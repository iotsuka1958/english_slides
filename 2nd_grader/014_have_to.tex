\documentclass[aspectratio=169,xcolor={dvipsnames,table}]{beamer}
\usepackage[no-math,deluxe,haranoaji]{luatexja-preset}
\renewcommand{\kanjifamilydefault}{\gtdefault}
\renewcommand{\emph}[1]{{\upshape\bfseries #1}}
\usetheme{metropolis}
\metroset{block=fill}
\setbeamertemplate{navigation symbols}{}
\setbeamertemplate{blocks}[rounded][shadow=false]
\usecolortheme[rgb={0.7,0.2,0.2}]{structure}
%%%%%%%%%%%%%%%%%%%%%%%%%%%
\usepackage{media9}
%%%%%%%%%%%%%%%%%%%%%%%%%%%
%% さまざまなアイコン
%%%%%%%%%%%%%%%%%%%%%%%%%%%
\usepackage{fontawesome}
%\usepackage{figchild}
\usepackage{twemojis}
\usepackage{utfsym}
\usepackage{bclogo}
\usepackage{marvosym}
\usepackage{fontmfizz}
\usepackage{pifont}
\usepackage{phaistos}
\usepackage{worldflags}
\usepackage{tabularray}
%%%%%%%%%%%%%%%%%%%%%%%%%%%
\usepackage{tikz}
\usetikzlibrary{backgrounds}
\usepackage{tcolorbox}
\usepackage{tikzpeople}
\usepackage{tikzducks}
\usepackage{circledsteps}
\usepackage{xcolor}
\usepackage{amsmath}
\usepackage{tipa}
\usepackage{pxrubrica}
\usepackage{manfnt}
%%%%%%%%%%%%%%%%%%%%%%%%%%
\usepackage[absolute,overlay]{textpos}  %% 任意の位置に図を配置
%\usepackage[grid=true,gridcolor=Maroon,subgridcolor=gray,gridunit=pt,texcoord]{eso-pic} %%場所決めのための格子
%%%%%%%%%%%%%%%%%%%%%%%%%%%
%% 場合分け
\usepackage{cases}
%%%%%%%%%%%%%%%%%%%%%%%%%%%
% \myAnch{<名前>}{<色>}{<テキスト>}
% 指定のテキストを指定の色の四角枠で囲み, 指定の名前をもつTikZの
% ノードとして出力する. 図には remeber picture 属性を付けている
% ので外部から参照可能である.
\newcommand*{\myAnch}[3]{%
  \tikz[remember picture,baseline=(#1.base)]
    \node[draw,rectangle,#2] (#1) {\normalcolor #3};
}
%%%%%%%%%%%%%%%%%%%%%%%%%%%
\newcommand{\dnote}[1]{%
    \noindent % I guess this is intended...
    \begin{tabular}{@{}m{0.13\textwidth}@{}m{0.87\textwidth}@{}}%
        \huge\textdbend &#1%
    \end{tabular}%
    \par % ... and this too.
}
%%%%%%%%%%%%%%%%%%%%%%%%%%%%
%% 音声リンク表示
\newcommand{\myaudio}[1]{\href{#1}{\faVolumeUp}}
%%%%%%%%%%%%%%%%%%%%%%%%%%%
% \myEmph コマンドの定義
%\newcommand{\myEmph}[3]{%
%    \textbf<#1>{\color<#1>{#2}{#3}}%
%}
\usepackage{xparse} % xparseパッケージの読み込み
\NewDocumentCommand{\myEmph}{O{} m m}{%
    \def\argOne{#1}%
    \ifx\argOne\empty
        \textbf{\color{#2}{#3}}% オプション引数が省略された場合
    \else
        \textbf<#1>{\color<#1>{#2}{#3}}% オプション引数が指定された場合
    \fi
}
%%%%%%%%%%%%%%%%%%%%%%%%%%%
%% 文末の上昇イントネーション記号 \myRisingPitch
%% 通常のイントネーション \myDownwardPitch
%% https://note.com/dan_oyama/n/n8be58e8797b2
%%%%%%%%%%%%%%%%%%%%%%%%%%%
\newcommand{\myRisingPitch}{
\begin{tikzpicture}[scale=0.3,baseline=0.3]
\draw[->,>=stealth] (0,0) to[bend right=45] (1,1);
\end{tikzpicture}
}
\newcommand{\myDownwardPitch}{
\begin{tikzpicture}[scale=0.3,baseline=0.3]
\draw[->,>=stealth] (0,1) to[bend left=45] (1,0);
\end{tikzpicture}
}
%%%%%%%%%%%%%%%%%%%%%%%%%%%
\title{English is fun.}
\subtitle{I have to make cookies for Jane's birthday.}
\author{}
\institute[]{}
\date[]

%%%%%%%%%%%%%%%%%%%%%%%%%%%%
%% TEXT
%%%%%%%%%%%%%%%%%%%%%%%%%%%%
\begin{document}
\begin{frame}[plain]
  \titlepage
\end{frame}

\section*{授業の流れ}
\begin{frame}[plain]
  \frametitle{授業の流れ}
  \tableofcontents
\end{frame}

%%%%%%%%%%%%%%%%%%%%%%%%%%%
\section{have to---}
\subsection{have to---(しなければならない」)}
%%%%%%%%%%%%%%%%%%%%%%%%%%
\begin{frame}[plain,t]\frametitle{have to ---(しなければならない)}
 \Large

\visible<1->{I \alt<1-2>{\myAnch{aux1}{white}{\bfseries must}}{\myAnch{AUX1}{Orange}{\bfseries must}}  \textcolor{Green}{\bfseries study} science.}
\hfill\visible<2->{{\scriptsize science \textipa{/s\'aI@ns/}: 理科、科学}}


\vspace{15pt}

\visible<4->{I \alt<1-4>{\myAnch{aux2}{white}{\bfseries have to}}{\myAnch{AUX2}{Orange}{\bfseries have to}}  \textcolor{Green}{\bfseries study} science.}%
\hfill\visible<5->{{\scriptsize \textbf{have to} \textipa{/h\'\ae ft@/}}}

%
\hfill{\tiny 0113}\,{\scriptsize \myaudio{./audio/014_have_to_01.mp3}}

\vspace{20pt}

\begin{block}<6->{Topics for Today}\small
義務「〜しなければならない」
\begin{itemize}\setbeamertemplate{items}[square]\small
 \item  {\bfseries must}  $+$ 動詞の原形\hfill{\scriptsize \textbf{must}は助動詞}
 \item \fcolorbox{BurntOrange}{white}{\bfseries have to}  $+$ 動詞の原形
 \item {\bfseries have to} \textipa{/h\'\ae ft@/}\hfill{}cf. have(持っている) \textipa{/h\'\ae v/}
 \end{itemize}
     \end{block}
%\hfill\visible<7->{\small {ところでmustとhave toってどうちがうの?}}

\visible<5->{\begin{tikzpicture}[remember picture,overlay] 
 \draw[line width=3pt,opacity=.5,BurntOrange,->] (aux1.south) to[out=-80, in=90] (aux2.north);
\end{tikzpicture}}
\end{frame}
%%%%%%%%%%%%%%%%%%%%%%%%%%%%%%%%
\begin{frame}[plain]{Exercises}

{\small あたえられた日本語の意味になるよう、カッコ内の語句を並べ替えましょう。なお、先頭の語は大文字で始めてください}%
\hfill{\tiny 0202}\,{\scriptsize \myaudio{./audio/014_have_to_02.mp3}}
\begin{enumerate}
 \item わたしは部屋を掃除しなければなりません。\\
( clean / have / I / to / my room )\hspace{20pt}
\visible<2->{I have to clean my room.}
 \item わたしたちは彼を助けなければなりません。\\
( help / him / to / we / have )\hspace{20pt}
\visible<3->{We have to help him.}
 \item あなたは両親の言うことを聞かなければなりません。\hfill{\scriptsize listen to ~を聞く}\\
( listen / your parents / to / to /have / you )\\
\visible<4->{You have to listen to your parents.}
 \item 彼らは制服を着なければなりません。
\\
( uniforms / wear / to / they / have )\hspace{20pt}
\visible<5->{They have to wear uniforms.}\\%
\hfill\visible<6>{{\scriptsize \textdbend\textdbend\,\,uniform($=$uni$+$form)\,\,\,unicycle($=$uni$+$cycle)つまり{\bfseries uni}は\,\,\Circled[fill color=yellow!50]{\,\,1、単\,\,}\,\,の意味}}
\end{enumerate} 
\end{frame}
%%%%%%%%%%%%%%%%%%%%%%%%%%%%%%%%%%%
\begin{frame}<1-13>[plain]{haveの三単現(復習)}
 \large

\begin{tblr}{
 colspec={ll},
 column{2} = {font = {\small}}
 }
\visible<1->{I have a cat.}&\visible<7->{一人称}\\
\visible<1->{We \alt<1>{(\phantom{~~have~~})}{have} a cat.}&\visible<8->{一人称}\\
\visible<1->{You  \alt<1-2>{(\phantom{~~have~~})}{have} a cat.}&\visible<9->{二人称}\\
\visible<1->{They \alt<1-3>{(\phantom{~~have~~})}{have} a cat.}&\visible<10->{三人称複数}\\
\visible<1->{\alt<1-4>{He}{\textcolor{BurntOrange}{\bfseries He}} \alt<1-4>{(\phantom{~~has~~})}{\textcolor{Green}{\bfseries has}} a cat.}&\visible<11->{三人称単数}\\
\visible<1->{\alt<1-5>{She}{\textcolor{BurntOrange}{\bfseries She}} \alt<1-5>{(\phantom{~~has~~})}{\textcolor{Green}{\bfseries has}} a cat.}&\visible<12->{三人称単数}
\end{tblr}

\begin{block}<13->{復習}\small
\begin{itemize}\setbeamertemplate{items}[square]\small
 \item   {\bfseries have}(持っている)は、主語が\textcolor{BurntOrange}{三人称単数}のときは\textcolor{Green}{\bfseries has}になります(いわゆる\kenten{三単現})
 \item {\bfseries have} \textipa{/h\'\ae v/}\hspace{50pt}{\bfseries has} \textipa{/h\'\ae z/}
 \end{itemize}
     \end{block}

\end{frame}
%%%%%%%%%%%%%%%%%%%%%%%%%%%%%%%%%
\begin{frame}<1-13>[plain]{have to--- (主語が三人称単数のとき)}
 \large

\begin{tblr}{
 colspec = {ll},
 column{2} = {font = {\small}}
}
\visible<1->{I have to study English.}&\visible<7->{一人称}\\
\visible<1->{We \alt<1>{(~~~~~~) (~~~~~~)}{have to} study English.}&\visible<8->{一人称}\\
\visible<1->{You  \alt<1-2>{(~~~~~~) (~~~~~~)}{have to} study English.}&\visible<9->{二人称}\\
\visible<1->{They \alt<1-3>{(~~~~~~) (~~~~~~)}{have to} study English.}&\visible<10->{三人称複数}\\\hline
\visible<1->{\alt<1-4>{He}{\textcolor{BurntOrange}{\bfseries He}} \alt<1-4>{(~~~~~~) (~~~~~~)}{\textcolor{Green}{\bfseries has to}} study English.}&\visible<11->{三人称単数}\\
\visible<1->{\alt<1-5>{She}{\textcolor{BurntOrange}{\bfseries She}} \alt<1-5>{(~~~~~~) (~~~~~~)}{\textcolor{Green}{\bfseries has to}}  study English.}&\visible<12->{三人称単数}
\end{tblr}

\hfill{\tiny 0251}\,{\scriptsize \myaudio{./audio/014_have_to_03.mp3}}
\begin{block}<13->{Topics for Today}
\begin{itemize}\setbeamertemplate{items}[square]\small
 \item   {\bfseries have to}(しなければならない)は、主語が\textcolor{BurntOrange}{三人称単数}のときは\textcolor{Green}{\bfseries has to}になります
 \item {\bfseries have to} \textipa{/h\'\ae ft@/}\,\,\,\,\,\,\,\,\,\,\,\,\,\,\,\,{\bfseries has to} \textipa{/h\'\ae st@/}
 \end{itemize}
     \end{block}

\end{frame}
%%%%%%%%%%%%%%%%%%%%%
\begin{frame}[plain]{Exercises}

{\small あたえられた日本語の意味になるよう、カッコ内の語句を並べ替えましょう。なお、先頭の語は大文字で始めてください}%
\hfill{\tiny 0205}\,{\scriptsize \myaudio{./audio/014_have_to_04.mp3}}
 \begin{enumerate}
  \item {\small 彼はこの本を読まなければならない。}
( read / this / to / has / he / book )\\
\visible<2->{He has to read this book.}
  \item {\small 彼女はレポートを書かなければならない。}
( write / a report / has / she / to )\\
\visible<3->{She has to write a report.}\hfill{}{\scriptsize report \textipa{/rIp\'O\textrhookschwa t/} レポート}
  \item {\small ベンは歯医者に行かなければならない。}
Ben ( the dentist / to / to / has / go )\\
\visible<4->{Ben has to go to the dentist.}\hfill{}{\scriptsize dentist \textipa{/d\'entIst/} 歯医者}  
\item {\small ベティとジャックは家を掃除しなければならない。}\\
Betty and Jack  ( clean /  have / the house / to )\\
\visible<5->{Betty and Jack have to clean the house.}
 \end{enumerate}
\end{frame}
%%%%%%%%%%%%%%%%%%%%%%%%%%
\section{have to---の疑問文}
%%%%%%%%%%%%%%%%%%%%%%%%%
\begin{frame}[plain]{have to---の疑問文}
 \large

\mbox{}\hspace{35pt}\visible<1->{You {\bfseries have to} study tonight.}%
\visible<1->{{\raisebox{-5pt}{\scalebox{1.4}{\myDownwardPitch}}}} 
\hfill\visible<1->{{\scriptsize tonight \textipa{/tUn\'aIt/} 今夜}}

\vspace{15pt}

\visible<2->{\myAnch{do}{Maroon}{\bfseries Do} you {\bfseries have to} study tonight \myAnch{question}{Maroon}{?}}
\visible<5->{\scalebox{1.4}{\myRisingPitch}}

\mbox{}\hspace{50pt}\visible<3->{\myAnch{txt1}{white}{\small 先頭に\textbf{Do}}}
\hspace{80pt}\visible<4->{\myAnch{txt2}{white}{\small 文末に`?'}}%
\visible<5->{{\small イントネーションは上昇調}}

\begin{tikzpicture}[remember picture, overlay]
\visible<3->{\draw[->, line width=2pt, opacity=.5, Maroon] (txt1.west) to[out=180, in=-90] (do.south);}
\visible<4->{\draw[->, line width=2pt, opacity=.5, Maroon] (txt2.west) to[out=180, in=-90] (question.south);}
\end{tikzpicture}

\hfill{\tiny 0112}\,{\scriptsize \myaudio{./audio/014_have_to_05.mp3}}

\begin{block}<6->{Today's Topic}
\begin{itemize}\setbeamertemplate{items}[square]\small
 \item   You {\bfseries have to}--の疑問文「〜しなければいけませんか」\\
\mbox{}\hfill{}$\text{{\bfseries Do} you {\bfseries have to}} + \text{動詞の原形}$\ldots{}\,\,\,?   
\end{itemize}
\end{block}
\end{frame}
%%%%%%%%%%%%%%%%%%%%%%
\begin{frame}[plain]\frametitle{Exercises}

{\small つぎの英文を疑問文にしましょう}

%\large
\begin{tabular}{rlcl}
 1& I have to clean the room. &$\rightarrow$ &\visible<2->{\myEmph[2-]{Maroon}{Do} I \myEmph[2-]{Maroon}{have to} clean the room\myEmph[2-]{Maroon}{?}} \\
 2& You have to study tonight.&$\rightarrow$ &\visible<3->{\myEmph[3-]{Maroon}{Do} you \myEmph[3-]{Maroon}{have to} study tonight\myEmph[3-]{Maroon}{?}} \\
 3& They have to go now.&$\rightarrow$ &\visible<4->{\myEmph[4-]{Maroon}{Do} they \myEmph[4-]{Maroon}{have to} go now\myEmph[4-]{Maroon}{?}}
\end{tabular}


\vfill
%
\hfill{\tiny 0229}\,{\scriptsize \myaudio{./audio/014_have_to_06.mp3}}

\begin{block}<5->{Topics for Today}
\begin{itemize}\setbeamertemplate{items}[square]\small
 \item   主語が一人称、二人称、三人称複数のとき$\text{\bfseries Do} + \text{S} + \text{\bfseries have to} + \text{動詞の原形}$\,\,\ldots{}\,\,\,?
 \item   文末に`?'をつける / イントネーションは上昇調
\end{itemize}
     \end{block}
\end{frame}
%%%%%%%%%%%%%%%%%%%%%
\begin{frame}[plain]{主語が三人称単数のとき}
 \large

\mbox{}\hspace{50pt}\alt<1>{You}{He} \alt<1-2>{have to}{\myAnch{T1}{orange}{\bfseries has to}} study hard. \only<2>{\ding{55}}%
\visible<3->{\raisebox{-5pt}{\scalebox{1.4}{\myDownwardPitch}}}
\hfill\visible<1->{{\scriptsize hard \textipa{/h\'A\textrhookschwa d/} 懸命に}}

\hspace{100pt}%
\visible<4->{%
\myAnch{T2}{orange}{\small 三人称単数のとき}
}

\vspace{20pt}

\onslide<5->{%
\myAnch{does}{orange}{\bfseries Does} he \fcolorbox{orange}{white}{\bfseries have to} study hard\myAnch{question}{orange}{?}
}
\onslide<8->{%
\scalebox{1.4}{\myRisingPitch}
}

\onslide<6->{%
\mbox{}\hspace{50pt}%
\myAnch{txt1}{white}{%
 \small
 \begin{tabular}[t]{@{}l}
  先頭にDoes(Do\,\,\ding{55})\\
  (Does he \textcolor{Maroon}{has} to\,\,\ding{55})
 \end{tabular}
 }
}
\hspace{60pt}%
\visible<7->{\myAnch{txt2}{white}{\small 文末に`?'}}

\begin{tikzpicture}[remember picture, overlay]
\visible<6->{\draw[->, thick, orange] (txt1) to[out=180, in=-90] (does.south);}
\visible<7->{\draw[->, thick, orange] (txt2.west) to[out=180, in=-45] (question.south);}
\end{tikzpicture}

\hfill{\tiny 0112}\,{\scriptsize \myaudio{./audio/014_have_to_07.mp3}}
\begin{block}<9->{Topics for Today}
\begin{itemize}\setbeamertemplate{items}[square]\small
 \item   主語が三人称単数のとき$\text{\bfseries Does} + \text{S} + \text{\bfseries have to} + \text{動詞の原形}$\,\,\ldots{}\,\,\,?
 \item   文末に`?'をつける / イントネーションは\myRisingPitch{}
\end{itemize}
     \end{block}
\end{frame}
%%%%%%%%%%%%%%%%%%%%%%%%%%%%%%%%%%
\begin{frame}<1-10>[plain]\frametitle{Exercises}

{\small つぎの文を疑問文にしましょう}%
\hfill{\tiny 0412}\,{\scriptsize \myaudio{./audio/014_have_to_08.mp3}}

 \begin{enumerate}
  \item<1-> You have to finish your homework.\\
        \onslide<5->{$\longrightarrow$\,\,\,\,\, Do you have to finish your homework?}\hfill\scalebox{.75}{\bcfleur\bcfleur}
  \item<1-> They have to go to Boston.\hspace{37.5pt}%
        \onslide<6->{$\longrightarrow$\,\,\,\,\, Do they have to go to Boston?}\hfill\scalebox{.25}{\worldflag{US}}
  \item<1-> She has to wake up early tomorrow.%
\hfill{}{\scriptsize wake up起きる}\\%
        \onslide<7->{$\longrightarrow$\,\,\,\,\, Does she have to wake up early tomorrow?}
  \item<1-> He has to drive to the station.\hspace{12pt}%
        \onslide<8->{$\longrightarrow$\,\,\,\,\, Does he have to drive to the station?}\hfill\faCar
  \item<1-> Your mother has to walk to school.\\
        \onslide<9->{$\longrightarrow$\,\,\,\,\, Does your mother have to walk to school?}\hfill\scalebox{1}{\PHpedestrian}\,\,\scalebox{2}{\twemoji{school}}
 \end{enumerate}

\begin{block}<2->{Topics for Today}
\begin{itemize}\setbeamertemplate{items}[square]\small
 \item<3->   主語が一人称、二人称、三人称複数のときは\hfill$\text{\bfseries Do} + \text{S} + \text{\bfseries have to} + \text{動詞の原形}$\,\,\ldots{}\,\,\,?\pause
 \item<4->   主語が三人称単数のときは\hfill$\text{\bfseries Does} + \text{S} + \text{\bfseries have to} + \text{動詞の原形}$\,\,\ldots{}\,\,\,?
\end{itemize}
\end{block}
\end{frame}
%%%%%%%%%%%%%%%%%%%%%%%%%%%%%%%
\section{疑問文への答え方}
\subsection{Do you have to 〜 ? と聞かれたら}
%%%%%%%%%%%%%%%%%%%%%%%%%%%%%%
 \begin{frame}[plain]{Do you have to〜 ? と聞かれたら}
 \Large

\visible<2->{Do you have to go home now?}

\vspace{30pt}


\mbox{}\hspace{80pt}%
\visible<3->{$\left\{\begin{tabular}{l}
         \visible<3->{\text{Yes, I do.}}\\
         \visible<4->{\text{No, I do not.}}\\
         \visible<5->{\text{(}= \text{No, I don't.)}}
        \end{tabular}\right.$}

\hfill{}\visible<6->{{\small Noのときdo notを縮めてNo, I \textcolor{Maroon}{\bfseries don't}.ともいいます}}

\hfill\tiny{0159}\,{\scriptsize \myaudio{./audio/014_have_to_09.mp3}}

% この場合は (363pt, 80pt) の位置に 0.4\linewidth の幅のブロックができる.
\begin{textblock*}{0.4\linewidth}(363pt,85pt)
    % TiKZを使った図形の描画
    \visible<4->{\begin{tikzpicture}
        \duck[laughing,bowtie,
strawhat=brown!50!white,
ribbon=black,
think={\scriptsize
 \begin{tabular}{c}
 一般動詞\\と同じ!
 \end{tabular}
},
bubblecolour=white!50!pink,
scale=1.1414]
    \end{tikzpicture}}
\end{textblock*}
\end{frame}

%%%%%%%%%%%%%%%%%%%%%%%%%%%%%%%%
\subsection{Does he  have to 〜 ? と聞かれたら}
%%%%%%%%%%%%%%%%%%%%%%%%%%%%%%%%
 \begin{frame}[plain]{Does he have to〜 ? と聞かれたら}
 \Large

\visible<2->{Does he have to get up early tomorrow?}

\vspace{30pt}


\mbox{}\hspace{80pt}%
\visible<3->{$\left\{\begin{tabular}{l}
         \visible<3->{\text{Yes, he does.}}\\
         \visible<4->{\text{No, he does not.}}\\
         \visible<5->{\text{(}= \text{No, he doesn't.)}}
        \end{tabular}\right.$}

\hfill\visible<6->{{\small Noのときdoes notを縮めてNo, he \textcolor{Maroon}{\bfseries doesn't}.ともいいます}}%

\hfill{\tiny 0202}\,{\scriptsize \myaudio{./audio/014_have_to_10.mp3}}
\end{frame}
%%%%%%%%%%%%%%%%%%%%%%%
\begin{frame}[plain]{Exercises}
例にならって、つぎの質問に対する答えを「はい」と「いいえ」の2通りつくりましょう%
\hfill{\scriptsize \myaudio{./audio/014_have_to_11.mp3}}

\begin{tabular}{@{}r@{\,\,\,\,}l@{\,\,\,\,}c@{\,\,\,\,}l@{\,\,\,}l}
\visible<1->{\scriptsize 例}& \visible<1->{Do you have to study every day?}& \visible<2->{$\rightarrow$}&\visible<3->{(1) Yes, I do.}&\visible<4->{(2) No, I don't.}\\[10pt]
\visible<1->{1}&\visible<1->{Does she have to read the book?\hspace{10pt}\raisebox{0pt}{\bcbook}}&\visible<5->{$\rightarrow$}&\visible<6->{(1) Yes, she does.}&\visible<7->{(2) No, she doesn't.}\\
\visible<1->{2}&\visible<1->{Does he have to cook dinner tonight?}&\visible<8->{$\rightarrow$}& \visible<9->{(1) Yes, he does.}&%
\visible<10->{(2) No, he doesn't.}\\
\visible<1->{3}&\visible<1->{Do they have to follow the rules?}&\visible<11->{$\rightarrow$}&\visible<12->{(1) Yes, they do.}&\visible<13->{(2) No, they don't.}\\
\multicolumn{2}{r}{{\scriptsize follow \textipa{/f\'AloU/} 従う、守る}}
\end{tabular}
\end{frame}
%%%%%%%%%%%%%%%%%%%%%%%%%%%%%%%
\section{have to---の否定文}
%%%%%%%%%%%%%%%%%%%%%%%%%%%%%%%%%%
\begin{frame}[plain]{have to---の否定文}
 \Large

\visible<1->{You \fcolorbox{BurntOrange}{white}{\bfseries have to} finish it today.\hfill{\scriptsize finish \textipa{/f\'InIS/} 終える}}

\visible<2->{You \textcolor{BrickRed}{{\bfseries do not}} \fcolorbox{BurntOrange}{white}{\bfseries have to} finish it today.}

\visible<3->{You \textcolor{BrickRed}{{\bfseries don't}} \fcolorbox{BurntOrange}{white}{\bfseries have to} finish it today.}


\visible<4->{{\small \fbox{終わらせなければならない}\,というわけではない}}\\
\hspace{120pt}\visible<5->{{\small $\longrightarrow$\,\,「終わらせる必要はない」「終わらせなくてもいい」}}

\hfill{\tiny 0140}\,{\scriptsize \myaudio{./audio/014_have_to_12.mp3}}

\begin{block}<6->{Topic for Today}
\begin{itemize}\setbeamertemplate{items}[square]\small
 \item   {\bfseries do not($=$don't) have to} $+$ 原形\hspace{20pt}「〜する必要はない」「しなくてもいい」\\
\hfill{}{\scriptsize もちろん三人称単数が主語なら{\bfseries does not($=$doesn't)\,\,have to}}
 \end{itemize}
     \end{block}

\end{frame}
%%%%%%%%%%%%%%%%%%%%%%%%%%%%%%
\begin{frame}[plain]{Exercises}
 次の各組の2文の意味の違いについて考えましょう%
\hfill{\tiny 0221}\,{\scriptsize \myaudio{./audio/014_have_to_13.mp3}}

\begin{enumerate}
 \item $\left\{\begin{tabular}{rl}
(A)&You have to study today.\\
(B)&You don't have to study today.
\end{tabular}
\right.$
 \item 
$\left\{\begin{tabular}{rl}
(A)&You have to do it alone$^{1}$.\\
(B)&You don't have to do it alone.
\end{tabular}
\right.$
 \item $\left\{\begin{tabular}{rl}
(A)&You have to be perfect$^{2}$.\\
(B)&You don't have to be perfect.
\end{tabular}
\right.$
\end{enumerate}

\vfill

\hrule

{\scriptsize 1\,\,\,alone \textipa{/@l\'oUn/} ひとりで}\\[-4pt]
{\scriptsize 2\,\,\,perfect \textipa{/p\'\textrhookschwa :fIkt/} 完璧な}
\end{frame}
%%%%%%%%%%%%%%%%%%%%%%%%%%%%%%%%%
\section{don't have toとmust not}
%%%%%%%%%%%%%%%%%%%%%%%%%%%%%%%%%
\begin{frame}[plain]{don't have toとmust not}
\Large
 
\visible<1->{You \textcolor{NavyBlue}{{\bfseries don't have to}} finish it today.}%
\hfill\visible<2->{{\small 今日それを終わらせなくてもいい。}}

\visible<3->{You \textcolor{Maroon}{\bfseries must not} eat in the library}%
\hfill\visible<4->{{\small 図書館でものを食べてはいけない。}}

\visible<5->{You \textcolor{Maroon}{\bfseries mustn't} eat in the library}%
\hfill{\scriptsize mustn't \textipa{/m\'2snt/}}

\vfill
%
\hfill{\tiny 0114}\,{\scriptsize \myaudio{./audio/014_have_to_14.mp3}}

\begin{block}<6->{Topics for Today}
\begin{itemize}\setbeamertemplate{items}[square]\small
 \item   {\bfseries do not($=$don't) have to}:「〜しなくてもいい」、「〜する必要はない」
 \item   {\bfseries must not / mustn=t}:「〜してはいけない」(禁止)
 \end{itemize}
     \end{block}
\end{frame}
%%%%%%%%%%%%%%%%%%%%%%%%%%%%%%%%%%%
\begin{frame}[plain,t]{Exercises}
 
つぎの2文が、それぞれどんな場面で使われるか想像してみましょう%
\hfill{\tiny 0109}\,{\scriptsize \myaudio{./audio/014_have_to_15.mp3}}


\begin{enumerate}
 \item You {\bfseries don't have to} eat it.
 \item You {\bfseries must not} eat it.
\end{enumerate}

\vspace{80pt}

\pause

\dnote{have to--- とmustは\kenten{義務}(~しなければならない)をあらわすという点でおなじです。いっぽう、don't have to---(~しなくてもいい)とmust not(~してはいけない)はまったく異なる意味ですから、しっかり区別することがだいじです}

\begin{textblock*}{0.4\linewidth}(300pt,62.5pt)
    % TiKZを使った図形の描画
    \visible<5->{\begin{tikzpicture}
  \duck[crazyhair=brown!60!black, glasses, eyebrow,
      %signpost={\small\textipa{/k@n/}},
      speech={\scriptsize メモメモ},
      laughing,
      jacket=Maroon!30, lapel, buttons,
      water=NavyBlue]
    \end{tikzpicture}}
\end{textblock*}
\end{frame}
%%%%%%%%%%%%%%%%%%%%%%%%%%%%%%%%%%%%
\section{まとめ}
\begin{frame}[plain]{まとめ1}
 \begin{block}<1->{基本}\small
\kenten{義務}をあらわす\,\Circled[fill color=white]{\bfseries \,have to\,}\,\,\,\,「〜しなければならない」\hfill{}cf. \textbf{must}
\begin{itemize}\setbeamertemplate{items}[square]\small
 \item $\text{\bfseries S} + \left\{\begin{array}{l}
			      \text{\bfseries have to}\\
			      \text{\bfseries has to}\end{array}
                            \right\} + \text{\bfseries 原形}$%
\hfill\begin{tabular}{l}
       {\scriptsize \textipa{/h\'\ae ft@/}}\\
       {\scriptsize \textipa{/h\'\ae st@/}}
      \end{tabular}
 \end{itemize}
     \end{block}

\begin{block}<2->{疑問文}
\begin{itemize}\setbeamertemplate{items}[square]\small
 \item $\left\{\begin{array}{l}
                  \text{\bfseries Do}\\
                  \text{\bfseries Does}
		\end{array}\right\} + \text{\bfseries S} + \text{\bfseries have to} + \text{\bfseries 原形\ldots\,\,?}$
 \item 答え方
\begin{itemize}\setbeamertemplate{items}[circle]
 \item 「はい」のとき\hspace{60pt}%
       $\text{\bfseries Yes, S} + \left\{\begin{array}{l}
				   \text{\bfseries do}\\
				   \text{\bfseries does}\\
				 \end{array}\right\} \text{.}$
 \item 「いいえ」のとき\hspace{50pt}%
       $\text{\bfseries No, S} + \left\{\begin{array}{l}
				   \text{\bfseries do not\,(} = \text{\bfseries don't)}\\
				   \text{\bfseries does not\,(} = \text{\bfseries doesn't)}\\
				 \end{array}\right\} \text{.}$
% \item 「いいえ」のとき\hfill{}\makebox[100pt][l]{\bfseries No, S $ + \text{\,don't($=$do not) / doesn't($=$ does not).}$}\hspace{120pt}\mbox{}
\end{itemize}
\end{itemize}
\end{block}
\end{frame}
%%%%%%%%%%%%%%%%%%%%%%%%%%%%%
\begin{frame}[plain,t]{まとめ2}
 \begin{block}<3->{do not($=$ don't) have toとmust not}
\begin{itemize}\setbeamertemplate{items}[square]\small
 \item<4->  $\text{\bfseries S} + \left\{
          \begin{array}{l}
	   \text{\bfseries do not($=$don't) have to}\\
	   \text{\bfseries does not($=$doesn't) have to}
	  \end{array}\right\} + \text{原形 .}$\hfill{}%
\begin{tabular}{l}
「〜しなくてもいい」\\
「〜する必要はない」
\end{tabular}
 \item<5-> $\text{\bfseries S} + \left\{
         \begin{array}{l}
	  \text{\bfseries must not}\\
	  \text{\bfseries mustn't}
	 \end{array}\right\} + \text{原形 .}$\hfill{}「〜してはいけない」(禁止)
 \end{itemize}
     \end{block}


\begin{textblock*}{0.4\linewidth}(50pt,150pt)
\visible<1->{\begin{tikzpicture}
\node[name=a,sailor,female,minimum size=1.5cm] {};
  \node[ellipse callout, xshift=1.6cm, draw,yshift= .6cm,
    callout absolute pointer={(a.mouth)},
    font=\tiny\bfseries, align=center, inner sep=1pt]
    {don't have to\\とmustn'tって\\違うんですか};
\end{tikzpicture}}
\end{textblock*}


\begin{textblock*}{0.4\linewidth}(125pt,165pt)
    % TiKZを使った図形の描画
    \visible<2->{\begin{tikzpicture}
      \node[name=k, shape=businessman, minimum size=55pt, mirrored] {};
  \node[ellipse callout, xshift=-1.5cm, draw,yshift= .2cm,
    callout absolute pointer={(k.mouth)},
    font=\tiny\bfseries, align=center, inner sep=1pt]
    {意味が\\まったく\\違いますよ};
    \end{tikzpicture}}
\end{textblock*}
\end{frame}
%%%%%%%%%%%%%%%%%%%%%%%%%%%%%%%
\end{document}
