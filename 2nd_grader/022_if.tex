

\documentclass[aspectratio=169,xcolor={dvipsnames,table}]{beamer}
\usepackage[no-math,deluxe,haranoaji]{luatexja-preset}
\renewcommand{\kanjifamilydefault}{\gtdefault}
\renewcommand{\emph}[1]{{\upshape\bfseries #1}}
\usetheme{metropolis}
\metroset{block=fill}
\setbeamertemplate{navigation symbols}{}
\setbeamertemplate{blocks}[rounded][shadow=false]
\usecolortheme[rgb={0.7,0.2,0.2}]{structure}
%%%%%%%%%%%%%%%%%%%%%%%%%%
%% Change alert block colors
%%% 1- Block title (background and text)
\setbeamercolor{block title alerted}{fg=mDarkTeal, bg=mLightBrown!45!yellow!45}
\setbeamercolor{block title example}{fg=magenta!10!black, bg=mLightGreen!70}
%%% 2- Block body (background)
\setbeamercolor{block body alerted}{bg=mLightBrown!25}
\setbeamercolor{block body example}{bg=mLightGreen!15}
%%%%%%%%%%%%%%%%%%%%%%%%%%%
%%%%%%%%%%%%%%%%%%%%%%%%%%%
%% さまざまなアイコン
%%%%%%%%%%%%%%%%%%%%%%%%%%%
%\usepackage{fontawesome}
\usepackage{fontawesome5}
\usepackage{figchild}
\usepackage{twemojis}
\usepackage{utfsym}
\usepackage{bclogo}
\usepackage{marvosym}
\usepackage{fontmfizz}
\usepackage{pifont}
\usepackage{phaistos}
\usepackage{worldflags}
\usepackage{jigsaw}
\usepackage{tikzlings}
\usepackage{tikzducks}
\usepackage{scsnowman}
\usepackage{epsdice}
\usepackage{halloweenmath}
\usepackage{svrsymbols}
\usepackage{countriesofeurope}
\usepackage{tipa}
\usepackage{manfnt}
%%%%%%%%%%%%%%%%%%%%%%%%%%%
\usepackage{tikz}
\usetikzlibrary{calc,patterns,decorations.pathmorphing,backgrounds}
\usepackage{tcolorbox}
\usepackage{tikzpeople}
\usepackage{circledsteps}
\usepackage{xcolor}
\usepackage{amsmath}
\usepackage{booktabs}
\usepackage{chronology}
\usepackage{signchart}
%%%%%%%%%%%%%%%%%%%%%%%%%%%
%% 場合分け
%%%%%%%%%%%%%%%%%%%%%%%%%%%
\usepackage{cases}
%%%%%%%%%%%%%%%%%%%%%%%%%%
\usepackage{pdfpages}
%%%%%%%%%%%%%%%%%%%%%%%%%%%
%% 音声リンク表示
\newcommand{\myaudio}[1]{\href{#1}{\faVolumeUp}}
%%%%%%%%%%%%%%%%%%%%%%%%%%
%% \myAnch{<名前>}{<色>}{<テキスト>}
%% 指定のテキストを指定の色の四角枠で囲み, 指定の名前をもつTikZの
%% ノードとして出力する. 図には remember picture 属性を付けている
%% ので外部から参照可能である.
\newcommand*{\myAnch}[3]{%
  \tikz[remember picture,baseline=(#1.base)]
    \node[draw,rectangle,line width=1pt,#2] (#1) {\normalcolor #3};
}
%%%%%%%%%%%%%%%%%%%%%%%%%%
%% \myEmph コマンドの定義
%%%%%%%%%%%%%%%%%%%%%%%%%%
%\newcommand{\myEmph}[3]{%
%    \textbf<#1>{\color<#1>{#2}{#3}}%
%}
\usepackage{xparse} % xparseパッケージの読み込み
\NewDocumentCommand{\myEmph}{O{} m m}{%
    \def\argOne{#1}%
    \ifx\argOne\empty
        \textbf{\color{#2}{#3}}% オプション引数が省略された場合
    \else
        \textbf<#1>{\color<#1>{#2}{#3}}% オプション引数が指定された場合
    \fi
}
%%%%%%%%%%%%%%%%%%%%%%%%%%%
%%%%%%%%%%%%%%%%%%%%%%%%%%%
%% 文末の上昇イントネーション記号 \myRisingPitch
%% 通常のイントネーション \myDownwardPitch
%% https://note.com/dan_oyama/n/n8be58e8797b2
%%%%%%%%%%%%%%%%%%%%%%%%%%%
\newcommand{\myRisingPitch}{
\begin{tikzpicture}[scale=0.3,baseline=0.3]
\draw[->,>=stealth] (0,0) to[bend right=45] (1,1);
\end{tikzpicture}
}
\newcommand{\myDownwardPitch}{
\begin{tikzpicture}[scale=0.3,baseline=0.3]
\draw[->,>=stealth] (0,1) to[bend left=45] (1,0);
\end{tikzpicture}
}
%%%%%%%%%%%%%%%%%%%%%%%%%%%%
%\AtBeginSection[%
%]{%
%  \begin{frame}[plain]\frametitle{授業の流れ}
%     \tableofcontents[currentsection]
%   \end{frame}%
%}

\usepackage{pxrubrica}
%%%%%%%%%%%%%%%%%%%%%%%%%%%
\title{English is fun.}
\subtitle{You will succeed if you work hard.}
\author{}
\institute[]{}
\date[]

%%%%%%%%%%%%%%%%%%%%%%%%%%%%
%% TEXT
%%%%%%%%%%%%%%%%%%%%%%%%%%%%
\begin{document}

\begin{frame}[plain]
  \titlepage
\end{frame}

%%%%%%%%%%%%%%%%%%%%%%%%%
\section*{授業の流れ}
\begin{frame}[plain]
  \frametitle{授業の流れ}
  \tableofcontents
\end{frame}

%%%%%%%%%%%%%%%%%%%%%%%%%%%
\section{if \textipa{/If/}}
\subsection{if}
%%%%%%%%%%%%%%%%%%%%%%%%%%%%%%%%%%%%%%%%%%%%%
\begin{frame}[plain]{もし~なら}
\large
 \begin{enumerate}
  \item You \myAnch{x_1}{white}{\myEmph[2-]{NavyBlue}{will succeed}} \alt<2->{\myAnch{y_1}{Maroon}{if you work hard}}{\myAnch{Y_1}{white}{if you work hard}}.\hfill\visible<7->{{\footnotesize {\bfseries S $+$ V}\,\,\Circled[fill color = white]{ if s $+$ v }}}
\item<3->  \alt<4->{\myAnch{y_2}{Maroon}{If you work hard}}{\myAnch{Y_2}{white}{If you work hard}},\,\, you  \myAnch{x_2}{white}{\myEmph[4-]{NavyBlue}{will succeed}}.\hfill\visible<7->{{\footnotesize \Circled[fill color = white]{ If s $+$ v },\,\, {\bfseries S $+$ V}}}
\end{enumerate}

\visible<2->%
{\begin{tikzpicture}[remember picture, overlay]
 \visible<2->{\draw[thick, Maroon, ->] (Y_1.north west) to[out=170, in=10] (x_1.north);} 
 \visible<4->{\draw[thick, Maroon, ->] (Y_2.north east) to[out=5, in=175] (x_2.north);} 
\end{tikzpicture}}


\hfill{\tiny 0118}\,{\scriptsize \myaudio{./audio/022_if_01.mp3}}
\begin{block}<5->{Topics for Today}\small
\begin{itemize}\setbeamertemplate{items}[square]\small
 \item   \visible<5->{\Circled[fill color = white]{ if s $+$ v }\,は「もし~なら」という意味}
 \item   \visible<6->{\Circled[fill color = white]{ if s $+$ v }\,は全体で「副詞」の役割}
 \item   \visible<7->{\Circled[fill color = white]{ if s $+$ v }\,は、前にも後ろにも位置します
\begin{enumerate}\setbeamertemplate{items}[circle]
	  \item {\bfseries S $+$ V}\,\,\Circled[fill color = white]{ if s $+$ v }
	  \item \Circled[fill color = white]{ If s $+$ v },\,\,{\bfseries S $+$ V}%
\hfill{}{カンマに注意!}
	 \end{enumerate}
	 }
 \item<8-> {\bfseries if}は2組の\,\,\Circled[fill color=white]{\,主語と動詞の組合せ\,}\,\,をつないでいます\visible<9->{$\rightarrow$\kenten{接続詞}}
\end{itemize}
     \end{block}
\end{frame}
%%%%%%%%%%%%%%%%%%%%%%%%%%%%%%
\begin{frame}{Exercises}
日本語の意味になるよう、(~~~~~)内の語句を並べ替えましょう。
先頭の語句は大文字ではじめてください%
\hfill{\tiny 0239}\,{\scriptsize \myaudio{./audio/022_if_02.mp3}}

 \begin{enumerate}
  \item ( you / water / if / drink ), you will feel better.{\small 水を飲めば、気持ちよくなりますよ。}\\
	\visible<2->{If you drink water, \textcolor{lightgray}{you will feel better.}}\hfill \visible<7->{\Circled[fill color = yellow!50]{ If s $+$ v }, {\bfseries S $+$ V}}
  \item (soccer / you / play / if ), you will have fun.{\small サッカーをすれば、楽しいですよ。}\\
	\visible<3->{If you play soccer, \textcolor{lightgray}{you will have fun.}}\hfill \visible<8->{\Circled[fill color = yellow!50]{ If s $+$ v }, {\bfseries S $+$ V}}
  \item (books / if / read / you ), you will learn a lot.{\small 本を読めば、おおくのことを学べる。}\\
	\visible<4->{If you read books, \textcolor{lightgray}{you will learn a lot.}}\hfill \visible<9->{\Circled[fill color = yellow!50]{ If s $+$ v }, {\bfseries S $+$ V}}
  \item We will watch a movie ( rains / it / if ).{\small 雨だったら、映画を見ます。}\\
	\visible<5->{\textcolor{lightgray}{We will watch a movie} if it rains.}\hfill\visible<10->{{\bfseries S $+$ V}\,\,\Circled[fill color = yellow!50]{ if s $+$ v }}
  \item We ( will be  / smile / if / happy / you ). {\small あなたが笑うと、わたしたちは幸福だ。}\\
	\visible<6->{\textcolor{lightgray}{We} will be happy if you smile.}\hfill \visible<11->{{\bfseries S $+$ V}\,\,\Circled[fill color = yellow!50]{ if s $+$ v }}
 \end{enumerate}
\end{frame}
%%%%%%%%%%%%%%%%%%%%%%%%%%%
\section{まとめ}
\begin{frame}[plain]{まとめ}
\begin{block}{Topics for Today}

if \textipa{/If/}
\begin{itemize}\setbeamertemplate{items}[square]\small
 \item   \visible{\Circled[fill color = white]{ if s $+$ v }\,は「もし~なら」という意味}
 \item   \visible{\Circled[fill color = white]{ if s $+$ v }\,は全体で「副詞」の役割}
 \item   \visible{\Circled[fill color = white]{ if s $+$ v }\,は、前にも後ろにも位置します}
	 \begin{enumerate}\setbeamertemplate{items}[circle]
	  \item {\bfseries S $+$ V}\,\,\Circled[fill color = white]{ if s $+$ v }
	  \item \Circled[fill color = white]{ If s $+$ v }\,,\,\,{\bfseries S $+$ V}\hfill{}コンマに注意!
.	 \end{enumerate}
 \item {\bfseries If}は2組の\,\,\Circled[fill color=white]{\,主語と動詞の組合せ\,}\,\,をつなぐ\kenten{接続詞}
\end{itemize}
     \end{block}

\pause
\textdbend If you work hard, you will succeed.で中心となる主語と動詞はどれですか
\end{frame}
%%%%%%%%%%%%%%%%%%%%%%%%%%%%%%%%
\end{document}
