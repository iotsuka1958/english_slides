\documentclass[aspectratio=169,xcolor={dvipsnames,table}]{beamer}
\usepackage[no-math,deluxe,haranoaji]{luatexja-preset}
\renewcommand{\kanjifamilydefault}{\gtdefault}
\renewcommand{\emph}[1]{{\upshape\bfseries #1}}
\usetheme{metropolis}
\metroset{block=fill}
\setbeamertemplate{navigation symbols}{}
\setbeamertemplate{blocks}[rounded][shadow=false]
\usecolortheme[rgb={0.7,0.2,0.2}]{structure}
%%%%%%%%%%%%%%%%%%%%%%%%%%
%% Change alert block colors
%%% 1- Block title (background and text)
\setbeamercolor{block title alerted}{fg=mDarkTeal, bg=mLightBrown!45!yellow!45}
\setbeamercolor{block title example}{fg=magenta!10!black, bg=mLightGreen!70}
%%% 2- Block body (background)
\setbeamercolor{block body alerted}{bg=mLightBrown!25}
\setbeamercolor{block body example}{bg=mLightGreen!15}
%%%%%%%%%%%%%%%%%%%%%%%%%%%
%%%%%%%%%%%%%%%%%%%%%%%%%%%
%% さまざまなアイコン
%%%%%%%%%%%%%%%%%%%%%%%%%%%
%\usepackage{fontawesome}
\usepackage{fontawesome5}
\usepackage{figchild}
\usepackage{twemojis}
\usepackage{utfsym}
\usepackage{bclogo}
\usepackage{marvosym}
\usepackage{fontmfizz}
\usepackage{pifont}
\usepackage{phaistos}
\usepackage{worldflags}
\usepackage{jigsaw}
\usepackage{tikzlings}
\usepackage{tikzducks}
\usepackage{scsnowman}
\usepackage{epsdice}
\usepackage{halloweenmath}
\usepackage{svrsymbols}
\usepackage{countriesofeurope}
\usepackage{tipa}
\usepackage{manfnt}
%%%%%%%%%%%%%%%%%%%%%%%%%%%
\usepackage{tikz}
\usetikzlibrary{calc,patterns,decorations.pathmorphing,backgrounds}
\usepackage{tcolorbox}
\usepackage{tikzpeople}
\usepackage{circledsteps}
\usepackage{xcolor}
\usepackage{amsmath}
\usepackage{booktabs}
\usepackage{chronology}
\usepackage{signchart}
%%%%%%%%%%%%%%%%%%%%%%%%%%%
%% 場合分け
%%%%%%%%%%%%%%%%%%%%%%%%%%%
\usepackage{cases}
%%%%%%%%%%%%%%%%%%%%%%%%%%
\usepackage{pdfpages}
%%%%%%%%%%%%%%%%%%%%%%%%%%%
%% 音声リンク表示
\newcommand{\myaudio}[1]{\href{#1}{\faVolumeUp}}
%%%%%%%%%%%%%%%%%%%%%%%%%%
%% \myAnch{<名前>}{<色>}{<テキスト>}
%% 指定のテキストを指定の色の四角枠で囲み, 指定の名前をもつTikZの
%% ノードとして出力する. 図には remember picture 属性を付けている
%% ので外部から参照可能である.
\newcommand*{\myAnch}[3]{%
  \tikz[remember picture,baseline=(#1.base)]
    \node[draw,rectangle,line width=1pt,#2] (#1) {\normalcolor #3};
}
%%%%%%%%%%%%%%%%%%%%%%%%%%
%% \myEmph コマンドの定義
%%%%%%%%%%%%%%%%%%%%%%%%%%
%\newcommand{\myEmph}[3]{%
%    \textbf<#1>{\color<#1>{#2}{#3}}%
%}
\usepackage{xparse} % xparseパッケージの読み込み
\NewDocumentCommand{\myEmph}{O{} m m}{%
    \def\argOne{#1}%
    \ifx\argOne\empty
        \textbf{\color{#2}{#3}}% オプション引数が省略された場合
    \else
        \textbf<#1>{\color<#1>{#2}{#3}}% オプション引数が指定された場合
    \fi
}
%%%%%%%%%%%%%%%%%%%%%%%%%%%
%%%%%%%%%%%%%%%%%%%%%%%%%%%
%% 文末の上昇イントネーション記号 \myRisingPitch
%% 通常のイントネーション \myDownwardPitch
%% https://note.com/dan_oyama/n/n8be58e8797b2
%%%%%%%%%%%%%%%%%%%%%%%%%%%
\newcommand{\myRisingPitch}{
\begin{tikzpicture}[scale=0.3,baseline=0.3]
\draw[->,>=stealth] (0,0) to[bend right=45] (1,1);
\end{tikzpicture}
}
\newcommand{\myDownwardPitch}{
\begin{tikzpicture}[scale=0.3,baseline=0.3]
\draw[->,>=stealth] (0,1) to[bend left=45] (1,0);
\end{tikzpicture}
}
%%%%%%%%%%%%%%%%%%%%%%%%%%%%
%\AtBeginSection[%
%]{%
%  \begin{frame}[plain]\frametitle{授業の流れ}
%     \tableofcontents[currentsection]
%   \end{frame}%
%}

%%%%%%%%%%%%%%%%%%%%%%%%%%%
\title{English is fun.}
\subtitle{Sleeping is necessary to us.}
\author{}
\institute[]{}
\date[]

%%%%%%%%%%%%%%%%%%%%%%%%%%%%
%% TEXT
%%%%%%%%%%%%%%%%%%%%%%%%%%%%
\begin{document}

\begin{frame}[plain]
  \titlepage
\end{frame}

\section*{授業の流れ}
\begin{frame}[plain]
  \frametitle{授業の流れ}
  \tableofcontents
\end{frame}

\section{動名詞}
\subsection{主語}
%%%%%%%%%%%%%%%%%%%%%%%%%%%%%%%%%%%%%%%%%%%%%
\begin{frame}[plain]{~すること}
 \large


\begin{enumerate}
 \item<1-> Exercise is good for your health.\hfill{\scriptsize execise: 運動 health: 健康}
 \item<2-> \alt<3->{\fbox{To eat vegetables}}{To eat vegetables} is good for your health.\hfill\visible<4->{\scriptsize 不定詞の名詞適用法}
 \item<5-> \alt<6->{\fbox{Eating vegetables}}{Eating vegetables} is good for your health.\hfill\visible<7->{{\scriptsize \fbox{Eating vegetables} $=$ S}}
 \item<8-> His dream is \alt<9->{\fbox{becoming a doctor}}{becoming a doctor}.\hfill\visible<10->{{\scriptsize His dream $=$ \fbox{becoming a doctor}}}
\end{enumerate}

\begin{exampleblock}{Topics for Today}
\begin{itemize}\small
 \item<11->   \Circled[fill color= white]{\,\,--ing\,\,} が「~すること」という意味を表し、名詞のはたらきをすることがある
 \item<12-> \Circled[fill color= white]{\,\,--ing\,\,}\,が、
      \begin{itemize}
       \item 文全体の主語になったり
       \item be動詞の直後にきたりすることがあります
      \end{itemize}
 \item <13-> この \Circled[fill color= white]{\,\,--ing\,\,}\,を「動名詞」といいます
 \end{itemize}
     \end{exampleblock}
\end{frame}
%%%%%%%%%%%%%%%%%%%
\begin{frame}[plain]{Exercises}
日本文の意味になるように(~~~~~)内の語句を並べ替えましょう
 \begin{enumerate}
  \item 走ることは健康にいい。\\
	( your health / is / good / running / for ).\\
	\visible<2->{Running is good for your health.}
  \item 川で泳ぐことは楽しい。\\
	(the river / fun / swimming / is / in ).\\
	\visible<3->{Swimming in the river is fun.}
  \item 私のお気に入りの趣味は読書です。\\
	My ( reading / hobby / is / favorite ).\\
	\visible<4->{My favorite hobby is reading.}
  \item 彼女の仕事はバスを運転することだ。
	Her ( driving / job / a bus / is ).\\
	\visible<5->{Her job is driving a bus.}

 \end{enumerate}
\end{frame}
%%%%%%%%%%%%%%%%%%%%%%%%%%%%%%%%%%%%%%%%
\begin{frame}[plain]{Exercises}
つぎの2文の意味を考えましょう\dbend

 \begin{enumerate}
  \item Her job was driving a bus.
  \item She was driving a bus at that time.
 \end{enumerate}
\end{frame}
%%%%%%%%%%%%%%%%%%%%%%%%%%%%%%%%%%%%%%%%
\begin{frame}[plain]{V $+$ --ing}
 \begin{enumerate}
  \item He started his business two years ago.\hfill{\scriptsize business: 商売}
  \item He started to laugh.\hfill{\scriptsize laugh: 笑う}
  \item He started laughing.

\begin{exampleblock}{Topic for Today}
\begin{itemize}\small
 \item \Circled[fill color= white]{\,\,--ing\,\,}\,\,が、「〜すること」の意味で
      \begin{itemize}
       \item 一般動詞の直後にくることがあります
      \end{itemize}
 \end{itemize}
     \end{exampleblock}
 \end{enumerate}
\end{frame}
%%%%%%%%%%%%%%%%%%%%%%%%%%%%
\begin{frame}[plain]{Exercises}
 \begin{enumerate}
  \item \begin{enumerate}
	 \item She likes music.
	 \item She likes to play the piano.
	 \item She likes playing the piano.      
	\end{enumerate}
  \item \begin{enumerate}
	 \item They began the lesson.
	 \item They began to study Spanish.
	 \item They began studying Spanish.
	\end{enumerate}
 \end{enumerate}
\end{frame}
%%%%%%%%%%%%%%%%%%%%%%%%%%%%%
\end{document}
