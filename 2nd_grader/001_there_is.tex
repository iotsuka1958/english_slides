\documentclass[aspectratio=169]{beamer}
\usepackage[no-math,deluxe,haranoaji]{luatexja-preset}
\renewcommand{\kanjifamilydefault}{\gtdefault}
\renewcommand{\emph}[1]{{\upshape\bfseries #1}}
\usetheme{metropolis}
\metroset{block=fill}
\setbeamertemplate{navigation symbols}{}
\usecolortheme[rgb={0.7,0.2,0.2}]{structure}
%%%%%%%%%%%%%%%%%%%%%%%%%%%
\usepackage{media9}
%%%%%%%%%%%%%%%%%%%%%%%%%%%
%% さまざまなアイコン
%%%%%%%%%%%%%%%%%%%%%%%%%%%
\usepackage{fontawesome}
\usepackage{figchild}
\usepackage{twemojis}
\usepackage{utfsym}
\usepackage{bclogo}
\usepackage{marvosym}
%%%%%%%%%%%%%%%%%%%%%%%%%%%
\usepackage{tikz}
\usetikzlibrary{backgrounds}
\usepackage{tcolorbox}
\usepackage{tikzpeople}
\usepackage{xcolor}
\usepackage{amsmath}
%%%%%%%%%%%%%%%%%%%%%%%%%%%
%% 場合分け
\usepackage{cases}
%%%%%%%%%%%%%%%%%%%%%%%%%%%
% \myAnch{<名前>}{<色>}{<テキスト>}
% 指定のテキストを指定の色の四角枠で囲み, 指定の名前をもつTikZの
% ノードとして出力する. 図には remeber picture 属性を付けている
% ので外部から参照可能である.
\newcommand*{\myAnch}[3]{%
  \tikz[remember picture,baseline=(#1.base)]
    \node[draw,rectangle,#2] (#1) {\normalcolor #3};
}
%%%%%%%%%%%%%%%%%%%%%%%%%%%%
%% 音声リンク表示
\newcommand{\myaudio}[1]{\href{#1}{\faVolumeUp}}
%%%%%%%%%%%%%%%%%%%%%%%%%%%
% \myEmph コマンドの定義
%\newcommand{\myEmph}[3]{%
%    \textbf<#1>{\color<#1>{#2}{#3}}%
%}
\usepackage{xparse} % xparseパッケージの読み込み
\NewDocumentCommand{\myEmph}{O{} m m}{%
    \def\argOne{#1}%
    \ifx\argOne\empty
        \textbf{\color{#2}{#3}}% オプション引数が省略された場合
    \else
        \textbf<#1>{\color<#1>{#2}{#3}}% オプション引数が指定された場合
    \fi
}
%%%%%%%%%%%%%%%%%%%%%%%%%%%
%% 文末の上昇イントネーション記号 \myRisingPitch
%% 通常のイントネーション \myDownwardPitch
%% https://note.com/dan_oyama/n/n8be58e8797b2
%%%%%%%%%%%%%%%%%%%%%%%%%%%
\newcommand{\myRisingPitch}{
\begin{tikzpicture}[scale=0.3,baseline=0.3]
\draw[->,>=stealth] (0,0) to[bend right=45] (1,1);
\end{tikzpicture}
}
\newcommand{\myDownwardPitch}{
\begin{tikzpicture}[scale=0.3,baseline=0.3]
\draw[->,>=stealth] (0,1) to[bend left=45] (1,0);
\end{tikzpicture}
}


%%%%%%%%%%%%%%%%%%%%%%%%%%%
\title{English is fun.\,\,{}---There is a book on the table.---}
\author{}
\institute[]{}
\date[]

%%%%%%%%%%%%%%%%%%%%%%%%%%%%
%% TEXT
%%%%%%%%%%%%%%%%%%%%%%%%%%%%
\begin{document}
\begin{frame}[plain]
  \titlepage
\end{frame}

\section*{授業の流れ}
\begin{frame}[plain]
  \frametitle{授業の流れ}
  \tableofcontents
\end{frame}

\section{〜がある}

\begin{frame}[plain]\frametitle{〜がある}
\begin{columns}
\begin{column}{.45\textwidth}
\IfFileExists{./images/one_book.jpg}{%
\rotatebox{0}{\includegraphics[width=.8\textwidth]{./images/one_book.jpg}
}}{naiyo}
\end{column}\pause
\begin{column}{.5\textwidth}\LARGE
There is a book on the table.
\end{column}
\end{columns}

\end{frame}


\begin{frame}[plain]\frametitle{〜がある}
\begin{columns}
\begin{column}{.45\textwidth}
\IfFileExists{./images/six_books.jpg}{%
\rotatebox{0}{\includegraphics[width=1.1\textwidth]{./images/six_books.jpg}
}}{naiyo}
\end{column}\pause
\begin{column}{.5\textwidth}\LARGE
There are six books on the table.
\end{column}
\end{columns}
\end{frame}

\subsection{There is X($=$単数) + 場所}
\begin{frame}[plain]\frametitle{There is a book on the table.}
 \Large

There \myEmph[4]{orange}{is} \myAnch{hako1}{black}{~X\,\,} $+$ \myAnch{hako2}{black}{\small 場所を表す表現} .\pause\hfill{} \fbox{~X~} が(どこどこに)ある\pause

\bigskip

\mbox{}\hspace{100pt}%
\myAnch{subject}{black}{%
$\left\{\begin{array}{l}
\text{a book}\\
\text{an apple}\\
\text{a ball}\\
\text{a cat \ldots}
	 \end{array} \right.$}
%
\hspace{50.pt}%
\myAnch{place}{black}{%
$\left\{\begin{array}{l}
\text{on the table}\\
\text{on the chair}\\
\text{under the chair}\\
\text{in the box \ldots}
	 \end{array} \right.$}




\begin{tikzpicture}[remember picture, overlay]
 \draw[thick,->] (subject.west) to[out=180,in=-90]node[midway,sloped,below] {\footnotesize \myEmph[4]{orange}{単数形}} (hako1.south);
 \draw[thick,->] (place.west) to[out=120,in=-45] (hako2.south east);
\end{tikzpicture}

\end{frame}


\subsection{There is X($=$複数) + 場所}
\begin{frame}[plain]\frametitle{There are two books on the table.}
 \Large

There \myEmph[3]{orange}{are} \myAnch{hako1}{black}{~X\,\,} $+$ \myAnch{hako2}{black}{\small 場所を表す表現} .\pause\hfill{} \fbox{~X~} が(どこどこに)ある\pause

\bigskip

\bigskip

\mbox{}\hspace{95pt}%
\myAnch{subject}{black}{%
$\left\{\begin{array}{l}
\text{two books}\\
\text{three apples}\\
\text{four balls}\\
\text{five cats \ldots}
	 \end{array} \right.$}
%
%\hspace{45pt}%
%\myAnch{place}{black}{%
%$\left\{\begin{array}{l}
%\text{on the table}\\
%\text{on the chair}\\
%\text{under the chair}\\
%\text{in the box \ldots}
%	 \end{array} \right.$}




\begin{tikzpicture}[remember picture, overlay]
 \draw[thick,->] (subject.west) to[out=180,in=-90]node[midway,sloped,below] {\footnotesize \myEmph[3]{orange}{複数形}} (hako1.south);
% \draw[thick,->] (place.north west) to[out=125,in=-45] (hako2.south east);
\end{tikzpicture}

\end{frame}

\subsection{Exercises}
\begin{frame}<1-15>[plain]\frametitle{Exercises}
\onslide*<1->{空所にisまたはareを補い、正しい英文にしましょう。あわせて、全体の意味を考えましょう。}
 % \setbeamercovered{transparent}
  \begin{enumerate}
   \item<2-> There (\onslide<3->{\myEmph[3]{orange}{~are~}}) three high schools in our city.
   \item<4->  There (\onslide<5->{\myEmph[5]{orange}{~is~}}) a new computer on the desk.
   \item<6-> There (\onslide<7->{\myEmph[7]{orange}{~are~}}) two boys in the park.
   \item<8-> There (\onslide<9->{\myEmph[9]{orange}{~is~}}) a cat by the tree now.
   \item<10-> There (\onslide<11->{\myEmph[11]{orange}{~is~}}) a station near my house.
  \end{enumerate}


\begin{exampleblock}<12->{Topics for Today}
\begin{itemize}
 \item<13-> There is \myEmph[11]{orange}{単数形}.
 \item<14-> There are \myEmph[12]{orange}{複数形}.
\end{itemize}
\end{exampleblock}

\vspace*{-15pt}
% Embed the sound file
\onslide<15>{%
\myaudio{audio/001_there_is_01.mp3}\,\,{}Listen carefully.(注意して聞いてください)

}
\end{frame}

\section{There was(were) X ...}
\begin{frame}[plain]{〜があった}

\begin{enumerate}
 \item There was a cat in the garden yesterday.\pause
 \item There was a river behind the house.\pause
 \item There were two big trees in the park.\pause
 \item There were three pictures on the wall.
\end{enumerate}

\pause
\begin{exampleblock}{Topics for Today}
過去のことをあらわすとき
\begin{itemize}
 \item There \textcolor{orange}{was 単数形}.\pause
 \item There \textcolor{orange}{were 複数形}.
\end{itemize}
\end{exampleblock}
\pause
\myaudio{audio/001_there_is_02.mp3}\,\,{}Listen carefully.(注意して聞いてください


\end{frame}

\section{There is(are, was, were) not X ...}
\begin{frame}[plain]{〜がない}

\begin{enumerate}[<+->]
 \item There is not a pen on the table.
 \item There was not a cat in the garden yesterday.
 \item There are not many cars on the road today.
 \item There were not many clouds in the sky yesterday.
\end{enumerate}

\pause 
\begin{exampleblock}<5->{Topics for Today}
「〜がない(なかった)」
\begin{itemize}
 \item There $\left\{\begin{array}{l}\text{is}\\\text{was}\end{array}
\right\}$ \textcolor{orange}{not} 単数形.\hspace{40pt}\pause
There $\left\{\begin{array}{l}\text{are}\\\text{were}\end{array}
\right\}$ \textcolor{orange}{not} 複数形.
\end{itemize}
\end{exampleblock}

\pause
% Embed the sound file
\myaudio{audio/001_there_is_03.mp3}\,\,{}Listen carefully.(注意して聞いてください\end{frame}

\section{Is there X ...?}
\subsection{〜がありますか}
\begin{frame}[plain]{〜がありますか?}
 \Large

疑問文のつくり方
\vspace{20pt}

\myAnch{there-1}{orange}{There} \myAnch{is-1}{olive}{is} a book on the table.
\vspace{50pt}\pause

\myAnch{is-2}{olive}{Is} \myAnch{there-2}{orange}{there} a book on the table \myAnch{question}{orange}{?}
\pause
\begin{tikzpicture}[remember picture, overlay]
 \draw[thick, orange, ->] (there-1.south) to[out=-90, in=90] (there-2.north); \pause
 \draw[thick, olive, ->] (is-1.south) to[out=-90, in=90] (is-2.north);
\pause
\node at (-6.5,-1) {逆にするだけ};
\pause
\node at (0.5,0.25) {\scalebox{2}{\myRisingPitch}};
\pause
\node at (2.75,-1) {?と最後のイントネーションに注意};
\end{tikzpicture}

\end{frame}

 \subsection{Exercises}
\begin{frame}[plain]{Exercises}

つぎの文の意味を考えましょう\pause
\begin{enumerate}[<+->]
 \item Is there a good restaurant in your town?
 \item Are there many students in your class?
 \item Was there a message for me?
 \item Were there many people at the concert last night?
\end{enumerate}

\pause
\myaudio{audio/001_there_is_04.mp3}\,\,{}Listen carefully.(注意して聞いてください\end{frame}


\begin{frame}[plain]\frametitle{Exercises}
各文を疑問文にしてください。\pause

  \begin{enumerate}
   \item There is a cat in the garden.\\
         \pause
         $\longrightarrow$ Is there a cat in the garden?\pause
   \item There are many flowers on the table.\\
         \pause
         $\longrightarrow$ Are there many flowers on the table?\pause
   \item There was a dog under the table.\\
         \pause
         $\longrightarrow$ Was there a dog under the table?\pause
   \item There were many people at the party.\\
         \pause
         $\longrightarrow$ Were there many people at the party?\pause
  \end{enumerate}


% Embed the sound file

\myaudio{audio/001_there_is_05.mp3}\,\,{}Listen carefully.(注意して聞いてください)


\end{frame}

\subsection{疑問文への答え方}
 \begin{frame}[plain]{Yer, there is. / No, there isn't.}
 \Large

疑問文にたいする答え方
\vspace{10pt}

\pause

Is there a dog in the garden?

\pause

$\left\{\begin{array}{l}
	 \text{Yes, there is.}\\\pause
         \text{No, there isn't.}
	\end{array}\right.$

\pause

\mbox{}\hfill{}{\small No, there isn't. はNo, there is not.ともいいます}
\end{frame}


 \begin{frame}[plain]{Yer, there was. / No, there wasn't.}
 \Large

疑問文にたいする答え方
\vspace{10pt}

\pause
Was there a cat on the desk?

\pause

$\left\{\begin{array}{l}
	 \text{Yes, there was.}\\\pause
         \text{No, there wasn't.}
	\end{array}\right.$

\pause

\mbox{}\hfill{}{\small No, there wasn't. はNo, there was not.ともいいます}
\end{frame}

 \begin{frame}[plain]{Yer, there ares. / No, there aren't.}
 \Large

疑問文にたいする答え方
\vspace{10pt}

\pause

Are there many people in the garden?

\pause 

$\left\{\begin{array}{l}
	 \text{Yes, there are.}\\\pause
         \text{No, there aren't.}
	\end{array}\right.$

\pause

\mbox{}\hfill{}{\small No, there aren't. はNo, there are not.ともいいます}
\end{frame}

 \begin{frame}[plain]{Yer, there were. / No, there weren't.}
 \Large

疑問文にたいする答え方
\vspace{10pt}

\pause

Were there many students in the classroom?

\pause

$\left\{\begin{array}{l}
	 \text{Yes, there were.}\\\pause
         \text{No, there weren't.}
	\end{array}\right.$

\pause

\mbox{}\hfill{}{\small No, there weren't. はNo, there were not.ともいいます}
\end{frame}


\begin{frame}[plain]\frametitle{Exercises}
(~~~~~~)内の語句を並べ替えて、日本文と同じ内容を表す英文を作成してください。なお、先頭の単語は大文字で始めてください。\pause

  \begin{enumerate}
   \item わたしたちのクラスには,
アメリカ人の生徒が二人います。\\
         \pause
         ( are / two American students / there ) in our class.\\
         \pause
         \onslide<4->{There are two American students in our class.}
         \pause
   \item 5年前は、その角に教会がありました。\\
         \pause
         ( a church / was / there ) on the corner five years ago.\\
         \pause
         \onslide<7->{There was a church on the corner five years ago.}
         \pause
   \item この近くに病院はありますか。\\
         \pause
         ( a hospital / is / there ) near here?\\
         \pause
         \onslide<10>{Is there a hospital near here?}
  \end{enumerate}


% Embed the sound file

\myaudio{audio/001_there_is_06.mp3}\,\,{}Listen carefully.(注意して聞いてください)


\end{frame}
\end{document}
