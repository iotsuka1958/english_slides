\documentclass[aspectratio=169]{beamer}
\usepackage[no-math,deluxe,haranoaji]{luatexja-preset}
\renewcommand{\kanjifamilydefault}{\gtdefault}
\renewcommand{\emph}[1]{{\upshape\bfseries #1}}
\usetheme{metropolis}
\metroset{block=fill}
\setbeamertemplate{navigation symbols}{}
\usecolortheme[rgb={0.7,0.2,0.2}]{structure}
%%%%%%%%%%%%%%%%%%%%%%%%%%%
\usepackage{media9}
%%%%%%%%%%%%%%%%%%%%%%%%%%%
%% さまざまなアイコン
%%%%%%%%%%%%%%%%%%%%%%%%%%%
\usepackage{fontawesome}
\usepackage{figchild}
\usepackage{twemojis}
\usepackage{utfsym}
\usepackage{bclogo}
\usepackage{marvosym}
%%%%%%%%%%%%%%%%%%%%%%%%%%%
\usepackage{tikz}
\usetikzlibrary{backgrounds}
\usepackage{tcolorbox}
\usepackage{tikzpeople}
\usepackage{xcolor}
\usepackage{amsmath}
%%%%%%%%%%%%%%%%%%%%%%%%%%%
%% 場合分け
\usepackage{cases}
%%%%%%%%%%%%%%%%%%%%%%%%%%%
% \myAnch{<名前>}{<色>}{<テキスト>}
% 指定のテキストを指定の色の四角枠で囲み, 指定の名前をもつTikZの
% ノードとして出力する. 図には remeber picture 属性を付けている
% ので外部から参照可能である.
\newcommand*{\myAnch}[3]{%
  \tikz[remember picture,baseline=(#1.base)]
    \node[draw,rectangle,#2] (#1) {\normalcolor #3};
}
%%%%%%%%%%%%%%%%%%%%%%%%%%%%
%% 音声リンク表示
\newcommand{\myaudio}[1]{\href{#1}{\faVolumeUp}}
%%%%%%%%%%%%%%%%%%%%%%%%%%%
% \myEmph コマンドの定義
%\newcommand{\myEmph}[3]{%
%    \textbf<#1>{\color<#1>{#2}{#3}}%
%}
\usepackage{xparse} % xparseパッケージの読み込み
\NewDocumentCommand{\myEmph}{O{} m m}{%
    \def\argOne{#1}%
    \ifx\argOne\empty
        \textbf{\color{#2}{#3}}% オプション引数が省略された場合
    \else
        \textbf<#1>{\color<#1>{#2}{#3}}% オプション引数が指定された場合
    \fi
}
%%%%%%%%%%%%%%%%%%%%%%%%%%%
\title{English is fun.\,\,{}---There is a book on the table.---}
\author{}
\institute[]{}
\date[]

%%%%%%%%%%%%%%%%%%%%%%%%%%%%
%% TEXT
%%%%%%%%%%%%%%%%%%%%%%%%%%%%
\begin{document}
\begin{frame}[plain]
  \titlepage
\end{frame}

\section*{授業の流れ}
\begin{frame}[plain]
  \frametitle{授業の流れ}
  \tableofcontents
\end{frame}

\section{〜がある}

\begin{frame}[plain]\frametitle{〜がある}
\begin{columns}
\begin{column}{.45\textwidth}
\IfFileExists{./images/one_book.jpg}{%
\rotatebox{0}{\includegraphics[width=.8\textwidth]{./images/one_book.jpg}
}}{naiyo}
\end{column}\pause
\begin{column}{.5\textwidth}\LARGE
There is a book on the table.
\end{column}
\end{columns}

\end{frame}


\begin{frame}[plain]\frametitle{〜がある}
\begin{columns}
\begin{column}{.45\textwidth}
\IfFileExists{./images/six_books.jpg}{%
\rotatebox{0}{\includegraphics[width=1.1\textwidth]{./images/six_books.jpg}
}}{naiyo}
\end{column}\pause
\begin{column}{.5\textwidth}\LARGE
There are six books on the table.
\end{column}
\end{columns}
\end{frame}

\subsection{There is X($=$単数) + 場所}
\begin{frame}[plain]\frametitle{There is a book on the table.}
 \Large

There \myEmph[4]{orange}{is} \myAnch{hako1}{black}{~X\,\,} $+$ \myAnch{hako2}{black}{\small 場所を表す表現} .\pause\hfill{} \fbox{~X~} が(どこどこに)ある\pause

\bigskip

\mbox{}\hspace{100pt}%
\myAnch{subject}{black}{%
$\left\{\begin{array}{l}
\text{a book}\\
\text{an apple}\\
\text{a ball}\\
\text{a cat \ldots}
	 \end{array} \right.$}
%
\hspace{50.pt}%
\myAnch{place}{black}{%
$\left\{\begin{array}{l}
\text{on the table}\\
\text{on the chair}\\
\text{onder the chair}\\
\text{in the box \ldots}
	 \end{array} \right.$}




\begin{tikzpicture}[remember picture, overlay]
 \draw[thick,->] (subject.west) to[out=180,in=-90]node[midway,sloped,below] {\footnotesize \myEmph[4]{orange}{単数形}} (hako1.south);
 \draw[thick,->] (place.west) to[out=120,in=-45] (hako2.south east);
\end{tikzpicture}

\end{frame}


\subsection{There is X($=$複数) + 場所}
\begin{frame}[plain]\frametitle{There are two books on the table.}
 \Large

There \myEmph[3]{orange}{are} \myAnch{hako1}{black}{~X\,\,} $+$ \myAnch{hako2}{black}{\small 場所を表す表現} .\pause\hfill{} \fbox{~X~} が(どこどこに)ある\pause

\bigskip

\bigskip

\mbox{}\hspace{95pt}%
\myAnch{subject}{black}{%
$\left\{\begin{array}{l}
\text{two books}\\
\text{three apples}\\
\text{four balls}\\
\text{five cats \ldots}
	 \end{array} \right.$}
%
%\hspace{45pt}%
%\myAnch{place}{black}{%
%$\left\{\begin{array}{l}
%\text{on the table}\\
%\text{on the chair}\\
%\text{under the chair}\\
%\text{in the box \ldots}
%	 \end{array} \right.$}




\begin{tikzpicture}[remember picture, overlay]
 \draw[thick,->] (subject.west) to[out=180,in=-90]node[midway,sloped,below] {\footnotesize \myEmph[3]{orange}{複数形}} (hako1.south);
% \draw[thick,->] (place.north west) to[out=125,in=-45] (hako2.south east);
\end{tikzpicture}

\end{frame}








5\begin{frame}<1-15>[plain]\frametitle{Exercises}
\onslide*<1->{空所にisまたはareを補い、正しい英文にしましょう。あわせて、全体の意味を考えましょう。}
 % \setbeamercovered{transparent}
  \begin{enumerate}
   \item<2-> There (\onslide<3->{\myEmph[3]{orange}{~are~}}) three high schools in our city.
   \item<4->  There (\onslide<5->{\myEmph[5]{orange}{~is~}}) a new computer on the desk.
   \item<6-> There (\onslide<7->{\myEmph[7]{orange}{~are~}}) two boys in the park this morning.
   \item<8-> There (\onslide<9->{\myEmph[9]{orange}{~is~}}) a cat by the tree now.
   \item<10-> There (\onslide<11->{\myEmph[11]{orange}{~is~}}) a station near my house.
  \end{enumerate}


\begin{exampleblock}<12->{Topics for Today}
\begin{itemize}
 \item<13-> There is \myEmph[11]{orange}{単数形}.
 \item<14-> There are \myEmph[12]{orange}{複数形}.
\end{itemize}
\end{exampleblock}

\vspace*{-15pt}
% Embed the sound file
\onslide<15>{%
\myaudio{audio/001_there_is__01.mp3}\,\,{}Listen carefully.(注意して聞いてください)

}
\end{frame}

\end{document}
