\documentclass[aspectratio=169,xcolor={dvipsnames,table}]{beamer}
\usepackage[no-math,deluxe,haranoaji]{luatexja-preset}
\renewcommand{\kanjifamilydefault}{\gtdefault}
\renewcommand{\emph}[1]{{\upshape\bfseries #1}}
\usetheme{metropolis}
\metroset{block=fill}
\setbeamertemplate{navigation symbols}{}
\setbeamertemplate{blocks}[rounded][shadow=false]
\usecolortheme[rgb={0.7,0.2,0.2}]{structure}
%%%%%%%%%%%%%%%%%%%%%%%%%%
%% Change alert block colors
%%% 1- Block title (background and text)
\setbeamercolor{block title alerted}{fg=mDarkTeal, bg=mLightBrown!45!yellow!45}
\setbeamercolor{block title example}{fg=magenta!10!black, bg=mLightGreen!70}
%%% 2- Block body (background)
\setbeamercolor{block body alerted}{bg=mLightBrown!25}
\setbeamercolor{block body example}{bg=mLightGreen!15}
%%%%%%%%%%%%%%%%%%%%%%%%%%%
%%%%%%%%%%%%%%%%%%%%%%%%%%%
%% さまざまなアイコン
%%%%%%%%%%%%%%%%%%%%%%%%%%%
%\usepackage{fontawesome}
\usepackage{fontawesome5}
\usepackage{figchild}
\usepackage{twemojis}
\usepackage{utfsym}
\usepackage{bclogo}
\usepackage{marvosym}
\usepackage{fontmfizz}
\usepackage{pifont}
\usepackage{phaistos}
\usepackage{worldflags}
\usepackage{jigsaw}
\usepackage{tikzlings}
\usepackage{tikzducks}
\usepackage{scsnowman}
\usepackage{epsdice}
\usepackage{halloweenmath}
\usepackage{svrsymbols}
\usepackage{countriesofeurope}
\usepackage{tipa}
\usepackage{manfnt}
%%%%%%%%%%%%%%%%%%%%%%%%%%%
\usepackage{tikz}
\usetikzlibrary{calc,patterns,decorations.pathmorphing,backgrounds}
\usepackage{tcolorbox}
\usepackage{tikzpeople}
\usepackage{circledsteps}
\usepackage{xcolor}
\usepackage{amsmath}
\usepackage{booktabs}
\usepackage{chronology}
\usepackage{signchart}
%%%%%%%%%%%%%%%%%%%%%%%%%%%
%% 場合分け
%%%%%%%%%%%%%%%%%%%%%%%%%%%
\usepackage{cases}
%%%%%%%%%%%%%%%%%%%%%%%%%%
\usepackage{pdfpages}
%%%%%%%%%%%%%%%%%%%%%%%%%%%
%% 音声リンク表示
\newcommand{\myaudio}[1]{\href{#1}{\faVolumeUp}}
%%%%%%%%%%%%%%%%%%%%%%%%%%
%% \myAnch{<名前>}{<色>}{<テキスト>}
%% 指定のテキストを指定の色の四角枠で囲み, 指定の名前をもつTikZの
%% ノードとして出力する. 図には remember picture 属性を付けている
%% ので外部から参照可能である.
\newcommand*{\myAnch}[3]{%
  \tikz[remember picture,baseline=(#1.base)]
    \node[draw,rectangle,line width=1pt,#2] (#1) {\normalcolor #3};
}
%%%%%%%%%%%%%%%%%%%%%%%%%%
%% \myEmph コマンドの定義
%%%%%%%%%%%%%%%%%%%%%%%%%%
%\newcommand{\myEmph}[3]{%
%    \textbf<#1>{\color<#1>{#2}{#3}}%
%}
\usepackage{xparse} % xparseパッケージの読み込み
\NewDocumentCommand{\myEmph}{O{} m m}{%
    \def\argOne{#1}%
    \ifx\argOne\empty
        \textbf{\color{#2}{#3}}% オプション引数が省略された場合
    \else
        \textbf<#1>{\color<#1>{#2}{#3}}% オプション引数が指定された場合
    \fi
}
%%%%%%%%%%%%%%%%%%%%%%%%%%%
%%%%%%%%%%%%%%%%%%%%%%%%%%%
%% 文末の上昇イントネーション記号 \myRisingPitch
%% 通常のイントネーション \myDownwardPitch
%% https://note.com/dan_oyama/n/n8be58e8797b2
%%%%%%%%%%%%%%%%%%%%%%%%%%%
\newcommand{\myRisingPitch}{
\begin{tikzpicture}[scale=0.3,baseline=0.3]
\draw[->,>=stealth] (0,0) to[bend right=45] (1,1);
\end{tikzpicture}
}
\newcommand{\myDownwardPitch}{
\begin{tikzpicture}[scale=0.3,baseline=0.3]
\draw[->,>=stealth] (0,1) to[bend left=45] (1,0);
\end{tikzpicture}
}
%%%%%%%%%%%%%%%%%%%%%%%%%%%%
%\AtBeginSection[%
%]{%
%  \begin{frame}[plain]\frametitle{授業の流れ}
%     \tableofcontents[currentsection]
%   \end{frame}%
%}

%%%%%%%%%%%%%%%%%%%%%%%%%%%
\title{English is fun.}
\subtitle{She was listening to the radio.}
\author{}
\institute[]{}
\date[]

%%%%%%%%%%%%%%%%%%%%%%%%%%%%
%% TEXT
%%%%%%%%%%%%%%%%%%%%%%%%%%%%
\begin{document}


\begin{frame}[plain]
  \titlepage
\end{frame}


\section*{授業の流れ}
\begin{frame}[plain]
  \frametitle{授業の流れ}
  \tableofcontents
\end{frame}

\section{過去進行形}

\subsection{過去進行形とは}

%%%%%%%%%%%%%%%%%%%%%%%%%%%%%%%%%%%%%%%%%%%%%
\begin{frame}[plain]{過去進行形}
 \begin{enumerate}
 \item \visible<1->{She \textcolor{ForestGreen}{\bfseries is sleeping} now. {\small 彼女はいま眠っています。\hfill{}(現在進行形)}}
\visible<2->{\signchart[width=10]{,,{\textcolor{ForestGreen}{いま}}}{,,,}}
 \item \visible<3->{She \textcolor{Maroon}{\bfseries was playing} tennis two hours ago.{\small 彼女は2時間前テニスをしていた。\\\mbox{}\hfill{}(過去進行形)}}
\visible<4->{\signchart[width=10,height=.5]{{\textcolor{Maroon}{{2時間前}}},,いま}{,,,}}
\end{enumerate}

\begin{tikzpicture}[overlay]
 %\draw[gray!50] (0,0) grid (12,5);
 \visible<2->{\draw[ForestGreen!70,line width=6pt,opacity=.5] (9.2,2.95) --(10.6,2.95);}
\visible<4->{ \draw[Maroon!70,line width=6pt,opacity=.7] (4.2,.9) -- (5.6,.9);}
\end{tikzpicture}

\visible<5->{{\small 
\mbox{}\hfill{%
}$\left\{\begin{tabular}[c]{l}
         was\\
         were
       \end{tabular}\right\} + \text{--ing}$
}
}


\vspace{10pt}

\visible<6->{%
\begin{exampleblock}{Topics for Today}
\pause
\begin{itemize}\setbeamertemplate{items}[square]\small
 \item 現在進行形(〜している)$\longrightarrow$今この瞬間に起こっていること
 \item 過去進行形(〜していた)$\longrightarrow$過去のある時点で起こっていること%
\hfill\myaudio{./audio/005_was_ing_intro_01.mp3}

\end{itemize}
     \end{exampleblock}
}
\end{frame}
%%%%%%%%%%%%%%%%%%%%%%%%%%%%%%%%%%%%%%%%%%%%%%
\begin{frame}[plain]{Exercises 1}
1は正しい方を選びましょう。2〜4は日本語の意味になるよう(~~~~~)内の語句を並べ替えましょう。先頭に来る語は大文字で書き始めてください
 \begin{enumerate}
  \item Bob ( studies / is studying / \alt<2->{\Circled[outer color = BurntOrange]{ was studying }}{was studying} ) English last night.
  \item ( is / she / sleeping ) now.彼女はいま眠っています。\\
\visible<3->{She is sleeping now.}
  \item ( I / waiting / was ) for the train then.私はそのとき電車を待っていました。\\
	\visible<4->{I was waiting for the train then.}%
\hfill{\scriptsize wait for ~を待つ}
  \item ( eating / dinner / was / with  / I  / my family ) at that time.\\
\mbox{}\hfill{}私はそのとき家族と夕飯を食べていました。\\
\visible<5->{I was eating dinner with my family at that time.}%
 \end{enumerate}
\hfill\myaudio{./audio/005_was_ing_intro_02.mp3}

\end{frame}
%%%%%%%%%%%%%%%%%%%%%%%%%%%%%%%%%%%%%%%%%%%%%
\begin{frame}[plain]{Exercises 2}
適当な動詞を(~~~~~)内に補い、英文を完成させましょう。必要に応じて適当な形に変えてください

\begin{columns}[t]
\begin{column}{.75\textwidth}
 \begin{enumerate}
  \item He was (~\alt<2->{playing}{\phantom{playing}}~) soccer at that time.
  \item She was  (~\alt<3->{making}{\phantom{making}}~) a delicious dinner for us then.
  \item I was  (~\alt<4->{writing}{\phantom{writing}}~) a letter to my friend at that time.
  \item They were  (~\alt<5->{swimming$^{1}$}{\phantom{swimming }}~) in the pool then.
  \item The dog was  (~\alt<7->{running$^{2}$}{\phantom{running }}~) in the yard at that time. \\
\hfill{}\visible<6->{{\scriptsize 1\visible<8->{,2} playingでもOK}}\hfill\mbox{}
 \end{enumerate} 
\end{column} 
\begin{column}{.225\textwidth}
次の中から\\選んでください
 \begin{tcolorbox}
  swim\\
  run\\
  make\\
  write\\
  play
 \end{tcolorbox}
\end{column}
\end{columns}

\visible<2->{\scriptsize play(遊ぶ) play the guitar(ギターを弾く) play soccer(サッカーをする)}

\hfill\myaudio{./audio/005_was_ing_intro_03.mp3}

\end{frame}
%%%%%%%%%%%%%%%%%%%%%%%%%%%%%%%%%%%%%%%%%%%%%
\begin{frame}[plain]{--ingのつくりかた(復習)}
 
\begin{center}
% \rowcolors{2}{NavyBlue!50}{yellow!40}
\begin{tblr}{
  colspec = {l l l l}, % 列の仕様
  row{5-6} = {bg=yellow!50}, % 1行目の背景色を黄色に設定
  row{7-8} = {bg=NavyBlue!40} % 3行目の背景色を薄い灰色に設定
}
\toprule
&{\small 原形}&{\small --ing}\\\midrule
1&\visible<1->{eat}&\visible<2->{{\small eating}}&\visible<2->{\scriptsize ingをつけるだけ (原則)}\\
2&\visible<1->{play}&\visible<3->{{\small playing}}&\\
3&\visible<1->{go}&\visible<4->{{\small going}}&\\
4&\visible<1->{make}&\visible<5->{{\small making}}&\visible<5->{\scriptsize eをとって--ing}\\
5&\visible<1->{write}&\visible<6->{{\small writing}}&\visible<6->{\scriptsize use $\rightarrow$\,using}\\
6&\visible<1->{swim}&\visible<7->{{\small swimming}}&\visible<7->{\scriptsize 子音字を重ねて--ing}\\
7&\visible<1->{run}&\visible<8->{{\small running}}&\visible<8->{\scriptsize sit $\rightarrow$\,sitting \ldots{}}\\
\bottomrule
\end{tblr}%
\end{center}
\hfill\myaudio{./audio/005_was_ing_intro_04.mp3}

\end{frame}
%%%%%%%%%%%%%%%%%%%%%%%%%%%%%%%%%%%%%%%%
\section{過去進行形の否定}
\begin{frame}[plain]{過去進行形の否定}
 \begin{enumerate}
  \item \begin{enumerate}
	 \item<1-> Bob was hungry.
	 \item<2-> Bob was \myEmph[2-]{Maroon}{not} hungry.
	\end{enumerate}
  \item \begin{enumerate}
	 \item<3-> Jane \myEmph[3-]{NavyBlue}{was waiting} for the train then.
	 \item<4-> Jane \myEmph[3-]{NavyBlue}{was} \myEmph[4-]{Maroon}{not} \myEmph[3-]{NavyBlue}{waiting} for the train then.
	\end{enumerate}
 \end{enumerate}

\begin{block}<5->{Topics for Today}\small
\begin{itemize}\setbeamertemplate{items}[square]
 \item  過去進行形\,\,\Circled[fill color = white]{\,\,$\text{be動詞}+\text{---ing}$\,\,}\,\,の否定$\longrightarrow$\,\,%
\Circled[fill color = white]{\,\,$\text{be動詞} + \text{\textcolor{Maroon}{\bfseries not}} + \text{---ing}$\,\,}
 \visible<6->{\item 短縮形wasn't ---ing, weren't ---ingが使われることもあります} \end{itemize}
      \end{block}
\hfill{}\onslide<7->{\scriptsize be動詞の直後に\textcolor{Maroon}{\bfseries not}がくるのは、be動詞の否定と同じです\hspace{1\zw}}


\hfill\myaudio{./audio/005_was_ing_intro_05.mp3}

\end{frame}
%%%%%%%%%%%%%%%%%%%%%%%%%%%%%%%%%%%%%%%%
\section{過去進行形の疑問文}
\begin{frame}[plain]{過去進行形の疑問文}
  \begin{enumerate}
  \item \begin{enumerate}
	 \item<1-> Bob was hungry.
	 \item<2-> Was Bob hungry?
	\end{enumerate}
  \item \begin{enumerate}
	 \item<3-> Jane \myEmph[3-]{NavyBlue}{was waiting} for the train then.
	 \item<4-> \myEmph[3-]{NavyBlue}{Was} Jane \myEmph[3-]{NavyBlue}{waiting} for the train then?
	 \item<5-> Yes, she was.\\
                   No, she was not.\\
                   No, she wasn't.
	\end{enumerate}
 \end{enumerate}

\begin{block}<6->{Topics for Today}
\begin{itemize}\setbeamertemplate{items}[square]\small
 \item 疑問文のつくり方: be動詞を先頭に
\hfill{}{{\bfseries Was} she {\bfseries sleeping} then?}
 \item 疑問文への答え方
\hfill{}Yes, she was.\\
\hfill{}No, she was not.\\
\hfill{}($=$ No, she wasn't.)
\end{itemize}
\end{block}

\hfill\myaudio{./audio/005_was_ing_intro_06.mp3}

\end{frame}
%%%%%%%%%%%%%%%%%%%%%%%%%%%%%%%%%%%%%%%%
\begin{frame}[plain]{Exercises}
 [~~~~~]内の指示にしたがって英文を書き換えましょう。
さらに3~5は、書き換えた疑問文に対する答えを「はい」と「いいえ」の2とおりつくりましょう\hfill\myaudio{./audio/005_was_ing_intro_07.mp3}

\begin{enumerate}
 \item Bob was playing the violin this morning. {\scriptsize [否定文にする]}\\
       \visible<2->{Bob was not playing the violin this morning.}
 \item We were studying science at that time. {\scriptsize [否定文にする]}\\
       \visible<3->{We were not studying science at that time.}
 \item He was walking in the park then. {\scriptsize [疑問文にする]}\\
       \visible<4->{Was he walking in the park then?}
       \visible<5->{--- Yes, he was. / No, he was not($= \text{wasn't}).$}
 \item She was swimming in the pool. {\scriptsize [疑問文にする]}\\
       \visible<6->{Was she swimming in the pool?}
       \visible<7->{--- Yes, she was. / No, she was not($= \text{wasn't}).$}
 \item The cats were sleeping then. {\scriptsize [疑問文にする]}\\
       \visible<8->{Were the cats sleeping then?}
       \visible<9->{--- Yes, they were. / No, they were not\makebox[0pt][l]{($= \text{weren't}$).}}

\end{enumerate}

\end{frame}
%%%%%%%%%%%%%%%%%%%%%%%%%%%%%%%%%%%%%%
\section{過去進行形のまとめ}
\begin{frame}[plain]{要点}
 
\begin{block}{was / were $+$ --ing}
\begin{itemize}\setbeamertemplate{items}[square]\small
 \item 過去進行形(〜していた)$\longrightarrow$過去のある時点で起こっていること\\%
\hfill{}She {\bfseries was sleeping} then.
 \item 否定文のつくり方: be動詞の直後にnot\hfill{}She {\bfseries was} \Circled[fill color = white]{\,not\,} {\bfseries sleeping} then.
 \item 疑問文のつくり方: be動詞を先頭に\hfill{}{\bfseries Was} she {\bfseries sleeping} then?
 \item 疑問文への答え方
\hfill{}Yes, she was.\\
\hfill{}No, she was not.\\
\hfill{}($=$ No, she wasn't.)
\end{itemize}
\hfill{{\scriptsize 否定文のつくり方・疑問文のつくり方・疑問文への答え方--みんなbe動詞の時と同じ}}
\end{block}

\end{frame}
%%%%%%%%%%%%%%%%%%%%%%%%%%%%%%%%%%%%%%%
\begin{frame}[plain]{疑問詞をもちいて}
 \begin{enumerate}
  \item What was she doing at that time? --- She was doing homework.
  \item Where was he running? --- In the park.
	\mbox{}\hfill{}cf. He was running in the park.
  \item Who was watching television? --- John was.
	\mbox{}\hfill{}cf. John was watching television.
 \end{enumerate}

\visible<5->{%
\begin{exampleblock}{Topics for Today}
\pause
\begin{itemize}\small
 \item $\text{疑問詞} + \text{be動詞} + \text{S} + \text{--ing\ldots ?}$
 \item ただし、疑問詞が主語のとき\\
\mbox{}\hspace{40pt}$\text{疑問詞($=$S)} + \text{be動詞} + \text{--ing\ldots ?}$

\end{itemize}
     \end{exampleblock}
}
\hfill\myaudio{./audio/005_was_ing_intro_08.mp3}
\end{frame}
%%%%%%%%%%%%%%%%%%%%%%%%%%%%%%%%%%%%%%%%
\begin{frame}[plain]{Exercises}
(~~~~~)の語句を並べ替えて対話を完成させましょう。文頭にくる語は大文字で書きはじめてください
\begin{enumerate}
 \item ( doing / you / were/ what ) last night?
--- I was studying history. \\
\visible<2->{What were you doing alast night?}
 \item ( was / what / doing / your brother ) then?
--- He was cleaning his room.\\
\visible<3->{What was your brother doing then?}
 \item ( was / the guitar / playing / who )? --- Paul was.\\
\visible<4->{Who was playing the guitar?}
\end{enumerate}
\hfill\myaudio{./audio/005_was_ing_intro_09.mp3}


\end{frame}
\end{document}
