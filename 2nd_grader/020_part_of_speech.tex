\documentclass[aspectratio=169,xcolor={dvipsnames,table}]{beamer}
\usepackage[no-math,deluxe,haranoaji]{luatexja-preset}
\renewcommand{\kanjifamilydefault}{\gtdefault}
\renewcommand{\emph}[1]{{\upshape\bfseries #1}}
\usetheme{metropolis}
\metroset{block=fill}
\setbeamertemplate{navigation symbols}{}
\setbeamertemplate{blocks}[rounded][shadow=false]
\usecolortheme[rgb={0.7,0.2,0.2}]{structure}
%%%%%%%%%%%%%%%%%%%%%%%%%%
%% Change alert block colors
%%% 1- Block title (background and text)
\setbeamercolor{block title alerted}{fg=mDarkTeal, bg=mLightBrown!45!yellow!45}
\setbeamercolor{block title example}{fg=magenta!10!black, bg=mLightGreen!70}
%%% 2- Block body (background)
\setbeamercolor{block body alerted}{bg=mLightBrown!25}
\setbeamercolor{block body example}{bg=mLightGreen!15}
%%%%%%%%%%%%%%%%%%%%%%%%%%%
%%%%%%%%%%%%%%%%%%%%%%%%%%%
%% さまざまなアイコン
%%%%%%%%%%%%%%%%%%%%%%%%%%%
%\usepackage{fontawesome}
\usepackage{fontawesome5}
\usepackage{figchild}
\usepackage{twemojis}
\usepackage{utfsym}
\usepackage{bclogo}
\usepackage{marvosym}
\usepackage{fontmfizz}
\usepackage{pifont}
\usepackage{phaistos}
\usepackage{worldflags}
\usepackage{jigsaw}
\usepackage{tikzlings}
\usepackage{tikzducks}
\usepackage{scsnowman}
\usepackage{epsdice}
\usepackage{halloweenmath}
\usepackage{svrsymbols}
\usepackage{countriesofeurope}
\usepackage{tipa}
\usepackage{manfnt}
%%%%%%%%%%%%%%%%%%%%%%%%%%%
\usepackage{tikz}
\usetikzlibrary{calc,patterns,decorations.pathmorphing,backgrounds}
\usepackage{tcolorbox}
\usepackage{tikzpeople}
\usepackage{circledsteps}
\usepackage{xcolor}
\usepackage{amsmath}
\usepackage{booktabs}
\usepackage{chronology}
\usepackage{signchart}
%%%%%%%%%%%%%%%%%%%%%%%%%%%
%% 場合分け
%%%%%%%%%%%%%%%%%%%%%%%%%%%
\usepackage{cases}
%%%%%%%%%%%%%%%%%%%%%%%%%%
\usepackage{pdfpages}
%%%%%%%%%%%%%%%%%%%%%%%%%%%
%% 音声リンク表示
\newcommand{\myaudio}[1]{\href{#1}{\faVolumeUp}}
%%%%%%%%%%%%%%%%%%%%%%%%%%
%% \myAnch{<名前>}{<色>}{<テキスト>}
%% 指定のテキストを指定の色の四角枠で囲み, 指定の名前をもつTikZの
%% ノードとして出力する. 図には remember picture 属性を付けている
%% ので外部から参照可能である.
\newcommand*{\myAnch}[3]{%
  \tikz[remember picture,baseline=(#1.base)]
    \node[draw,rectangle,line width=1pt,#2] (#1) {\normalcolor #3};
}
%%%%%%%%%%%%%%%%%%%%%%%%%%
%% \myEmph コマンドの定義
%%%%%%%%%%%%%%%%%%%%%%%%%%
%\newcommand{\myEmph}[3]{%
%    \textbf<#1>{\color<#1>{#2}{#3}}%
%}
\usepackage{xparse} % xparseパッケージの読み込み
\NewDocumentCommand{\myEmph}{O{} m m}{%
    \def\argOne{#1}%
    \ifx\argOne\empty
        \textbf{\color{#2}{#3}}% オプション引数が省略された場合
    \else
        \textbf<#1>{\color<#1>{#2}{#3}}% オプション引数が指定された場合
    \fi
}
%%%%%%%%%%%%%%%%%%%%%%%%%%%
%%%%%%%%%%%%%%%%%%%%%%%%%%%
%% 文末の上昇イントネーション記号 \myRisingPitch
%% 通常のイントネーション \myDownwardPitch
%% https://note.com/dan_oyama/n/n8be58e8797b2
%%%%%%%%%%%%%%%%%%%%%%%%%%%
\newcommand{\myRisingPitch}{
\begin{tikzpicture}[scale=0.3,baseline=0.3]
\draw[->,>=stealth] (0,0) to[bend right=45] (1,1);
\end{tikzpicture}
}
\newcommand{\myDownwardPitch}{
\begin{tikzpicture}[scale=0.3,baseline=0.3]
\draw[->,>=stealth] (0,1) to[bend left=45] (1,0);
\end{tikzpicture}
}
%%%%%%%%%%%%%%%%%%%%%%%%%%%%
%\AtBeginSection[%
%]{%
%  \begin{frame}[plain]\frametitle{授業の流れ}
%     \tableofcontents[currentsection]
%   \end{frame}%
%}

\usepackage{pxrubrica}
%%%%%%%%%%%%%%%%%%%%%%%%%%%
\title{English is fun.}
\subtitle{The quick brown fox jumps over the lazy dog.}
\author{}
\institute[]{}
\date[]

%%%%%%%%%%%%%%%%%%%%%%%%%%%%
%% TEXT
%%%%%%%%%%%%%%%%%%%%%%%%%%%%
\begin{document}

\begin{frame}[plain]
  \titlepage
\end{frame}

\section*{授業の流れ}
\begin{frame}[plain]
  \frametitle{授業の流れ}
  \tableofcontents
\end{frame}

\section{品詞}
\subsection{品詞とは}
%%%%%%%%%%%%%%%%%%%%%%%%%%%%%%%%%%%%%%%%%%%%%
\begin{frame}[plain]{品詞}
\large 
\begin{description}
 \item<2->[名詞] 人やモノの名前を表す語\hfill\visible<3->{child, cat, desk, Tokyo \ldots}
  \item<4->[代名詞] 名詞の代わりに使う語\hfill\visible<5->{I, we, you he, she, it, they, this, that \ldots}
 \item<6->[動詞] 動作や状態を表す語\hfill\visible<7->{run, eat, study, have, be \ldots{}}
 \item<8->[形容詞] 名詞をくわしく説明する語\hfill\visible<9->{big, new, happy, good \ldots}
 \item<10->[副詞] 動詞や形容詞などをくわしく説明する語\hfill\visible<11->{quickly, here, then\ldots}
 \item<12->[前置詞] 名詞や代名詞と組み合わせて$+\alpha$の意味をくわえる\\
\hfill\visible<13->{in, on, under, with, from, to \ldots}
\end{description}

\vfill

\mbox{}\hfill\visible<10->{ほかにもありますが、まずはこれをしっかり理解しましょう}
\end{frame}
%%%%%%%%%%%%%%%%%%%%%%%%%%%%%%%%%%%%%%%%%%%%%%
\begin{frame}[plain]{Exercises}
それぞれの英文の単語の品詞がなにかかんがえましょう%
\mbox{}\hfill{\scriptsize \myaudio{./audio/020_part_of_speech_01.mp3}}

\begin{columns}[t]
\begin{column}{.7\textwidth}
  \begin{enumerate}
  \item Small cats jump quickly.\\
	\visible<2->{{\small \Circled{ 形 }\hspace{12pt}\Circled{ 名 } \Circled{ 動 }\hspace{12pt}\Circled{ 副 }}}
  \item Tall trees grow slowly.\\
	\visible<3->{{\small \Circled{ 形 } \Circled{ 名 }\hspace{8pt}\Circled{ 動 }\hspace{10pt}\Circled{ 副 }}}
  \item Big dogs bark loudly.\\
	\visible<4->{{\small \Circled{ 形 } \Circled{ 名 } \Circled{ 動 }\hspace{10pt}\Circled{ 副 }}}
  \item George had very old coins.\\
	\visible<5->{{\small \Circled{ 名 }\hspace{18pt}\Circled{ 動 } \Circled{ 副 } \Circled{ 形 }} \Circled{ 名 }}
  \item Jennifer was very happy then.\\
	\visible<6->{{\small \Circled{ 名 }\hspace{22pt}\Circled{ 動 } \Circled{ 副 }\hspace{8pt}\Circled{ 形 }\hspace{10pt}\Circled{ 副 }}}
 \end{enumerate}
\end{column}
\begin{column}{.25\textwidth}
 \begin{tcolorbox}
  \begin{tabular}{cl}
    \Circled{ 名 }&名詞\\
    \Circled{ 動 }&動詞\\
    \Circled{ 形 }&形容詞\\
    \Circled{ 副 }&副詞
  \end{tabular}
 \end{tcolorbox}
\end{column}
\end{columns}
\end{frame}
%%%%%%%%%%%%%%%%%%%%%%%%%%%%%%%%
\begin{frame}[plain]{代名詞}
\Large

代名詞: 名詞の代わりに使う語

 \begin{enumerate} 
  \item \begin{enumerate}
	 \item<1-> Dave is my classmate.\hspace{5pt} Dave likes basketball.\\
	      \mbox{}
	 \item<2-> \myAnch{X}{white}{Dave}is my classmate.\hspace{10pt}\myAnch{x}{Maroon}{He} likes basketball.\\
	      \mbox{}
	\end{enumerate}  \item \begin{enumerate}
	 \item<4-> I ate an apple.\hspace{5pt} The apple was delicious.%
	      \hfill{\scriptsize delicious: おいしい}\\
	      \mbox{}
	 \item<5-> I ate \myAnch{Y}{white}{an apple.}\hspace{10pt} \myAnch{y}{Maroon}{It} was delicious.
	\end{enumerate}
 \end{enumerate}

\begin{tikzpicture}[remember picture, overlay]
% Calculate intermediate points
  \coordinate (X1) at ($(X) + (0,15pt)$); % 10pt above X
  \coordinate (x1) at ($(x) + (0,15pt)$); % 10pt above x
  % Draw the arrow with right angles
  \visible<3->{\draw[<-,Maroon] (X) -- (X1) -- (x1) -- (x);}
%%%%%%%%%%%
% Calculate intermediate points
  \coordinate (Y1) at ($(Y) + (0,15pt)$); % 10pt above Y
  \coordinate (y1) at ($(y) + (0,15pt)$); % 10pt above y
  % Draw the arrow with right angles
  \visible<6->{\draw[<-,Maroon] (Y) -- (Y1) -- (y1) -- (y);}
\end{tikzpicture}
\mbox{}\hfill{\scriptsize \myaudio{./audio/020_part_of_speech_03.mp3}}
\end{frame}
%%%%%%%%%%%%%%%%%%%%%%%%
%%%%%%%%%%%%%%%%%%%%%%%%%%%%%%%
\begin{frame}[plain]{前置詞}
\hfill\myaudio{../1st_grader/audio/004_verb_object_4.mp3}
 \begin{enumerate}
  \item<1-> He plays tennis.
  \item<2-> He plays tennis \Circled[fill color = gray, outer color=gray, inner color=white]{\,\,with his friends\,\,}.%
\hfill{\scriptsize with: ~といっしょに}
  \item<3-> He plays tennis \Circled[fill color = gray, outer color=gray, inner color=white]{\,\,in the park\,\,}.%
\hfill{\scriptsize in: ~の中で}
  \item<4-> He plays tennis \Circled[fill color = gray, outer color=gray, inner color=white]{\,\,after school\,\,}.%
\hfill{\scriptsize after: ~の後に}
  \item<5-> He plays tennis \Circled[fill color = gray, outer color=gray, inner color=white]{\,\,with his friends\,\,} \Circled[fill color = gray, outer color=gray, inner color=white]{\,\,in the park\,\,} \Circled[fill color = gray, outer color=gray, inner color=white]{\,\,after school\,\,}.

 \end{enumerate}

\visible<6->{%
\begin{exampleblock}{Topics for Today}\small
with, in, after などを\kenten{前置詞}といいます%
\hfill{}「(代)名詞の\kenten{前}に\kenten{置}く\kenten{詞}」
\begin{itemize}\setbeamertemplate{items}[square]\small
 \item<7-> 裏返していうと「前置詞の\kenten{後}にはかならず(代)\kenten{名詞}がくる」ということです
 \item<8-> \Circled[fill color = gray, outer color=gray, inner color=white]{\,\,$\text{前置詞}+\text{(代)名詞}$\,\,}\,は、文に$+\alpha$の意味をつけくわえます\\
 \hfill\visible<9->{{\scriptsize 前置詞はいろいろあります。ちょっとずつ覚えましょう}}
\end{itemize}
     \end{exampleblock}
}
\end{frame}
%%%%%%%%%%%%%%%%%%%
%%%%%%%%%%%%%%%%%%%%%%%%
\begin{frame}[plain]{その他}
 \begin{description}
  \item[冠詞]  名詞の前に置く\hfill{}a / an, theだけ
  \item[助動詞] 動詞の意味を補う語\hfill{}can, will, must \ldots 
 \end{description}
\end{frame}
%%%%%%%%%%%%%%%%%%%%%%
\begin{frame}[plain]{冠詞}
\Large

冠詞: a / an , the

 \begin{enumerate}
  \item $\text{a / an} + \text{単数形の名詞}$\,(1つの~)
	\begin{enumerate}
	 \item a book\hfill\textipa{/\textschwa /}
	 \item an apple\hfill\textipa{/\textschwa n/}
	\end{enumerate}
  \item $\text{the} + \text{名詞}$\,(その~)\hspace{10pt}{\small (単数形・複数形どちらにもつく)}
	\begin{enumerate}
	 \item the book / the books\hfill\textipa{/\dh\textschwa/}
	 \item the apple / the apples\hfill\textipa{/\dh i/}
	\end{enumerate}
 \end{enumerate}
\mbox{}\hfill{\scriptsize \myaudio{./audio/020_part_of_speech_02.mp3}}
\end{frame}
%%%%%%%%%%%%%%%%%%%%%%%%%%%%%
\begin{frame}[plain]{助動詞}
 \Large

助動詞: 動詞の意味を補う

\begin{enumerate}
 \item \begin{enumerate}
	\item I play the guitar.
	\item I \myEmph[2-]{Maroon}{will} play the guitar tomorrow.
       \end{enumerate}
 \item \begin{enumerate}
	\item You play the guitar.
	\item You \myEmph[3-]{Maroon}{must} play the guitar.
       \end{enumerate}
 \item \begin{enumerate}
	\item He plays the guitar.
	\item He \myEmph[4-]{Maroon}{can} play the guitar.
       \end{enumerate}
\end{enumerate}
\mbox{}\hfill{\scriptsize \myaudio{./audio/020_part_of_speech_04.mp3}}
\end{frame}
%%%%%%%%%%%%%%%%%%%%%%%%
\end{document}

