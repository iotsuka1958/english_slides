\documentclass[aspectratio=169,xcolor={dvipsnames,table}]{beamer}
\usepackage[no-math,deluxe,haranoaji]{luatexja-preset}
\renewcommand{\kanjifamilydefault}{\gtdefault}
\renewcommand{\emph}[1]{{\upshape\bfseries #1}}
\usetheme{metropolis}
\metroset{block=fill}
\setbeamertemplate{navigation symbols}{}
\setbeamertemplate{blocks}[rounded][shadow=false]
\usecolortheme[rgb={0.7,0.2,0.2}]{structure}
%%%%%%%%%%%%%%%%%%%%%%%%%%
%% Change alert block colors
%%% 1- Block title (background and text)
\setbeamercolor{block title alerted}{fg=mDarkTeal, bg=mLightBrown!45!yellow!45}
\setbeamercolor{block title example}{fg=magenta!10!black, bg=mLightGreen!70}
%%% 2- Block body (background)
\setbeamercolor{block body alerted}{bg=mLightBrown!25}
\setbeamercolor{block body example}{bg=mLightGreen!15}
%%%%%%%%%%%%%%%%%%%%%%%%%%%
%%%%%%%%%%%%%%%%%%%%%%%%%%%
%% さまざまなアイコン
%%%%%%%%%%%%%%%%%%%%%%%%%%%
%\usepackage{fontawesome}
\usepackage{fontawesome5}
\usepackage{figchild}
\usepackage{twemojis}
\usepackage{utfsym}
\usepackage{bclogo}
\usepackage{marvosym}
\usepackage{fontmfizz}
\usepackage{pifont}
\usepackage{phaistos}
\usepackage{worldflags}
\usepackage{jigsaw}
\usepackage{tikzlings}
\usepackage{tikzducks}
\usepackage{scsnowman}
\usepackage{epsdice}
\usepackage{halloweenmath}
\usepackage{svrsymbols}
\usepackage{countriesofeurope}
\usepackage{tipa}
%%%%%%%%%%%%%%%%%%%%%%%%%%%
\usepackage{tikz}
\usetikzlibrary{calc,patterns,decorations.pathmorphing,backgrounds}
\usepackage{tcolorbox}
\usepackage{tikzpeople}
\usepackage{circledsteps}
\usepackage{xcolor}
\usepackage{amsmath}
\usepackage{booktabs}
\usepackage{chronology}
\usepackage{signchart}
%%%%%%%%%%%%%%%%%%%%%%%%%%%
%% 場合分け
%%%%%%%%%%%%%%%%%%%%%%%%%%%
\usepackage{cases}
%%%%%%%%%%%%%%%%%%%%%%%%%%
\usepackage{pdfpages}
%%%%%%%%%%%%%%%%%%%%%%%%%%%
%% 音声リンク表示
\newcommand{\myaudio}[1]{\href{#1}{\faVolumeUp}}
%%%%%%%%%%%%%%%%%%%%%%%%%%
%% \myAnch{<名前>}{<色>}{<テキスト>}
%% 指定のテキストを指定の色の四角枠で囲み, 指定の名前をもつTikZの
%% ノードとして出力する. 図には remember picture 属性を付けている
%% ので外部から参照可能である.
\newcommand*{\myAnch}[3]{%
  \tikz[remember picture,baseline=(#1.base)]
    \node[draw,rectangle,line width=1pt,#2] (#1) {\normalcolor #3};
}
%%%%%%%%%%%%%%%%%%%%%%%%%%
%% \myEmph コマンドの定義
%%%%%%%%%%%%%%%%%%%%%%%%%%
%\newcommand{\myEmph}[3]{%
%    \textbf<#1>{\color<#1>{#2}{#3}}%
%}
\usepackage{xparse} % xparseパッケージの読み込み
\NewDocumentCommand{\myEmph}{O{} m m}{%
    \def\argOne{#1}%
    \ifx\argOne\empty
        \textbf{\color{#2}{#3}}% オプション引数が省略された場合
    \else
        \textbf<#1>{\color<#1>{#2}{#3}}% オプション引数が指定された場合
    \fi
}
%%%%%%%%%%%%%%%%%%%%%%%%%%%
%%%%%%%%%%%%%%%%%%%%%%%%%%%
%% 文末の上昇イントネーション記号 \myRisingPitch
%% 通常のイントネーション \myDownwardPitch
%% https://note.com/dan_oyama/n/n8be58e8797b2
%%%%%%%%%%%%%%%%%%%%%%%%%%%
\newcommand{\myRisingPitch}{
\begin{tikzpicture}[scale=0.3,baseline=0.3]
\draw[->,>=stealth] (0,0) to[bend right=45] (1,1);
\end{tikzpicture}
}
\newcommand{\myDownwardPitch}{
\begin{tikzpicture}[scale=0.3,baseline=0.3]
\draw[->,>=stealth] (0,1) to[bend left=45] (1,0);
\end{tikzpicture}
}
%%%%%%%%%%%%%%%%%%%%%%%%%%%%
%\AtBeginSection[%
%]{%
%  \begin{frame}[plain]\frametitle{授業の流れ}
%     \tableofcontents[currentsection]
%   \end{frame}%
%}

\usepackage{pxrubrica}
%%%%%%%%%%%%%%%%%%%%%%%%%%%
\title{English is fun.}
\subtitle{She was happy because she passed the exam.}
\author{}
\institute[]{}
\date[]

%%%%%%%%%%%%%%%%%%%%%%%%%%%%
%% TEXT
%%%%%%%%%%%%%%%%%%%%%%%%%%%%
\begin{document}

\begin{frame}[plain]
  \titlepage
\end{frame}

\section*{授業の流れ}
\begin{frame}[plain]
  \frametitle{授業の流れ}
  \tableofcontents
\end{frame}

%%%%%%%%%%%%%%%%%%%%%%%%%%%%%%%%%%%%%%%%%%%%%
\section{理由}
\subsection{because S $+$ V}
%%%%%%%%%%%%%%%%%%%%%%%%%%%%%%%%%%%%%%%%%%%%%
\begin{frame}[plain]{理由をあらわす表現}
\large
 \begin{enumerate}
  \item She  \myAnch{x_1}{white}{\myEmph[2-]{NavyBlue}{smiled}} \alt<2->{\myAnch{y_1}{Maroon}{\textbf{because} his joke was funny}}{\myAnch{Y_1}{white}{\textbf{because} his joke was funny}}.\hfill{}\visible<6->{{\footnotesize {\bfseries S $+$ V}\,\,\,\,\Circled[fill color = yellow!50]{ because s $+$ v }}}
\item  \alt<3->{\myAnch{y_2}{Maroon}{\textbf{Because} his joke was funny}}{\myAnch{Y_2}{white}{\textbf{Becuase} his joke was funny}},\,\, she  \myAnch{x_2}{white}{\myEmph[3-]{NavyBlue}{smiled}}.\hfill{}\visible<6->{{\footnotesize \Circled[fill color = yellow!50]{ Because s $+$ v }\,,\,\,\,{\bfseries S $+$ V}}}
\end{enumerate}

\begin{tikzpicture}[remember picture, overlay]
 \visible<2->{\draw[thick, Maroon, ->] (Y_1.north west) to[out=170, in=10] (x_1.north);} 
 \visible<3->{\draw[thick, Maroon, ->] (Y_2.north east) to[out=5, in=175] (x_2.north);} 
\end{tikzpicture}

\hfill{\tiny 0122}\,{\scriptsize \myaudio{./audio/023_because_01.mp3}}

\begin{block}<4->{Topics for Today}\small
\begin{itemize}\setbeamertemplate{items}[square]\small
 \item   \visible<4->{\Circled[fill color = white]{ because s $+$ v }\,は「~なので」という意味で理由を表す}
 \item   \visible<5->{\Circled[fill color = white]{ because s $+$ v }\,は全体で「副詞」の役割}
 \item   \visible<6->{\Circled[fill color = white]{ because s $+$ v }\,は、前にも後ろにも位置します
	 \begin{enumerate}\setbeamertemplate{items}[circle]
	  \item {\bfseries S $+$ V}\,\,\Circled[fill color = white]{ because s $+$ v }\,\,\textbf{.}
	  \item \Circled[fill color = white]{ Because s $+$ v }\,,\,\,{\bfseries S $+$ V\,\,.}
	 \end{enumerate}
	 }
\end{itemize}
     \end{block}
\end{frame}
%%%%%%%%%%%%%%%%%%%%%%%%%%%%%%
\begin{frame}{Exercises}

{\small 日本語の意味になるよう、(~~~~~)内の語句を並べ替えましょう}%
\hfill{\tiny 0217}\,{\scriptsize \myaudio{./audio/023_because_02.mp3}}
%先頭の語句は大文字ではじめてください
 \begin{enumerate}
  \item {\small 雨が降っていたので、わたしは家にいた。}\hfill{\scriptsize stay \textipa{/st\'eI/} とどまる}\\
	I stayed home ( was / because / raining / it ) .\\
	\visible<2->{\textcolor{lightgray}{I stayed home} because it was raining.}\hfill \visible<3->{{\bfseries S $+$ V}\,\,\Circled[fill color = yellow!50]{ because s $+$ v }}
  \item {\small 試験に受かったので、彼女は幸せだった。}\hfill{\scriptsize exam \textipa{/Igz\'\ae m/} 試験}\\
	She was happy ( the exam / because / passed / she ) .\\
	\visible<4->{\textcolor{lightgray}{She was happy} because she passed the exam.} \hfill \visible<5->{{\bfseries S $+$ V}\,\,\Circled[fill color = yellow!50]{ because s $+$ v }}
 \item {\small 眠かったので、彼女は早く寝た。}\hfill{\scriptsize sleepy \textipa{/sl\'\i:pi/} 眠い go to bed: 寝る}\\
Because ( sleepy / went / was / she / she ) to bed early.\\
	\visible<6->{\textcolor{lightgray}{Because} she was sleepy, she went \textcolor{lightgray}{to bed early.}}\hfill \visible<7->{\Circled[fill color = yellow!50]{ Because s $+$ v }, {\bfseries S $+$ V}}
  \item {\small 昼食を食べなかったので、彼ははらぺこだった。}\hfill{\scriptsize hungry \textipa{/h\'2Ngri/} おなかがすいた}\\
Because ( he / he / was / eat / hungry / lunch / didn't ) .\\
	\visible<8->{\textcolor{lightgray}{Because} he didn't eat lunch, he was hungry.}\hfill \visible<9->{\Circled[fill color = yellow!50]{ Because s $+$ v }, {\bfseries S $+$ V}}
 \end{enumerate}
\end{frame}
%%%%%%%%%%%%%%%%%%%%%%%%%%%%%
\begin{frame}[plain]{\fbox{because s $+$ v}\,は理由を表しますが}
 
\begin{enumerate}
 \item<1-> He told her a joke. \textbf{Because} his joke was funny, she smiled.
 \item<2-> 彼は彼女にジョークをいった。なぜなら彼のジョークはおもしろかった。
彼女は笑いました。\visible<3->{$\longleftarrow$ この解釈のどこがいけないかわかりますか}
 \item<3-> 彼は彼女にジョークをいった。彼のジョークはおもしろかったので、彼女は笑いました。
 \end{enumerate}

\textdbend
何に対する理由かはっきりさせましょう!\\
\hspace*{16pt}Because his joke was funnyはshe smiledの理由です


\vspace*{40pt}

\hfill{\scriptsize tell 人 ~(\,\ldots\,\,に~をいう)}\\
\hfill{\scriptsize told \textipa{/t\'oUld/ tellの過去形}}
\end{frame}
%%%%%%%%%%%%%%%%%%%%%%%%%%%%%
\section{接続詞のまとめ}
%%%%%%%%%%%%%%%%%%%%%%%%%%%%%
\begin{frame}[plain]{まとめ}
\begin{block}{接続詞}
 \begin{itemize}\setbeamertemplate{items}[square]\small
  \item<1-> \textbf{when}(時), \textbf{if}(条件), \textbf{because}(理由)は2組のS $+$ Vをつなぐ\kenten{接続詞}
 \item<2->   \Circled[fill color = white]{ 接 s $+$ v }\,は、全体で「副詞」の役割
 \item<3->   \Circled[fill color = white]{ 接 s $+$ v }\,は、前にも後ろにも位置します
	 \begin{itemize}\setbeamertemplate{items}[circle]
	  \setlength{\itemsep}{4pt}
	  \item {\bfseries S $+$ V}\,\,\Circled[fill color = white]{ 接 s $+$ v }\,\textbf{.}
	  \item \Circled[fill color = white]{ 接 s $+$ v }\,,\,\,{\bfseries S $+$ V}\,.\hspace{20pt}\visible<4->{$\longleftarrow$\,コンマに注意!}
	 \end{itemize}
 \end{itemize}
\end{block}

\begin{enumerate}\small
 \item<5-> He was hungry \textbf{when} he got up this morning.\\
$\Longleftrightarrow$ \textbf{When} he got up this morning, he was hungry.
 \item<6-> You will succeed \textbf{if} you work hard.\\
$\Longleftrightarrow$ \textbf{If} you work hard, you will succeed.
 \item<7-> She smiled \textbf{because} his joke was funny.\\
$\Longleftrightarrow$ \textbf{Because} his joke was funny, she smiled.
\end{enumerate}
\hfill{\tiny 0255}\,{\scriptsize \myaudio{./audio/023_because_03.mp3}}
\end{frame}
\end{document}
