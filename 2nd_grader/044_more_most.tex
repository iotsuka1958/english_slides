\documentclass[aspectratio=169,xcolor={dvipsnames,table}]{beamer}
\usepackage[no-math,deluxe,haranoaji]{luatexja-preset}
\renewcommand{\kanjifamilydefault}{\gtdefault}
\renewcommand{\emph}[1]{{\upshape\bfseries #1}}
\usetheme{metropolis}
\metroset{block=fill}
\setbeamertemplate{navigation symbols}{}
\setbeamertemplate{blocks}[rounded][shadow=false]
\usecolortheme[rgb={0.7,0.2,0.2}]{structure}
%%%%%%%%%%%%%%%%%%%%%%%%%%
%% Change alert block colors
%%% 1- Block title (background and text)
\setbeamercolor{block title alerted}{fg=mDarkTeal, bg=mLightBrown!45!yellow!45}
\setbeamercolor{block title example}{fg=magenta!10!black, bg=mLightGreen!70}
%%% 2- Block body (background)
\setbeamercolor{block body alerted}{bg=mLightBrown!25}
\setbeamercolor{block body example}{bg=mLightGreen!15}
%%%%%%%%%%%%%%%%%%%%%%%%%%%
\usepackage[absolute,overlay]{textpos}
%\usepackage[grid=true,gridcolor=Maroon,subgridcolor=gray,gridunit=pt,texcoord]{eso-pic} %場所決めのためのgrid表示
%%%%%%%%%%%%%%%%%%%%%%%%%%%
%% さまざまなアイコン
%%%%%%%%%%%%%%%%%%%%%%%%%%%
%\usepackage{fontawesome}
\usepackage{fontawesome5}
\usepackage{figchild}
\usepackage{twemojis}
\usepackage{utfsym}
\usepackage{bclogo}
\usepackage{marvosym}
\usepackage{fontmfizz}
\usepackage{pifont}
\usepackage{phaistos}
\usepackage{worldflags}
\usepackage{jigsaw}
\usepackage{tikzlings}
\usepackage{tikzducks}
\usepackage{scsnowman}
\usepackage{epsdice}
\usepackage{halloweenmath}
\usepackage{svrsymbols}
\usepackage{countriesofeurope}
\usepackage{tipa}
%%%%%%%%%%%%%%%%%%%%%%%%%%%
\usepackage{tikz}
\usetikzlibrary{calc,patterns,decorations.pathmorphing,backgrounds}
\usepackage{tcolorbox}
\usepackage{tikzpeople}
\usepackage{circledsteps}
\usepackage{xcolor}
\usepackage{amsmath}
\usepackage{booktabs}
\usepackage{chronology}
\usepackage{signchart}
%%%%%%%%%%%%%%%%%%%%%%%%%%%
%% 場合分け
%%%%%%%%%%%%%%%%%%%%%%%%%%%
\usepackage{cases}
%%%%%%%%%%%%%%%%%%%%%%%%%%
\usepackage{pdfpages}
%%%%%%%%%%%%%%%%%%%%%%%%%%%
%% 音声リンク表示
\newcommand{\myaudio}[1]{\href{#1}{\faVolumeUp}}
%%%%%%%%%%%%%%%%%%%%%%%%%%
%% \myAnch{<名前>}{<色>}{<テキスト>}
%% 指定のテキストを指定の色の四角枠で囲み, 指定の名前をもつTikZの
%% ノードとして出力する. 図には remember picture 属性を付けている
%% ので外部から参照可能である.
\newcommand*{\myAnch}[3]{%
  \tikz[remember picture,baseline=(#1.base)]
    \node[draw,rectangle,line width=1pt,#2] (#1) {\normalcolor #3};
}
%%%%%%%%%%%%%%%%%%%%%%%%%%
%% \myEmph コマンドの定義
%%%%%%%%%%%%%%%%%%%%%%%%%%
%\newcommand{\myEmph}[3]{%
%    \textbf<#1>{\color<#1>{#2}{#3}}%
%}
\usepackage{xparse} % xparseパッケージの読み込み
\NewDocumentCommand{\myEmph}{O{} m m}{%
    \def\argOne{#1}%
    \ifx\argOne\empty
        \textbf{\color{#2}{#3}}% オプション引数が省略された場合
    \else
        \textbf<#1>{\color<#1>{#2}{#3}}% オプション引数が指定された場合
    \fi
}
%%%%%%%%%%%%%%%%%%%%%%%%%%%
%%%%%%%%%%%%%%%%%%%%%%%%%%%
%% 文末の上昇イントネーション記号 \myRisingPitch
%% 通常のイントネーション \myDownwardPitch
%% https://note.com/dan_oyama/n/n8be58e8797b2
%%%%%%%%%%%%%%%%%%%%%%%%%%%
\newcommand{\myRisingPitch}{
\begin{tikzpicture}[scale=0.3,baseline=0.3]
\draw[->,>=stealth] (0,0) to[bend right=45] (1,1);
\end{tikzpicture}
}
\newcommand{\myDownwardPitch}{
\begin{tikzpicture}[scale=0.3,baseline=0.3]
\draw[->,>=stealth] (0,1) to[bend left=45] (1,0);
\end{tikzpicture}
}
%%%%%%%%%%%%%%%%%%%%%%%%%%%%
%\AtBeginSection[%
%]{%
%  \begin{frame}[plain]\frametitle{授業の流れ}
%     \tableofcontents[currentsection]
%   \end{frame}%
%}

%%%%%%%%%%%%%%%%%%%%%%%%%%%
\title{English is fun.}
\subtitle{Health is more important than money.}
\author{}
\institute[]{}
\date[]

%%%%%%%%%%%%%%%%%%%%%%%%%%%%
%% TEXT
%%%%%%%%%%%%%%%%%%%%%%%%%%%%
\begin{document}

\begin{frame}[plain]
  \titlepage
\end{frame}

\section*{授業の流れ}
\begin{frame}[plain]
  \frametitle{授業の流れ}
  \tableofcontents
\end{frame}

\section{more, most}
\subsection{more ---, most ---}
%%%%%%%%%%%%%%%%%%%%%%%%%%%%%%%%%%%%%%%%%%%%%
\begin{frame}[plain]{more ---, most ---}
\begin{enumerate}
 \item<1-> \begin{enumerate}
	\item<1-> Jennifer is tall\myEmph[1-]{Maroon}{er} than Emily.%
\hfill{\tiny 0209}{\scriptsize \myaudio{./audio/044_more_most_01.mp3}}
	\item<1-> The river is wid\myEmph[1-]{Maroon}{er} than a mile near here.
	\item<1-> Tokyo is busi\myEmph[1-]{Maroon}{er} than my city.
	\item<1-> Whales are bigg\myEmph[1-]{Maroon}{er} than dolphines.
       \end{enumerate}
 \item<2-> \begin{enumerate}
	\item<2-> Health is important.%
\hfill{\tiny 0154}\,{\scriptsize \myaudio{./audio/044_more_most_02.mp3}}
	\item<3-> \only<4->{*}\visible<3->{Health is important\myEmph[3-]{Maroon}{er} than money.}\hfill\visible<4->{{\scriptsize importanterとはいわない}}
	\item<5-> Health is \myEmph[5-]{Maroon}{more} important than money.\hfill{\scriptsize more \textipa{/m\'O\textrhookschwa /}}
	\item<6-> Health is \myEmph[6-]{Maroon}{the most} important.\hfill\visible<6->{{\scriptsize importantestとはいわない}}\\
\hfill{\scriptsize most \textipa{/m\'oUst/}}
       \end{enumerate}
\end{enumerate}

\begin{block}<7->{Topics for Today}
\begin{description}[第2のパターン]\small
 \item<7->[\textcolor{black}{原則:}] ---er, ---est%
	    \hfill{}\makebox[200pt][l]{tall, wide, busy, big \ldots}
 \item<8->[\textcolor{black}{第2のパターン:}]  more ---,\,\,\,\,\,\,most ---
	    \hfill{}\makebox[200pt][l]{important, beautiful, difficult, interesting \ldots}\\
	    \mbox{}\hspace{120pt}\visible<9->{\Circled[fill color = white]{\small \,\,比較的長い単語は第2のパターン\,\,}}
 \end{description}
     \end{block}

\end{frame}
%%%%%%%%%%%%%%%%%%%%%%%%%
\begin{frame}[plain]{Exercises}

{\small 選択肢の中の語を用いて日本語の意味になるようにしましょう。カッコ内は1語とは限りません}%
\hfill{\tiny 0247}\,{\scriptsize \myaudio{./audio/044_more_most_03.mp3}}
\begin{columns}[t]
 \begin{column}{.75\textwidth}
   \begin{enumerate}
  \item 理科は数学よりもわたしにはむずかしい。\\
	Science is (~~\alt<2->{\textcolor{Maroon}{\bfseries more difficult}}{\phantom{more difficult}}~~) than math for me.
  \item 私の庭は春にもっとも美しい。\\
	My garden is (~~\alt<3->{\textcolor{Maroon}{\bfseries the most beautiful}}{\phantom{the most beautiful}}~~) in spring.
  \item 多くの国でコーヒーはお茶よりも人気がある。\\
	Coffee is (~~\alt<4->{\textcolor{Maroon}{\bfseries more popular}}{\phantom{more popular}}~~) than tea in many countries.
  \item ダイアモンドは真珠よりも高価だ。\\
	A diamond is (~~\alt<5->{\textcolor{Maroon}{\bfseries more expensive}}{\phantom{more expensive}}~~) than a pearl.
  \item モナリザはもっとも有名な絵画だ。\\
	The Mona Lisa is (~~\alt<6->{\textcolor{Maroon}{\bfseries the most famous}}{\phantom{the most famous}}~~) painting.
 \end{enumerate}
 \end{column}
\begin{column}{.2\textwidth}
\begin{tcolorbox}
 expensive\\
beautiful\\
famous\\
difficult\\
popular
\end{tcolorbox}
\end{column}
\end{columns}
\end{frame}
%%%%%%%%%%%%%%%%%%%%%%%%%%%%%%%%%%%%%%%%%%%%%
\section{長い単語の比較級・最上級}
%%%%%%%%%%%%%%%%%%%%%%%%%%%%%%%%%%%%%%%%%%%%%
\begin{frame}[plain,t,label=table1]{長い単語の比較級・最上級のつくりかた}
more \textipa{/m\'O\textrhookschwa /}\hspace{40pt}most \textipa{/m\'oUst/}

\centering
  \begin{tblr}{colspec={lll},
% 表の最上と最下に太さ 0.08em の横罫線
hline{1,Z} = { 0.08em },
hline{2} = { 0.05em },
row{odd}={gray9},
row{1} = { halign = c, font = { \sffamily\bfseries }, bg = gray6, fg = white }
}
原級&比較級&最上級\\
difficult&\visible<2->{more difficult}&\visible<3->{most difficult}\\
beautiful&\visible<4->{more beautiful}&\visible<5->{most beautiful}\\
popular&\visible<6->{more popular}&\visible<7->{most popular}\\
expensive&\visible<8->{more expensive}&\visible<9->{most expensive}\\
famous&\visible<10->{more famous}&\visible<11->{most famous}\\
important&\visible<12->{more important}&\visible<13->{most important}\\
useful&\visible<14->{more useful}&\visible<15->{most useful}\\
interesting&\visible<16->{more interesting}&\visible<17->{most interesting}\\
   \end{tblr}

\hfill{\tiny 0643}\,{\scriptsize \myaudio{./audio/044_more_most_04.mp3}}

\begin{textblock*}{0.3\linewidth}(340pt,135pt)
\visible<18->{\begin{tikzpicture}
\pig[
signpost=\scalebox{0.5}{
\parbox{2.2cm}{\color{black}
\centering 長い単語は\\{\Large more ---}\\{\Large most ---}}},
signcolour= brown!70!gray,
signback=white!80!brown
]
\end{tikzpicture}}
\end{textblock*}

\end{frame}
%%%%%%%%%%%%%%%%%%%%%%%%%%%%%%%%%%%%%%
\begin{frame}[plain,t]{---lyで終わるとき}
\large

 \begin{enumerate}
  \item \begin{enumerate}
	 \item Bob spoke \myEmph[1-]{Maroon}{more} slow\myEmph[1-]{Maroon}{ly} than Jane.%
\hfill{\tiny 0144}\,{\scriptsize \myaudio{./audio/044_more_most_05.mp3}}

	 \item Jane spoke \myEmph[1-]{Maroon}{more} quick\myEmph[1-]{Maroon}{ly} than Bob.
	 \item Please speak \myEmph[1-]{Maroon}{more} quiet\myEmph[1-]{Maroon}{ly}.
	\end{enumerate}
  \item \begin{enumerate}
	 \item<2-> Jennifer got up \myEmph[2-]{Maroon}{earlier} than Emily.%
	      \hfill\visible<3->{{\scriptsize more earlyとはいわない}}
	 \item<4-> Jennifer got up the \myEmph[4-]{Maroon}{earliest} of all.%
	      \hfill\visible<5->{{\scriptsize most earlyとはいわない}}
	\end{enumerate}
 \end{enumerate}
\vspace{-8pt}
\hfill{\tiny 0117}\,{\scriptsize \myaudio{./audio/044_more_most_06.mp3}}

\vspace{60pt}

\begin{block}<7->{Topics for Today}
 \begin{itemize}\setbeamertemplate{items}[square]\small
  \item<7-> --{\bfseries ly}で終わる副詞の比較級は{\bfseries more} ---, 最上級は{\bfseries most} ---
  \item<8-> ただし{\bfseries early}は{\bfseries earlier}, {\bfseries earliest}
 \end{itemize}
     \end{block}

\begin{textblock*}{0.2\linewidth}(320pt,120pt)
\visible<6->{\begin{tikzpicture}
\duck[signpost=\scalebox{0.3}{
\parbox{2.5cm}{\color{black}\centering
{\Huge\bfseries more\\[3pt] most}}},
signcolour=brown!70!gray,
signback=white!80!brown,
graduate=gray!20!black,
tassel=red!70!black,
speech={\tiny --{\bfseries ly}のとき}
]
\end{tikzpicture}}
\end{textblock*}
\end{frame}
%%%%%%%%%%%%%%%%%%%%%
\begin{frame}[plain]{Exercises}

{\small 日本語の意味になるよう(~~~~~~)内の語を並べ替えましょう。なお $[ +1 ]$とある場合は不足している1語を補いましょう。$[ -1 ]$とある場合は余計な1語が含まれています}%
\hfill{\tiny 0139}\,{\scriptsize \myaudio{./audio/044_more_most_07.mp3}}

 \begin{enumerate}
  \item 彼は静かに話しました。\\
	He ( quiet / quietly / spoke ) . $[-1]$\\
	\visible<2->{He \textcolor{Maroon}{\bfseries spoke quietly}.\,\,\,(✕\,quiet)}
  \item もっと静かに話してください。\\
	Please ( quietly / speak ) . $[+1]$\\
	\visible<3->{Please \textcolor{Maroon}{\bfseries speak more quietly}.}
  \item 彼女は3人のなかでもっとも早く起きた。\\
	She got up ( the / of / most / earliest ) the three. $[-1]$\\
	\visible<4->{She got up \textcolor{Maroon}{\bfseries the earliest} of the three.\,\,\,(✕\,most)}
 \end{enumerate}
\end{frame}
%%%%%%%%%%%%%%%%%%%%%%%%%%%
\section{more ---, most --- のまとめ}

\begin{frame}[plain]{まとめ}

\begin{block}<1->{比較級・最上級をつくる2つのパターン}
\begin{description}[第2のパターン]\small
 \item[\textcolor{black}{原則:}] ---er, ---est%
	    \hfill{}\makebox[200pt][l]{tall, wide, busy, big \ldots}
 \item[\textcolor{black}{第2のパターン:}]  more ---,\,\,\,\,\,\,most ---
	    \hfill{}\makebox[200pt][l]{important, beautiful, difficult, interesting \ldots}\\
	    \mbox{}\hspace{120pt}{\Circled[fill color = white]{\small \,\,比較的長い単語は第2のパターン\,\,}}
 \end{description}
     \end{block}
 
\begin{block}<2->{---lyのとき}
 \begin{itemize}\setbeamertemplate{items}[square]\small
  \item --lyで終わる副詞の比較級は{\bfseries more} ---, 最上級は{\bfseries most} ---
  \item ただし{\bfseries early}は{\bfseries earlier}, {\bfseries earliest}
 \end{itemize}
     \end{block}

\begin{block}<3->{Pronunciation}
 \begin{itemize}\setbeamertemplate{items}[square]\small
  \item {\bfseries more} \textipa{/m\'O\textrhookschwa /}
  \item {\bfseries most} \textipa{/m\'oUst/}
\end{itemize}
 
\end{block}
\end{frame}
%%%%%%%%%%%%%%%%%%%%%%%%%%
\againframe<18>{table1}
\end{document}
