\documentclass[aspectratio=169,xcolor={dvipsnames,table}]{beamer}
\usepackage[no-math,deluxe,haranoaji]{luatexja-preset}
\renewcommand{\kanjifamilydefault}{\gtdefault}
\renewcommand{\emph}[1]{{\upshape\bfseries #1}}
\usetheme{metropolis}
\metroset{block=fill}
\setbeamertemplate{navigation symbols}{}
\setbeamertemplate{blocks}[rounded][shadow=false]
\usecolortheme[rgb={0.7,0.2,0.2}]{structure}
%%%%%%%%%%%%%%%%%%%%%%%%%%
%% Change alert block colors
%%% 1- Block title (background and text)
\setbeamercolor{block title alerted}{fg=mDarkTeal, bg=mLightBrown!45!yellow!45}
\setbeamercolor{block title example}{fg=magenta!10!black, bg=mLightGreen!70}
%%% 2- Block body (background)
\setbeamercolor{block body alerted}{bg=mLightBrown!25}
\setbeamercolor{block body example}{bg=mLightGreen!15}
%%%%%%%%%%%%%%%%%%%%%%%%%%%
\usepackage[absolute,overlay]{textpos}
%\usepackage[grid=true,gridcolor=Maroon,subgridcolor=gray,gridunit=pt,texcoord]{eso-pic} %場所決めのためのgrid表示
%%%%%%%%%%%%%%%%%%%%%%%%%%%
%% さまざまなアイコン
%%%%%%%%%%%%%%%%%%%%%%%%%%%
%\usepackage{fontawesome}
\usepackage{fontawesome5}
\usepackage{figchild}
\usepackage{twemojis}
\usepackage{utfsym}
\usepackage{bclogo}
\usepackage{marvosym}
\usepackage{fontmfizz}
\usepackage{pifont}
\usepackage{phaistos}
\usepackage{worldflags}
\usepackage{jigsaw}
\usepackage{tikzlings}
\usepackage{tikzducks}
\usepackage{scsnowman}
\usepackage{epsdice}
\usepackage{halloweenmath}
\usepackage{svrsymbols}
\usepackage{countriesofeurope}
\usepackage{tipa}
%%%%%%%%%%%%%%%%%%%%%%%%%%%
\usepackage{tikz}
\usetikzlibrary{calc,patterns,decorations.pathmorphing,backgrounds}
\usepackage{tcolorbox}
\usepackage{tikzpeople}
\usepackage{circledsteps}
\usepackage{xcolor}
\usepackage{amsmath}
\usepackage{booktabs}
\usepackage{chronology}
\usepackage{signchart}
%%%%%%%%%%%%%%%%%%%%%%%%%%%
%% 場合分け
%%%%%%%%%%%%%%%%%%%%%%%%%%%
\usepackage{cases}
%%%%%%%%%%%%%%%%%%%%%%%%%%
\usepackage{pdfpages}
%%%%%%%%%%%%%%%%%%%%%%%%%%%
%% 音声リンク表示
\newcommand{\myaudio}[1]{\href{#1}{\faVolumeUp}}
%%%%%%%%%%%%%%%%%%%%%%%%%%
%% \myAnch{<名前>}{<色>}{<テキスト>}
%% 指定のテキストを指定の色の四角枠で囲み, 指定の名前をもつTikZの
%% ノードとして出力する. 図には remember picture 属性を付けている
%% ので外部から参照可能である.
\newcommand*{\myAnch}[3]{%
  \tikz[remember picture,baseline=(#1.base)]
    \node[draw,rectangle,line width=1pt,#2] (#1) {\normalcolor #3};
}
%%%%%%%%%%%%%%%%%%%%%%%%%%
%% \myEmph コマンドの定義
%%%%%%%%%%%%%%%%%%%%%%%%%%
%\newcommand{\myEmph}[3]{%
%    \textbf<#1>{\color<#1>{#2}{#3}}%
%}
\usepackage{xparse} % xparseパッケージの読み込み
\NewDocumentCommand{\myEmph}{O{} m m}{%
    \def\argOne{#1}%
    \ifx\argOne\empty
        \textbf{\color{#2}{#3}}% オプション引数が省略された場合
    \else
        \textbf<#1>{\color<#1>{#2}{#3}}% オプション引数が指定された場合
    \fi
}
%%%%%%%%%%%%%%%%%%%%%%%%%%%
%%%%%%%%%%%%%%%%%%%%%%%%%%%
%% 文末の上昇イントネーション記号 \myRisingPitch
%% 通常のイントネーション \myDownwardPitch
%% https://note.com/dan_oyama/n/n8be58e8797b2
%%%%%%%%%%%%%%%%%%%%%%%%%%%
\newcommand{\myRisingPitch}{
\begin{tikzpicture}[scale=0.3,baseline=0.3]
\draw[->,>=stealth] (0,0) to[bend right=45] (1,1);
\end{tikzpicture}
}
\newcommand{\myDownwardPitch}{
\begin{tikzpicture}[scale=0.3,baseline=0.3]
\draw[->,>=stealth] (0,1) to[bend left=45] (1,0);
\end{tikzpicture}
}
%%%%%%%%%%%%%%%%%%%%%%%%%%%%
%\AtBeginSection[%
%]{%
%  \begin{frame}[plain]\frametitle{授業の流れ}
%     \tableofcontents[currentsection]
%   \end{frame}%
%}

\usepackage{pxrubrica}
%%%%%%%%%%%%%%%%%%%%%%%%%%%
\title{English is fun.}
\subtitle{I want something to drink.}
\author{}
\institute[]{}
\date[]

%%%%%%%%%%%%%%%%%%%%%%%%%%%%
%% TEXT
%%%%%%%%%%%%%%%%%%%%%%%%%%%%
\begin{document}

\begin{frame}[plain]
  \titlepage
\end{frame}

\section*{授業の流れ}
\begin{frame}[plain]
  \frametitle{授業の流れ}
  \tableofcontents
\end{frame}

\section{形容詞的用法}
\subsection{名詞 $\longleftarrow$ \fbox{to不定詞}}
%%%%%%%%%%%%%%%%%%%%%%%%%%%%%%%%%%%%%%%%%%%%%
\begin{frame}[plain]{名詞 $\longleftarrow$ \fbox{to不定詞}}
 \begin{enumerate}
  \item \begin{enumerate}
	 \item<1-> He doesn't have time to read books.%
\hfill{\tiny 0137}\,{\scriptsize \myaudio{./audio/033_infinitive_adj_01.mp3}}\\
	       \mbox{}
	 \item<2-> He doesn't have \myAnch{a1}{white}{\myEmph[3-]{NavyBlue}{time}} \myAnch{b1}{Maroon}{\textbf{to read} books}.\visible<3->{本を読む時間}
 	\end{enumerate}
  \item \begin{enumerate}
	 \item<4-> I want something to drink.\\
	       \mbox{}
	 \item<5-> I want \myAnch{a2}{white}{\myEmph[6-]{NavyBlue}{something}} \myAnch{b2}{Maroon}{\textbf{to drink}}.\visible<6->{なにか飲むもの}
	\end{enumerate}
  \item \begin{enumerate}
	 \item<7-> I need someone to help me.\\
	       \mbox{}
	 \item<8-> I need \myAnch{a3}{white}{\myEmph[9-]{NavyBlue}{someone}} \myAnch{b3}{Maroon}{\textbf{to help} me}.\visible<9->{私を助けてくれるだれか}
	\end{enumerate}
 \end{enumerate}

\begin{tikzpicture}[remember picture, overlay]
% Calculate intermediate points
  \coordinate (A1) at ($(a1) + (0,15pt)$); % 10pt above a1
  \coordinate (B1) at ($(b1) + (0,15pt)$); % 10pt above b1
  % Draw the arrow with right angles
  \visible<3->{\draw[<-,Maroon] (a1) -- (A1) -- (B1) -- (b1);}
%%%%%%%%%%%
% Calculate intermediate points
  \coordinate (A2) at ($(a2) + (0,15pt)$); % 10pt above a2
  \coordinate (B2) at ($(b2) + (0,15pt)$); % 10pt above b2
  % Draw the arrow with right angles
  \visible<6->{\draw[<-,Maroon] (a2) -- (A2) -- (B2) -- (b2);}
%%%%%%%%%%%
% Calculate intermediate points
  \coordinate (A3) at ($(a3) + (0,15pt)$); % 10pt above a2
  \coordinate (B3) at ($(b3) + (0,15pt)$); % 10pt above b2
  % Draw the arrow with right angles
  \visible<9->{\draw[<-,Maroon] (a3) -- (A3) -- (B3) -- (b3);}
\end{tikzpicture}


\begin{block}<10->{Topics for Today}
\begin{itemize}\setbeamertemplate{items}[square]\small
\setlength{\itemsep}{4pt}
 \item<11-> to不定詞が名詞を後ろから修飾することがある\hfill{}homework \textbf{to do}\\
\hfill\hfill{}名詞 $\longleftarrow$ \Circled[fill color=white]{\,to不定詞\,}\hfill~
 \item<12-> to不定詞の\kenten{形容詞的用法}といいます
 \end{itemize}
     \end{block}
\end{frame}
%%%%%%%%%%%%%%%%%%%
\begin{frame}[plain]{Exercises}
意味がとおるように(~~~~~)内の語句を並べ替えましょう%
\hfill{\tiny 0245}\,{\scriptsize \myaudio{./audio/033_infinitive_adj_02.mp3}}
 \begin{enumerate}
  \item<1->{He has ( help / many friends / to ) him.}\\
       \visible<2->{\textcolor{gray}{He has} many friends to help \textcolor{gray}{him}.}%
\hfill\visible<3->{{\scriptsize many friends $\longleftarrow$\,\fbox{to help him}}}
  \item<1-> She has a lot of ( to / do / things ) today.%
       \hfill{\scriptsize a lot of ~: たくさんの~}\\
	\visible<4->{\textcolor{gray}{She has a lot of} things to do \textcolor{gray}{today}.}%
\hfill\visible<5->{{\scriptsize a lot of things $\longleftarrow$\,\fbox{to do}}}
  \item<1-> I am hungry. I ( eat / want something / to )\\
	\visible<6->{\textcolor{gray}{I am hungry. I} want something to eat}.%
\hfill\visible<7->{{\scriptsize something $\longleftarrow$\,\fbox{to eat}}}
  \item<1-> There are ( places / see / many / to ) in New York.\\
	\visible<8->{\textcolor{gray}{There are} many places to see \textcolor{gray}{in New York}.}%
\hfill\visible<9->{{\scriptsize many places $\longleftarrow$\,\fbox{to see}}}
  \item<1-> They have ( a plan / to / to / go ) Boston.\\
	\visible<10->{\textcolor{gray}{They have} a plan to go to \textcolor{gray}{Boston}.}
\hfill\visible<11->{{\scriptsize a plan $\longleftarrow$\,\fbox{to go to Boston}}}
 \end{enumerate} 
\end{frame}
%%%%%%%%%%%%%%%%%%%%%%
\section{聞いてみよう、読んでみよう{\tiny 0032}\,{\scriptsize \myaudio{./audio/033_infinitive_adj_03.mp3}}}
%%%%%%%%%%%%%%%%%%%%%%
%%%%%%%%%%%%%%%%%%%%%%%%%%%%%%%
\begin{frame}[plain]
 
\includegraphics[width=1.01\textwidth]{./images/nanobanana-output/033_infinitive_adj_reading.png}

\vspace{-15pt}

\hfill{\tiny 0032}\,{\scriptsize \myaudio{./audio/033_infinitive_adj_03.mp3}}

\end{frame}
%%%%%%%%%%%%%%%%%%%
%%%%%%%%%%%%%%%%%%%%%%
\begin{frame}[plain,t]{Exercises}

\begin{tcolorbox}[colframe=ForestGreen,
  colback=ForestGreen!10!white,
  colbacktitle=ForestGreen!20!white,
  coltitle=black, %fonttitle=\bfseries,
before upper={\setlength{\parindent}{1.25em}},
 title=次の英文を読みましょう\mbox{}\hfill{\tiny 0032}\,{\scriptsize \myaudio{./audio/033_infinitive_adj_03.mp3}}
]
Tom is a junior high school student.
He is very busy today.
First, he has a lot of homework to do.
He studies math and English for two hours.

After that, he wants something to eat, so he makes a sandwich.
In the afternoon, he has a plan to go to the park.
He has many friends to meet there.
He does not have time to watch TV today, but he is happy.
He has many fun things to do.
\end{tcolorbox}
\end{frame}
%%%%%%%%%%%%%%%%%%%%%%%%%%%%%%%%%
\begin{frame}[plain]{大意}
 
\begin{tcolorbox}
 トムは中学生です。今日はとても忙しいです。
まず、しなければいけないたくさんの宿題があります。
彼は2時間、数学と英語の勉強をします。

 そのあと、なにか食べるものがほしくなったので、サンドイッチを作ります。
午後には、公園へ行く予定があります。
そこで会う友だちがたくさんいます。
今日はテレビを見る時間はありませんが、彼はうれしい気持ちです。
やりたい楽しいことがたくさんあるからです。
\end{tcolorbox}

\end{frame}
%%%%%%%%%%%%%%%%%%%%%%%%%%%%%%%%%%%%%%%%%%%%%
\section{to不定詞の形容詞的用法(まとめ)}
%%%%%%%%%%%%%%%%%%%%%%%%%%%%%%%%%%%%%%%%%%%%%
\begin{frame}[plain]{to不定詞の形容詞的用法のまとめ}
 
\begin{block}{Topics for Today}
\begin{itemize}\setbeamertemplate{items}[square]\small
 \item to不定詞が名詞を後ろから修飾することがある\\
\hfill\hfill{}名詞 $\longleftarrow$ \Circled[fill color=white]{\,to不定詞\,}\hfill~
 \item to不定詞の\kenten{形容詞的用法}といいます


\hfill{}{\scriptsize He doesn't have time \textbf{to read} books.}\\
\hfill{}{\scriptsize I want something \textbf{to drink}.}\\
\hfill{}{\scriptsize I need someone \textbf{to help} me.}

 \end{itemize}
     \end{block}
\hfill{\tiny 0137}\,{\scriptsize \myaudio{./audio/033_infinitive_adj_01.mp3}}\

\end{frame}
%%%%%%%%%%%%%%%%%%%%%%%%%%%%
%%%%%%%%%%%%%%%%%%%%%%%%%%%%%%%%%%%%%%%%%%%%%
\begin{frame}[plain]{to不定詞とは}

\large
\visible<2->{動詞とは本来、主語と組み合わせて使うもの}
\hfill\visible<3->{\myEmph[3-]{NavyBlue}{I} \myEmph[3-]{Maroon}{play} the piano and \myEmph[3-]{NavyBlue}{she} \myEmph[3-]{Maroon}{sings}.}

\visible<4->{だが、特別に

\begin{tcolorbox}[colback=yellow!10!white]
\myAnch{v}{white}{動詞}%
\hspace{25pt}%
\begin{tabular}{ll}
 \visible<4->{\myAnch{n}{white}{名詞のはたらき}}&\visible<5->{\fcolorbox{black}{white}{\textbf{To get} up early} is important.}\\
 \visible<4->{\myAnch{adj}{white}{形容詞のはたらき}}&\visible<6->{I want something \fcolorbox{black}{white}{\textbf{to drink}}.}\\
 \visible<4->{\myAnch{adv}{white}{副詞のはたらき}}&\visible<7->{She studied hard \fcolorbox{black}{white}{\textbf{to pass} the exam}}.
\end{tabular}
\end{tcolorbox}

をすることがある
}

\begin{tikzpicture}[remember picture, overlay]
 \visible<4->{\draw[line width=2pt,opacity=.75, gray, ->] (v.east) to[out=0, in=180] (n.west);} 
 \visible<4->{\draw[line width=2pt,opacity=.75, gray, ->] (v.east) to[out=0, in=180] (adj.west);} 
 \visible<4->{\draw[line width=2pt,opacity=.75, gray, ->] (v.east) to[out=0, in=180] (adv.west);} 
\end{tikzpicture}

%\vspace{-5pt}

%\hfill%
%\visible<9->{to不定詞:\Circled[fill color = yellow!10!white]{\,$\text{to} + \text{動詞の原形}$\,}}

\visible<8->{
\Circled[fill color = yellow!10!white]{\,$\text{to} + \text{動詞の原形}$\,}
のことをto\kenten{不定詞}といいます\hfill{\bfseries to-infinitive}\,\textdbend
}

\hfill{\tiny 0140}\,{\scriptsize \myaudio{./audio/031_infinitive_intro_01.mp3}}

\begin{textblock*}{0.4\linewidth}(250pt,170pt)
\visible<9->{\begin{tikzpicture}
\duck[signpost=\scalebox{0.3}{
\parbox{2.5cm}{\color{black}
{\Large to $+$原形}\\{\Large $=$ to不定詞}}},
signcolour=brown!70!gray,
signback=white!80!brown,
graduate=gray!20!black,
tassel=red!70!black,
speech={\tiny メモメモ}
]
\end{tikzpicture}}
\end{textblock*}
\end{frame}
%%%%%%%%%%%%%%%%%%%%%%%%
%%%%%%%%%%%%%%%%%%%%%%%%%%%%%%%%%%%%%%
\begin{frame}[plain]
 
\hfill{\tiny audio\_overview 1623}\,{\scriptsize \myaudio{./audio/overview/033_infinitive_adj_audio_overview.m4a}}
\end{frame}
%%%%%%%%%%%%%%%%%%%%%%%%%%%%%%%%%%%%%%%
\end{document}

