\documentclass[aspectratio=169,xcolor={dvipsnames,table}]{beamer}
\usepackage[no-math,deluxe,haranoaji]{luatexja-preset}
\renewcommand{\kanjifamilydefault}{\gtdefault}
\renewcommand{\emph}[1]{{\upshape\bfseries #1}}
\usetheme{metropolis}
\metroset{block=fill}
\setbeamertemplate{navigation symbols}{}
\setbeamertemplate{blocks}[rounded][shadow=false]
\usecolortheme[rgb={0.7,0.2,0.2}]{structure}
%%%%%%%%%%%%%%%%%%%%%%%%%%
%% Change alert block colors
%%% 1- Block title (background and text)
\setbeamercolor{block title alerted}{fg=mDarkTeal, bg=mLightBrown!45!yellow!45}
\setbeamercolor{block title example}{fg=magenta!10!black, bg=mLightGreen!70}
%%% 2- Block body (background)
\setbeamercolor{block body alerted}{bg=mLightBrown!25}
\setbeamercolor{block body example}{bg=mLightGreen!15}
%%%%%%%%%%%%%%%%%%%%%%%%%%%
%%%%%%%%%%%%%%%%%%%%%%%%%%%
%% さまざまなアイコン
%%%%%%%%%%%%%%%%%%%%%%%%%%%
%\usepackage{fontawesome}
\usepackage{fontawesome5}
\usepackage{figchild}
\usepackage{twemojis}
\usepackage{utfsym}
\usepackage{bclogo}
\usepackage{marvosym}
\usepackage{fontmfizz}
\usepackage{pifont}
\usepackage{phaistos}
\usepackage{worldflags}
\usepackage{jigsaw}
\usepackage{tikzlings}
\usepackage{tikzducks}
\usepackage{scsnowman}
\usepackage{epsdice}
\usepackage{halloweenmath}
\usepackage{svrsymbols}
\usepackage{countriesofeurope}
\usepackage{tipa}
\usepackage{manfnt}
%%%%%%%%%%%%%%%%%%%%%%%%%%%
\usepackage{tikz}
\usetikzlibrary{calc,patterns,decorations.pathmorphing,backgrounds}
\usepackage{tcolorbox}
\usepackage{tikzpeople}
\usepackage{circledsteps}
\usepackage{xcolor}
\usepackage{amsmath}
\usepackage{booktabs}
\usepackage{chronology}
\usepackage{signchart}
%%%%%%%%%%%%%%%%%%%%%%%%%%%
%% 場合分け
%%%%%%%%%%%%%%%%%%%%%%%%%%%
\usepackage{cases}
%%%%%%%%%%%%%%%%%%%%%%%%%%
\usepackage{pdfpages}
%%%%%%%%%%%%%%%%%%%%%%%%%%%
%% 音声リンク表示
\newcommand{\myaudio}[1]{\href{#1}{\faVolumeUp}}
%%%%%%%%%%%%%%%%%%%%%%%%%%
%% \myAnch{<名前>}{<色>}{<テキスト>}
%% 指定のテキストを指定の色の四角枠で囲み, 指定の名前をもつTikZの
%% ノードとして出力する. 図には remember picture 属性を付けている
%% ので外部から参照可能である.
\newcommand*{\myAnch}[3]{%
  \tikz[remember picture,baseline=(#1.base)]
    \node[draw,rectangle,line width=1pt,#2] (#1) {\normalcolor #3};
}
%%%%%%%%%%%%%%%%%%%%%%%%%%
%% \myEmph コマンドの定義
%%%%%%%%%%%%%%%%%%%%%%%%%%
%\newcommand{\myEmph}[3]{%
%    \textbf<#1>{\color<#1>{#2}{#3}}%
%}
\usepackage{xparse} % xparseパッケージの読み込み
\NewDocumentCommand{\myEmph}{O{} m m}{%
    \def\argOne{#1}%
    \ifx\argOne\empty
        \textbf{\color{#2}{#3}}% オプション引数が省略された場合
    \else
        \textbf<#1>{\color<#1>{#2}{#3}}% オプション引数が指定された場合
    \fi
}
%%%%%%%%%%%%%%%%%%%%%%%%%%%
%%%%%%%%%%%%%%%%%%%%%%%%%%%
%% 文末の上昇イントネーション記号 \myRisingPitch
%% 通常のイントネーション \myDownwardPitch
%% https://note.com/dan_oyama/n/n8be58e8797b2
%%%%%%%%%%%%%%%%%%%%%%%%%%%
\newcommand{\myRisingPitch}{
\begin{tikzpicture}[scale=0.3,baseline=0.3]
\draw[->,>=stealth] (0,0) to[bend right=45] (1,1);
\end{tikzpicture}
}
\newcommand{\myDownwardPitch}{
\begin{tikzpicture}[scale=0.3,baseline=0.3]
\draw[->,>=stealth] (0,1) to[bend left=45] (1,0);
\end{tikzpicture}
}
%%%%%%%%%%%%%%%%%%%%%%%%%%%%
%\AtBeginSection[%
%]{%
%  \begin{frame}[plain]\frametitle{授業の流れ}
%     \tableofcontents[currentsection]
%   \end{frame}%
%}

\usepackage{pxrubrica}
%%%%%%%%%%%%%%%%%%%%%%%%%%%
\title{English is fun.}
\subtitle{I want something to drink.}
\author{}
\institute[]{}
\date[]

%%%%%%%%%%%%%%%%%%%%%%%%%%%%
%% TEXT
%%%%%%%%%%%%%%%%%%%%%%%%%%%%
\begin{document}

\begin{frame}[plain]
  \titlepage
\end{frame}

\section*{授業の流れ}
\begin{frame}[plain]
  \frametitle{授業の流れ}
  \tableofcontents
\end{frame}

\section{形容詞的用法}
\subsection{形容詞的用法}
%%%%%%%%%%%%%%%%%%%%%%%%%%%%%%%%%%%%%%%%%%%%%
\begin{frame}[plain]{形容詞的用法}
 \begin{enumerate}
  \item \begin{enumerate}
	 \item<1-> He doesn't have time to read books.%
\hfill{\tiny 0137}\,{\scriptsize \myaudio{./audio/033_infinitive_adj_01.mp3}}\\
	       \mbox{}
	 \item<2-> He doesn't have \myAnch{a1}{white}{\myEmph[3-]{NavyBlue}{time}} \myAnch{b1}{Maroon}{to read books}.\visible<3->{本を読む時間}
 	\end{enumerate}
  \item \begin{enumerate}
	 \item<4-> I want something to drink.\\
	       \mbox{}
	 \item<5-> I want \myAnch{a2}{white}{\myEmph[6-]{NavyBlue}{something}} \myAnch{b2}{Maroon}{to drink}.\visible<6->{なにか飲むもの}
	\end{enumerate}
  \item \begin{enumerate}
	 \item<7-> I need someone to help me.\\
	       \mbox{}
	 \item<8-> I need \myAnch{a3}{white}{\myEmph[9-]{NavyBlue}{someone}} \myAnch{b3}{Maroon}{to help me}.\visible<9->{私を助けてくれるだれか}
	\end{enumerate}
 \end{enumerate}

\begin{tikzpicture}[remember picture, overlay]
% Calculate intermediate points
  \coordinate (A1) at ($(a1) + (0,15pt)$); % 10pt above a1
  \coordinate (B1) at ($(b1) + (0,15pt)$); % 10pt above b1
  % Draw the arrow with right angles
  \visible<3->{\draw[<-,Maroon] (a1) -- (A1) -- (B1) -- (b1);}
%%%%%%%%%%%
% Calculate intermediate points
  \coordinate (A2) at ($(a2) + (0,15pt)$); % 10pt above a2
  \coordinate (B2) at ($(b2) + (0,15pt)$); % 10pt above b2
  % Draw the arrow with right angles
  \visible<6->{\draw[<-,Maroon] (a2) -- (A2) -- (B2) -- (b2);}
%%%%%%%%%%%
% Calculate intermediate points
  \coordinate (A3) at ($(a3) + (0,15pt)$); % 10pt above a2
  \coordinate (B3) at ($(b3) + (0,15pt)$); % 10pt above b2
  % Draw the arrow with right angles
  \visible<9->{\draw[<-,Maroon] (a3) -- (A3) -- (B3) -- (b3);}
\end{tikzpicture}

\visible<10->{%
\begin{block}{Topic for Today}
\begin{itemize}\setbeamertemplate{items}[square]\small
 \item to不定詞つまり\Circled[fill color= white]{\,$\text{to} + \text{動詞の原形}$\,} が、名詞を後ろから修飾することがある\\
\hfill{}to不定詞の\kenten{形容詞的用法}といいます
 \end{itemize}
     \end{block}
}
\end{frame}
%%%%%%%%%%%%%%%%%%%
\begin{frame}[plain]{Exercises}
意味がとおるように(~~~~~)内の語句を並べ替えましょう%
\hfill{\tiny 0245}\,{\scriptsize \myaudio{./audio/033_infinitive_adj_02.mp3}}
 \begin{enumerate}
  \item<1->{He has ( help / many friends / to ) him.}\\
       \visible<2->{\textcolor{lightgray}{He has} many friends to help \textcolor{lightgray}{him}.}
  \item<1-> She has a lot of ( to / do / things ) today.\\
	\visible<3->{\textcolor{lightgray}{She has a lot of} things to do \textcolor{lightgray}{today}.}
  \item<1-> I am hungry. I ( eat / want something / to )\\
	\visible<4->{\textcolor{lightgray}{I am hungry. I} want something to eat}.
  \item<1-> There are ( places / see / many / to ) in New York.\\
	\visible<5->{\textcolor{lightgray}{There are} many places to see \textcolor{lightgray}{in New York}.}
  \item<1-> They have ( a plan / to / to / go ) Boston.\\
	\visible<6->{\textcolor{lightgray}{They have} a plan to go to \textcolor{lightgray}{Boston}.}
 \end{enumerate} 
\end{frame}
%%%%%%%%%%%%%%%%%%%%%%%%%%%%%%%%%%%%%%%%%%%%%
\section{まとめ}
%%%%%%%%%%%%%%%%%%%%%%%%%%%%%%%%%%%%%%%%%%%%%
\begin{frame}[plain]{to不定詞の形容詞的用法のまとめ}
 
\begin{block}{Topic for Today}
\begin{itemize}\setbeamertemplate{items}[square]\small
 \item to不定詞つまり\Circled[fill color= white]{\,$\text{to} + \text{動詞の原形}$\,} が、名詞を後ろから修飾することがある\\
\hfill{}to不定詞の\kenten{形容詞的用法}といいます


\hfill{}{\scriptsize He doesn't have time to read books.}\\
\hfill{}{\scriptsize I want something to drink.}\\
\hfill{}{\scriptsize I need someone to help me.}

 \end{itemize}
     \end{block}
\hfill{\tiny 0137}\,{\scriptsize \myaudio{./audio/033_infinitive_adj_01.mp3}}\

\end{frame}


















%%%%%%%%%%%%%%%%%%%%%%%%%%%%%%%
\begin{frame}[plain]{to不定詞とは}
\large
\visible<2->{動詞とは本来、主語と組み合わせて使うもの
\mbox{}\hfill{}\myEmph[2-]{NavyBlue}{I} \myEmph[2-]{Maroon}{play} the piano and \myEmph[2-]{NavyBlue}{she} \myEmph[2-]{Maroon}{sings}.}

\visible<3->{だが、特別に

\vfill

\myAnch{v}{white}{動詞}}%
\hspace{30pt}%
\begin{tabular}{ll}
 \visible<3->{\myAnch{n}{white}{名詞のはたらき}: }&\visible<5->{\fbox{To get up early} is important.}\\
 \visible<3->{\myAnch{adj}{white}{形容詞のはたらき}: }&\visible<6->{I want something \fbox{to drink}.}\\
 \visible<3->{\myAnch{adv}{white}{副詞のはたらき}: }&\visible<7->{She studied hard \fbox{to pass the exam}}.
\end{tabular}

\visible<3->%
{\begin{tikzpicture}[remember picture, overlay]
 \visible<3->{\draw[line width=2pt,opacity=.75, Maroon, ->] (v.east) to[out=0, in=180] (n.west);} 
 \visible<3->{\draw[line width=2pt,opacity=.75, Maroon, ->] (v.east) to[out=0, in=180] (adj.west);} 
 \visible<3->{\draw[line width=2pt,opacity=.75, Maroon, ->] (v.east) to[out=0, in=180] (adv.west);} 
\end{tikzpicture}}

\vspace{-5pt}

\mbox{}\hfill%
\visible<8->{\begin{minipage}{.4\textwidth}
\visible<9->{to不定詞}:\Circled[fill color = white]{\,$\text{to} + \text{動詞の原形}$\,}
\end{minipage}}

\hfill{\tiny 0140}\,{\scriptsize \myaudio{./audio/031_infinitive_intro_01.mp3}}
\end{frame}
%%%%%%%%%%%%%%%%%%%%%%%%
\end{document}

