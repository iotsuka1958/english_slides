\documentclass[aspectratio=169,xcolor={dvipsnames,table}]{beamer}
\usepackage[no-math,deluxe,haranoaji]{luatexja-preset}
\renewcommand{\kanjifamilydefault}{\gtdefault}
\renewcommand{\emph}[1]{{\upshape\bfseries #1}}
\usetheme{metropolis}
\metroset{block=fill}
\setbeamertemplate{navigation symbols}{}
\setbeamertemplate{blocks}[rounded][shadow=false]
\usecolortheme[rgb={0.7,0.2,0.2}]{structure}
%%%%%%%%%%%%%%%%%%%%%%%%%%%
\usepackage{media9}
%%%%%%%%%%%%%%%%%%%%%%%%%%%
%% さまざまなアイコン
%%%%%%%%%%%%%%%%%%%%%%%%%%%
\usepackage{fontawesome}
\usepackage{figchild}
\usepackage{twemojis}
\usepackage{utfsym}
\usepackage{bclogo}
\usepackage{marvosym}
\usepackage{fontmfizz}
\usepackage{pifont}
\usepackage{phaistos}
\usepackage{worldflags}
%%%%%%%%%%%%%%%%%%%%%%%%%%%
\usepackage{tikz}
\usetikzlibrary{backgrounds}
\usepackage{tcolorbox}
\usepackage{tikzpeople}
\usepackage{circledsteps}
\usepackage{xcolor}
\usepackage{amsmath}
\usepackage{booktabs}
%%%%%%%%%%%%%%%%%%%%%%%%%%%
%% 場合分け
\usepackage{cases}
%%%%%%%%%%%%%%%%%%%%%%%%%%%
% \myAnch{<名前>}{<色>}{<テキスト>}
% 指定のテキストを指定の色の四角枠で囲み, 指定の名前をもつTikZの
% ノードとして出力する. 図には remeber picture 属性を付けている
% ので外部から参照可能である.
\newcommand*{\myAnch}[3]{%
  \tikz[remember picture,baseline=(#1.base)]
    \node[draw,rectangle,#2] (#1) {\normalcolor #3};
}
%%%%%%%%%%%%%%%%%%%%%%%%%%%%
%% 音声リンク表示
\newcommand{\myaudio}[1]{\href{#1}{\faVolumeUp}}
%%%%%%%%%%%%%%%%%%%%%%%%%%%
% \myEmph コマンドの定義
%\newcommand{\myEmph}[3]{%
%    \textbf<#1>{\color<#1>{#2}{#3}}%
%}
\usepackage{xparse} % xparseパッケージの読み込み
\NewDocumentCommand{\myEmph}{O{} m m}{%
    \def\argOne{#1}%
    \ifx\argOne\empty
        \textbf{\color{#2}{#3}}% オプション引数が省略された場合
    \else
        \textbf<#1>{\color<#1>{#2}{#3}}% オプション引数が指定された場合
    \fi
}
%%%%%%%%%%%%%%%%%%%%%%%%%%%
%% 文末の上昇イントネーション記号 \myRisingPitch
%% 通常のイントネーション \myDownwardPitch
%% https://note.com/dan_oyama/n/n8be58e8797b2
%%%%%%%%%%%%%%%%%%%%%%%%%%%
\newcommand{\myRisingPitch}{
\begin{tikzpicture}[scale=0.3,baseline=0.3]
\draw[->,>=stealth] (0,0) to[bend right=45] (1,1);
\end{tikzpicture}
}
\newcommand{\myDownwardPitch}{
\begin{tikzpicture}[scale=0.3,baseline=0.3]
\draw[->,>=stealth] (0,1) to[bend left=45] (1,0);
\end{tikzpicture}
}
%%%%%%%%%%%%%%%%%%%%%%%%%%%
\title{English is fun.\,\,{}--- Cheese is made from milk. ---}
\author{}
\institute[]{}
\date[]

%%%%%%%%%%%%%%%%%%%%%%%%%%%%
%% TEXT
%%%%%%%%%%%%%%%%%%%%%%%%%%%%
\begin{document}
\begin{frame}[plain]
  \titlepage
\end{frame}

\section*{授業の流れ}
\begin{frame}[plain]
  \frametitle{授業の流れ}
  \tableofcontents
\end{frame}

\section{受け身}

\subsection{受け身とは}

\begin{frame}[plain]{受け身とは}
 \Large

\begin{enumerate}
 \item<1-> \begin{enumerate}
	\item<1-> 先生は彼女をほめました。
	\item<2-> 彼女は先生にほめられました。
       \end{enumerate}
 \item<3-> \begin{enumerate}
	\item<3-> ジョンはジェニファーを愛しています。
	\item<4-> ジェニファーはジョンに愛されています。
       \end{enumerate}
 \item<5-> \begin{enumerate}
	\item<5-> 牛乳からチーズを作ります。
	\item<6-> チーズは牛乳から作られます。
       \end{enumerate}
\end{enumerate}

\begin{exampleblock}<7->{Topic for Today}
\begin{itemize}\setbeamertemplate{items}[square]\small
 \item 受け身とは「〜される」の意味を表します
\end{itemize}
     \end{exampleblock}

\end{frame}
%%%%%%%%%%%%%%%%%%%%%%%%%%%%%%%%%%%
\subsection{be $+$ 過去分詞}
\begin{frame}[plain]{be $+$ 過去分詞}
 \Large
\begin{enumerate}
 \item<1-> They speak English.\hfill{}{\small 彼らは英語を話す。}
 \item<2-> They spoke English.\hspace{1\zw}{{\small 過去形}}\hfill{\small 彼らは英語を話した。}
 \item<3-> English \textcolor{Maroon}{\bfseries is} \textcolor{NavyBlue}{\bfseries spoken} there.\hspace{1\zw}{{\small 過去分詞}}\hfill{\small そこでは英語が話されている。}
\end{enumerate}

\begin{center}
 
\visible<4->{{\small%
\begin{tabular}{lll}
{\small 原形}&{\small 過去形}&{\small 過去分詞}\\\hline
speak&spoke&spoken
\end{tabular}%
}}
\end{center}

\vfill

\hfill\myaudio{./audio/051_passive_01.mp3}
\begin{exampleblock}<5->{Topics for Today}
\begin{itemize}\setbeamertemplate{items}[square]\small
 \item 「過去分詞」は「受け身」の意味を表します
 \item `be動詞$+$ 過去分詞' $\longrightarrow$\,\,\,\,受け身(〜される)
\end{itemize}
     \end{exampleblock}
\end{frame}
%%%%%%%%%%%%%%%%%%%%%%%%%%%%%%%%
\begin{frame}[plain]{Exercises}
 次の各文の意味を考えましょう\hfill\myaudio{./audio/051_passive_02.mp3}


\begin{enumerate}
 \item The picture was painted by her.\hfill{}(painted: paintの過去分詞)
 \item The car is washed every Sunday.\hfill{}(washed: washの過去分詞)
 \item The computer is used by students.\hfill{}(used: useの過去分詞)
 \item The cat is loved by the family.\hfill{}(loved: loveの過去分詞)
 \item The house was built ten years ago.\hfill{}(built: buildの過去分詞)
 \item The toy was made in Canada.\hfill{}(made: makeの過去分詞)
 \item The book is read by many people.\hfill{}(read: readの過去分詞)
\end{enumerate}

\pause

これらの動詞の過去分詞は、過去形と同じです

\vspace{-5pt}

過去形がしっかり身についていればだいじょうぶ
\end{frame}
%%%%%%%%%%%%%%%%%%%%%%%%%%%%%%%%%%%%
\begin{frame}[plain]{Exercises}
 次の各文の意味を考えましょう\hfill\myaudio{./audio/051_passive_03.mp3}


 \begin{enumerate}
\item English is spoken in many countries.\hfill{}(spoken: speakの過去分詞)
\item The photo was taken by my friend.\hfill{}(taken: takeの過去分詞 by: 〜によって)
\item The window was broken by the ball.\hfill{} (broken: breakの過去分詞)
\item The cake was eaten by the family.\hfill{}(eaten: eatの過去分詞)
\item The movie was seen by many people.\hfill{}(seen: seeの過去分詞)
\item The letter was written by my father.\hfill{}(written: writeの過去分詞)
\item She is known as a great singer.\hfill{}(known: knowの過去分詞 as: 〜として)
\end{enumerate}

\pause

これらの動詞は、過去形と過去分詞がちがいます

\end{frame}

\begin{frame}[plain]{過去形 $\neq$ 過去分詞}
 過去形と過去分詞が異なる動詞の表をつくりましょう\hfill\myaudio{./audio/051_passive_04.mp3}


\begin{center}
 
\rowcolors{2}{NavyBlue!50}{yellow!50}
\begin{tabular}{lll}\toprule
{\small 原形}&{\small 過去形}&{\small 過去分詞}\\\midrule
\visible<1->{speak}&\visible<2->{spoke}&\visible<3->{spoken}\\
\visible<1->{take}&\visible<4->{took}&\visible<5->{taken}\\
\visible<1->{break}&\visible<6->{broke}&\visible<7->{broken}\\
\visible<1->{eat}&\visible<8->{ate}&\visible<9->{eaten}\\
\visible<1->{see}&\visible<10->{saw}&\visible<11->{seen}\\
\visible<1->{write}&\visible<12->{wrote}&\visible<13->{written}\\
\visible<1->{know}&\visible<14->{knew}&\visible<15->{known}\\\bottomrule
\end{tabular}%
\end{center}

\end{frame}

\begin{frame}<1-7>[plain]{Exercises}

あたえられた日本文の意味になるよう、下の枠から動詞を選び適当な形にして(~~~~~~~~~~)に補いましょう\hfill\myaudio{./audio/051_passive_05.mp3}


\begin{enumerate}
 \item The song is \alt<1>{(~~\phantom{loved}~~)}{(~~loved~~)} by many people.その歌は多くの人から愛されている。
 \item The vase was \alt<1-2>{(~~\phantom{broken}~~)}{(~~broken~~)} by the cat. 花瓶がネコにこわされた。
 \item The photo was \alt<1-3>{(~~\phantom{taken}~~)}{(~~taken~~)} by my mother.その写真をとったのは私の母です。
 \item Spanish is \alt<1-4>{(~~\phantom{spoken}~~)}{(~~spoken~~)} in Mexico.メキシコではスペイン語が話されている。
 \item Sushi is \alt<1-5>{(~~\phantom{eaten}~~)}{(~~eaten~~)} in the United States.アメリカではすしが食べられている。
 \item The book was \alt<1-6>{(~~\phantom{written}~~)}{(~~written~~)} by Hemingway. \\その本はヘミングウェイによって書かれた。
\end{enumerate}

\begin{tcolorbox}\centering
 eat~~~~~~~~~~love~~~~~~~~~~write~~~~~~~~~~speak~~~~~~~~~~break~~~~~~~~~~take
\end{tcolorbox}
\end{frame}


\subsection{受け身の否定}
\begin{frame}[plain]{受け身の否定}
\Large

\visible<1->{Spanish \textcolor{NavyBlue}{\bfseries is spoken} in the country.}\hfill\visible<2->{{\small country: 国}}

\visible<3->{Spanish \textcolor{NavyBlue}{\bfseries is} \textcolor{Maroon}{\bfseries not} \textcolor{NavyBlue}{\bfseries spoken} in the country.} 

\vfill

\hfill\myaudio{./audio/051_passive_06.mp3}

\begin{exampleblock}{Topic for Today}
\begin{itemize}\setbeamertemplate{items}[square]\small
 \item 「受け身」の否定 $\longrightarrow$\,\,\,\,`be動詞$+$ \textcolor{Maroon}{not} $+$ 過去分詞'
\end{itemize}
     \end{exampleblock}

\end{frame}
%%%%%%%%%%%%%%%%%%%%%%%%%%%%%%%%%
\subsection{受け身の疑問文}
\begin{frame}[plain]{受け身の疑問文のつくり方}
\Large

\visible<1->{English \alt<1-3>{\myAnch{be1}{white}{is}}{\myAnch{BE1}{orange}{is}} spoken there.}\hfill\visible<2->{{\small there: そこでは}}

\vspace{15pt}

\visible<3->{\alt<1-3>{\myAnch{be2}{white}{Is}}{\myAnch{BE2}{orange}{Is}} English spoken there?} 

\visible<4->{%
\begin{tikzpicture}[remember picture,overlay]
 \draw[thick,orange,->] (be1.south) to[out=-90, in=90] (be2.north);
\end{tikzpicture}
}

\hfill\myaudio{./audio/051_passive_07.mp3}

\begin{exampleblock}<5->{Topics for Today}\small
受け身の疑問文のつくり方
\begin{itemize}\setbeamertemplate{items}[square]
 \item<3->  be動詞を先頭にする
 \item<4-> 文末に`?'をつける   イントネーションは\myRisingPitch
\end{itemize}
      \end{exampleblock}

\end{frame}
%%%%%%%%%%%%%%%%%%%%%%%%%%%%%
\begin{frame}[plain]{Exercises}
 
日本文を参考にして(~~~~~~~~)の語を並べかえ、英文を完成させましょう。
ただし先頭に来る単語は大文字で始めてください\hfill\myaudio{./audio/051_passive_08.mp3}




\begin{enumerate}
 \item 家は毎日掃除されますか。
( cleaned / the house / is ) every day?\\
\visible<2->{Is the house cleaned every day?}
 \item その写真はハワイで撮られましたか。
( the picture / taken /  was ) in Hawaii?\\
\visible<3->{Was the picture taken in Hawaii?}
 \item チーズは牛乳から作られますか。
( cheese / made / from / milk / is ) ?\\
\visible<4->{Is cheese made from milk?}
 \item 彼は良い歌手として知られていましたか。\\
( he / known /  was ) as a good singer?\\
\visible<5->{Was he known as a good singer?}
\end{enumerate}

\end{frame}

\section{まとめ}
\begin{frame}[plain]{要点}
 
\begin{exampleblock}{Topics for Today}
\begin{itemize}\setbeamertemplate{items}[square]\small
 \item `be動詞$+$ 過去分詞' $\longrightarrow$\,\,\,\,受け身(〜される)
 \item 「過去分詞」は「受け身」の意味を表します
        \begin{itemize}
	 \item 過去形と過去分詞が同じことが多い\\
\mbox{}\hfill{}(例: paint--painted--painted, love--loved--loved, make--made--made)
	 \item でも、異なることもある\\
\mbox{}\hfill{}(例: eat--ate--eaten, write--wrote--written, see--saw--seen, break--broke--broken)
	\end{itemize}
 \item 「受け身」の否定文と疑問文の作り方は、be動詞を用いた普通の文と同じです
\end{itemize}
     \end{exampleblock}
\end{frame}





\end{document}


\begin{frame}[plain]{受け身とは}
 \Large

He makes cheese from milk.\hfill{}{\small 彼は牛乳からチーズを作る。}

They made cheese from milk.\hfill{}{\small 彼は牛乳からチーズを作った。}

Cheese is made from milk.\hfill{\small チーズは牛乳から作られる。}

\end{frame}

