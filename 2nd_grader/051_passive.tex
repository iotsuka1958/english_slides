\documentclass[aspectratio=169,xcolor={dvipsnames,table}]{beamer}
\usepackage[no-math,deluxe,haranoaji]{luatexja-preset}
\renewcommand{\kanjifamilydefault}{\gtdefault}
\renewcommand{\emph}[1]{{\upshape\bfseries #1}}
\usetheme{metropolis}
\metroset{block=fill}
\setbeamertemplate{navigation symbols}{}
\setbeamertemplate{blocks}[rounded][shadow=false]
\usecolortheme[rgb={0.7,0.2,0.2}]{structure}
%%%%%%%%%%%%%%%%%%%%%%%%%%%
\usepackage{media9}
%%%%%%%%%%%%%%%%%%%%%%%%%%%
%% さまざまなアイコン
%%%%%%%%%%%%%%%%%%%%%%%%%%%
\usepackage{fontawesome}
\usepackage{figchild}
\usepackage{twemojis}
\usepackage{utfsym}
\usepackage{bclogo}
\usepackage{marvosym}
\usepackage{fontmfizz}
\usepackage{pifont}
\usepackage{phaistos}
\usepackage{worldflags}
%%%%%%%%%%%%%%%%%%%%%%%%%%%
\usepackage{tikz}
\usetikzlibrary{backgrounds}
\usetikzlibrary{tikzmark}
\usepackage{tcolorbox}
\usepackage{tikzpeople}
\usepackage{tikzducks}
\usepackage{circledsteps}
\usepackage{xcolor}
\usepackage{amsmath}
\usepackage{booktabs}
\usepackage{pxrubrica}
\usepackage{tipa}
\usepackage{manfnt}
%%%%%%%%%%%%%%%%%%%%%%%%%%%
%% 場合分け
\usepackage{cases}
%%%%%%%%%%%%%%%%%%%%%%%%%%%
% \myAnch{<名前>}{<色>}{<テキスト>}
% 指定のテキストを指定の色の四角枠で囲み, 指定の名前をもつTikZの
% ノードとして出力する. 図には remeber picture 属性を付けている
% ので外部から参照可能である.
\newcommand*{\myAnch}[3]{%
  \tikz[remember picture,baseline=(#1.base)]
    \node[draw,rectangle,#2] (#1) {\normalcolor #3};
}
%%%%%%%%%%%%%%%%%%%%%%%%%%%%
%% 音声リンク表示
\newcommand{\myaudio}[1]{\href{#1}{\faVolumeUp}}
%%%%%%%%%%%%%%%%%%%%%%%%%%%
% \myEmph コマンドの定義
%\newcommand{\myEmph}[3]{%
%    \textbf<#1>{\color<#1>{#2}{#3}}%
%}
\usepackage{xparse} % xparseパッケージの読み込み
\NewDocumentCommand{\myEmph}{O{} m m}{%
    \def\argOne{#1}%
    \ifx\argOne\empty
        \textbf{\color{#2}{#3}}% オプション引数が省略された場合
    \else
        \textbf<#1>{\color<#1>{#2}{#3}}% オプション引数が指定された場合
    \fi
}
%%%%%%%%%%%%%%%%%%%%%%%%%%%
%% 文末の上昇イントネーション記号 \myRisingPitch
%% 通常のイントネーション \myDownwardPitch
%% https://note.com/dan_oyama/n/n8be58e8797b2
%%%%%%%%%%%%%%%%%%%%%%%%%%%
\newcommand{\myRisingPitch}{
\begin{tikzpicture}[scale=0.3,baseline=0.3]
\draw[->,>=stealth] (0,0) to[bend right=45] (1,1);
\end{tikzpicture}
}
\newcommand{\myDownwardPitch}{
\begin{tikzpicture}[scale=0.3,baseline=0.3]
\draw[->,>=stealth] (0,1) to[bend left=45] (1,0);
\end{tikzpicture}
}
%%%%%%%%%%%%%%%%%%%%%%%%%%%
\title{English is fun.}
\subtitle{Cheese is made from milk.}
\author{}
\institute[]{}
\date[]

%%%%%%%%%%%%%%%%%%%%%%%%%%%%
%% TEXT
%%%%%%%%%%%%%%%%%%%%%%%%%%%%
\begin{document}
\begin{frame}[plain]
  \titlepage
\end{frame}

\section*{授業の流れ}
\begin{frame}[plain]
  \frametitle{授業の流れ}
  \tableofcontents
\end{frame}

\section{受け身}

\subsection{受け身とは}

\begin{frame}[plain]{受け身とは}
 \Large

\begin{enumerate}
 \item<1-> \begin{enumerate}
	\item<1-> 先生は彼女をほめました。
	\item<2-> 彼女は先生にほめられました。
       \end{enumerate}
 \item<3-> \begin{enumerate}
	\item<3-> ジョンはジェニファーを愛しています。
	\item<4-> ジェニファーはジョンに愛されています。
       \end{enumerate}
 \item<5-> \begin{enumerate}
	\item<5-> 牛乳からチーズを作ります。
	\item<6-> チーズは牛乳から作られます。
       \end{enumerate}
\end{enumerate}

\begin{exampleblock}<7->{Topic for Today}
\begin{itemize}\setbeamertemplate{items}[square]\small
 \item \kenten{受け身}とは「〜される」の意味を表します
\end{itemize}
     \end{exampleblock}

\end{frame}
%%%%%%%%%%%%%%%%%%%%%%%%%%%%%%%%%%%
\section{受け身の基本:be $+$ 過去分詞}
\begin{frame}[plain]{be $+$ 過去分詞}
 \large
\begin{enumerate}
 \item<1-> They \textbf{speak} English.\hspace{1\zw}{{\small 現在形}}\hfill{}{\small 彼らは英語を話す。}
 \item<2-> They \textbf{spoke} English.\hspace{1\zw}{{\small 過去形}}\hfill{\small 彼らは英語を話した。}
 \item<3-> English \textcolor{Maroon}{\bfseries is} \textcolor{NavyBlue}{\bfseries spoken} there.\hspace{1\zw}{{\small 過去分詞}}\hfill{\small そこでは英語が話されている。}
\end{enumerate}

\begin{center}
 
\visible<4->{{\small%
\begin{tabular}{lll}
{\small 原形}&{\small 過去形}&{\small 過去分詞}\\\hline
speak&spoke&spoken
\end{tabular}%
}}
\end{center}

%\vfill

\hfill{\tiny 0133}\,{\scriptsize \myaudio{./audio/051_passive_01.mp3}}
\begin{exampleblock}<5->{Topics for Today}
\begin{itemize}\setbeamertemplate{items}[square]\small
 \item 「過去分詞」は「受け身」の意味を表します\\
\hfill{\textdbend\,過去分詞は過去とは関係ありません}\scalebox{2}{😭}
 \item \Circled[fill color=white]{\,be動詞$+$ 過去分詞\,}\, $\longrightarrow$\,\,\,\,受け身(〜される)
\end{itemize}
     \end{exampleblock}
\end{frame}
%%%%%%%%%%%%%%%%%%%%%%%%%%%%%%%%
\begin{frame}[plain]{Exercises}
 
{\small 次の各文の意味を考えましょう}\hfill{\tiny 0325}\,{\scriptsize \myaudio{./audio/051_passive_02.mp3}}

\begin{enumerate}
 \item The picture \textbf{was painted} by her.\hfill{}{ \scriptsize painted \textipa{/p\'eIntId/}: paintの過去分詞}
 \item The car \textbf{is washed} every Sunday.\hfill{}{\scriptsize washed \textipa{/w\'ASt/}: washの過去分詞}
 \item The computer \textbf{is used} by students.\hfill{}{\scriptsize used \textipa{/j\'u:zd/} : useの過去分詞}
 \item The cat \textbf{is loved by} the family.\hfill{}{\scriptsize loved \textipa{/l\'\textturnv vd/} : loveの過去分詞}
 \item The house \textbf{was built} ten years ago.\hfill{}{\scriptsize built \textipa{/b\'Ilt/} : buildの過去分詞}
 \item The toy \textbf{was made} in Canada.\hfill{}{\scriptsize made \textipa{/m\'eId/} : makeの過去分詞}
 \item The book \textbf{is read} by many people.\hfill{}{\scriptsize read \textipa{/r\'ed/} : readの過去分詞}
\end{enumerate}

\hfill{\scriptsize by ~(~によって)$\longleftarrow$受け身の文でよく使われる表現です}

\pause

{\small これらの動詞の過去分詞は、過去形と同じ}

\vspace{-5pt}

{\small 過去形がしっかり身についていればだいじょうぶ}
\end{frame}
%%%%%%%%%%%%%%%%%%%%%%%%%%%%%%%%%%%
\begin{frame}[plain]{過去形 $=$ 過去分詞}

 \begin{enumerate}
  \item paint --- painted --- painted\tikzmark{A1}%
\hfill\visible<2->{{\scriptsize \tikzmark{B1}最後がedで終わります}}
  \item wash --- washed --- washed\tikzmark{A2}
  \item use --- used --- used\tikzmark{A3}
  \item love --- loved --- loved\tikzmark{A4}%
  \item build --- built --- built\tikzmark{A5}%
\hfill\visible<4->{{\scriptsize \tikzmark{B2}不規則動詞です}}
  \item make --- made --- made\tikzmark{A6}
  \item read --- read --- read\tikzmark{A7}
 \end{enumerate}

\begin{tikzpicture}[remember picture,overlay]
 \visible<2->{\draw[<-,opacity=0.5, line width=2pt,Maroon] ([xshift=2pt,yshift=3pt]pic cs:A1) to node[sloped,above,text=black,font=\tiny,pos=.4]{原形にedつけて} ([xshift=-2pt, yshift=3pt] pic cs:B1);}
 \visible<2->{\draw[<-,opacity=0.5, line width=2pt,Maroon] ([xshift=2pt,yshift=3pt]pic cs:A2) to ([xshift=-2pt, yshift=3pt] pic cs:B1);}
 \visible<3->{\draw[<-,opacity=0.5, line width=2pt,Maroon,dashed] ([xshift=2pt,yshift=3pt]pic cs:A3) to ([xshift=-2pt, yshift=3pt] pic cs:B1);}
 \visible<3->{\draw[<-,opacity=0.5, line width=2pt,Maroon,dashed] ([xshift=2pt,yshift=3pt]pic cs:A4) to node[sloped,above,text=black,font=\tiny,pos=.2]{原形にdつけて} ([xshift=-2pt, yshift=3pt] pic cs:B1);}
 %%%%%%%%%%%%%%%%%%%%%%
 \visible<4->{\draw[<-,opacity=0.5, line width=2pt,NavyBlue] ([xshift=2pt,yshift=3pt]pic cs:A5) to ([xshift=-2pt, yshift=3pt] pic cs:B2);}
 \visible<4->{\draw[<-,opacity=0.5, line width=2pt,NavyBlue] ([xshift=2pt,yshift=3pt]pic cs:A6) to ([xshift=-2pt, yshift=3pt] pic cs:B2);}
 \visible<4->{\draw[<-,opacity=0.5, line width=2pt,NavyBlue] ([xshift=2pt,yshift=3pt]pic cs:A7) to ([xshift=-2pt, yshift=3pt] pic cs:B2);}
\end{tikzpicture}

\visible<5->{{\scriptsize \textdbend 覚えるには、なんども口ずさんだり、紙に書いたりするのが効果的です!}}
\hfill{\tiny 0428}\,{\scriptsize \myaudio{./audio/051_passive_02a.mp3}}
\end{frame}
%%%%%%%%%%%%%%%%%%%%%%%%%%%%%%%%%%%%
\begin{frame}[plain]{Exercises}
 次の各文の意味を考えましょう\hfill{\tiny 0328}\,{\scriptsize \myaudio{./audio/051_passive_03.mp3}}


 \begin{enumerate}
\item English \textbf{is spoken} in many countries.\hfill{}{\scriptsize spoken \textipa{/sp\'oUkn/} : speakの過去分詞}
\item The photo \textbf{was taken} by my friend.\hfill{}{\scriptsize taken \textipa{/t\'eIkn/} : takeの過去分詞 by: 〜によって}
\item The window \textbf{was broken} by the ball.\hfill{} {\scriptsize broken \textipa{/br\'oUkn/} : breakの過去分詞}
\item The cake \textbf{was eaten} by the family.\hfill{}{\scriptsize eaten \textipa{/\'\i:tn/} : eatの過去分詞}
\item The movie \textbf{was seen} by many people.\hfill{}{\scriptsize seen \textipa{/s\'\i:n/} : seeの過去分詞}
\item The letter \textbf{was written} by my father.\hfill{}{\scriptsize written \textipa{/r\'Itn/} : writeの過去分詞}
\item She \textbf{is known} as a great singer.\hfill{}{\scriptsize known \textipa{/n\'oUn/} : knowの過去分詞 as: 〜として}
\end{enumerate}

\pause

\hfill{}{\small これらの動詞は、過去形と過去分詞がちがいます}

\end{frame}
%%%%%%%%%%%%%%%%%%%%%%%%
\begin{frame}[plain,label=table]{過去形 $\neq$ 過去分詞}
 過去形と過去分詞が異なる動詞の表%
\hfill{\tiny 0616}\,{\scriptsize \myaudio{./audio/051_passive_04.mp3}}


\begin{center}
 
\rowcolors{2}{NavyBlue!50}{yellow!50}
\begin{tabular}{lll}\toprule
{\small 原形      }&{\small 過去形     }&{\small 過去分詞    }\\\midrule
\visible<1->{speak \textipa{/sp\'\i:k/} }&\visible<2->{spoke \textipa{/sp\'oUk/} }&\visible<3->{spoken \textipa{/sp\'oUkn/} }\\
\visible<1->{take \textipa{/t\'eIk/} }&\visible<4->{took \textipa{/t\'Uk/} }&\visible<5->{taken \textipa{/t\'eIkn/} }\\
\visible<1->{break \textipa{/br\'eIk/} }&\visible<6->{broke \textipa{/br\'oUk/} }&\visible<7->{broken \textipa{/br\'oUkn/} }\\
\visible<1->{eat \textipa{/\'\i:t/} }&\visible<8->{ate \textipa{/\'eIt/} }&\visible<9->{eaten \textipa{/\'\i:tn/} }\\
\visible<1->{see \textipa{/s\'\i:/} }&\visible<10->{saw \textipa{/s\'O:/} }&\visible<11->{seen \textipa{/s\'\i:n/} }\\
\visible<1->{write \textipa{/r\'aIt/} }&\visible<12->{wrote \textipa{/r\'oUt/} }&\visible<13->{written \textipa{/r\'Itn/} }\\
\visible<1->{know \textipa{/n\'oU/} }&\visible<14->{knew \textipa{/nj\'u:/} }&\visible<15->{known \textipa{/n\'oUn/} }\\\bottomrule
\end{tabular}%
\end{center}

\hfill{}\visible<16->{{\scriptsize \textdbend 過去分詞は---enで終わることが多い}}

\vspace{-8pt}

\hfill{}\visible<17->{{\scriptsize 覚えるには、なんども口ずさんだり、紙に書いたりするのが効果的です!}}
\end{frame}
%%%%%%%%%%%%%%%%%%%%%%
\begin{frame}<1-7>[plain]{Exercises}

{\small あたえられた日本文の意味になるよう、下の枠から動詞を選び適当な形にして(~~~~~~~~~~)に補いましょう}\hfill{\tiny 0302}\,{\scriptsize \myaudio{./audio/051_passive_05.mp3}
}
\begin{enumerate}
 \item The song is \alt<1>{(~~\phantom{loved}~~)}{(~~loved~~)} by many people.\hfill{\small その歌は多くの人から愛されている。}
 \item The vase was \alt<1-2>{(~~\phantom{broken}~~)}{(~~broken~~)} by the cat. \hfill{\small 花瓶がネコにこわされた。}
 \item The photo was \alt<1-3>{(~~\phantom{taken}~~)}{(~~taken~~)} by my mother.\hfill{\small その写真をとったのは私の母です。}
 \item Spanish is \alt<1-4>{(~~\phantom{spoken}~~)}{(~~spoken~~)} in Mexico.\hfill{\small メキシコではスペイン語が話されている。}
 \item Sushi is \alt<1-5>{(~~\phantom{eaten}~~)}{(~~eaten~~)} in the United States.\hfill{\small アメリカではすしが食べられている。}
 \item The book was \alt<1-6>{(~~\phantom{written}~~)}{(~~written~~)} by Hemingway. \\
\hfill{\small その本はヘミングウェイによって書かれた。}
\end{enumerate}

\begin{tcolorbox}\centering
 eat~~~~~~~~~~love~~~~~~~~~~write~~~~~~~~~~speak~~~~~~~~~~break~~~~~~~~~~take
\end{tcolorbox}
\end{frame}
%%%%%%%%%%%%%%%%%%%%%%%%%%%%
\section{受け身の否定}
%%%%%%%%%%%%%%%%%%%%%%%%%%%%
\begin{frame}[plain]{否定を表す語}
 \Large

否定を表す語:\pause {\LARGE\bfseries not}\hspace{20pt}\textipa{/n\'At/}


\hfill\begin{tikzpicture}
\duck[tshirt=black,
stripes={\stripes[color=white]},
football,
speech={\tiny 口を縦に全開},
bubblecolour=cyan!20!white,
think={ア},
laughing
]
\end{tikzpicture}


\end{frame}
%%%%%%%%%%%%%%%%%%%%%%%%%%%%%%%%%
\begin{frame}[plain]{受け身の否定}
\large

\visible<1->{Spanish \textcolor{NavyBlue}{\bfseries is spoken} in the country.}\hfill\visible<2->{{\scriptsize country \textipa{/k\'\textturnv ntri/} 国}}

\visible<3->{Spanish \textcolor{NavyBlue}{\bfseries is} \textcolor{Maroon}{\bfseries not} \textcolor{NavyBlue}{\bfseries spoken} in the country.} 

\vfill

\hfill{\tiny 0117}\,{\scriptsize \myaudio{./audio/051_passive_06.mp3}}

\begin{exampleblock}{Topic for Today}
\begin{itemize}\setbeamertemplate{items}[square]\small
 \item 「受け身」の否定 $\longrightarrow$\,\,\,\,`be動詞$+$ \textcolor{Maroon}{{\bfseries not}} $+$ 過去分詞'
\end{itemize}
     \end{exampleblock}

\end{frame}
%%%%%%%%%%%%%%%%%%%%%%%%%%%%%%%%%
\section{受け身の疑問文}
\begin{frame}[plain]{受け身の疑問文のつくり方}
\large

\visible<1->{English \alt<1-3>{\myAnch{be1}{white}{is}}{\myAnch{BE1}{orange}{is}} spoken there.}\hfill\visible<2->{{\scriptsize there \textipa{/D\'e\textrhookschwa /} そこでは}}

\vspace{15pt}

\visible<3->{\alt<1-3>{\myAnch{be2}{white}{Is}}{\myAnch{BE2}{orange}{Is}} English spoken there?}%
\hfill%
\visible<5->{\begin{tikzpicture}
\duck[signpost=\scalebox{0.3}{
\parbox{2.5cm}{\color{black}
前に\\出すだけ}},
signcolour=brown!70!gray,
signback=white!80!brown,
graduate=gray!20!black,
tassel=red!70!black,
speech={\tiny 語順がだいじ}
]
\end{tikzpicture}}  

\visible<4->{%
\begin{tikzpicture}[remember picture,overlay]
 \draw[thick,orange,->,line width=2pt,opacity=.5] (be1.south) to[out=-90, in=90] node[sloped,above,text=black,font=\tiny,pos=.4]{be動詞を前に} (be2.north);
\end{tikzpicture}
}

\vspace{-20pt}

\hfill{\tiny 0111}\,{\scriptsize \myaudio{./audio/051_passive_07.mp3}}

\begin{exampleblock}<6->{Topics for Today}\small
受け身の疑問文のつくり方
\begin{itemize}\setbeamertemplate{items}[square]
 \item<3->  be動詞を先頭へ\hfill{\scriptsize ふつうのbe動詞の文と同じ(He is kind.\,$\rightarrow$\,Is he kind?)}
 \item<4-> 文末に`?'をつける   イントネーションは\myRisingPitch
\end{itemize}
      \end{exampleblock}

\end{frame}
%%%%%%%%%%%%%%%%%%%%%%%%%%%%%
\subsection{受け身の疑問文への答え方}
\begin{frame}[plain]{受け身の疑問文への答え方}
 \Large

Is English spoken in the country?

\vspace{20pt}
\pause

\mbox{}\hspace{100pt}$\left\{\begin{tabular}{ll}
         \text{Yes, it is.}&\\\pause
         \text{No, it is not.}&\\\pause
         \text{(}= \text{No, it isn't.)}&
        \end{tabular}\right.$



\end{frame}
%%%%%%%%%%%%%%%%%%%%%%%%%%%%%
\begin{frame}[plain]{Exercises}
 
{\small 日本文を参考にして(~~~~~~~~)の語を並べかえ、英文を完成させましょう。
ただし先頭に来る単語は大文字で始めてください}\hfill{\tiny 0200}\,{\scriptsize \myaudio{./audio/051_passive_08.mp3}}

\begin{enumerate}
 \item {\small 家は毎日掃除されますか。}
( cleaned / the house / is ) every day?\\
\visible<2->{Is the house cleaned every day?}
 \item {\small その写真はハワイで撮られましたか。}
( the picture / taken /  was ) in Hawaii?\\
\visible<3->{Was the picture taken in Hawaii?}
 \item {\small チーズは牛乳から作られますか}。
( cheese / made / from / milk / is ) ?\\
\visible<4->{Is cheese made from milk?}
 \item {\small 彼は良い歌手として知られていましたか。}
( he / known /  was ) as a good singer?\\
\visible<5->{Was he known as a good singer?}
\end{enumerate}

\end{frame}

\section{まとめ}
\begin{frame}[plain]{要点}
 
\begin{exampleblock}{Topics for Today}
\begin{itemize}\setbeamertemplate{items}[square]\small
 \item `be動詞$+$ 過去分詞' $\longrightarrow$\,\,\,\,受け身(〜される)
 \item 「過去分詞」は「受け身」の意味を表します
        \begin{itemize}
	 \item 過去形と過去分詞が同じことが多い\\
\mbox{}\hfill{}(例: paint--painted--painted, love--loved--loved, make--made--made)
	 \item でも、異なることもある\\
\mbox{}\hfill{}(例: eat--ate--eaten, write--wrote--written, see--saw--seen, break--broke--broken)
	\end{itemize}
 \item 「受け身」の否定文と疑問文の作り方は、be動詞を用いた普通の文と同じです
\end{itemize}
     \end{exampleblock}
\end{frame}
%%%%%%%%%%%%%%%%%%%%%
\againframe<17>{table}




\end{document}


\begin{frame}[plain]{受け身とは}
 \Large

He makes cheese from milk.\hfill{}{\small 彼は牛乳からチーズを作る。}

They made cheese from milk.\hfill{}{\small 彼は牛乳からチーズを作った。}

Cheese is made from milk.\hfill{\small チーズは牛乳から作られる。}

\end{frame}

