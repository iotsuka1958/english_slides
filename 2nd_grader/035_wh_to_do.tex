\documentclass[aspectratio=169,xcolor={dvipsnames,table}]{beamer}
\usepackage[no-math,deluxe,haranoaji]{luatexja-preset}
\renewcommand{\kanjifamilydefault}{\gtdefault}
\renewcommand{\emph}[1]{{\upshape\bfseries #1}}
\usetheme{metropolis}
\metroset{block=fill}
\setbeamertemplate{navigation symbols}{}
\setbeamertemplate{blocks}[rounded][shadow=false]
\usecolortheme[rgb={0.7,0.2,0.2}]{structure}
%%%%%%%%%%%%%%%%%%%%%%%%%%
%% Change alert block colors
%%% 1- Block title (background and text)
\setbeamercolor{block title alerted}{fg=mDarkTeal, bg=mLightBrown!45!yellow!45}
\setbeamercolor{block title example}{fg=magenta!10!black, bg=mLightGreen!70}
%%% 2- Block body (background)
\setbeamercolor{block body alerted}{bg=mLightBrown!25}
\setbeamercolor{block body example}{bg=mLightGreen!15}
%%%%%%%%%%%%%%%%%%%%%%%%%%%
%%%%%%%%%%%%%%%%%%%%%%%%%%%
%% さまざまなアイコン
%%%%%%%%%%%%%%%%%%%%%%%%%%%
%\usepackage{fontawesome}
\usepackage{fontawesome5}
\usepackage{figchild}
\usepackage{twemojis}
\usepackage{utfsym}
\usepackage{bclogo}
\usepackage{marvosym}
\usepackage{fontmfizz}
\usepackage{pifont}
\usepackage{phaistos}
\usepackage{worldflags}
\usepackage{jigsaw}
\usepackage{tikzlings}
\usepackage{tikzducks}
\usepackage{scsnowman}
\usepackage{epsdice}
\usepackage{halloweenmath}
\usepackage{svrsymbols}
\usepackage{countriesofeurope}
\usepackage{tipa}
\usepackage{manfnt}
%%%%%%%%%%%%%%%%%%%%%%%%%%%
\usepackage{tikz}
\usetikzlibrary{calc,patterns,decorations.pathmorphing,backgrounds}
\usepackage{tcolorbox}
\usepackage{tikzpeople}
\usepackage{circledsteps}
\usepackage{xcolor}
\usepackage{amsmath}
\usepackage{booktabs}
\usepackage{chronology}
\usepackage{signchart}
%%%%%%%%%%%%%%%%%%%%%%%%%%%
%% 場合分け
%%%%%%%%%%%%%%%%%%%%%%%%%%%
\usepackage{cases}
%%%%%%%%%%%%%%%%%%%%%%%%%%
\usepackage{pdfpages}
%%%%%%%%%%%%%%%%%%%%%%%%%%%
%% 音声リンク表示
\newcommand{\myaudio}[1]{\href{#1}{\faVolumeUp}}
%%%%%%%%%%%%%%%%%%%%%%%%%%
%% \myAnch{<名前>}{<色>}{<テキスト>}
%% 指定のテキストを指定の色の四角枠で囲み, 指定の名前をもつTikZの
%% ノードとして出力する. 図には remember picture 属性を付けている
%% ので外部から参照可能である.
\newcommand*{\myAnch}[3]{%
  \tikz[remember picture,baseline=(#1.base)]
    \node[draw,rectangle,line width=1pt,#2] (#1) {\normalcolor #3};
}
%%%%%%%%%%%%%%%%%%%%%%%%%%
%% \myEmph コマンドの定義
%%%%%%%%%%%%%%%%%%%%%%%%%%
%\newcommand{\myEmph}[3]{%
%    \textbf<#1>{\color<#1>{#2}{#3}}%
%}
\usepackage{xparse} % xparseパッケージの読み込み
\NewDocumentCommand{\myEmph}{O{} m m}{%
    \def\argOne{#1}%
    \ifx\argOne\empty
        \textbf{\color{#2}{#3}}% オプション引数が省略された場合
    \else
        \textbf<#1>{\color<#1>{#2}{#3}}% オプション引数が指定された場合
    \fi
}
%%%%%%%%%%%%%%%%%%%%%%%%%%%
%%%%%%%%%%%%%%%%%%%%%%%%%%%
%% 文末の上昇イントネーション記号 \myRisingPitch
%% 通常のイントネーション \myDownwardPitch
%% https://note.com/dan_oyama/n/n8be58e8797b2
%%%%%%%%%%%%%%%%%%%%%%%%%%%
\newcommand{\myRisingPitch}{
\begin{tikzpicture}[scale=0.3,baseline=0.3]
\draw[->,>=stealth] (0,0) to[bend right=45] (1,1);
\end{tikzpicture}
}
\newcommand{\myDownwardPitch}{
\begin{tikzpicture}[scale=0.3,baseline=0.3]
\draw[->,>=stealth] (0,1) to[bend left=45] (1,0);
\end{tikzpicture}
}
%%%%%%%%%%%%%%%%%%%%%%%%%%%%
%\AtBeginSection[%
%]{%
%  \begin{frame}[plain]\frametitle{授業の流れ}
%     \tableofcontents[currentsection]
%   \end{frame}%
%}

%%%%%%%%%%%%%%%%%%%%%%%%%%%
\title{English is fun.}
\subtitle{I know how to make a sadwitch.}
\author{}
\institute[]{}
\date[]

%%%%%%%%%%%%%%%%%%%%%%%%%%%%
%% TEXT
%%%%%%%%%%%%%%%%%%%%%%%%%%%%
\begin{document}

\begin{frame}[plain]
  \titlepage
\end{frame}

\section*{授業の流れ}
\begin{frame}[plain]
  \frametitle{授業の流れ}
  \tableofcontents
\end{frame}

\section{Wh $+$ to --}
\subsection{how to --}
%%%%%%%%%%%%%%%%%%%%%%%%%%%%%%%%%%%%%%%%%%%%%
\begin{frame}[plain]{how to --}
 \begin{enumerate}
  \item<1-> I know it.
  \item<2-> I know the recipe.\hfill{\scriptsize recipe: レシピ}
  \item<3-> I know \myAnch{a1}{NavyBlue}{\myEmph[4-]{Maroon}{how to} make a sandwitch}.
 \end{enumerate}

\visible<5->{%
\begin{exampleblock}{Topics for Today}
\begin{itemize}\small
 \item \Circled[fill color=white]{ how to 動詞の原形 }\,は、
「どのように~したらいいか」、「~する方法」の意味
 \item  \Circled[fill color=white]{ how to 動詞の原形 }\, の全体が名詞のはたらき
 \end{itemize}
     \end{exampleblock}
}
\end{frame}
%%%%%%%%%%%%%%%%%%%%%
\begin{frame}[plain]{Exercises}
日本語の意味になるよう(~~~~~~)内の語句を並べ替えましょう。先頭に来る語は大文字で書き始めてください。
 \begin{enumerate}
  \item わたしは寿司の作り方を知りません。\\
	( make / I / to / know / sushi / how / don't ) .\\
	\visible<2->{I don't know how to make sushi.}
  \item チェスの遊び方を知っていますか。\\
	( chess / know / play / you / to / do / how ) ?\\
	\visible<3->{Do you know how to play chess?}
  \item この地図には駅への行き方がのっています。\hfill{\scriptsize get to ~: ~へ到着する}\\
	This map ( how / get / shows / to / to ) the station.\\
	\visible<4->{This map shows how to get to the station.}
 \end{enumerate}
\end{frame}
%%%%%%%%%%%%%%%%%%%%%
\begin{frame}[plain]{Wh $+$ to --}
\begin{enumerate}
 \item<1->  When she was five, she learned \fcolorbox{NavyBlue}{white}{\textcolor{Maroon}{how to} swim}.
 \item<2-> I was thinking about \fcolorbox{NavyBlue}{white}{\textcolor{Maroon}{when to} start the meeting}.
 \item<3-> They discussed \fcolorbox{NavyBlue}{white}{\textcolor{Maroon}{where to} go next Sunday}.\hfill{\scriptsize discuss: 議論する}
 \item<4-> He didn't know \fcolorbox{NavyBlue}{white}{\textcolor{Maroon}{what to} wear for hot weather}.\hfill{\scriptsize weather: 天候}
 \item<5-> I didn't know \fcolorbox{NavyBlue}{white}{\textcolor{Maroon}{who to} believe}.\hfill{\scriptsize believe: 信じる}
\end{enumerate}

 \visible<6->{%
\begin{exampleblock}{Topics for Today}
\begin{itemize}\small
 \item \Circled[fill color=white]{ 疑問詞 $+$ to 動詞の原形 }\,では、how以外の疑問詞も使われます
 \item  \Circled[fill color=white]{ 疑問詞 $+$ to 動詞の原形 }\, の全体が名詞のはたらき
 \end{itemize}
     \end{exampleblock}
}
\end{frame}
%%%%%%%%%%%%%%%%%%%%%%%%%%
\begin{frame}[plain]{Wh $+$ to --}
 \begin{tblr}{colspec={ll}}
  how to $+$ 動詞の原形&どのように~したらいいか、~する方法\\
  when to $+$ 動詞の原形&いつ~したらいいか\\
  where to $+$ 動詞の原形&どこで(どこへ)~したらいいか\\
  what to $+$ 動詞の原形&なにを~したらいいか\\
  who to $+$ 動詞の原形&だれを~したらいいか
   \end{tblr}
\end{frame}
%%%%%%%%%%%%%%%%%%%%%%%%%%%%%
\begin{frame}[plain]{Exercises}
日本語の意味になるよう(~~~~~~)内の単語を並べ替えましょう

 \begin{enumerate}
  \item eメールの送り方を知っていますか。\\%
	Do you know ( send / to / an e-mail / how ) ?\hfill{}
	\visible<2->{\textcolor{BurntOrange}{\bfseries how to send an e-mail}}
  \item いつ彼を訪ねるべきかかんがえています。\\
	I am thinking about ( him / visit / when / to ) .\hfill{}
	\visible<3->{\textcolor{BurntOrange}{\bfseries when to visit him}}
  \item 彼女はどこでチケットを買たらいいかたずねた。\hfill{\scriptsize buy: 買う}\\
	She asked ( the ticket / where / buy / to  ) .\hfill{}
	\visible<4->{\textcolor{BurntOrange}{\bfseries where to buy the ticket}}\\
  \item 彼女は何を買ったらいいか決めていなかった。\hfill{\scriptsize decide: 決める}\\
	She didn't decide (buy / what / to ) .
	\hfill\visible<5->{\textcolor{BurntOrange}{\bfseries what to buy}}
  \item だれをパーティに招待したらいいか、私はわからなかった。\hfill{\scriptsize invite: 招待する}\\
	I didn't know ( invite / to / who ) to the party.\hfill{}
	\visible<6->{\textcolor{BurntOrange}{\bfseries who to invite}}
 \end{enumerate}
\end{frame}
\end{document}
