\documentclass[aspectratio=169,xcolor={dvipsnames,table}]{beamer}
\usepackage[no-math,deluxe,haranoaji]{luatexja-preset}
\renewcommand{\kanjifamilydefault}{\gtdefault}
\renewcommand{\emph}[1]{{\upshape\bfseries #1}}
\usetheme{metropolis}
\metroset{block=fill}
\setbeamertemplate{navigation symbols}{}
\setbeamertemplate{blocks}[rounded][shadow=false]
\usecolortheme[rgb={0.7,0.2,0.2}]{structure}
%%%%%%%%%%%%%%%%%%%%%%%%%%
%% Change alert block colors
%%% 1- Block title (background and text)
\setbeamercolor{block title alerted}{fg=mDarkTeal, bg=mLightBrown!45!yellow!45}
\setbeamercolor{block title example}{fg=magenta!10!black, bg=mLightGreen!70}
%%% 2- Block body (background)
\setbeamercolor{block body alerted}{bg=mLightBrown!25}
\setbeamercolor{block body example}{bg=mLightGreen!15}
%%%%%%%%%%%%%%%%%%%%%%%%%%%
%%%%%%%%%%%%%%%%%%%%%%%%%%%
%% さまざまなアイコン
%%%%%%%%%%%%%%%%%%%%%%%%%%%
%\usepackage{fontawesome}
\usepackage{fontawesome5}
\usepackage{figchild}
\usepackage{twemojis}
\usepackage{utfsym}
\usepackage{bclogo}
\usepackage{marvosym}
\usepackage{fontmfizz}
\usepackage{pifont}
\usepackage{phaistos}
\usepackage{worldflags}
\usepackage{jigsaw}
\usepackage{tikzlings}
\usepackage{tikzducks}
\usepackage{scsnowman}
\usepackage{epsdice}
\usepackage{halloweenmath}
\usepackage{svrsymbols}
\usepackage{countriesofeurope}
\usepackage{tipa}
%%%%%%%%%%%%%%%%%%%%%%%%%%%
\usepackage{tikz}
\usetikzlibrary{calc,patterns,decorations.pathmorphing,backgrounds}
\usepackage{tcolorbox}
\usepackage{tikzpeople}
\usepackage{circledsteps}
\usepackage{xcolor}
\usepackage{amsmath}
\usepackage{booktabs}
\usepackage{chronology}
\usepackage{signchart}
%%%%%%%%%%%%%%%%%%%%%%%%%%%
%% 場合分け
%%%%%%%%%%%%%%%%%%%%%%%%%%%
\usepackage{cases}
%%%%%%%%%%%%%%%%%%%%%%%%%%
\usepackage{pdfpages}
%%%%%%%%%%%%%%%%%%%%%%%%%%%
%% 音声リンク表示
\newcommand{\myaudio}[1]{\href{#1}{\faVolumeUp}}
%%%%%%%%%%%%%%%%%%%%%%%%%%
%% \myAnch{<名前>}{<色>}{<テキスト>}
%% 指定のテキストを指定の色の四角枠で囲み, 指定の名前をもつTikZの
%% ノードとして出力する. 図には remember picture 属性を付けている
%% ので外部から参照可能である.
\newcommand*{\myAnch}[3]{%
  \tikz[remember picture,baseline=(#1.base)]
    \node[draw,rectangle,line width=1pt,#2] (#1) {\normalcolor #3};
}
%%%%%%%%%%%%%%%%%%%%%%%%%%
%% \myEmph コマンドの定義
%%%%%%%%%%%%%%%%%%%%%%%%%%
%\newcommand{\myEmph}[3]{%
%    \textbf<#1>{\color<#1>{#2}{#3}}%
%}
\usepackage{xparse} % xparseパッケージの読み込み
\NewDocumentCommand{\myEmph}{O{} m m}{%
    \def\argOne{#1}%
    \ifx\argOne\empty
        \textbf{\color{#2}{#3}}% オプション引数が省略された場合
    \else
        \textbf<#1>{\color<#1>{#2}{#3}}% オプション引数が指定された場合
    \fi
}
%%%%%%%%%%%%%%%%%%%%%%%%%%%
%%%%%%%%%%%%%%%%%%%%%%%%%%%
%% 文末の上昇イントネーション記号 \myRisingPitch
%% 通常のイントネーション \myDownwardPitch
%% https://note.com/dan_oyama/n/n8be58e8797b2
%%%%%%%%%%%%%%%%%%%%%%%%%%%
\newcommand{\myRisingPitch}{
\begin{tikzpicture}[scale=0.3,baseline=0.3]
\draw[->,>=stealth] (0,0) to[bend right=45] (1,1);
\end{tikzpicture}
}
\newcommand{\myDownwardPitch}{
\begin{tikzpicture}[scale=0.3,baseline=0.3]
\draw[->,>=stealth] (0,1) to[bend left=45] (1,0);
\end{tikzpicture}
}
%%%%%%%%%%%%%%%%%%%%%%%%%%%%
%\AtBeginSection[%
%]{%
%  \begin{frame}[plain]\frametitle{授業の流れ}
%     \tableofcontents[currentsection]
%   \end{frame}%
%}

%%%%%%%%%%%%%%%%%%%%%%%%%%%
\title{English is fun.}
\subtitle{I know how to make a sandwitch.}
\author{}
\institute[]{}
\date[]

%%%%%%%%%%%%%%%%%%%%%%%%%%%%
%% TEXT
%%%%%%%%%%%%%%%%%%%%%%%%%%%%
\begin{document}

\begin{frame}[plain]
  \titlepage
\end{frame}
%%%%%%%%%%%%%%%%%%%%%%%%%%%
%\section*{授業の流れ}
%\begin{frame}[plain]
%  \frametitle{授業の流れ}
%  \tableofcontents
%\end{frame}
%%%%%%%%%%%%%%%%%%%%%%%%%%%%%%%%%%%%%%%%%%%%
\section{how $+$ to --}
%%%%%%%%%%%%%%%%%%%%%%%%%%%%%%%%%%%%%%%%%%%%%
\begin{frame}[plain]{how to --}
 \begin{enumerate}
  \item<1-> I know it.\hfill{}it $=$ \textbf{O}
  \item<2-> I know the recipe.\hfill{\scriptsize recipe \textipa{/r\'es@pi/} レシピ}
  \item<3-> I know \myAnch{a1}{NavyBlue}{\myEmph[4-]{Maroon}{how to} make a sandwitch}.\hfill{\scriptsize sandwitch \textipa{/s\'\ae n(d)wItS/} サンドイッチ}
 \end{enumerate}
%
\hfill{\tiny 0133}\,{\scriptsize \myaudio{./audio/035_wh_to_do_01.mp3}}
%
\begin{block}<5->{Topics for Today}
\begin{itemize}\setbeamertemplate{items}[square]\small
 \item \Circled[fill color=white]{ how $+$ to不定詞 }\,は、
「どのように~したらいいか」、「~する方法」の意味
 \item  \Circled[fill color=white]{ how $+$ to不定詞 }\, の全体が名詞のはたらき
 \end{itemize}
     \end{block}
\end{frame}
%%%%%%%%%%%%%%%%%%%%
\begin{frame}[plain]{A BLT sandwitch}

\begin{columns}
\begin{column}{.55\textwidth} 
\includegraphics[height=\textheight]{./images/tomato_sandwitch.jpg}
\end{column}
%%%%%%%%%%%%%%%%%%
\begin{column}{.4\textwidth}
\begin{description}
 \item[B] Bacon 
 \item[L] Lettuce 
 \item[T] Tomato 
\end{description}

\bigskip

It's a classic American sandwich, usually served on toasted bread with mayonnaise.
\end{column}
\end{columns}
\end{frame}
%%%%%%%%%%%%%%%%%%%%%
\begin{frame}[plain]{Exercises}
日本語の意味になるよう(~~~~~~)内の語句を並べ替えましょう。先頭に来る語は大文字で書き始めてください%
\hfill{\tiny 0140}\,{\scriptsize \myaudio{./audio/035_wh_to_do_02.mp3}}
 \begin{enumerate}
  \item わたしは寿司の作り方を知りません。\\
	( make / I / to / know / sushi / how / don't ) .\\
	\visible<2->{I don't know how to make sushi.}
  \item チェスの遊び方を知っていますか。\\
	( chess / know / play / you / to / do / how ) ?\\
	\visible<3->{Do you know how to play chess?}
  \item この地図には駅への行き方がのっています。\hfill{\scriptsize show: 示す get to ~: ~に着く}\\
	This map ( how / get / shows / to / to ) the station.\\
	\visible<4->{This map shows how to get to the station.}\\
	\hfill\visible<5->{\scriptsize 「この地図は駅への行き方を示している」という発想です}
 \end{enumerate}
\end{frame}
%%%%%%%%%%%%%%%%%%%%%
\begin{frame}[plain]{Chess}

\begin{columns}
\begin{column}{.6\textwidth} 
\includegraphics[height=\textheight]{./images/playing_chess.jpg}
%\includegraphics[height=\textheight]{../1st_grader/images/chess.jpg}
\end{column}
%%%%%%%%%%%%%%%%%%
\begin{column}{.35\textwidth}
チェスは2人で遊ぶボードゲーム
\end{column}
\end{columns}
\end{frame}
%%%%%%%%%%%%%%%%%%%%%
\section{wh-word $+$ to --}
%%%%%%%%%%%%%%%%%
\begin{frame}[plain]{Wh-word $+$ to --}
\begin{enumerate}
 \item<1->  When she was five, she learned \fcolorbox{NavyBlue}{white}{\textcolor{Maroon}{how to} swim}.%
\hfill{\tiny 0303}\,{\scriptsize \myaudio{./audio/035_wh_to_do_03.mp3}}
 \item<2-> I was thinking about \fcolorbox{NavyBlue}{white}{\textcolor{Maroon}{when to} start the meeting}.\hfill{\scriptsize think about ~について考える}
 \item<3-> They discussed \fcolorbox{NavyBlue}{white}{\textcolor{Maroon}{where to} go next Sunday}.\hfill{\scriptsize discuss \textipa{/dIsk\'\textturnv s/} 議論する}
 \item<4-> He didn't know \fcolorbox{NavyBlue}{white}{\textcolor{Maroon}{what to} wear for hot weather}.\hfill{\scriptsize weather \textipa{/w\'eD\textrhookschwa /} 天候}
 \item<5-> I didn't know \fcolorbox{NavyBlue}{white}{\textcolor{Maroon}{who to} believe}.\hfill{\scriptsize believe \textipa{/b@l\'i:v/} 信じる}
\item<6-> Tell me \fcolorbox{NavyBlue}{white}{\textcolor{Maroon}{which to} choose}.\hfill{\scriptsize tell me: 教えて choose \textipa{/tS\'u:z/} 選ぶ}
\end{enumerate}

 \begin{block}<7->{Topics for Today}
\begin{itemize}\setbeamertemplate{items}[square]\small
 \item \Circled[fill color=white]{ 疑問詞 $+$ to不定詞 }\,で、how以外の疑問詞wh-wordも使われます
 \item  \Circled[fill color=white]{ 疑問詞 $+$ to不定詞 }\, の全体が名詞のはたらき
 \end{itemize}
     \end{block}

\end{frame}
%%%%%%%%%%%%%%%%%%%%%%%%%%
\begin{frame}[plain,label=table]{Wh $+$ to --}
 \begin{tblr}{colspec={ll}}
  \Circled[fill color=yellow!20]{ how $+$ to不定詞 }&どのように~したらいいか、~する方法\\
  \Circled[fill color=yellow!20]{ when $+$ to不定詞 }&いつ~したらいいか\\
  \Circled[fill color=yellow!20]{ where $+$ to不定詞 }&どこで(どこへ)~したらいいか\\
  \Circled[fill color=yellow!20]{ what $+$ to不定詞 }&なにを~したらいいか\\
  \Circled[fill color=yellow!20]{ who $+$ to不定詞 }&だれを~したらいいか\\
  \Circled[fill color=yellow!20]{ which $+$ to不定詞 }&どちら(どれ)を~したらいいか
   \end{tblr}
\end{frame}
%%%%%%%%%%%%%%%%%%%%%%%%%%%%%
\begin{frame}[plain]{Exercises}
日本語の意味になるよう(~~~~~~)内の単語を並べ替えましょう%
\hfill{\tiny 0232}\,{\scriptsize \myaudio{./audio/035_wh_to_do_04.mp3}}

 \begin{enumerate}
  \item eメールの送り方を知っていますか。\hfill{\scriptsize send \textipa{/s\'end/} 送る}\\%
	Do you know ( send / to / an e-mail / how ) ?\hfill{}
	\visible<2->{\textcolor{BurntOrange}{\bfseries how to send an e-mail}}
  \item いつ彼を訪ねるべきかかんがえています。\\
	I am thinking about ( him / visit / when / to ) .\hfill{}
	\visible<3->{\textcolor{BurntOrange}{\bfseries when to visit him}}
  \item 彼女はどこでチケットを買ったらいいかたずねた。\hfill{\scriptsize buy \textipa{/b\'aI/} 買う}\\
	She asked ( the ticket / where / buy / to  ) .\hfill{}
	\visible<4->{\textcolor{BurntOrange}{\bfseries where to buy the ticket}}\\
  \item 彼女は何を買ったらいいか決めていなかった。\hfill{\scriptsize decide \textipa{/dIs\'aId/} 決める}\\
	She didn't decide ( buy / what / to ) .
	\hfill\visible<5->{\textcolor{BurntOrange}{\bfseries what to buy}}
  \item だれをパーティに招待したらいいか、私はわからなかった。\hfill{\scriptsize invite \textipa{/Inv\'aIt/} 招待する}\\
	I didn't know ( invite / to / who ) to the party.\hfill{}
	\visible<6->{\textcolor{BurntOrange}{\bfseries who to invite}}
 \end{enumerate}
\end{frame}
%%%%%%%%%%%%%%%%%%%%%%%%%
\section{which 名詞 $+$ to --}
\begin{frame}[plain]{which 名詞 $+$ to --}

\dbend

\bigskip

 \begin{enumerate}
  \item<1-> Tell me \fcolorbox{NavyBlue}{white}{the answer}.\hfill{\scriptsize tell me: 教えて}
  \item<2-> Tell me \fcolorbox{NavyBlue}{white}{\textcolor{Maroon}{which to} choose}.\hfill{\scriptsize choose \textipa{/tS\'u:z/} 選ぶ}
  \item<3-> Tell me \fcolorbox{NavyBlue}{white}{\textcolor{Maroon}{which} book \textcolor{Maroon}{to} choose}.
  \item<4-> Tell me \fcolorbox{NavyBlue}{white}{\textcolor{Maroon}{which} book \textcolor{Maroon}{to} read}.
  \item<5-> Tell me \fcolorbox{NavyBlue}{white}{\textcolor{Maroon}{which} book \textcolor{Maroon}{to} buy}.
 \end{enumerate}
%
\vfill

\begin{block}<6->{Topics for Today}
\begin{itemize}\setbeamertemplate{items}[square]\small
 \item \textbf{which}は名詞をしたがえることがあります
 \item \Circled[fill color=white]{ which 名詞 $+$ to不定詞 }\,は、
「どちらの(どの)\ldots{}\,を~したらいいか」の意味
 \end{itemize}
     \end{block}
\hfill{\tiny 0225}\,{\scriptsize \myaudio{./audio/035_wh_to_do_05.mp3}}

\end{frame}
%%%%%%%%%%%%%%%%%%%%%%
\section{聞いてみよう、読んでみよう{\tiny 0034}\,{\scriptsize \myaudio{./audio/035_wh_to_do_reading.mp3}}}
%%%%%%%%%%%%%%%%%%%%%%
%%%%%%%%%%%%%%%%%%%%%%%%%%%%%%%
\begin{frame}[plain]
 
\includegraphics[width=1.01\textwidth]{./images/nanobanana-output/035_wh_to_do_reading.png}

\vspace{-15pt}

\hfill{\tiny 0034}\,{\scriptsize \myaudio{./audio/035_wh_to_do_reading.mp3}}

\end{frame}
%%%%%%%%%%%%%%%%%%%
%%%%%%%%%%%%%%%%%%%%%%
\begin{frame}[plain,t]{Exercises}

\begin{tcolorbox}[colframe=ForestGreen,
  colback=ForestGreen!10!white,
  colbacktitle=ForestGreen!40!white,
  coltitle=black, %fonttitle=\bfseries,
before upper={\setlength{\parindent}{1.25em}},
 title=次の英文を読みましょう\mbox{}\hfill{\tiny 0032}\,{\scriptsize \myaudio{./audio/035_wh_to_do_reading.mp3}}
]

I want to have a lunch party next Sunday.
However\footnotemark[1], I have a problem.
I don't know what to make for my friends.

My friend, Yumi, helps me.
She knows how to make delicious sandwiches.
We look at a map on my phone to find where to buy fresh bread.
We also discuss\footnotemark[2] when to start cooking.

At the supermarket, there are many kinds\footnotemark[3] of cheese.
I don't know which to choose\footnotemark[4].
Yumi tells me the best one.
Now we are ready!
\vspace*{15pt}

\scriptsize
$^{1}$however:しかし\hfill%
$^{2}$discuss:話し合う\hfill%
$^{3}$however:kind\hfill%
$^{4}$choose:選ぶ
\end{tcolorbox}
\end{frame}
%%%%%%%%%%%%%%%%%%%%%%%%%%%%%%%%%
\begin{frame}[plain]{大意}
 
\begin{tcolorbox}
 わたしは、つぎの日曜日にランチ・パーティを開きたいとおもっています。
でも、わたしには悩みの種があります。
ともだちに何をつくってあげればいいのかわからないのです。

 ともだちのユミがわたしを助けてくれます。
彼女はおいしいサンドイッチのつくりかたを知っています。
焼きたてのパンをどこで買えばいいのか知るために、わたしたちはわたしの携帯のマップを見ます。
また、いつ調理を始めればいいか話し合います。

スーパーマーケットにはたくさんの種類のチーズがあります。
わたしにはどれを選べばいいのかわかりません。
ユミがいちばん上等のチーズを教えてくれます。
さあ、準備完了です。
\end{tcolorbox}

\end{frame}
%%%%%%%%%%%%%%%%%%%%%%%%%%%%%%%
%%%%%%%%%%%%%%%%%%%%%%%%
\section{疑問詞 $+$ to --- のまとめ}
%%%%%%%%%%%%%%%%%%%%%%%
\begin{frame}[plain]{まとめ}
  \begin{block}{Topics for Today}
\begin{itemize}\setbeamertemplate{items}[square]\small
 \item 疑問詞とto不定詞がセットになって\,\Circled[fill color=white]{ 疑問詞 $+$ to不定詞 }\,となることがあります
 \item  \Circled[fill color=white]{ 疑問詞 $+$ to不定詞 }\, は全体が名詞のはたらきをします
 \end{itemize}
     \end{block}

\begin{enumerate}
 \item<2->  When she was five, she learned \fcolorbox{NavyBlue}{white}{\textcolor{Maroon}{how to} swim}.%
\hfill{\tiny 0303}\,{\scriptsize \myaudio{./audio/035_wh_to_do_03.mp3}}
 \item<3-> I was thinking about \fcolorbox{NavyBlue}{white}{\textcolor{Maroon}{when to} start the meeting}.\hfill{\scriptsize think about ~について考える}
 \item<4-> They discussed \fcolorbox{NavyBlue}{white}{\textcolor{Maroon}{where to} go next Sunday}.\hfill{\scriptsize discuss \textipa{/dIsk\'\textturnv s/} 議論する}
 \item<5-> He didn't know \fcolorbox{NavyBlue}{white}{\textcolor{Maroon}{what to} wear for hot weather}.\hfill{\scriptsize weather \textipa{/w\'eD\textrhookschwa /} 天候}
 \item<6-> I didn't know \fcolorbox{NavyBlue}{white}{\textcolor{Maroon}{who to} believe}.\hfill{\scriptsize believe \textipa{/b@l\'i:v/} 信じる}
\item<7-> Tell me \fcolorbox{NavyBlue}{white}{\textcolor{Maroon}{which to} choose}.\hfill{\scriptsize tell me: 教えて choose \textipa{/tS\'u:z/} 選ぶ}
\end{enumerate}

\hfill\visible<8->{\scriptsize whichはTell me \textbf{which book to} choose.のように名詞をしたがえることがあります}
\end{frame}
%%%%%%%%%%%%%%%%%%%%%%%
\againframe{table}
%%%%%%%%%%%%%%%%%%%
%%%%%%%%%%%%%%%%%%%
\begin{frame}[plain]

 {\tiny audio\_overview 1715}\,{\scriptsize \myaudio{./audio/overview/035_wh_to_do_audio_overview.m4a}}
\end{frame}
%%%%%%%%%%%%%%%%%%
\end{document}
