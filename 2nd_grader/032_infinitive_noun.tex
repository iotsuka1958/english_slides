\documentclass[aspectratio=169,xcolor={dvipsnames,table}]{beamer}
\usepackage[no-math,deluxe,haranoaji]{luatexja-preset}
\renewcommand{\kanjifamilydefault}{\gtdefault}
\renewcommand{\emph}[1]{{\upshape\bfseries #1}}
\usetheme{metropolis}
\metroset{block=fill}
\setbeamertemplate{navigation symbols}{}
\setbeamertemplate{blocks}[rounded][shadow=false]
\usecolortheme[rgb={0.7,0.2,0.2}]{structure}
%%%%%%%%%%%%%%%%%%%%%%%%%%
%% Change alert block colors
%%% 1- Block title (background and text)
\setbeamercolor{block title alerted}{fg=mDarkTeal, bg=mLightBrown!45!yellow!45}
\setbeamercolor{block title example}{fg=magenta!10!black, bg=mLightGreen!70}
%%% 2- Block body (background)
\setbeamercolor{block body alerted}{bg=mLightBrown!25}
\setbeamercolor{block body example}{bg=mLightGreen!15}
%%%%%%%%%%%%%%%%%%%%%%%%%%%
\usepackage[absolute,overlay]{textpos}  %% 任意の位置に図を配置
%\usepackage[grid=true,gridcolor=Maroon,subgridcolor=gray,gridunit=pt,texcoord]{eso-pic} %%場所決めのための格子
%%%%%%%%%%%%%%%%%%%%%%%%%%%
%% さまざまなアイコン
%%%%%%%%%%%%%%%%%%%%%%%%%%%
%\usepackage{fontawesome}
\usepackage{fontawesome5}
\usepackage{figchild}
\usepackage{twemojis}
\usepackage{utfsym}
\usepackage{bclogo}
\usepackage{marvosym}
\usepackage{fontmfizz}
\usepackage{pifont}
\usepackage{phaistos}
\usepackage{worldflags}
\usepackage{jigsaw}
\usepackage{tikzlings}
\usepackage{tikzducks}
\usepackage{scsnowman}
\usepackage{epsdice}
\usepackage{halloweenmath}
\usepackage{svrsymbols}
\usepackage{countriesofeurope}
\usepackage{tipa}
\usepackage{manfnt}
%%%%%%%%%%%%%%%%%%%%%%%%%%%
\usepackage{tikz}
\usetikzlibrary{calc,patterns,decorations.pathmorphing,backgrounds}
\usepackage{tcolorbox}
\usepackage{tikzpeople}
\usepackage{circledsteps}
\usepackage{xcolor}
\usepackage{amsmath}
\usepackage{booktabs}
\usepackage{chronology}
\usepackage{signchart}
%%%%%%%%%%%%%%%%%%%%%%%%%%%
%% 場合分け
%%%%%%%%%%%%%%%%%%%%%%%%%%%
\usepackage{cases}
%%%%%%%%%%%%%%%%%%%%%%%%%%
\usepackage{pdfpages}
%%%%%%%%%%%%%%%%%%%%%%%%%%%
%% 音声リンク表示
\newcommand{\myaudio}[1]{\href{#1}{\faVolumeUp}}
%%%%%%%%%%%%%%%%%%%%%%%%%%
%% \myAnch{<名前>}{<色>}{<テキスト>}
%% 指定のテキストを指定の色の四角枠で囲み, 指定の名前をもつTikZの
%% ノードとして出力する. 図には remember picture 属性を付けている
%% ので外部から参照可能である.
\newcommand*{\myAnch}[3]{%
  \tikz[remember picture,baseline=(#1.base)]
    \node[draw,rectangle,line width=1pt,#2] (#1) {\normalcolor #3};
}
%%%%%%%%%%%%%%%%%%%%%%%%%%
%% \myEmph コマンドの定義
%%%%%%%%%%%%%%%%%%%%%%%%%%
%\newcommand{\myEmph}[3]{%
%    \textbf<#1>{\color<#1>{#2}{#3}}%
%}
\usepackage{xparse} % xparseパッケージの読み込み
\NewDocumentCommand{\myEmph}{O{} m m}{%
    \def\argOne{#1}%
    \ifx\argOne\empty
        \textbf{\color{#2}{#3}}% オプション引数が省略された場合
    \else
        \textbf<#1>{\color<#1>{#2}{#3}}% オプション引数が指定された場合
    \fi
}
%%%%%%%%%%%%%%%%%%%%%%%%%%%
%%%%%%%%%%%%%%%%%%%%%%%%%%%
%% 文末の上昇イントネーション記号 \myRisingPitch
%% 通常のイントネーション \myDownwardPitch
%% https://note.com/dan_oyama/n/n8be58e8797b2
%%%%%%%%%%%%%%%%%%%%%%%%%%%
\newcommand{\myRisingPitch}{
\begin{tikzpicture}[scale=0.3,baseline=0.3]
\draw[->,>=stealth] (0,0) to[bend right=45] (1,1);
\end{tikzpicture}
}
\newcommand{\myDownwardPitch}{
\begin{tikzpicture}[scale=0.3,baseline=0.3]
\draw[->,>=stealth] (0,1) to[bend left=45] (1,0);
\end{tikzpicture}
}
%%%%%%%%%%%%%%%%%%%%%%%%%%%%
%\AtBeginSection[%
%]{%
%  \begin{frame}[plain]\frametitle{授業の流れ}
%     \tableofcontents[currentsection]
%   \end{frame}%
%}

\usepackage{pxrubrica}
%%%%%%%%%%%%%%%%%%%%%%%%%%%
\title{English is fun.}
\subtitle{To study science is fun.}
\author{}
\institute[]{}
\date[]

%%%%%%%%%%%%%%%%%%%%%%%%%%%%
%% TEXT
%%%%%%%%%%%%%%%%%%%%%%%%%%%%
\begin{document}

\begin{frame}[plain]
  \titlepage
\end{frame}

\section*{授業の流れ}
\begin{frame}[plain]
  \frametitle{授業の流れ}
  \tableofcontents
\end{frame}

%%%%%%%%%%%%%%%%%%%%%%%%%%%%%%%%%%%%%%%%%%
\section{名詞的用法1}
\subsection{to不定詞 $=$ Sなど}
%%%%%%%%%%%%%%%%%%%%%%%%%%%%%%%%%%%%%%%%%%%
\begin{frame}[plain]{~すること}
 \begin{enumerate}
  \item \begin{enumerate}
	 \item<1-> Science is fun.%
\hfill{\tiny 0251}\,{\scriptsize \myaudio{./audio/032_infinitive_noun_01.mp3}}

	 \item<2-> \alt<3->{\fbox{\textbf{To study} science}}{To study science } is fun.	
	\end{enumerate}
  \item \begin{enumerate}
	 \item<4-> His idea is important.%
	      \hfill{\scriptsize important \textipa{/Imp\'O\textrhookschwa tnt/} 重要な}
	 \item<5-> \alt<6->{\fbox{\textbf{To understand} his idea}}{To understand his idea } is important.\mbox{}\hfill{\scriptsize understand \textipa{/\`\textturnv nd\textrhookschwa st\'\ae nd/} 理解する}
	\end{enumerate}
  \item \begin{enumerate}
	 \item<7-> The plan was a success.\mbox{}\hfill{\scriptsize success\textipa{/s@ks\'es/} 成功}
	 \item <8-> The plan was \alt<9->{\fbox{\textbf{to make} a hospital}}{to make a hospital}\,.%
\hfill{\scriptsize hospital \textipa{/h\'Aspitl/} 病院}
	\end{enumerate}
 \end{enumerate}

\visible<10->{%
\begin{block}{Topics for Today}
\begin{itemize}\setbeamertemplate{items}[square]\small
 \item<10->   to不定詞つまり\,\Circled[fill color= white]{\,$\text{to} + \text{動詞の原形}$\,}\,が「~すること」という意味を表し、名詞のはたらきをすることがあります
 %\item<11-> to不定詞つまり\,\Circled[fill color= white]{\,$\text{to} + \text{動詞の原形}$}\,が
      \begin{itemize}\setbeamertemplate{items}[circle]
       \item 文全体の主語になったり
       \item be動詞の直後にきたりします
      \end{itemize}
 \item to不定詞の\kenten{名詞的用法}といいます
 \end{itemize}
     \end{block}
}
\end{frame}
%%%%%%%%%%%%%%%%%%%%%%%%%%%%%%%%%%%%%%%%%%%%%
\section{名詞的用法2}
\subsection{to不定詞 $=$ 目的語}
%%%%%%%%%%%%%%%%%%%%%%%%%%%%%%%%%%%%%%%%%%%%%
\begin{frame}[plain]{一般動詞 $+$ to不定詞}
\large 

\begin{enumerate}
 \item \begin{enumerate}
	\item<1-> I like \fbox{~~~X~~~}\,.%
	\item<2-> I like cookies.\hfill{}O $=$ cookies
	\item<3-> I like \alt<4->{\fbox{\textbf{to make} cookies}}{to make cookies}\,.\hfill{}O $=$ \fbox{\textbf{to make} cookies}
       \end{enumerate}
 \item \begin{enumerate}
	\item<5-> He likes cookies.
	\item<6-> He likes \alt<7->{\fbox{\textbf{to make} cookies}}{to make cookies}\,.
       \end{enumerate}
 \item \begin{enumerate}
	\item<8-> She liked cookies.
	\item<9-> She liked \alt<10->{\fbox{\textbf{to make} cookies}}{to make cookies}\,.
       \end{enumerate} 
\end{enumerate}

\vspace*{40pt}

\begin{block}<11->{Topic for Today}
\begin{itemize}\setbeamertemplate{items}[square]\small
% \item<11->   \Circled[fill color= white]{\,$\text{to} + \text{動詞の原形}$\,} が「~すること」という意味を表し、名詞のはたらきをすることがある
 \item<11->   名詞的用法のto不定詞が一般動詞の目的語になることがある
% \item<12-> 主語と組み合わせる動詞\,(\,I like \ldots{}, He likes \ldots{}, She liked \ldots\,)とはちがって、\\
%いつでも\,\Circled[fill color= white]{\,$\text{to} + \text{動詞の原形}$\,}\\
 \end{itemize}
     \end{block}

\vspace{-10pt}

\hfill{\tiny 0300}\,{\scriptsize \myaudio{./audio/032_infinitive_noun_02.mp3}}


\begin{textblock*}{0.4\linewidth}(318pt,110pt)
\visible<9->{\begin{tikzpicture}
\duck[scale=1.2,
signpost=\scalebox{0.33}{
\parbox{2.5cm}{\color{black}
V $+$ to不定詞}},
signcolour=brown!70!gray,
signback=white!80!brown,
graduate=gray!20!black,
tassel=red!70!black,
speech={\tiny メモメモ}
]
\end{tikzpicture}}
\end{textblock*}

\end{frame}
%%%%%%%%%%%%%%%%%%%%%
\subsection{頻出表現}
%%%%%%%%%%%%%%%%%%%%%
\begin{frame}[plain,t]{一般動詞 $+$ to不定詞}
\begin{enumerate}\setlength{\itemsep}{-1pt}
 \item I \textbf{like to} read books before bed.\hfill{\tiny 0312}\,{\scriptsize \myaudio{./audio/032_infinitive_noun_02a.mp3}}

 \item We will \textbf{begin to} study for the exam soon.%
       \hfill{\scriptsize exam \textipa{/Igz\'\ae m/} 試験}
 \item She \textbf{started to} read a new book yesterday.
 \item They \textbf{decided to} have a party for his birthday.%
\hfill{\scriptsize decide \textipa{/dIs\'aId/}決める、決心する}
 \item I \textbf{need to} finish my homework before dinner.
\hfill{\scriptsize need \textipa{/n\'\i:d/} 必要がある}
 \item Did they \textbf{continue to} play soccer until dark?
\hfill{\scriptsize continue \textipa{/k@nt\'Inju:/} 続ける}\\
\hfill{\scriptsize until \textipa{/@nt\'Il/} ~まで until dark 暗くなるまで}
\end{enumerate}

\begin{block}<2->{頻出表現}\small
 $\left\{
 \begin{tabular}[c]{>{\bfseries }l}
       like\\[-1pt]begin\\[-1pt]start\\[-1pt]decide\\[-1pt]need\\[-1pt]continue
 \end{tabular}
\right\}+\text{\textbf{to}不定詞}$ 
\end{block}

\begin{textblock*}{0.4\linewidth}(300pt,158pt)
\visible<3->{\begin{tikzpicture}
\pig[
signpost=\scalebox{0.5}{
\parbox{2.2cm}{\color{black}
\centering 身につけると\\表現の幅が\\広がります!}},
signcolour= brown!70!gray,
signback=white!80!brown
]
\end{tikzpicture}}
\end{textblock*}
\end{frame}
%%%%%%%%%%%%%%%%%%%%%%
\begin{frame}[plain,label=vtodo]{頻出表現}\large
 
\[
\vcenter{
  \hbox{$
    \left\{
    \begin{tabular}[c]{>{\bfseries}l}
      like\\
      begin\\
      start\\
      decide\\
      need\\
      continue
    \end{tabular}
    \right\} + \text{\textbf{to}不定詞}
  $}
}
\hspace{80pt}
\vcenter{
  \hbox{
    \begin{tikzpicture}
      \duck[
        laughing,
        bowtie,
        strawhat=brown!50!white,
        ribbon=black,
        think={\scriptsize よくでます},
        bubblecolour=white!50!pink
      ]
    \end{tikzpicture}
  }
}
\]

\bigskip

\hfill\textdbend\,\,{\small 名詞的用法のto不定詞が一般動詞の目的語($=$ O)になっています}

\end{frame}
%%%%%%%%%%%%%%%%%%%%%%
\begin{frame}[plain,t]{Exercises}

{\small つぎの英文を読みましょう}%
\hfill{\tiny 0536}\,{\scriptsize \myaudio{./audio/032_infinitive_noun_03.mp3}}

Every summer, Jane and her friends planned\footnote{plan: 計画する} fun activities\footnote{fun activities: 楽しい活動}.
Jane {\bfseries wanted to visit} many countries.
Tom {\bfseries liked to draw} pictures of the places they visited\footnote{the places they visited: 訪れた場所}.
Mary {\bfseries began to pack}\footnote{pack: つめる} her bags early.
John {\bfseries started to read} travel books\footnote{travel books: 旅行の本}.
They were so excited\footnote{excited: わくわくして}.
They {\bfseries decided to save}\footnote{save: 貯蓄する} money for their trip.
During their trip, they {\bfseries needed to try} new foods.
After the trip, they { \bfseries continued}\footnote{continue: 続ける} {\bfseries to talk} about their memories.
\end{frame}
%%%%%%%%%%%%%%%%%%%%%%
\begin{frame}[plain,t]{Exercises}

{\small つぎの英文を読みましょう}%
\hfill{\tiny 0536}\,{\scriptsize \myaudio{./audio/032_infinitive_noun_03.mp3}}

\begin{tcolorbox}
Every summer, Jane and her friends planned\footnotemark[1] fun activities\footnotemark[2].
Jane {\bfseries wanted to visit} many countries.
Tom {\bfseries liked to draw}\footnotemark[3] pictures of the places they visited\footnotemark[4].
Mary {\bfseries began to pack}\footnotemark[5] her bags early.
John {\bfseries started to read} travel books\footnotemark[6].
They were so excited\footnotemark[7].
They {\bfseries decided to save}\footnotemark[8] money for their trip.
During\footnotemark[9] their trip, they {\bfseries needed to try} new foods.
After the trip, they {\bfseries continued}\footnotemark[10] {\bfseries to talk} about their memories.
\end{tcolorbox}

\vspace*{40pt}
\scriptsize
$^{1}$plan: 計画する\hfill{}$^{2}$fun activity: 楽しい活動\hfill{}%
$^{3}$draw: (絵を)描く\hfill$^{4}$the places they visited: 彼らが訪ねた場所\\%
$^{5}$pack: 詰める\hfill{}$^{6}$travel books: 旅行の本\hfill{}%
$^{7}$excited: わくわくして\hfill{}$^{8}$save: 貯蓄する\hfill%
$^{9}$during: ~の間\\
$^{10}$continue: 続ける

\end{frame}
%%%%%%%%%%%%%%%%%%%%%%%%%%%%%%%%%
\section{to不定詞の名詞的用法(まとめ)}
\begin{frame}[plain]{まとめ}
 
\begin{block}{to不定詞の名詞的用法}
\begin{itemize}\setbeamertemplate{items}[square]\small
 \item  to不定詞つまり\,\Circled[fill color= white]{\,$\text{to} + \text{動詞の原形}$\,}\,が「~すること」という意味を表し、名詞のはたらきをすることがあります%
 \hfill{}to不定詞の\kenten{名詞的用法}といいます
 %\item<11-> to不定詞つまり\,\Circled[fill color= white]{\,$\text{to} + \text{動詞の原形}$}\,が
 \item 具体的には次の場合があります
      \begin{enumerate}\setbeamertemplate{items}[circle]
       \item 文全体の主語\hfill{\footnotesize \fbox{\textbf{To play} tennis} is fun.}
       \item be動詞の直後\hfill{\footnotesize The plan was \fbox{\textbf{to climb} Mt. Fuji.}}
       \item 一般動詞の直後\hfill{\footnotesize He began \fbox{\textbf{to write} a letter}.}
      \end{enumerate}
 \end{itemize}
     \end{block}

\hfill{\tiny 0139}\,{\scriptsize \myaudio{./audio/032_infinitive_noun_04.mp3}}
\end{frame}
%%%%%%%%%%%%%%%%%%%%%%%%%%%
\againframe{vtodo}
%%%%%%%%%%%%%%%%%%%%%%%%%%
%%%%%%%%%%%%%%%%%%%%%%%%%%%%%%%%%%%%%%%%%%%%%
\begin{frame}[plain]{to不定詞とは}

\large
\visible<2->{動詞とは本来、主語と組み合わせて使うもの}
\hfill\visible<3->{\myEmph[3-]{NavyBlue}{I} \myEmph[3-]{Maroon}{play} the piano and \myEmph[3-]{NavyBlue}{she} \myEmph[3-]{Maroon}{sings}.}

\visible<4->{だが、特別に

\begin{tcolorbox}[colback=yellow!10!white]
\myAnch{v}{white}{動詞}%
\hspace{25pt}%
\begin{tabular}{ll}
 \visible<4->{\myAnch{n}{white}{名詞のはたらき}}&\visible<5->{\fcolorbox{black}{white}{\textbf{To get} up early} is important.}\\
 \visible<4->{\myAnch{adj}{white}{形容詞のはたらき}}&\visible<6->{I want something \fcolorbox{black}{white}{\textbf{to drink}}.}\\
 \visible<4->{\myAnch{adv}{white}{副詞のはたらき}}&\visible<7->{She studied hard \fcolorbox{black}{white}{\textbf{to pass} the exam}}.
\end{tabular}
\end{tcolorbox}

をすることがある
}

\begin{tikzpicture}[remember picture, overlay]
 \visible<4->{\draw[line width=2pt,opacity=.75, gray, ->] (v.east) to[out=0, in=180] (n.west);} 
 \visible<4->{\draw[line width=2pt,opacity=.75, gray, ->] (v.east) to[out=0, in=180] (adj.west);} 
 \visible<4->{\draw[line width=2pt,opacity=.75, gray, ->] (v.east) to[out=0, in=180] (adv.west);} 
\end{tikzpicture}

%\vspace{-5pt}

%\hfill%
%\visible<9->{to不定詞:\Circled[fill color = yellow!10!white]{\,$\text{to} + \text{動詞の原形}$\,}}

\visible<8->{
\Circled[fill color = yellow!10!white]{\,$\text{to} + \text{動詞の原形}$\,}
のことをto\kenten{不定詞}といいます
}

\hfill{\tiny 0140}\,{\scriptsize \myaudio{./audio/031_infinitive_intro_01.mp3}}

\begin{textblock*}{0.4\linewidth}(330pt,170pt)
\visible<9->{\begin{tikzpicture}
\duck[signpost=\scalebox{0.3}{
\parbox{2.5cm}{\color{black}
{\Large to $+$原形}\\{\Large $=$ to不定詞}}},
signcolour=brown!70!gray,
signback=white!80!brown,
graduate=gray!20!black,
tassel=red!70!black,
speech={\tiny メモメモ}
]
\end{tikzpicture}}
\end{textblock*}
\end{frame}
%%%%%%%%%%%%%%%%%%%%%%%%
\end{document}
