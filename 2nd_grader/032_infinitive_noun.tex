\documentclass[aspectratio=169,xcolor={dvipsnames,table}]{beamer}
\usepackage[no-math,deluxe,haranoaji]{luatexja-preset}
\renewcommand{\kanjifamilydefault}{\gtdefault}
\renewcommand{\emph}[1]{{\upshape\bfseries #1}}
\usetheme{metropolis}
\metroset{block=fill}
\setbeamertemplate{navigation symbols}{}
\setbeamertemplate{blocks}[rounded][shadow=false]
\usecolortheme[rgb={0.7,0.2,0.2}]{structure}
%%%%%%%%%%%%%%%%%%%%%%%%%%
%% Change alert block colors
%%% 1- Block title (background and text)
\setbeamercolor{block title alerted}{fg=mDarkTeal, bg=mLightBrown!45!yellow!45}
\setbeamercolor{block title example}{fg=magenta!10!black, bg=mLightGreen!70}
%%% 2- Block body (background)
\setbeamercolor{block body alerted}{bg=mLightBrown!25}
\setbeamercolor{block body example}{bg=mLightGreen!15}
%%%%%%%%%%%%%%%%%%%%%%%%%%%
%%%%%%%%%%%%%%%%%%%%%%%%%%%
%% さまざまなアイコン
%%%%%%%%%%%%%%%%%%%%%%%%%%%
%\usepackage{fontawesome}
\usepackage{fontawesome5}
\usepackage{figchild}
\usepackage{twemojis}
\usepackage{utfsym}
\usepackage{bclogo}
\usepackage{marvosym}
\usepackage{fontmfizz}
\usepackage{pifont}
\usepackage{phaistos}
\usepackage{worldflags}
\usepackage{jigsaw}
\usepackage{tikzlings}
\usepackage{tikzducks}
\usepackage{scsnowman}
\usepackage{epsdice}
\usepackage{halloweenmath}
\usepackage{svrsymbols}
\usepackage{countriesofeurope}
\usepackage{tipa}
%%%%%%%%%%%%%%%%%%%%%%%%%%%
\usepackage{tikz}
\usetikzlibrary{calc,patterns,decorations.pathmorphing,backgrounds}
\usepackage{tcolorbox}
\usepackage{tikzpeople}
\usepackage{circledsteps}
\usepackage{xcolor}
\usepackage{amsmath}
\usepackage{booktabs}
\usepackage{chronology}
\usepackage{signchart}
%%%%%%%%%%%%%%%%%%%%%%%%%%%
%% 場合分け
%%%%%%%%%%%%%%%%%%%%%%%%%%%
\usepackage{cases}
%%%%%%%%%%%%%%%%%%%%%%%%%%
\usepackage{pdfpages}
%%%%%%%%%%%%%%%%%%%%%%%%%%%
%% 音声リンク表示
\newcommand{\myaudio}[1]{\href{#1}{\faVolumeUp}}
%%%%%%%%%%%%%%%%%%%%%%%%%%
%% \myAnch{<名前>}{<色>}{<テキスト>}
%% 指定のテキストを指定の色の四角枠で囲み, 指定の名前をもつTikZの
%% ノードとして出力する. 図には remember picture 属性を付けている
%% ので外部から参照可能である.
\newcommand*{\myAnch}[3]{%
  \tikz[remember picture,baseline=(#1.base)]
    \node[draw,rectangle,line width=1pt,#2] (#1) {\normalcolor #3};
}
%%%%%%%%%%%%%%%%%%%%%%%%%%
%% \myEmph コマンドの定義
%%%%%%%%%%%%%%%%%%%%%%%%%%
%\newcommand{\myEmph}[3]{%
%    \textbf<#1>{\color<#1>{#2}{#3}}%
%}
\usepackage{xparse} % xparseパッケージの読み込み
\NewDocumentCommand{\myEmph}{O{} m m}{%
    \def\argOne{#1}%
    \ifx\argOne\empty
        \textbf{\color{#2}{#3}}% オプション引数が省略された場合
    \else
        \textbf<#1>{\color<#1>{#2}{#3}}% オプション引数が指定された場合
    \fi
}
%%%%%%%%%%%%%%%%%%%%%%%%%%%
%%%%%%%%%%%%%%%%%%%%%%%%%%%
%% 文末の上昇イントネーション記号 \myRisingPitch
%% 通常のイントネーション \myDownwardPitch
%% https://note.com/dan_oyama/n/n8be58e8797b2
%%%%%%%%%%%%%%%%%%%%%%%%%%%
\newcommand{\myRisingPitch}{
\begin{tikzpicture}[scale=0.3,baseline=0.3]
\draw[->,>=stealth] (0,0) to[bend right=45] (1,1);
\end{tikzpicture}
}
\newcommand{\myDownwardPitch}{
\begin{tikzpicture}[scale=0.3,baseline=0.3]
\draw[->,>=stealth] (0,1) to[bend left=45] (1,0);
\end{tikzpicture}
}
%%%%%%%%%%%%%%%%%%%%%%%%%%%%
%\AtBeginSection[%
%]{%
%  \begin{frame}[plain]\frametitle{授業の流れ}
%     \tableofcontents[currentsection]
%   \end{frame}%
%}

\usepackage{pxrubrica}
%%%%%%%%%%%%%%%%%%%%%%%%%%%
\title{English is fun.}
\subtitle{To study science is fun.}
\author{}
\institute[]{}
\date[]

%%%%%%%%%%%%%%%%%%%%%%%%%%%%
%% TEXT
%%%%%%%%%%%%%%%%%%%%%%%%%%%%
\begin{document}

\begin{frame}[plain]
  \titlepage
\end{frame}

\section*{授業の流れ}
\begin{frame}[plain]
  \frametitle{授業の流れ}
  \tableofcontents
\end{frame}

\section{名詞的用法}
\subsection{名詞的用法}
%%%%%%%%%%%%%%%%%%%%%%%%%%%%%%%%%%%%%%%%%%%%
\begin{frame}[plain]{~すること}
 \begin{enumerate}
  \item \begin{enumerate}
	 \item<1-> Science is fun.%
\hfill{\tiny 0251}\,{\scriptsize \myaudio{./audio/032_infinitive_noun_01.mp3}}

	 \item<2-> \alt<3->{\fbox{To study science}}{To study science } is fun.	
	\end{enumerate}
  \item \begin{enumerate}
	 \item<4-> His idea is important.%
	      \hfill{\scriptsize important \textipa{/Imp\'O\textrhookschwa tnt/} 重要な}
	 \item<5-> \alt<6->{\fbox{To understand his idea}}{To understand his idea } is important.\mbox{}\hfill{\scriptsize understand \textipa{/\`\textturnv nd\textrhookschwa st\'\ae nd/} 理解する}
	\end{enumerate}
  \item \begin{enumerate}
	 \item<7-> The plan was a success.\mbox{}\hfill{\scriptsize success\textipa{/s@ks\'es/} 成功}
	 \item <8-> The plan was \alt<9->{\fbox{to make a hospital}}{to make a hospital}\,.%
\hfill{\scriptsize hospital \textipa{/h\'Aspitl/} 病院}
	\end{enumerate}
 \end{enumerate}

\visible<10->{%
\begin{block}{Topics for Today ---名詞的用法---}
\begin{itemize}\setbeamertemplate{items}[square]\small
 \item<10->   to不定詞つまり\,\Circled[fill color= white]{\,$\text{to} + \text{動詞の原形}$\,}\,が「~すること」という意味を表し、名詞のはたらきをすることがあります
 %\item<11-> to不定詞つまり\,\Circled[fill color= white]{\,$\text{to} + \text{動詞の原形}$}\,が
      \begin{itemize}\setbeamertemplate{items}[circle]
       \item 文全体の主語になったり
       \item be動詞の直後にきたりします
      \end{itemize}
 \item to不定詞の\kenten{名詞的用法}といいます
 \end{itemize}
     \end{block}
}
\end{frame}
%%%%%%%%%%%%%%%%%%%%%%%%%%%%%%%%%%%%%%%%%%%%%
\begin{frame}[plain]{一般動詞 $+$ to不定詞}
\large 

\begin{enumerate}
 \item \begin{enumerate}
	\item<1-> I like \fbox{~~~X~~~}\,.%
\hfill{\tiny 0300}\,{\scriptsize \myaudio{./audio/032_infinitive_noun_02.mp3}}
	\item<2-> I like cookies.\hfill{}O $=$ cookies
	\item<3-> I like \alt<4->{\fbox{to make cookies}}{to make cookies}\,.\hfill{}O $=$ \fbox{to make cookies}
       \end{enumerate}
 \item \begin{enumerate}
	\item<5-> He likes cookies.
	\item<6-> He likes \alt<7->{\fbox{to make cookies}}{to make cookies}\,.
       \end{enumerate}
 \item \begin{enumerate}
	\item<8-> She liked cookies.
	\item<9-> She liked \alt<10->{\fbox{to make cookies}}{to make cookies}\,.
       \end{enumerate} 
\end{enumerate}

\visible<11->{%
\begin{block}{Topics for Today ---名詞的用法---}
\begin{itemize}\setbeamertemplate{items}[square]\small
% \item<11->   \Circled[fill color= white]{\,$\text{to} + \text{動詞の原形}$\,} が「~すること」という意味を表し、名詞のはたらきをすることがある
 \item<11->   名詞的用法のto不定詞が一般動詞の目的語になることがある
% \item<12-> 主語と組み合わせる動詞\,(\,I like \ldots{}, He likes \ldots{}, She liked \ldots\,)とはちがって、\\
%いつでも\,\Circled[fill color= white]{\,$\text{to} + \text{動詞の原形}$\,}\\
 \end{itemize}
     \end{block}
}
\end{frame}
%%%%%%%%%%%%%%%%%%%%%%
\begin{frame}[plain]{Exercises}

{\small つぎの英文を読みましょう}%
\hfill{\tiny 0536}\,{\scriptsize \myaudio{./audio/032_infinitive_noun_03.mp3}}

Every summer, Jane and her friends planned\footnote{plan: 計画する} fun activities\footnote{fun activities: 楽しい活動}.
Jane {\bfseries wanted to visit} many countries.
Tom {\bfseries liked to draw} pictures of the places they visited\footnote{the places they visited: 訪れた場所}.
Mary {\bfseries began to pack}\footnote{pack: つめる} her bags early.
John {\bfseries started to read} travel books\footnote{travel books: 旅行の本}.
They were so excited\footnote{excited: わくわくして}.
They {\bfseries decided to save}\footnote{save: 貯蓄する} money for their trip.
During their trip, they {\bfseries needed to try} new foods.
After the trip, they { \bfseries continued}\footnote{continue: 続ける} {\bfseries to talk} about their memories.
\end{frame}
\end{document}
