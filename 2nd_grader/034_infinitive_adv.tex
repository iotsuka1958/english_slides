\documentclass[aspectratio=169,xcolor={dvipsnames,table}]{beamer}
\usepackage[no-math,deluxe,haranoaji]{luatexja-preset}
\renewcommand{\kanjifamilydefault}{\gtdefault}
\renewcommand{\emph}[1]{{\upshape\bfseries #1}}
\usetheme{metropolis}
\metroset{block=fill}
\setbeamertemplate{navigation symbols}{}
\setbeamertemplate{blocks}[rounded][shadow=false]
\usecolortheme[rgb={0.7,0.2,0.2}]{structure}
%%%%%%%%%%%%%%%%%%%%%%%%%%
%% Change alert block colors
%%% 1- Block title (background and text)
\setbeamercolor{block title alerted}{fg=mDarkTeal, bg=mLightBrown!45!yellow!45}

\setbeamercolor{block title example}{fg=magenta!10!black, bg=mLightGreen!70}
%%% 2- Block body (background)
\setbeamercolor{block body alerted}{bg=mLightBrown!25}
\setbeamercolor{block body example}{bg=mLightGreen!15}
%%%%%%%%%%%%%%%%%%%%%%%%%%%
%%%%%%%%%%%%%%%%%%%%%%%%%%%
%% さまざまなアイコン
%%%%%%%%%%%%%%%%%%%%%%%%%%%
%\usepackage{fontawesome}
\usepackage{fontawesome5}
\usepackage{figchild}
\usepackage{twemojis}
\usepackage{utfsym}
\usepackage{bclogo}
\usepackage{marvosym}
\usepackage{fontmfizz}
\usepackage{pifont}
\usepackage{phaistos}
\usepackage{worldflags}
\usepackage{jigsaw}
\usepackage{tikzlings}
\usepackage{tikzducks}
\usepackage{scsnowman}
\usepackage{epsdice}
\usepackage{halloweenmath}
\usepackage{svrsymbols}
\usepackage{countriesofeurope}
\usepackage{tipa}
\usepackage{manfnt}
%%%%%%%%%%%%%%%%%%%%%%%%%%%
\usepackage{tikz}
\usetikzlibrary{calc,patterns,decorations.pathmorphing,backgrounds}
\usepackage{tcolorbox}
\usepackage{tikzpeople}
\usepackage{circledsteps}
\usepackage{xcolor}
\usepackage{amsmath}
\usepackage{booktabs}
\usepackage{chronology}
\usepackage{signchart}
%%%%%%%%%%%%%%%%%%%%%%%%%%%
%% 場合分け
%%%%%%%%%%%%%%%%%%%%%%%%%%%
\usepackage{cases}
%%%%%%%%%%%%%%%%%%%%%%%%%%
\usepackage{pdfpages}
%%%%%%%%%%%%%%%%%%%%%%%%%%%
%% 音声リンク表示
\newcommand{\myaudio}[1]{\href{#1}{\faVolumeUp}}
%%%%%%%%%%%%%%%%%%%%%%%%%%
%% \myAnch{<名前>}{<色>}{<テキスト>}
%% 指定のテキストを指定の色の四角枠で囲み, 指定の名前をもつTikZの
%% ノードとして出力する. 図には remember picture 属性を付けている
%% ので外部から参照可能である.
\newcommand*{\myAnch}[3]{%
  \tikz[remember picture,baseline=(#1.base)]
    \node[draw,rectangle,line width=1pt,#2] (#1) {\normalcolor #3};
}
%%%%%%%%%%%%%%%%%%%%%%%%%%
%% \myEmph コマンドの定義
%%%%%%%%%%%%%%%%%%%%%%%%%%
%\newcommand{\myEmph}[3]{%
%    \textbf<#1>{\color<#1>{#2}{#3}}%
%}
\usepackage{xparse} % xparseパッケージの読み込み
\NewDocumentCommand{\myEmph}{O{} m m}{%
    \def\argOne{#1}%
    \ifx\argOne\empty
        \textbf{\color{#2}{#3}}% オプション引数が省略された場合
    \else
        \textbf<#1>{\color<#1>{#2}{#3}}% オプション引数が指定された場合
    \fi
}
%%%%%%%%%%%%%%%%%%%%%%%%%%%
%%%%%%%%%%%%%%%%%%%%%%%%%%%
%% 文末の上昇イントネーション記号 \myRisingPitch
%% 通常のイントネーション \myDownwardPitch
%% https://note.com/dan_oyama/n/n8be58e8797b2
%%%%%%%%%%%%%%%%%%%%%%%%%%%
\newcommand{\myRisingPitch}{
\begin{tikzpicture}[scale=0.3,baseline=0.3]
\draw[->,>=stealth] (0,0) to[bend right=45] (1,1);
\end{tikzpicture}
}
\newcommand{\myDownwardPitch}{
\begin{tikzpicture}[scale=0.3,baseline=0.3]
\draw[->,>=stealth] (0,1) to[bend left=45] (1,0);
\end{tikzpicture}
}
%%%%%%%%%%%%%%%%%%%%%%%%%%%%
%\AtBeginSection[%
%]{%
%  \begin{frame}[plain]\frametitle{授業の流れ}
%     \tableofcontents[currentsection]
%   \end{frame}%
%}

%%%%%%%%%%%%%%%%%%%%%%%%%%%
\title{English is fun.}
\subtitle{She studied hard to pass the exam.}
\author{}
\institute[]{}
\date[]

%%%%%%%%%%%%%%%%%%%%%%%%%%%%
%% TEXT
%%%%%%%%%%%%%%%%%%%%%%%%%%%%
\begin{document}

\begin{frame}[plain]
  \titlepage
\end{frame}

\section*{授業の流れ}
\begin{frame}[plain]
  \frametitle{授業の流れ}
  \tableofcontents
\end{frame}

\section{副詞的用法}
\subsection{目的}
%%%%%%%%%%%%%%%%%%%%%%%%%%%%%%%%%%%%%%%%%%%%%
\begin{frame}[plain]{~するために}
 \large


\begin{enumerate}
 \item<1-> You must work hard.\hfill{\scriptsize work \textipa{/w\'\textrhookschwa :k/} 働く、勉強する}
 \item<2-> You must \myAnch{a}{white}{work}hard \,\,\,\myAnch{b}{Maroon}{to pass the exam}.\hfill{\scriptsize pass \textipa{/p\'\ae s/} (試験に)受かる}

\end{enumerate}

\begin{tikzpicture}[remember picture, overlay]
% Calculate intermediate points
  \coordinate (a1) at ($(a) + (0,-25pt)$); % 25pt below a
  \coordinate (b1) at ($(b) + (0,-25pt)$); % 25pt below b
  % Draw the arrow with right angles
  \visible<3->{\draw[<-,Maroon,line width=2pt, opacity=.5] (a) -- (a1) -- (b1) -- (b);}
\end{tikzpicture}

%
\hfill{\tiny 0113}\,{\scriptsize \myaudio{./audio/034_infinitive_adv_01.mp3}}
\visible<4->{%
\begin{block}{Topic for Today ---副詞的用法---}
\begin{itemize}\setbeamertemplate{items}[square]\small
 \item \Circled[fill color= yellow!50]{\,$\text{to} + \text{動詞の原形}$\,} が、
「〜するために」の意味で副詞としてはたらくことがある
 \end{itemize}
     \end{block}
}
\end{frame}
%%%%%%%%%%%%%%%%%%%%%%%%%%%%%%%%%%%%%%%%%%%%%
\begin{frame}[plain]{Exercises}

{\small つぎの文の意味を考えましょう}%

 \begin{enumerate}
  \item I got up early to catch the first train.\hfill{\scriptsize catch \textipa{/k\'\ae\textteshlig/} 間に合う}
  \item She went to the station to meet her parents.\hfill{\scriptsize meet \textipa{/m\'\i:t/} 出迎える}
  \item To save money, we eat at home.\hfill{\scriptsize save \textipa{/s\'eIv/ 節約する}}
  \item To help his parents, he cleans the house every day.\hfill{\scriptsize clean \textipa{/kl\'\i:n/ 掃除する}}
 \end{enumerate}

\visible<2->{%
\begin{block}{Topic for Today ---副詞的用法---}
\begin{itemize}\setbeamertemplate{items}[square]\small
 \item 目的をあらわすto不定詞は、文の最後にも先頭にもくる
       \begin{itemize}\setbeamertemplate{items}[circle]\small
\setlength{\itemsep}{5pt} % 項目間
	\item $\text{
S} + \text{V}\,\,\,\,$\Circled[fill color= yellow!40]{\,$\text{to} + \text{動詞の原形}$\,}.%
\hfill{\scriptsize The door opened \textbf{suddenly}.}
	\item \Circled[fill color= yellow!30]{\,$\text{To} + \text{動詞の原形}$\,},\,\, $\text{S} + \text{V}$.%
\hfill{\scriptsize \textbf{Suddenly}, the door opened.}
       \end{itemize}
 \end{itemize}
     \end{block}
}
 
\hfill{\tiny 0218}\,{\scriptsize \myaudio{./audio/034_infinitive_adv_02.mp3}}

\end{frame}
%%%%%%%%%%%%%%%%%%%%%%%%%%%%%%%%%%%
\begin{frame}[plain]{Exercises}

{\small 日本語の意味になるよう(~~~~~~)内の語句を並べ替えましょう。先頭の語は大文字で始めてください}%
\hfill{\tiny 0212}\,{\scriptsize \myaudio{./audio/034_infinitive_adv_03.mp3}}
 \begin{enumerate}
  \item {\small 彼らは野球をするために公園へ行った。}\\
	They ( play / to / to / the park / went / baseball ) .\\
	\visible<2->{They went to the park to play baseball.}
  \item {\small 彼はフランスへ行くためにフランス語を勉強した。}\\
	He ( to / to / French / studied / go / France ) .\\
	\visible<3->{He studied French to go to France.}
  \item {\small 彼女は新しい自転車を買うためにお金を節約した。}\\
	To ( saved / bike / money / new / she / buy / a ) .\\
	\visible<4->{To buy a new bike, she saved money.}
  \item {\small そのバスに間に合うようにわたしたちは走った。}
	\hfill{}( bus / catch / to / the ), we ran.\\
	\visible<5->{To catch the bus, we ran.}
 \end{enumerate}
\end{frame}
%%%%%%%%%%%%%%%%%%%%%%%%%%%%%%%%%%%%%
\begin{frame}[plain]{to不定詞で文がはじまったとき}\large
 {\small つぎの2文の意味を考えましょう\textdbend}

\begin{enumerate}
 \item \alt<2->{\fbox{To study English}}{To study English}\,\,\,is very important.
\hfill{}%
\visible<4->{\Circled[fill color= yellow!30]{\,$\text{To} + \text{動詞の原形}$\,}\, ($=$S)\,\, $+$ V}

 \item \alt<2->{\fbox{To study English}}{To study English}\,,\,\,\,he went to Australia.\hfill%
\visible<5->{\Circled[fill color= yellow!30]{\,$\text{To} + \text{動詞の原形}$\,},\,\,\,\, $\text{S} + \text {V}$}\hspace{13pt}\mbox{}\\
\hfill{\scriptsize Australia \textipa{/O:str\'eIlj@/} オーストラリア}
\end{enumerate}
%
\hfill{\tiny 0120}\,{\scriptsize \myaudio{./audio/034_infinitive_adv_04.mp3}}


\bigskip 

\hfill\begin{minipage}{.4\textwidth}
\begin{itemize}\small
 \item<3->[hint 1] 主語と述語動詞を確認しよう
 \item<3->[hint 2] to不定詞は何的用法か考えよう
\end{itemize}
\end{minipage}
\end{frame}
%%%%%%%%%%%%%%%%%%%%%%%%%%%%%%%%%
\end{document}

