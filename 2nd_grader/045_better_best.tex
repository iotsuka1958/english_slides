\documentclass[aspectratio=169,xcolor={dvipsnames,table}]{beamer}
\usepackage[no-math,deluxe,haranoaji]{luatexja-preset}
\renewcommand{\kanjifamilydefault}{\gtdefault}
\renewcommand{\emph}[1]{{\upshape\bfseries #1}}
\usetheme{metropolis}
\metroset{block=fill}
\setbeamertemplate{navigation symbols}{}
\setbeamertemplate{blocks}[rounded][shadow=false]
\usecolortheme[rgb={0.7,0.2,0.2}]{structure}
%%%%%%%%%%%%%%%%%%%%%%%%%%
%% Change alert block colors
%%% 1- Block title (background and text)
\setbeamercolor{block title alerted}{fg=mDarkTeal, bg=mLightBrown!45!yellow!45}
\setbeamercolor{block title example}{fg=magenta!10!black, bg=mLightGreen!70}
%%% 2- Block body (background)
\setbeamercolor{block body alerted}{bg=mLightBrown!25}
\setbeamercolor{block body example}{bg=mLightGreen!15}
%%%%%%%%%%%%%%%%%%%%%%%%%%%
\usepackage[absolute,overlay]{textpos}
%\usepackage[grid=true,gridcolor=Maroon,subgridcolor=gray,gridunit=pt,texcoord]{eso-pic} %場所決めのためのgrid表示
%%%%%%%%%%%%%%%%%%%%%%%%%%%
%% さまざまなアイコン
%%%%%%%%%%%%%%%%%%%%%%%%%%%
%\usepackage{fontawesome}
\usepackage{fontawesome5}
\usepackage{figchild}
\usepackage{twemojis}
\usepackage{utfsym}
\usepackage{bclogo}
\usepackage{marvosym}
\usepackage{fontmfizz}
\usepackage{pifont}
\usepackage{phaistos}
\usepackage{worldflags}
\usepackage{jigsaw}
\usepackage{tikzlings}
\usepackage{tikzducks}
\usepackage{scsnowman}
\usepackage{epsdice}
\usepackage{halloweenmath}
\usepackage{svrsymbols}
\usepackage{countriesofeurope}
\usepackage{tipa}
\usepackage{manfnt}
%%%%%%%%%%%%%%%%%%%%%%%%%%%
\usepackage{tikz}
\usetikzlibrary{calc,patterns,decorations.pathmorphing,backgrounds}
\usepackage{tcolorbox}
\usepackage{tikzpeople}
\usepackage{circledsteps}
\usepackage{xcolor}
\usepackage{amsmath}
\usepackage{booktabs}
\usepackage{chronology}
\usepackage{signchart}
%%%%%%%%%%%%%%%%%%%%%%%%%%%
%% 場合分け
%%%%%%%%%%%%%%%%%%%%%%%%%%%
\usepackage{cases}
%%%%%%%%%%%%%%%%%%%%%%%%%%
\usepackage{pdfpages}
%%%%%%%%%%%%%%%%%%%%%%%%%%%
%% 音声リンク表示
\newcommand{\myaudio}[1]{\href{#1}{\faVolumeUp}}
%%%%%%%%%%%%%%%%%%%%%%%%%%
%% \myAnch{<名前>}{<色>}{<テキスト>}
%% 指定のテキストを指定の色の四角枠で囲み, 指定の名前をもつTikZの
%% ノードとして出力する. 図には remember picture 属性を付けている
%% ので外部から参照可能である.
\newcommand*{\myAnch}[3]{%
  \tikz[remember picture,baseline=(#1.base)]
    \node[draw,rectangle,line width=1pt,#2] (#1) {\normalcolor #3};
}
%%%%%%%%%%%%%%%%%%%%%%%%%%
%% \myEmph コマンドの定義
%%%%%%%%%%%%%%%%%%%%%%%%%%
%\newcommand{\myEmph}[3]{%
%    \textbf<#1>{\color<#1>{#2}{#3}}%
%}
\usepackage{xparse} % xparseパッケージの読み込み
\NewDocumentCommand{\myEmph}{O{} m m}{%
    \def\argOne{#1}%
    \ifx\argOne\empty
        \textbf{\color{#2}{#3}}% オプション引数が省略された場合
    \else
        \textbf<#1>{\color<#1>{#2}{#3}}% オプション引数が指定された場合
    \fi
}
%%%%%%%%%%%%%%%%%%%%%%%%%%%
%%%%%%%%%%%%%%%%%%%%%%%%%%%
%% 文末の上昇イントネーション記号 \myRisingPitch
%% 通常のイントネーション \myDownwardPitch
%% https://note.com/dan_oyama/n/n8be58e8797b2
%%%%%%%%%%%%%%%%%%%%%%%%%%%
\newcommand{\myRisingPitch}{
\begin{tikzpicture}[scale=0.3,baseline=0.3]
\draw[->,>=stealth] (0,0) to[bend right=45] (1,1);
\end{tikzpicture}
}
\newcommand{\myDownwardPitch}{
\begin{tikzpicture}[scale=0.3,baseline=0.3]
\draw[->,>=stealth] (0,1) to[bend left=45] (1,0);
\end{tikzpicture}
}
%%%%%%%%%%%%%%%%%%%%%%%%%%%%
%\AtBeginSection[%
%]{%
%  \begin{frame}[plain]\frametitle{授業の流れ}
%     \tableofcontents[currentsection]
%   \end{frame}%
%}

%%%%%%%%%%%%%%%%%%%%%%%%%%%
\title{English is fun.}
\subtitle{Billy is the best pianist.}
\author{}
\institute[]{}
\date[]

%%%%%%%%%%%%%%%%%%%%%%%%%%%%
%% TEXT
%%%%%%%%%%%%%%%%%%%%%%%%%%%%
\begin{document}

\begin{frame}[plain]
  \titlepage
\end{frame}

%%%%%%%%%%%%%%%%%%%%%%%%%%%%%%%%
\section*{授業の流れ}
\begin{frame}[plain]
  \frametitle{授業の流れ}
  \tableofcontents
\end{frame}
%%%%%%%%%%%%%%%%%%%%%%%%%%%%%%%%
\section{不規則変化する比較級・最上級}
\subsection{wellとgood}
%%%%%%%%%%%%%%%%%%%%%%%%%%%%%%%%%%%%%%%%%%%%%
\begin{frame}[plain]{wellとgood}
 \begin{enumerate}
  \item \begin{enumerate}
	 \item<1-> Billy plays the piano \myEmph[1-]{Maroon}{well}.
%	 \item<2-> Billy plays the piano \myEmph[2-]{Maroon}{as} well \myEmph[2-]{Maroon}{as} John.
	 \item<2-> Billy plays the piano \myEmph[2-]{Maroon}{better} than John.
	 \item<3-> Billy plays the piano \myEmph[3-]{Maroon}{the best} in his school.%
\hfill{\tiny 0146}\,{\scriptsize \myaudio{./audio/045_better_best_01.mp3}}
	\end{enumerate}
  \item \begin{enumerate}
	 \item<4-> Janis is a \myEmph[4-]{Maroon}{good} singer.
	 \item<5-> Janis is a \myEmph[5-]{Maroon}{better} singer than Linda.
	 \item<6-> Janis is \myEmph[6-]{Maroon}{the best} singer of all.
\hfill[\tiny 0141]\,{\scriptsize \myaudio{./audio/045_better_best_02.mp3}}
	\end{enumerate}
 \end{enumerate}

\begin{block}<7->{Topics for Today}\small
副詞{\bfseries well}(じょうずに)と形容詞{\bfseries good}(よい)は不規則な変化をします%
\hfill{\tiny 0148}\,{\scriptsize \myaudio{./audio/045_better_best_03.mp3}}

\begin{itemize}\setbeamertemplate{items}[square]\small
 \item {\bfseries well} \textipa{/w\'el/} --- {\bfseries better} \textipa{/b\'et\textrhookschwa /} --- {\bfseries best} \textipa{/b\'est/}
 \item {\bfseries good} \textipa{/g\'Ud/} --- {\bfseries better} \textipa{/b\'et\textrhookschwa /}--- {\bfseries best} \textipa{/b\'est/}
 \end{itemize}
     \end{block}
\end{frame}
%%%%%%%%%%%%%
\begin{frame}[plain]{Exercises}

{\small 日本語の意味になるよう空所に適語を補ってください}%
\hfill{\tiny 0207}\,{\scriptsize \myaudio{./audio/045_better_best_04.mp3}}
\begin{enumerate}
 \item ジャニスはリンダよりも歌うのがうまい。\\
       Janis sings (~~\alt<2->{\textcolor{Maroon}{\bfseries better}}{\phantom{better}}~~) than Linda.\\
       $=$\,\,Janis is a  (~~\alt<3->{\textcolor{Maroon}{\bfseries better}}{\phantom{better}}~~)  (~~\alt<3->{\textcolor{Maroon}{\bfseries singer}}{\phantom{singer}}~~) than Linda.
 \item ダイアナはみんなのなかでもっともピアノがうまい。\\
       Diana plays the piano the (~~\alt<4->{\textcolor{Maroon}{\bfseries best}}{\phantom{best}}~~) of all.\\
       $=$\,\,Diana is (~~\alt<5->{\textcolor{Maroon}{\bfseries the}}{\phantom{the}}~~)  (~~\alt<5->{\textcolor{Maroon}{\bfseries best}}{\phantom{best}}~~) pianist of all.
\end{enumerate}
\end{frame}
%%%%%%%%%%%%%%%%%%%%%%%%%%%%%%%%%%%%%%%%%
\subsection{manyとmuch}
%%%%%%%%%%%%%%%%%%%%%%%%%%%%%%%%%%%%%%%%
\begin{frame}[plain]{manyとmuch}
 \begin{enumerate}
  \item \begin{enumerate}
	 \item<1-> Bob ate too \myEmph[1-]{Maroon}{many} cookies.\hfill{\scriptsize cookie:数えられる名詞}
	 \item<2-> Bob ate \myEmph[2-]{Maroon}{more} cookies than John.
	 \item<3-> Bob ate \myEmph[3-]{Maroon}{the most} cookies in his family.\hfill{\scriptsize \myaudio{./audio/045_better_best_05.mp3}}

	\end{enumerate}
  \item \begin{enumerate}
	 \item<4-> Janis drank too \myEmph[4-]{Maroon}{much} tea.\hfill{\scriptsize tea: 数えられない名詞}
	 \item<5-> Janis drank \myEmph[5-]{Maroon}{more} tea than Linda.
	 \item<6-> Janis drank \myEmph[6-]{Maroon}{the most} tea of all.%
\hfill{\scriptsize \myaudio{./audio/045_better_best_06.mp3}}


	\end{enumerate}
 \end{enumerate}
%
\begin{block}<7->{Topics for Today}\small
形容詞{\bfseries many}(多数の)と{\bfseries much}(多量の)は不規則な変化をします%
\hfill{\tiny 0149}\,{\scriptsize \myaudio{./audio/045_better_best_07.mp3}}

\begin{itemize}\setbeamertemplate{items}[square]\small
 \item {\bfseries many} \textipa{/m\'eni/} --- {\bfseries more}  \textipa{/m\'O\textrhookschwa /} --- {\bfseries most}  \textipa{/m\'oUst/} 
 \item {\bfseries much} \textipa{/m\'\textturnv tS/}  --- {\bfseries more}  \textipa{/m\'O\textrhookschwa /} --- {\bfseries most}  \textipa{/m\'oUst/} 
 \end{itemize}
     \end{block}
\end{frame}
%%%%%%%%%%%%%%%%%%%%%%%
\begin{frame}[plain]{Exercises}

{\small 1.1と1.2は正しいほうを選びましょう。2.1~は空所に適語を補ってください}%
\hfill{\tiny 0349}\,{\scriptsize \myaudio{./audio/045_better_best_08.mp3}}
 \begin{enumerate}
  \item \begin{enumerate}
	 \item<1-> Paul ate too ( \alt<2->{\Circled[outer color = Maroon]{many}}{ many } / much ) hotdogs.
	 \item<1-> Paul drank too ( many / \alt<3->{\Circled[outer color = Maroon]{much}}{ much }) water.
	\end{enumerate}
  \item \begin{enumerate}
	 \item<1-> ポールはジョンよりたくさんのホットドッグを食べた。\\
	       Paul ate (~~\alt<4->{\myEmph[4-]{Maroon}{more}}{\phantom{many}}~~) hotdogs (~~\alt{\myEmph[4-]{Maroon}{than}}{\phantom{than}}~~) John.
	 \item<1-> ポールはジョンよりたくさんのコーヒーを飲んだ。\\
	       Paul drank (~~\alt<5->{\myEmph[5-]{Maroon}{more}}{\phantom{many}}~~) coffee (~~\alt<5->{\myEmph[5-]{Maroon}{than}}{\phantom{than}}~~) John.
	\end{enumerate}
  \item \begin{enumerate}
	 \item<1-> ポールがもっともおおくのホットドッグを食べた。\\
	       Paul ate (~~\alt<6->{\myEmph[6-]{Maroon}{the}}{\phantom{the}}~~) (~~\alt<6->{\myEmph[6-]{Maroon}{most}}{\phantom{most}}~~) hotdogs.
	 \item<1-> ポールがいちばん大量のコーヒーを飲んだ。\\
	       Paul drank (~~\alt<7->{\myEmph[7-]{Maroon}{the}}{\phantom{the}}~~) (~~\alt<7->{\myEmph[7-]{Maroon}{most}}{\phantom{most}}~~) coffee.
	\end{enumerate}
  \item \begin{enumerate}
	 \item<1-> ポールはジョン同数のホットドッグを食べた。\\
	       Paul ate (~~\alt<8->{\myEmph[8-]{Maroon}{as}}{\phantom{as}}~~) (~~\alt<8->{\myEmph[8-]{Maroon}{many}}{\phantom{many}}~~) hotdogs as John.
	 \item<1-> ポールはジョンと同量のコーヒーを飲んだ。\\
	       Paul drank (~~\alt<9->{\myEmph[9-]{Maroon}{as}}{\phantom{as}}~~) (~~\alt<9->{\myEmph[9-]{Maroon}{much}}{\phantom{much}}~~) coffee as John.
	\end{enumerate}
 \end{enumerate}
\end{frame}
%%%%%%%%%%%%%%%%%%%%%%%%%%%%%%%%%%%%%%%%%%%%%
\begin{frame}[plain]{不規則な比較級・最上級}
\centering
  \begin{tblr}{colspec={lll},
% 表の最上と最下に太さ 0.08em の横罫線
hline{1,Z} = { 0.08em },
hline{2} = { 0.05em },
row{odd}={gray9},
row{1} = { halign = c, font = { \sffamily\bfseries }, bg = gray6, fg = white }
}
原級&比較級&最上級\\
good&\visible<2->{better}&\visible<3->{best}\\
well&\visible<4->{better}&\visible<5->{best}\\
many&\visible<6->{more}&\visible<7->{most}\\
much&\visible<8->{more}&\visible<9->{most}\\
   \end{tblr}

%
\hfill{\tiny 0313}\,{\scriptsize \myaudio{./audio/045_better_best_09.mp3}}
\end{frame}
%%%%%%%%%%%%%%%%%%%%%%%%%%%%%%%%%%%%%%
\section{まとめ}
\begin{frame}[plain]{まとめ}
 \begin{block}{well, good}\small
副詞{\bfseries well}(じょうずに)と形容詞{\bfseries good}(よい)は不規則な変化をします%
\hfill{\tiny 0148}\,{\scriptsize \myaudio{./audio/045_better_best_03.mp3}}

\begin{itemize}\setbeamertemplate{items}[square]\small
 \item {\bfseries well} \textipa{/w\'el/} --- {\bfseries better} \textipa{/b\'et\textrhookschwa /} --- {\bfseries best} \textipa{/b\'est/}
 \item {\bfseries good} \textipa{/g\'Ud/} --- {\bfseries better} \textipa{/b\'et\textrhookschwa /}--- {\bfseries best} \textipa{/b\'est/}
 \end{itemize}
     \end{block}

\begin{block}{many, much}\small
形容詞{\bfseries many}(多数の)と{\bfseries much}(多量の)は不規則な変化をします%
\hfill{\tiny 0149}\,{\scriptsize \myaudio{./audio/045_better_best_07.mp3}}

\begin{itemize}\setbeamertemplate{items}[square]\small
 \item {\bfseries many} \textipa{/m\'eni/} --- {\bfseries more}  \textipa{/m\'O\textrhookschwa /} --- {\bfseries most}  \textipa{/m\'oUst/} 
 \item {\bfseries much} \textipa{/m\'\textturnv tS/}  --- {\bfseries more}  \textipa{/m\'O\textrhookschwa /} --- {\bfseries most}  \textipa{/m\'oUst/} 
 \end{itemize}
     \end{block}
\end{frame}
%%%%%%%%%%%%%%%%%%%%%%%%%%%%%%%%%%%%%
\end{document}

