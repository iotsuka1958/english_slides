\documentclass[aspectratio=169,xcolor={dvipsnames,table}]{beamer}
\usepackage[no-math,deluxe,haranoaji]{luatexja-preset}
\renewcommand{\kanjifamilydefault}{\gtdefault}
\renewcommand{\emph}[1]{{\upshape\bfseries #1}}
\usetheme{metropolis}
\metroset{block=fill}
\setbeamertemplate{navigation symbols}{}
\usecolortheme[rgb={0.7,0.2,0.2}]{structure}
%%%%%%%%%%%%%%%%%%%%%%%%%%%
\usepackage{media9}
%%%%%%%%%%%%%%%%%%%%%%%%%%%
%% さまざまなアイコン
%%%%%%%%%%%%%%%%%%%%%%%%%%%
\usepackage{fontawesome}
\usepackage{figchild}
\usepackage{twemojis}
\usepackage{utfsym}
\usepackage{bclogo}
\usepackage{marvosym}
\usepackage{fontmfizz}
\usepackage{pifont}
\usepackage{phaistos}
\usepackage{worldflags}
%%%%%%%%%%%%%%%%%%%%%%%%%%%
\usepackage{tikz}
\usetikzlibrary{backgrounds}
\usepackage{tcolorbox}
\usepackage{circledsteps}
\usepackage{xcolor}
\usepackage{amsmath}
%%%%%%%%%%%%%%%%%%%%%%%%%%%
%% 場合分け
\usepackage{cases}
%%%%%%%%%%%%%%%%%%%%%%%%%%%
% \myAnch{<名前>}{<色>}{<テキスト>}
% 指定のテキストを指定の色の四角枠で囲み, 指定の名前をもつTikZの
% ノードとして出力する. 図には remeber picture 属性を付けている
% ので外部から参照可能である.
\newcommand*{\myAnch}[3]{%
  \tikz[remember picture,baseline=(#1.base)]
    \node[draw,rectangle,#2] (#1) {\normalcolor #3};
}
%%%%%%%%%%%%%%%%%%%%%%%%%%%%
%% 音声リンク表示
\newcommand{\myaudio}[1]{\href{#1}{\faVolumeUp}}
%%%%%%%%%%%%%%%%%%%%%%%%%%%
% \myEmph コマンドの定義
%\newcommand{\myEmph}[3]{%
%    \textbf<#1>{\color<#1>{#2}{#3}}%
%}
\usepackage{xparse} % xparseパッケージの読み込み
\NewDocumentCommand{\myEmph}{O{} m m}{%
    \def\argOne{#1}%
    \ifx\argOne\empty
        \textbf{\color{#2}{#3}}% オプション引数が省略された場合
    \else
        \textbf<#1>{\color<#1>{#2}{#3}}% オプション引数が指定された場合
    \fi
}
%%%%%%%%%%%%%%%%%%%%%%%%%%%
%% 文末の上昇イントネーション記号 \myRisingPitch
%% 通常のイントネーション \myDownwardPitch
%% https://note.com/dan_oyama/n/n8be58e8797b2
%%%%%%%%%%%%%%%%%%%%%%%%%%%
\newcommand{\myRisingPitch}{
\begin{tikzpicture}[scale=0.3,baseline=0.3]
\draw[->,>=stealth] (0,0) to[bend right=45] (1,1);
\end{tikzpicture}
}
\newcommand{\myDownwardPitch}{
\begin{tikzpicture}[scale=0.3,baseline=0.3]
\draw[->,>=stealth] (0,1) to[bend left=45] (1,0);
\end{tikzpicture}
}
%%%%%%%%%%%%%%%%%%%%%%%%%%%
\title{English is fun.\,\,{}--- It will be rainy tomorrow. ---}
\author{}
\institute[]{}
\date[]

%%%%%%%%%%%%%%%%%%%%%%%%%%%%
%% TEXT
%%%%%%%%%%%%%%%%%%%%%%%%%%%%
\begin{document}


\begin{frame}[plain]
  \titlepage
\end{frame}


\section*{授業の流れ}
\begin{frame}[plain]
  \frametitle{授業の流れ}
  \tableofcontents
\end{frame}



\section{will}

\subsection{未来を表すwill}
\begin{frame}[plain]{未来を表すwill}
\Large
\begin{enumerate}
 \item<1-> I played tennis yesterday.\hfill{}{\small 過去のこと}
 \item<2-> I play tennis tomorrow.\hfill{}{\small 現在のこと}
 \item<3-> I \alt<1-3>{\myAnch{aux1}{white}{am going to}}{\myAnch{AUX1}{Orange}{am going to}} play tennis tomorrow.\hfill{\small 未来のこと}%
 \item<4-> I  \alt<1-3>{\myAnch{aux1}{white}{will}}{\myAnch{AUX1}{Orange}{will}} play tennis tomorrow.

\end{enumerate}

\begin{exampleblock}<5->{Topic for Today}
\small
$\text{will}+\text{原形}$も未来のことを表します。
\end{exampleblock}
\end{frame}


\begin{frame}[plain]\frametitle{Exercises}

つぎの英文の意味を考えましょう。

\begin{enumerate}
 \item I will play baseball tomorrow.
 \item We will have a party on Saturday.
 \item You will be a good singer.
 \item He will drive to the airport.
 \item She will call him later.
 \item They will clean the room.
\end{enumerate}

\pause

\vfill

\begin{exampleblock}{Topics for Today}
\pause
\begin{itemize}\small
 \item   主語がなんであってもwill $+$ 動詞($=\text{そのままの形}$)
 \end{itemize}
     \end{exampleblock}
\end{frame}


\begin{frame}[plain]{Exercises}

次の英文の(~~~~~~~~)内から動詞の正しい形を選び、○で囲みましょう。 

\begin{enumerate}
 \item I will ( \alt<2->{\Circled[outer color=orange]{~~study~~}}{~~study~~}  /~~studies~~/~~studied~~) English at home.
 \item He will ( \alt<3->{\Circled[outer color=orange]{~~play~~}}{~~play~~} /~~plays~~/~~is playing~~) baseball tomorrow.
 \item They always (  \alt<4->{\Circled[outer color=orange]{~~read~~}}{~~read~~} /~~reads~~/~~reading~~) books in the library after school.
 \item She will (~~is~~/~~are~~/ \alt<5->{\Circled[outer color=orange]{~~be~~}}{~~be~~} ) free this month.
 \item He will (~~am~~/~~is~~/ \alt<6->{\Circled[outer color=orange]{~~be~~}}{~~be~~} ) fourteen years old next month.
\end{enumerate}

\end{frame}

\subsection{will not}
\begin{frame}[plain]{willの否定文}
\Large

I will \myEmph[5-]{OliveGreen}{go} to the party tomorrow.

\pause

I \textcolor{Orange}{\bfseries will not}  \myEmph[5-]{OliveGreen}{go} to the party tomorrow.

\pause

I \textcolor{Orange}{\bfseries won't}  \myEmph[5-]{OliveGreen}{go} to the party tomorrow.

\pause

\begin{exampleblock}{Topic for Today}

\begin{itemize}\small
 \item   willの否定$\longrightarrow${\,\,\,}$\left\{\begin{tabular}{l}\text{will not}\\\text{won't}\end{tabular}\right\} + \text{動詞(=\text{そのままの形})}$
 \end{itemize}
     \end{exampleblock}
\end{frame}

\begin{frame}[plain]{Exercises}
 
\begin{enumerate}
 \item わたしは今夜コーヒーを飲みません。\\
 I ~~( coffee / drink / will / not )~~tonight.\\
\visible<2->{I will not drink coffee tonight.}
 \item 彼女は今夜、テレビを見ません。\\
She~~( watch / will / not / TV )~~tonight.\\
\visible<3->{She will not watch TV tonight.}
 \item 彼らは新しい車を買わないでしょう。\\
 They~~( buy / will / not / a new car ).\\
\visible<4->{They will not buy a new car.}
 \item 明日は雨が降らないでしょう。\\
It~~( be / rainy / not / will )~~tomorrow.\\
\visible<5->{It will not be rainy tomorrow.}
\end{enumerate}
\end{frame}

\begin{frame}[plain]{willの疑問文}

\myAnch{s-1}{orange}{He} \myAnch{aux-1}{olive}{will} play tennis this weekend. \scalebox{2}{\myDownwardPitch}
\vspace{30pt}\pause

\myAnch{aux-2}{olive}{Will} \myAnch{s-2}{orange}{he} play tennis this weekend\myAnch{question}{orange}{?}
\pause
\begin{tikzpicture}[remember picture, overlay]
 \draw[thick, orange, ->] (s-1.south) to[out=-90, in=90] (s-2.north); 
 \draw[thick, olive, ->] (aux-1.south) to[out=-90, in=90] (aux-2.north);
%\pause
%\visible<8->{\node at (0.5,0.25) {\scalebox{2}{\myRisingPitch}};}
\end{tikzpicture}

\vspace{-25pt}
\mbox{}\hspace{200pt}\visible<6->{\scalebox{2}{\myRisingPitch}}

\hspace{40pt}\visible<5->{\small 主語の前にWill}
\hspace{80pt}\visible<6-> {\small ?と最後のイントネーションに注意}

\vfill

\begin{exampleblock}<7->{Topic for Today}
\visible<8->{%
\begin{itemize}\small
 \item   willの疑問文」$\longrightarrow${\,\,\,}$\text{Will} + \text{主語} + \text{動詞(}=\text{そのままの形)}$ \ldots{}\,\,?
 \end{itemize}
}
     \end{exampleblock}
\end{frame}

\begin{frame}[plain]{Exercises}

\begin{enumerate}
 \item 今夜彼らは映画を見るつもりですか。\\
 ( a movie / they / will / watch ) tonight?  \\
\visible<2->{Will they watch a movie tonight?}
 \item 彼女は後で友達に電話するつもりですか。\\
 ( call / will / she / her friend ) later? \\
 \visible<3->{Will she call her friend later?}
 \item 今週末は雨が降るでしょうか。\\
 ( rain / will / it ) this weekend?  \\
 \visible<4->{Will it rain  this weekend?}
 \item 明日は忙しいですか。\\
 ( will / busy  / you / be ) tomorrow?  \\
 \visible<5->{Will you be busy tomorrow?}
\end{enumerate}
 \end{frame}


\subsection{Will you 〜? と聞かれたら}
\begin{frame}[plain]{Will you 〜? と聞かれたら}
 \Large

Will you be busy tomorrow?

\vspace{20pt}
\pause

\mbox{}\hspace{100pt}$\left\{\begin{tabular}{l}
         \text{Yes, I will.}\\\pause
         \text{No, I will not.}\\\pause
         \text{(}= \text{No, I won't.)}
        \end{tabular}\right.$

\end{frame}


\subsection{willとbe going to--}
\begin{frame}[plain]{willとbe going to--}
 \Large

\visible<1->{will / be going to--}%
\hfill{}\visible<2->{どちらも未来のことを表します}

\vfill

\visible<3->{%
$\left\{
\begin{tabular}{@{}rl}
A:& What's your plan for tomorrow?\\
B:& I \textcolor{Orange}{\bfseries am going to } meet her.
\end{tabular}\right.$%
}
\hfill{}\visible<4->{{\small あらかじめ計画したこと}}


\visible<5->{%
$\left\{
\begin{tabular}{@{}rl}
A:& I am hungry.\\
B:& I \textcolor{OliveGreen}{\bfseries will} make pancakes.
\end{tabular}\right.%
}$
\hfill{}\visible<6->{{\small 今やると決めたこと}}
\end{frame}

\end{document}
