\documentclass[aspectratio=169,xcolor={dvipsnames,table}]{beamer}
\usepackage[no-math,deluxe,haranoaji]{luatexja-preset}
\renewcommand{\kanjifamilydefault}{\gtdefault}
\renewcommand{\emph}[1]{{\upshape\bfseries #1}}
\usetheme{metropolis}
\metroset{block=fill}
\setbeamertemplate{navigation symbols}{}
\setbeamertemplate{blocks}[rounded][shadow=false]
\usecolortheme[rgb={0.7,0.2,0.2}]{structure}
%%%%%%%%%%%%%%%%%%%%%%%%%%%
\usepackage{media9}
%%%%%%%%%%%%%%%%%%%%%%%%%%%
%% さまざまなアイコン
%%%%%%%%%%%%%%%%%%%%%%%%%%%
\usepackage{fontawesome}
\usepackage{figchild}
\usepackage{twemojis}
\usepackage{utfsym}
\usepackage{bclogo}
\usepackage{marvosym}
\usepackage{fontmfizz}
\usepackage{pifont}
\usepackage{phaistos}
\usepackage{worldflags}
%%%%%%%%%%%%%%%%%%%%%%%%%%%
\usepackage{tikz}
\usetikzlibrary{backgrounds}
\usepackage{tcolorbox}
\usepackage{circledsteps}
\usepackage{xcolor}
\usepackage{amsmath}
\usepackage{pxrubrica}
\usepackage{tipa}
\usepackage{manfnt}
%%%%%%%%%%%%%%%%%%%%%%%%%%%
%% 場合分け
\usepackage{cases}
%%%%%%%%%%%%%%%%%%%%%%%%%%%
% \myAnch{<名前>}{<色>}{<テキスト>}
% 指定のテキストを指定の色の四角枠で囲み, 指定の名前をもつTikZの
% ノードとして出力する. 図には remeber picture 属性を付けている
% ので外部から参照可能である.
\newcommand*{\myAnch}[3]{%
  \tikz[remember picture,baseline=(#1.base)]
    \node[draw,rectangle,#2] (#1) {\normalcolor #3};
}
%%%%%%%%%%%%%%%%%%%%%%%%%%%%
%% 音声リンク表示
\newcommand{\myaudio}[1]{\href{#1}{\faVolumeUp}}
%%%%%%%%%%%%%%%%%%%%%%%%%%%
% \myEmph コマンドの定義
%\newcommand{\myEmph}[3]{%
%    \textbf<#1>{\color<#1>{#2}{#3}}%
%}
\usepackage{xparse} % xparseパッケージの読み込み
\NewDocumentCommand{\myEmph}{O{} m m}{%
    \def\argOne{#1}%
    \ifx\argOne\empty
        \textbf{\color{#2}{#3}}% オプション引数が省略された場合
    \else
        \textbf<#1>{\color<#1>{#2}{#3}}% オプション引数が指定された場合
    \fi
}
%%%%%%%%%%%%%%%%%%%%%%%%%%%
%% 文末の上昇イントネーション記号 \myRisingPitch
%% 通常のイントネーション \myDownwardPitch
%% https://note.com/dan_oyama/n/n8be58e8797b2
%%%%%%%%%%%%%%%%%%%%%%%%%%%
\newcommand{\myRisingPitch}{
\begin{tikzpicture}[scale=0.3,baseline=0.3]
\draw[->,>=stealth] (0,0) to[bend right=45] (1,1);
\end{tikzpicture}
}
\newcommand{\myDownwardPitch}{
\begin{tikzpicture}[scale=0.3,baseline=0.3]
\draw[->,>=stealth] (0,1) to[bend left=45] (1,0);
\end{tikzpicture}
}
%%%%%%%%%%%%%%%%%%%%%%%%%%%
\title{English is fun.}
\subtitle{It will be rainy tomorrow.}
\author{}
\institute[]{}
\date[]

%%%%%%%%%%%%%%%%%%%%%%%%%%%%
%% TEXT
%%%%%%%%%%%%%%%%%%%%%%%%%%%%
\begin{document}


\begin{frame}[plain]
  \titlepage
\end{frame}


\section*{授業の流れ}
\begin{frame}[plain]
  \frametitle{授業の流れ}
  \tableofcontents
\end{frame}


%%%%%%%%%%%%%%%%%%%%%%%%%%%%%%%
\section{will}
\subsection{未来を表すwill}
%%%%%%%%%%%%%%%%%%%%%%%%%%%%%%
\begin{frame}[plain]{未来を表すwill}
\Large
\begin{enumerate}
 \item<1-> I played tennis yesterday.\hfill{}{\small 過去のこと}
 \item<2-> I play tennis.\hfill{}{\small 現在のこと}
 \item<3-> I \alt<1-2>{\myAnch{aux1}{white}{am going to}}{\myAnch{AUX1}{Maroon}{\bfseries am going to}} play tennis tomorrow.\hfill{\small 未来のこと}%
 \item<4-> I  \alt<1-3>{\myAnch{aux1}{white}{will}}{\myAnch{AUX1}{Maroon}{\bfseries will}} play tennis tomorrow.

\end{enumerate}

\begin{block}<5->{Topics for Today}
\small
\begin{itemize}\setbeamertemplate{items}[square]\small
 \item \Circled[fill color=white]{\,$\text{\bfseries will}+\text{原形}$\,}\,も\kenten{未来}のことを表します
 \item \textbf{will} \textipa{/w\'Il/}
\end{itemize}
\end{block}

\mbox{}\hfill{\tiny 0202}\,{\scriptsize \myaudio{./audio/012_will_01.mp3}} 
\end{frame}
%%%%%%%%%%%%%%%%%%%%%%%%%
\begin{frame}[plain]\frametitle{Exercises}

{\small つぎの英文の意味を考えましょう}

\begin{enumerate}
 \item I {\bfseries will} play baseball tomorrow.
 \item We  {\bfseries will} have a party on Saturday.
 \item You {\bfseries will} be a good singer.\hfill{\scriptsize singer \textipa{/s\'i\ng \textrhookschwa/} 歌手}
 \item He {\bfseries will} drive to the airport.\hfill{\scriptsize airport \textipa{/\'e\textrhookschwa p\`O\textrhookschwa t/} 空港}
 \item She {\bfseries will} call him later.\hfill{\scriptsize call \textipa{/k\'O:l/} 電話する\hspace{8pt}later \textipa{/l\'eIt\textrhookschwa /} 後で}
 \item They {\bfseries will} clean the room.\hfill{\scriptsize clean \textipa{/kl\'\i:n/} 掃除する}
\end{enumerate}

\vfill

\begin{block}<2->{Topics for Today}
\begin{itemize}\setbeamertemplate{items}[square]\small
 \item<2->  主語がなんであっても\,\Circled[fill color=white]{\,$\text{\bfseries will}+\text{原形}$\,}
 \item<3-> {\bfseries will}のように動詞と組み合わせて動詞の意味を補足する語を\kenten{助動詞}といいます
 \end{itemize}
     \end{block}

\mbox{}\hfill{\tiny 0251}\,{\scriptsize \myaudio{./audio/012_will_02.mp3}}
\end{frame}
%%%%%%%%%%%%%%%%%%%%%%%%
\begin{frame}[plain]{Exercises}

{\small 次の英文の(~~~~~~~~)内から動詞の正しい形を選び、○で囲みましょう}

\begin{enumerate}
 \item I will ( \alt<2->{\Circled[outer color=Maroon]{~~study~~}}{~~study~~}  /~~studies~~/~~studied~~) English at home.
 \item He will ( \alt<3->{\Circled[outer color=Maroon]{~~play~~}}{~~play~~} /~~plays~~/~~is playing~~) baseball tomorrow.
 \item They always (  \alt<4->{\Circled[outer color=Maroon]{~~read~~}}{~~read~~} /~~reads~~/~~reading~~) books in the library after school.\\
\hfill{\scriptsize always \textipa{/\'O:lweIz/} いつも\hspace{8pt}library \textipa{/l\'aIbreri/} 図書館}
 \item She will (~~is~~/~~are~~/ \alt<5->{\Circled[outer color=Maroon]{~~be~~}}{~~be~~} ) free this month.\hfill{\scriptsize free \textipa{/fr\'\i:/} 暇で}
 \item He will (~~am~~/~~is~~/ \alt<6->{\Circled[outer color=Maroon]{~~be~~}}{~~be~~} ) fourteen years old next month.
\end{enumerate}
\mbox{}\hfill{\tiny 0237}\,{\scriptsize \myaudio{./audio/012_will_03.mp3}} 
\end{frame}
%%%%%%%%%%%%%%%%%%%%%%%%%
\section{短縮形}
%%%%%%%%%%%%%%%
\begin{frame}[plain]\frametitle{willの短縮形}

\begin{enumerate}
 \item {\bfseries I'll} play baseball tomorrow.
 \item {\bfseries We'll} have a party on Saturday.
 \item {\bfseries You'll} be a  good singer.\hfill{\scriptsize singer \textipa{/s\'i\ng \textrhookschwa/} 歌手}
 \item {\bfseries He'll} drive to the airport.\hfill{\scriptsize airport \textipa{/\'e\textrhookschwa p\`O\textrhookschwa t/} 空港}
 \item {\bfseries She'll} call him later.\hfill{\scriptsize call \textipa{/k\'O:l/} 電話する\hspace{8pt}later \textipa{/l\'eIt\textrhookschwa /} 後で}
 \item {\bfseries They'll} clean the room.\hfill{\scriptsize clean \textipa{/kl\'\i:n/} 掃除する}
\end{enumerate}

\vfill

\begin{block}<2->{Topic for Today}
\begin{itemize}\setbeamertemplate{items}[square]\small
 \item  助動詞willにも短縮形があります\
\hfill{}\Circled[fill color=white]{\,S\,} $+$ will $\longrightarrow$ \Circled[fill color=white]{\,S\,}\,'ll\hfill\mbox{}
 \end{itemize}
     \end{block}

\mbox{}\hfill{\tiny 0248}\,{\scriptsize \myaudio{./audio/012_will_03a.mp3}}
\end{frame}
%%%%%%%%%%%%%%%%%%%%%%%%
\section{否定}
%%%%%%%%%%%%%%%%%%%%%%%%
\subsection{will not}
%%%%%%%%%%%%%%%%%%%%%%%%%%
\begin{frame}[plain]{否定を表すnot}
 \Large

否定を表すことば: {\LARGE\bfseries not}\hspace{20pt}\textipa{/n\'At/}
\end{frame}
%%%%%%%%%%%%%%%%%%%%%%%%%%
\begin{frame}[plain]{willの否定文}
\Large

\begin{enumerate}
 \item<1-> I will go to the party tomorrow.
 \item<2-> I \textcolor{Maroon}{\bfseries will not}  go to the party tomorrow.
 \item<3-> I \textcolor{Maroon}{\bfseries won't} go to the party tomorrow.
\end{enumerate}

\bigskip

\begin{block}<4->{Topics for Today}
\begin{itemize}\setbeamertemplate{items}[square]\small
 \item<4->   {\bfseries will}の否定$\longrightarrow${\,\,\,}$\left\{\begin{tabular}{l}\text{{\bfseries will not}}\\\text{{\bfseries won't}}\end{tabular}\right\} + \text{原形}$
 \item<5-> {\bfseries will not} \textipa{/w\'Il n\'At/}\hspace{25pt}{\bfseries won't} \textipa{/w\'oUnt/}
 \end{itemize}
     \end{block}
\mbox{}\hfill{\tiny 0141}\,{\scriptsize \myaudio{./audio/012_will_04.mp3}}
\end{frame}
%%%%%%%%%%%%%%%%%%%%%%%%%%%%%%%%%%%%5
\begin{frame}[plain]{Exercises}

{\small あたえられた日本文の意味になるよう(~~~~~~~~)内の語句を並べ替えましょう}%
\mbox{}\hfill{\tiny 0207}\,{\scriptsize \myaudio{./audio/012_will_05.mp3}}

\begin{enumerate}
 \item わたしは今夜コーヒーを飲みません。\\
 I ~~( coffee / drink / will / not )~~tonight.\hfill{\scriptsize tonight \textipa{/tUn\'aIt/} 今夜}\\
\visible<2->{I will not drink coffee tonight.}
 \item 彼女は今夜、テレビを見ません。\\
She~~( watch / will / not / TV )~~tonight.\\
\visible<3->{She will not watch TV tonight.}
 \item 彼らは新しい車を買わないでしょう。\\
 They~~( car / buy / will / new / not / a ).\hfill{\scriptsize buy \textipa{/b\'aI/} 買う}\\
\visible<4->{They will not buy a new car.}
 \item 明日は雨が降らないでしょう。\\
It~~( be / rainy / not / will )~~tomorrow.\\
\visible<5->{It will not be rainy tomorrow.}
\end{enumerate}

\end{frame}
%%%%%%%%%%%%%%%%%%%%%%%%%%%%%%%
\section{willの疑問文}
%%%%%%%%%%%%%%%%%%%%%%%%%%%%%%%
\begin{frame}[plain]{willの疑問文}

\hspace{32pt}\myAnch{s-1}{Maroon}{He} \myAnch{aux-1}{olive}{will} play tennis this weekend. \scalebox{2}{\myDownwardPitch}\hfill{\scriptsize weekend \textipa{/w\'\i:k\`end/} 週末}
\vspace{30pt}\pause

\myAnch{aux-2}{olive}{Will} \myAnch{s-2}{Maroon}{he} play tennis this weekend\myAnch{question}{orange}{?}
\pause
\begin{tikzpicture}[remember picture, overlay]
 \draw[line width=2pt,opacity=.5, Maroon, ->] (s-1.south) to (s-2.north); 
 \draw[line width=2pt, opacity=.5, olive, ->] (aux-1.south) to[out=-90, in=90] (aux-2.north);
%\pause
%\visible<8->{\node at (0.5,0.25) {\scalebox{2}{\myRisingPitch}};}
\end{tikzpicture}

\vspace{-25pt}
\mbox{}\hspace{200pt}\visible<6->{\scalebox{2}{\myRisingPitch}}

\hspace{40pt}\visible<5->{\small Willを先頭に}
\hspace{80pt}\visible<6-> {\small ?と最後のイントネーションに注意}

\vfill

\begin{block}<7->{Topic for Today}
     \begin{itemize}\setbeamertemplate{items}[square]\small
      \item   {\bfseries will}の疑問文 $\longrightarrow${\,\,\,}$\text{{\bfseries Will}} + \text{主語} + \text{原形}$ \ldots{}\,\,?
      \end{itemize}
\end{block}

\mbox{}\hfill{\tiny 0115}\,{\scriptsize \myaudio{./audio/012_will_06.mp3}}
\end{frame}
%%%%%%%%%%%%%%%%%%%%%
\begin{frame}[plain]{Exercises}

{\small あたえられた日本文の意味になるよう(~~~~~~~~)内の語句を並べ替えましょう。
先頭に来る語は大文字ではじめてください}\mbox{}\hfill{\tiny 0200}\,{\scriptsize \myaudio{./audio/012_will_07.mp3}}

\begin{enumerate}
 \item 今夜彼らは映画を見るつもりですか。\\
 ( a movie / they / will / watch ) tonight?  \\
\visible<2->{Will they watch a movie tonight?}
 \item 彼女は後で友達に電話するつもりですか。\\
 ( call / will / she / her friend ) later?\hfill{\scriptsize later \textipa{/l\'eIt\textrhookschwa /} 後で} \\
 \visible<3->{Will she call her friend later?}
 \item 今週末は雨が降るでしょうか。\\
 ( rain / will / it ) this weekend?\hfill{\scriptsize rain \textipa{/r\'eIn/} \Circled{動} [itを主語にして]雨が降る}\\
 \visible<4->{Will it rain  this weekend?}
 \item 明日は忙しいですか。\\
 ( will / busy  / you / be ) tomorrow?  \\
 \visible<5->{Will you be busy tomorrow?}
\end{enumerate}
 \end{frame}
%%%%%%%%%%%%%%%%%%%%%%%%%%%%%%%%%%%%%

\subsection{Will you 〜? と聞かれたら}
\begin{frame}[plain]{Will you 〜? と聞かれたら}
 \Large

Will you be busy tomorrow?

\vspace{20pt}
\pause

\mbox{}\hspace{100pt}$\left\{\begin{tabular}{l}
         \text{Yes, I will.}\\\pause
         \text{No, I will not.}\\\pause
         \text{(}= \text{No, I won't.)}
        \end{tabular}\right.$

\mbox{}\hfill{\tiny 0143}\,{\scriptsize \myaudio{./audio/012_will_08.mp3}}
\end{frame}

%%%%%%%%%%%%%%%%%%%%%%%%%
\section{まとめ}
\subsection{要点}
%%%%%%%%%%%%%%%%%%%%%
\begin{frame}[plain]{まとめ}
 \begin{block}{will}
\small
\begin{itemize}\setbeamertemplate{items}[square]\small
 \item \Circled[fill color=white]{\,$\text{\bfseries will}+\text{原形}$\,}\,は\kenten{未来}のことを表します\hfill{\bfseries will} \textipa{/w\'Il/}
 \item   助動詞{\bfseries will}も短縮形があります\hfill{}\Circled[fill color=white]{\,S\,} $+$ will $\longrightarrow$ \Circled[fill color=white]{\,S\,}\,'ll\hfill\hfill\mbox{}\\
\hfill{}I will $\rightarrow$ I'll\\
\mbox{}\hfill{}You will $\rightarrow$ You'll\\
\mbox{}\hfill{}He will $\rightarrow$ He'll \ldots
 \item   {\bfseries will}の否定$\longrightarrow${\,\,\,}$\left\{\begin{tabular}{l}\text{{\bfseries will not}}\\\text{{\bfseries won't}}\end{tabular}\right\} + \text{原形}$\hfill{\bfseries will not} \textipa{/w\'Il n\'At/}\hspace{15pt}{\bfseries won't} \textipa{/w\'oUnt/}
 \item   {\bfseries will}の疑問文 $\longrightarrow${\,\,\,}$\text{{\bfseries Will}} + \text{主語} + \text{原形}$ \ldots{}\,\,?

\end{itemize}
\end{block}
\end{frame}
%%%%%%%%%%%%%%%%%%%%%%%%%%%%%
\subsection{ところでwillとbe going to--}
\begin{frame}[plain]{willとbe going to--}
 \Large

\visible<1->{will / be going to--}%
\hfill{}\visible<1->{{\small どちらも\kenten{未来}のことを表します}}

\bigskip

\visible<2->{%
$\left\{
\begin{tabular}{@{}rl}
A:& What's your plan for tomorrow?\\
B:& I \textcolor{Maroon}{\bfseries am going to } meet her.
\end{tabular}\right.$%
}
\hfill{}\visible<5->{{\small あらかじめ計画したこと}}


\visible<4->{%
$\left\{
\begin{tabular}{@{}rl}
A:& I am hungry.\\
B:& I \textcolor{OliveGreen}{\bfseries will} make pancakes.
\end{tabular}\right.%
}$
\hfill{}\visible<6->{{\small 今やると決めたこと}}

\dbend\dbend\hfill{\tiny 0154}\,{\scriptsize \myaudio{./audio/012_will_09.mp3}}
\end{frame}
%%%%%%%%%%%%%%%%%%%%%
\end{document}
