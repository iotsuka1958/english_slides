\documentclass[aspectratio=169,xcolor={dvipsnames,table}]{beamer}
\usepackage[no-math,deluxe,haranoaji]{luatexja-preset}
\renewcommand{\kanjifamilydefault}{\gtdefault}
\renewcommand{\emph}[1]{{\upshape\bfseries #1}}
\usetheme{metropolis}
\metroset{block=fill}
\setbeamertemplate{navigation symbols}{}
\setbeamertemplate{blocks}[rounded][shadow=false]
\usecolortheme[rgb={0.7,0.2,0.2}]{structure}
%%%%%%%%%%%%%%%%%%%%%%%%%%
%% Change alert block colors
%%% 1- Block title (background and text)
\setbeamercolor{block title alerted}{fg=mDarkTeal, bg=mLightBrown!45!yellow!45}
\setbeamercolor{block title example}{fg=magenta!10!black, bg=mLightGreen!70}
%%% 2- Block body (background)
\setbeamercolor{block body alerted}{bg=mLightBrown!25}
\setbeamercolor{block body example}{bg=mLightGreen!15}
%%%%%%%%%%%%%%%%%%%%%%%%%%%
%%%%%%%%%%%%%%%%%%%%%%%%%%%
%% さまざまなアイコン
%%%%%%%%%%%%%%%%%%%%%%%%%%%
%\usepackage{fontawesome}
\usepackage{fontawesome5}
\usepackage{figchild}
\usepackage{twemojis}
\usepackage{utfsym}
\usepackage{bclogo}
\usepackage{marvosym}
\usepackage{fontmfizz}
\usepackage{pifont}
\usepackage{phaistos}
\usepackage{worldflags}
\usepackage{jigsaw}
\usepackage{tikzlings}
\usepackage{tikzducks}
\usepackage{scsnowman}
\usepackage{epsdice}
\usepackage{halloweenmath}
\usepackage{svrsymbols}
\usepackage{countriesofeurope}
\usepackage{tipa}
\usepackage{manfnt}
%%%%%%%%%%%%%%%%%%%%%%%%%%%
\usepackage{tikz}
\usetikzlibrary{calc,patterns,decorations.pathmorphing,backgrounds}
\usepackage{tcolorbox}
\usepackage{tikzpeople}
\usepackage{circledsteps}
\usepackage{xcolor}
\usepackage{amsmath}
\usepackage{booktabs}
\usepackage{chronology}
\usepackage{signchart}
%%%%%%%%%%%%%%%%%%%%%%%%%%%
%% 場合分け
%%%%%%%%%%%%%%%%%%%%%%%%%%%
\usepackage{cases}
%%%%%%%%%%%%%%%%%%%%%%%%%%
\usepackage{pdfpages}
%%%%%%%%%%%%%%%%%%%%%%%%%%%
%% 音声リンク表示
\newcommand{\myaudio}[1]{\href{#1}{\faVolumeUp}}
%%%%%%%%%%%%%%%%%%%%%%%%%%
%% \myAnch{<名前>}{<色>}{<テキスト>}
%% 指定のテキストを指定の色の四角枠で囲み, 指定の名前をもつTikZの
%% ノードとして出力する. 図には remember picture 属性を付けている
%% ので外部から参照可能である.
\newcommand*{\myAnch}[3]{%
  \tikz[remember picture,baseline=(#1.base)]
    \node[draw,rectangle,line width=1pt,#2] (#1) {\normalcolor #3};
}
%%%%%%%%%%%%%%%%%%%%%%%%%%
%% \myEmph コマンドの定義
%%%%%%%%%%%%%%%%%%%%%%%%%%
%\newcommand{\myEmph}[3]{%
%    \textbf<#1>{\color<#1>{#2}{#3}}%
%}
\usepackage{xparse} % xparseパッケージの読み込み
\NewDocumentCommand{\myEmph}{O{} m m}{%
    \def\argOne{#1}%
    \ifx\argOne\empty
        \textbf{\color{#2}{#3}}% オプション引数が省略された場合
    \else
        \textbf<#1>{\color<#1>{#2}{#3}}% オプション引数が指定された場合
    \fi
}
%%%%%%%%%%%%%%%%%%%%%%%%%%%
%%%%%%%%%%%%%%%%%%%%%%%%%%%
%% 文末の上昇イントネーション記号 \myRisingPitch
%% 通常のイントネーション \myDownwardPitch
%% https://note.com/dan_oyama/n/n8be58e8797b2
%%%%%%%%%%%%%%%%%%%%%%%%%%%
\newcommand{\myRisingPitch}{
\begin{tikzpicture}[scale=0.3,baseline=0.3]
\draw[->,>=stealth] (0,0) to[bend right=45] (1,1);
\end{tikzpicture}
}
\newcommand{\myDownwardPitch}{
\begin{tikzpicture}[scale=0.3,baseline=0.3]
\draw[->,>=stealth] (0,1) to[bend left=45] (1,0);
\end{tikzpicture}
}
%%%%%%%%%%%%%%%%%%%%%%%%%%%%
%\AtBeginSection[%
%]{%
%  \begin{frame}[plain]\frametitle{授業の流れ}
%     \tableofcontents[currentsection]
%   \end{frame}%
%}

%%%%%%%%%%%%%%%%%%%%%%%%%%%
\title{English is fun.}
\subtitle{I was watching television when he came.}
\author{}
\institute[]{}
\date[]

%%%%%%%%%%%%%%%%%%%%%%%%%%%%
%% TEXT
%%%%%%%%%%%%%%%%%%%%%%%%%%%%
\begin{document}


\begin{frame}[plain]
  \titlepage
\end{frame}


\section*{授業の流れ}
\begin{frame}[plain]
  \frametitle{授業の流れ}
  \tableofcontents
\end{frame}

\section{when \textipa{/w\'en/}}
\subsection{when}
%%%%%%%%%%%%%%%%%%%%%%%%%%%%%%%%%%%%%%%%%%%%%
\begin{frame}[plain]{~するとき}
\Large
 \begin{enumerate}
  \item<1-> I \myAnch{a_1}{white}{\myEmph[2-]{NavyBlue}{was watching}} television \alt<2->{\myAnch{b_1}{Maroon}{then}}{\myAnch{B_1}{white}{then}}.%
\hfill{}\visible<2->{{\scriptsize then \textipa{/D\'en/} そのとき(副詞)}}
  \item<3-> I \myAnch{a_2}{white}{\myEmph[4-]{NavyBlue}{was watching}} television  \alt<4->{\myAnch{b_2}{Maroon}{at that time}}{\myAnch{B_2}{white}{at that time}}.
  \item<5-> I \myAnch{a_3}{white}{\myEmph[6-]{NavyBlue}{was watching}} television   \alt<6->{\myAnch{b_3}{Maroon}{when he came}}{\myAnch{B_3}{white}{when he came}}.
\end{enumerate}
\mbox{}\hfill{\scriptsize \myaudio{./audio/021_when_01.mp3}}
\visible<2->%
{\begin{tikzpicture}[remember picture, overlay]
 \visible<2->{\draw[thick, Maroon, ->] (B_1.north west) to[out=170, in=10] (a_1.north);} 
 \visible<4->{\draw[thick, Maroon, ->] (B_2.north west) to[out=170, in=10] (a_2.north);} 
 \visible<6->{\draw[thick, Maroon, ->] (B_3.north west) to[out=170, in=10] (a_3.north);} 
\end{tikzpicture}}

\begin{block}<7->{Topics for Today}\small
\begin{itemize}\setbeamertemplate{items}[square]\small
 \item   \Circled[fill color = white]{ when s $+$ v }\,は「〜が~するとき」という意味
 \item   \Circled[fill color = white]{ when s $+$ v }\,は全体で「副詞」の役割
\end{itemize}
     \end{block}
\end{frame}
%%%%%%%%%%%%%%%%%%%%%%%%
\begin{frame}[plain]{Exercises}
日本語の意味になるよう、(~~~~~)内の語句を並べ替えましょう%
\mbox{}\hfill{\scriptsize \myaudio{./audio/021_when_02.mp3}}

\begin{enumerate}
 \item 若いころ母はテニスがうまかった。\\
       My mother was a good tennis player ( was / she / when / young ).\\
       \visible<2->{\textcolor{lightgray}{My mother was a good tennis player} when she was young.}
 \item わたしが見かけたとき、彼は妹といっしょだった。\\
       He was with his sister ( him / saw / when / I ). \\
       \visible<3->{\textcolor{lightgray}{He was with his sister} when I saw him.}
 \item  彼女は14歳のとき、京都にいました。\\
	She ( was / was / Kyoto / she / fourteen / when / in ).\\
       \visible<4->{\textcolor{lightgray}{She} was in Kyoto when she was fourteen.}
\end{enumerate}
\end{frame}
%%%%%%%%%%%%%%%%%%%%%%%%%%%%%%
\section{\fbox{when S $+$ V}\,の位置}
\begin{frame}[plain]{\fbox{when S $+$ V}\,の位置}
\large
 \begin{enumerate}
  \item \begin{enumerate}
	 \item The door  \myAnch{x_1}{white}{\myEmph[2-]{NavyBlue}{opened}} \alt<2->{\myAnch{y_1}{Maroon}{suddenly}}{\myAnch{Y_1}{white}{suddenly}}.%
	       \hfill{\scriptsize suddenly \textipa{/s\'\textturnv dnli/} とつぜん(副詞)}
	 \item  \alt<3->{\myAnch{y_2}{Maroon}{Suddenly}}{\myAnch{Y_2}{white}{Suddenly}},\,\, the door  \myAnch{x_2}{white}{\myEmph[3-]{NavyBlue}{opened}}.
	\end{enumerate}
  \item \begin{enumerate}
	 \item I \myAnch{x_3}{white}{\myEmph[3-]{NavyBlue}{was reading}} a book \alt<4->{\myAnch{y_3}{Maroon}{at that time}}{\myAnch{Y_3}{white}{at that time}}.
	 \item \alt<5->{\myAnch{y_4}{Maroon}{At that time}}{\myAnch{Y_4}{white}{At that time}},\,\, I \myAnch{x_4}{white}{\myEmph[4-]{NavyBlue}{was reading}} a book.
	\end{enumerate}
  \item \begin{enumerate}
	 \item I \myAnch{x_5}{white}{\myEmph[5-]{NavyBlue}{was sleeping}}
\alt<6->{\myAnch{y_5}{Maroon}{when he came}}{\myAnch{Y_5}{white}{when he came}}.\hfill{}\visible<9->{{\bfseries S $+$ V}\,\,\Circled[fill color = white]{ when s $+$ v }}
	 \item \alt<7->{\myAnch{y_6}{Maroon}{When he came}}{\myAnch{Y_6}{white}{When he came}},\,\, I \myAnch{x_6}{white}{\myEmph[6-]{NavyBlue}{was sleeping}}.\hfill{}\visible<9->{\Circled[fill color = white]{ when s $+$ v }, {\bfseries S $+$ V}}
	\end{enumerate}
 \end{enumerate}

\visible<2->%
{\begin{tikzpicture}[remember picture, overlay]
 \visible<2->{\draw[thick, Maroon, ->] (Y_1.north west) to[out=170, in=10] (x_1.north);} 
 \visible<3->{\draw[thick, Maroon, ->] (Y_2.north east) to[out=5, in=175] (x_2.north);} 
 \visible<4->{\draw[thick, Maroon, ->] (Y_3.north west) to[out=170, in=10] (x_3.north);}
 \visible<5->{\draw[thick, Maroon, ->] (Y_4.north east) to[out=5, in=175] (x_4.north);}  
 \visible<6->{\draw[thick, Maroon, ->] (Y_5.north west) to[out=170, in=10] (x_5.north);}
 \visible<7->{\draw[thick, Maroon, ->] (Y_6.north east) to[out=5, in=175] (x_6.north);}  
\end{tikzpicture}}

\vspace{-20pt}

\begin{block}<8->{Topics for Today}\small
\begin{itemize}\setbeamertemplate{items}[square]\small
 \item   \visible<8->{\Circled[fill color = white]{ when s $+$ v }\,は「〜が~するとき」という意味}%
\mbox{}\hfill{\scriptsize \myaudio{./audio/021_when_03.mp3}}
 \item   \visible<8->{\Circled[fill color = white]{ when s $+$ v }\,は全体で「副詞」の役割}
 \item   \visible<9->{\Circled[fill color = white]{ when s $+$ v }\,は、前にも後ろにも位置します}
	 \begin{enumerate}
	  \item<9-> {\bfseries S $+$ V}\,\,\Circled[fill color = white]{ when s $+$ v }
	  \item<9-> \Circled[fill color = white]{ When s $+$ v }\,,\,\,{\bfseries S $+$ V}
	 \end{enumerate}
\end{itemize}
     \end{block}
\end{frame}
%%%%%%%%%%%%%%%%%%%%%
\begin{frame}[plain]{Exercises}
つぎの英文を、同じ意味になるようにWhenではじめて書き換えましょう%
\mbox{}\hfill{\scriptsize \myaudio{./audio/021_when_04.mp3}}
\begin{enumerate}
 \item I was five when my brother was born.\\
       $\longrightarrow$\, When \visible<2->{my brother was born, I was five.}
 \item He was with his sister when I saw him.\\
       $\longrightarrow$\, When \visible<3->{I saw him, he was with his sister.}
 \item She was studying when I called her.\\
       $\longrightarrow$\, When \visible<4->{I called her, she was studying.}
\end{enumerate}
\end{frame}
%%%%%%%%%%%%%%%%%%%%%%%%%%%
\section{まとめ}
\begin{frame}[plain]{まとめ}
\begin{exampleblock}{Topics for Today}
\begin{itemize}
 \item   \visible{\Circled[fill color = white]{ when s $+$ v }\,は「〜が~するとき」という意味}
 \item   \visible{\Circled[fill color = white]{ when s $+$ v }\,は全体で「副詞」の役割}
 \item   \visible{\Circled[fill color = white]{ when s $+$ v }\,は、前にも後ろにも位置します}
	 \begin{enumerate}
	  \item {\bfseries S $+$ V}\,\,\Circled[fill color = white]{ when s $+$ v }
	  \item \Circled[fill color = white]{ When s $+$ v },\,\,{\bfseries S $+$ V}
.	 \end{enumerate}
\end{itemize}
     \end{exampleblock}
\end{frame}
\end{document}
