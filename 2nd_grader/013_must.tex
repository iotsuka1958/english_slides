\documentclass[aspectratio=169,xcolor={dvipsnames,table}]{beamer}
\usepackage[no-math,deluxe,haranoaji]{luatexja-preset}
\renewcommand{\kanjifamilydefault}{\gtdefault}
\renewcommand{\emph}[1]{{\upshape\bfseries #1}}
\usetheme{metropolis}
\metroset{block=fill}
\setbeamertemplate{navigation symbols}{}
\setbeamertemplate{blocks}[rounded][shadow=false]
\usecolortheme[rgb={0.7,0.2,0.2}]{structure}
%%%%%%%%%%%%%%%%%%%%%%%%%%%
\usepackage{media9}
%%%%%%%%%%%%%%%%%%%%%%%%%%%
%% さまざまなアイコン
%%%%%%%%%%%%%%%%%%%%%%%%%%%
\usepackage{fontawesome}
\usepackage{figchild}
\usepackage{twemojis}
\usepackage{utfsym}
\usepackage{bclogo}
\usepackage{marvosym}
\usepackage{fontmfizz}
\usepackage{pifont}
\usepackage{phaistos}
\usepackage{worldflags}
%%%%%%%%%%%%%%%%%%%%%%%%%%%
\usepackage{tikz}
\usetikzlibrary{backgrounds}
\usepackage{tcolorbox}
\usepackage{tikzpeople}
\usepackage{circledsteps}
\usepackage{xcolor}
\usepackage{amsmath}
\usepackage{tipa}
%%%%%%%%%%%%%%%%%%%%%%%%%%%
%% 場合分け
\usepackage{cases}
%%%%%%%%%%%%%%%%%%%%%%%%%%%
% \myAnch{<名前>}{<色>}{<テキスト>}
% 指定のテキストを指定の色の四角枠で囲み, 指定の名前をもつTikZの
% ノードとして出力する. 図には remeber picture 属性を付けている
% ので外部から参照可能である.
\newcommand*{\myAnch}[3]{%
  \tikz[remember picture,baseline=(#1.base)]
    \node[draw,rectangle,#2] (#1) {\normalcolor #3};
}
%%%%%%%%%%%%%%%%%%%%%%%%%%%%
%% 音声リンク表示
\newcommand{\myaudio}[1]{\href{#1}{\faVolumeUp}}
%%%%%%%%%%%%%%%%%%%%%%%%%%%
% \myEmph コマンドの定義
%\newcommand{\myEmph}[3]{%
%    \textbf<#1>{\color<#1>{#2}{#3}}%
%}
\usepackage{xparse} % xparseパッケージの読み込み
\NewDocumentCommand{\myEmph}{O{} m m}{%
    \def\argOne{#1}%
    \ifx\argOne\empty
        \textbf{\color{#2}{#3}}% オプション引数が省略された場合
    \else
        \textbf<#1>{\color<#1>{#2}{#3}}% オプション引数が指定された場合
    \fi
}
%%%%%%%%%%%%%%%%%%%%%%%%%%%
%% 文末の上昇イントネーション記号 \myRisingPitch
%% 通常のイントネーション \myDownwardPitch
%% https://note.com/dan_oyama/n/n8be58e8797b2
%%%%%%%%%%%%%%%%%%%%%%%%%%%
\newcommand{\myRisingPitch}{
\begin{tikzpicture}[scale=0.3,baseline=0.3]
\draw[->,>=stealth] (0,0) to[bend right=45] (1,1);
\end{tikzpicture}
}
\newcommand{\myDownwardPitch}{
\begin{tikzpicture}[scale=0.3,baseline=0.3]
\draw[->,>=stealth] (0,1) to[bend left=45] (1,0);
\end{tikzpicture}
}
%%%%%%%%%%%%%%%%%%%%%%%%%%%
\title{English is fun.}
\subtitle{You must watch this movie. It's so funny.}
\author{}
\institute[]{}
\date[]

%%%%%%%%%%%%%%%%%%%%%%%%%%%%
%% TEXT
%%%%%%%%%%%%%%%%%%%%%%%%%%%%
\begin{document}
\begin{frame}[plain]
  \titlepage
\end{frame}

\section*{授業の流れ}
\begin{frame}[plain]
  \frametitle{授業の流れ}
  \tableofcontents
\end{frame}

%%%%%%%%%%%%%%%%%%%%%%%%%%%%%%
\section{must}
\subsection{mustの意味}
%%%%%%%%%%%%%%%%%%%%%%%%%%%%%
\begin{frame}[plain]\frametitle{わたしは〜しなければならない}
 \large\setlength{\fboxrule}{1pt}

\visible<1->{%
I \textcolor{Green}{\bfseries study} math every day.}
\hspace{20pt}\visible<2->{{\footnotesize study: 勉強する math: 数学}}
\hfill{}\visible<3->{{\small 現在の習慣}}

\vspace{10pt}

\visible<4->{I \alt<1-4>{\myAnch{aux1}{white}{\bfseries will}}{\myAnch{AUX1}{Orange}{\bfseries will}}%
\textcolor{Green}{\bfseries study} math tonight.}%
\hspace{20pt}\visible<5->{{\footnotesize tonight: 今晩}}
\hfill{}\visible<6->{{\small 未来のこと}}

\vspace{10pt}

\visible<7->{I \alt<1-7>{\myAnch{aux2}{white}{{\bfseries must}}}{\myAnch{AUX2}{Orange}{{\bfseries must}}}  \textcolor{Green}{\bfseries study} math hard.}\hspace{25pt}\visible<7->{{\scriptsize \textipa{/m\'\textturnv st/}}}\hfill\visible<8->{{\small 義務}}


\vfill

\begin{block}<9->{Topics for Today}
\begin{itemize}\setbeamertemplate{items}[square]\small
 \item   「わたしは〜しなければならない」$\longrightarrow${\,\,\,}I {\bfseries must} $+$ 原形\mbox{}
 \item {\bfseries must} \textipa{/m\'\textturnv st/}
 \end{itemize}
     \end{block}
\hfill{\tiny 0138}\,{\scriptsize \myaudio{./audio/013_must_01.mp3}}

\visible<8->{%
\begin{tikzpicture}[remember picture, overlay]
 \draw[line width=2pt,opacity=.5,BurntOrange, ->] (aux1.south) -- (aux2.north) ;
\end{tikzpicture}%
}

\visible<9->{\small {\bfseries will}や{\bfseries can}, {\bfseries must}のように動詞と組み合わせて使用することで、$+\alpha$(プラスアルファ)の意味を付け加える語を\textcolor{BurntOrange}{助動詞}といいます}
\end{frame}
%%%%%%%%%%%%%%%%%%%%
\begin{frame}[plain]{Exercises}
あたえられた日本文の意味になるようカッコ内の語句を並べ替えましょう\hfill{\tiny 0224}\,{\scriptsize \myaudio{./audio/013_must_02.mp3}}
\begin{enumerate}
 \item {\small 私は宿題を終わらせなければならない。}
( finish / my homework / I / must )\\
\visible<2->{I must finish my homework.}\hfill{}{\scriptsize finish \textipa{/f\'InIS/} 終える}
 \item {\small 私は部屋を掃除しなければなりません。}
( my room / I / must / clean )\\
\visible<3->{I must clean my room.}\hfill{\scriptsize clean \textipa{/kl\'\i:n/} 掃除する}
 \item {\small 私は早く寝なければなりません。}
( to / bed / I / must / go / early )\\
\visible<4->{I must go to bed early.}\hfill{}{\scriptsize go to bed: 寝る}
 \item {\small 私はこの本を読まなければなりません。}
( must / this / I / read / book )\\
\visible<5->{I must read this book.}
 \item {\small 私はじゅうぶんな睡眠をとらなければなりません。}\\
( enough sleep / get / must / I )\hfill{\scriptsize enough sleep: じゅうぶんな睡眠}\\
\visible<6->{I must get enough sleep.}\hfill{}{\scriptsize enough \textipa{/In\'\textturnv f/} じゅうぶんな}
\end{enumerate}
\end{frame}
%%%%%%%%%%%%%%%%%%%%%%%%
\subsection{mustに続く動詞の形}
%%%%%%%%%%%%%%%%%%%%%%
 \begin{frame}[plain]\frametitle{Heの場合}
 \large\setlength{\fboxrule}{1pt}

\visible<1->{%
He \textcolor{Green}{\bfseries play}\textcolor{BurntOrange}{s} the guitar every day.}
\hspace{20pt}\visible<3->{{\footnotesize \textcolor{BurntOrange}{三人称単数現在の`s'}}}
\hfill{}\visible<2->{{\small 現在の習慣}}

\vspace{10pt}

\visible<4->{He \alt<1-4>{\myAnch{aux3}{white}{\bfseries will}}{\myAnch{AUX1}{Orange}{\bfseries will}}%
\textcolor{Green}{\bfseries play} the guitar tonight.}%
\hspace{20pt}\visible<5->{{\footnotesize tonight: 今晩}}
\hfill{}\visible<6->{{\small 未来のこと}}

\vspace{10pt}

\visible<7->{He \alt<1-7>{\myAnch{aux4}{white}{\bfseries must}}{\myAnch{AUX2}{Orange}{\bfseries must}}  \textcolor{Green}{\bfseries play} the guitar.}\hfill\visible<8->{{\small 義務}}


\vfill

\begin{block}<9->{Topic for Today}
\begin{itemize}\setbeamertemplate{items}[square]\small
 \item   主語がなんであっても$\longrightarrow${\,\,\,}$\text{S}+\text{\bfseries must}+\text{原形}$\\
\mbox{}\hfill{}*He must play\Circled[fill color=white]{s} \ldots とはいわない\hspace{1\zw}\mbox{}
 \end{itemize}
     \end{block}

\hfill{\tiny 0139}\,{\scriptsize \myaudio{./audio/013_must_03.mp3}}

\hfill\visible<9->{{\small {\bfseries will}や{\bfseries can}の場合と同じです}}

\visible<8->{%
\begin{tikzpicture}[remember picture, overlay]
 \draw[line width=2pt,opacity=.5,BurntOrange, ->] (aux3.south) -- (aux4.north) ;
\end{tikzpicture}%
}
\end{frame}
%%%%%%%%%%%%%%%%%%%%%%%%
\section{~してはいけない(禁止)}
%%%%%%%%%%%%%%%%%%%%%%%%%%
\begin{frame}[plain]{否定を表すnot}
 \Large

否定を表すことば: {\LARGE\bfseries not}\hspace{20pt}\textipa{/n\'At/}
\end{frame}
%%%%%%%%%%%%%%%%%%%%%%%%%%
%%%%%%%%%%%%%%%%%%%%%%%
\begin{frame}[plain,t]{must not}
 
\large

\visible<1->{You \textcolor{NavyBlue}{\bfseries must} \textcolor{ForestGreen}{\bfseries be} quiet in the library.}%
\hfill\visible<2->{{\scriptsize 図書館では静かにしなければいけない。(義務)}}

\visible<3->{You \textcolor{NavyBlue}{\bfseries must} \textcolor{Maroon}{\bfseries not} \textcolor{ForestGreen}{\bfseries eat} in the library.}%
\hfill\visible<4->{{\scriptsize 図書館ではものを食べてはいけない。(禁止)}}

\visible<5->{You \textcolor{NavyBlue}{\bfseries must}\textcolor{Maroon}{\bfseries n't} \textcolor{ForestGreen}{\bfseries eat} in the library.}\hfill\visible<6->{{\scriptsize 短縮形 mustn't \textipa{/m\'\textturnv snt/}}}

\hfill{\scriptsize quiet \textipa{/kw\'aI@t/} 静かな}

\vfill

\begin{block}<6->{Topics for Today}
\begin{itemize}\setbeamertemplate{items}[square]\small
 \item \textbf{must} は「義務」を表す\hspace{27pt}〜しなければならない
 \item \textbf{must not} は「禁止」を表す\hspace{10pt}〜してはならない
 \item 短縮形\textbf{mustn't} \textipa{/m\'\textturnv snt/}
 \end{itemize}
     \end{block}
\hfill{\tiny 0139}\,{\scriptsize \myaudio{./audio/013_must_04.mp3}}

\end{frame}
%%%%%%%%%%%%
\begin{frame}<1-4>[plain]{Exercises}
 
あたえられた日本語の意味になるように、(~~~~~~)に適当な語を補いましょう
\hfill{\tiny 0141}\,{\scriptsize \myaudio{./audio/013_must_05.mp3}}


\begin{enumerate}
 \item {\small 寝る前にコーヒーを飲んではいけない。}%
\hfill{}{\scriptsize before bed 寝る前に}\\
You \alt<1>{(~~\phantom{must}~~) (~~\phantom{not}~~)}{(~~must~~) (~~not~~)} drink coffee before bed.
 \item {\small 暗くなってから泳いではいけません。}\hfill{}{\scriptsize after dark 暗くなってから}\\
 You \alt<1-2>{(~~\phantom{mustn't}~~)}{(~~mustn't~~)} swim after dark.
 \item {\small 図書館で走ってはいけません。}%
\hfill{}{\scriptsize library \textipa{/l\'aIbreri/} 図書館}\\
 You \alt<1-3>{(~~\phantom{mustn't}~~) (~~\phantom{run}~~)}{(~~mustn't~~) (~~run~~)}  in the library.
\end{enumerate}
\end{frame}

%%%%%%%%%%%%%%%%%%%%%%%%%%%
\section{助動詞mustのまとめ}
\begin{frame}[plain]{まとめ}
 \Large

\begin{block}{Topics for Today}
\small
\begin{itemize}\setbeamertemplate{items}[square]
 \item<2->  助動詞{\bfseries must} は「義務」(〜しなければならない)を表す%
\hfill{}{\scriptsize \textipa{/m\'\textturnv st/}}
 \item<3->  {\bfseries must not}($=\text{\bfseries mustn't}$)\,は「禁止」(〜してはならない)を表す%
\hfill{}{\scriptsize \textipa{/m\'\textturnv snt/}}
 \item<4->  {\bfseries must} / {\bfseries must not}($=\text{\bfseries mustn't}$)\,に続く動詞は原形
\end{itemize}
\end{block}

\vfill

\visible<5->{{\small {\bfseries will}や{\bfseries can}, {\bfseries must}のように動詞と組み合わせて使用することで、$+\alpha$(プラスアルファ)の意味を付け加える語を\textcolor{Maroon}{助動詞}といいます。}}

\end{frame}
%%%%%%%%%%%%%%%%%%%%%%%%
\begin{frame}[plain]
 \large

\begin{enumerate}
 \item You \textbf{must} study hard.
 \item You \textbf{must} finish your homework.
 \item You \textbf{must} clean you room.
\end{enumerate}

\end{frame}
%%%%%%%%%%%%%%%%%%%%%%%%%%%%%%%%%%%%
\begin{frame}[plain]{Pancakes}
\begin{columns}
\begin{column}{.5\textwidth}
 \includegraphics[height=\textheight]{./images/pancakes.jpg}
\end{column}
%%%%%%%%%%
\begin{column}{.4\textwidth}\large
$\left\{
\begin{tabular}{@{}rl}
A:& You \textbf{must} eat these. \\
&They're delicious.\\
B:& Thanks.
\end{tabular}\right.$

\bigskip

\bigskip

\bigskip

\hfill{\scriptsize delicious \textipa{/dIl\'IS@s/} \Circled{\,形\,}\,おいしい}
\end{column}
\end{columns}
\end{frame}
\end{document}
