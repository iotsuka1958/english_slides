\documentclass[aspectratio=169,xcolor={dvipsnames,table}]{beamer}
\usepackage[no-math,deluxe,haranoaji]{luatexja-preset}
\renewcommand{\kanjifamilydefault}{\gtdefault}
\renewcommand{\emph}[1]{{\upshape\bfseries #1}}
\usetheme{metropolis}
\metroset{block=fill}
\setbeamertemplate{navigation symbols}{}
\setbeamertemplate{blocks}[rounded][shadow=false]
\usecolortheme[rgb={0.7,0.2,0.2}]{structure}
%%%%%%%%%%%%%%%%%%%%%%%%%%
%% Change alert block colors
%%% 1- Block title (background and text)
\setbeamercolor{block title alerted}{fg=mDarkTeal, bg=mLightBrown!45!yellow!45}
\setbeamercolor{block title example}{fg=magenta!10!black, bg=mLightGreen!70}
%%% 2- Block body (background)
\setbeamercolor{block body alerted}{bg=mLightBrown!25}
\setbeamercolor{block body example}{bg=mLightGreen!15}
%%%%%%%%%%%%%%%%%%%%%%%%%%%
%%%%%%%%%%%%%%%%%%%%%%%%%%%
%% さまざまなアイコン
%%%%%%%%%%%%%%%%%%%%%%%%%%%
%\usepackage{fontawesome}
\usepackage{fontawesome5}
\usepackage{figchild}
\usepackage{twemojis}
\usepackage{utfsym}
\usepackage{bclogo}
\usepackage{marvosym}
\usepackage{fontmfizz}
\usepackage{pifont}
\usepackage{phaistos}
\usepackage{worldflags}
\usepackage{jigsaw}
\usepackage{tikzlings}
\usepackage{tikzducks}
\usepackage{scsnowman}
\usepackage{epsdice}
\usepackage{halloweenmath}
\usepackage{svrsymbols}
\usepackage{countriesofeurope}
\usepackage{tipa}
\usepackage{manfnt}
%%%%%%%%%%%%%%%%%%%%%%%%%%%
\usepackage{tikz}
\usetikzlibrary{calc,patterns,decorations.pathmorphing,backgrounds}
\usepackage{tcolorbox}
\usepackage{tikzpeople}
\usepackage{circledsteps}
\usepackage{xcolor}
\usepackage{amsmath}
\usepackage{booktabs}
\usepackage{chronology}
\usepackage{signchart}
%%%%%%%%%%%%%%%%%%%%%%%%%%%
%% 場合分け
%%%%%%%%%%%%%%%%%%%%%%%%%%%
\usepackage{cases}
%%%%%%%%%%%%%%%%%%%%%%%%%%
\usepackage{pdfpages}
%%%%%%%%%%%%%%%%%%%%%%%%%%%
%% 音声リンク表示
\newcommand{\myaudio}[1]{\href{#1}{\faVolumeUp}}
%%%%%%%%%%%%%%%%%%%%%%%%%%
%% \myAnch{<名前>}{<色>}{<テキスト>}
%% 指定のテキストを指定の色の四角枠で囲み, 指定の名前をもつTikZの
%% ノードとして出力する. 図には remember picture 属性を付けている
%% ので外部から参照可能である.
\newcommand*{\myAnch}[3]{%
  \tikz[remember picture,baseline=(#1.base)]
    \node[draw,rectangle,line width=1pt,#2] (#1) {\normalcolor #3};
}
%%%%%%%%%%%%%%%%%%%%%%%%%%
%% \myEmph コマンドの定義
%%%%%%%%%%%%%%%%%%%%%%%%%%
%\newcommand{\myEmph}[3]{%
%    \textbf<#1>{\color<#1>{#2}{#3}}%
%}
\usepackage{xparse} % xparseパッケージの読み込み
\NewDocumentCommand{\myEmph}{O{} m m}{%
    \def\argOne{#1}%
    \ifx\argOne\empty
        \textbf{\color{#2}{#3}}% オプション引数が省略された場合
    \else
        \textbf<#1>{\color<#1>{#2}{#3}}% オプション引数が指定された場合
    \fi
}
%%%%%%%%%%%%%%%%%%%%%%%%%%%
%%%%%%%%%%%%%%%%%%%%%%%%%%%
%% 文末の上昇イントネーション記号 \myRisingPitch
%% 通常のイントネーション \myDownwardPitch
%% https://note.com/dan_oyama/n/n8be58e8797b2
%%%%%%%%%%%%%%%%%%%%%%%%%%%
\newcommand{\myRisingPitch}{
\begin{tikzpicture}[scale=0.3,baseline=0.3]
\draw[->,>=stealth] (0,0) to[bend right=45] (1,1);
\end{tikzpicture}
}
\newcommand{\myDownwardPitch}{
\begin{tikzpicture}[scale=0.3,baseline=0.3]
\draw[->,>=stealth] (0,1) to[bend left=45] (1,0);
\end{tikzpicture}
}
%%%%%%%%%%%%%%%%%%%%%%%%%%%%
%\AtBeginSection[%
%]{%
%  \begin{frame}[plain]\frametitle{授業の流れ}
%     \tableofcontents[currentsection]
%   \end{frame}%
%}

%%%%%%%%%%%%%%%%%%%%%%%%%%%
\title{English is fun.}
\subtitle{I like to play the piano. I want to be a pianist.}
\author{}
\institute[]{}
\date[]

%%%%%%%%%%%%%%%%%%%%%%%%%%%%
%% TEXT
%%%%%%%%%%%%%%%%%%%%%%%%%%%%
\begin{document}


\begin{frame}[plain]
  \titlepage
\end{frame}


\section*{授業の流れ}
\begin{frame}[plain]
  \frametitle{授業の流れ}
  \tableofcontents
\end{frame}

\section{不定詞}
\subsection{不定詞とは}
%%%%%%%%%%%%%%%%%%%%%%%%%%%%%%%%%%%%%%%%%%%%%
\begin{frame}[plain]{おさらい}
\Large

Small cats jump quickly.\\
	\visible<2->{{\small \Circled{ 形 }\hspace{25pt}\Circled{ 名 }\hspace{15pt}\Circled{ 動 }\hspace{24pt}\Circled{ 副 }}}

\end{frame}
%%%%%%%%%%%%%%%%%%%%%%%%%%%%%%%%%%%%%%%%%%%%%
\begin{frame}[plain]{不定詞とは}
\large
\visible<2->{動詞とは本来、主語と組み合わせて使うもの
\mbox{}\hfill{}\myEmph[2-]{Maroon}{I} \myEmph[2-]{Maroon}{play} the piano and \myEmph[2-]{Maroon}{she} \myEmph[2-]{Maroon}{sings}.}

\visible<3->{だが、特別に

\vfill

\myAnch{v}{white}{動詞}}%
\hspace{30pt}%
\begin{tabular}{ll}
 \visible<3->{\myAnch{n}{white}{名詞のはたらき}: }&\visible<5->{\fbox{To get up early} is important.}\\
 \visible<3->{\myAnch{adj}{white}{形容詞のはたらき}: }&\visible<6->{I want something \fbox{to drink}.}\\
 \visible<3->{\myAnch{adv}{white}{副詞のはたらき}: }&\visible<7->{She studied hard \fbox{to pass the exam}}.
\end{tabular}

\visible<3->%
{\begin{tikzpicture}[remember picture, overlay]
 \visible<3->{\draw[thick, Maroon, ->] (v.east) to[out=0, in=180] (n.west);} 
 \visible<3->{\draw[thick, Maroon, ->] (v.east) to[out=0, in=180] (adj.west);} 
 \visible<3->{\draw[thick, Maroon, ->] (v.east) to[out=0, in=180] (adv.west);} 
\end{tikzpicture}}

\vspace{-5pt}

\mbox{}\hfill%
\visible<8->{\begin{minipage}{.35\textwidth}
\visible<9->{不定詞}:\Circled[fill color = white]{\,$\text{to} + \text{動詞の原形}$\,}
\end{minipage}}

\hfill{\scriptsize \myaudio{./audio/031_infinitive_intro_01.mp3}}
\end{frame}
%%%%%%%%%%%%%%%%%%%%%%%%
\begin{frame}[plain]{不定詞}
\large 

\begin{enumerate}
 \item \begin{enumerate}
	\item<1-> I like \fbox{~~~X~~~}\,.
	\item<2-> I like cookies.
	\item<3-> I like \alt<4->{\fbox{to make cookies}}{to make cookies}\,.
       \end{enumerate}
 \item \begin{enumerate}
	\item<5-> He likes cookies.
	\item<6-> He likes \alt<7->{\fbox{to make cookies}}{to make cookies}\,.
       \end{enumerate}
 \item \begin{enumerate}
	\item<8-> She liked cookies.
	\item<9-> She liked \alt<10->{\fbox{to make cookies}}{to make cookies}\,.
       \end{enumerate} 
\end{enumerate}

\visible<11->{%
\begin{exampleblock}{Topics for Today ---名詞的用法---}
\begin{itemize}\small
 \item<11-> 主語と組み合わせる動詞\,(\,I like \ldots{}, He likes \ldots{}, She liked \ldots\,)とはちがって、\\
いつでも\,\Circled[fill color= white]{\,$\text{to} + \text{動詞の原形}$\,}\\
 \end{itemize}
     \end{exampleblock}
}
\mbox{}\hfill{\scriptsize \myaudio{./audio/031_infinitive_intro_02.mp3}}
\end{frame}
\end{document}
