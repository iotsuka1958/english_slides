\documentclass[aspectratio=169,xcolor={dvipsnames,table}]{beamer}
\usepackage[no-math,deluxe,haranoaji]{luatexja-preset}
\renewcommand{\kanjifamilydefault}{\gtdefault}
\renewcommand{\emph}[1]{{\upshape\bfseries #1}}
\usetheme{metropolis}
\metroset{block=fill}
\setbeamertemplate{navigation symbols}{}
\setbeamertemplate{blocks}[rounded][shadow=false]
\usecolortheme[rgb={0.7,0.2,0.2}]{structure}
%%%%%%%%%%%%%%%%%%%%%%%%%%
%% Change alert block colors
%%% 1- Block title (background and text)
\setbeamercolor{block title alerted}{fg=mDarkTeal, bg=mLightBrown!45!yellow!45}
\setbeamercolor{block title example}{fg=magenta!10!black, bg=mLightGreen!70}
%%% 2- Block body (background)
\setbeamercolor{block body alerted}{bg=mLightBrown!25}
\setbeamercolor{block body example}{bg=mLightGreen!15}
%%%%%%%%%%%%%%%%%%%%%%%%%%%
%%%%%%%%%%%%%%%%%%%%%%%%%%%
%% さまざまなアイコン
%%%%%%%%%%%%%%%%%%%%%%%%%%%
%\usepackage{fontawesome}
\usepackage{fontawesome5}
\usepackage{figchild}
\usepackage{twemojis}
\usepackage{utfsym}
\usepackage{bclogo}
\usepackage{marvosym}
\usepackage{fontmfizz}
\usepackage{pifont}
\usepackage{phaistos}
\usepackage{worldflags}
\usepackage{jigsaw}
\usepackage{tikzlings}
\usepackage{tikzducks}
\usepackage{scsnowman}
\usepackage{epsdice}
\usepackage{halloweenmath}
\usepackage{svrsymbols}
\usepackage{countriesofeurope}
\usepackage{tipa}
\usepackage{manfnt}
%%%%%%%%%%%%%%%%%%%%%%%%%%%
\usepackage{tikz}
\usetikzlibrary{calc,patterns,decorations.pathmorphing,backgrounds}
\usepackage{tcolorbox}
\usepackage{tikzpeople}
\usepackage{circledsteps}
\usepackage{xcolor}
\usepackage{amsmath}
\usepackage{booktabs}
\usepackage{chronology}
\usepackage{signchart}
%%%%%%%%%%%%%%%%%%%%%%%%%%%
%% 場合分け
%%%%%%%%%%%%%%%%%%%%%%%%%%%
\usepackage{cases}
%%%%%%%%%%%%%%%%%%%%%%%%%%
\usepackage{pdfpages}
%%%%%%%%%%%%%%%%%%%%%%%%%%%
%% 音声リンク表示
\newcommand{\myaudio}[1]{\href{#1}{\faVolumeUp}}
%%%%%%%%%%%%%%%%%%%%%%%%%%
%% \myAnch{<名前>}{<色>}{<テキスト>}
%% 指定のテキストを指定の色の四角枠で囲み, 指定の名前をもつTikZの
%% ノードとして出力する. 図には remember picture 属性を付けている
%% ので外部から参照可能である.
\newcommand*{\myAnch}[3]{%
  \tikz[remember picture,baseline=(#1.base)]
    \node[draw,rectangle,line width=1pt,#2] (#1) {\normalcolor #3};
}
%%%%%%%%%%%%%%%%%%%%%%%%%%
%% \myEmph コマンドの定義
%%%%%%%%%%%%%%%%%%%%%%%%%%
%\newcommand{\myEmph}[3]{%
%    \textbf<#1>{\color<#1>{#2}{#3}}%
%}
\usepackage{xparse} % xparseパッケージの読み込み
\NewDocumentCommand{\myEmph}{O{} m m}{%
    \def\argOne{#1}%
    \ifx\argOne\empty
        \textbf{\color{#2}{#3}}% オプション引数が省略された場合
    \else
        \textbf<#1>{\color<#1>{#2}{#3}}% オプション引数が指定された場合
    \fi
}
%%%%%%%%%%%%%%%%%%%%%%%%%%%
%%%%%%%%%%%%%%%%%%%%%%%%%%%
%% 文末の上昇イントネーション記号 \myRisingPitch
%% 通常のイントネーション \myDownwardPitch
%% https://note.com/dan_oyama/n/n8be58e8797b2
%%%%%%%%%%%%%%%%%%%%%%%%%%%
\newcommand{\myRisingPitch}{
\begin{tikzpicture}[scale=0.3,baseline=0.3]
\draw[->,>=stealth] (0,0) to[bend right=45] (1,1);
\end{tikzpicture}
}
\newcommand{\myDownwardPitch}{
\begin{tikzpicture}[scale=0.3,baseline=0.3]
\draw[->,>=stealth] (0,1) to[bend left=45] (1,0);
\end{tikzpicture}
}
%%%%%%%%%%%%%%%%%%%%%%%%%%%%
%\AtBeginSection[%
%]{%
%  \begin{frame}[plain]\frametitle{授業の流れ}
%     \tableofcontents[currentsection]
%   \end{frame}%
%}

%%%%%%%%%%%%%%%%%%%%%%%%%%%
\title{English is fun.}
\subtitle{John is as tall as his father.}
\author{}
\institute[]{}
\date[]

%%%%%%%%%%%%%%%%%%%%%%%%%%%%
%% TEXT
%%%%%%%%%%%%%%%%%%%%%%%%%%%%
\begin{document}

\begin{frame}[plain]
  \titlepage
\end{frame}

\section*{授業の流れ}
\begin{frame}[plain]
  \frametitle{授業の流れ}
  \tableofcontents
\end{frame}

\section{同じくらい~}
\subsection{as ~ as \ldots{}}
%%%%%%%%%%%%%%%%%%%%%%%%%%%%%%%%%%%%%%%%%%%%%
\begin{frame}[plain]{as ~ as \ldots{}}
 \Large

\begin{enumerate}
 \item \begin{enumerate}
	\item<1-> John is \myEmph[1-]{NavyBlue}{tall}.
	\item<2-> John is \myEmph[2-]{Maroon}{as} \myEmph[2-]{NavyBlue}{tall} \myEmph[2-]{Maroon}{as} his brother.\hfill{\textipa{/@z/} \textipa{/\'\ae z/ }}
       \end{enumerate}
 \item \begin{enumerate}
	\item<3-> Jennifer runs \myEmph[3-]{NavyBlue}{fast}. 
	\item<4-> Jennifer runs \myEmph[4-]{Maroon}{as} \myEmph[4-]{NavyBlue}{fast} \myEmph[4-]{Maroon}{as} Emily.
       \end{enumerate}
\end{enumerate}
%
\hfill{\scriptsize \myaudio{./audio/041_as_as_01.mp3}}
\vfill

\visible<5->{%
\begin{block}{Topics for Today}
\begin{itemize}\setbeamertemplate{items}[square]\small
 \item AとBの程度が同じとき($\text{A}=\text{B}$のとき)は\\
\mbox{}\hspace{120pt} $\text{\Circled[fill color=white]{A}\,\,\,\,\,\ldots\,\,\,\,\,\myEmph[5-]{Maroon}{as}\,\,}+\left\{\begin{tblr}{l}
	    \text{形容詞}\\
	    \text{副詞}
	 \end{tblr}\right\} + \text{\myEmph[5-]{Maroon}{as}\,\,\,\,\,\Circled[fill color=white]{B}}$
 \item 述語動詞はbe動詞のこともあれば一般動詞のこともあります
 \end{itemize}
     \end{block}
}
\end{frame}
%%%%%%%%%%%%%%%%%%%%%%%%%%%%%%%%%%%%%%%%%%%%%%
\begin{frame}[plain]{Exercises}
\small
日本語の意味になるよう(~~~~~~)内の語句を並べ替えましょう。なお $[ +1 ]$とある場合は不足している1語を補ってください。$[ -1 ]$とある場合は余計な1語が含まれています%
\hfill{\scriptsize \myaudio{./audio/041_as_as_02.mp3}}

\normalsize
 \begin{enumerate}
  \item {\small その国ではバスケットボールは野球と同じくらい人気がある。}\hfill{\scriptsize popular \textipa{/p\'ApjUl\textrhookschwa /} 人気がある}\\
	Basketball is ( popular / as / as ) baseball in the country.\hfill{\scriptsize country \textipa{/k\'\textturnv ntri/} 国}\\
	\hspace{80pt}\visible<2->{\myEmph[2-]{BurntOrange}{as popular as}}
  \item {\small ジョージはジョンと同じくらい有名だ。}\hfill{\scriptsize famous \textipa{/f\'eIm@s/} 有名な}\\
	George ( famous / as / is / as ) John.\\
	\hspace*{55pt}\visible<3->{\myEmph[3-]{BurntOrange}{is as famous as}}
  \item {\small ゴリラは象と同じくらいの速さで走る。}\\
	Gorillas ( run / is / fast / as / as ) elephants. $[-1]$\\
	\hspace{65pt}\visible<4->{\myEmph[4-]{BurntOrange}{run as fast as} ✕ is}
  \item {\small ジェームズはエリックと同じくらいギターをじょうずに弾いた。}\\
	James played the guitar ( as / well ) Eric. $[+1]$\\
	\hspace{125pt}\visible<5->{\myEmph[5-]{BurntOrange}{as well as}}
 \end{enumerate}
\end{frame}
%%%%%%%%%%%%%%%%%%%%%%%
\begin{frame}[plain]{not as ~ as\hspace{10pt} \visible<4->{/\hspace{10pt}not so ~ as}}
 \begin{enumerate}
  \item<1-> Tokyo Tower is \myEmph[1-]{Maroon}{not} \myEmph[1-]{NavyBlue}{as} high \myEmph[1-]{NavyBlue}{as} Tokyo Sky Tree.\hfill{\scriptsize high \textipa{/h\'aI/} 高い}
  \item<3-> Tokyo Tower is \myEmph[2-]{Maroon}{not} \myEmph[2-]{NavyBlue}{so} high \myEmph[2-]{NavyBlue}{as} Tokyo Sky Tree.
 \end{enumerate}
%
\hfill{\scriptsize \myaudio{./audio/041_as_as_03.mp3}}
\vfill

\visible<2->{%
\begin{block}{Topics for Today}
\begin{itemize}\setbeamertemplate{items}[square]\small
 \item<2-> $\text{\Circled[fill color=white]{A}\,\,\,\,\,\ldots\,\,\,\,\,\myEmph[2-]{Maroon}{not as}\,\,}+\left\{\begin{array}{l}
	    \text{形容詞}\\
	    \text{副詞}
	 \end{array}\right\} + \text{\myEmph[2-]{Maroon}{as}\,\,\,\,\,\Circled[fill color=white]{B}}$%
\,\,$\Longrightarrow$\,\,「AはBほど~でない」の意味
 \item<4-> 最初のasがsoになることがある\\
\hspace{100pt}%
$\text{\Circled[fill color=white]{A}\,\,\,\,\,\ldots\,\,\,\,\,\myEmph[4-]{Maroon}{not so}\,\,}+\left\{\begin{array}{l}
	    \text{形容詞}\\
	    \text{副詞}
	 \end{array}\right\} + \text{\myEmph[4-]{Maroon}{as}\,\,\,\,\,\Circled[fill color=white]{B}}$%

 \end{itemize}
     \end{block}
}
\end{frame}
%%%%%%%%%%%%%%%%%%%%%%%
\begin{frame}[plain]{Exercises}\small
日本語の意味になるよう(~~~~~~)内の語句を並べ替えましょう%
\hfill{\scriptsize \myaudio{./audio/041_as_as_04.mp3}}

\normalsize
 \begin{enumerate}
  \item {\small わたしはピーターほど忙しくありません。}%
\hfill{\scriptsize buzy \textipa{/b\'Izi/} 忙しい }\\
	I am ( busy / as / as / not )  Peter. \\
	\hspace{45pt}\visible<2->{\myEmph[2-]{BurntOrange}{not as busy as}}
  \item {\small その映画はその本ほどおもしろくない。}\hfill{\scriptsize interesting \textipa{/\'Intr@stiN/} おもしろい}\\
	The movie is ( as / so / not / interesting ) the book.\\
	\hspace{75pt}\visible<3->{\myEmph[3-]{BurntOrange}{not so interesting as}}
  \item {\small 彼女はジェームズほどピアノの演奏がうまくない。}\\
	She does not play the piano ( as / as / well ) James.\\
	\hspace{160pt}\visible<3->{\myEmph[3-]{BurntOrange}{as well as}}
 \end{enumerate}
\end{frame}
\end{document}
