\documentclass[aspectratio=169,xcolor={dvipsnames,table}]{beamer}
\usepackage[no-math,deluxe,haranoaji]{luatexja-preset}
\renewcommand{\kanjifamilydefault}{\gtdefault}
\renewcommand{\emph}[1]{{\upshape\bfseries #1}}
\usetheme{metropolis}
\metroset{block=fill}
\setbeamertemplate{navigation symbols}{}
\setbeamertemplate{blocks}[rounded][shadow=false]
\usecolortheme[rgb={0.7,0.2,0.2}]{structure}
%%%%%%%%%%%%%%%%%%%%%%%%%%
%% Change alert block colors
%%% 1- Block title (background and text)
\setbeamercolor{block title alerted}{fg=mDarkTeal, bg=mLightBrown!45!yellow!45}
\setbeamercolor{block title example}{fg=magenta!10!black, bg=mLightGreen!70}
%%% 2- Block body (background)
\setbeamercolor{block body alerted}{bg=mLightBrown!25}
\setbeamercolor{block body example}{bg=mLightGreen!15}
%%%%%%%%%%%%%%%%%%%%%%%%%%%
%%%%%%%%%%%%%%%%%%%%%%%%%%%
%% さまざまなアイコン
%%%%%%%%%%%%%%%%%%%%%%%%%%%
%\usepackage{fontawesome}
\usepackage{fontawesome5}
\usepackage{figchild}
\usepackage{twemojis}
\usepackage{utfsym}
\usepackage{bclogo}
\usepackage{marvosym}
\usepackage{fontmfizz}
\usepackage{pifont}
\usepackage{phaistos}
\usepackage{worldflags}
\usepackage{jigsaw}
\usepackage{tikzlings}
\usepackage{tikzducks}
\usepackage{scsnowman}
\usepackage{epsdice}
\usepackage{halloweenmath}
\usepackage{svrsymbols}
\usepackage{countriesofeurope}
\usepackage{tipa}
\usepackage{manfnt}
%%%%%%%%%%%%%%%%%%%%%%%%%%%
\usepackage{tikz}
\usetikzlibrary{calc,patterns,decorations.pathmorphing,backgrounds}
\usepackage{tcolorbox}
\usepackage{tikzpeople}
\usepackage{circledsteps}
\usepackage{xcolor}
\usepackage{amsmath}
\usepackage{booktabs}
\usepackage{chronology}
\usepackage{signchart}
%%%%%%%%%%%%%%%%%%%%%%%%%%%
%% 場合分け
%%%%%%%%%%%%%%%%%%%%%%%%%%%
\usepackage{cases}
%%%%%%%%%%%%%%%%%%%%%%%%%%
\usepackage{pdfpages}
%%%%%%%%%%%%%%%%%%%%%%%%%%%
%% 音声リンク表示
\newcommand{\myaudio}[1]{\href{#1}{\faVolumeUp}}
%%%%%%%%%%%%%%%%%%%%%%%%%%
%% \myAnch{<名前>}{<色>}{<テキスト>}
%% 指定のテキストを指定の色の四角枠で囲み, 指定の名前をもつTikZの
%% ノードとして出力する. 図には remember picture 属性を付けている
%% ので外部から参照可能である.
\newcommand*{\myAnch}[3]{%
  \tikz[remember picture,baseline=(#1.base)]
    \node[draw,rectangle,line width=1pt,#2] (#1) {\normalcolor #3};
}
%%%%%%%%%%%%%%%%%%%%%%%%%%
%% \myEmph コマンドの定義
%%%%%%%%%%%%%%%%%%%%%%%%%%
%\newcommand{\myEmph}[3]{%
%    \textbf<#1>{\color<#1>{#2}{#3}}%
%}
\usepackage{xparse} % xparseパッケージの読み込み
\NewDocumentCommand{\myEmph}{O{} m m}{%
    \def\argOne{#1}%
    \ifx\argOne\empty
        \textbf{\color{#2}{#3}}% オプション引数が省略された場合
    \else
        \textbf<#1>{\color<#1>{#2}{#3}}% オプション引数が指定された場合
    \fi
}
%%%%%%%%%%%%%%%%%%%%%%%%%%%
%%%%%%%%%%%%%%%%%%%%%%%%%%%
%% 文末の上昇イントネーション記号 \myRisingPitch
%% 通常のイントネーション \myDownwardPitch
%% https://note.com/dan_oyama/n/n8be58e8797b2
%%%%%%%%%%%%%%%%%%%%%%%%%%%
\newcommand{\myRisingPitch}{
\begin{tikzpicture}[scale=0.3,baseline=0.3]
\draw[->,>=stealth] (0,0) to[bend right=45] (1,1);
\end{tikzpicture}
}
\newcommand{\myDownwardPitch}{
\begin{tikzpicture}[scale=0.3,baseline=0.3]
\draw[->,>=stealth] (0,1) to[bend left=45] (1,0);
\end{tikzpicture}
}
%%%%%%%%%%%%%%%%%%%%%%%%%%%%
%\AtBeginSection[%
%]{%
%  \begin{frame}[plain]\frametitle{授業の流れ}
%     \tableofcontents[currentsection]
%   \end{frame}%
%}

%%%%%%%%%%%%%%%%%%%%%%%%%%%
\title{English is fun.}
\subtitle{Mt. Everest is the highest in the world.}
\author{}
\institute[]{}
\date[]

%%%%%%%%%%%%%%%%%%%%%%%%%%%%
%% TEXT
%%%%%%%%%%%%%%%%%%%%%%%%%%%%
\begin{document}

\begin{frame}[plain]
  \titlepage
\end{frame}

\section*{授業の流れ}
\begin{frame}[plain]
  \frametitle{授業の流れ}
  \tableofcontents
\end{frame}

\section{最上級}
\subsection{もっとも〜だ}
%%%%%%%%%%%%%%%%%%%%%%%%%%%%%%%%%%%%%%%%%%%%%
\begin{frame}[plain]{--est}
 \large

\begin{enumerate}
 \item<1-> The Nile River is \only<2->{\myEmph[2-]{Maroon}{the}} long\only<2->{\myEmph[2-]{Maroon}{est} in the world}.
 \item<1-> Mt. Fuji is \only<3->{\myEmph[3-]{Maroon}{the}} high\only<3->{\myEmph[3-]{Maroon}{est} in Japan}.
 \item<1-> Jennifer is \only<4->{\myEmph[4-]{Maroon}{the}} tall\only<4->{\myEmph[4-]{Maroon}{est} in her school}.
 \item<1-> Peter is \only<5->{\myEmph[5-]{Maroon}{the}} young\only<5->{\myEmph[5-]{Maroon}{est} of the three brothers}.
 \item<1-> George swims \only<6->{\myEmph[6-]{Maroon}{the}} fast\only<6->{\myEmph[6-]{Maroon}{est} of all the students}.
\end{enumerate}

\visible<7->{%
\begin{exampleblock}{Topics for Today}
\begin{itemize}\small
 \item 「もっとも〜だ」は\,\,\,\,\,$\text{the} +%
\Circled[fill color=white]{\,\,\,\,\,\left\{\begin{array}{l}
            形容詞\\
            副詞
         \end{array}\right\} + \text{\myEmph[5-]{Maroon}{est}\,\,\,\,}}$
\,\,\,\,\,で表します。「最上級」といいます
 \item \fcolorbox{white}{white}{the $+$ 最上級}のあとには\,\,$\left\{
\begin{array}{l}
 \text{in} + \text{範囲を表す語}\\
 \text{of} + \text{複数}
\end{array}
\right\}$が続くことがあります
 \end{itemize}

     \end{exampleblock}
}

\end{frame}
%%%%%%%%%%%%%%%%%%%%%%%%%%%%%%%%%%%%%%%%%%%%%%
\begin{frame}[plain]{Exercises}
日本語の意味になるよう空所に適当な語を選択肢から選んで補いましょう。\\必要に応じて変化させてください。

\begin{columns}[t]
 \begin{column}{.78\textwidth}
   \begin{enumerate}
  \item 私の祖父は家族で最年長です。\\
	My grandfather is the (~~\alt<2->{\myEmph[2-]{BurntOrange}{oldest}}{\phantom{oldest}}~~) in my family. 
    \item この教室は学校のなかでいちばん小さいです。\\
	This classroom is the (~~\alt<3->{\myEmph[3-]{BurntOrange}{smallest}}{\phantom{smallest}}~~) in our school. 
    \item このバッグはその店でいちばん安いです。\\
	The bag is the (~~\alt<4->{\myEmph[4-]{BurntOrange}{cheapest}}{\phantom{cheapest}}~~) in the store.
    \item エミリーが5人の生徒のなかでいちばん高くジャンプした。\\
	Emily jumped the (~~\alt<5->{\myEmph[5-]{BurntOrange}{highest}}{\phantom{highest}}~~) of the five students.

\end{enumerate}

 \end{column}
%%%%%%%%%%%%%%
\begin{column}{.2\textwidth}
 \begin{tcolorbox}
  old\\
  high\\
  small\\
  big\\
  cheap
 \end{tcolorbox}
\end{column}
\end{columns}
\end{frame}
%%%%%%%%%%%%%%%%%%%%%%%%%%%%%%%%%%%%%%%%%
 \begin{frame}[plain]{Exercises}
日本語の意味になるよう(~~~~~~)に適語を補いましょう
  \begin{enumerate}
   \item ナイル川は世界でいちばん長い。\\
	 The Nile River is (~~\alt<2->{the}{\phantom{the}}~~) (~~\alt<2->{longest}{\phantom{longest}}~~) (~~\alt<2->{in}{\phantom{in}}~~) the world.
   \item ジェニファーは学校でいちばん背が高い。\\
	 Jennifer is (~~\alt<3->{the}{\phantom{the}}~~) (~~\alt<3->{tallest}{\phantom{tallest}}~~) (~~\alt<2->{in}{\phantom{in}}~~) her school.
   \item ボブは3人の生徒の中でいちばんはやく走った。\\
	 Bob ran the (~~\alt<4->{fastest}{\phantom{fastest}}~~) (~~\alt<4->{of}{\phantom{of}}~~) the three students.
   \item ピーターはすべてのメンバーの中で最年少だった。\\
	 Peter was  (~~\alt<5->{the}{\phantom{the}}~~) (~~\alt<5->{youngest}{\phantom{youngest}}~~) (~~\alt<5->{of}{\phantom{in}}~~) all the members.
  \end{enumerate}
 \end{frame}
%%%%%%%%%%%%%%%%%%%%%%%%%%%%%%%%%%%%%%%%%
\begin{frame}[plain]{最上級のつくり方}
 \begin{enumerate}
  \item<1-> The park is \only<2->{\myEmph[2-]{Maroon}{the}} \alt<2->{larg\myEmph[2-]{Maroon}{est} in the city}{large}.
  \item<1-> The question is \only<3->{\myEmph[3-]{Maroon}{the}} \alt<3->{eas\myEmph[3-]{Maroon}{iest} of all}{easy}.
  \item<1-> The box was \only<4->{\myEmph[4-]{Maroon}{the}} \alt<4->{heav\myEmph[3-]{Maroon}{iest} of all}{heavy}.
  \item<1-> John is \only<5->{\myEmph[5-]{Maroon}{the}} \alt<5->{bus\myEmph[2-]{Maroon}{iest} in the hospital}{busy}.
  \item<1-> Jennifer got up \only<6->{\myEmph[6-]{Maroon}{the}} \alt<6->{earl\myEmph[2-]{Maroon}{iest} in her family}{early}.
  \item<1-> My dog is \only<7->{\myEmph[7-]{Maroon}{the}} \alt<7->{bi\myEmph[2-]{Maroon}{ggest} in our neighborhood}{big}.
 \end{enumerate}

\visible<8->{%
\begin{exampleblock}{Topics for Today}
{\small 原則は語尾に--est。
ただし}
\begin{itemize}\small
 \item 最後がeのときは--stだけつける\\\hfill{}large -- largest, wide -- widest, nice --nicest, fine --finest 
 \item 最後のyをiにして--estをつけるもの\\\hfill{}easy -- easiest, heavy -- heaviest, busy -- busiest, happy -- happiest, early -- earliest 
 \item 最後の文字を重ねて --erをつけるもの\hfill{}hot -- hottest, big -- biggest 

 \end{itemize}
     \end{exampleblock}
}

\end{frame}
%%%%%%%%%%%%%%%%%%%%%%%%%%%%%%%%%%%%%%%%%%%%
\begin{frame}[plain]{最上級のつくりかた}

\begin{columns}
%%%%%%%%%%%%%%%%%%
\begin{column}[T]{.3\textwidth}
最後がeのとき

\bigskip

   \begin{tblr}{colspec={ll},
% 表の最上と最下に太さ 0.08em の横罫線
hline{1,Z} = { 0.08em },
hline{2} = { 0.05em },
row{odd}={gray9},
row{1} = { halign = c, font = { \sffamily\bfseries }, bg = gray6, fg = white }
}
原級&最上級\\
large&\visible<2->{largest}\\
wide&\visible<3->{widest}\\
nice&\visible<4->{nicest}\\
fine&\visible<5->{finest}\\
   \end{tblr}
\end{column}
%%%%%%%%%%%%%%%%%%%%%%%%%%
\begin{column}[T]{.3\textwidth}
最後がyのとき

\bigskip

  \begin{tblr}{colspec={ll},
% 表の最上と最下に太さ 0.08em の横罫線
hline{1,Z} = { 0.08em },
hline{2} = { 0.05em },
row{odd}={gray9},
row{1} = { halign = c, font = { \sffamily\bfseries }, bg = gray6, fg = white }
}
原級&最上級\\
easy&\visible<6->{easiest}\\
heavy&\visible<7->{heaviest}\\
busy&\visible<8->{busiest}\\
happy&\visible<9->{happiest}\\
early&\visible<10->{earliest}\\
   \end{tblr}
\end{column}
%%%%%%%%%%%%%%%%%%%%%%
\begin{column}[T]{.3\textwidth}
最後の文字を重ねるもの

\bigskip

   \begin{tblr}{colspec={ll},
% 表の最上と最下に太さ 0.08em の横罫線
hline{1,Z} = { 0.08em },
hline{2} = { 0.05em },
row{odd}={gray9},
row{1} = { halign = c, font = { \sffamily\bfseries }, bg = gray6, fg = white }
}
原級&最上級\\
hot&\visible<11->{hottest}\\
big&\visible<12->{biggest}
   \end{tblr}
\end{column}
%%%%%%%%%%%%%%%%%%%%%%%%%%%%%%
\end{columns}

\vfill

\hfill\visible<13->{微妙な差異はありますが、語尾が--estであることにかわりありませんね}
\end{frame}
%%%%%%%%%%%%%%%%%%%%%%%%%%%%%
\begin{frame}[plain]{Exercises}
日本語の意味になるよう(~~~~~~)に適語を補いましょう
 \begin{enumerate}
  \item その象は動物園でいちばん大きい動物だ。\\
	The elephant is the (~~\alt<2->{\textcolor{BurntOrange}{largest}}{\phantom{largest}}~~) animal in the zoo. \visible<2->{または\textcolor{BurntOrange}{biggest}}
  \item その店では月曜日がいちばん忙しい日です。\\
	Monday is the (~~\alt<3->{\textcolor{BurntOrange}{busiest}}{\phantom{busiest}}~~) day at the store.
  \item デイブはメンバーの中でいちばん早く起きた。\\
	Dave got up the (~~\alt<4->{\textcolor{BurntOrange}{earliest}}{\phantom{earliest}}~~) of all the members.
  \item 砂漠(desert)は地球上でもっとも暑い場所です。\\
	The desert is the (~~\alt<5->{\textcolor{BurntOrange}{hottest}}{\phantom{hottest}}~~) place on Earth.
  \item わたしにとって秋がもっとも幸せな季節です。\\
	Fall  is the (~~\alt<6->{\textcolor{BurntOrange}{happiest}}{\phantom{happiest}}~~) season for me.
 \end{enumerate}
\end{frame}
\end{document}
