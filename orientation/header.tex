%% PDFメタデータの文字化け防止
% https://blog.miz-ar.info/2015/09/latex-hyperref-tips/
% https://tex.stackexchange.com/questions/24445/hyperref-lualatex-and-unicode-bookmarks-issue-garbled-page-numbers-in-ar-for-l
\hypersetup{%
  pdfencoding=auto
}

%% Fonts
\usefonttheme[onlymath]{serif}
\usepackage[T1]{fontenc}
\usepackage{textcomp}
%\usepackage{arev}
\usepackage[scale=1.0]{tgheros}  % San serif font
\usepackage[scaled]{beramono}    % Monospace font

%% Japanese font
\usepackage{luatexja-otf}
\usepackage[match,deluxe,expert,haranoaji,nfssonly]{luatexja-preset} % Notoフォント使用
\renewcommand{\kanjifamilydefault}{\gtdefault}

%%
\setbeamerfont{title}{size=\huge, series=\bfseries}
\setbeamerfont{frametitle}{size=\Large, series=\bfseries}

%% https://tex.stackexchange.com/questions/62202/change-background-colour-of-verbatim-environment
\let\oldv\verbatim
\let\oldendv\endverbatim
\def\verbatim{\par\setbox0\vbox\bgroup\oldv}
\def\endverbatim{\oldendv\egroup\fboxsep0pt \noindent\colorbox[gray]{0.95}{\usebox0}\par}

%% https://stackoverflow.com/questions/38323331/code-chunk-font-size-in-beamer-with-knitr-and-latex
%% change fontsize of R code
\let\oldShaded\Shaded
\let\endoldShaded\endShaded 
\renewenvironment{Shaded}{\footnotesize\oldShaded}{\endoldShaded}

\usepackage{listings}
\lstset{%
  frame = shadow,
  backgroundcolor = {\color[gray]{.95}},
  basicstyle = {\small\ttfamily},
  breaklines = true,
  upquote = true
}
%%%%%%%%%%%%%%
\usepackage{fontawesome}
%%%%%%%%%%%%%%%
\usepackage{booktabs}
\usepackage{tikz}
\usepackage{pxpgfmark} % remember picture を可能にする
\usetikzlibrary{decorations,backgrounds,decorations.pathmorphing, shapes,positioning,fit,calc,spy}
\newcommand*\myCrossedOut[2]{%
  \tikz[baseline=(T.base)]
    \node[draw=#1, ultra thick, shape=cross out, decorate,
      inner sep=2pt, outer sep=0pt,
      decoration={random steps, segment length=2pt, amplitude=0.4pt}]
      (T) {#2};}
%%%%%%%%%%%%%%%%%%%%%%%%%%%%
\usetikzlibrary{decorations.text}     
%%%%%%%%%%%%%%%%%%%%%%%%%
\definecolor{softblue}{RGB}{87, 117, 144}
\definecolor{gentlegreen}{RGB}{125, 158, 128}
\definecolor{lightpurple}{RGB}{156, 134, 170}
\definecolor{warmorange}{RGB}{202, 137, 98}
\definecolor{gentlered}{RGB}{169, 105, 102}
\definecolor{softred}{RGB}{169, 0, 0}
\definecolor{darkred}{RGB}{207, 89, 89}      % ダークレッド
\definecolor{brightorange}{RGB}{235, 116, 70}     % ブライトオレンジ
\definecolor{lightorange}{RGB}{242, 150, 90}     % ライトオレンジ
\definecolor{gold}{RGB}{255, 174, 66}     % ゴールド
\definecolor{coral}{RGB}{247, 126, 101}    % コーラル
\definecolor{mintgreen}{RGB}{124, 156, 124}  % ミントグリーン
\definecolor{khaki}{RGB}{153, 141, 116}  % カーキ
\definecolor{mocha}{RGB}{141, 124, 113}  % モカ
\definecolor{slateblue}{RGB}{124, 123, 156}  % スレートブルー
\definecolor{lightyellow}{RGB}{242, 207, 90}    % ライトイエロー
\definecolor{yellowgold}{RGB}{235, 189, 70}    % イエローゴールド
\definecolor{mediumyellow}{RGB}{224, 171, 46}    % メディアムイエロー
\definecolor{darkyellow}{RGB}{217, 153, 25}    % ダークイエロー
\definecolor{goldenyellow}{RGB}{204, 135, 15}    % ゴールデンイエロー
