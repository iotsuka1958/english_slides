% Options for packages loaded elsewhere
\PassOptionsToPackage{unicode}{hyperref}
\PassOptionsToPackage{hyphens}{url}
%
\documentclass[
  ignorenonframetext,
  aspectratio=169,
  xcolor=dvipsnames]{beamer}
\usepackage{pgfpages}
\setbeamertemplate{caption}[numbered]
\setbeamertemplate{caption label separator}{: }
\setbeamercolor{caption name}{fg=normal text.fg}
\beamertemplatenavigationsymbolsempty
% Prevent slide breaks in the middle of a paragraph
\widowpenalties 1 10000
\raggedbottom
\setbeamertemplate{part page}{
  \centering
  \begin{beamercolorbox}[sep=16pt,center]{part title}
    \usebeamerfont{part title}\insertpart\par
  \end{beamercolorbox}
}
\setbeamertemplate{section page}{
  \centering
  \begin{beamercolorbox}[sep=12pt,center]{part title}
    \usebeamerfont{section title}\insertsection\par
  \end{beamercolorbox}
}
\setbeamertemplate{subsection page}{
  \centering
  \begin{beamercolorbox}[sep=8pt,center]{part title}
    \usebeamerfont{subsection title}\insertsubsection\par
  \end{beamercolorbox}
}
\AtBeginPart{
  \frame{\partpage}
}
\AtBeginSection{
  \ifbibliography
  \else
    \frame{\sectionpage}
  \fi
}
\AtBeginSubsection{
  \frame{\subsectionpage}
}
\usepackage{amsmath,amssymb}
\usepackage{iftex}
\ifPDFTeX
  \usepackage[T1]{fontenc}
  \usepackage[utf8]{inputenc}
  \usepackage{textcomp} % provide euro and other symbols
\else % if luatex or xetex
  \usepackage{unicode-math} % this also loads fontspec
  \defaultfontfeatures{Scale=MatchLowercase}
  \defaultfontfeatures[\rmfamily]{Ligatures=TeX,Scale=1}
\fi
\usepackage{lmodern}
\usetheme[]{metropolis}
\ifPDFTeX\else
  % xetex/luatex font selection
\fi
% Use upquote if available, for straight quotes in verbatim environments
\IfFileExists{upquote.sty}{\usepackage{upquote}}{}
\IfFileExists{microtype.sty}{% use microtype if available
  \usepackage[]{microtype}
  \UseMicrotypeSet[protrusion]{basicmath} % disable protrusion for tt fonts
}{}
\makeatletter
\@ifundefined{KOMAClassName}{% if non-KOMA class
  \IfFileExists{parskip.sty}{%
    \usepackage{parskip}
  }{% else
    \setlength{\parindent}{0pt}
    \setlength{\parskip}{6pt plus 2pt minus 1pt}}
}{% if KOMA class
  \KOMAoptions{parskip=half}}
\makeatother
\usepackage{xcolor}
\newif\ifbibliography
\usepackage{color}
\usepackage{fancyvrb}
\newcommand{\VerbBar}{|}
\newcommand{\VERB}{\Verb[commandchars=\\\{\}]}
\DefineVerbatimEnvironment{Highlighting}{Verbatim}{commandchars=\\\{\}}
% Add ',fontsize=\small' for more characters per line
\usepackage{framed}
\definecolor{shadecolor}{RGB}{248,248,248}
\newenvironment{Shaded}{\begin{snugshade}}{\end{snugshade}}
\newcommand{\AlertTok}[1]{\textcolor[rgb]{0.94,0.16,0.16}{#1}}
\newcommand{\AnnotationTok}[1]{\textcolor[rgb]{0.56,0.35,0.01}{\textbf{\textit{#1}}}}
\newcommand{\AttributeTok}[1]{\textcolor[rgb]{0.13,0.29,0.53}{#1}}
\newcommand{\BaseNTok}[1]{\textcolor[rgb]{0.00,0.00,0.81}{#1}}
\newcommand{\BuiltInTok}[1]{#1}
\newcommand{\CharTok}[1]{\textcolor[rgb]{0.31,0.60,0.02}{#1}}
\newcommand{\CommentTok}[1]{\textcolor[rgb]{0.56,0.35,0.01}{\textit{#1}}}
\newcommand{\CommentVarTok}[1]{\textcolor[rgb]{0.56,0.35,0.01}{\textbf{\textit{#1}}}}
\newcommand{\ConstantTok}[1]{\textcolor[rgb]{0.56,0.35,0.01}{#1}}
\newcommand{\ControlFlowTok}[1]{\textcolor[rgb]{0.13,0.29,0.53}{\textbf{#1}}}
\newcommand{\DataTypeTok}[1]{\textcolor[rgb]{0.13,0.29,0.53}{#1}}
\newcommand{\DecValTok}[1]{\textcolor[rgb]{0.00,0.00,0.81}{#1}}
\newcommand{\DocumentationTok}[1]{\textcolor[rgb]{0.56,0.35,0.01}{\textbf{\textit{#1}}}}
\newcommand{\ErrorTok}[1]{\textcolor[rgb]{0.64,0.00,0.00}{\textbf{#1}}}
\newcommand{\ExtensionTok}[1]{#1}
\newcommand{\FloatTok}[1]{\textcolor[rgb]{0.00,0.00,0.81}{#1}}
\newcommand{\FunctionTok}[1]{\textcolor[rgb]{0.13,0.29,0.53}{\textbf{#1}}}
\newcommand{\ImportTok}[1]{#1}
\newcommand{\InformationTok}[1]{\textcolor[rgb]{0.56,0.35,0.01}{\textbf{\textit{#1}}}}
\newcommand{\KeywordTok}[1]{\textcolor[rgb]{0.13,0.29,0.53}{\textbf{#1}}}
\newcommand{\NormalTok}[1]{#1}
\newcommand{\OperatorTok}[1]{\textcolor[rgb]{0.81,0.36,0.00}{\textbf{#1}}}
\newcommand{\OtherTok}[1]{\textcolor[rgb]{0.56,0.35,0.01}{#1}}
\newcommand{\PreprocessorTok}[1]{\textcolor[rgb]{0.56,0.35,0.01}{\textit{#1}}}
\newcommand{\RegionMarkerTok}[1]{#1}
\newcommand{\SpecialCharTok}[1]{\textcolor[rgb]{0.81,0.36,0.00}{\textbf{#1}}}
\newcommand{\SpecialStringTok}[1]{\textcolor[rgb]{0.31,0.60,0.02}{#1}}
\newcommand{\StringTok}[1]{\textcolor[rgb]{0.31,0.60,0.02}{#1}}
\newcommand{\VariableTok}[1]{\textcolor[rgb]{0.00,0.00,0.00}{#1}}
\newcommand{\VerbatimStringTok}[1]{\textcolor[rgb]{0.31,0.60,0.02}{#1}}
\newcommand{\WarningTok}[1]{\textcolor[rgb]{0.56,0.35,0.01}{\textbf{\textit{#1}}}}
\usepackage{graphicx}
\makeatletter
\def\maxwidth{\ifdim\Gin@nat@width>\linewidth\linewidth\else\Gin@nat@width\fi}
\def\maxheight{\ifdim\Gin@nat@height>\textheight\textheight\else\Gin@nat@height\fi}
\makeatother
% Scale images if necessary, so that they will not overflow the page
% margins by default, and it is still possible to overwrite the defaults
% using explicit options in \includegraphics[width, height, ...]{}
\setkeys{Gin}{width=\maxwidth,height=\maxheight,keepaspectratio}
% Set default figure placement to htbp
\makeatletter
\def\fps@figure{htbp}
\makeatother
\setlength{\emergencystretch}{3em} % prevent overfull lines
\providecommand{\tightlist}{%
  \setlength{\itemsep}{0pt}\setlength{\parskip}{0pt}}
\setcounter{secnumdepth}{-\maxdimen} % remove section numbering
%% PDFメタデータの文字化け防止
% https://blog.miz-ar.info/2015/09/latex-hyperref-tips/
% https://tex.stackexchange.com/questions/24445/hyperref-lualatex-and-unicode-bookmarks-issue-garbled-page-numbers-in-ar-for-l
\hypersetup{%
  pdfencoding=auto
}

%% Fonts
\usefonttheme[onlymath]{serif}
\usepackage[T1]{fontenc}
\usepackage{textcomp}
%\usepackage{arev}
\usepackage[scale=1.0]{tgheros}  % San serif font
\usepackage[scaled]{beramono}    % Monospace font

%% Japanese font
\usepackage{luatexja-otf}
\usepackage[match,deluxe,expert,haranoaji,nfssonly]{luatexja-preset} % Notoフォント使用
\renewcommand{\kanjifamilydefault}{\gtdefault}

%%
\setbeamerfont{title}{size=\huge, series=\bfseries}
\setbeamerfont{frametitle}{size=\Large, series=\bfseries}

%% https://tex.stackexchange.com/questions/62202/change-background-colour-of-verbatim-environment
\let\oldv\verbatim
\let\oldendv\endverbatim
\def\verbatim{\par\setbox0\vbox\bgroup\oldv}
\def\endverbatim{\oldendv\egroup\fboxsep0pt \noindent\colorbox[gray]{0.95}{\usebox0}\par}

%% https://stackoverflow.com/questions/38323331/code-chunk-font-size-in-beamer-with-knitr-and-latex
%% change fontsize of R code
\let\oldShaded\Shaded
\let\endoldShaded\endShaded 
\renewenvironment{Shaded}{\footnotesize\oldShaded}{\endoldShaded}

\usepackage{listings}
\lstset{%
  frame = shadow,
  backgroundcolor = {\color[gray]{.95}},
  basicstyle = {\small\ttfamily},
  breaklines = true,
  upquote = true
}
%%%%%%%%%%%%%%
\usepackage{fontawesome}
%%%%%%%%%%%%%%%
\usepackage{booktabs}
\usepackage{tikz}
\usepackage{pxpgfmark} % remember picture を可能にする
\usetikzlibrary{decorations,backgrounds,decorations.pathmorphing, shapes,positioning,fit,calc,spy}
\newcommand*\myCrossedOut[2]{%
  \tikz[baseline=(T.base)]
    \node[draw=#1, ultra thick, shape=cross out, decorate,
      inner sep=2pt, outer sep=0pt,
      decoration={random steps, segment length=2pt, amplitude=0.4pt}]
      (T) {#2};}
%%%%%%%%%%%%%%%%%%%%%%%%%%%%
\usetikzlibrary{decorations.text}     
%%%%%%%%%%%%%%%%%%%%%%%%%
\definecolor{softblue}{RGB}{87, 117, 144}
\definecolor{gentlegreen}{RGB}{125, 158, 128}
\definecolor{lightpurple}{RGB}{156, 134, 170}
\definecolor{warmorange}{RGB}{202, 137, 98}
\definecolor{gentlered}{RGB}{169, 105, 102}
\definecolor{softred}{RGB}{169, 0, 0}
\definecolor{darkred}{RGB}{207, 89, 89}      % ダークレッド
\definecolor{brightorange}{RGB}{235, 116, 70}     % ブライトオレンジ
\definecolor{lightorange}{RGB}{242, 150, 90}     % ライトオレンジ
\definecolor{gold}{RGB}{255, 174, 66}     % ゴールド
\definecolor{coral}{RGB}{247, 126, 101}    % コーラル
\definecolor{mintgreen}{RGB}{124, 156, 124}  % ミントグリーン
\definecolor{khaki}{RGB}{153, 141, 116}  % カーキ
\definecolor{mocha}{RGB}{141, 124, 113}  % モカ
\definecolor{slateblue}{RGB}{124, 123, 156}  % スレートブルー
\definecolor{lightyellow}{RGB}{242, 207, 90}    % ライトイエロー
\definecolor{yellowgold}{RGB}{235, 189, 70}    % イエローゴールド
\definecolor{mediumyellow}{RGB}{224, 171, 46}    % メディアムイエロー
\definecolor{darkyellow}{RGB}{217, 153, 25}    % ダークイエロー
\definecolor{goldenyellow}{RGB}{204, 135, 15}    % ゴールデンイエロー
%%%%%%%%%%%%%%%%%
\usepackage{array}
%% セルの内容を縦中央にそろえる
% カスタム列指定子を定義
\newcolumntype{C}[1]{>{\centering\arraybackslash}m{#1}}
\newcolumntype{L}[1]{>{\raggedright\arraybackslash}m{#1}}
%%%%%%%%%%%%%%%%%
\usepackage{pxrubrica}
\ifLuaTeX
  \usepackage{selnolig}  % disable illegal ligatures
\fi
\usepackage{bookmark}
\IfFileExists{xurl.sty}{\usepackage{xurl}}{} % add URL line breaks if available
\urlstyle{same}
\hypersetup{
  pdftitle={English is fun},
  hidelinks,
  pdfcreator={LaTeX via pandoc}}

\title{English is fun}
\subtitle{orientation}
\author{}
\date{\vspace{-2.5em}2025}

\begin{document}
\frame{\titlepage}

\begin{frame}{}
\phantomsection\label{section}
\thispagestyle{empty}
\Large

\raggedright

予定の時刻になったらはじまります

\textbullet  音声を流しています

\textbullet  聞こえていますか 

\vfill

\raggedleft

The lesson will begin at the scheduled time.

\vspace{-6pt}

We are playing audio.

\vspace{-6pt}

Can you hear it?
\end{frame}

\begin{frame}{}
\phantomsection\label{section-1}
\thispagestyle{empty}
\titlepage
\end{frame}

\begin{frame}{きょうの予定}
\phantomsection\label{ux304dux3087ux3046ux306eux4e88ux5b9a}
\thispagestyle{empty}
\LARGE

\begin{tabular}{rl}
1&安心して学ぶ場にするために\\
2&授業の進め方や内容などについて
\end{tabular}
\end{frame}

\begin{frame}{自己紹介}
\phantomsection\label{ux81eaux5df1ux7d39ux4ecb}
\thispagestyle{empty}
\Large

\pause

大塚一朗(Otsuka Ichiro)

\pause

\begin{itemize}[<+->]
\tightlist
\item
  千葉LOTTE Marinesを応援しています
\item
  ミステリーが好き

  \begin{itemize}[<+->]
  \tightlist
  \item
    本・ 映画・TVドラマ
  \end{itemize}
\item
  音楽はだいたいなんでも好き

  \begin{itemize}[<+->]
  \tightlist
  \item
    BachとThe Beach Boysをよく聴きます ← 首尾一貫していない
  \end{itemize}
\item
  プログラミングが好き。しろうとレベルですが、楽しい
\item
  やっぱりネコが好き \pause
\end{itemize}

いっしょに英語の勉強を\textcolor{Orange}{\bfseries 楽しく}していきましょう
\end{frame}

\begin{frame}{この時間は}
\phantomsection\label{ux3053ux306eux6642ux9593ux306f}
\Huge

\centering

English is fun.
\end{frame}

\section{みんなが安心して楽しく学べるように}\label{ux307fux3093ux306aux304cux5b89ux5fc3ux3057ux3066ux697dux3057ux304fux5b66ux3079ux308bux3088ux3046ux306b}

\begin{frame}{約束}
\phantomsection\label{ux7d04ux675f}
\LARGE

みんなが\textcolor{Maroon}{安心して楽しく}学べるように

\pause

約束をしましょう
\end{frame}

\begin{frame}{約束}
\phantomsection\label{ux7d04ux675f-1}
\Large

\begin{tabular}{ll}
約束1& 個人情報は書き込みません\\\pause
約束2& 他の人が傷つくような発言はしません\\\pause
約束3& ログインに必要な情報は他の人には教えません\\\pause
約束4& パスワードはたいせつに管理します\\\pause
約束5& 教材や動画等は、許可なく録画・録音したり\\
&      SNSなどにのせません
\end{tabular}
\end{frame}

\begin{frame}{安心して学ぶための約束1}
\phantomsection\label{ux5b89ux5fc3ux3057ux3066ux5b66ux3076ux305fux3081ux306eux7d04ux675f1}
\LARGE

\pause

「個人情報」って、耳にしたことはありますか?

\pause

「個人情報」ってなんでしょうか?
\end{frame}

\begin{frame}{個人情報とは}
\phantomsection\label{ux500bux4ebaux60c5ux5831ux3068ux306f}
\LARGE

生存する個人に関する情報で、

特定の個人を識別できる情報

\vfill
\pause

うーん、わかったようなわからないような
\end{frame}

\begin{frame}{具体的には}
\phantomsection\label{ux5177ux4f53ux7684ux306bux306f}
\LARGE

個人情報って、具体的にはなんなの?
\end{frame}

\begin{frame}{具体的には}
\phantomsection\label{ux5177ux4f53ux7684ux306bux306f-1}
\begin{tabular}
{C{.175\textwidth}  <{\onslide<2->} C{.18\textwidth}  <{\onslide<3->} C{.27\textwidth}  <{\onslide<4->} C{.27\textwidth}  <{\onslide}}
\noalign{\hrule height 1pt}
氏名&顔写真&
\begin{tabular}{@{}c@{}}
住所\\
{\scriptsize 氏名と組み合わせた場合}
\end{tabular}&
\begin{tabular}{@{}c@{}}
生年月日\\
{\scriptsize 氏名と組み合わせた場合}
\end{tabular}\\
\midrule
\ruby{穂毛}{ほ|げ} ホゲ\ruby{雄}{お}&\includegraphics[width=.2\textwidth]{../images/mr_hogehoge.pdf}&
\begin{tabular}{@{}c@{}}
千葉県○○市○○町\\
穂毛 ホゲ雄
\end{tabular}&
\begin{tabular}{@{}c@{}}
2010年○月○日\\
穂毛 ホゲ雄
\end{tabular}\\
\bottomrule
\end{tabular}
\end{frame}

\begin{frame}{安心して学ぶための約束1}
\phantomsection\label{ux5b89ux5fc3ux3057ux3066ux5b66ux3076ux305fux3081ux306eux7d04ux675f1-1}
\LARGE

個人情報は書き込まないようにしましょう

\pause

\bigskip

自分やともだちの名前、学校名なども書き込みません
\end{frame}

\begin{frame}{安心して学ぶための約束2}
\phantomsection\label{ux5b89ux5fc3ux3057ux3066ux5b66ux3076ux305fux3081ux306eux7d04ux675f2}
\LARGE

他の人が傷つくようなことはいいません
\end{frame}

\begin{frame}{安心して学ぶための約束3}
\phantomsection\label{ux5b89ux5fc3ux3057ux3066ux5b66ux3076ux305fux3081ux306eux7d04ux675f3}
\LARGE

ログインに必要な情報

\pause
\bigskip

\raggedleft

たいせつに管理しましょう
\end{frame}

\begin{frame}{安心して学ぶための約束4}
\phantomsection\label{ux5b89ux5fc3ux3057ux3066ux5b66ux3076ux305fux3081ux306eux7d04ux675f4}
\LARGE

パスワード

\pause

\begin{itemize}[<+->]
\tightlist
\item
  安全なパスワードにしましょう
\item
  他の人には教えないようにしましょう
\item
  自分で忘れないようにしましょう
\end{itemize}
\end{frame}

\begin{frame}{安心して学ぶための約束5}
\phantomsection\label{ux5b89ux5fc3ux3057ux3066ux5b66ux3076ux305fux3081ux306eux7d04ux675f5}
\LARGE

教材や動画等は許可なく録画・録音したり、

SNSなどにのせないようにしましょう
\end{frame}

\begin{frame}{}
\phantomsection\label{section-2}
\Large

\begin{tabular}{ll}
約束1& 個人情報は書き込みません\\
約束2& 他の人が傷つくような発言はしません\\
約束3& ログインに必要な情報は他の人には教えません\\
約束4& パスワードはたいせつに管理します\\
約束5& 教材や動画等は許可なく録画・録音したり\\
&      SNSなどにのせません
\end{tabular}

\vfill

これらの\textcolor{NavyBlue}{約束を守って}、
\textcolor{Maroon}{安心して楽しく}勉強しましょう!
\end{frame}

\section{授業について}\label{ux6388ux696dux306bux3064ux3044ux3066}

\begin{frame}{授業について}
\phantomsection\label{ux6388ux696dux306bux3064ux3044ux3066-1}
\LARGE

\begin{itemize}[<+->]
\tightlist
\item
  授業の進め方について
\item
  基礎基本について
\item
  準備するもの
\item
  予習・復習について
\item
  英語の音声について
\end{itemize}
\end{frame}

\begin{frame}{授業の進め方}
\phantomsection\label{ux6388ux696dux306eux9032ux3081ux65b9}
\LARGE

スライドで進めていきます

\pause
\vfill

\raggedleft

誰かを\textcolor{Maroon}{指すことはありません}

\pause

安心して授業に参加してください
\end{frame}

\begin{frame}{授業の内容}
\phantomsection\label{ux6388ux696dux306eux5185ux5bb9}
\LARGE

\textcolor{Maroon}{基礎基本をだいじに}します

\vfill

\raggedleft

やさしいことからはじめて着実に
\end{frame}

\begin{frame}{準備するもの}
\phantomsection\label{ux6e96ux5099ux3059ux308bux3082ux306e}
\LARGE
\begin{columns}
\begin{column}{.25\textwidth}
  \begin{itemize}
    \item ノート
    \item 筆記具
  \end{itemize}
\end{column}
\begin{column}{.7\textwidth}
\raggedleft
\includegraphics[width=.95\textwidth]{../images/notebook.jpg}
\end{column}
\end{columns}
\tiny
\raggedleft

``With your pen and notebook, you blow me away\ldots{}'' by lethaargic
is licensed under CC BY 2.0. To view a copy of this license, visit
\url{https://creativecommons.org/licenses/by/2.0/?ref=openverse}.
\end{frame}

\begin{frame}{予習・復習について}
\phantomsection\label{ux4e88ux7fd2ux5fa9ux7fd2ux306bux3064ux3044ux3066}
\LARGE

予習\pause → 必要ありません

\pause

\bigskip

復習\pause → 効果的です

\phantom{    }余裕があればくらいの軽い気持ちで
\end{frame}

\begin{frame}{音声をだいじにしよう}
\phantomsection\label{ux97f3ux58f0ux3092ux3060ux3044ux3058ux306bux3057ux3088ux3046}
\LARGE

英語の音声に親しみましょう

\begin{itemize}[<+->]
\tightlist
\item
  \href{./audio/listen.mp3}{Please listen carefully.}
\item
  \href{./audio/repeat.mp3}{Please repeat after me.}
\end{itemize}
\end{frame}

\begin{frame}{授業について(まとめ)}
\phantomsection\label{ux6388ux696dux306bux3064ux3044ux3066ux307eux3068ux3081}
\Large

\begin{itemize}
\item 授業の進め方
\item 基礎基本をだいじに
\item 筆記用具・ノート
\item 予習・復習について
\item 英語の音声に親しみましょう
      \begin{itemize}
         \item  \href{./audio/listen.mp3}{Please listen carefully.}
         \item  \href{./audio/repeat.mp3}{Please repeat after me.}
      \end{itemize}
\end{itemize}
\end{frame}

\begin{frame}{きょうのまとめ}
\phantomsection\label{ux304dux3087ux3046ux306eux307eux3068ux3081}
\LARGE

\begin{itemize}
\item  安心して学ぶ場にするための約束
\item  授業について
\end{itemize}
\end{frame}

\section{See you next time.}\label{see-you-next-time.}

\begin{frame}{}
\phantomsection\label{section-3}
\end{frame}

\begin{frame}{Hello, World!}
\phantomsection\label{hello-world}
\Large

Hello, World!

\includegraphics[width=0.9\linewidth]{orientation_files/figure-beamer/unnamed-chunk-3-1}
\end{frame}

\begin{frame}{Japan is here}
\phantomsection\label{japan-is-here}
\Large

\textcolor{orange}{\bfseries Japan} is here.

\includegraphics[width=0.9\linewidth]{orientation_files/figure-beamer/unnamed-chunk-4-1}
\end{frame}

\begin{frame}{世界のことば}
\phantomsection\label{ux4e16ux754cux306eux3053ux3068ux3070}
\Large

日本では日本語

\pause

世界にはどんなことばがありますか
\end{frame}

\begin{frame}{日本語以外のことばをあげてみよう}
\phantomsection\label{ux65e5ux672cux8a9eux4ee5ux5916ux306eux3053ux3068ux3070ux3092ux3042ux3052ux3066ux307fux3088ux3046}
\Large

\begin{itemize}[<+->]
\tightlist
\item
  日本語
\item
  英語
\item
  中国語
\item
  韓国語
\item
  フランス語
\item
  ドイツ語
\item
  スペイン語
\item
  アラビア語 \ldots
\end{itemize}

\pause

ほかにもたくさんあります
\end{frame}

\begin{frame}{Quiz}
\phantomsection\label{quiz}
\Large

ところで、世界にはいくつくらいことばがあるの?

\pause

\vfill

\centering
\begin{tabular}{lr}
(A) &7\\\pause
(B) &70\\\pause
(C) &700\\\pause
(D) &7,000
\end{tabular}

\vfill
\end{frame}

\begin{frame}{Answer}
\phantomsection\label{answer}
\Huge
\vfill

\centering
7,164

\vfill

\raggedleft
\scriptsize

\url{https://www.ethnologue.com/insights/how-many-languages/による}
\end{frame}

\begin{frame}{びっくり}
\phantomsection\label{ux3073ux3063ux304fux308a}
\Large

\begin{columns}
    \begin{column}{0.475\textwidth}
      ええー、そんなにあるの
    \end{column}
    \begin{column}{0.475\textwidth}
      \IfFileExists{surprised.png}{\includegraphics[width=.6\textwidth]{surprised.png}}{\relax}
            {\tiny Designed by Wannapik}
      \end{column}
  \end{columns}
\end{frame}

\begin{frame}{そのうちのひとつ}
\phantomsection\label{ux305dux306eux3046ux3061ux306eux3072ux3068ux3064}
\Large

ということは\pause

日本語は\Huge \phantom{0}\(\frac{1}{7,164}\)\phantom{0}\Large ということですね

\vfill
\pause

でもそれをいったら、\pause 英語もフランス語もドイツ語も\pause
スペイン語もアラビア語もヒンディー語も\pause 中国語も韓国語も、\pause
みーんな\Huge\phantom{0}\(\frac{1}{7,164}\)\phantom{0}\Large ということです

\vfill
\pause

どの言語が優れているとかいうようなことはまったくありません
\end{frame}

\begin{frame}{そのことばを話す人の数}
\phantomsection\label{ux305dux306eux3053ux3068ux3070ux3092ux8a71ux3059ux4ebaux306eux6570}
\Large

どのことばも7,164あることばのひとつ、\pause
つまり\Huge\phantom{0}\(\frac{1}{7,164}\)\phantom{0}

\Large
\pause

とはいっても、
それぞれのことばを話している人の数も同じということになるでしょうか

\pause

そのことばを話す人の数がおおいことばもあれば、
話す人があまりいないことばもあるはずです。
\end{frame}

\begin{frame}{Quiz}
\phantomsection\label{quiz-1}
\Large

では、ここでもうひとつの問題です。
つぎの4つの言語で、話す人の数がおおい順に
並べると、どうなるでしょうか\pause

ただし生まれてはじめて身につけたことばという条件で

\pause

\begin{itemize}[<+->]
\tightlist
\item
  中国語(Mandarin Chinese)
\item
  スペイン語(Spanish)
\item
  英語(English)
\item
  ヒンディー語(Hindi)
\end{itemize}
\end{frame}

\begin{frame}{生まれて初めて身につけたことば}
\phantomsection\label{ux751fux307eux308cux3066ux521dux3081ux3066ux8eabux306bux3064ux3051ux305fux3053ux3068ux3070}
\includegraphics[height=0.95\textheight]{orientation_files/figure-beamer/unnamed-chunk-5-1}
\end{frame}

\begin{frame}{第一言語として話す人の数がおおいことば}
\phantomsection\label{ux7b2cux4e00ux8a00ux8a9eux3068ux3057ux3066ux8a71ux3059ux4ebaux306eux6570ux304cux304aux304aux3044ux3053ux3068ux3070}
\raggedleft

(単位: 100万人)

\vfill

\Large
\centering

\begin{tabular}{lr}
\toprule
ことば & 話者数\\
\midrule
Mandarin Chinese & 940\\
Spanish & 485\\
English & 380\\
Hindi & 345\\
\bottomrule
\end{tabular}

\vfill

\raggedleft
\scriptsize

\url{https://www.ethnologue.com/insights/most-spoken-language/による}
\end{frame}

\begin{frame}{複数のことばを話す人もいますよね}
\phantomsection\label{ux8907ux6570ux306eux3053ux3068ux3070ux3092ux8a71ux3059ux4ebaux3082ux3044ux307eux3059ux3088ux306d}
\Large

第一言語はAということばだけれど、

そのほかにも第二言語としてBということばを話す人がいます
\end{frame}

\begin{frame}{第一言語ではないけど、そのことば話します}
\phantomsection\label{ux7b2cux4e00ux8a00ux8a9eux3067ux306fux306aux3044ux3051ux3069ux305dux306eux3053ux3068ux3070ux8a71ux3057ux307eux3059}
\Large

第一言語ではないけれど、そのことばを話します

そういう人をあわせると、 どうなるでしょうか
\end{frame}

\begin{frame}{第一言語として話す人の数はこうでしたね}
\phantomsection\label{ux7b2cux4e00ux8a00ux8a9eux3068ux3057ux3066ux8a71ux3059ux4ebaux306eux6570ux306fux3053ux3046ux3067ux3057ux305fux306d}
\includegraphics[height=0.95\textheight]{orientation_files/figure-beamer/unnamed-chunk-7-1}
\end{frame}

\begin{frame}{第一言語ではない人を合わせるとこうなります}
\phantomsection\label{ux7b2cux4e00ux8a00ux8a9eux3067ux306fux306aux3044ux4ebaux3092ux5408ux308fux305bux308bux3068ux3053ux3046ux306aux308aux307eux3059}
\includegraphics[height=0.95\textheight]{orientation_files/figure-beamer/unnamed-chunk-8-1}
\end{frame}

\begin{frame}{つまり}
\phantomsection\label{ux3064ux307eux308a}
\raggedleft

(単位: 100万人)

\vfill

\Large

\begin{tabular}{lrrr}
\toprule
ことば & 第一言語 & それ以外 & Total\\
\midrule
English & 380 & 1,080 & 1,460\\
Mandarin Chinese & 940 & 199 & 1,139\\
Hindi & 345 & 265 & 610\\
Spanish & 485 & 74 & 559\\
\bottomrule
\end{tabular}

\vfill

\scriptsize

\url{https://www.ethnologue.com/insights/most-spoken-language/による}
\end{frame}

\begin{frame}{話す人を全部あわせると}
\phantomsection\label{ux8a71ux3059ux4ebaux3092ux5168ux90e8ux3042ux308fux305bux308bux3068}
\Large

世界には7,164のことばがありますが、

話す人をぜんぶあわせると\pause 英語を話す人がいちばんおおい\pause
2位は中国語\pause

\centering
\vfill

\begin{tabular}{lr}\toprule
ことば&話者数\\
\midrule
英語&14億6000万人\\
中国語 &11億3900万人\\
\bottomrule
\end{tabular}
\end{frame}

\begin{frame}{空間的にはどうかしら}
\phantomsection\label{ux7a7aux9593ux7684ux306bux306fux3069ux3046ux304bux3057ux3089}
\Large

中国語と英語を話す人が

世界のどこに住んでいるか見てみましょう。
\end{frame}

\begin{frame}{Chinese}
\phantomsection\label{chinese}
\includegraphics{orientation_files/figure-beamer/unnamed-chunk-10-1.pdf}
\end{frame}

\begin{frame}{中国語を話す人}
\phantomsection\label{ux4e2dux56fdux8a9eux3092ux8a71ux3059ux4eba}
\Large

中国語を話す人は中国に集中していますね
\end{frame}

\begin{frame}{English}
\phantomsection\label{english}
\includegraphics{orientation_files/figure-beamer/unnamed-chunk-11-1.pdf}
\end{frame}

\begin{frame}{English}
\phantomsection\label{english-1}
\Large

英語を話す人は世界中にいます
\end{frame}

\begin{frame}{つまり}
\phantomsection\label{ux3064ux307eux308a-1}
\Large

中国語を話す人は、中国に集中しています\pause

いっぽう、英語は広く世界中で話されています
\end{frame}

\begin{frame}{わたしたちが学ぶ英語とは}
\phantomsection\label{ux308fux305fux3057ux305fux3061ux304cux5b66ux3076ux82f1ux8a9eux3068ux306f}
\Large

\begin{itemize}[<+->]
\tightlist
\item
  英語は話す人の数が\textcolor{ForestGreen}{いちばんおおい}
\item
  英語は\textcolor{ForestGreen}{広く世界中で}話されている
\end{itemize}

つまり\pause

\textcolor{Maroon}{英語は世界の共通語}といっていいだろう
\end{frame}

\begin{frame}{英語は世界の共通語}
\phantomsection\label{ux82f1ux8a9eux306fux4e16ux754cux306eux5171ux901aux8a9e}
\Large

さまざまな分野で使われています

\begin{itemize}[<+->]
\tightlist
\item
  ビジネス
\item
  医療
\item
  スポーツ
\item
  芸術(文学・音楽) などなど
\end{itemize}
\end{frame}

\begin{frame}{英語を身につけると}
\phantomsection\label{ux82f1ux8a9eux3092ux8eabux306bux3064ux3051ux308bux3068}
\Large

\textcolor{Maroon}{\bfseries 楽しい}ことがたくさんあります\pause

いっしょに\textcolor{Maroon}{\bfseries 楽しく}英語の勉強をしましょう
\end{frame}

\begin{frame}{英語は楽しい}
\phantomsection\label{ux82f1ux8a9eux306fux697dux3057ux3044}
\Huge

\centering

English is fun.
\end{frame}

\begin{frame}{}
\phantomsection\label{section-4}
\end{frame}

\begin{frame}[fragile]{Slide with R Output}
\phantomsection\label{slide-with-r-output}
\begin{Shaded}
\begin{Highlighting}[]
\FunctionTok{summary}\NormalTok{(cars)}
\end{Highlighting}
\end{Shaded}

\begin{verbatim}
##      speed           dist       
##  Min.   : 4.0   Min.   :  2.00  
##  1st Qu.:12.0   1st Qu.: 26.00  
##  Median :15.0   Median : 36.00  
##  Mean   :15.4   Mean   : 42.98  
##  3rd Qu.:19.0   3rd Qu.: 56.00  
##  Max.   :25.0   Max.   :120.00
\end{verbatim}
\end{frame}

\end{document}
