\documentclass[aspectratio=169,xcolor={dvipsnames,table}]{beamer}
\usepackage[no-math,deluxe,haranoaji]{luatexja-preset}
\renewcommand{\kanjifamilydefault}{\gtdefault}
\renewcommand{\emph}[1]{{\upshape\bfseries #1}}
\usetheme{metropolis}
\usetheme{metropolis}
\metroset{block=fill}
%%%%%%%%%%%%%%%%%%%%%%%%%%
\setbeamertemplate{navigation symbols}{}
\usecolortheme[rgb={0.7,0.2,0.2}]{structure}
%%%%%%%%%%%%%%%%%%%%%%%%%%
%% Change alert block colors
%%% 1- Block title (background and text)
\setbeamercolor{block title alerted}{fg=mDarkTeal, bg=mLightBrown!45!yellow!45}
\setbeamercolor{block title example}{fg=magenta!10!black, bg=mLightGreen!70}
%%% 2- Block body (background)
\setbeamercolor{block body alerted}{bg=mLightBrown!25}
\setbeamercolor{block body example}{bg=mLightGreen!15}
%%%%%%%%%%%%%%%%%%%%%%%%%%%
%%%%%%%%%%%%%%%%%%%%%%%%%%%
%% さまざまなアイコン
%%%%%%%%%%%%%%%%%%%%%%%%%%%
%\usepackage{fontawesome}
\usepackage{fontawesome5}
\usepackage{figchild}
\usepackage{twemojis}
\usepackage{utfsym}
\usepackage{bclogo}
\usepackage{marvosym}
\usepackage{fontmfizz}
\usepackage{pifont}
\usepackage{phaistos}
\usepackage{worldflags}
\usepackage{jigsaw}
\usepackage{tikzlings}
\usepackage{tikzducks}
\usepackage{scsnowman}
\usepackage{epsdice}
\usepackage{halloweenmath}
\usepackage{svrsymbols}
\usepackage{countriesofeurope}
\usepackage{tipa}
\usepackage{manfnt}
%%%%%%%%%%%%%%%%%%%%%%%%%%%
\usepackage{tikz}
\usetikzlibrary{calc,patterns,decorations.pathmorphing,backgrounds}
\usepackage{tcolorbox}
\usepackage{tikzpeople}
\usepackage{circledsteps}
\usepackage{xcolor}
\usepackage{amsmath}
\usepackage{booktabs}
\usepackage{chronology}
\usepackage{signchart}
%%%%%%%%%%%%%%%%%%%%%%%%%%%
%% 場合分け
%%%%%%%%%%%%%%%%%%%%%%%%%%%
\usepackage{cases}
%%%%%%%%%%%%%%%%%%%%%%%%%%
\usepackage{pdfpages}
%%%%%%%%%%%%%%%%%%%%%%%%%%%
%% 音声リンク表示
\newcommand{\myaudio}[1]{\href{#1}{\faVolumeUp}}
%%%%%%%%%%%%%%%%%%%%%%%%%%
%% \myAnch{<名前>}{<色>}{<テキスト>}
%% 指定のテキストを指定の色の四角枠で囲み, 指定の名前をもつTikZの
%% ノードとして出力する. 図には remember picture 属性を付けている
%% ので外部から参照可能である.
\newcommand*{\myAnch}[3]{%
  \tikz[remember picture,baseline=(#1.base)]
    \node[draw,rectangle,line width=1pt,#2] (#1) {\normalcolor #3};
}
%%%%%%%%%%%%%%%%%%%%%%%%%%
%% \myEmph コマンドの定義
%%%%%%%%%%%%%%%%%%%%%%%%%%
%\newcommand{\myEmph}[3]{%
%    \textbf<#1>{\color<#1>{#2}{#3}}%
%}
\usepackage{xparse} % xparseパッケージの読み込み
\NewDocumentCommand{\myEmph}{O{} m m}{%
    \def\argOne{#1}%
    \ifx\argOne\empty
        \textbf{\color{#2}{#3}}% オプション引数が省略された場合
    \else
        \textbf<#1>{\color<#1>{#2}{#3}}% オプション引数が指定された場合
    \fi
}
%%%%%%%%%%%%%%%%%%%%%%%%%%%
%%%%%%%%%%%%%%%%%%%%%%%%%%%
%% 文末の上昇イントネーション記号 \myRisingPitch
%% 通常のイントネーション \myDownwardPitch
%% https://note.com/dan_oyama/n/n8be58e8797b2
%%%%%%%%%%%%%%%%%%%%%%%%%%%
\newcommand{\myRisingPitch}{
\begin{tikzpicture}[scale=0.3,baseline=0.3]
\draw[->,>=stealth] (0,0) to[bend right=45] (1,1);
\end{tikzpicture}
}
\newcommand{\myDownwardPitch}{
\begin{tikzpicture}[scale=0.3,baseline=0.3]
\draw[->,>=stealth] (0,1) to[bend left=45] (1,0);
\end{tikzpicture}
}
%%%%%%%%%%%%%%%%%%%%%%%%%%%%
%\AtBeginSection[%
%]{%
%  \begin{frame}[plain]\frametitle{授業の流れ}
%     \tableofcontents[currentsection]
%   \end{frame}%
%}

\usepackage{vowel}
\usepackage{lua-ul}
\usepackage{pxrubrica}
\usepackage{tikzducks}
\usetikzlibrary{decorations.pathmorphing}
\usetikzlibrary{ducks}
\usepackage{scsnowman}
\usepackage{tikzlings}
%%%%%%%%%%%%%%%%%%%%%%%%%%%
\makeatletter
\newcommand*{\themonth}{\two@digits\month}
\newcommand*{\theday}{\two@digits\day}
\makeatother
\newcommand{\mytoday}{{\the\year}--{\themonth}--{\theday}}
%%%%%%%%%%%%%%%%%%%%%%%%%%%
\title{English is fun.}
\subtitle{Pronunciation---vowel---}
\author{}
\institute[]{}
\date[]

%%%%%%%%%%%%%%%%%%%%%%%%%%%%
%% TEXT
%%%%%%%%%%%%%%%%%%%%%%%%%%%%
\begin{document}
%%%%%%%%%%%%%%%%%%%%%%%%%%%
%%%%%%%%%%%%%%%%%%%%%%%%%%%%%%%%%%%%%%%%%%%%%%%%%%%%%
% 背景色をグレイに変更
%\setbeamercolor{background canvas}{bg=gray}
\setbeamercolor{background canvas}{bg=black}
\begin{frame}
%\centering
\raggedleft
  \textcolor{white}{\Huge\bfseries English is fun.}

\vfill

\raggedleft
% \textcolor{white}{\LARGE\bfseries 2024--11--26}
% \textcolor{white}{\LARGE\bfseries \today}
 \textcolor{white}{\LARGE\bfseries \mytoday}

\vfill
\vfill
\vfill

\raggedleft
\textcolor{white}{\large The lesson will begin at the scheduled time.}

%\textcolor{white}{\large 可能なら、鏡を用意してください}
\end{frame}
\setbeamercolor{background canvas}{bg=}
%%%%%%%%%%%%%%%%%%%%%%%%%%
%%%%%%%%%%%%%%%%%%%
%%% youtube
%%%%%%%%%%%%%%%%%%%%%%%%%%%%%%%%%%%%%%%%%%%%%%%%%%%%%
% 背景色をグレイに変更
%\setbeamercolor{background canvas}{bg=gray}
\setbeamercolor{background canvas}{bg=black}

\begin{frame}
%\centering
\raggedleft
  \textcolor{white}{\Huge\bfseries \textcolor{yellow}{E}nglish is fun.}

\vfill

\vfill

\raggedleft
 \textcolor{white}{\LARGE\bfseries \textcolor{yellow}{H}ello, everybody!}

 \textcolor{white}{\LARGE\bfseries \textcolor{yellow}{H}ow are you today?}

\raggedleft
 \textcolor{white}{\LARGE\bfseries \textcolor{yellow}{A}re you ready to start?}

 \textcolor{white}{\LARGE\bfseries \textcolor{yellow}{L}et's begin today's lesson.}

\vfill

\raggedleft
% \textcolor{white}{\LARGE\bfseries 2024--11--26}
% \textcolor{white}{\LARGE\bfseries \today}
 \textcolor{white}{\Large \bfseries \mytoday}

\hyperlink{today}{\beamergotobutton{Today's Pronunciation}}%%todayにジャンプ
\end{frame}
\setbeamercolor{background canvas}{bg=}
%%%%%%%%%%%%%%%%%%%%%%%%%
\begin{frame}[plain]{授業の進め方}\large

\pause
\begin{enumerate}
 \item<2-> スライドで進めていきます
 \item<3-> だれかを指名することはありません
\end{enumerate}

\bigskip

\hfill{}\visible<4->{安心して授業に参加してください!
\begin{tikzpicture}
\duck[laughing,bowtie,
strawhat=brown!50!white,
ribbon=black,
think={\scriptsize Enjoy!},
bubblecolour=white!50!pink]
\end{tikzpicture}}
\end{frame}
%%%%%%%%%%%%%%%%%%%%%%%%%
\begin{frame}[plain]{だいじなこと}\large
\pause

基礎基本をたいせつにしよう!

\pause

\hfill{}やさしいことからはじめて着実に
\begin{tikzpicture}
\duck[signpost=\scalebox{0.25}{
\parbox{2.5cm}{\color{black}
Slow but steady wins the race.}},
signcolour=brown!70!gray,
signback=white!80!brown,
graduate=gray!20!black,
tassel=red!70!black
]
\end{tikzpicture} 
\end{frame}
%%%%%%%%%%%%%%%%%%%%%%%%%
\begin{frame}[plain]{準備するもの}
\pause
 \begin{itemize}\setbeamertemplate{items}[square]
  \item<2-> ノート
  \item<3-> 筆記具(黒以外に2色あるとなおいい)
 \end{itemize}


\hfill%
{\begin{tikzpicture}
\duck[squareglasses=blue!50!black,
speech={\scriptsize \twemoji{spiral notepad}\,{$+$}\,\usymW{2710}{.37cm}},
laughing
]
\end{tikzpicture}}
\end{frame}
%%%%%%%%%%%%%%%%%%%%%%%%%%
\begin{frame}[plain]{予習復習}\large
 \begin{itemize}\setbeamertemplate{items}[square]
  \item<1-> 予習\visible<2->{$\longrightarrow$\,必要}\visible<3->{ありません} 
  \item<4-> 復習\visible<5->{$\longrightarrow$\,効果的!} 
 \end{itemize}

\hfill%
\visible<6->{\begin{tikzpicture}
\duck[tshirt,
jacket=gray,
bowtie,
crazyhair,
speech={\tiny できる範囲で},
laughing,
signpost=\scalebox{0.4}{
\parbox{2cm}{
気楽に\\
気が\\向いたら}},
]
\end{tikzpicture}}
\end{frame}
%%%%%%%%%%%%%%%%%%%%%%%%%
\begin{frame}[plain]{音声をだいじにしよう}\large

英語の音声に親しみましょう

\begin{enumerate}
 \item<2-> Please listen carefully.\hfill{\scriptsize \myaudio{./audio/listen.mp3}}
 \item<3-> Please repeat after me.\hfill{\scriptsize \myaudio{./audio/repeat.mp3}}
\end{enumerate}
 
\hfill%
\begin{tikzpicture}
\duck[signpost=\scalebox{0.4}{
\parbox{2cm}{\color{black}
\centering よく使う\\ 指示です}},
signcolour=brown!70!gray,
signback=white!80!brown,
sombrero=orange!70!yellow,
sombreroa=green!70!blue,
sombrerob=red,
sombreroc=blue,
glasses,
icecream=brown,
flavoura=green!50!brown,
flavourb=white,
flavourc=red,
speech={\tiny \parbox{2cm}{\centering Lsten carefully.\\Repeat after me.}}]\end{tikzpicture}
\end{frame}
%%%%%%%%%%%%%%%%%%%%%%%%%
\begin{frame}[plain,t]{授業について(まとめ)}
 \begin{itemize}\setbeamertemplate{items}[square]
  \item 授業の進め方
	\begin{itemize}\setbeamertemplate{items}[circle]
	 \item スライド
	 \item 指名はありません
	\end{itemize}
  \item 基礎基本をだいじに
  \item 筆記具・ノート
	\begin{itemize}\setbeamertemplate{items}[circle]
	 \item 筆記具は黒以外に2色あるとなおいい
	\end{itemize}
  \item 復習
	\begin{itemize}\setbeamertemplate{items}[circle]
	 \item 効果的
	 \item でも、余裕があればくらいの気持ちで
	\end{itemize}
  \item 英語の音声に親しみましょう
	\begin{itemize}\setbeamertemplate{items}[circle]
	 \item Please listen carefully.\hspace{24pt}{\scriptsize \myaudio{./audio/listen.mp3}}
	 \item Please repeat after me.\hspace{20pt}{\scriptsize \myaudio{./audio/repeat.mp3}}
	\end{itemize}
 \end{itemize}


\vspace{-2.5cm}

\hfill\begin{tikzpicture}
\bear[
scale=.75,
signpost={\scriptsize よろしく},
signcolour= brown!50!black,
signback=green!40!black
]
\end{tikzpicture}

\end{frame}
%%%%%%%%%%%%%%%%%%%%%%%%%%%%%
\section{資料の置き場}
%%%%%%%%%%%%%%%%%%%%%%%%%%%%%
\begin{frame}[plain,t]{資料の置き場}
 
\includegraphics[height=.9\textheight
]{./screenshot_1.png}

\vspace*{-.925\textheight}

\visible<2->{%
\hspace{45mm}\textcolor{red!60}{{\LARGE ここをクリック!}}

\vspace*{-3.3mm}

\hspace{39.6mm}\textcolor{red!60}{{\LARGE 〇}
}}


\end{frame}
%%%%%%%%%%%%%%%%%%%%%%%%%%%%%
\begin{frame}[plain,t]{資料の置き場}
 
\includegraphics[height=.9\textheight]{./screenshot_2.png}

\vspace*{-.575\textheight}

\visible<2->{%
\hspace{77mm}\textcolor{red!60}{{\LARGE ここをクリック!}}

\vspace*{-3.3mm}

\hspace{53.5mm}\textcolor{red!60}{\scalebox{4}[1]{{\LARGE 〇}}}
}

\end{frame}
%%%%%%%%%%%%%%%%%%%%%%%%%%%%%
\begin{frame}[plain]{資料の置き場}
 
\includegraphics[height=.9\textheight
]{./screenshot_3.png}

\vspace*{-.16\textheight}

\visible<2->{%
\hspace{80mm}\textcolor{red!60}{{\LARGE \,$\leftarrow$\,ここをクリック!}}
}
\end{frame}
%%%%%%%%%%%%%%%%%%%%%%%%%%%%%
\end{document}
