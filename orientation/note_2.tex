\documentclass[12pt]{jlreq}
%%%%%%%%%%%%%%%%%%%%%%%%%%%%
%% 欧文TTF/OTFフォントを利用するにはfontspec.styをロードする必要あり
%% 和文TTF/OTFフォントを利用するにはluatexja-fontspec.styをロードする必要あり
%% luatexja-fontspec.styはfontspec.styをないぶてきにロードする
%% lualatex-ja-preset.sty は luatexja-fontspec.styをロードする
%% つまり次の1行でluatexja-fontspec.sty, fontspec.styも自動的にロードされる
\usepackage[no-math,deluxe,expert,haranoaji]{luatexja-preset}
%%%%
\usepackage{graphicx}
\usepackage{xcolor}
\usepackage{pxrubrica}
\usepackage[default]{fontsetup}
%%%% tabular環境の改良版
\usepackage{tabularray}
\UseTblrLibrary{booktabs}
%%%% ハイパーリンク
%%%% hyperref.sty は preamble の最後で読み込む
\usepackage{hyperref}
\usepackage{xurl}
\hypersetup{
  bookmarks=true,
  bookmarksnumbered=true,
  pdfauthor={iotsuka1958}
}
%%%%%%%%%%%%%%%%%%%%%%%%%%%%%
\usepackage{tikz}
\usetikzlibrary{arrows}
%%%%%%%%%%%%%%%%%%%%%%%%%%%%%
\usepackage{luatexja-otf}
\ltjsetparameter{jacharrange={-2}}
%%%%%%%%%%%%%%%%%%%%%%%%%%%%%
\usepackage{array}
%%%%%%%%%%%%%%%%%%%%%%
% カスタム列指定子を定義
\newcolumntype{C}[1]{>{\centering\arraybackslash}m{#1}}
\newcolumntype{L}[1]{>{\raggedright\arraybackslash}m{#1}}
%%%%%%%%%%%%%%%%%%%%%%%%%%%%%
\begin{document}
%%%%%%%%%%%%%%%%%%%%%%%%%%%%%

\begin{center}\Large
 オリエンテーションmanuscript
\end{center}

\bigskip


\section{でだし}
みなさん、こんにちは。

エデュオプちばの英語の授業にようこそ。

前回は、第1回目の授業ということで
\begin{enumerate}
 \item 安心して学ぶ場にするために
 \item 授業の進め方や内容など
\end{enumerate}
について、お話しました。


\section{英語ってどんなことば}

きょうは2時間目です。

きょうの予定はごらんのとおりです。

本格的な授業にはいるまえに、きょうは「わたしたちが学ぶ英語ってどういうことば」
なのか考えてみたいとおもいます。

とちゅうでクイズをいくつかはさみながら、進めていきます。

リラックスして参加してください。

\subsection{世界地図}
いまごらんいただいているのは世界地図です。

この地図の中で、わたしたちがいいまいるのはどこでしょうか。

\subsection{Japan is here}

Japan is here.日本はここだよ。日本をオレンジで示しました。

あらためて世界全体のなかでみると、日本がごく小さな島国だということがよくわかりますね。

\subsection{Quiz}

さて日本では日本語が話されていますね。

みなさんにクイズを出します

日本語以外のことばにはどんなことばがありますか。
おもいつくままにあげてみましょう。

ノートにおもいつくことばをかいてみましょう。
よろしければチャットで、おもいつくことばを書き込んでみてください。

(ここで1分??)

いかがですか。

わたしたちが勉強することになっている英語がありますよね。

お隣の国でいえば、中国語とか韓国語とか。

ヨーロッパでいうと、フランス語ドイツ語スペイン語とか。


アラビア語もありますよね。

このほかのことばをあげてくれたみなさんもいるでしょう。

\subsection{世界にはいくつのことばがあるの}

ではもうひとつのクイズです。

世界にはいくつのことばがあるでしょうか。

ここはあてずっぽうでいいので、みなさんの答えを決めてください。

ノートに答えを書いてください。
よろしければチャットに書き込んでください。
いいですか、あてずっぽうでいいですよ。
はずれたってはずかしくなんかありませんから。

(30秒とる)

では、ヒントです。
これから選択肢をいくつか出しますから、そこから選んでください。

(A)7つ

いかがでしょうか。
7つだと思う人はいますか。

もうひとつの選択肢です。

(B) 70

いかがでしょうか。こんなところでしょうか。

次の選択肢です。

(C) 700

こんなにありますか?

では、最後の選択肢。

(D) 7,000


このなかのどれかが正解です。
みなさんの答えを決めてください。
なんどもいいますが、あてずっぽうでいいですよ。

さあいいですか。Your final answer

なんと7,164。

ですからこのクイズは(D)の7,000を選んだ人が正解でした。

あの、こまかいことをいうと、いつ誰が知らべたかによって、微妙に違うのですが、
でもだいたいこの7000くらいということになってます。



わたしの正直な気持ちは、
「ええーっ、そんなにあるの」、そういう気持ちです。

もうびっくりです。

あの、これ、ちょっぴりおもしろいクイズですから、こんどは
みなさんが誰かにこのクイズだすとおもしろいです。

話がわきにそれました。
話をもとにもどしますね。



世界には7,164のことばがある。

ということは、つまりです、
日本語は${1}/{7164}$にすぎないということになります。

でも、でもです、そういうのであれば、

日本語だけじゃありません、
英語もフランス語もドイツ語もスペイン語
もアラビア語もヒンディー語も中国語も韓国語も、みーんな
$1/7,164$ということです

どのことばも、そ$1/7,164$の存在ですが、そのことばを話す人にとってはかけがえのないものです。
それから、気をつけてほしいのは、どのことばのほうが優れているとか劣っているというようなことはないことにも注意しておきましょう。。

\hrulefill

さて、どのことばも 7,164 あることばのひとつ、つまり
${1}/{7,164}$という点では同じだといいましたが、
とはいってもです、
それぞれのことばを話している人の数も同じとい
うことになるでしょうか
そのことばを話す人の数がおおいことばもあれば、話す人があま
りいないことばもあるはずです。

\subsection{quiz}

では、ここでもうひとつの問題です。つぎの4つの言語で、話す
人の数がおおい順に並べると、どうなるでしょうか

ただし生まれてはじめて身につけたことばという条件で

\begin{itemize}
 \item  中国語 (Mandarin Chinese)
 \item  スペイン語 (Spanish)
 \item  英語 (English)
 \item  ヒンディー語 (Hindi)
\end{itemize}

皆さんで自分なりの答えを決めましょう。

\hrulefill


答えです。

グラフにしました。

上から順番に、
中国語、スペイン語、英語、ヒンディー語です。

これを見ると中国語が圧倒的におおいですねえ。

これ、単位は100万人ですから、中国語を話す人は1,000の目盛りのあたりですから
1,000$\times$100万人、ていうことはだいたい10億人くらいです。
スペイン語はその半分ですからまあだいたい5億人、英語とヒンディー語はまあだいたい
4億人ぐらいでしょうか。

\hrulefill

これは表にしたものです。

世界ではどうやら中国語を話す人がおおいようです。

であれば、外国語としてどうせ学ぶのであれば中国語がいいんじゃないのかなあと
思う人もいるかもしれませんね。
どうせ勉強するなら、話す人がおおいことばがよさそうだ、というわけでしょうか。


\hrulefill

ところで、2つ以上のことばを話す人がいますよね。

たとえば、
生まれて初めて身につけた第一言語は日本語だけれど、
家族の仕事の関係でアメリカに住んでいたので第二言語として英語も話すという人がいます。
なかにはさらに、韓国語も話せるよなんていう人もいます。

\hrulefill

さっき第一言語だと中国語を話す人がダントツでおおかったんですが、
第二言語とか、とにかくそのことばを話す人を加えるとどうでしょうか。

\hrulefill

第一言語として話す人の数はこうでしたね。

ここに第二言語とか第三言語とか、とにかくそのことばを話す人を加えてみます。

せーの。


\hrulefill

結果はこうです。

なんと逆転しましたね。

さっき3位だった英語が1位になりました。

\hrulefill

表にするとこうです。

英語は第一言語だけだと4億人弱で3位でしたが、
第二言語とか第三言語とか、とにかく英語を話す人が108、単位は100万人ですから、
10億8000万人、
これを加えると、
14億6000万人いるわけです。
英語は第一言語以外で話す人がすごくおおいことがわかります。

いっぽう、中国語は第一言語に限れば940、単位は100万人ですから9億4000万人でダントツで
1位でしたが、第二言語とか第三言語とかで話す人は1億9900万人で、英語に比較すればわずかです。
で、あわせると11億3900万人ということです。

\hrulefill

もう一度整理します。

世界には 7,164 のことばがありますが、
話す人をぜんぶあわせると英語が3位から1位へ
英語が1位で中国語が2位

\hrulefill

これまでは話す人の数という
ではつぎに英語と中国のそれぞれを話す人が世界のどのあたりにいるのかみてみましょう



まず中国語から

\hrulefill

中国語を公用語としている国と地域を
オレンジで示しました。

ほぼ中国に一極集中ですね。

\hrulefill

どうやら、
中国語を話す人は中国に集中しているといってよさそうです。

もちろん世界の各地にある中華街China townには中国語を話す人がいるはずですが、
全体的に見れば、中国語を話す人は中国に集中しているといってよさそうです。

では英語はどうでしょうか。

\hrulefill

英語を公用語としている国と地域、事実上の公用語となっているところをブルーで示しました。

\hrulefill

英語を話す人は世界中にいることがわかります

\hrulefill

中国語を話す人は中国に集中しています、つまり局所的

いっぽう、英語は広く世界中で話されています

\hrulefill

ここまでを整理します。

わたしたちが学ぶ英語とは

\begin{itemize}
 \item 英語は話す人の数がいちばんおおい

 \item 英語は広く世界中で話されている
\end{itemize}

 

つまり英語は世界の共通語といっていいだろうということになりますね。



\hrulefill

じっさい、英語はさまざまな分野で使われています

\begin{itemize}
 \item ビジネス
 \item 医療
 \item スポーツ
 \item 芸術(文学・音楽) などなど
\end{itemize}
さまざまな分野で英語が共通のことばになっています


\hrulefill

英語を学ぶと、いまいったようなさまざまな分野において、
楽しいことにであえることになるわけです。

さあいっしょに楽しく英語の勉強をしましょう。

\hrulefill

English is fun.

英語は楽しいぞー。



\subsection*{最後に}

それでは、つぎの授業でお会いしましょう。


\hrulefill
\hrulefill
\hrulefill
\hrulefill

\end{document}
