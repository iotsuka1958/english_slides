% Options for packages loaded elsewhere
\PassOptionsToPackage{unicode}{hyperref}
\PassOptionsToPackage{hyphens}{url}
%
\documentclass[
  ignorenonframetext,
  aspectratio=169,
  xcolor=dvipsnames]{beamer}
\usepackage{pgfpages}
\setbeamertemplate{caption}[numbered]
\setbeamertemplate{caption label separator}{: }
\setbeamercolor{caption name}{fg=normal text.fg}
\beamertemplatenavigationsymbolsempty
% Prevent slide breaks in the middle of a paragraph
\widowpenalties 1 10000
\raggedbottom
\setbeamertemplate{part page}{
  \centering
  \begin{beamercolorbox}[sep=16pt,center]{part title}
    \usebeamerfont{part title}\insertpart\par
  \end{beamercolorbox}
}
\setbeamertemplate{section page}{
  \centering
  \begin{beamercolorbox}[sep=12pt,center]{part title}
    \usebeamerfont{section title}\insertsection\par
  \end{beamercolorbox}
}
\setbeamertemplate{subsection page}{
  \centering
  \begin{beamercolorbox}[sep=8pt,center]{part title}
    \usebeamerfont{subsection title}\insertsubsection\par
  \end{beamercolorbox}
}
\AtBeginPart{
  \frame{\partpage}
}
\AtBeginSection{
  \ifbibliography
  \else
    \frame{\sectionpage}
  \fi
}
\AtBeginSubsection{
  \frame{\subsectionpage}
}
\usepackage{amsmath,amssymb}
\usepackage{iftex}
\ifPDFTeX
  \usepackage[T1]{fontenc}
  \usepackage[utf8]{inputenc}
  \usepackage{textcomp} % provide euro and other symbols
\else % if luatex or xetex
  \usepackage{unicode-math} % this also loads fontspec
  \defaultfontfeatures{Scale=MatchLowercase}
  \defaultfontfeatures[\rmfamily]{Ligatures=TeX,Scale=1}
\fi
\usepackage{lmodern}
\usetheme[]{metropolis}
\ifPDFTeX\else
  % xetex/luatex font selection
\fi
% Use upquote if available, for straight quotes in verbatim environments
\IfFileExists{upquote.sty}{\usepackage{upquote}}{}
\IfFileExists{microtype.sty}{% use microtype if available
  \usepackage[]{microtype}
  \UseMicrotypeSet[protrusion]{basicmath} % disable protrusion for tt fonts
}{}
\makeatletter
\@ifundefined{KOMAClassName}{% if non-KOMA class
  \IfFileExists{parskip.sty}{%
    \usepackage{parskip}
  }{% else
    \setlength{\parindent}{0pt}
    \setlength{\parskip}{6pt plus 2pt minus 1pt}}
}{% if KOMA class
  \KOMAoptions{parskip=half}}
\makeatother
\usepackage{xcolor}
\newif\ifbibliography
\usepackage{color}
\usepackage{fancyvrb}
\newcommand{\VerbBar}{|}
\newcommand{\VERB}{\Verb[commandchars=\\\{\}]}
\DefineVerbatimEnvironment{Highlighting}{Verbatim}{commandchars=\\\{\}}
% Add ',fontsize=\small' for more characters per line
\usepackage{framed}
\definecolor{shadecolor}{RGB}{248,248,248}
\newenvironment{Shaded}{\begin{snugshade}}{\end{snugshade}}
\newcommand{\AlertTok}[1]{\textcolor[rgb]{0.94,0.16,0.16}{#1}}
\newcommand{\AnnotationTok}[1]{\textcolor[rgb]{0.56,0.35,0.01}{\textbf{\textit{#1}}}}
\newcommand{\AttributeTok}[1]{\textcolor[rgb]{0.13,0.29,0.53}{#1}}
\newcommand{\BaseNTok}[1]{\textcolor[rgb]{0.00,0.00,0.81}{#1}}
\newcommand{\BuiltInTok}[1]{#1}
\newcommand{\CharTok}[1]{\textcolor[rgb]{0.31,0.60,0.02}{#1}}
\newcommand{\CommentTok}[1]{\textcolor[rgb]{0.56,0.35,0.01}{\textit{#1}}}
\newcommand{\CommentVarTok}[1]{\textcolor[rgb]{0.56,0.35,0.01}{\textbf{\textit{#1}}}}
\newcommand{\ConstantTok}[1]{\textcolor[rgb]{0.56,0.35,0.01}{#1}}
\newcommand{\ControlFlowTok}[1]{\textcolor[rgb]{0.13,0.29,0.53}{\textbf{#1}}}
\newcommand{\DataTypeTok}[1]{\textcolor[rgb]{0.13,0.29,0.53}{#1}}
\newcommand{\DecValTok}[1]{\textcolor[rgb]{0.00,0.00,0.81}{#1}}
\newcommand{\DocumentationTok}[1]{\textcolor[rgb]{0.56,0.35,0.01}{\textbf{\textit{#1}}}}
\newcommand{\ErrorTok}[1]{\textcolor[rgb]{0.64,0.00,0.00}{\textbf{#1}}}
\newcommand{\ExtensionTok}[1]{#1}
\newcommand{\FloatTok}[1]{\textcolor[rgb]{0.00,0.00,0.81}{#1}}
\newcommand{\FunctionTok}[1]{\textcolor[rgb]{0.13,0.29,0.53}{\textbf{#1}}}
\newcommand{\ImportTok}[1]{#1}
\newcommand{\InformationTok}[1]{\textcolor[rgb]{0.56,0.35,0.01}{\textbf{\textit{#1}}}}
\newcommand{\KeywordTok}[1]{\textcolor[rgb]{0.13,0.29,0.53}{\textbf{#1}}}
\newcommand{\NormalTok}[1]{#1}
\newcommand{\OperatorTok}[1]{\textcolor[rgb]{0.81,0.36,0.00}{\textbf{#1}}}
\newcommand{\OtherTok}[1]{\textcolor[rgb]{0.56,0.35,0.01}{#1}}
\newcommand{\PreprocessorTok}[1]{\textcolor[rgb]{0.56,0.35,0.01}{\textit{#1}}}
\newcommand{\RegionMarkerTok}[1]{#1}
\newcommand{\SpecialCharTok}[1]{\textcolor[rgb]{0.81,0.36,0.00}{\textbf{#1}}}
\newcommand{\SpecialStringTok}[1]{\textcolor[rgb]{0.31,0.60,0.02}{#1}}
\newcommand{\StringTok}[1]{\textcolor[rgb]{0.31,0.60,0.02}{#1}}
\newcommand{\VariableTok}[1]{\textcolor[rgb]{0.00,0.00,0.00}{#1}}
\newcommand{\VerbatimStringTok}[1]{\textcolor[rgb]{0.31,0.60,0.02}{#1}}
\newcommand{\WarningTok}[1]{\textcolor[rgb]{0.56,0.35,0.01}{\textbf{\textit{#1}}}}
\usepackage{graphicx}
\makeatletter
\def\maxwidth{\ifdim\Gin@nat@width>\linewidth\linewidth\else\Gin@nat@width\fi}
\def\maxheight{\ifdim\Gin@nat@height>\textheight\textheight\else\Gin@nat@height\fi}
\makeatother
% Scale images if necessary, so that they will not overflow the page
% margins by default, and it is still possible to overwrite the defaults
% using explicit options in \includegraphics[width, height, ...]{}
\setkeys{Gin}{width=\maxwidth,height=\maxheight,keepaspectratio}
% Set default figure placement to htbp
\makeatletter
\def\fps@figure{htbp}
\makeatother
\setlength{\emergencystretch}{3em} % prevent overfull lines
\providecommand{\tightlist}{%
  \setlength{\itemsep}{0pt}\setlength{\parskip}{0pt}}
\setcounter{secnumdepth}{-\maxdimen} % remove section numbering
%% PDFメタデータの文字化け防止
% https://blog.miz-ar.info/2015/09/latex-hyperref-tips/
% https://tex.stackexchange.com/questions/24445/hyperref-lualatex-and-unicode-bookmarks-issue-garbled-page-numbers-in-ar-for-l
\hypersetup{%
  pdfencoding=auto
}

%% Fonts
\usefonttheme[onlymath]{serif}
\usepackage[T1]{fontenc}
\usepackage{textcomp}
%\usepackage{arev}
\usepackage[scale=1.0]{tgheros}  % San serif font
\usepackage[scaled]{beramono}    % Monospace font

%% Japanese font
\usepackage{luatexja-otf}
\usepackage[match,deluxe,expert,haranoaji,nfssonly]{luatexja-preset} % Notoフォント使用
\renewcommand{\kanjifamilydefault}{\gtdefault}

%%
\setbeamerfont{title}{size=\huge, series=\bfseries}
\setbeamerfont{frametitle}{size=\Large, series=\bfseries}

%% https://tex.stackexchange.com/questions/62202/change-background-colour-of-verbatim-environment
\let\oldv\verbatim
\let\oldendv\endverbatim
\def\verbatim{\par\setbox0\vbox\bgroup\oldv}
\def\endverbatim{\oldendv\egroup\fboxsep0pt \noindent\colorbox[gray]{0.95}{\usebox0}\par}

%% https://stackoverflow.com/questions/38323331/code-chunk-font-size-in-beamer-with-knitr-and-latex
%% change fontsize of R code
\let\oldShaded\Shaded
\let\endoldShaded\endShaded 
\renewenvironment{Shaded}{\footnotesize\oldShaded}{\endoldShaded}

\usepackage{listings}
\lstset{%
  frame = shadow,
  backgroundcolor = {\color[gray]{.95}},
  basicstyle = {\small\ttfamily},
  breaklines = true,
  upquote = true
}
%%%%%%%%%%%%%%
\usepackage{fontawesome}
%%%%%%%%%%%%%%%
\usepackage{booktabs}
\usepackage{tikz}
\usepackage{pxpgfmark} % remember picture を可能にする
\usetikzlibrary{decorations,backgrounds,decorations.pathmorphing, shapes,positioning,fit,calc,spy}
\newcommand*\myCrossedOut[2]{%
  \tikz[baseline=(T.base)]
    \node[draw=#1, ultra thick, shape=cross out, decorate,
      inner sep=2pt, outer sep=0pt,
      decoration={random steps, segment length=2pt, amplitude=0.4pt}]
      (T) {#2};}
%%%%%%%%%%%%%%%%%%%%%%%%%%%%
\usetikzlibrary{decorations.text}     
%%%%%%%%%%%%%%%%%%%%%%%%%
\definecolor{softblue}{RGB}{87, 117, 144}
\definecolor{gentlegreen}{RGB}{125, 158, 128}
\definecolor{lightpurple}{RGB}{156, 134, 170}
\definecolor{warmorange}{RGB}{202, 137, 98}
\definecolor{gentlered}{RGB}{169, 105, 102}
\definecolor{softred}{RGB}{169, 0, 0}
\definecolor{darkred}{RGB}{207, 89, 89}      % ダークレッド
\definecolor{brightorange}{RGB}{235, 116, 70}     % ブライトオレンジ
\definecolor{lightorange}{RGB}{242, 150, 90}     % ライトオレンジ
\definecolor{gold}{RGB}{255, 174, 66}     % ゴールド
\definecolor{coral}{RGB}{247, 126, 101}    % コーラル
\definecolor{mintgreen}{RGB}{124, 156, 124}  % ミントグリーン
\definecolor{khaki}{RGB}{153, 141, 116}  % カーキ
\definecolor{mocha}{RGB}{141, 124, 113}  % モカ
\definecolor{slateblue}{RGB}{124, 123, 156}  % スレートブルー
\definecolor{lightyellow}{RGB}{242, 207, 90}    % ライトイエロー
\definecolor{yellowgold}{RGB}{235, 189, 70}    % イエローゴールド
\definecolor{mediumyellow}{RGB}{224, 171, 46}    % メディアムイエロー
\definecolor{darkyellow}{RGB}{217, 153, 25}    % ダークイエロー
\definecolor{goldenyellow}{RGB}{204, 135, 15}    % ゴールデンイエロー
%%%%%%%%%%%%%%%%%
\usepackage{array}
%% セルの内容を縦中央にそろえる
% カスタム列指定子を定義
\newcolumntype{C}[1]{>{\centering\arraybackslash}m{#1}}
\newcolumntype{L}[1]{>{\raggedright\arraybackslash}m{#1}}
%%%%%%%%%%%%%%%%%
\usepackage{pxrubrica}
\ifLuaTeX
  \usepackage{selnolig}  % disable illegal ligatures
\fi
\usepackage{bookmark}
\IfFileExists{xurl.sty}{\usepackage{xurl}}{} % add URL line breaks if available
\urlstyle{same}
\hypersetup{
  pdftitle={English is fun},
  hidelinks,
  pdfcreator={LaTeX via pandoc}}

\title{English is fun}
\author{}
\date{\vspace{-2.5em}2024}

\begin{document}
\frame{\titlepage}

\begin{frame}{}
\phantomsection\label{section}
\thispagestyle{empty}
\Large

\raggedright

予定の時刻になったらはじまります

\textbullet  音声を流しています

\textbullet  聞こえていますか 

\vfill

\raggedleft

The lesson will begin at the scheduled time.

\vspace{-6pt}

We are playing audio.

\vspace{-6pt}

Can you hear it?
\end{frame}

\begin{frame}{}
\phantomsection\label{section-1}
\thispagestyle{empty}
\titlepage
\end{frame}

\begin{frame}{前回の復習}
\phantomsection\label{ux524dux56deux306eux5fa9ux7fd2}
\thispagestyle{empty}
\LARGE

\begin{tabular}{rl}
1&安心して学ぶ場にするために\\
2&授業の進め方や内容など
\end{tabular}
\end{frame}

\begin{frame}{きょうの予定}
\phantomsection\label{ux304dux3087ux3046ux306eux4e88ux5b9a}
\thispagestyle{empty}
\Large

\begin{tabular}{rl}
1&世界のことば\\
2&世界にはどんなことばがありますか\\
3&世界にはいくつのことばがあるの\\
4&そのことばを話す人はどこに住んでいるの\\
5&わたしたちが学ぶ英語とは
\end{tabular}
\end{frame}

\begin{frame}{世界のことば}
\phantomsection\label{ux4e16ux754cux306eux3053ux3068ux3070}
\Large

Hello, World!

\includegraphics[width=0.9\linewidth]{orientation_2_files/figure-beamer/unnamed-chunk-3-1}
\end{frame}

\begin{frame}{Japan is here}
\phantomsection\label{japan-is-here}
\Large

\textcolor{orange}{\bfseries Japan} is here.

\includegraphics[width=0.9\linewidth]{orientation_2_files/figure-beamer/unnamed-chunk-4-1}
\end{frame}

\begin{frame}{世界のことば}
\phantomsection\label{ux4e16ux754cux306eux3053ux3068ux3070-1}
\Large

日本では日本語

\pause

Quiz: 世界にはどんなことばがありますか
\end{frame}

\begin{frame}{日本語以外のことばをあげてみよう}
\phantomsection\label{ux65e5ux672cux8a9eux4ee5ux5916ux306eux3053ux3068ux3070ux3092ux3042ux3052ux3066ux307fux3088ux3046}
\Large

\begin{itemize}[<+->]
\tightlist
\item
  日本語
\item
  英語
\item
  中国語
\item
  韓国語
\item
  フランス語
\item
  ドイツ語
\item
  スペイン語
\item
  アラビア語 \ldots
\end{itemize}

\pause

ほかにもたくさんあります
\end{frame}

\begin{frame}{世界にはいくつのことばがあるの}
\phantomsection\label{ux4e16ux754cux306bux306fux3044ux304fux3064ux306eux3053ux3068ux3070ux304cux3042ux308bux306e}
\Large

Quiz: ところで、世界にはいくつくらいことばがあるの?

\pause

\vfill

\centering
\begin{tabular}{lr}
(A) &7\\\pause
(B) &70\\\pause
(C) &700\\\pause
(D) &7,000
\end{tabular}

\vfill
\end{frame}

\begin{frame}{Answer}
\phantomsection\label{answer}
\Huge
\vfill

\centering
7,164

\vfill

\raggedleft
\scriptsize

\url{https://www.ethnologue.com/insights/how-many-languages/による}
\end{frame}

\begin{frame}{びっくり}
\phantomsection\label{ux3073ux3063ux304fux308a}
\Large

\begin{columns}
    \begin{column}{0.475\textwidth}
      ええー、そんなにあるの
    \end{column}
    \begin{column}{0.475\textwidth}
      \IfFileExists{surprised.png}{\includegraphics[width=.6\textwidth]{surprised.png}}{\relax}
            {\tiny Designed by Wannapik}
      \end{column}
  \end{columns}
\end{frame}

\begin{frame}{そのうちのひとつ}
\phantomsection\label{ux305dux306eux3046ux3061ux306eux3072ux3068ux3064}
\Large

ということは\pause

日本語は\Huge \phantom{0}\(\frac{1}{7,164}\)\phantom{0}\Large ということですね

\vfill
\pause

でもそれをいったら、\pause 英語もフランス語もドイツ語も\pause
スペイン語もアラビア語もヒンディー語も\pause 中国語も韓国語も、\pause
みーんな\Huge\phantom{0}\(\frac{1}{7,164}\)\phantom{0}\Large ということです

\vfill
\pause

どの言語が優れているとかいうようなことはまったくありません
\end{frame}

\begin{frame}{どのことばを話す人がおおいの}
\phantomsection\label{ux3069ux306eux3053ux3068ux3070ux3092ux8a71ux3059ux4ebaux304cux304aux304aux3044ux306e}
\Large

どのことばも7,164あることばのひとつ、\pause
つまり\Huge\phantom{0}\(\frac{1}{7,164}\)\phantom{0}

\Large
\pause

とはいっても、
それぞれのことばを話している人の数も同じということになるでしょうか

\pause

そのことばを話す人の数がおおいことばもあれば、
話す人があまりいないことばもあるはずです。
\end{frame}

\begin{frame}{Quiz}
\phantomsection\label{quiz}
\Large

では、ここでもうひとつの問題です。
つぎの4つの言語で、話す人の数がおおい順に
並べると、どうなるでしょうか\pause

ただし生まれてはじめて身につけたことばという条件で

\pause

\begin{itemize}[<+->]
\tightlist
\item
  中国語(Mandarin Chinese)
\item
  スペイン語(Spanish)
\item
  英語(English)
\item
  ヒンディー語(Hindi)
\end{itemize}
\end{frame}

\begin{frame}{生まれて初めて身につけたことば}
\phantomsection\label{ux751fux307eux308cux3066ux521dux3081ux3066ux8eabux306bux3064ux3051ux305fux3053ux3068ux3070}
\includegraphics[height=0.95\textheight]{orientation_2_files/figure-beamer/unnamed-chunk-5-1}
\end{frame}

\begin{frame}{第一言語として話す人の数がおおいことば}
\phantomsection\label{ux7b2cux4e00ux8a00ux8a9eux3068ux3057ux3066ux8a71ux3059ux4ebaux306eux6570ux304cux304aux304aux3044ux3053ux3068ux3070}
\raggedleft

(単位: 100万人)

\vfill

\Large
\centering

\begin{tabular}{lr}
\toprule
ことば & 話者数\\
\midrule
Mandarin Chinese & 940\\
Spanish & 485\\
English & 380\\
Hindi & 345\\
\bottomrule
\end{tabular}

\vfill

\raggedleft
\scriptsize

\url{https://www.ethnologue.com/insights/most-spoken-language/による}
\end{frame}

\begin{frame}{複数のことばを話す人もいますよね}
\phantomsection\label{ux8907ux6570ux306eux3053ux3068ux3070ux3092ux8a71ux3059ux4ebaux3082ux3044ux307eux3059ux3088ux306d}
\Large

第一言語はAということばだけれど、

そのほかにも第二言語としてBということばを話す人がいます
\end{frame}

\begin{frame}{第一言語ではないけど、そのことば話します}
\phantomsection\label{ux7b2cux4e00ux8a00ux8a9eux3067ux306fux306aux3044ux3051ux3069ux305dux306eux3053ux3068ux3070ux8a71ux3057ux307eux3059}
\Large

第一言語ではないけれど、そのことばを話します

そういう人をあわせると、 どうなるでしょうか
\end{frame}

\begin{frame}{第一言語として話す人の数はこうでしたね}
\phantomsection\label{ux7b2cux4e00ux8a00ux8a9eux3068ux3057ux3066ux8a71ux3059ux4ebaux306eux6570ux306fux3053ux3046ux3067ux3057ux305fux306d}
\includegraphics[height=0.95\textheight]{orientation_2_files/figure-beamer/unnamed-chunk-7-1}
\end{frame}

\begin{frame}{第一言語ではない人を合わせるとこうなります}
\phantomsection\label{ux7b2cux4e00ux8a00ux8a9eux3067ux306fux306aux3044ux4ebaux3092ux5408ux308fux305bux308bux3068ux3053ux3046ux306aux308aux307eux3059}
\includegraphics[height=0.95\textheight]{orientation_2_files/figure-beamer/unnamed-chunk-8-1}
\end{frame}

\begin{frame}{あれあれ}
\phantomsection\label{ux3042ux308cux3042ux308c}
\Huge

大逆転
\end{frame}

\begin{frame}{つまり}
\phantomsection\label{ux3064ux307eux308a}
\raggedleft

(単位: 100万人)

\vfill

\Large

\begin{tabular}{lrrr}
\toprule
ことば & 第一言語 & それ以外 & Total\\
\midrule
English & 380 & 1,080 & 1,460\\
Mandarin Chinese & 940 & 199 & 1,139\\
Hindi & 345 & 265 & 610\\
Spanish & 485 & 74 & 559\\
\bottomrule
\end{tabular}

\vfill

\scriptsize

\url{https://www.ethnologue.com/insights/most-spoken-language/による}
\end{frame}

\begin{frame}{話す人を全部あわせると}
\phantomsection\label{ux8a71ux3059ux4ebaux3092ux5168ux90e8ux3042ux308fux305bux308bux3068}
\Large

世界には7,164のことばがありますが、

話す人をぜんぶあわせると\pause
\textcolor{Maroon}{英語を話す人がいちばんおおい}\pause

2位は中国語\pause

\centering
\vfill

\begin{tabular}{lr}\toprule
ことば&話者数\\
\midrule
英語&14億6000万人\\
中国語 &11億3900万人\\
\bottomrule
\end{tabular}
\end{frame}

\begin{frame}{そのことばを話す人はどこに住んでいるの}
\phantomsection\label{ux305dux306eux3053ux3068ux3070ux3092ux8a71ux3059ux4ebaux306fux3069ux3053ux306bux4f4fux3093ux3067ux3044ux308bux306e}
\Large

中国語と英語を話す人が

世界のどこに住んでいるか見てみましょう。
\end{frame}

\begin{frame}{Chinese}
\phantomsection\label{chinese}
\includegraphics{orientation_2_files/figure-beamer/unnamed-chunk-10-1.pdf}
\end{frame}

\begin{frame}{中国語を話す人}
\phantomsection\label{ux4e2dux56fdux8a9eux3092ux8a71ux3059ux4eba}
\Large

\textcolor{Maroon}{中国語}を話す人は\textcolor{Maroon}{中国に集中}していますね
\end{frame}

\begin{frame}{English}
\phantomsection\label{english}
\includegraphics{orientation_2_files/figure-beamer/unnamed-chunk-11-1.pdf}
\end{frame}

\begin{frame}{English}
\phantomsection\label{english-1}
\Large

\textcolor{NavyBlue}{英語}を話す人は\textcolor{NavyBlue}{世界中にいます}
\end{frame}

\begin{frame}{つまり}
\phantomsection\label{ux3064ux307eux308a-1}
\Large

\textcolor{Maroon}{中国語}を話す人は、\textcolor{Maroon}{中国に集中}しています\pause

いっぽう、\textcolor{NavyBlue}{英語は広く世界中で話されています}
\end{frame}

\begin{frame}{わたしたちが勉強する英語ってどんなことばなの}
\phantomsection\label{ux308fux305fux3057ux305fux3061ux304cux52c9ux5f37ux3059ux308bux82f1ux8a9eux3063ux3066ux3069ux3093ux306aux3053ux3068ux3070ux306aux306e}
\Large

\begin{itemize}[<+->]
\tightlist
\item
  英語は話す人の数が\textcolor{NavyBlue}{いちばんおおい}
\item
  英語は\textcolor{NavyBlue}{広く世界中で}話されている
\end{itemize}

\pause

つまり\pause

\textcolor{NavyBlue}{英語は世界の共通語}といっていいだろう
\end{frame}

\begin{frame}{英語は世界の共通語}
\phantomsection\label{ux82f1ux8a9eux306fux4e16ux754cux306eux5171ux901aux8a9e}
\Large

さまざまな分野で使われています

\begin{itemize}[<+->]
\tightlist
\item
  ビジネス
\item
  医療
\item
  スポーツ
\item
  芸術(文学・音楽) などなど
\end{itemize}
\end{frame}

\begin{frame}{英語を身につけると}
\phantomsection\label{ux82f1ux8a9eux3092ux8eabux306bux3064ux3051ux308bux3068}
\Large

\textcolor{Maroon}{\bfseries 楽しい}ことがたくさんあります\pause

いっしょに\textcolor{Maroon}{\bfseries 楽しく}英語の勉強をしましょう
\end{frame}

\begin{frame}{}
\phantomsection\label{section-2}
\Huge

\centering

English is fun.
\end{frame}

\section{See you next time.}\label{see-you-next-time.}

\begin{frame}{}
\phantomsection\label{section-3}
\end{frame}

\begin{frame}[fragile]{Slide with R Output}
\phantomsection\label{slide-with-r-output}
\begin{Shaded}
\begin{Highlighting}[]
\FunctionTok{summary}\NormalTok{(cars)}
\end{Highlighting}
\end{Shaded}

\begin{verbatim}
##      speed           dist       
##  Min.   : 4.0   Min.   :  2.00  
##  1st Qu.:12.0   1st Qu.: 26.00  
##  Median :15.0   Median : 36.00  
##  Mean   :15.4   Mean   : 42.98  
##  3rd Qu.:19.0   3rd Qu.: 56.00  
##  Max.   :25.0   Max.   :120.00
\end{verbatim}
\end{frame}

\end{document}
