\documentclass[12pt]{jlreq}
%%%%%%%%%%%%%%%%%%%%%%%%%%%%
%% 欧文TTF/OTFフォントを利用するにはfontspec.styをロードする必要あり
%% 和文TTF/OTFフォントを利用するにはluatexja-fontspec.styをロードする必要あり
%% luatexja-fontspec.styはfontspec.styをないぶてきにロードする
%% lualatex-ja-preset.sty は luatexja-fontspec.styをロードする
%% つまり次の1行でluatexja-fontspec.sty, fontspec.styも自動的にロードされる
\usepackage[no-math,deluxe,expert,haranoaji]{luatexja-preset}
%%%%
\usepackage{graphicx}
\usepackage{xcolor}
\usepackage{pxrubrica}
\usepackage[default]{fontsetup}
%%%% tabular環境の改良版
\usepackage{tabularray}
\UseTblrLibrary{booktabs}
%%%% ハイパーリンク
%%%% hyperref.sty は preamble の最後で読み込む
\usepackage{hyperref}
\usepackage{xurl}
\hypersetup{
  bookmarks=true,
  bookmarksnumbered=true,
  pdfauthor={iotsuka1958}
}
%%%%%%%%%%%%%%%%%%%%%%%%%%%%%
\usepackage{tikz}
\usetikzlibrary{arrows}
%%%%%%%%%%%%%%%%%%%%%%%%%%%%%
\usepackage{luatexja-otf}
\ltjsetparameter{jacharrange={-2}}
%%%%%%%%%%%%%%%%%%%%%%%%%%%%%
\usepackage{array}
%%%%%%%%%%%%%%%%%%%%%%
% カスタム列指定子を定義
\newcolumntype{C}[1]{>{\centering\arraybackslash}m{#1}}
\newcolumntype{L}[1]{>{\raggedright\arraybackslash}m{#1}}
%%%%%%%%%%%%%%%%%%%%%%%%%%%%%
\begin{document}
%%%%%%%%%%%%%%%%%%%%%%%%%%%%%

\begin{center}\Large
 オリエンテーションmanuscript
\end{center}

\bigskip

\section{でだし}
みなさん、こんにちは。

きょうからエデュオプちばの授業がスタートします。
英語を担当する大塚です。

この時間が楽しい時間になるようつとめたいとかんがえています。
どうぞよろしくお願いします。

きょうははじめての授業ですが、
きょうの予定はこうです。

おおづかみにいうと、
2つの柱があります。

1つ目は、
エデュオプちばが安心して学べる場所になるための約束について
お話します。

2つ目は
授業の具体的な進め方などについてお話するつもりです。

緊張することはありませんよ。
気を楽にして、リラックスして聞いてきください。


\newpage
\section{自己紹介}

わたしの自己紹介をします。

オオツカ、下の名前はイチロウ、大塚一朗といいます。

わたしが好きなことについてお話します。

いま千葉ロッテマリーンズといいました。
わたしは千葉ロッテマリーンズを応援しています。
昨年は千葉市の幕張にあるマリンスタジアムで20回現地で応援しました。
ぜひ優勝してほしいとおもっています。

それからミステリー、日本語でいうと推理小説がすきです。
本を読むのはもちろんですが、映画やテレビドラマもミステリーものが好きで、よく見ています。

好きな音楽は、だいたいなんでも好きですが、
よく聴くのはバッハ、クラシックです。イタリアンコンチェルトが特に好きです。それからビーチボーイズです。アメリカのロックバンドです。
あかるく美しいハーモニーが魅力です。クラシックとロックでは、首尾一貫していませんが、好きなものはしかたないですよね。
みなさんも、もし興味があったらぜひ聴いてみてください。

コンピュータでプログラムをつくるのも好きですが、素人レベルです。
でも、楽しいです。

最後に、先週の全体オリエンテーションのときにもいいましたが、
ネコがだいすきです。

さて、とにかく楽しいことや好きなものにかこまれていると幸せな気持ちになります。

この英語の時間も楽しい時間にしたいとおもっています。

ということで、この時間のタイトルを
English is fun.(英語って楽しいぞ)
としました。

%%%%%%%%%%%%%%
\newpage
\section{安心して学べる場にするために}


みんなが安心して楽しく学べるように、みなさんといくつか約束をしておきたいとおもいます。

\subsection{誹謗中傷}

他の人が傷つくような発言はしないようにしましょう。


\subsection{個人情報}

みなさん、個人情報って聞いたことがありますか。

個人情報ってなんのことでしょうか。


生存する個人に関する情報で、
特定の個人を識別できるような情報のことでをいいます。

でも、なんとなくわったようなわからないような。

具体的に考えてみましょう。

\paragraph{氏名}
たとえば、氏名。
これはまさに個人が特定できてしまいますから、
個人情報そのものですね。

\paragraph{顔写真}
では、顔写真はどうでしょうか。

これも、すぐに個人が特定できてしまいます。
顔写真も個人情報です。

\paragraph{住所}
じゃあ、住所はどうでしょうか。
氏名と組み合わせたときは個人を特定できてしまいますので、個人情報です。

\paragraph{生年月日}
生年月日はどうでしょうか。
これもそれだけでは個人は特定できないかもしれませんが、
氏名と組み合わせれば個人が特定できてしまいますから個人情報です。

ここにあげたような個人情報は、書き込まないようにしましょう。

自分や友達の名前、学校名などをけっして書き込まないように
注意してください。


\subsection{誹謗中傷はしない}


みんなが安心して学ぶための約束2です。

他の人が傷つくようなことはけっしていわないようにしましょう。



\subsection{ログインに必要な情報}
みんなが安心して学ぶための約束3です。

ログインに必要な情報はたいせつに管理しましょう。

こういった情報はだれかに聞かれても教えないことがたいせつです。

たとえば誰かから「ねえねえ、誰にも言わないから教えてよ」などといわれても、
「これは大事なものだから教えることはできないんだ」といいましょう。

\subsection{パスワード管理}
みんなが安心して学ぶための約束4です。

安全なパスワードを設定しましょう。
他の人が推測できないようなパスワードが安全です。


パスワードは人に教えないようにしましょう。
せっかく安全なパスワードを設定しても、他の人に漏れてしまったら
意味がありませんよね。

ときどきあるのが、自分でパスワードを忘れてしまうことです。
こればかりは、他の人に聞いてもわかりません。
自分で忘れてしまうことのないようにしましょう
忘れてしまうとログインできなくなってしまいます。



\subsection{録音録画}
みんなが安心して学ぶための約束5です。

エデュオプちばから配信された教材や動画等を許可なく録画・録音したり、SNSなどにのせないようにしましょう。

\subsection{約束--まとめ}

約束をもう一度見てください。

これらの約束を守って、安心して楽しく学んでいきましょう。


%%%%%%%%%%%%%%%
\newpage
\section{授業の進め方}
つぎに授業の進め方についてお話します。


\subsection{全般的に}

授業はスライドを用いて進めます。画面を見ながらわたしの話に耳をかたむけてください。

わたしからみなさんに、いろいろと質問をすることがあります。
そのときは、みなさんが自分のアタマであれこれ考えることがだいじです。


なお、授業で個別にみなさんの誰かをさすようなことはありませんので、
安心して授業にのぞんでください。

\subsection{基礎基本}
授業では、基礎基本をだいじにします。

ごくかんたんなところから進めていきます。
もしかするとみなさんのなかには、
「それはもう知ってるよ」とか
「なあんだ、ずいぶんかんたんだなあ」という感想をもつ人も
いるかもしれません。

でも、やさしいことからはじめて、
ひとつひとつ身につけていくことで、しだいに力がついていきます。
気がついたら、すいぶん英語ができるようになってきたな、と思うようになるはずです。
いっしょにあせらず気を楽にしてに進めていきましょう。

\subsection{準備するもの}

ノートと筆記用具の準備をしてください。

わたしから「ここをノートに写しましょう」という指示をだすことがあります。
そのときは、指示に従ってスライドの内容をノートに写してください。

また、特に指示がなくても、みなさんがだいじたとおもったことは、どんどん
ノートにまとめていきましょう。

このノートは、みなさんのだいじな宝物になるとおもいます。

ノートや筆記具はみなさんが使いやすいものなら、なんでもいいですよ。
好みのものを選んでください。

筆記用具の色は、黒以外にも何色かあると、べんりだとおもいます。
ラインマーカー、蛍光ペンともいいますが、みなさんの好みに応じて使ってみると、
ノートをまとめるのにいいかもしれません。

\subsection{予習復習について}

予習復習についてです。
この授業では予習をする必要はありません。
気軽に参加してくださればけっこうです。

もし余裕があれば、復習は効果的だとおもいます。
ただし、「かならず復習しなくっちゃ」などと力む必要はありません。
肩の力を抜いて、「余裕があったら復習でもしようかな」くらいの
気軽な気持ちでいいとおもいます。


\subsection{音声をだいじに}
つぎに音声についてです。

英語の音声に親しむことをだいじにします。

ですから、実際の英語の音声に耳をかたむけてもらう時間をもうけます。
そのときはplease listen carefully.というフレーズが合図です。
Please listen carefully.は「注意して聞いてください、注意して耳を傾けてください」という意味のフレーズです。
このフレーズがあったら、続く音声によく耳をかたむけるようにしてください。
ここで実際の音声を流します。

(音声を流す)

いかがですか。
授業ではこのフレーズを頻繁に使います。
このフレーズがあったら、続く音声によく耳をかたむけるようにしてください。

もう一度聞いてください。

(音声を流す)

では、もうひとつのフレーズです。
Please repeat after me.というフレーズです。
これは「私のあとでリピートしてください」という意味です。Please repeat after me.という
フレーズがあったら、音声によく耳を傾けて、自分で口に出して発音してください。
ここで実際の音声を聞いてもらいます。

(音声を流す)

このフレーズを合図に、自分で実際に口にだして発音しましょう。
もう一度流します。

(音声を流す)


\subsection{授業について(まとめ)}
授業について、ご覧のようなことについてお話しました。

\section{本日のまとめ}

きょうははじめての授業ということで、
\begin{itemize}
\item  安心して学ぶ場にするための約束
\item  授業の進め方
\end{itemize}
についてお話しました。

それでは、つぎの授業でお会いしましょう。
お待ちしています。
%%%%%%%%%%%%%%
\newpage
\section{英語ってどんなことば}

\hrulefill

きょうは最初の時間ですので、英語の本格的な授業に入る前にいろいろとお話したいとおもいます。

いまごらんいただいているのは世界地図です。

この地図の中で、わたしたちがいいまいるのはどこでしょうか。

\hrulefill

Japan is here.日本はここだよ。日本をオレンジで示しました。

あらためて世界全体のなかでみると、日本がごく小さな島国だということがよくわかりますね。

\hrulefill

さて日本では日本語が話されていますね。

では、世界にはどんなことばがありますか。

\hrulefill

みなさんにクイズを出します

日本語以外のことばにはどんなことばがありますか。
おもいつくままにあげてみましょう。

もし手元に書くものがあれば、なければいいですよ。
もしあれば、紙におもいつくことばをかいてみましょう。

(ここで2分)

いかがですか。

わたしたちが勉強することになっている英語がありますよね。

お隣の国でいえば、中国語とか韓国語とか。

ヨーロッパでいうと、フランス語ドイツ語スペイン語とか。


アラビア語もありますよね。

このほかのことばをあげてくれたみなさんもいるでしょう。

\hrulefill

ではもうひとつのクイズです。

世界にはいくつのことばがあるでしょうか。

ここはあてずっぽうでいいので、みなさんの答えを決めてください。

やはり紙がある人は自分の答えを書いてください。
いいですか、あてずっぽうでいいですよ。
はずれたってはずかしくなんかありませんから。

(30秒とる)

では、ヒントです。
これから選択肢をいくつか出しますから、そこから選んでください。

(A)7つ

いかがでしょうか。
7つだと思う人はいますか。

もうひとつの選択肢です。

(B) 70

いかがでしょうか。こんなところでしょうか。

次の選択肢です。

(C) 700

こんなにありますか?

では、最後の選択肢。

(D) 7,000


このなかのどれかが正解です。
みなさんの答えを決めてください。
なんどもいいますが、あてずっぽうでいいですよ。

さあいいですか。Your final answer

\hrulefill

なんと7,164。

ですからこのクイズは(D)の7,000を選んだ人が正解でした。

あの、こまかいことをいうと、いつ誰が知らべたかによって、微妙に違うのですが、
でもだいたいこの7000くらいということになってます。

\hrulefill

わたしの正直な気持ちは、
「ええーっ、そんなにあるの」、そういう気持ちです。

もうびっくりです。

あの、これ、ちょっぴりおもしろいクイズですから、こんどは
みなさんが誰かにこのクイズだすとおもしろいです。

話がわきにそれました。
話をもとにもどしますね。

\hrulefill

世界には7,164のことばがある。

ということは、つまりです、
日本語は${1}/{7164}$にすぎないということになります。

でも、でもです、そういうのであれば、

日本語だけじゃありません、
英語もフランス語もドイツ語もスペイン語
もアラビア語もヒンディー語も中国語も韓国語も、みーんな
$1/7,164$ということです

どのことばも、そ$1/7,164$の存在ですが、そのことばを話す人にとってはかけがえのないものです。
それから、気をつけてほしいのは、どのことばのほうが優れているとか劣っているというようなことはないことにも注意しておきましょう。。

\hrulefill

さて、どのことばも 7,164 あることばのひとつ、つまり
${1}/{7,164}$という点では同じだといいましたが、
とはいってもです、
それぞれのことばを話している人の数も同じとい
うことになるでしょうか
そのことばを話す人の数がおおいことばもあれば、話す人があま
りいないことばもあるはずです。

\hrulefill

では、ここでもうひとつの問題です。つぎの4つの言語で、話す
人の数がおおい順に並べると、どうなるでしょうか

ただし生まれてはじめて身につけたことばという条件で

\begin{itemize}
 \item  中国語 (Mandarin Chinese)
 \item  スペイン語 (Spanish)
 \item  英語 (English)
 \item  ヒンディー語 (Hindi)
\end{itemize}

皆さんで自分なりの答えを決めましょう。

\hrulefill

答えです。

グラフにしました。

上から順番に、
中国語、スペイン語、英語、ヒンディー語です。

これを見ると中国語が圧倒的におおいですねえ。

これ、単位は100万人ですから、中国語を話す人は1,000の目盛りのあたりですから
1,000$\times$100万人、ていうことはだいたい10億人くらいです。
スペイン語はその半分ですからまあだいたい5億人、英語とヒンディー語はまあだいたい
4億人ぐらいでしょうか。

\hrulefill

これは表にしたものです。

世界ではどうやら中国語を話す人がおおいようです。

であれば、外国語としてどうせ学ぶのであれば中国語がいいんじゃないのかなあと
思う人もいるかもしれませんね。
どうせ勉強するなら、話す人がおおいことばがよさそうだ、というわけでしょうか。


\hrulefill

ところで、2つ以上のことばを話す人がいますよね。

たとえば、
生まれて初めて身につけた第一言語は日本語だけれど、
家族の仕事の関係でアメリカに住んでいたので第二言語として英語も話すという人がいます。
なかにはさらに、韓国語も話せるよなんていう人もいます。

\hrulefill

さっき第一言語だと中国語を話す人がダントツでおおかったんですが、
第二言語とか、とにかくそのことばを話す人を加えるとどうでしょうか。

\hrulefill

第一言語として話す人の数はこうでしたね。

ここに第二言語とか第三言語とか、とにかくそのことばを話す人を加えてみます。

せーの。


\hrulefill

結果はこうです。

なんと逆転しましたね。

さっき3位だった英語が1位になりました。

\hrulefill

表にするとこうです。

英語は第一言語だけだと4億人弱で3位でしたが、
第二言語とか第三言語とか、とにかく英語を話す人が108、単位は100万人ですから、
10億8000万人、
これを加えると、
14億6000万人いるわけです。
英語は第一言語以外で話す人がすごくおおいことがわかります。

いっぽう、中国語は第一言語に限れば940、単位は100万人ですから9億4000万人でダントツで
1位でしたが、第二言語とか第三言語とかで話す人は1億9900人で、英語に比較すればわずかです。
で、あわせると11億3900万人ということです。

\hrulefill

もう一度整理します。

世界には 7,164 のことばがありますが、
話す人をぜんぶあわせると英語が3位から1位へ
英語が1位で中国語が2位

\hrulefill

これまでは話す人の数という
ではつぎに英語と中国のそれぞれを話す人が世界のどのあたりにいるのかみてみましょう



まず中国語から

\hrulefill

中国語を公用語としている国と地域を
オレンジで示しました。

ほぼ中国に一極集中ですね。

\hrulefill

どうやら、
中国語を話す人は中国に集中しているといってよさそうです。

もちろん世界の各地にある中華街China townには中国語を話す人がいるはずですが、
全体的に見れば、中国語を話す人は中国に集中しているといってよさそうです。

では英語はどうでしょうか。

\hrulefill

英語を公用語としている国と地域、事実上の公用語となっているところをブルーで示しました。

\hrulefill

英語を話す人は世界中にいることがわかります

\hrulefill

中国語を話す人は中国に集中しています、つまり局所的

いっぽう、英語は広く世界中で話されています

\hrulefill

ここまでを整理します。

わたしたちが学ぶ英語とは

\begin{itemize}
 \item 英語は話す人の数がいちばんおおい

 \item 英語は広く世界中で話されている
\end{itemize}

 

つまり英語は世界の共通語といっていいだろうということになりますね。



\hrulefill

じっさい、英語はさまざまな分野で使われています

\begin{itemize}
 \item ビジネス
 \item 医療
 \item スポーツ
 \item 芸術(文学・音楽) などなど
\end{itemize}
さまざまな分野で英語が共通のことばになっています


\hrulefill

英語を学ぶと、いまいったようなさまざまな分野において、
楽しいことにであえることになるわけです。

さあいっしょに楽しく英語の勉強をしましょう。

\hrulefill

English is fun.

英語は楽しいぞー。



\subsection*{最後に}

それでは、つぎの授業でお会いしましょう。


\hrulefill
\hrulefill
\hrulefill
\hrulefill

\end{document}
