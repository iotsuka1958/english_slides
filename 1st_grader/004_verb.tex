\documentclass[aspectratio=169]{beamer}
\usepackage[no-math,deluxe,haranoaji]{luatexja-preset}
\renewcommand{\kanjifamilydefault}{\gtdefault}
\renewcommand{\emph}[1]{{\upshape\bfseries #1}}
\usetheme{metropolis}
\metroset{block=fill}
\setbeamertemplate{navigation symbols}{}
\usecolortheme[rgb={0.7,0.2,0.2}]{structure}
%%%%%%%%%%%%%%%%%%%%%%%%%%%
%% さまざまなアイコン
%%%%%%%%%%%%%%%%%%%%%%%%%%%
\usepackage{fontawesome}
%%%%%%%%%%%%%%%%%%%%%%%%%%%
\usepackage{tikz}
\usetikzlibrary{backgrounds}
\usepackage{xcolor}
\usepackage{amsmath}
%%%%%%%%%%%%%%%%%%%%%%%%%%%
%% 場合分け
\usepackage{cases}
%%%%%%%%%%%%%%%%%%%%%%%%%%%
%% 音声リンク表示
\newcommand{\myaudio}[1]{\href{#1}{\faVolumeUp}}
%%%%%%%%%%%%%%%%%%%%%%%%%%%
% \myEmph コマンドの定義
%\newcommand{\myEmph}[3]{%
%    \textbf<#1>{\color<#1>{#2}{#3}}%
%}
\usepackage{xparse} % xparseパッケージの読み込み
\NewDocumentCommand{\myEmph}{O{} m m}{%
    \def\argOne{#1}%
    \ifx\argOne\empty
        \textbf{\color{#2}{#3}}% オプション引数が省略された場合
    \else
        \textbf<#1>{\color<#1>{#2}{#3}}% オプション引数が指定された場合
    \fi
}
%%%%%%%%%%%%%%%%%%%%%%%%%%%
\title{English is fun.\,\,{}---I play the guitar.---}
\author{}
\institute[]{}
\date[]

%%%%%%%%%%%%%%%%%%%%%%%%%%%%
%% TEXT
%%%%%%%%%%%%%%%%%%%%%%%%%%%%
\begin{document}
\begin{frame}[plain]
  \titlepage
\end{frame}

\section*{授業の流れ}
\begin{frame}[plain]
  \frametitle{授業の流れ}
  \tableofcontents
\end{frame}

\section{一般動詞}
\begin{frame}<1-36>[plain,label=example]\frametitle{一般動詞って?}
 % \setbeamercovered{transparent}
  \begin{enumerate}
   \item<1-> \myEmph[14]{red}{I} \myEmph[15,27,28]{blue}{like} basketball. \onslide*<2>{わたしはバスケットボールが好きです。}\onslide*<30>{\footnotesize  like: 好き、好む basketball: バスケットボール}
   \item<3-> \myEmph[16]{red}{We} \myEmph[17,27,28]{blue}{walk} to school. \onslide*<4>{われわれは歩いて学校へ行きます。}\onslide*<31>{\footnotesize  walk: 歩く}
   \item<5-> \myEmph[18]{red}{They} \myEmph[19,27,28]{blue}{speak} English in Australia. \onslide*<6>{オースタラリアでは英語が話されています。}\onslide*<32>{\footnotesize  speak: 話す Australia:オーストラリア}
   \item<7-> \myEmph[20]{red}{I} \myEmph[21,27,28]{blue}{drink} tea every morning. \onslide*<8>{わたしは毎朝お茶を飲みます。}\onslide*<33>{\footnotesize  drink: 飲む tea: お茶 every morning: 毎朝}
   \item<9-> \myEmph[22]{red}{They} \myEmph[23,27,28]{blue}{study} math every day. \onslide*<10>{彼らは毎日数学を勉強します。}\onslide*<34>{\footnotesize study: 勉強する math: 数学 every day: 毎日}
   \item<11-> \myEmph[24]{red}{I} \myEmph[25,27,28]{blue}{play} the guitar. \onslide*<12>{わたしはギターをひきます。}\onslide*<35>{\footnotesize  play: (楽器などを)演奏する guitar: ギター}
  \end{enumerate}

\bigskip

\begin{exampleblock}<28->{Topics for Today}
\begin{itemize}
 \item<1->  be動詞以外を一般動詞といいます
 \item<2-> 一般動詞の意味はいろいろです
\end{itemize}
      \end{exampleblock}


% Embed the sound file
\onslide<36>{%
\myaudio{./audio/004_verb_01.mp3}\,\,{}Listen carefully.(注意して聞いてください)
}

\end{frame}

\section{be動詞と一般動詞}
\begin{frame}<1-11>[plain]\frametitle{be動詞と一般動詞}

\begin{columns}
\begin{column}[t]{.38\textwidth}
\begin{block}{be動詞}
be動詞
{\footnotesize
$
 \left\{
\begin{array}{l}
\text{am}\\
\text{are}\\
\text{is}
\end{array}\right\}
$
}
は$=$の意味
\end{block}
\end{column}
\pause
\begin{column}[t]{.52\textwidth}
\begin{block}{一般動詞(be動詞以外のすべて)}
\begin{tabular}[t]{ll}
go(行く)&walk(歩く)\pause\\
have(持っている)&like(好き)\pause\\
speak(話す)&study(勉強する)\pause\\
eat(食べる)& drink(飲む)\pause\\
\multicolumn{2}{l}{play(遊ぶ、演奏する)\pause}\\
\multicolumn{2}{l}{ほかにもいろいろあります}\\
\end{tabular}

\end{block}
\end{column}
\end{columns}


\bigskip
\pause
\hyperlink{example}{\beamerreturnbutton{pronunciation}}
\begin{exampleblock}{Topics for Today}
\begin{itemize}[<+- |alert@+>]
 \item  be動詞以外を一般動詞といいます
 \item 一般動詞はたくさんあります。意味はいろいろです
\end{itemize}
     \end{exampleblock}

\end{frame}


\begin{frame}[plain]\frametitle{be動詞と一般動詞}
\begin{tikzpicture}

% グリッドを描画(5mm刻み)
%\draw[step=1cm, gray!20, very thin] (-6,0) grid (6,5);

\node[rectangle, rounded corners, draw=black, fill=white] (verb) at (-3,5) {動詞};



% 左側のノード(be動詞)
\node[circle, draw=black, line width=1pt, fill=yellow!30, minimum size=2cm] (be) at (-2.5,2.5) {be動詞{\footnotesize ($=$)}};

% 右側のノード(一般動詞)
\node[rectangle, draw=black, line width=1pt, fill=pink!30, minimum width=4cm, minimum height=1.5cm, inner sep=20pt] (general) at (4,2.5) {一般動詞\,\,{}$\left\{\begin{array}{ll}
\text{go}&\text{walk}\\
\text{have}&\text{like}\\
\text{speak}&\text{study}\\
\text{eat}& \text{drink}\\
\text{play}&\text{ほかにもたくさん}\\
\end{array}\right.$};


% 大きな長方形を描画して2つのノードを包む
\begin{scope}[on background layer]
\draw[draw=black!50, fill=blue!10, rounded corners, line width=1.5pt]
    ([xshift=-25pt,yshift=50pt]be.north west) rectangle ([xshift=8pt,yshift=-12pt]general.south east);
\end{scope}


\end{tikzpicture}

\end{frame}



\section{わたし、あなた、それ以外}
\begin{frame}[plain]\frametitle{人称}
\begin{description}
\item[1人称] 話し手(自分)のこと\pause{}\,\,{} $\longrightarrow$ Iとweだけ\pause
\item[2人称] 聞き手(相手)のこと\pause{}\,\,{} $\longrightarrow$ youだけ\pause
\item[3人称] 1人称、2人称以外\pause{}\hspace{15pt} $\longrightarrow$ \begin{tabular}[t]{@{}l}
he\\she\\it\\they\\the teacher\\this pencil\\the cats \end{tabular}
\end{description}
\end{frame}


\section{Exercise}
\begin{frame}<1-21>[plain]\frametitle{Exercises}
 % \setbeamercovered{transparent}
  \begin{enumerate}
   \item<1-> I am a student. \onslide*<2>{わたしは生徒です。}\onslide*<15-21>{\footnotesize  a:1つの、1人の student: 生徒、学生}
   \item<3-> You play the guitar very well. \onslide*<4>{あなたはギターをとても上手にひく。}\onslide*<16-21>{\footnotesize  guitar: ギター very well: とてもじょうずに}
  \item<5-> We drink coffee every morning. \onslide*<6>{われわれは毎朝コーヒーを飲みます。}\onslide*<17-21>{\footnotesize drink: 飲む coffee: コーヒー every morning: 毎朝}
    \item<7-> They go to school by bus. \onslide*<8>{われわれはバスで通学します。}\onslide*<18-21>{\footnotesize  by bus: バスで}
   \item<9-> The sky is blue. \onslide*<10>{空は青い。}\onslide*<19-21>{\footnotesize  the sky: 空 blue: 青い}
   \item<11-> Birds sing in the morning. \onslide*<12>{鳥は朝鳴きます。}\onslide*<20-21>{\footnotesize  bird: 鳥 sing: 歌う}
  \end{enumerate}

\bigskip

\begin{exampleblock}<14->{Topics for Today}
\begin{itemize}[<+->]
 \item 1人称はIとweだけ
 \item 2人称はyouだけ
 \item それ以外はぜんぶ3人称です
\end{itemize}
\end{exampleblock}


% Embed the sound file
\onslide<21>{%
\myaudio{audio/002_be_01.mp3}\,\,{}Listen carefully.(注意して聞いてください)

}
\end{frame}

\end{document}
!20
