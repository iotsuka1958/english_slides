\documentclass[aspectratio=169,xcolor={dvipsnames,table}]{beamer}
\usepackage[no-math,deluxe,haranoaji]{luatexja-preset}
\renewcommand{\kanjifamilydefault}{\gtdefault}
\renewcommand{\emph}[1]{{\upshape\bfseries #1}}
%\usetheme{moloch}
\usetheme{metropolis}
\metroset{block=fill}
\setbeamertemplate{navigation symbols}{}
\setbeamertemplate{blocks}[rounded][shadow=false]
\usecolortheme[rgb={0.7,0.2,0.2}]{structure}
%%%%%%%%%%%%%%%%%%%%%%%%%%%
%% さまざまなアイコン
%%%%%%%%%%%%%%%%%%%%%%%%%%%
\usepackage{fontawesome}
\usepackage{tipa}
\usepackage{twemojis}
\usepackage{figchild}
\usepackage{utfsym}
%%%%%%%%%%%%%%%%%%%%%%%%%%%
\usepackage{tikz}
\usetikzlibrary{backgrounds,decorations.pathmorphing, shapes}
\usepackage{tcolorbox}
\usepackage{xcolor}
\usepackage{amsmath}
\usepackage{manfnt}
\usepackage{array}
\usepackage{pxrubrica}
%%%%%%%%%%%%%%%%%%%%%%%%%%%
\newcommand*\myCrossedOut[2]{%
  \tikz[baseline=(T.base)]
    \node[draw=#1, thick, shape=cross out, decorate,
      inner sep=2pt, outer sep=0pt,
      decoration={random steps, segment length=2pt, amplitude=0.4pt}]
      (T) {#2};}
%%%%%%%%%%%%%%%%%%%%%%%%%%%
\usepackage{booktabs}
\usepackage{circledsteps}

%%%%%%%%%%%%%%%%%%%%%%%%%%%
%% 場合分け
\usepackage{cases}
%%%%%%%%%%%%%%%%%%%%%%%%%%%
\newcommand{\dnote}[1]{%
    \noindent % I guess this is intended...
    \begin{tabular}{@{}m{0.13\textwidth}@{}m{0.87\textwidth}@{}}%
        \huge\textdbend &#1%
    \end{tabular}%
    \par % ... and this too.
}
%%%
\newcommand{\ddnote}[1]{%
    \noindent % I guess this is intended...
    \begin{tabular}{@{}m{0.2\textwidth}@{}m{0.8\textwidth}@{}}%
        \huge\textdbend\,\textdbend &#1%
    \end{tabular}%
    \par % ... and this too.
}
%%%%%%%%%%%%%%%%%%%%%%%%
% \myAnch{<名前>}{<色>}{<テキスト>}
% 指定のテキストを指定の色の四角枠で囲み, 指定の名前をもつTikZの
% ノードとして出力する. 図には remeber picture 属性を付けている
% ので外部から参照可能である.
\newcommand*{\myAnch}[3]{%
  \tikz[remember picture,baseline=(#1.base)]
    \node[draw,rectangle,#2,thick] (#1) {\normalcolor #3};
}
%%%%%%%%%%%%%%%%%%%%%%%%%%%
%%%%%%%%%%%%%%%%%%%%%%%%%%%%
%% 音声リンク表示
\newcommand{\myaudio}[1]{\href{#1}{\faVolumeUp}}
%%%%%%%%%%%%%%%%%%%%%%%%%%%
% \myEmph コマンドの定義
%\newcommand{\myEmph}[3]{%
%    \textbf<#1>{\color<#1>{#2}{#3}}%
%}
\usepackage{xparse} % xparseパッケージの読み込み
\NewDocumentCommand{\myEmph}{O{} m m}{%
    \def\argOne{#1}%
    \ifx\argOne\empty
        \textbf{\color{#2}{#3}}% オプション引数が省略された場合
    \else
        \textbf<#1>{\color<#1>{#2}{#3}}% オプション引数が指定された場合
    \fi
}
%%%%%%%%%%%%%%%%%%%%%%%%%%
%% Change alert block colors
%%% 1- Block title (background and text)
\setbeamercolor{block title alerted}{fg=mDarkTeal, bg=mLightBrown!45!yellow!45}
\setbeamercolor{block title example}{fg=magenta!10!black, bg=mLightGreen!70}
%%% 2- Block body (background)
\setbeamercolor{block body alerted}{bg=mLightBrown!25}
\setbeamercolor{block body example}{bg=mLightGreen!15}
%%%%%%%%%%%%%%%%%%%%%%%%%%%
\AtBeginSection[%
]{%
  \begin{frame}[plain]\frametitle{授業の流れ}
     \tableofcontents[currentsection]
   \end{frame}%
}
%%%%%%%%%%%%%%%%%%%%%%%%%%%
\title{English is fun.}
\subtitle{I play the guitar, and she sings.}
\author{}
\institute[]{}
\date[]

%%%%%%%%%%%%%%%%%%%%%%%%%%%%
%% TEXT
%%%%%%%%%%%%%%%%%%%%%%%%%%%%
\begin{document}
%%%%%%%%%%%%%%%%%%%%%%%%%%%
%%%%%%%%%%%%%%%%%%%%%%%%%%%%%%
\begin{frame}[label=title]
%\phantomsection\label{section-1}
\thispagestyle{empty}
\titlepage
\end{frame}
%%%%%%%%%%%%%%%%%%%%%%%%%%%%%%
\section*{授業の流れ}
\begin{frame}[plain]
  \frametitle{授業の流れ}
  \tableofcontents
\end{frame}
%%%%%%%%%%%%%%%%%%%%%%%%%%%%%%
\section{一般動詞とは}
\begin{frame}[plain,label=is_verb]\frametitle{一般動詞って?}
 % \setbeamercovered{transparent}
  \begin{enumerate}
   \item<1-> \myEmph[14-]{Maroon}{I} \myEmph[14-]{NavyBlue}{like} basketball. \onslide*<2-12>{{\small わたしはバスケットボールが好きです。}}\hfill\onslide*<21->{\footnotesize  like: 好き、好む basketball: バスケットボール}
   \item<3-> \myEmph[15-]{Maroon}{We} \myEmph[15-]{NavyBlue}{walk} to work. \onslide*<4-12>{{\small われわれは歩いて職場へ行きます。}}\hfill\onslide*<22->{\footnotesize  walk: 歩く}
   \item<5-> \myEmph[16-]{Maroon}{They} \myEmph[16-]{NavyBlue}{speak} English in Australia. \onslide*<6-12>{{\small オーストラリアでは英語が話されています。}}\hfill\onslide*<23->{\footnotesize  speak: 話す Australia:オーストラリア}
   \item<7-> \myEmph[17-]{Maroon}{I} \myEmph[17-]{NavyBlue}{drink} tea every morning. \onslide*<8-12>{{\small わたしは毎朝お茶を飲みます。}}\hfill\onslide*<24->{\footnotesize  drink: 飲む tea: お茶 every morning: 毎朝}
   \item<9-> \myEmph[18-]{Maroon}{They} \myEmph[18-]{NavyBlue}{study} math every day. \onslide*<10-12>{{\small 彼らは毎日数学を勉強します。}}\hfill\onslide*<25->{\footnotesize study: 勉強する math: 数学 every day: 毎日}
   \item<11-> \myEmph[19-]{Maroon}{I} \myEmph[19-]{NavyBlue}{play} the guitar. \onslide*<12>{{\small わたしはギターをひきます。}}\hfill\onslide*<26->{\footnotesize  play: (楽器などを)演奏する guitar: ギター}
  \end{enumerate}

\bigskip

\begin{block}<20->{Topics for Today}
\begin{itemize}\setbeamertemplate{items}[square]
 \item  {be動詞以外を\kenten{一般動詞}といいます}
 \item<27-> {一般動詞の意味はいろいろです}
\end{itemize}
      \end{block}


% Embed the sound file

\hfill{\tiny 0253}\,\scriptsize \myaudio{./audio/004_verb_01.mp3}

\end{frame}
%%%%%%%%%%%%%%%%%%%%%%%%%%%%%%%
\section{be動詞と一般動詞}
%%%%%%%%%%%%%%%%%%%%%%%%%%%
\begin{frame}[plain,label=point1]\frametitle{be動詞と一般動詞}

\begin{columns}
\begin{column}[t]{.38\textwidth}
\begin{exampleblock}{be動詞}
be動詞
{\footnotesize
$
 \left\{
\begin{array}{l}
\text{am}\\
\text{are}\\
\text{is}
\end{array}\right\}
$
}
は$=$の働き
\end{exampleblock}
\end{column}
\pause
\begin{column}[t]{.52\textwidth}
\begin{alertblock}{一般動詞(be動詞以外のすべて)}
\pause
\begin{tabular}[t]{ll}
go(行く)&walk(歩く)\pause\\
have(持っている)&like(好き)\pause\\
speak(話す)&study(勉強する)\pause\\
eat(食べる)& drink(飲む)\pause\\
\multicolumn{2}{l}{play(遊ぶ、演奏する)\pause}\\
\multicolumn{2}{r}{ほかにもいろいろあります}\\
\end{tabular}
\end{alertblock}
\end{column}
\end{columns}


\bigskip
\pause
\begin{block}{Topics for Today}
\setbeamercolor{alerted text}{fg=Maroon} 
\begin{itemize}[<+- |alert@+>]\setbeamertemplate{items}[square]
 \item  be動詞以外を一般動詞といいます
 \item 一般動詞はたくさんあります。意味はいろいろです
\end{itemize}
     \end{block}

\end{frame}
%%%%%%%%%%%%%%%%%%%%%%%%%%
\begin{frame}[plain,label=figure]\frametitle{be動詞と一般動詞}
\begin{tikzpicture}

% グリッドを描画(5mm刻み)
%\draw[step=1cm, gray!20, very thin] (-6,0) grid (6,5);

\node[rectangle, rounded corners, draw=black, fill=white] (verb) at (-3,5) {動詞};

% 左側のノード(be動詞)
\node[circle, draw=black, line width=1pt, fill=yellow!30, minimum size=2cm] (be) at (-2.5,2.5) {be動詞{\footnotesize ($=$)}};

% 右側のノード(一般動詞)
\node[rectangle, draw=black, line width=1pt, fill=pink!30, minimum width=4cm, minimum height=1.5cm, inner sep=20pt] (general) at (4,2.5) {一般動詞\,\,{}$\left\{\begin{array}{ll}
\text{go}&\text{walk}\\
\text{have}&\text{like}\\
\text{speak}&\text{study}\\
\text{eat}& \text{drink}\\
\text{play}&\text{ほかにもたくさん}\\
\end{array}\right.$};


% 大きな長方形を描画して2つのノードを包む
\begin{scope}[on background layer]
\draw[draw=black!50, fill=NavyBlue!10, rounded corners, line width=1.5pt]
    ([xshift=-25pt,yshift=50pt]be.north west) rectangle ([xshift=8pt,yshift=-12pt]general.south east);
\end{scope}


\end{tikzpicture}

\end{frame}
%%%%%%%%%%%%%%%%%%%%%%%%%%%%%%%%%%%%%%%%%%%%%%%%%%
\section{わたし、あなた、それ以外}
\subsection{人称とは}
\begin{frame}[plain,label=ninsyo]\frametitle{人称}

\begin{alertblock}{人称とは}
\begin{description}
\item[1人称] 話し手(自分)のこと\pause{}\,\,{} $\longrightarrow$ \textbf{I}と\textbf{we}だけ\pause
\item[2人称] 聞き手(相手)のこと\pause{}\,\,{} $\longrightarrow$ \textbf{you}だけ\pause
\item[3人称] 1人称、2人称以外\pause{}\hspace{15pt} $\longrightarrow$%
 \begin{tabular}[t]{@{\,\,}l}
(I, we, you以外なんでも)\\\pause
he\\\pause
she\\\pause
it\\\pause
they\\\pause
the teacher\\\pause
this pencil\\\pause
the cats\\\pause
\ldots\hfill{}きりがない
 \end{tabular}
\end{description}
\end{alertblock}
\end{frame}

\begin{frame}[plain,t]\frametitle{Exercises}
\onslide*<1->{つぎの文の主語は何人称ですか?}\hfill{\tiny 0253}\,{\scriptsize \myaudio{audio/004_verb_02.mp3}}

 % \setbeamercovered{transparent}
  \begin{enumerate}
   \item<1-> \myEmph[2,8]{Maroon}{I} am a student. \onslide*<2-10>{{\small わたしは生徒です。}}\hfill\onslide*<11->{\scriptsize  a:1つの student: 生徒、学生}
   \item<1-> \myEmph[3,9]{Maroon}{You} play the guitar very well. \onslide*<3-10>{{\small あなたはギターをとても上手にひく。}}\hfill\onslide*<12->{\scriptsize  guitar: ギター very well: とてもじょうずに}
  \item<1-> \myEmph[4,8]{Maroon}{We} drink coffee every morning. \onslide*<4-10>{{\small われわれは毎朝コーヒーを飲みます。}}\hfill\onslide*<13->{\scriptsize drink: 飲む coffee: コーヒー every morning: 毎朝}
    \item<1-> \myEmph[5,10]{Maroon}{They} go to the park by bus. \onslide*<5-10>{{\small 彼らはバスで公園に行く}。}\hfill\onslide*<14->{\scriptsize  by bus: バスで}
   \item<1-> \myEmph[6,10]{Maroon}{The sky} is blue. \onslide*<6-10>{{\small 空は青い。}}\hfill\onslide*<15->{\scriptsize  the sky: 空 blue: 青い}
   \item<1-> \myEmph[7,10]{Maroon}{Birds} sing in the morning. \onslide*<7-10>{{\small 鳥は朝歌います。}}\hfill\onslide*<16->{\scriptsize  bird: 鳥 sing: 歌う}
  \end{enumerate}

\begin{block}<8->{Topics for Today}
\begin{itemize}\setbeamertemplate{items}[square]
 \item<8-> {1人称はIとweだけ}
 \item<9-> {2人称はyouだけ}
 \item<10-> {それ以外はぜんぶ3人称です}
\end{itemize}
\end{block}
\end{frame}
%%%%%%%%%%%%%%%%%%%%%%%%%%%%%%%%%%%%%%
\section{主語によって動詞のかたちが定まる}
\begin{frame}[plain]\frametitle{主語によって動詞のかたちが定まる}
 % \setbeamercovered{transparent}
  \begin{enumerate}
   \item<1-> \myEmph[2-13]{Maroon}{I} \myEmph[2-13]{NavyBlue}{like} music. \hfill\onslide*<2->{\scriptsize  like: 好き、好む music: 音楽}
   \item<1-> \myEmph[3-13]{Maroon}{We} \myEmph[3-13]{NavyBlue}{like} music.
   \item<1-> \myEmph[4-13]{Maroon}{You} \myEmph[4-13]{NavyBlue}{like} music.
   \item<1-> \myEmph[5-]{Maroon}{He} \myEmph[5-]{NavyBlue}{likes} music.
   \item<1-> \myEmph[6-]{Maroon}{She} \myEmph[6-]{NavyBlue}{likes} music.
   \item<1-> \myEmph[7-]{Maroon}{Paul} \myEmph[7-]{NavyBlue}{likes} music.
   \item<1-> \myEmph[8-13]{Maroon}{They} \myEmph[8-13]{NavyBlue}{like} music.
  \end{enumerate}
\begin{block}<9->{Topics for Today}\small
\begin{itemize}\setbeamertemplate{items}[square]
 \item<10-> 主語が1人称のとき、動詞はそのまま\hfill{I \textbf{sing}. / We \textbf{sing}.}
 \item<11-> 主語が2人称のとき、動詞はそのまま\hfill{You \textbf{sing}.}
 \item<12-> \myEmph[14-]{Maroon}{主語が3人称のとき、単数なら\,---s\,となります}\hfill{She \textbf{sings}. / He \textbf{sings}.}
 \item<13->[] \hspace{96pt}複数ならそのまま\hfill{They \textbf{sing}.}
\end{itemize}
      \end{block}
\end{frame}
%%%%%%%%%%%%%%%%%%%%%%%%%%%%%%%%%%%%%%%%%%%%%%%%%%
\section{3単現}
\subsection{3単現のs}
\begin{frame}<1-5>[plain,label=3tangen]\frametitle{3単現のs}
\Large

\myEmph[2]{Maroon}{現在}のことを表す文では\\
主語が\myEmph[2]{Maroon}{3人称}\myEmph[2]{Maroon}{単数}のとき、動詞の最後に\myEmph[2]{Maroon}{s}をつけます

\pause

\mbox{}\hfill$\longrightarrow{}$これを\myAnch{T1}{Maroon}{3単現の\Circled[fill color=Maroon!30]{\,S\,}}といいます\pause

\vspace{20pt}

She play\myAnch{T2}{Maroon}{s}\,{}the guitar.

\mbox{}\hfill{}He live\myAnch{T3}{Maroon}{s}\,{}in Boston.

\mbox{}\hfill{}Jane drink\myAnch{T4}{Maroon}{s}\,{}coffee.\hfill\mbox{}

% ノード間を結ぶ矢印を別のTikZ環境で描く
\begin{tikzpicture}[remember picture,overlay]
\onslide<3->{\draw[->,Maroon,line width=2pt, opacity=.5] (T1.south)to[out=-110,in=40](T2.north);}
{\onslide<4->{\draw[->,Maroon,line width=2pt, opacity=.5] (T1.south)to[out=-90,in=120](T3.north);}
\onslide<5->\draw[->,Maroon,line width=2pt, opacity=.5] (T1.south)to[out=-90,in=90](T4.north);}
\end{tikzpicture}
\end{frame}
%%%%%%%%%%%%%%%%%%%%%%%%%%%%%%%%%%%%%
\begin{frame}[plain,label=ex]\frametitle{Exercises}

{\small つぎの空所に適当な動詞を選んで補って、現在のことを表す英文をつくってください。なお、必要があれば、形を変えてください}

\bigskip

\begin{columns}
\begin{column}{.5\textwidth}
 % \setbeamercovered{transparent}
  \begin{enumerate}
%  \item I~~(\onslide<2,8,9>{\textcolor{orange}{~have~}})~~two cats.
   \item We~~(\onslide<2->{\textcolor{orange}{~go~}})~~to the park by bus.
   \item  She~~(\onslide<3->{\textcolor{orange}{~drinks~}})~~coffee every morning.
   \item Tom~~(\onslide<4->{\textcolor{orange}{~eats~}})~~bread for breakfast.
   \item Jennifer~~(\onslide<5->{\textcolor{orange}{~lives~}})~~in Boston.
   \item They~~(\onslide<6->{\textcolor{orange}{~speak~}})~~English.
  \end{enumerate}
\end{column}
\begin{column}{.45\textwidth}
\begin{tcolorbox}[title=この中から選んでください]
live / go / speak / drink / eat
\end{tcolorbox}
\end{column}
\end{columns}

\bigskip

% Embed the sound file
\mbox{}\hfill{\tiny 0228}\,{\scriptsize \myaudio{audio/004_verb_03.mp3}}

\end{frame}
%%%%%%%%%%%%%%%%%%%%%%%%%%%%%%%%%%%%%%%%%%%%

\subsection{s以外の3単現}
\begin{frame}[plain,label=exception]\frametitle{s以外の3単現}

  \begin{enumerate}
   \item<1-> She \myEmph[2-12,13,14]{Maroon}{drinks} coffee every morning. \hspace{10pt}\onslide<2,13-14,19>{(drink \textcolor{Maroon}{$+$ s })\hfill{}原則}
   \item<3-> He \myEmph[3-12,4,15,16]{Maroon}{goes} to the station by bus. \hspace{10pt}\onslide<4-10,15,16,19>{(go \textcolor{Maroon}{ $+$ es })\hfill\myCrossedOut{Maroon}{gos}}
  \item<5-> John \myEmph[5-12,15,16]{Maroon}{teaches} math. \hspace{10pt}\onslide<6-10,15,16,19>{(teach \textcolor{Maroon}{ $+$ es })\hfill\myCrossedOut{Maroon}{teachs}}
    \item<7-> My father \myEmph[7-12,15,16]{Maroon}{washes} his car every Sunday. \hspace{10pt}\onslide<8-10,15,16,19>{(wash \textcolor{Maroon}{ $+$ es })\hfill\myCrossedOut{Maroon}{washs}}
   \item<9-> Jane \myEmph[9-12,15,16]{Maroon}{watches} television after supper. \hspace{10pt}\onslide<10,15,16,19>{(watch \textcolor{Maroon}{ $+$ es })\hfill\myCrossedOut{Maroon}{watchs}}
   \item<11-> She \myEmph[11-12,17,18]{Maroon}{has} two cats. \hspace{10pt}\onslide<12,17,18,19>{(haveの3単現は\textcolor{Maroon}{has})\hfill\myCrossedOut{Maroon}{haves}}
  \end{enumerate}

\vspace{-3pt}

\setbeamercovered{transparent}
\begin{block}<13->{Topics for Today}
\begin{itemize}\setbeamertemplate{items}[square]
 \item<1-> \myEmph[13]{Maroon}{3単現の原則:\,\,---s}\\
\mbox{}\hspace{70pt}\onslide<14->{{\scriptsize 例えば}{\small drinks, eats, likes, lives, plays, speaks, walks\ldots{}\,\,{\scriptsize $\leftarrow$覚える必要なし}}}
 \item<2-> \onslide<15->{\myEmph[15]{Maroon}{一部の動詞:\,\,---es}}\\
\mbox{}\hspace{70pt}\onslide<16->{{\small goes, teaches, washes, watches}\,\,{\small $\leftarrow$覚えよう!}}
 \item<3-> \onslide<17->{\myEmph[17]{Maroon}{haveの3単現:\,\, has}\hspace{20pt}}\onslide<18->{\small $\longleftarrow$\, これも覚えよう!\myCrossedOut{Maroon}{haves}}
\end{itemize}
      \end{block}

\vspace{-15pt}

\mbox{}\hfill{\tiny 0300}\,{\scriptsize \myaudio{audio/004_verb_04.mp3}}


\end{frame}
%%%%%%%%%%%%%%%%%%%%%%%%%%%%%%%%%%%%%%
%%%%%%%%%%%%%%%%%%%%%%%%%%%%%%
%\againframe<19>{exception}
%%%%%%%%%%%%%%%%%%%%%%%%%%%%%%%%%%%%%%%%%
\begin{frame}[plain]{注意するべき3単現}
 \begin{center}
 \rowcolors{2}{NavyBlue!40}{yellow!40}
\begin{tabular}{rlll}\toprule
&{\small 原形}&{\small 3単現}&{\small 発音}\\\midrule
1&\visible<1->{play}&\visible<2->{{\small plays}}&\visible<9->{\textipa{/z/}}\\
2&\visible<1->{drink}&\visible<3->{{\small drinks}}&\visible<10->{\textipa{/s/}}\\
3&\visible<1->{go}&\visible<4->{{\small goes}}&\visible<12->{\textipa{/z/}}\\
4&\visible<1->{teach}&\visible<5->{{\small teaches}}&\visible<13->{\textipa{/\textsci{}z/}}\\
5&\visible<1->{wash}&\visible<6->{{\small washes}}&\visible<14->{\textipa{/\textsci{}z/}}\\
6&\visible<1->{watch}&\visible<7->{{\small watches}}&\visible<15->{\textipa{/\textsci{}z/}}\\
7&\visible<1->{have}&\visible<8->{{\small has}}&\visible<17->{\textipa{/h\'\ae z/}}\\
\bottomrule
\end{tabular}%
\end{center}

\visible<11->{\scriptsize 3単現の\,\Circled{\,s\,}\,の発音は\textipa{/z/}と濁るときと\textipa{/s/}と濁らないときがあります}

\visible<16->{{\scriptsize 3単現の\,\Circled{\,es\,}\,の発音は\textipa{/Iz/}となるときがあります}}



\mbox{}\hfill{\tiny 0422}\,{\scriptsize \myaudio{audio/004_verb_041.mp3}}

\end{frame}
%%%%%%%%%%%%%%%%%%%%%%%%%%%%%%%

%%%%%%%%%%%%%%%%%%%%%%%%%%%%%%%%%%%%%%%%%
\begin{frame}[plain,label=zzz]\frametitle{Exercises}

{\small 空所に適当な動詞を補いましょう。なお、必要があれば、形を変えてください}%
\mbox{}\hfill{\tiny 0306}\,{\scriptsize \myaudio{audio/004_verb_05.mp3}}

 % \setbeamercovered{transparent}
  \begin{enumerate}
   \item George (\onslide<2->{\textcolor{orange}{~has~}}) two cats.
{\footnotesize ジョージはネコを2匹飼っています。}\visible<2->{\myCrossedOut{Maroon}{haves}}
   \item My mother (\onslide<3->{\textcolor{orange}{~teaches~}}) music.{\footnotesize 母は音楽を教えています。}
   \item  She (\onslide<4->{\textcolor{orange}{~washes~}}) her car every Saturday.{\footnotesize 彼女は毎週土曜日に洗車します。}
   \item Tom (\onslide<5->{\textcolor{orange}{~goes~}}) to church every Sunday.{\footnotesize トムは毎週日曜日に教会へ行きます。}
   \item Jennifer (\onslide<6->{\textcolor{orange}{~watches~}}) television after supper.{\footnotesize ジェニファーは夕食後テレビをみます。}
   \item He (\onslide<8->{\textcolor{orange}{~studies~}}) English every day.{\footnotesize 彼は毎日英語を勉強します。} \onslide<7->{\myCrossedOut{Maroon}{studys}}\\
\mbox{}\hfill\onslide<9->{\scalebox{.25}{\includegraphics{./images/dbend.png}}\,\,\,{\small \textcolor{Maroon}{study}の3単現は\textcolor{Maroon}{studies}(\,\Circled{\,y\,}\,を\,\Circled{\,i\,}\,に変えて\,\Circled{\,es\,}\,をつける)}}
  \end{enumerate}

\vspace*{-10pt}

\begin{tcolorbox}[title=この中から選んでください]
\centering
 watch~~~/~~~~study~~~~/~~~have~~~~/~~~teach~~~~/~~~go~~~~/~~~wash
\end{tcolorbox}

\vspace{-5pt}
\end{frame}
%%%%%%%%%%%%%%%%%%%%%%%%%%%%%%%%%%%%%%%%%%
{
  \usebackgroundtemplate{\includegraphics[width=\paperwidth]{./images/Borne_Michelin_Virages.jpeg}}
  \begin{frame}[b]
    \frametitle{}
\tiny
\raggedleft
  \textcolor{white}{ ``Borne Michelin Virages'' by Roulex45 is licensed under CC BY-SA 3.0. }\\
   \textcolor{white}{To view a copy of this license, visit \url{https://creativecommons.org/licenses/by-sa/3.0/}}.
  \end{frame}
}
%%%%%%%%%%%%%%%%%%%%%%%%%%%%%%%%%%%%%%%%%%%%
\begin{frame}[plain]{Dangerous Bend}{危険な曲がり角}

\dnote{ちょっとむずかしい事項ですという標識\par{}でも、しっかり身につけると実力アップにつながります}

\bigskip

\ddnote{かなりむずかしい事項ですという標識\par{}いまはまだ理解できなくてもだいじょうぶ}


\vspace{40pt}

\small

\raggedleft
Dangerous Bendとは「危険な曲がり角」という意味
\end{frame}
%%%%%%%%%%%%%%%%%%%%%%%%%%%%%%%%%%%%%%%%%%%

%%%%%%%%%%%%%%%%%%%%%%%%%%%%%%%%%%%%%%%%%%%

\begin{frame}<16-18>[plain,label=hyo]{注意するべき3単現}


 \begin{center}
 \rowcolors{2}{NavyBlue!40}{yellow!40}
\begin{tabular}{rlll}\toprule
&{\small 原形}&{\small 3単現}&{\small 発音}\\\midrule
1&\visible<1->{play}&\visible<2->{{\small plays}}&\visible<10->{\textipa{/z/}}\\
2&\visible<1->{drink}&\visible<3->{{\small drinks}}&\visible<11->{\textipa{/s/}}\\
3&\visible<1->{go}&\visible<4->{{\small goes}}&\visible<12->{\textipa{/z/}}\\
4&\visible<1->{teach}&\visible<5->{{\small teaches}}&\visible<13->{\textipa{/\textsci{}z/}}\\
5&\visible<1->{wash}&\visible<6->{{\small washes}}&\visible<14->{\textipa{/\textsci{}z/}}\\
6&\visible<1->{watch}&\visible<7->{{\small watches}}&\visible<15->{\textipa{/\textsci{}z/}}\\
7&\visible<1->{have}&\visible<8->{{\small has}}&\visible<16->{\textipa{/z/}}\\
8&\visible<16->{study}&\visible<17->{{\small studies}}&\visible<18->{\textipa{/z/}}\\
\bottomrule
\end{tabular}%
\end{center}


\vspace{-40pt}
\visible<1-18>{\raisebox{\height}{\dbend}\,studyを追加しよう$\longrightarrow$}



\mbox{}\hfill{\tiny 0541}\,{\scriptsize \myaudio{audio/004_verb_06.mp3}}

\end{frame}
%%%%%%%%%%%%%%%%%%%%%%%%%%%%%%%%%%%%%%
\section{まとめ}
\againframe<11>[plain]{point1}
\againframe<13>[plain]{ninsyo}
\againframe<5>[plain]{3tangen}
\againframe<19>[plain]{hyo}
\end{document}


