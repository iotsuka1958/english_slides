\documentclass[aspectratio=169]{beamer}
\usepackage[no-math,deluxe,haranoaji]{luatexja-preset}
\renewcommand{\kanjifamilydefault}{\gtdefault}
\renewcommand{\emph}[1]{{\upshape\bfseries #1}}
\usetheme{metropolis}
\metroset{block=fill}
\setbeamertemplate{navigation symbols}{}
\usecolortheme[rgb={0.7,0.2,0.2}]{structure}
%%%%%%%%%%%%%%%%%%%%%%%%%%%
%% さまざまなアイコン
%%%%%%%%%%%%%%%%%%%%%%%%%%%
\usepackage{fontawesome}
%%%%%%%%%%%%%%%%%%%%%%%%%%%
\usepackage{tikz}
\usetikzlibrary{backgrounds}
\usepackage{tcolorbox}
\usepackage{xcolor}
\usepackage{amsmath}
%%%%%%%%%%%%%%%%%%%%%%%%%%%
%% 場合分け
\usepackage{cases}
%%%%%%%%%%%%%%%%%%%%%%%%%%%
% \myAnch{<名前>}{<色>}{<テキスト>}
% 指定のテキストを指定の色の四角枠で囲み, 指定の名前をもつTikZの
% ノードとして出力する. 図には remeber picture 属性を付けている
% ので外部から参照可能である.
\newcommand*{\myAnch}[3]{%
  \tikz[remember picture,baseline=(#1.base)]
    \node[draw,rectangle,#2] (#1) {\normalcolor #3};
}
%%%%%%%%%%%%%%%%%%%%%%%%%%%%
%% 音声リンク表示
\newcommand{\myaudio}[1]{\href{#1}{\faVolumeUp}}
%%%%%%%%%%%%%%%%%%%%%%%%%%%
% \myEmph コマンドの定義
%\newcommand{\myEmph}[3]{%
%    \textbf<#1>{\color<#1>{#2}{#3}}%
%}
\usepackage{xparse} % xparseパッケージの読み込み
\NewDocumentCommand{\myEmph}{O{} m m}{%
    \def\argOne{#1}%
    \ifx\argOne\empty
        \textbf{\color{#2}{#3}}% オプション引数が省略された場合
    \else
        \textbf<#1>{\color<#1>{#2}{#3}}% オプション引数が指定された場合
    \fi
}
%%%%%%%%%%%%%%%%%%%%%%%%%%%
\title{English is fun.\,\,{}---I play the guitar.---}
\author{}
\institute[]{}
\date[]

%%%%%%%%%%%%%%%%%%%%%%%%%%%%
%% TEXT
%%%%%%%%%%%%%%%%%%%%%%%%%%%%
\begin{document}
\begin{frame}[plain]
  \titlepage
\end{frame}

\section*{授業の流れ}
\begin{frame}[plain]
  \frametitle{授業の流れ}
  \tableofcontents
\end{frame}

\section{一般動詞とは}
\begin{frame}<1-37>[plain,label=example]\frametitle{一般動詞って?}
 % \setbeamercovered{transparent}
  \begin{enumerate}
   \item<1-> \myEmph[14]{red}{I} \myEmph[15,27-29]{blue}{like} basketball. \onslide*<2>{わたしはバスケットボールが好きです。}\onslide*<31>{\footnotesize  like: 好き、好む basketball: バスケットボール}
   \item<3-> \myEmph[16]{red}{We} \myEmph[17,27-29]{blue}{walk} to school. \onslide*<4>{われわれは歩いて学校へ行きます。}\onslide*<32>{\footnotesize  walk: 歩く}
   \item<5-> \myEmph[18]{red}{They} \myEmph[19,27-29]{blue}{speak} English in Australia. \onslide*<6>{オースタラリアでは英語が話されています。}\onslide*<33>{\footnotesize  speak: 話す Australia:オーストラリア}
   \item<7-> \myEmph[20]{red}{I} \myEmph[21,27-29]{blue}{drink} tea every morning. \onslide*<8>{わたしは毎朝お茶を飲みます。}\onslide*<34>{\footnotesize  drink: 飲む tea: お茶 every morning: 毎朝}
   \item<9-> \myEmph[22]{red}{They} \myEmph[23,27-29]{blue}{study} math every day. \onslide*<10>{彼らは毎日数学を勉強します。}\onslide*<35>{\footnotesize study: 勉強する math: 数学 every day: 毎日}
   \item<11-> \myEmph[24]{red}{I} \myEmph[25,27-29]{blue}{play} the guitar. \onslide*<12>{わたしはギターをひきます。}\onslide*<36>{\footnotesize  play: (楽器などを)演奏する guitar: ギター}
  \end{enumerate}

\bigskip

\begin{exampleblock}<28->{Topics for Today}
\begin{itemize}
 \item<1->  \myEmph[28]{orange}{be動詞以外を一般動詞といいます}
 \item<2-> \myEmph[29]{orange}{一般動詞の意味はいろいろです}
\end{itemize}
      \end{exampleblock}


% Embed the sound file
\onslide<37>{%
\myaudio{./audio/004_verb_01.mp3}\,\,{}Listen carefully.(注意して聞いてください)
}

\end{frame}

\section{be動詞と一般動詞}
\begin{frame}<1-11>[plain]\frametitle{be動詞と一般動詞}

\begin{columns}
\begin{column}[t]{.38\textwidth}
\begin{block}{be動詞}
be動詞
{\footnotesize
$
 \left\{
\begin{array}{l}
\text{am}\\
\text{are}\\
\text{is}
\end{array}\right\}
$
}
は$=$の意味
\end{block}
\end{column}
\pause
\begin{column}[t]{.52\textwidth}
\begin{block}{一般動詞(be動詞以外のすべて)}
\begin{tabular}[t]{ll}
go(行く)&walk(歩く)\pause\\
have(持っている)&like(好き)\pause\\
speak(話す)&study(勉強する)\pause\\
eat(食べる)& drink(飲む)\pause\\
\multicolumn{2}{l}{play(遊ぶ、演奏する)\pause}\\
\multicolumn{2}{l}{ほかにもいろいろあります}\\
\end{tabular}

\end{block}
\end{column}
\end{columns}


\bigskip
\pause
\hyperlink{example}{\beamerreturnbutton{pronunciation}}
\begin{exampleblock}{Topics for Today}
\begin{itemize}[<+- |alert@+>]
 \item  be動詞以外を一般動詞といいます
 \item 一般動詞はたくさんあります。意味はいろいろです
\end{itemize}
     \end{exampleblock}

\end{frame}


\begin{frame}[plain]\frametitle{be動詞と一般動詞}
\begin{tikzpicture}

% グリッドを描画(5mm刻み)
%\draw[step=1cm, gray!20, very thin] (-6,0) grid (6,5);

\node[rectangle, rounded corners, draw=black, fill=white] (verb) at (-3,5) {動詞};



% 左側のノード(be動詞)
\node[circle, draw=black, line width=1pt, fill=yellow!30, minimum size=2cm] (be) at (-2.5,2.5) {be動詞{\footnotesize ($=$)}};

% 右側のノード(一般動詞)
\node[rectangle, draw=black, line width=1pt, fill=pink!30, minimum width=4cm, minimum height=1.5cm, inner sep=20pt] (general) at (4,2.5) {一般動詞\,\,{}$\left\{\begin{array}{ll}
\text{go}&\text{walk}\\
\text{have}&\text{like}\\
\text{speak}&\text{study}\\
\text{eat}& \text{drink}\\
\text{play}&\text{ほかにもたくさん}\\
\end{array}\right.$};


% 大きな長方形を描画して2つのノードを包む
\begin{scope}[on background layer]
\draw[draw=black!50, fill=blue!10, rounded corners, line width=1.5pt]
    ([xshift=-25pt,yshift=50pt]be.north west) rectangle ([xshift=8pt,yshift=-12pt]general.south east);
\end{scope}


\end{tikzpicture}

\end{frame}


\section{わたし、あなた、それ以外}
\subsection{人称とは}
\begin{frame}[plain]\frametitle{人称}
\begin{description}
\item[1人称] 話し手(自分)のこと\pause{}\,\,{} $\longrightarrow$ Iとweだけ\pause
\item[2人称] 聞き手(相手)のこと\pause{}\,\,{} $\longrightarrow$ youだけ\pause
\item[3人称] 1人称、2人称以外\pause{}\hspace{15pt} $\longrightarrow$ \begin{tabular}[t]{@{}l}
he\\she\\it\\they\\the teacher\\this pencil\\the cats \end{tabular}
\end{description}
\end{frame}

\begin{frame}<1-29>[plain]\frametitle{Exercises}
\onslide*<1->{つぎの文の主語は何人称ですか?}
 % \setbeamercovered{transparent}
  \begin{enumerate}
   \item<1-> \myEmph[17]{red}{I} am a student. \onslide*<2>{わたしは生徒です。}\onslide*<23-28>{\footnotesize  a:1つの、1人の student: 生徒、学生}
   \item<3-> \myEmph[18]{red}{You} play the guitar very well. \onslide*<4>{あなたはギターをとても上手にひく。}\onslide*<24-28>{\footnotesize  guitar: ギター very well: とてもじょうずに}
  \item<5-> \myEmph[19]{red}{We} drink coffee every morning. \onslide*<6>{われわれは毎朝コーヒーを飲みます。}\onslide*<25-28>{\footnotesize drink: 飲む coffee: コーヒー every morning: 毎朝}
    \item<7-> \myEmph[20]{red}{They} go to school by bus. \onslide*<8>{われわれはバスで通学します。}\onslide*<26-28>{\footnotesize  by bus: バスで}
   \item<9-> \myEmph[21]{red}{The} sky is blue. \onslide*<10>{空は青い。}\onslide*<27-28>{\footnotesize  the sky: 空 blue: 青い}
   \item<11-> \myEmph[22]{red}{Birds} sing in the morning. \onslide*<12>{鳥は朝鳴きます。}\onslide*<28>{\footnotesize  bird: 鳥 sing: 歌う}
  \end{enumerate}


\begin{exampleblock}<14->{Topics for Today}
\begin{itemize}
 \item<14-> \myEmph[14]{orange}{1人称はIとweだけ}
 \item<15-> \myEmph[15]{orange}{2人称はyouだけ}
 \item<16->  \myEmph[16]{orange}{それ以外はぜんぶ3人称です}
\end{itemize}
\end{exampleblock}

\vspace*{-15pt}
% Embed the sound file
\onslide<29>{%
\myaudio{audio/004_verb_02.mp3}\,\,{}Listen carefully.(注意して聞いてください)

}
\end{frame}

\section{動詞のかたち}
\begin{frame}<1-20>[plain]\frametitle{主語によって動詞のかたちが定まる}
 % \setbeamercovered{transparent}
  \begin{enumerate}
   \item<1-> \myEmph[9,17]{red}{I} \myEmph[9,17]{blue}{like} music. \onslide*<2>{\footnotesize  like: 好き、好む music: 音楽}
   \item<3-> \myEmph[10,17]{red}{We} \myEmph[10,17]{blue}{like} music.
   \item<4-> \myEmph[11,17]{red}{You} \myEmph[11,17]{blue}{like} music.
   \item<5-> \myEmph[12,18]{red}{He} \myEmph[12,18]{blue}{likes} music.
   \item<6-> \myEmph[13,18]{red}{She} \myEmph[13,18]{blue}{likes} music.
   \item<7-> \myEmph[14,18]{red}{Paul} \myEmph[14,18]{blue}{likes} music.
   \item<8-> \myEmph[15,19]{red}{They} \myEmph[15,19]{blue}{like} music.
  \end{enumerate}
\begin{exampleblock}<17->{Topics for Today}
\begin{itemize}
 \item<1-> \myEmph[17]{orange}{主語が1人称、2人称のとき、動詞はそのまま}
 \item<2-> \myEmph[18]{orange}{主語が3人称で単数のとき、---sとなります}\myEmph[19]{orange}{(複数ならそのままです)}
\end{itemize}
      \end{exampleblock}


\end{frame}

\section{3単現}
\subsection{3単現のs}
\begin{frame}<1-3>[plain]\frametitle{3単現のs}
\Large

\myEmph[2]{orange}{現在}を表す文では、
主語が\myEmph[2]{orange}{3人称}で\myEmph[2]{orange}{単数}のとき\\動詞の最後に\myEmph[2]{orange}{--s}をつけます

\pause

\mbox{}\hfill$\longrightarrow{}$これを\myAnch{T1}{orange}{3単現の`s'}といいます\pause

She play\myAnch{T2}{orange}{s} \,\,{}the guitar.

% ノード間を結ぶ矢印を別のTikZ環境で描く
\begin{tikzpicture}[remember picture,overlay]
\draw[->,orange] (T1.south)to[out=-120,in=-30](T2.south);
\end{tikzpicture}
\end{frame}

\begin{frame}<1-9>[plain]\frametitle{Exercises}
つぎの空所に適当な動詞を選んで補って、現在のことを表す英文をつくってください。なお、必要があれば、形を変えてください

\begin{columns}
\begin{column}{.5\textwidth}
 % \setbeamercovered{transparent}
  \begin{enumerate}
   \item I (\onslide<2,8,9>{\textcolor{orange}{~have~}}) two cats.
   \item We (\onslide<3,8,9>{\textcolor{orange}{~go~}}) to school by bus.
   \item  She (\onslide<4,8,9>{\textcolor{orange}{~drinks~}}) coffee every morning.
   \item Tom (\onslide<5,8,9>{\textcolor{orange}{~eats~}}) bread for breakfast.
   \item Jennifer (\onslide<6,8,9>{\textcolor{orange}{~lives~}}) in Boston.
   \item They (\onslide<7,8,9>{\textcolor{orange}{~speak~}}) English.
  \end{enumerate}
\end{column}
\begin{column}{.45\textwidth}
\begin{tcolorbox}[title=この中から選んでください]
live, go, have, speak, drink, eat
\end{tcolorbox}
\end{column}
\end{columns}

% Embed the sound file
\onslide<9>{%
\myaudio{audio/004_verb_03.mp3}\,\,{}Listen carefully.(注意して聞いてください)

}
\end{frame}

\subsection{s以外の3単現}
\begin{frame}<1-16>[plain]\frametitle{s以外の3単現}

  \begin{enumerate}
   \item<1-> She \myEmph[2]{orange}{drinks} coffee every morning. \onslide<2>{\textcolor{orange}{(drink $+$ s )}}
   \item<3-> He \myEmph[3]{orange}{goes} to school by bus. \onslide<4>{\textcolor{orange}{(go $+$ es )}}
  \item<5-> John \myEmph[5]{orange}{teaches} math. \onslide<6>{\textcolor{orange}{(teach $+$ es )}}
    \item<7-> My father \myEmph[7]{orange}{washes} his car every Sunday. \onslide<8>{\textcolor{orange}{(wash $+$ es )}}
   \item<9-> Jane \myEmph[9]{orange}{watches} television after supper. \onslide<10>{\textcolor{orange}{(watch $+$ es )}}
   \item<11-> She \myEmph[11]{orange}{has} two cats. \onslide<12>{\textcolor{orange}{(haveの3単元はhas)}}
  \end{enumerate}

\begin{exampleblock}<13->{Topics for Today}
\begin{itemize}
 \item<1-> \myEmph[13]{orange}{3単現の原則: ---s}
 \item<2-> \myEmph[14]{orange}{一部の動詞((go, teach, wash, watch\ldots{})の3単現: ---es}
 \item<3-> \myEmph[15]{orange}{haveの3単現は: has}
\end{itemize}
      \end{exampleblock}

% Embed the sound file
\onslide<16>{%
\myaudio{audio/004_verb_04.mp3}\,\,{}Listen carefully.(注意して聞いてください)
}
\end{frame}

\begin{frame}<1-9>[plain]\frametitle{Exercises}
空所に適当な動詞を補ってください。なお、必要があれば、形を変えてください

 % \setbeamercovered{transparent}
  \begin{enumerate}
   \item George (\onslide<2,8,9>{\textcolor{orange}{~has~}}) two cats.
{\footnotesize ジョージはネコを2匹飼っています。}
   \item My mother (\onslide<3,8,9>{\textcolor{orange}{~teaches~}}) music.{\footnotesize 母は音楽を教えています。}
   \item  She (\onslide<4,8,9>{\textcolor{orange}{~washes~}}) her car every Saturday.{\footnotesize 彼女は毎週土曜日に洗車します。}
   \item Tom (\onslide<5,8,9>{\textcolor{orange}{~goes~}}) to church every Sunday.{\footnotesize トムは毎週日曜日に教会へ行きます。}
   \item Jennifer (\onslide<6,8,9>{\textcolor{orange}{~watches~}}) television after supper.{\footnotesize ジェニファーは夕食後テレビをみます。}
   \item He (\onslide<7,8,9>{\textcolor{orange}{~studies~}}) English every day.{\footnotesize 彼は毎日英語を勉強します。} \onslide<9>{\textcolor{orange}{studyの3単現はstdies}}
  \end{enumerate}

\begin{tcolorbox}[title=この中から選んでください]
\centering
 watch,~~~~~~~~study,~~~~~~~~have,~~~~~~~~teach,~~~~~~~~go,~~~~~~~~wash
\end{tcolorbox}

% Embed the sound file
\onslide<9>{%
\myaudio{audio/004_verb_05.mp3}\,\,{}Listen carefully.(注意して聞いてください)

}
\end{frame}
\end{document}



