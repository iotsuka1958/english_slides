\documentclass[aspectratio=169]{beamer}
\usepackage[no-math,deluxe,haranoaji]{luatexja-preset}
\renewcommand{\kanjifamilydefault}{\gtdefault}
\renewcommand{\emph}[1]{{\upshape\bfseries #1}}
\usetheme{metropolis}
\metroset{block=fill}
\setbeamertemplate{navigation symbols}{}
\setbeamertemplate{blocks}[rounded][shadow=false]
\usecolortheme[rgb={0.7,0.2,0.2}]{structure}
%%%%%%%%%%%%%%%%%%%%%%%%%%
%% Change alert block colors
%%% 1- Block title (background and text)
\setbeamercolor{block title alerted}{fg=mDarkTeal, bg=mLightBrown!45!yellow!45}
\setbeamercolor{block title example}{fg=magenta!10!black, bg=mLightGreen!70}
%%% 2- Block body (background)
\setbeamercolor{block body alerted}{bg=mLightBrown!25}
\setbeamercolor{block body example}{bg=mLightGreen!15}
%%%%%%%%%%%%%%%%%%%%%%%%%%%
%%%%%%%%%%%%%%%%%%%%%%%%%%%
%% さまざまなアイコン
%%%%%%%%%%%%%%%%%%%%%%%%%%%
%\usepackage{fontawesome}
\usepackage{fontawesome5}
\usepackage{figchild}
\usepackage{twemojis}
\usepackage{utfsym}
\usepackage{bclogo}
\usepackage{marvosym}
\usepackage{fontmfizz}
\usepackage{pifont}
\usepackage{phaistos}
\usepackage{worldflags}
\usepackage{jigsaw}
\usepackage{tikzlings}
\usepackage{tikzducks}
\usepackage{scsnowman}
\usepackage{epsdice}
\usepackage{halloweenmath}
\usepackage{svrsymbols}
\usepackage{countriesofeurope}
\usepackage{tipa}
\usepackage{manfnt}
%%%%%%%%%%%%%%%%%%%%%%%%%%%
\usepackage{tikz}
\usetikzlibrary{calc,patterns,decorations.pathmorphing,backgrounds}
\usepackage{tcolorbox}
\usepackage{tikzpeople}
\usepackage{circledsteps}
\usepackage{xcolor}
\usepackage{amsmath}
\usepackage{booktabs}
\usepackage{chronology}
\usepackage{signchart}
%%%%%%%%%%%%%%%%%%%%%%%%%%%
%% 場合分け
%%%%%%%%%%%%%%%%%%%%%%%%%%%
\usepackage{cases}
%%%%%%%%%%%%%%%%%%%%%%%%%%
\usepackage{pdfpages}
%%%%%%%%%%%%%%%%%%%%%%%%%%%
%% 音声リンク表示
\newcommand{\myaudio}[1]{\href{#1}{\faVolumeUp}}
%%%%%%%%%%%%%%%%%%%%%%%%%%
%% \myAnch{<名前>}{<色>}{<テキスト>}
%% 指定のテキストを指定の色の四角枠で囲み, 指定の名前をもつTikZの
%% ノードとして出力する. 図には remember picture 属性を付けている
%% ので外部から参照可能である.
\newcommand*{\myAnch}[3]{%
  \tikz[remember picture,baseline=(#1.base)]
    \node[draw,rectangle,line width=1pt,#2] (#1) {\normalcolor #3};
}
%%%%%%%%%%%%%%%%%%%%%%%%%%
%% \myEmph コマンドの定義
%%%%%%%%%%%%%%%%%%%%%%%%%%
%\newcommand{\myEmph}[3]{%
%    \textbf<#1>{\color<#1>{#2}{#3}}%
%}
\usepackage{xparse} % xparseパッケージの読み込み
\NewDocumentCommand{\myEmph}{O{} m m}{%
    \def\argOne{#1}%
    \ifx\argOne\empty
        \textbf{\color{#2}{#3}}% オプション引数が省略された場合
    \else
        \textbf<#1>{\color<#1>{#2}{#3}}% オプション引数が指定された場合
    \fi
}
%%%%%%%%%%%%%%%%%%%%%%%%%%%
%%%%%%%%%%%%%%%%%%%%%%%%%%%
%% 文末の上昇イントネーション記号 \myRisingPitch
%% 通常のイントネーション \myDownwardPitch
%% https://note.com/dan_oyama/n/n8be58e8797b2
%%%%%%%%%%%%%%%%%%%%%%%%%%%
\newcommand{\myRisingPitch}{
\begin{tikzpicture}[scale=0.3,baseline=0.3]
\draw[->,>=stealth] (0,0) to[bend right=45] (1,1);
\end{tikzpicture}
}
\newcommand{\myDownwardPitch}{
\begin{tikzpicture}[scale=0.3,baseline=0.3]
\draw[->,>=stealth] (0,1) to[bend left=45] (1,0);
\end{tikzpicture}
}
%%%%%%%%%%%%%%%%%%%%%%%%%%%%
%\AtBeginSection[%
%]{%
%  \begin{frame}[plain]\frametitle{授業の流れ}
%     \tableofcontents[currentsection]
%   \end{frame}%
%}

\usepackage{pxrubrica}
%%%%%%%%%%%%%%%%%%%%%%%%%%%
\title{English is fun.}
\subtitle{Who is that man?}
\author{}
\institute[]{}
\date[]

%%%%%%%%%%%%%%%%%%%%%%%%%%%%
%% TEXT
%%%%%%%%%%%%%%%%%%%%%%%%%%%%
\begin{document}
\begin{frame}[plain]
  \titlepage
\end{frame}

\section*{授業の流れ}
\begin{frame}[plain]
  \frametitle{授業の流れ}
  \tableofcontents
\end{frame}
%%%%%%%%%%%%%%%%%%%%%%%%%%%
\section{Who}
%%%%%%%%%%%%%%%%%%%%%%%
\subsection{Who $+$ be動詞 \ldots ?}
\begin{frame}[plain]{Who is at the door?}
\Large

 \mbox{}\hspace{0pt}\alt<2->{\myAnch{FOCUS_2}{orange}{Someone}}{\myAnch{focus_2}{white}{Someone}} is at the door.
%\pause
%\hfill{}cf. \pause%
%Is someone at the door?\pause

%\mbox{}\hfill{\normalsize YesまたはNoで答える疑問文}

\vspace{7pt}

\pause

\visible<3->{\myAnch{wh_2}{orange}{Who} is at the door \myAnch{question_2}{orange}{?}}
\visible<3->{\scalebox{1.4}{\myDownwardPitch}}

\hfill{}\visible<3->{{\scriptsize cf. Is Paul at the door?\scalebox{1.14}{\myRisingPitch}}}




\visible<4->{%
\begin{tikzpicture}[remember picture, overlay]
\draw[->, thick, orange] (focus_2.south) to[out=-90, in=90] (wh_2.north);
\end{tikzpicture}
}

\visible<4->{%
\begin{exampleblock}{Topics for Today}
\begin{itemize}\setbeamertemplate{items}[square]\small
 \item 「だれ」と聞くとき$\longrightarrow$\,\,\,Who%
\pause\hfill{}\kenten{疑問詞}といいます
 \item   文末に`?'をつける(イントネーションは\myDownwardPitch{}\,\,)
\end{itemize}
     \end{exampleblock}
}
%
\mbox{}\hfill\myaudio{./audio/013_who_01.mp3}

\end{frame}
%%%%%%%%%%%%%%%%%%%%%%%%%%%
\begin{frame}[plain]{Exercises}
日本語の意味になるよう語を並べ替えてください。
文の先頭に来る語は大文字で書きはじめましょう\mbox{}\hfill\myaudio{./audio/013_who_02.mp3}

 \begin{enumerate}
  \item  $[$ the / is / who / in / kitchen $]$ ? 誰がキッチンにいるのですか?\\\visible<2->{Who is in the kitchen?}
  \item  $[$ who / hungry / is $]$ ? 空腹なのは誰ですか?\\\visible<3->{Who is hungry?}
  \item  $[$ best / your / who / friend / is $]$ ? あなたの親友は誰ですか?\\
\visible<4->{Who is your best friend?}
  \item  $[$ you / are / who $]$ ? 君は誰だ?\\
\visible<5->{Who are you? $\longleftarrow $失礼な聞き方です}
\end{enumerate}

\begin{block}<5->{Who $+$ be動詞 \ldots ?}
 \begin{itemize}\setbeamertemplate{items}[square]
  \item Who is \ldots{}?\,\, Who are \ldots{}?\hfill{}「~はだれですか」
 \end{itemize}
\end{block}
\end{frame}
%%%%%%%%%%%%%%%%%%%%%%%%%%%%%%%%%%%
\subsection{Who $+$ 一般動詞 \ldots ?}
\begin{frame}[plain]{Who teaches music?}
 \Large

\alt<2->{\myAnch{FOCUS}{orange}{Mr Nash}}{\myAnch{focus}{white}{Mr Nash}} teaches music.
%\pause
%\hfill{}cf. \pause%
%Does Mr Nash teach science?\pause
%
%\mbox{}\hfill{\normalsize YesまたはNoで答える疑問文}

\vspace{20pt}

\pause

\visible<3->{\myAnch{wh}{orange}{Who} teaches music \myAnch{question}{orange}{?}}
\visible<4->{\scalebox{1.4}{\myDownwardPitch}}

%\pause

%\mbox{}\hspace{30pt}\myAnch{txt1}{white}{\small 先頭にWho}

\visible<3->{%
\begin{tikzpicture}[remember picture, overlay]
\draw[->, thick, orange] (focus.south) to[out=-95, in=90] (wh.north);
\end{tikzpicture}
}
\visible<5->{%
\begin{exampleblock}{Topic for Today}
\pause
\begin{itemize}\setbeamertemplate{items}[square]\small
 \item Who $+\,\,\text{一般動詞}$ \ldots\,?%
% \item   文末に`?'をつける(イントネーションは\myDownwardPitch{}\,\,)
\end{itemize}
     \end{exampleblock}
}
\visible<5->{%
\mbox{}\hfill\myaudio{./audio/013_who_03.mp3}
}
\end{frame}
%%%%%%%%%%%%%%%%%%%%%%%%%%
%%%%%%%%%%%%%%%%%%%%%%%%%%%
\begin{frame}[plain]{Exercises}
日本語の意味になるよう語を並べ替えてください。
文の先頭に来る語は大文字で書きはじめましょう

\begin{enumerate}
 \item $[$ the / plays / piano / who  $]$ ?\,\, だれがピアノを弾きますか\\
       \visible<2->{Who plays the piano?}
 \item $[$ dinner / cooks / who  $]$ ?\,\, だれが夕食をつくりますか\\
       \visible<3->{Who cooks dinner?}
 \item $[$ the / who / in / house / lives $]$ ?\,\, だれがその家に住んでいますか\\
       \visible<4->{Who lives in the house?}
\end{enumerate}

\mbox{}\hfill\myaudio{./audio/013_who_04.mp3}
\end{frame}
%%%%%%%%%%%%%%%%%%%%%%%%%%%%%%%%%%%%%%%%%%%%%%
\section{まとめ}
\begin{frame}[plain]{まとめ}
 
\begin{block}{Topics for Today}
\pause
\begin{itemize}\setbeamertemplate{items}[square]\small
 \item 「だれ」と聞くとき$\longrightarrow$\,\,\,疑問詞のwhoをもちいます
        \begin{enumerate}
	 \item $\text{Who}+\text{be動詞\ldots ?}$
	 \item Who $+\,\text{一般動詞}$ \ldots\,?
	\end{enumerate}
 \item 文末には`?'
 \item イントネーションは下降調\,\myDownwardPitch{}\,\,
\end{itemize}
     \end{block}
\end{frame}
%%%%%%%%%%%%%%%%%%%%%%%%%%
\end{document}
