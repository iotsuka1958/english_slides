\documentclass[aspectratio=169,xcolor={dvipsnames,table}]{beamer}
\usepackage[no-math,deluxe,haranoaji]{luatexja-preset}
\renewcommand{\kanjifamilydefault}{\gtdefault}
\renewcommand{\emph}[1]{{\upshape\bfseries #1}}
\usetheme{metropolis}
\metroset{block=fill}
\setbeamertemplate{navigation symbols}{}
\setbeamertemplate{blocks}[rounded][shadow=false]
\usecolortheme[rgb={0.7,0.2,0.2}]{structure}
%%%%%%%%%%%%%%%%%%%%%%%%%%%
\usepackage{media9}
%%%%%%%%%%%%%%%%%%%%%%%%%%%
%% さまざまなアイコン
%%%%%%%%%%%%%%%%%%%%%%%%%%%
\usepackage{fontawesome}
\usepackage{figchild}
\usepackage{twemojis}
\usepackage{utfsym}
\usepackage{bclogo}
\usepackage{marvosym}
\usepackage{fontmfizz}
\usepackage{pifont}
\usepackage{phaistos}
\usepackage{worldflags}
%%%%%%%%%%%%%%%%%%%%%%%%%%%
\usepackage{tikz}
\usetikzlibrary{backgrounds}
\usepackage{tcolorbox}
\usepackage{tikzpeople}
\usepackage{circledsteps}
\usepackage{xcolor}
\usepackage{amsmath}
\usepackage{booktabs}
\usepackage{circledsteps}
\usepackage{tipa}
\usepackage{highlightx}
%%%%%%%%%%%%%%%%%%%%%%%%%%%
%% 場合分け
\usepackage{cases}
%%%%%%%%%%%%%%%%%%%%%%%%%%%
% \myAnch{<名前>}{<色>}{<テキスト>}
% 指定のテキストを指定の色の四角枠で囲み, 指定の名前をもつTikZの
% ノードとして出力する. 図には remeber picture 属性を付けている
% ので外部から参照可能である.
\newcommand*{\myAnch}[3]{%
  \tikz[remember picture,baseline=(#1.base)]
    \node[draw,rectangle,#2] (#1) {\normalcolor #3};
}
%%%%%%%%%%%%%%%%%%%%%%%%%%%%
%% 音声リンク表示
\newcommand{\myaudio}[1]{\href{#1}{\faVolumeUp}}
%%%%%%%%%%%%%%%%%%%%%%%%%%%
% \myEmph コマンドの定義
%\newcommand{\myEmph}[3]{%
%    \textbf<#1>{\color<#1>{#2}{#3}}%
%}
\usepackage{xparse} % xparseパッケージの読み込み
\NewDocumentCommand{\myEmph}{O{} m m}{%
    \def\argOne{#1}%
    \ifx\argOne\empty
        \textbf{\color{#2}{#3}}% オプション引数が省略された場合
    \else
        \textbf<#1>{\color<#1>{#2}{#3}}% オプション引数が指定された場合
    \fi
}
%%%%%%%%%%%%%%%%%%%%%%%%%%%
%% 文末の上昇イントネーション記号 \myRisingPitch
%% 通常のイントネーション \myDownwardPitch
%% https://note.com/dan_oyama/n/n8be58e8797b2
%%%%%%%%%%%%%%%%%%%%%%%%%%%
\newcommand{\myRisingPitch}{
\begin{tikzpicture}[scale=0.3,baseline=0.3]
\draw[->,>=stealth] (0,0) to[bend right=45] (1,1);
\end{tikzpicture}
}
\newcommand{\myDownwardPitch}{
\begin{tikzpicture}[scale=0.3,baseline=0.3]
\draw[->,>=stealth] (0,1) to[bend left=45] (1,0);
\end{tikzpicture}
}
%%%%%%%%%%%%%%%%%%%%%%%%%%%
\title{English is fun.}
\subtitle{Did you eat sushi last night?}
\author{}
\institute[]{}
\date[]

%%%%%%%%%%%%%%%%%%%%%%%%%%%%
%% TEXT
%%%%%%%%%%%%%%%%%%%%%%%%%%%%
\begin{document}
\begin{frame}[plain]
  \titlepage
\end{frame}

\section*{授業の流れ}
\begin{frame}[plain]
  \frametitle{授業の流れ}
  \tableofcontents
\end{frame}

\section{一般動詞の現在形の疑問文(復習)}
%%%%%%%%%%%%%%%%%%%%%%%%%%%%%%%%%%%%%%%%%%%% 
\begin{frame}<1-10>[plain]\frametitle{一般動詞の疑問文(現在形)}

{\small つぎの文を疑問文にしましょう}

 \begin{enumerate}
  \item<1-> You like flowers.\hspace{59.7pt}
        \onslide<6->{$\longrightarrow$\,\,\,\,\, Do you like flowers?\hfill\scalebox{.75}{\bcfleur\bcfleur}}
  \item<1-> They live in Boston.\hspace{47.5pt}%
        \onslide<7->{$\longrightarrow$\,\,\,\,\, Do they live in Boston?\hfill\scalebox{.25}{\worldflag{US}}}
  \item<1-> She teaches science.\hspace{42pt}%
        \onslide<8->{$\longrightarrow$\,\,\,\,\, Does she teach science?\hfill\scalebox{1.75}{\twemoji{woman scientist}}}
  \item<1-> He has  a car.\hspace{80.5pt}%
        \onslide<9->{$\longrightarrow$\,\,\,\,\, Does he have a car?\hfill\faCar}
  \item<1-> Our teacher walks to school.
        \onslide<10->{$\longrightarrow$\,\,\,\,\, Does our teacher walk to school? \hfill\scalebox{.67}{\PHpedestrian}\,\,\scalebox{1.5}{\twemoji{school}}}
 \end{enumerate}

\begin{block}<2->{一般動詞の現在形の疑問文}
\begin{itemize}\setbeamertemplate{items}[square]\small
 \item<3->   先頭に \Circled[fill color = white]{\,Do\,} を置く\pause
 \item<4->   主語が三人称単数のときは \Circled[fill color = white]{\,Does\,}
\end{itemize}

\hfill%
\visible<5->{$\HighlightFormula{\left\{
\begin{tabular}{l}
 Do\\
 Does
\end{tabular}
\right\} + \text{S} + \text{原形 \ldots ?}}$}%
\hfill\mbox{}

\end{block}
\vspace{-10pt}
% Embed the sound file
%
 \mbox{}\hfill{\tiny 0331}\,{\scriptsize \myaudio{./audio/027_past_did_you_01.mp3}}

\end{frame}


\section{一般動詞の過去形の疑問文}

\subsection{一般動詞の過去形の疑問文のつくり方}
\begin{frame}[plain]{一般動詞の過去形の疑問文}
 \Large

\begin{enumerate}
 \item \begin{enumerate}
	\item  \visible<1->{They play tennis after school. $\rightarrow$} \visible<2->{\myEmph[3-]{Maroon}{Do} they play tennis after school?}
	\item  \visible<1->{She plays tennis after school. $\rightarrow$} \visible<4->{\myEmph[5-]{ForestGreen}{Does} she play tennis after school?}\\
\mbox{}\hfill\visible<6->{\footnotesize 現在形の疑問文は、主語によって\textcolor{Maroon}{\bfseries Do}と\textcolor{ForestGreen}{\bfseries Does}を使い分け}
       \end{enumerate}
 \item \begin{enumerate}
	\item  \visible<7->{They played tennis last week. $\rightarrow$} \visible<8->{\myEmph[9-]{NavyBlue}{Did} they play tennis last week?}
	\item  \visible<7->{She played tennis last week. $\rightarrow$} \visible<10->{\myEmph[11-]{NavyBlue}{Did} she play tennis last week?}\\
\mbox{}\hfill\visible<12->{\footnotesize 過去形の疑問文は主語がなんであっても\textcolor{NavyBlue}{\bfseries $\text{Did} + \text{S} + \text{原形} \ldots ?$}}
       \end{enumerate}
\end{enumerate}

\visible<13->{%
\begin{block}{Topics for Today}
\small
一般動詞の過去形の疑問文
\begin{itemize}\setbeamertemplate{items}[square]
 \item  \textcolor{NavyBlue}{\bfseries $\text{Did} + \text{S} + \text{原形} \ldots ?$}\hfill{}Did she eat sushi in Japan?\\
\hfill{}*Did she \Circled[fill color=white]{\,ate\,} sushi in Japan?
 \item  主語がなんであっても同じです
\end{itemize}
      \end{block}
}
\mbox{}\hfill{\tiny 0250}\,{\scriptsize \myaudio{./audio/027_past_did_you_02.mp3}}
\end{frame}
%%%%%%%%%%%%%%%%%%%%%%%%%%%%%%%%%%%%%%%%%%%
\subsection{疑問文への答え方}
%%%%%%%%%%%%%%%%%%%%%%%%%%%%%%%%%%%%%%%%%%%
 \begin{frame}[plain]{疑問文への答え方}
 \Large
\pause

Did  you study science yesterday?

\vspace{20pt}
\pause

\mbox{}\hspace{100pt}$\left\{\begin{tabular}{l}
         \text{Yes, I did.}\\\pause
         \text{No, I did not.}\\\pause
         \text{(}= \text{No, I didn't.)}
        \end{tabular}\right.$

\pause

\mbox{}\hfill{}{\small Noのときdid notを縮めてNo, I \textcolor{NavyBlue}{didn't}.ともいいます}\,\,{\tiny 0202}\,{\scriptsize \myaudio{./audio/027_past_did_you_03.mp3}}

\pause
\begin{block}{Topics for Today}
\small
疑問文への答え方
\begin{itemize}\setbeamertemplate{items}[square]
 \item  \textcolor{NavyBlue}{\bfseries Yes, S did.} または\textcolor{NavyBlue}{\bfseries No, S did not($=\text{didn't}$).}
 \item  主語がなんであっても同じです
\end{itemize}
      \end{block}

\end{frame}
%%%%%%%%%%%%%%%%%%%%%%%%%
\begin{frame}[plain]\frametitle{Exercises}

\small {つぎの文を疑問文にしましょう}

 \begin{enumerate}
  \item<1-> You liked the movie.\hspace{59.7pt}
        \onslide<2->{$\longrightarrow$\,\,\,\,\, Did you like the movie?\hfill\scalebox{.75}{\bcfleur\bcfleur}}
  \item<1-> They lived in Boston.\hspace{62pt}%
        \onslide<3->{$\longrightarrow$\,\,\,\,\, Did they live in Boston?\hfill\scalebox{.25}{\worldflag{US}}}
  \item<1-> She wrote a letter to him.\hspace{42pt}%
        \onslide<4->{$\longrightarrow$\,\,\,\,\, Did she write a letter to him?}
  \item<1-> He had  a car two years ago.\hspace{30.5pt}%
        \onslide<5->{$\longrightarrow$\,\,\,\,\, Did he have a car two years ago?\hfill\faCar}
  \item<1-> Your father went to  Australia last year.\\
 \mbox{}\hfill\onslide<6->{$\longrightarrow$\,\,\,\,\, Did your father go to Australia last year? }
 \end{enumerate}

\begin{block}<7->{Topics for Today}
\small
一般動詞の過去形の疑問文
\begin{itemize}\setbeamertemplate{items}[square]
 \item  \textcolor{NavyBlue}{\bfseries $\text{Did} + \text{S} + \text{原形} \ldots ?$}
 \item  主語がなんであっても同じです
\end{itemize}
      \end{block}

\vspace{-10pt}
% Embed the sound file
\mbox{}\hfill{\tiny 0341}\,{\scriptsize \myaudio{./audio/027_past_did_you_04.mp3}}


\end{frame}
%%%%%%%%%%%%%%%%%%%%%%%%%%%%%%%
\section{まとめ---一般動詞の過去形の疑問文---}
\begin{frame}[plain]{まとめ}

\begin{block}{一般動詞の過去形の疑問文のつくり方}\small
\begin{itemize}\setbeamertemplate{items}[square]
 \item   先頭に\,\,\Circled[fill color = white]{\,\,Did\,\,}\,\,を置きます%
\hfill{\scriptsize 主語が何であっても同じです}\\[10pt]
	 \Circled[fill color = white]{\,\,Did\,\,}\,\,$+$\,S\,$+$\,V\,{\scriptsize ($=$\,原形)} \ldots\,\,\,?\hfill{}Did you eat sushi?\\
\hfill{}*Did you \Circled{\textbf{ate}} sushi?
% \item   文末に`?'をつける
\end{itemize}
     \end{block}

\pause

\begin{block}{一般動詞の過去形の疑問文への答え方}
\small
\begin{itemize}\setbeamertemplate{items}[square]
 \item 「はい」のとき\hfill{}\makebox[10em][l]{Yes, S $ + \text{\,did.}$}\hspace{120pt}\mbox{}
 \item 「いいえ」のとき\hfill{}\makebox[100pt][l]{No, S $ + \text{\,did not\,($=$ didn't).}$}\hspace{120pt}\mbox{}
\end{itemize}
      \end{block}

\pause

\begin{block}{Pronunciation}
 \mbox{}\hfill{}did \textipa{/d\'Id/}
\hspace{25pt}not \textipa{/n\'At/}
\hspace{25pt}didn't \textipa{/d\'Idnt/}\hfill{}\mbox{}
\end{block}
\end{frame}
%%%%%%%%%%%%%%%%%%%%%%%%%%%%%%%%%


%%%%%%%%%%%%%%%%%%%%%%%%%%%%%%%%
\end{document}
