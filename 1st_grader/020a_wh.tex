\documentclass[aspectratio=169,xcolor={dvipsnames,table}]{beamer}
\usepackage[no-math,deluxe,haranoaji]{luatexja-preset}
\renewcommand{\kanjifamilydefault}{\gtdefault}
\renewcommand{\emph}[1]{{\upshape\bfseries #1}}
\usetheme{metropolis}
\metroset{block=fill}
\setbeamertemplate{navigation symbols}{}
\setbeamertemplate{blocks}[rounded][shadow=false]
\usecolortheme[rgb={0.7,0.2,0.2}]{structure}
%%%%%%%%%%%%%%%%%%%%%%%%%%
%% Change alert block colors
%%% 1- Block title (background and text)
\setbeamercolor{block title alerted}{fg=mDarkTeal, bg=mLightBrown!45!yellow!45}
\setbeamercolor{block title example}{fg=magenta!10!black, bg=mLightGreen!60}
%%% 2- Block body (background)
\setbeamercolor{block body alerted}{bg=mLightBrown!25}
\setbeamercolor{block body example}{bg=mLightGreen!15}
%%%%%%%%%%%%%%%%%%%%%%%%%%%
%%%%%%%%%%%%%%%%%%%%%%%%%%%
%% さまざまなアイコン
%%%%%%%%%%%%%%%%%%%%%%%%%%%
%\usepackage{fontawesome}
\usepackage{fontawesome5}
\usepackage{figchild}
\usepackage{twemojis}
\usepackage{utfsym}
\usepackage{bclogo}
\usepackage{marvosym}
\usepackage{fontmfizz}
\usepackage{pifont}
\usepackage{phaistos}
\usepackage{worldflags}
\usepackage{jigsaw}
\usepackage{tikzlings}
\usepackage{tikzducks}
\usepackage{scsnowman}
\usepackage{epsdice}
\usepackage{halloweenmath}
\usepackage{svrsymbols}
\usepackage{countriesofeurope}
\usepackage{tipa}
\usepackage{manfnt}
%%%%%%%%%%%%%%%%%%%%%%%%%%%
\usepackage{tikz}
\usetikzlibrary{calc,patterns,decorations.pathmorphing,backgrounds}
\usepackage{tcolorbox}
\usepackage{tikzpeople}
\usepackage{circledsteps}
\usepackage{xcolor}
\usepackage{amsmath}
\usepackage{booktabs}
\usepackage{chronology}
\usepackage{signchart}
%%%%%%%%%%%%%%%%%%%%%%%%%%%
%% 場合分け
%%%%%%%%%%%%%%%%%%%%%%%%%%%
\usepackage{cases}
%%%%%%%%%%%%%%%%%%%%%%%%%%
\usepackage{pdfpages}
%%%%%%%%%%%%%%%%%%%%%%%%%%%
%% 音声リンク表示
\newcommand{\myaudio}[1]{\href{#1}{\faVolumeUp}}
%%%%%%%%%%%%%%%%%%%%%%%%%%
%% \myAnch{<名前>}{<色>}{<テキスト>}
%% 指定のテキストを指定の色の四角枠で囲み, 指定の名前をもつTikZの
%% ノードとして出力する. 図には remember picture 属性を付けている
%% ので外部から参照可能である.
\newcommand*{\myAnch}[3]{%
  \tikz[remember picture,baseline=(#1.base)]
    \node[draw,rectangle,line width=1pt,#2] (#1) {\normalcolor #3};
}
%%%%%%%%%%%%%%%%%%%%%%%%%%
%% \myEmph コマンドの定義
%%%%%%%%%%%%%%%%%%%%%%%%%%
%\newcommand{\myEmph}[3]{%
%    \textbf<#1>{\color<#1>{#2}{#3}}%
%}
\usepackage{xparse} % xparseパッケージの読み込み
\NewDocumentCommand{\myEmph}{O{} m m}{%
    \def\argOne{#1}%
    \ifx\argOne\empty
        \textbf{\color{#2}{#3}}% オプション引数が省略された場合
    \else
        \textbf<#1>{\color<#1>{#2}{#3}}% オプション引数が指定された場合
    \fi
}
%%%%%%%%%%%%%%%%%%%%%%%%%%%
%%%%%%%%%%%%%%%%%%%%%%%%%%%
%% 文末の上昇イントネーション記号 \myRisingPitch
%% 通常のイントネーション \myDownwardPitch
%% https://note.com/dan_oyama/n/n8be58e8797b2
%%%%%%%%%%%%%%%%%%%%%%%%%%%
\newcommand{\myRisingPitch}{
\begin{tikzpicture}[scale=0.3,baseline=0.3]
\draw[->,>=stealth] (0,0) to[bend right=45] (1,1);
\end{tikzpicture}
}
\newcommand{\myDownwardPitch}{
\begin{tikzpicture}[scale=0.3,baseline=0.3]
\draw[->,>=stealth] (0,1) to[bend left=45] (1,0);
\end{tikzpicture}
}
%%%%%%%%%%%%%%%%%%%%%%%%%%%%
%\AtBeginSection[%
%]{%
%  \begin{frame}[plain]\frametitle{授業の流れ}
%     \tableofcontents[currentsection]
%   \end{frame}%
%}

\usepackage{pxrubrica}
%%%%%%%%%%%%%%%%%%%%%%%%%%%
\title{English is fun.}
\subtitle{いつだれがどこでなにをどんなふうにどうして?}
\author{}
\institute[]{}
\date[]

%%%%%%%%%%%%%%%%%%%%%%%%%%%%
%% TEXT
%%%%%%%%%%%%%%%%%%%%%%%%%%%%
\begin{document}

\begin{frame}[plain]
  \titlepage
\end{frame}

%\section*{授業の流れ}
%\begin{frame}[plain]
%  \frametitle{授業の流れ}
%  \tableofcontents
%\end{frame}

%%%%%%%%%%%%%%%%%%%%%%%%%%%%%%%%%%%%%%
\begin{frame}[plain]{疑問詞}
 

\begin{tblr}{
  width = { 1\linewidth },
  hline{1,Z} = { 0.08em },
  hline{2} = { 0.05em },
  colspec = { X[1,l]X[2,l]X[1,l]X[4,l] },
  row{odd} = { gray9 },
  row{1} = { halign = c, bg = gray6, fg = white},
}
 疑問詞&意味&発音&例文\\
who&だれ&\textipa{/h\'u:/}&Who teaches music?\\
whose&だれの&\textipa{/h\'u:z/}&Whose songs do you like?\\
how&どのように\par{}どれくらい&\textipa{/h\'aU/}&How does she go to work?\par{}How long is the river?\\
when&いつ&\textipa{/w\'en/}&When does the meeting begin?\\
where&どこで&\textipa{/w\'e\textrhookschwa /}&Where do they live?\\
why&なぜ&\textipa{/w\'aI/}&Why do you like jazz?\\
what&なに&\textipa{/w\'At/}&What do you drink every morning?\\
which&どちら、どれ&\textipa{/w\'ItS/}&Which do you like, pizza or pasta?\\
\end{tblr}

\hfill{\scriptsize \myaudio{./audio/020a_wh_01.mp3}}
\end{frame}
%%%%%%%%%%%%%%%%%%%%%%%%%%%%%%%%%%%
\begin{frame}[plain]{\textdbend doがあったりなかったり}
 \begin{enumerate}
  \item<1-> \begin{enumerate}
	 \item<1-> \myAnch{focus_1}{Maroon}{She} teaches music.\\
\mbox{}\\
\mbox{}
	 \item<2-> \myAnch{wh_1}{Maroon}{Who} teaches music?
	\end{enumerate}
  \item<3-> \begin{enumerate}
	 \item<3-> She teaches \myAnch{focus_2}{NavyBlue}{music}.\\
\mbox{}\\
\mbox{}
	 \item<4-> \myAnch{wh_2}{NavyBlue}{What} does she teach?
	\end{enumerate}
 \end{enumerate}

\begin{tikzpicture}[remember picture, overlay]
\visible<2->{\draw[->, line width=3pt,opacity=.5, Maroon] (focus_1.south) to[out=-90, in=90]  node[sloped,above,text=black,font=\tiny,pos=.667]{} (wh_1.north);}
\visible<4->{\draw[->, line width=3pt,opacity=.5, NavyBlue] (focus_2.south) to[out=-90, in=60]  node[sloped,above,text=black,font=\tiny,pos=.667]{先頭へ} (wh_2.north);}
\end{tikzpicture}


\begin{block}<5->{Topics for Today}
\begin{itemize}\setbeamertemplate{items}[square]\small
 \item もともと先頭にあったとき$\rightarrow$疑問詞にするだけ
 \item もともと先頭になかったとき$\rightarrow$疑問詞にしてから
       \begin{itemize}
	\item その疑問詞を先頭に置く
	\item さらに、そのあとを\,\,\Circled[fill color = white]{\,\,do(does) $+$ S $+$ 原形\,\,}\,\,にします
       \end{itemize}
\end{itemize}
 
\end{block}
\end{frame}
%%%%%%%%%%%%%%%%%%%%%
\begin{frame}[plain]{Exercises}
\textdbend\textdbend つぎの各文の意味をかんがえましょう

\begin{enumerate}
 \item \begin{enumerate}
		  \item Who loves you?\hfill{}cf. She loves you.
		  \item Who do you love?\hfill{}cf. You love Pat. 
		 \end{enumerate}
 \item \begin{enumerate}
	\item Carelessness causes accidents.\hfill{\scriptsize carelessness \textipa{/k\'e\textrhookschwa l@sn@s/} 不注意}
	\item What causes accidents?\hfill{\scriptsize cause \textipa{/k\'O:z/} 引き起こす}
       \end{enumerate}
 \item \begin{enumerate}
	\item You eat bread for breakfast.
	\item What do you eat for breakfast?
       \end{enumerate}
\end{enumerate}
\hfill{\scriptsize \myaudio{./audio/020a_wh_02.mp3}}

\end{frame}
%%%%%%%%%%%%%%%%%%%%%%%%%%
\end{document}
