\documentclass[aspectratio=169,xcolor={dvipsnames,table}]{beamer}
\usepackage[no-math,deluxe,haranoaji]{luatexja-preset}
\renewcommand{\kanjifamilydefault}{\gtdefault}
\renewcommand{\emph}[1]{{\upshape\bfseries #1}}
\usetheme{metropolis}
\metroset{block=fill}
\setbeamertemplate{navigation symbols}{}
\setbeamertemplate{blocks}[rounded][shadow=false]
\usecolortheme[rgb={0.7,0.2,0.2}]{structure}
%%%%%%%%%%%%%%%%%%%%%%%%%%%
%%%%%%%%%%%%%%%%%%%%%%%%%%%
%% さまざまなアイコン
%%%%%%%%%%%%%%%%%%%%%%%%%%%
%\usepackage{fontawesome}
\usepackage{fontawesome5}
\usepackage{figchild}
\usepackage{twemojis}
\usepackage{utfsym}
\usepackage{bclogo}
\usepackage{marvosym}
\usepackage{fontmfizz}
\usepackage{pifont}
\usepackage{phaistos}
\usepackage{worldflags}
\usepackage{jigsaw}
\usepackage{tikzlings}
\usepackage{tikzducks}
\usepackage{scsnowman}
\usepackage{epsdice}
\usepackage{halloweenmath}
\usepackage{svrsymbols}
\usepackage{countriesofeurope}
\usepackage{tipa}
\usepackage{manfnt}
%%%%%%%%%%%%%%%%%%%%%%%%%%%
\usepackage{tikz}
\usetikzlibrary{calc,patterns,decorations.pathmorphing,backgrounds}
\usepackage{tcolorbox}
\usepackage{tikzpeople}
\usepackage{circledsteps}
\usepackage{xcolor}
\usepackage{amsmath}
\usepackage{booktabs}
\usepackage{chronology}
\usepackage{signchart}
%%%%%%%%%%%%%%%%%%%%%%%%%%%
%% 場合分け
%%%%%%%%%%%%%%%%%%%%%%%%%%%
\usepackage{cases}
%%%%%%%%%%%%%%%%%%%%%%%%%%
\usepackage{pdfpages}
%%%%%%%%%%%%%%%%%%%%%%%%%%%
%% 音声リンク表示
\newcommand{\myaudio}[1]{\href{#1}{\faVolumeUp}}
%%%%%%%%%%%%%%%%%%%%%%%%%%
%% \myAnch{<名前>}{<色>}{<テキスト>}
%% 指定のテキストを指定の色の四角枠で囲み, 指定の名前をもつTikZの
%% ノードとして出力する. 図には remember picture 属性を付けている
%% ので外部から参照可能である.
\newcommand*{\myAnch}[3]{%
  \tikz[remember picture,baseline=(#1.base)]
    \node[draw,rectangle,line width=1pt,#2] (#1) {\normalcolor #3};
}
%%%%%%%%%%%%%%%%%%%%%%%%%%
%% \myEmph コマンドの定義
%%%%%%%%%%%%%%%%%%%%%%%%%%
%\newcommand{\myEmph}[3]{%
%    \textbf<#1>{\color<#1>{#2}{#3}}%
%}
\usepackage{xparse} % xparseパッケージの読み込み
\NewDocumentCommand{\myEmph}{O{} m m}{%
    \def\argOne{#1}%
    \ifx\argOne\empty
        \textbf{\color{#2}{#3}}% オプション引数が省略された場合
    \else
        \textbf<#1>{\color<#1>{#2}{#3}}% オプション引数が指定された場合
    \fi
}
%%%%%%%%%%%%%%%%%%%%%%%%%%%
%%%%%%%%%%%%%%%%%%%%%%%%%%%
%% 文末の上昇イントネーション記号 \myRisingPitch
%% 通常のイントネーション \myDownwardPitch
%% https://note.com/dan_oyama/n/n8be58e8797b2
%%%%%%%%%%%%%%%%%%%%%%%%%%%
\newcommand{\myRisingPitch}{
\begin{tikzpicture}[scale=0.3,baseline=0.3]
\draw[->,>=stealth] (0,0) to[bend right=45] (1,1);
\end{tikzpicture}
}
\newcommand{\myDownwardPitch}{
\begin{tikzpicture}[scale=0.3,baseline=0.3]
\draw[->,>=stealth] (0,1) to[bend left=45] (1,0);
\end{tikzpicture}
}
%%%%%%%%%%%%%%%%%%%%%%%%%%%%
%\AtBeginSection[%
%]{%
%  \begin{frame}[plain]\frametitle{授業の流れ}
%     \tableofcontents[currentsection]
%   \end{frame}%
%}

%%%%%%%%%%%%%%%%%%%%%%%%%%%
\title{English is fun.}
\subtitle{When do you play baseball?}
\author{}
\institute[]{}
\date[]

%%%%%%%%%%%%%%%%%%%%%%%%%%%%
%% TEXT
%%%%%%%%%%%%%%%%%%%%%%%%%%%%
\begin{document}
\begin{frame}[plain]
  \titlepage
\end{frame}

\section*{授業の流れ}
\begin{frame}[plain]
  \frametitle{授業の流れ}
  \tableofcontents
\end{frame}

%%%%%%%%%%%%%%%%%%%%%%%%%%%%%%%
\section{when \textipa{/w\'en/}}
\subsection{When is your birthday?: be動詞のとき}
\begin{frame}[plain]{When is your birthday?} \Large

be動詞のとき

\mbox{}\hspace{55pt}Your birthday is  \alt<3->{\myAnch{FOCUS}{orange}{August 7th}}{\myAnch{focus}{white}{August 7th}}.%
\hfill{\scriptsize August \textipa{/\'\textopeno :g@st/} 8月}

\pause


\vspace{7pt}

\mbox{}\hfill{}{\small cf. \myEmph[6-]{Maroon}{Is your birthday} August 7th? \scalebox{1.1}{\myRisingPitch}}%

\vspace{-5pt}

\hfill{\scriptsize YesまたはNoで答える疑問文}

\pause

\visible<4->{\myAnch{wh}{orange}{When} \myEmph[6-]{Maroon}{is your birthday} \myAnch{question}{orange}{?}}
\visible<6->{\scalebox{1.4}{\myDownwardPitch}}

\pause

%\mbox{}\hspace{30pt}\myAnch{txt1}{white}{\small 先頭にWho}

\visible<5->{%
\begin{tikzpicture}[remember picture, overlay]
\draw[->, line width = 3pt, opacity = .5, orange] (focus.south) to[out=-90, in=90] (wh.north);
\end{tikzpicture}
}
\vspace{-10pt}

\visible<7->{%
\begin{block}{Topics for Today}
\pause
\begin{itemize}\setbeamertemplate{items}[square]\small
 \item 疑問詞{\bfseries when}(いつ?)を先頭に置く\hfill\textipa{/w\'en/}
 \item When is 〜?「〜はいつ」
 \item {\bfseries When}の後ろは、YesまたはNoで答える疑問文の語順と同じ
 \item   文末に`?'をつける(イントネーションは\myDownwardPitch{}\,\,)%
\hfill{\tiny 0138}\,{\scriptsize \myaudio{./audio/014_when_01.mp3}}
\end{itemize}
     \end{block}
}

\end{frame}
%%%%%%%%%%%%%%%%%%%%%%%%%%%%%%%%%%%%%%%%%%%%%%
\subsection{When do you play tennis?: 一般動詞のとき}
\begin{frame}[plain]\frametitle{When do you play tennis?}
\Large
一般動詞のとき

\pause

\mbox{}\hspace{55pt}%
You play tennis \alt<4->{\myAnch{FOCUS2}{orange}{on Sundays}}{\myAnch{focus2}{white}{on Sundays}}.

\pause

\mbox{}\hfill%
{\small cf. \myEmph[6-]{Maroon}{Do you play} tennis on Sundays? \scalebox{1.4}{\myRisingPitch}}

\vspace{-5pt}

\hfill\visible<3->{{\small Yes/Noで答える疑問文}}

\visible<5->{\myAnch{WH2}{orange}{When} \myEmph[6-]{Maroon}{do you play} tennis\myAnch{question2}{orange}{?}}

\visible<5->{%
\begin{tikzpicture}[remember picture, overlay]
 \draw[line width = 3pt, opacity = .5, orange, ->] (focus2.south) to[out=-90, in=90] (WH2.north);
\end{tikzpicture}
}

\vspace{-20pt}

\visible<7->{%
\begin{block}{Topics for Today}
\pause
\begin{itemize}\setbeamertemplate{items}[square]\small
 \item 疑問詞{\bfseries When}を先頭に置く
 \item {\bfseries When}の後ろはYes/Noで答える疑問文の語順と同じ\hfill{}When do you 原形 \ldots ?\\
\hfill{}When does she 原形 \ldots ?
 \item   文末に`?'をつける(イントネーションは\myDownwardPitch{}\,\,)\\
\hfill{\tiny 0137}\,{\scriptsize \myaudio{./audio/014_when_02.mp3}}
\end{itemize}
     \end{block}
}
\end{frame}
%%%%%%%%%%%%%%%%%%
\subsection{Exercises}
\begin{frame}[plain]{Exercises 1}
つぎの文の意味を考えましょう

\begin{tabular}{rll}
1& When is the concert? &{\scriptsize concert: コンサート} \\
2& When is the party?& \\
3& When do you get up? & {\scriptsize get up: 起きる}\\
4& When does the meeting begin?&{\scriptsize meeting: 会議 begin: 始まる} \\
\end{tabular}


\mbox{}\hfill{\tiny 0155}\,{\scriptsize \myaudio{./audio/014_when_03.mp3}}

\end{frame}

\begin{frame}[plain]{Exercises 2}
 次の質問に対する答えとしてもっとも適切なものを,下のア~ウの中から選びましょう

\begin{enumerate}
 \item When do you usually\footnote{usually \textipa{/j\'u:Zu@li/} いつも} practice\footnote{practice \textipa{/pr\'\ae ktIs/} 練習する} basketball?\hspace{10pt}\visible<2->{--- On Saturdays.}\hspace{21pt}\visible<3->{イ}
 \item When is St. Valentine's Day\footnote{バレンタインデー}?\hspace{80pt}\visible<4->{--- It's February\footnote{February \textipa{/f\'ebju\`eri/} 2月} 14th.}\hspace{12pt}\visible<5->{ウ}
 \item When does the school year start in Japan?\hspace{15pt}\visible<6->{--- In April.}\hspace{50pt}\visible<7->{ア}
\end{enumerate}

\begin{tcolorbox}
\centering
ア In April.~~~~~~~~%
イ On Saturdays.~~~~~~~~%
ウ It's February 14th. 
\end{tcolorbox}

%
\mbox{}\hfill{\tiny 0235}\,{\scriptsize \myaudio{./audio/014_when_04.mp3}}


\end{frame}


\begin{frame}[plain]{Exercises 3}
 (~~~~~~~~) 内の語句を並べかえ、AとBの対話を完成させましょう。なお、先頭の語は大文字で始めてください

\begin{enumerate}
 \item A: ( Children's Day / when / is ) ? 
\hspace{137.4pt}B: It's May 5th.\\
\phantom{A: }\visible<2->{When is Children's Day?}
 \item A: ( birthday / your / father's / when / is ) ?
\hspace{83.5pt}B: It's July 2nd.\\
\phantom{A: }\visible<3->{When is your father's birthday?}
 \item A: ( to school / does / Mr. Brown / come / when ) ?
\hspace{47pt}B: Around 7:00.\\
\phantom{A: }\visible<4->{When does Mr. Brown come to school?}
 \item A: ( does / the violin / when / play / your brother ) ?
\hspace{39.7pt}B: After dinner.\\
\phantom{A: }\visible<5->{When does your brother play the violin?}
\end{enumerate}

\mbox{}\hfill{\tiny 0256}\,{\scriptsize \myaudio{./audio/014_when_05.mp3}}

\end{frame}
%%%%%%%%%%%%%%%%%%%%%%%%%%%%%%%%
\section{疑問詞when \textipa{/w\'en/} のまとめ}
\begin{frame}[plain]{まとめ}
 \begin{block}{When ~? \textipa{/w\'en/}}
\begin{description}[    ]
 \item[be動詞] When is ~?
 \item[一般動詞] When do you ~? / When does he ~?
\end{description}
  
 \end{block}
\end{frame}
\end{document}
