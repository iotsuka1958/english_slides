\documentclass[aspectratio=169,xcolor={dvipsnames,table}]{beamer}
\usepackage[no-math,deluxe,haranoaji]{luatexja-preset}
\renewcommand{\kanjifamilydefault}{\gtdefault}
\renewcommand{\emph}[1]{{\upshape\bfseries #1}}
\usetheme{metropolis}
\metroset{block=fill}
\setbeamertemplate{navigation symbols}{}
\setbeamertemplate{blocks}[rounded][shadow=false]
\usecolortheme[rgb={0.7,0.2,0.2}]{structure}
%%%%%%%%%%%%%%%%%%%%%%%%%%%
%%%%%%%%%%%%%%%%%%%%%%%%%%%
%% さまざまなアイコン
%%%%%%%%%%%%%%%%%%%%%%%%%%%
%\usepackage{fontawesome}
\usepackage{fontawesome5}
\usepackage{figchild}
\usepackage{twemojis}
\usepackage{utfsym}
\usepackage{bclogo}
\usepackage{marvosym}
\usepackage{fontmfizz}
\usepackage{pifont}
\usepackage{phaistos}
\usepackage{worldflags}
\usepackage{jigsaw}
\usepackage{tikzlings}
\usepackage{tikzducks}
\usepackage{scsnowman}
\usepackage{epsdice}
\usepackage{halloweenmath}
\usepackage{svrsymbols}
\usepackage{countriesofeurope}
\usepackage{tipa}
%%%%%%%%%%%%%%%%%%%%%%%%%%%
\usepackage{tikz}
\usetikzlibrary{calc,patterns,decorations.pathmorphing,backgrounds}
\usepackage{tcolorbox}
\usepackage{tikzpeople}
\usepackage{circledsteps}
\usepackage{xcolor}
\usepackage{amsmath}
\usepackage{booktabs}
\usepackage{chronology}
\usepackage{signchart}
%%%%%%%%%%%%%%%%%%%%%%%%%%%
%% 場合分け
%%%%%%%%%%%%%%%%%%%%%%%%%%%
\usepackage{cases}
%%%%%%%%%%%%%%%%%%%%%%%%%%
\usepackage{pdfpages}
%%%%%%%%%%%%%%%%%%%%%%%%%%%
%% 音声リンク表示
\newcommand{\myaudio}[1]{\href{#1}{\faVolumeUp}}
%%%%%%%%%%%%%%%%%%%%%%%%%%
%% \myAnch{<名前>}{<色>}{<テキスト>}
%% 指定のテキストを指定の色の四角枠で囲み, 指定の名前をもつTikZの
%% ノードとして出力する. 図には remember picture 属性を付けている
%% ので外部から参照可能である.
\newcommand*{\myAnch}[3]{%
  \tikz[remember picture,baseline=(#1.base)]
    \node[draw,rectangle,line width=1pt,#2] (#1) {\normalcolor #3};
}
%%%%%%%%%%%%%%%%%%%%%%%%%%
%% \myEmph コマンドの定義
%%%%%%%%%%%%%%%%%%%%%%%%%%
%\newcommand{\myEmph}[3]{%
%    \textbf<#1>{\color<#1>{#2}{#3}}%
%}
\usepackage{xparse} % xparseパッケージの読み込み
\NewDocumentCommand{\myEmph}{O{} m m}{%
    \def\argOne{#1}%
    \ifx\argOne\empty
        \textbf{\color{#2}{#3}}% オプション引数が省略された場合
    \else
        \textbf<#1>{\color<#1>{#2}{#3}}% オプション引数が指定された場合
    \fi
}
%%%%%%%%%%%%%%%%%%%%%%%%%%%
%%%%%%%%%%%%%%%%%%%%%%%%%%%
%% 文末の上昇イントネーション記号 \myRisingPitch
%% 通常のイントネーション \myDownwardPitch
%% https://note.com/dan_oyama/n/n8be58e8797b2
%%%%%%%%%%%%%%%%%%%%%%%%%%%
\newcommand{\myRisingPitch}{
\begin{tikzpicture}[scale=0.3,baseline=0.3]
\draw[->,>=stealth] (0,0) to[bend right=45] (1,1);
\end{tikzpicture}
}
\newcommand{\myDownwardPitch}{
\begin{tikzpicture}[scale=0.3,baseline=0.3]
\draw[->,>=stealth] (0,1) to[bend left=45] (1,0);
\end{tikzpicture}
}
%%%%%%%%%%%%%%%%%%%%%%%%%%%%
%\AtBeginSection[%
%]{%
%  \begin{frame}[plain]\frametitle{授業の流れ}
%     \tableofcontents[currentsection]
%   \end{frame}%
%}

\usepackage{highlightx}
%%%%%%%%%%%%%%%%%%%%%%%%%%%
\title{English is fun.}
\subtitle{When do you play baseball?}
\author{}
\institute[]{}
\date[]

%%%%%%%%%%%%%%%%%%%%%%%%%%%%
%% TEXT
%%%%%%%%%%%%%%%%%%%%%%%%%%%%
\begin{document}
\begin{frame}[plain]
  \titlepage
\end{frame}

\section*{授業の流れ}
\begin{frame}[plain]
  \frametitle{授業の流れ}
  \tableofcontents
\end{frame}

%%%%%%%%%%%%%%%%%%%%%%%%%%%%%%%
\section{when \textipa{/w\'en/}}
\subsection{When is your birthday?: be動詞のとき}
\begin{frame}[plain]{~はいつ?}
be動詞のとき

\mbox{}\hspace{55pt}Your birthday is  \alt<3->{\myAnch{FOCUS}{NavyBlue}{August 7th}}{\myAnch{focus}{white}{August 7th}}.%
\hfill{\scriptsize August \textipa{/\'\textopeno :g@st/} 8月}

\pause


\vspace{7pt}

\mbox{}\hfill{}{\small cf. \myEmph[5-]{Maroon}{Is your birthday} August 7th? \scalebox{1.1}{\myRisingPitch}}%

\vspace{-5pt}

\hfill{\scriptsize Yes/Noで答える疑問文}

\visible<4->{\myAnch{wh}{NavyBlue}{When} \myEmph[5-]{Maroon}{is your birthday} ?}%\myAnch{question}{orange}{?}}
\visible<6->{\scalebox{1.4}{\myDownwardPitch}}

\pause

%\mbox{}\hspace{30pt}\myAnch{txt1}{white}{\small 先頭にWho}

\visible<4->{%
\begin{tikzpicture}[remember picture, overlay]
\draw[->, line width = 3pt, opacity = .5, NavyBlue] (focus.south) to[out=-90, in=90] node[sloped,above,text=black,font=\tiny,pos=.6]{Whenに置き換えて先頭へ} node[sloped,below,text=black,font=\tiny,pos=.4]{その後は疑問文の語順} (wh.north);
\end{tikzpicture}
}
\vspace{-10pt}

\begin{block}<7->{Topics for Today}\small
\begin{itemize}\setbeamertemplate{items}[square]\small
 \item<8-> 疑問詞{\bfseries when}(いつ)を先頭に置く\hfill\textipa{/w\'en/}
% \item When is 〜?「〜はいつ」
 \item<9-> {\bfseries When}の後ろは、Yes/Noで答える疑問文の語順と同じ
 \item<10->   文末に`?'をつける / イントネーションは\myDownwardPitch{}(下降調)%
\end{itemize}
     \end{block}
\hfill{\tiny 0138}\,{\scriptsize \myaudio{./audio/014_when_01.mp3}}

\end{frame}
%%%%%%%%%%%%%%%%%%%%%%%%%%%%%%%%%%%%%%%%%%%%%%
\subsection{When do you play tennis?: 一般動詞のとき}
\begin{frame}[plain]\frametitle{When do you play tennis?}
%\Large
一般動詞のとき

\pause

\mbox{}\hspace{55pt}%
You play tennis \alt<4->{\myAnch{FOCUS2}{NavyBlue}{on Sundays}}{\myAnch{focus2}{white}{on Sundays}}.

\pause

\mbox{}\hfill%
{\small cf. \myEmph[6-]{Maroon}{Do you play} tennis on Sundays? \scalebox{1.4}{\myRisingPitch}}

\vspace{-5pt}

\hfill\visible<3->{{\small Yes/Noで答える疑問文}}

\visible<5->{\myAnch{WH2}{NavyBlue}{When} \myEmph[6-]{Maroon}{do you play} tennis?}%\myAnch{question2}{orange}{?}}
\visible<7->{\scalebox{1.4}{\myDownwardPitch}}

\visible<5->{%
\begin{tikzpicture}[remember picture, overlay]
 \draw[line width = 3pt, opacity = .5, NavyBlue, ->] (focus2.south) to[out=-90, in=90] node[sloped,above,text=black,font=\tiny,pos=.6]{Whenに置き換えて先頭へ} node[sloped,below,text=black,font=\tiny,pos=.4]{その後は疑問文の語順}(WH2.north);
\end{tikzpicture}
}

\vspace{-20pt}

\begin{block}<8->{Topics for Today}
\pause
\begin{itemize}\setbeamertemplate{items}[square]\small
 \item<9-> 疑問詞{\bfseries When}を先頭に置く
 \item<10-> {\bfseries When}の後ろはYes/Noで答える疑問文の語順と同じ

\hfill{}When\,\,\,$\left\{ \begin{tabular}{l}
	  do / does\\
	  did\\
	  will\\
	 \end{tabular}\right\}$%
\,\,$+$ S $+$ 原形 \ldots ? 


%\hfill{}When do you 原形 \ldots ?\\
%\hfill{}When does she 原形 \ldots ?\\
%\hfill{}When did they 原形 \ldots ?
 \item<11->   文末に`?'をつける / イントネーションは\myDownwardPitch{}\\
\end{itemize}
     \end{block}
\hfill{\tiny 0137}\,{\scriptsize \myaudio{./audio/014_when_02.mp3}}
\end{frame}
%%%%%%%%%%%%%%%%%%
\subsection{Exercises}
\begin{frame}[plain]{Exercises 1}
つぎの文の意味を考えましょう

\begin{tabular}{rll}
1& When is the concert? &{\scriptsize concert: コンサート} \\
2& When is the party?& \\
3& When do you get up? & {\scriptsize get up: 起きる}\\
4& When does the meeting begin?&{\scriptsize meeting: 会議 begin: 始まる} \\
\end{tabular}


\mbox{}\hfill{\tiny 0155}\,{\scriptsize \myaudio{./audio/014_when_03.mp3}}

\end{frame}

\begin{frame}[plain]{Exercises 2}

 {\small 次の質問に対する答えとしてもっとも適切なものを、下のア~ウの中から選びましょう}

\begin{enumerate}
 \item When do you usually\footnote{usually \textipa{/j\'u:Zu@li/} いつも} practice\footnote{practice \textipa{/pr\'\ae ktIs/} 練習する} basketball?\hspace{10pt}\visible<2->{--- On Saturdays.}\hspace{21pt}\visible<3->{イ}
 \item When is St. Valentine's Day\footnote{バレンタインデー}?\hspace{80pt}\visible<4->{--- It's February 14th.}\hspace{12pt}\visible<5->{ウ}
 \item When does the school year start in Japan?\hspace{15pt}\visible<6->{--- In April.}\hspace{50pt}\visible<7->{ア}
\end{enumerate}

\bigskip

\begin{tcolorbox}
\centering
ア In April.~~~~~~~~%
イ On Saturdays.~~~~~~~~%
ウ It's February 14th. 
\end{tcolorbox}

%
\mbox{}\hfill{\tiny 0235}\,{\scriptsize \myaudio{./audio/014_when_04.mp3}}


\end{frame}


\begin{frame}[plain]{Exercises 3}
 (~~~~~~~~) 内の語句を並べかえ、AとBの対話を完成させましょう。なお、先頭の語は大文字で始めてください

\begin{enumerate}
 \item A: ( Children's Day / when / is ) ? 
\hspace{137.4pt}B: It's May 5th.\\
\phantom{A: }\visible<2->{When is Children's Day?}
 \item A: ( birthday / your / father's / when / is ) ?
\hspace{83.5pt}B: It's July 2nd.\\
\phantom{A: }\visible<3->{When is your father's birthday?}
 \item A: ( to school / does / Mr. Brown / come / when ) ?
\hspace{47pt}B: Around\footnotemark[1] 7:00.\\
\phantom{A: }\visible<4->{When does Mr. Brown come to school?}
 \item A: ( does / the violin / when / play / your brother ) ?
\hspace{39.7pt}B: After dinner.\\
\phantom{A: }\visible<5->{When does your brother play the violin?}
\end{enumerate}

\mbox{}\hfill{\tiny 0256}\,{\scriptsize \myaudio{./audio/014_when_05.mp3}}

\footnotetext[1]{(時刻) ~ごろ}
\end{frame}
%%%%%%%%%%%%%%%%%%%%%%%%%%%%%%%%%%%%%%%%%%%%%%%%%%
\section{曜日}
%%%%%%%%%%%%%%%%%%%%%%%%%%%%%%%%%%%%%%%%%%%%%%%%%%
\begin{frame}[plain]{曜日}
\centering
\begin{tblr}{
  colspec = {rll}, 
%  column{2} = {fg=blue},   % 第7列の文字を青に
 row{odd} = {bg=azure8},
 row{1} = { bg=azure3, fg=white},
 row{2} = {fg=Maroon!80},    % 第2列の文字を赤に
 row{Z} = {fg=azure3},
 hline{Z} = {0.08em},    % \toprule, \midrule, \bottomrule
%  hline{3} = {0.5pt}       % もう1つの \midrule
 baseline=t,
 cell{1}{3} = {halign=r}
}
 & & {\tiny 0415}\,{\scriptsize \myaudio{../misc/audio/002_day_month_season_01.mp3}}\\
  日 & Sunday & \textipa{/s\'\textturnv nd\`eI/}\\
  月 & Monday & \textipa{/m\'\textturnv nd\`eI/}\\
  火 & Tuesday & \textipa{/t{\it j}\'u:zd\`eI/}\\ 
  水 & Wednesday & \textipa{/w\'enzd\`eI/}\\
  木 & Thursday & \textipa{/T\'\textrhookschwa :zd\`eI/}\\
  金 & Friday & \textipa{/fr\'aId\`eI/}\\
  土 & Saturday & \textipa{/s\'\ae t\textrhookschwa d\`eI/}\\
\end{tblr}
\end{frame}
%%%%%%%%%%%%%%%%%%%%%%%%%%%%%%%%%%%%%%%%%%%%%%%%%%
\section{month \textipa{/m\'\textturnv nT/}   season \textipa{/s\'\i:zn/}}
%%%%%%%%%%%%%%%%%%%%%%%%%%%%%%%%%%%%%%%%%%%%%%%%%%
%%%%%%%%%%%%%%%%%%%%%%%%%%%%%%%
\begin{frame}[plain]
\small

\begin{columns}
\begin{column}{.55\linewidth}
\begin{tblr}{
  colspec = {rlll}, 
%  column{2} = {fg=blue},   % 第7列の文字を青に
 row{1} = { bg=azure3, fg=white},
 row{2-3,13} = {bg=azure8},
 row{4-6} = {bg=SpringGreen!60},    
 row{7-9} = {bg=Goldenrod!60},    
 row{10-12} = {bg=Maroon!60!black!80, fg=white},
 column{4} = {bg=white},    
% row{Z} = {fg=azure3},
% hline{Z} = {0.08em},    % \toprule, \midrule, \bottomrule
%  hline{3} = {0.5pt}       % もう1つの \midrule
 baseline=t,
 cell{1}{3} = {halign=r},
 cells={cmd=\onslide<\arabic{rownum}->} %%%%tabularrayとpauseが衝突することを回避する方法→https://github.com/lvjr/tabularray/issues/226
}
    & month& {\scriptsize \textipa{/m\'\textturnv nT/}\hspace{15pt}{\tiny 0358}\,\myaudio{../misc/audio/002_day_month_season_02a.mp3}}\\
  1月 & January & \textipa{/dZ\'\ae nju\`eri/}&\myAnch{fuyu1}{white}{}\\
  2月 & February & \textipa{/f\'eb{\it r}u\`eri/}\\
  3月 & March & \textipa{/m\'A\textrhookschwa tS/}\\ 
  4月 & April & \textipa{/\'eIpr@l/}&\myAnch{haru}{white}{}\\
  5月 & May & \textipa{/m\'eI/}\\
  6月 & June & \textipa{/dZ\'u:n/}\\
  7月 & July & \textipa{/dZUl\'aI/}&\myAnch{natsu}{white}{}\\
  8月 & August & \textipa{/\'O:g@st/}\\ 
  9月 & September & \textipa{/sept\'emb\textrhookschwa /}\\
  10月 & October & \textipa{/Akt\'oUb\textrhookschwa /}&\myAnch{aki}{white}{}\\
  11月 & November & \textipa{/nouv\'emb\textrhookschwa /}\\
  12月 & December & \textipa{/dIs\'emb\textrhookschwa /}&\myAnch{fuyu2}{white}{}\\
\end{tblr}
\end{column}
\begin{column}<14->{.4\linewidth}
\begin{tblr}{
  colspec = {lll}, 
%  column{2} = {fg=blue},   % 第7列の文字を青に
row{1} = { bg=azure3, fg=white},
 row{5} = {bg=azure8},
 row{2} = {bg=SpringGreen!60},    
 row{3} = {bg=Goldenrod!60},    
 row{4} = {bg=Maroon!60!black!80, fg=white},
column{1} = {bg=white},
  baseline=t,
 cells={cmd=\onslide<\arabic{rownum}->} %%%%tabularrayとpauseが衝突することを回避する方法→https://github.com/lvjr/tabularray/issues/226
}
    & season& {\scriptsize \textipa{/s\'\i:zn/}\hspace{20pt}{\tiny 0217}\,\myaudio{../misc/audio/002_day_month_season_02b.mp3}}\\
 \myAnch{spring}{white}{}&spring&\textipa{/spr\'IN/}\\
 \myAnch{summer}{white}{}&summer&\textipa{/s\'\textturnv m\textrhookschwa /}\\
 \myAnch{fall}{white}{}&fall / autumn&\textipa{/f\'O:l/} \textipa{/\'O:t@m/}\\
 \myAnch{winter}{white}{}&winter&\textipa{/w\'Int\textrhookschwa/}\\\
\end{tblr}
\end{column}
\end{columns}

\begin{tikzpicture}[remember picture,overlay]
 \onslide<15->{\draw[<-,line width=3pt,SpringGreen,opacity=.75] (haru.east) -- (spring.west);}
 \onslide<16->{\draw[<-,line width=3pt,Goldenrod,opacity=.75] (natsu.east) -- (summer.west);}
 {\onslide<17->\draw[<-,line width=3pt,Maroon,opacity=.75] (aki.east) -- (fall.west);}
 \onslide<18->{\draw[<-,line width=3pt,azure3,opacity=.75] (fuyu1.east) to[out=0,in=180] (winter.west);}
 \onslide<18->{\draw[<-,line width=3pt,azure3,opacity=.75] (fuyu2.315) to[out=0,in=200] (winter.west);}
\end{tikzpicture}
\end{frame}
%%%%%%%%%%%%%%%%%%%%%%%%%%%%%%%
%%%%%%%%%%%%%%%%%%%%%%%%%%%%%%%%
\section{疑問詞when \textipa{/w\'en/} のまとめ}
\begin{frame}[plain]{まとめ}
 \begin{block}{When ~? \textipa{/w\'en/}}
\begin{description}[    ]
 \item[be動詞]<2-> $\HighlightFormula{\text{When is ~?}}$%

\begin{enumerate}
 \item When is your birthday?\\
\mbox{}\\
\mbox{}
\end{enumerate}

 \item[一般動詞]<3-> $\HighlightFormula{\text{When}\left\{ \begin{tabular}{l}
	  do / does\\
	  did\\
	  will\\
	 \end{tabular}\right\} + \text{S} + \text{原形 \ldots ?}}$%
       \begin{enumerate}\setcounter{enumi}{1}
       \item<4-> When do you play tennis?
       \item<5-> When does she study science?
       \item<6-> When did he go to America?
       \item<7-> When will they come back?
      \end{enumerate}
\end{description}
 \end{block}
\mbox{}\hfill{\tiny 0220}\,{\scriptsize \myaudio{./audio/014_when_06.mp3}}

\end{frame}
%%%%%%%%%%%%%%%%%%%%%%%%%
\end{document}
