\documentclass[aspectratio=169,xcolor={dvipsnames,table}]{beamer}
\usepackage[no-math,deluxe,haranoaji]{luatexja-preset}
\renewcommand{\kanjifamilydefault}{\gtdefault}
\renewcommand{\emph}[1]{{\upshape\bfseries #1}}
\usetheme{metropolis}
\metroset{block=fill}
\setbeamertemplate{navigation symbols}{}
\setbeamertemplate{blocks}[rounded][shadow=false]
\usecolortheme[rgb={0.7,0.2,0.2}]{structure}
%%%%%%%%%%%%%%%%%%%%%%%%%%%
\usepackage{media9}
%%%%%%%%%%%%%%%%%%%%%%%%%%%
%% さまざまなアイコン
%%%%%%%%%%%%%%%%%%%%%%%%%%%
\usepackage{fontawesome}
\usepackage{figchild}
\usepackage{twemojis}
\usepackage{utfsym}
\usepackage{bclogo}
\usepackage{marvosym}
\usepackage{fontmfizz}
\usepackage{pifont}
\usepackage{phaistos}
\usepackage{worldflags}
%%%%%%%%%%%%%%%%%%%%%%%%%%%
\usepackage{tikz}
\usetikzlibrary{backgrounds}
\usepackage{tcolorbox}
\usepackage{tikzpeople}
\usepackage{circledsteps}
\usepackage{xcolor}
\usepackage{amsmath}
\usepackage{booktabs}
\usepackage{tipa}
%%%%%%%%%%%%%%%%%%%%%%%%%%%
%% 場合分け
\usepackage{cases}
%%%%%%%%%%%%%%%%%%%%%%%%%%%
% \myAnch{<名前>}{<色>}{<テキスト>}
% 指定のテキストを指定の色の四角枠で囲み, 指定の名前をもつTikZの
% ノードとして出力する. 図には remeber picture 属性を付けている
% ので外部から参照可能である.
\newcommand*{\myAnch}[3]{%
  \tikz[remember picture,baseline=(#1.base)]
    \node[draw,rectangle,#2] (#1) {\normalcolor #3};
}
%%%%%%%%%%%%%%%%%%%%%%%%%%%%
%% 音声リンク表示
\newcommand{\myaudio}[1]{\href{#1}{\faVolumeUp}}
%%%%%%%%%%%%%%%%%%%%%%%%%%%
% \myEmph コマンドの定義
%\newcommand{\myEmph}[3]{%
%    \textbf<#1>{\color<#1>{#2}{#3}}%
%}
\usepackage{xparse} % xparseパッケージの読み込み
\NewDocumentCommand{\myEmph}{O{} m m}{%
    \def\argOne{#1}%
    \ifx\argOne\empty
        \textbf{\color{#2}{#3}}% オプション引数が省略された場合
    \else
        \textbf<#1>{\color<#1>{#2}{#3}}% オプション引数が指定された場合
    \fi
}
%%%%%%%%%%%%%%%%%%%%%%%%%%%
%% 文末の上昇イントネーション記号 \myRisingPitch
%% 通常のイントネーション \myDownwardPitch
%% https://note.com/dan_oyama/n/n8be58e8797b2
%%%%%%%%%%%%%%%%%%%%%%%%%%%
\newcommand{\myRisingPitch}{
\begin{tikzpicture}[scale=0.3,baseline=0.3]
\draw[->,>=stealth] (0,0) to[bend right=45] (1,1);
\end{tikzpicture}
}
\newcommand{\myDownwardPitch}{
\begin{tikzpicture}[scale=0.3,baseline=0.3]
\draw[->,>=stealth] (0,1) to[bend left=45] (1,0);
\end{tikzpicture}
}
%%%%%%%%%%%%%%%%%%%%%%%%%%%
\title{English is fun.}
\subtitle{I was busy yesterday.}
\author{}
\institute[]{}
\date[]

%%%%%%%%%%%%%%%%%%%%%%%%%%%%
%% TEXT
%%%%%%%%%%%%%%%%%%%%%%%%%%%%
\begin{document}
\begin{frame}[plain]
  \titlepage
\end{frame}

\section*{授業の流れ}
\begin{frame}[plain]
  \frametitle{授業の流れ}
  \tableofcontents
\end{frame}


\section{be動詞の現在形(復習)}

\begin{frame}[plain]{be動詞の現在形(復習)}

次の英文の(~~~~~~~~)内から正しいものを選び○で囲みましょう。

\begin{enumerate}
 \item I ( \alt<2->{\Circled[outer color=orange]{am}}{am} / are  / is ) hungry.
 \item You ( am / \alt<3->{\Circled[outer color=orange]{are}}{are} / is ) very kind.
 \item She ( am / are / \alt<4->{\Circled[outer color=orange]{is}}{is} ) from Australia.
 \item Tom ( am / are / \alt<5->{\Circled[outer color=orange]{is}}{is} ) busy today.
 \item The flowers( am / is / \alt<6->{\Circled[outer color=orange]{are}}{are} ) beautiful.
\end{enumerate}

\hfill\myaudio{./audio/024_past_be_01.mp3}

\begin{exampleblock}{Topic for Today}
\pause
\begin{itemize}\small
 \item be動詞の現在形は主語によって使い分けがあります
\end{itemize}
     \end{exampleblock}

\end{frame}



\begin{frame}[plain]\frametitle{be動詞の現在形(復習)}

\begin{block}{現在形のbe動詞の使い分け}

{\large

\begin{numcases}{\text{主語が\,\,\,\,}}
 \text{\mbox{}\,\,{}I}&$\longrightarrow$\,\,\,\,\,\,{}\text{am}\\
 \text{\mbox{}\,\,{}you}&$\longrightarrow$\,\,\,\,\,\,{}\text{are}\\
 \text{\mbox{}\,\,{}単数(1つ、1人)}&$\longrightarrow$\,\,\,\,\,\,{}\text{is}\\
 \text{\mbox{}\,\,{}複数(2つ、2人以上)}&$\longrightarrow$\,\,\,\,\,\,{}\text{are}
\end{numcases}
}
\end{block}
\end{frame}

\begin{frame}[plain]\frametitle{使い分けを図にすると}
 \centering
\begin{tikzpicture}
% 補助グリッドを描画
%\draw[step=1cm, gray!20, very thin] (-6,-2) grid (6,6);
% ノードの定義と配置
\node[circle, draw=black, fill=yellow!30, minimum size=20mm, line width=1pt] (A) at (0,0) {\LARGE be動詞};\pause
\node[circle, draw=black, fill=pink!30,minimum size=15mm, line width=1pt] (B) at (-6,0) {\LARGE am};\pause
\node[circle, draw=black, fill=blue!30, minimum size=15mm, line width=1pt] (C) at (0,5) {\LARGE are};\pause
\node[circle, draw=black, fill=green!30, minimum size=15mm, line width=1pt] (D) at (6,0) {\LARGE is};\pause

% ノード間の線の描画
\draw[-latex, line width=1.5pt] (A) -- node[above] {Iのとき} (B);\pause
\draw[-latex, line width=1.5pt] (A) -- node[sloped, above] {youのとき}  node[sloped, below] {複数なら} (C);\pause
\draw[-latex, line width=1.5pt] (A) -- node[above] {単数なら} node[below] {} (D);
\end{tikzpicture}
\end{frame}

\section{be動詞の過去形}
\subsection{過去形(wasとwere)の使い分け}
\begin{frame}[plain]{be動詞の過去形}

\begin{tabular}{rll}
&\multicolumn{1}{c}{現在形}&\multicolumn{1}{c}{過去形}\\
1&\visible<1->{I am busy now.\hspace{30pt}{\scriptsize busy \textipa{/b\'Izi/} 忙しい\hspace*{20pt}\mbox{}}}&       \visible<2->{I \textcolor{Maroon}{\bfseries was} busy yesterday.}\\
2&\visible<1->{We are busy now.}&     \visible<3->{We \textcolor{NavyBlue}{\bfseries were} busy yesterday}.\\
3&\visible<1->{You are busy now.}&    \visible<4->{You \textcolor{NavyBlue}{\bfseries were} busy yesterday.}\\
4&\visible<1->{He is busy now.}&      \visible<5->{He \textcolor{Maroon}{\bfseries was} busy yesterday.}\\
5&\visible<1->{She is busy now.}&     \visible<6->{She \textcolor{Maroon}{\bfseries was} busy yesterday.}\\
6&\visible<1->{They are busy now.}&   \visible<7->{They \textcolor{NavyBlue}{\bfseries were} busy yesterday.}
\end{tabular}

\onslide<8->{現在形は3つ(am, are, is)ありましたが、過去形は2つ(was, were)だけです}

\hfill\myaudio{./audio/024_past_be_02.mp3}

\end{frame}


\begin{frame}[plain]\frametitle{be動詞の過去形の使い分け}

\begin{block}{使い分け}

{
\setcounter{equation}{0}
\begin{numcases}{\text{主語が\,\,\,\,}}
 \text{\mbox{}\,\,{}I}&$\longrightarrow$\,\,\,\,\,\,{}\text{was}\\
 \text{\mbox{}\,\,{}you}&$\longrightarrow$\,\,\,\,\,\,{}\text{were}\\
 \text{\mbox{}\,\,{}単数(1つ、1人)}&$\longrightarrow$\,\,\,\,\,\,{}\text{was}\\
 \text{\mbox{}\,\,{}複数(2つ、2人以上)}&$\longrightarrow$\,\,\,\,\,\,{}\text{were}
\end{numcases}
}
\end{block}
\end{frame}


\begin{frame}[plain]\frametitle{過去形の使い分けを図にすると}
 \centering
\begin{tikzpicture}
% 補助グリッドを描画
%\draw[step=1cm, gray!20, very thin] (-6,-2) grid (6,6);
% ノードの定義と配置
\node[circle, draw=black, fill=yellow!30, minimum size=20mm, line width=1pt] (A) at (0,0) {\LARGE be動詞};\pause
\node[circle, draw=black, fill=Maroon!30,minimum size=20mm, line width=1pt] (B) at (-6,0) {\LARGE was};\pause
%\node[circle, draw=black, fill=blue!30, minimum size=15mm, line width=1pt] (C) at (0,5) {\LARGE are};\pause
\node[circle, draw=black, fill=NavyBlue!30, minimum size=20mm, line width=1pt] (D) at (6,0) {\LARGE were};\pause

% ノード間の線の描画
\draw[-latex, line width=1.5pt] (A) -- node[above] {Iのとき} node[below] {単数なら}(B);\pause
%\draw[-latex, line width=1.5pt] (A) -- node[sloped, above] {Youのとき}  node[sloped, below] {複数形なら} (C);\pause
\draw[-latex, line width=1.5pt] (A) -- node[above] {youのとき} node[below] {複数なら} (D);
\end{tikzpicture}
\end{frame}

\begin{frame}[plain]{Exercises}
 
次の英文の(~~~~~~~~)内から正しいものを選び○で囲みましょう。

\begin{enumerate}
 \item I ( am/ is /are / \alt<2->{\Circled[outer color=orange]{was}}{was} / were ) hungry yesterday.
 \item They ( am / is / are / was / \alt<3->{\Circled[outer color=orange]{were}}{were} ) busy yesterday.
 \item She  ( am/ is /are / \alt<4->{\Circled[outer color=orange]{was}}{was} / were ) tired last night.\hfill{}{\small tired \textipa{/t\'aI\textrhookschwa d/} 疲れて}
 \item Tom ( am / \alt<5->{\Circled[outer color=orange]{is}}{is} / are / were ) busy today.
 \item They( am / is / are / was / \alt<6->{\Circled[outer color=orange]{were}}{were} ) at home then.\\
\mbox{}\hfill{}{\small at home: 在宅して then: そのとき}
\end{enumerate}

\hfill\myaudio{./audio/024_past_be_03.mp3}

\end{frame}


\section{否定文}
\subsection{否定文のつくり方}
\begin{frame}[plain]\frametitle{was / wereの否定}
\Large

\begin{enumerate}
 \item \begin{enumerate}\Large
	\item \visible<1->{I \textcolor{NavyBlue}{\bfseries was} hungry yesterday.\hfill{\scriptsize hungry\,\textipa{/h\'2Ngri/}\,空腹の}}
	\item \visible<2->{I \textcolor{NavyBlue}{\bfseries was} \textcolor{Maroon}{\bfseries not} hungry yesterday.}
       \end{enumerate}
 \item \begin{enumerate}\Large
	\item \visible<1->{They \textcolor{NavyBlue}{\bfseries were} tired last night.\hfill{}{\scriptsize tired\,\textipa{/t\'aI\textrhookschwa d/}\,疲れて}}
	\item \visible<3->{They \textcolor{NavyBlue}{\bfseries were} \textcolor{Maroon}{\bfseries not} tired last night.}
       \end{enumerate}
\end{enumerate}


\visible<4->{\begin{exampleblock}{Topics for Today}
\small
\begin{itemize}
 \item \visible<5->{否定文にするには、be動詞の直後に\,\Circled[fill color = white]{\,\,not\,\,}\,を置きます(現在形のときと同じですね)}
 \item \visible<6->{縮めて was not $\rightarrow$\,\,\,\textcolor{BurntOrange}{\bfseries wasn't} /  were not $\rightarrow$ \textcolor{BurntOrange}{\bfseries weren't}}
\end{itemize}
      \end{exampleblock}}

\hfill\myaudio{./audio/024_past_be_04.mp3}

\end{frame}


\begin{frame}[plain]{Exercises}
あたえられた日本語の意味になるよう、カッコ内の語を並べ替えましょう。なお、先頭の語は大文字で始めてください

\begin{enumerate}
 \item 私は昨日お腹が空いていませんでした。
( not / was / I / hungry ) yesterday.\\
\visible<2->{I was not hungry yesterday.}
 \item 
彼らは昨晩忙しくなかった。
( not / they / busy / were ) last night.\\
\visible<3->{They were not busy last night.}
 \item 
彼女は疲れていませんでした。
( not / she / was / tired ) .\\
\visible<4->{She was not tired.}
 \item 
あなたは当時家にいませんでした。
( you / not / home / at / were ) then.\\
\visible<5->{You were not at home then.}\hfill{\scriptsize at home: 在宅して}
\end{enumerate}

\hfill\myaudio{./audio/024_past_be_05.mp3}

\end{frame}


\section{疑問文}
\subsection{疑問文のつくり方}
\begin{frame}[plain]\frametitle{was / wereの疑問文}
\Large

\begin{enumerate}
 \item \begin{enumerate}\Large
	\item \visible<1->{You \textcolor{NavyBlue}{\bfseries were} hungry.}
	\item \visible<2->{\textcolor{NavyBlue}{\bfseries Were} you hungry?}
       \end{enumerate}
 \item \begin{enumerate}\Large
	\item \visible<1->{She \textcolor{NavyBlue}{\bfseries was} tired last night.}
	\item \visible<3->{\textcolor{NavyBlue}{\bfseries Was} she tired last night?}
       \end{enumerate}
\end{enumerate}

\visible<4->{\begin{exampleblock}{Topics for Today}
\small
\begin{itemize}
 \item \visible<5->{疑問文にするには、be動詞を先頭に置きます}
 \item \visible<6->{現在形のときと同じです}
\end{itemize}
      \end{exampleblock}}

\hfill\myaudio{./audio/024_past_be_06.mp3}

\end{frame}

%%%%%%%%%%%%%%%%%%%%%%%
\begin{frame}[plain]{Exercises}
あたえられた日本語の意味になるよう、カッコ内の語句を並べ替えましょう。なお、先頭の語は大文字で始めてください。 
 \begin{enumerate}
	    \item あなたは昨日お腹が空いていましたか?
( you / were / hungry ) yesterday?\\
\visible<2->{Were you hungry yesterday?}
    \item 彼らは昨晩忙しかったですか?
( they / busy / were ) last night?\\
\visible<3->{Were they busy last night?}
	    \item 彼女は昨日疲れていましたか?
( she / tired / was ) yesterday?\\
\visible<4->{Was she tired yesterday?}
	    \item あなたは当時家にいましたか?
( you / at / home / were ) then?\\
\visible<5->{Were you at home then?}

 \end{enumerate}
\hfill\myaudio{./audio/024_past_be_07.mp3}

\end{frame}

%%%%%%%%%%%%%%%%%%%%%
\subsection{疑問文への答え方}
\begin{frame}<1-16>[plain]\frametitle{疑問文への答え方}
例にならって、つぎの質問に対する答えを「はい」と「いいえ」の2通りつくりましょう
\mbox{}\hfill\myaudio{./audio/024_past_be_08z.mp3}


\begin{tabular}{rlcll}
\visible<1->{例}& \visible<1->{Were you busy?}& \visible<2->{$\rightarrow$}&\visible<3->{(1) Yes, I was.}&\visible<3->{(2) No, I wasn't.}\\
\visible<1->{1}&\visible<1->{Was she tired last night?}&\visible<5->{$\rightarrow$}&\visible<6->{(1) Yes, she was.}&\visible<7->{(2) No, she wasn't.}\\
\visible<1->{2}&\visible<1->{Were they at home then?}&\visible<8->{$\rightarrow$}& \visible<9->{(1) Yes, they were.}&\visible<10->{(2) No, they weren't.}\\
\visible<1->{3}&\visible<1->{Was Peter in Japan then?}&\visible<11->{$\rightarrow$}&\visible<12->{(1) Yes, he was.}&\visible<13->{(2) No, he wasn't.}\\
\visible<1->{4}&\visible<1->{Was it rainy yesterday?}&\visible<14->{$\rightarrow$}&\visible<15->{(1) Yes, it was.}&\visible<16->{(2) No, it wasn't.}\\
&&&&{\scriptsize rainy\,\textipa{/r\'eIni/}\,雨の}
\end{tabular}

\begin{exampleblock}<4->{Topics for Today}
\small
\begin{itemize}
 \item Yes, S $+ \left\{\begin{array}{l}
		  \text{was}\\
		\text{were}\end{array}\right\}$\,\,.
\hspace{20pt}
No, S $+ \left\{\begin{array}{l}
		  \text{was not($=$ wasn't)}\\
		\text{were not($=$ weren't)}\end{array}\right\}$\,\,.
\end{itemize}
      \end{exampleblock}
\end{frame}

\end{document}
