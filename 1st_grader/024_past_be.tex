\documentclass[aspectratio=169,xcolor={dvipsnames,table}]{beamer}
\usepackage[no-math,deluxe,haranoaji]{luatexja-preset}
\renewcommand{\kanjifamilydefault}{\gtdefault}
\renewcommand{\emph}[1]{{\upshape\bfseries #1}}
\usetheme{metropolis}
\metroset{block=fill}
\setbeamertemplate{navigation symbols}{}
\setbeamertemplate{blocks}[rounded][shadow=false]
\usecolortheme[rgb={0.7,0.2,0.2}]{structure}
%%%%%%%%%%%%%%%%%%%%%%%%%%
%% Change alert block colors
%%% 1- Block title (background and text)
\setbeamercolor{block title alerted}{fg=mDarkTeal, bg=mLightBrown!45!yellow!45}
\setbeamercolor{block title example}{fg=magenta!10!black, bg=mLightGreen!60}
%%% 2- Block body (background)
\setbeamercolor{block body alerted}{bg=mLightBrown!25}
\setbeamercolor{block body example}{bg=mLightGreen!15}
%%%%%%%%%%%%%%%%%%%%%%%%%%%
%%%%%%%%%%%%%%%%%%%%%%%%%%%
\usepackage{pxrubrica}
%%%%%%%%%%%%%%%%%%%%%%%%%%%
%% さまざまなアイコン
%%%%%%%%%%%%%%%%%%%%%%%%%%%
\usepackage{fontawesome}
\usepackage{figchild}
\usepackage{twemojis}
\usepackage{utfsym}
\usepackage{bclogo}
\usepackage{marvosym}
\usepackage{fontmfizz}
\usepackage{pifont}
\usepackage{phaistos}
\usepackage{worldflags}
%%%%%%%%%%%%%%%%%%%%%%%%%%%
\usepackage{tikz}
\usetikzlibrary{backgrounds}
\usepackage{tcolorbox}
\usetikzlibrary{tikzmark}
\usepackage{tikzpeople}
\usepackage{tikzducks}
\usepackage{circledsteps}
\usepackage{xcolor}
\usepackage{amsmath}
\usepackage{booktabs}
\usepackage{tipa}
\usepackage{tabularray}
%%%%%%%%%%%%%%%%%%%%%%%%%%%
%% 場合分け
\usepackage{cases}
%%%%%%%%%%%%%%%%%%%%%%%%%%%
% \myAnch{<名前>}{<色>}{<テキスト>}
% 指定のテキストを指定の色の四角枠で囲み, 指定の名前をもつTikZの
% ノードとして出力する. 図には remeber picture 属性を付けている
% ので外部から参照可能である.
\newcommand*{\myAnch}[3]{%
  \tikz[remember picture,baseline=(#1.base)]
    \node[draw,rectangle,#2] (#1) {\normalcolor #3};
}
%%%%%%%%%%%%%%%%%%%%%%%%%%%%
%% 音声リンク表示
\newcommand{\myaudio}[1]{\href{#1}{\faVolumeUp}}
%%%%%%%%%%%%%%%%%%%%%%%%%%%
% \myEmph コマンドの定義
%\newcommand{\myEmph}[3]{%
%    \textbf<#1>{\color<#1>{#2}{#3}}%
%}
\usepackage{xparse} % xparseパッケージの読み込み
\NewDocumentCommand{\myEmph}{O{} m m}{%
    \def\argOne{#1}%
    \ifx\argOne\empty
        \textbf{\color{#2}{#3}}% オプション引数が省略された場合
    \else
        \textbf<#1>{\color<#1>{#2}{#3}}% オプション引数が指定された場合
    \fi
}
%%%%%%%%%%%%%%%%%%%%%%%%%%%
%% 文末の上昇イントネーション記号 \myRisingPitch
%% 通常のイントネーション \myDownwardPitch
%% https://note.com/darcledn_oyama/n/n8be58e8797b2
%%%%%%%%%%%%%%%%%%%%%%%%%%%
\newcommand{\myRisingPitch}{
\begin{tikzpicture}[scale=0.3,baseline=0.3]
\draw[->,>=stealth] (0,0) to[bend right=45] (1,1);
\end{tikzpicture}
}
\newcommand{\myDownwardPitch}{
\begin{tikzpicture}[scale=0.3,baseline=0.3]
\draw[->,>=stealth] (0,1) to[bend left=45] (1,0);
\end{tikzpicture}
}
%%%%%%%%%%%%%%%%%%%%%%%%%%%
\title{English is fun.}
\subtitle{I was busy yesterday.}
\author{}
\institute[]{}
\date[]

%%%%%%%%%%%%%%%%%%%%%%%%%%%%
%% TEXT
%%%%%%%%%%%%%%%%%%%%%%%%%%%%
\begin{document}
\begin{frame}[plain]
  \titlepage
\end{frame}

\section*{授業の流れ}
\begin{frame}[plain]
  \frametitle{授業の流れ}
  \tableofcontents
\end{frame}


\section{be動詞の現在形(復習)}

\begin{frame}[plain]{be動詞の現在形(復習)}

次の英文の(~~~~~~~~)内から正しいものを選び○で囲みましょう

\begin{enumerate}
 \item I ( \alt<2->{\Circled[outer color=orange]{am}}{am} / are  / is ) hungry.
 \item You ( am / \alt<3->{\Circled[outer color=orange]{are}}{are} / is ) very kind.
 \item She ( am / are / \alt<4->{\Circled[outer color=orange]{is}}{is} ) from Australia.\hfill{\scriptsize from ~の出身だ}
 \item Tom ( am / are / \alt<5->{\Circled[outer color=orange]{is}}{is} ) busy today.
 \item The flowers( am / is / \alt<6->{\Circled[outer color=orange]{are}}{are} ) beautiful.
\end{enumerate}

\hfill{\tiny 0222}{\scriptsize \myaudio{./audio/024_past_be_01.mp3}}

\begin{block}{Topics for Today}
\pause
\begin{itemize}\setbeamertemplate{items}[square]\small
 \item be動詞は $=$ の意味です
 \item be動詞の現在形は主語によって使い分けがあります
\end{itemize}
     \end{block}

\end{frame}
%%%%%%%%%%%%%%%%%%%%%%%%%%%%
\begin{frame}[plain]\frametitle{be動詞の現在形(復習)}

\begin{block}{現在形のbe動詞の使い分け}

{\large

\begin{numcases}{\text{主語が\,\,\,\,}}
 \text{\mbox{}\,\,{}I}&$\longrightarrow$\,\,\,\,\,\,{}\text{am}\\
 \text{\mbox{}\,\,{}you}&$\longrightarrow$\,\,\,\,\,\,{}\text{are}\\
 \text{\mbox{}\,\,{}単数(1つ、1人)}&$\longrightarrow$\,\,\,\,\,\,{}\text{is}\\
 \text{\mbox{}\,\,{}複数(2つ、2人以上)}&$\longrightarrow$\,\,\,\,\,\,{}\text{are}
\end{numcases}
}
\end{block}
\end{frame}

\begin{frame}[plain]\frametitle{使い分けを図にすると}
 \centering
\begin{tikzpicture}
% 補助グリッドを描画
%\draw[step=1cm, gray!20, very thin] (-6,-2) grid (6,6);
% ノードの定義と配置
\node[circle, draw=black, fill=yellow!30, minimum size=20mm, line width=1pt] (A) at (0,0) {\LARGE be動詞};\pause
\node[circle, draw=black, fill=pink!30,minimum size=15mm, line width=1pt] (B) at (-6,0) {\LARGE am};\pause
\node[circle, draw=black, fill=blue!30, minimum size=15mm, line width=1pt] (C) at (0,5) {\LARGE are};\pause
\node[circle, draw=black, fill=green!30, minimum size=15mm, line width=1pt] (D) at (6,0) {\LARGE is};\pause

% ノード間の線の描画
\draw[-latex, line width=1.5pt] (A) -- node[above] {Iのとき} (B);\pause
\draw[-latex, line width=1.5pt] (A) -- node[sloped, above] {youのとき}  node[sloped, below] {複数なら} (C);\pause
\draw[-latex, line width=1.5pt] (A) -- node[above] {単数なら} node[below] {} (D);
\end{tikzpicture}
\end{frame}

\section{be動詞の過去形}
\subsection{過去形(wasとwere)の使い分け}
\begin{frame}[plain]{be動詞の過去形}

\begin{tabular}{rll}
&\multicolumn{1}{c}{現在形}&\multicolumn{1}{c}{過去形}\\
1&\visible<1->{I am busy now.\hspace{30pt}{\scriptsize busy \textipa{/b\'Izi/} 忙しい\hspace*{20pt}\mbox{}}}&       \visible<2->{I \textcolor{Maroon}{\bfseries was} busy yesterday.}\\
2&\visible<1->{We are busy now.}&     \visible<3->{We \textcolor{NavyBlue}{\bfseries were} busy yesterday}.\\
3&\visible<1->{You are busy now.}&    \visible<4->{You \textcolor{NavyBlue}{\bfseries were} busy yesterday.}\\
4&\visible<1->{He is busy now.}&      \visible<5->{He \textcolor{Maroon}{\bfseries was} busy yesterday.}\\
5&\visible<1->{She is busy now.}&     \visible<6->{She \textcolor{Maroon}{\bfseries was} busy yesterday.}\\
6&\visible<1->{They are busy now.}&   \visible<7->{They \textcolor{NavyBlue}{\bfseries were} busy yesterday.}
\end{tabular}

\onslide<8->{現在形は3つ(am, are, is)ありましたが、過去形は2つ(was, were)だけです}

\hfill{\tiny 0419}{\scriptsize \myaudio{./audio/024_past_be_02.mp3}}

\end{frame}


\begin{frame}[plain]\frametitle{be動詞の過去形の使い分け}

\begin{block}{使い分け}

{
\setcounter{equation}{0}
\begin{numcases}{\text{主語が\,\,\,\,}}
 \text{\mbox{}\,\,{}I}&$\longrightarrow$\,\,\,\,\,\,{}\text{was}\\
 \text{\mbox{}\,\,{}you}&$\longrightarrow$\,\,\,\,\,\,{}\text{were}\\
 \text{\mbox{}\,\,{}単数(1つ、1人)}&$\longrightarrow$\,\,\,\,\,\,{}\text{was}\\
 \text{\mbox{}\,\,{}複数(2つ、2人以上)}&$\longrightarrow$\,\,\,\,\,\,{}\text{were}
\end{numcases}
}
\end{block}
\end{frame}


\begin{frame}[plain]\frametitle{過去形の使い分けを図にすると}
 \centering
\begin{tikzpicture}
% 補助グリッドを描画
%\draw[step=1cm, gray!20, very thin] (-6,-2) grid (6,6);
% ノードの定義と配置
\node[circle, draw=black, fill=yellow!30, minimum size=20mm, line width=1pt] (A) at (0,0) {\LARGE be動詞};\pause
\node[circle, draw=black, fill=Maroon!30,minimum size=20mm, line width=1pt] (B) at (-6,0) {\LARGE was};\pause
%\node[circle, draw=black, fill=blue!30, minimum size=15mm, line width=1pt] (C) at (0,5) {\LARGE are};\pause
\node[circle, draw=black, fill=NavyBlue!30, minimum size=20mm, line width=1pt] (D) at (6,0) {\LARGE were};\pause

% ノード間の線の描画
\draw[-latex, line width=1.5pt] (A) -- node[above] {Iのとき} node[below] {単数なら}(B);\pause
%\draw[-latex, line width=1.5pt] (A) -- node[sloped, above] {Youのとき}  node[sloped, below] {複数形なら} (C);\pause
\draw[-latex, line width=1.5pt] (A) -- node[above] {youのとき} node[below] {複数なら} (D);
\end{tikzpicture}
\end{frame}
%%%%%%%%%%%%%%%%%%%%%%%%%%%%%%%%%%
\begin{frame}[plain]{be動詞の現在形と過去形}
\Large
 \centering
\begin{tblr}{
  colspec={lll},
 row{odd} = {bg=azure8},
  row{1}={font=\bfseries, bg=azure3, fg=white},
  hline{Z} = {0.08em},    % \toprule, \midrule, \bottomrule
%  hline{3} = {0.5pt}       % もう1つの \midrule
}
主語 & 現在形 & 過去形 \\
I & am & was \\
you & are & were \\
単数(1人、1つ) & is & was \\
複数(2人、2つ以上) & are & were \\
\end{tblr}

\end{frame}
%%%%%%%%%%%%%%%%%%%%%%%%%%%%%%%%%%
\begin{frame}[plain]{Exercises}
 
次の英文の(~~~~~~~~)内から正しいものを選び○で囲みましょう

\begin{enumerate}
 \item I ( am/ is /are / \alt<2->{\Circled[outer color=orange]{was}}{was} / were ) hungry yesterday.
 \item They ( am / is / are / was / \alt<3->{\Circled[outer color=orange]{were}}{were} ) busy yesterday.
 \item She  ( am/ is /are / \alt<4->{\Circled[outer color=orange]{was}}{was} / were ) tired last night.\hfill{}{\scriptsize tired \textipa{/t\'aI\textrhookschwa d/} 疲れて}
 \item Tom ( am / \alt<5->{\Circled[outer color=orange]{is}}{is} / are / were ) busy today.\,\,\,{\scriptsize Tomはきょういそがしい}
 \item They( am / is / are / was / \alt<6->{\Circled[outer color=orange]{were}}{were} ) at home then.\\
       \mbox{}\hfill{}{\scriptsize at home: 在宅して then: そのとき}
\end{enumerate}

\hfill{\tiny 0206}{\scriptsize \myaudio{./audio/024_past_be_03.mp3}}

\end{frame}

\section{否定文}
\subsection{否定文のつくり方}
%%%%%%%%%%%%%%%%%%%%%%%%%%%%%%%%%
\begin{frame}<1>[plain,label=negative]\frametitle{was / wereの否定}
\large

\begin{enumerate}
 \item \begin{enumerate}\large
	\item \visible<1->{I \textcolor{NavyBlue}{\bfseries was} hungry yesterday.\hfill{\scriptsize hungry\,\textipa{/h\'2Ngri/}\,空腹の}}
	\item \visible<2->{I \textcolor{NavyBlue}{\bfseries was} \textcolor{Maroon}{\bfseries not} hungry yesterday.}
       \end{enumerate}
 \item \begin{enumerate}\large
	\item \visible<1->{They \textcolor{NavyBlue}{\bfseries were} tired last night.\hfill{}{\scriptsize tired\,\textipa{/t\'aI\textrhookschwa d/}\,疲れて\hspace{10pt}last \textipa{/l\'\ae st/} 最後の}}
	\item \visible<3->{They \textcolor{NavyBlue}{\bfseries were} \textcolor{Maroon}{\bfseries not} tired last night.}
       \end{enumerate}
\end{enumerate}

\bigskip

\visible<4->{\begin{block}{否定文のつくり方}
\small
\begin{itemize}\setbeamertemplate{items}[square]
 \item \visible<5->{否定文にするには、be動詞の直後に\,\Circled[fill color = white]{\,\,\textbf{not}\,\,}\,を置きます\\
\hfill{\scriptsize 現在形のときと同じです}}
 \item \visible<6->{縮めて was not $\rightarrow$\,\,\,\textcolor{BurntOrange}{\bfseries wasn't} /  were not $\rightarrow$ \textcolor{BurntOrange}{\bfseries weren't}}\\%
\hfill\visible<7->{\scriptsize {\kenten{短縮形}といいます}}
\end{itemize}
      \end{block}}

\hfill{\tiny 0144}\,{\scriptsize \myaudio{./audio/024_past_be_04.mp3}}

\end{frame}
%%%%%%%%%%%%%%%%%%%%%%%%%%%%
\begin{frame}[plain]{否定を表す語}
 \Large

否定を表す語:\pause {\LARGE\bfseries not}\hspace{20pt}\textipa{/n\'At/}


\hfill\begin{tikzpicture}
\duck[tshirt=black,
stripes={\stripes[color=white]},
football,
speech={\tiny 口を縦に全開},
bubblecolour=cyan!20!white,
think={ア},
laughing
]
\end{tikzpicture}


\end{frame}
%%%%%%%%%%%%%%%%%%%%%%%%%%%%%%%%%
\againframe[plain]{negative}
%%%%%%%%%%%%%%%%%%%%%%%%%%%%%%%%%

\begin{frame}[plain]{Exercises}

{\small あたえられた日本語の意味になるよう、カッコ内の語を並べ替えましょう。なお、先頭の語は大文字で始めてください}

\begin{enumerate}
 \item {\small 私は昨日お腹が空いていませんでした。}
( not / was / I / hungry ) yesterday.\\
\visible<2->{I was not hungry yesterday.}
 \item 
{\small 彼らは昨晩忙しくなかった。}
( not / they / busy / were ) last night.\\
\visible<3->{They were not busy last night.}
 \item 
{\small 彼女は疲れていませんでした。}
( not / she / was / tired ) .\\
\visible<4->{She was not tired.}
 \item 
{\small あなたは当時家にいませんでした。}
( you / not / home / at / were ) then.\\
\visible<5->{You were not at home then.}\hfill{\scriptsize at home: 在宅して}
\end{enumerate}

\hfill{\tiny 0204}\,{\scriptsize \myaudio{./audio/024_past_be_05.mp3}}
\end{frame}


\section{疑問文}
\subsection{疑問文のつくり方}
\begin{frame}[plain]\frametitle{was / wereの疑問文}
\Large

\begin{columns}
\begin{column}{.6\textwidth}
\begin{enumerate}
 \item \begin{enumerate}\large
	\item \visible<1->{You \tikzmark{were}\textcolor{NavyBlue}{\bfseries were} hungry.}\\[5pt]
\mbox{}
	\item \visible<2->{\tikzmark{q_were}\textcolor{NavyBlue}{\bfseries Were} you hungry?}
       \end{enumerate}
 \item \begin{enumerate}\large
	\item \visible<1->{She \tikzmark{was}\textcolor{NavyBlue}{\bfseries was} tired last night.}\\[5pt]
\mbox{}
	\item \visible<3->{\tikzmark{q_was}\textcolor{NavyBlue}{\bfseries Was} she tired last night?}
       \end{enumerate}
\end{enumerate}
\end{column}
%%%%%%%%%%%%%
\begin{column}<2->{.35\textwidth}
\begin{tikzpicture}
\duck[signpost=\scalebox{0.3}{
\parbox{2.5cm}{\color{black}
前に\\出すだけ}},
signcolour=brown!70!gray,
signback=white!80!brown,
graduate=gray!20!black,
tassel=red!70!black,
speech={\tiny 語順がだいじ}
]
\end{tikzpicture} 
\end{column}
\end{columns}

\bigskip

\visible<4->{\begin{block}{疑問文のつくり方}
\small
\setbeamertemplate{items}[square]
\begin{itemize}
 \item \visible<5->{疑問文にするには、be動詞を先頭に置きます}\\
\hfill\visible<6->{{\scriptsize 現在形のときと同じです}}
 \item<7-> \Circled[fill color=white]{\,S $+$ V\,}\,が\,\Circled[fill color=white]{\,V $+$ S\,}\,と順番が変わります\\
\hfill{{\scriptsize \kenten{倒置}といいます}}
\end{itemize}
      \end{block}}

\vspace*{-20pt}

\hfill{\tiny 0137}\,{\scriptsize \myaudio{./audio/024_past_be_06.mp3}}

\begin{tikzpicture}[remember picture,overlay]
 \visible<2->{\draw[->,opacity=0.6,line width=2pt] ([xshift=8pt,yshift=-2pt]pic cs:were) to[bend right] node[sloped,above,text=black,font=\tiny,pos=.6]{前に} ([xshift=10pt, yshift=8pt] pic cs:q_were);}
 \visible<3->{\draw[->,opacity=0.6,line width=2pt] ([xshift=8pt,yshift=-2pt]pic cs:was) to[bend right] node[sloped,above,text=black,font=\tiny,pos=.6]{前に} ([xshift=10pt, yshift=8pt] pic cs:q_was);}
\end{tikzpicture}
\end{frame}
%%%%%%%%%%%%%%%%%%%%%%
\begin{frame}[plain]{記号}
 
       \begin{description}
	\item[主語:] S\hfill{subjectの頭文字} 
	\item[動詞:] V\hfill{verbの頭文字}
%	\item[目的語:] O \hfill{objectの頭文字}
       \end{description}
\end{frame}
%%%%%%%%%%%%%%%%%%%%%%%
\begin{frame}[plain]{Exercises}

{\small あたえられた日本語の意味になるよう、カッコ内の語句を並べ替えましょう。なお、先頭の語は大文字で始めてください}
 \begin{enumerate}
    \item {\small あなたは昨日お腹が空いていましたか?}
( you / were / hungry ) yesterday?\\
\visible<2->{Were you hungry yesterday?}
    \item {\small 彼らは昨晩忙しかったですか?}
( they / busy / were ) last night?\\
\visible<3->{Were they busy last night?}
    \item {\small 彼女は昨日疲れていましたか?}
( she / tired / was ) yesterday?\\
\visible<4->{Was she tired yesterday?}
    \item {\small あなたは当時家にいましたか?}
( you / at / home / were ) then?\\
\visible<5->{Were you at home then?}

 \end{enumerate}
\hfill{\tiny 0157}\,{\scriptsize \myaudio{./audio/024_past_be_07.mp3}}

\end{frame}

%%%%%%%%%%%%%%%%%%%%%
\subsection{疑問文への答え方}
\begin{frame}<1-16>[plain]\frametitle{疑問文への答え方}

{\small 例にならい、質問に対する答えを「はい」と「いいえ」の2通りつくりましょう}
\mbox{}\hfill{\tiny 0339}\,{\scriptsize \myaudio{./audio/024_past_be_08z.mp3}}

\begin{tabular}{rlcll}
\visible<1->{例}& \visible<1->{Were you busy?}& \visible<2->{$\rightarrow$}&\visible<2->{(1) Yes, I was.}&\visible<2->{(2) No, I wasn't.}\\
\visible<1->{1}&\visible<1->{Was she tired last night?}&\visible<5->{$\rightarrow$}&\visible<6->{(1) Yes, she was.}&\visible<7->{(2) No, she wasn't.}\\
\visible<1->{2}&\visible<1->{Were they at home then?}&\visible<8->{$\rightarrow$}& \visible<9->{(1) Yes, they were.}&\visible<10->{(2) No, they weren't.}\\
\visible<1->{3}&\visible<1->{Was Peter in Japan then?}&\visible<11->{$\rightarrow$}&\visible<12->{(1) Yes, he was.}&\visible<13->{(2) No, he wasn't.}\\
\visible<1->{4}&\visible<1->{Was it rainy yesterday?}&\visible<14->{$\rightarrow$}&\visible<15->{(1) Yes, it was.}&\visible<16->{(2) No, it wasn't.}\\
&\multicolumn{1}{r}{{\scriptsize rainy\,\textipa{/r\'eIni/}\,雨の}}
\end{tabular}

\begin{block}<3->{疑問文への答え方}
\small
\begin{itemize}\setbeamertemplate{items}[square]
 \item Yes, S $+ \left\{\begin{array}{l}
		  \text{was}\\
		\text{were}\end{array}\right\}$\,\,.
\hspace{40pt}
No, S $+ \left\{\begin{array}{l}
		  \text{was not($=$ wasn't)}\\
		\text{were not($=$ weren't)}\end{array}\right\}$\,\,.
\end{itemize}

\hfill{\scriptsize Sは主語(subject)という意味の記号です}
      \end{block}
\end{frame}
%%%%%%%%%%%%%%%%%%%%%%%%%%%%%%%%%%%%
\section{まとめ}
\begin{frame}[plain]{この単元で学んだこと1}\small

 \begin{block}{be動詞の過去形の使い分け}
{
\setcounter{equation}{0}
\begin{numcases}{\text{主語できまります\,\,\,\,}}
 \text{\mbox{}\,\,{}I}&$\longrightarrow$\,\,\,\,\,\,{}\text{was}\\
 \text{\mbox{}\,\,{}you}&$\longrightarrow$\,\,\,\,\,\,{}\text{were}\\
 \text{\mbox{}\,\,{}単数(1人、1つ)}&$\longrightarrow$\,\,\,\,\,\,{}\text{was}\\
 \text{\mbox{}\,\,{}複数(2人以上、2つ)}&$\longrightarrow$\,\,\,\,\,\,{}\text{were}
\end{numcases}
}
\end{block}

\pause

現在形と過去形の使い分けを1つの表にまとめてみます

\hfill\begin{tblr}{
  colspec={lll},
 row{odd} = {bg=azure8},
  row{1}={font=\bfseries, bg=azure3, fg=white},
  hline{Z} = {0.08em},    % \toprule, \midrule, \bottomrule
%  hline{3} = {0.5pt}       % もう1つの \midrule
baseline=t
}
主語 & 現在形 & 過去形 \\
I & am & was \\
you & are & were \\
単数(1人、1つ) & is & was \\
複数(2人、2つ以上) & are & were \\
\end{tblr}


\end{frame}
%%%%%%%%%%%%%%%%%%%%%%%%%
\begin{frame}[plain]{この単元で学んだこと2}

\begin{block}<1->{否定文のつくり方\hspace{80pt}not \textipa{/n\'At/}}
\small
\setbeamertemplate{items}[square]
\begin{itemize}
 \item 否定文にするには、be動詞の直後に\,\,\Circled[fill color = white]{\,\,not\,\,}\,\,を置きます\hfill{\scriptsize 現在形のときと同じです}\\
\hfill{}I \textbf{was not} tired.  / You \textbf{were not} hungry.
 \item 縮めて was not $\rightarrow$\,\,\,\textcolor{Maroon}{\bfseries wasn't}\,\,\,\,\,\,/\,\,\,\,\,\,were not $\rightarrow$ \textcolor{Maroon}{\bfseries weren't}%
\hfill{\scriptsize \kenten{短縮形}といいます}\\
\hfill{}I \textbf{wasn't} busy. / They \textbf{weren't} at home.
\end{itemize}
      \end{block}


\begin{block}<2->{疑問文のつくり方・疑問文への答え方}
\small
\begin{itemize}\setbeamertemplate{items}[square]
 \item 疑問文にするには、be動詞を先頭に置きます\hfill{\scriptsize 現在形のときと同じです}\\
\hfill{}{\bfseries Were} you hungry?
 \item 疑問文への答え方
%KY\setbeamertemplate{items}[square]
\begin{itemize}\setbeamertemplate{items}[circle]
 \item Yes, S $+ \left\{\begin{array}{l}
		  \text{was}\\
		\text{were}\end{array}\right\}$\,\,.
\hspace{20pt}
No, S $+ \left\{\begin{array}{l}
		  \text{was not($=$ wasn't)}\\
		\text{were not($=$ weren't)}\end{array}\right\}$\,\,.
\end{itemize}
\end{itemize}
      \end{block}
\end{frame}

\end{document}
