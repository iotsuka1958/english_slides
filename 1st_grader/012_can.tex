\documentclass[aspectratio=169,xcolor={dvipsnames,table}]{beamer}
\usepackage[no-math,deluxe,haranoaji]{luatexja-preset}
\renewcommand{\kanjifamilydefault}{\gtdefault}
\renewcommand{\emph}[1]{{\upshape\bfseries #1}}
\usetheme{metropolis}
\metroset{block=fill}
\setbeamertemplate{navigation symbols}{}
\setbeamertemplate{blocks}[rounded][shadow=false]
\usecolortheme[rgb={0.7,0.2,0.2}]{structure}
%%%%%%%%%%%%%%%%%%%%%%%%%%%
\usepackage{media9}
%%%%%%%%%%%%%%%%%%%%%%%%%%%
%% さまざまなアイコン
%%%%%%%%%%%%%%%%%%%%%%%%%%%
\usepackage{fontawesome}
%\usepackage{figchild}
\usepackage{twemojis}
\usepackage{utfsym}
\usepackage{bclogo}
\usepackage{marvosym}
\usepackage{fontmfizz}
\usepackage{pifont}
\usepackage{phaistos}
\usepackage{worldflags}
\usepackage{manfnt}
%%%%%%%%%%%%%%%%%%%%%%%%%%%
\usepackage{tikz}
\usetikzlibrary{backgrounds}
\usepackage{tcolorbox}
\usepackage{tikzpeople}
\usepackage{tikzducks}
\usepackage{circledsteps}
\usepackage{xcolor}
\usepackage{amsmath}
\usepackage{pxrubrica}
\usepackage{tipa}
%%%%%%%%%%%%%%%%%%%%%%%%%%
\usepackage[absolute,overlay]{textpos}  %% 任意の位置に図を配置
%\usepackage[grid=true,gridcolor=Maroon,subgridcolor=gray,gridunit=pt,texcoord]{eso-pic} %%場所決めのための格子
%%%%%%%%%%%%%%%%%%%%%%%%%%%
%%%%%%%%%%%%%%%%%%%%%%%%%%%
%% 場合分け
\usepackage{cases}
%%%%%%%%%%%%%%%%%%%%%%%%%%%
% \myAnch{<名前>}{<色>}{<テキスト>}
% 指定のテキストを指定の色の四角枠で囲み, 指定の名前をもつTikZの
% ノードとして出力する. 図には remeber picture 属性を付けている
% ので外部から参照可能である.
\newcommand*{\myAnch}[3]{%
  \tikz[remember picture,baseline=(#1.base)]
    \node[draw,rectangle,#2] (#1) {\normalcolor #3};
}
%%%%%%%%%%%%%%%%%%%%%%%%%%%%
%% 音声リンク表示
\newcommand{\myaudio}[1]{\href{#1}{\faVolumeUp}}
%%%%%%%%%%%%%%%%%%%%%%%%%%%
% \myEmph コマンドの定義
%\newcommand{\myEmph}[3]{%
%    \textbf<#1>{\color<#1>{#2}{#3}}%
%}
\usepackage{xparse} % xparseパッケージの読み込み
\NewDocumentCommand{\myEmph}{O{} m m}{%
    \def\argOne{#1}%
    \ifx\argOne\empty
        \textbf{\color{#2}{#3}}% オプション引数が省略された場合
    \else
        \textbf<#1>{\color<#1>{#2}{#3}}% オプション引数が指定された場合
    \fi
}
%%%%%%%%%%%%%%%%%%%%%%%%%%%
%% 文末の上昇イントネーション記号 \myRisingPitch
%% 通常のイントネーション \myDownwardPitch
%% https://note.com/dan_oyama/n/n8be58e8797b2
%%%%%%%%%%%%%%%%%%%%%%%%%%%
\newcommand{\myRisingPitch}{
\begin{tikzpicture}[scale=0.3,baseline=0.3]
\draw[->,>=stealth] (0,0) to[bend right=45] (1,1);
\end{tikzpicture}
}
\newcommand{\myDownwardPitch}{
\begin{tikzpicture}[scale=0.3,baseline=0.3]
\draw[->,>=stealth] (0,1) to[bend left=45] (1,0);
\end{tikzpicture}
}
%%%%%%%%%%%%%%%%%%%%%%%%%%%
\title{English is fun.}
\subtitle{I can play the guitar. She can sing well.}
\author{}
\institute[]{}
\date[]

%%%%%%%%%%%%%%%%%%%%%%%%%%%%
%% TEXT
%%%%%%%%%%%%%%%%%%%%%%%%%%%%
\begin{document}
\begin{frame}[plain]
  \titlepage
\end{frame}

\section*{授業の流れ}
\begin{frame}[plain]
  \frametitle{授業の流れ}
  \tableofcontents
\end{frame}
%%%%%%%%%%%%%%%%%%%%%%%%%%%%%%
\section{The Pronunciation of Can}
%%%%%%%%%%%%%%%%%%%%%%%%%%%%%%
\begin{frame}[plain]{canの発音}
\LARGE
can

\hfill\visible<3->{\textipa{/k@n/}}\hfill\visible<2->{\textipa{/k\'\ae n/}}\hfill\mbox{}

\vspace{90pt}

\begin{block}<5->{Topic for Today}\small
 \begin{itemize}\setbeamertemplate{items}[square]\small
  \item \textbf{can}には2とおりの発音があります%
\hfill{\scriptsize 使い分けが気になりますね}

 \end{itemize}
\end{block}

% この場合は (363pt, 100pt) の位置に 0.4\linewidth の幅のブロックができる.
\begin{textblock*}{0.4\linewidth}(363pt,100pt)
    % TiKZを使った図形の描画
    \visible<4->{\begin{tikzpicture}
        \duck[laughing,bowtie,
strawhat=brown!50!white,
ribbon=black,
think={\scriptsize 2つ!},
bubblecolour=white!50!pink,
scale=1.1414]
    \end{tikzpicture}}
\end{textblock*}

\end{frame}
%%%%%%%%%%%%%%%%%%%%%%%%%%%%%%
\subsection{〜できる}
%%%%%%%%%%%%%%%%%%%%%%%%%%%%%
\begin{frame}[plain]\frametitle{〜できる}
 \Large

\begin{enumerate}
 \item<2-> I play the guitar every day. \hfill{\scriptsize every day: 毎日}
 \item<3-> I \textcolor{Maroon}{\bfseries can} {\bfseries play} the guitar.
\end{enumerate}

\vspace{60pt}

\begin{block}<4->{Topics for Today}
\begin{itemize}\setbeamertemplate{items}[square]\small
 \item<4->   {\bfseries can} $+$ 原形$\longrightarrow${\,\,\,}「〜することができる」
% \item \textipa{/k@n/ /k\'\ae n/}
 \item<6-> \<「~できる」というときは\textipa{/k@n/}が普通です
 \item<7-> \textbf{can}のように動詞と組み合わせて動詞の意味を補足する語を\kenten{助動詞}といいます
 \end{itemize}
     \end{block}

\vspace*{-20pt}

\mbox{}\hfill{\tiny 0102}\,{\scriptsize \myaudio{./audio/012_can_01.mp3}} 

% この場合は (363pt, 100pt) の位置に 0.4\linewidth の幅のブロックができる.
\begin{textblock*}{0.4\linewidth}(363pt,62.5pt)
    % TiKZを使った図形の描画
    \visible<5->{\begin{tikzpicture}
        \duck[laughing,bowtie,
strawhat=brown!50!white,
ribbon=black,
think={\scriptsize どっち?},
bubblecolour=white!50!pink,
scale=1.1414]
    \end{tikzpicture}}
\end{textblock*}

\end{frame}
%%%%%%%%%%%%%%%%%%%%%%%%%%%%
\begin{frame}[plain]\frametitle{助動詞can}

\begin{enumerate}
 \item I \textcolor{Maroon}{\bfseries can} {\bfseries play} the guitar.
 \item We \textcolor{Maroon}{\bfseries can} {\bfseries speak} English.
 \item You \textcolor{Maroon}{\bfseries can} {\bfseries sing} very well.
 \item He \textcolor{Maroon}{\bfseries can} {\bfseries swim}.
 \item She \textcolor{Maroon}{\bfseries can} {\bfseries read} French.
 \item They \textcolor{Maroon}{\bfseries can} {\bfseries dance} very well.
\end{enumerate}

\vfill

\begin{block}<3->{Topics for Today}
\begin{itemize}\setbeamertemplate{items}[square]\small
% \item<2->   {\bfseries can} $+$ 原形$\longrightarrow${\,\,\,}「〜することができる」
 \item<3->   主語がなんであっても\textbf{can} $+$ 原形\\
%\hfill\visible<3->{\scriptsize canのように動詞と組み合わせて動詞の意味を補足する語を\kenten{助動詞}といいます}
% \item<4-> \<「~できる」というときは\,\textipa{/k@n/}\,が普通です
 \end{itemize}
\end{block}

\vspace*{-10pt}

\hfill{\tiny 0240}\,{\scriptsize \myaudio{./audio/012_can_02.mp3}}

\begin{textblock*}{0.4\linewidth}(300pt,62.5pt)
    % TiKZを使った図形の描画
    \visible<5->{\begin{tikzpicture}
  \duck[crazyhair=brown!60!black, glasses, eyebrow,
    signpost={\small\textipa{/k@n/}}, speech={\scriptsize できる!}, laughing,
    jacket=Maroon!30, lapel, buttons, water=NavyBlue]
    \end{tikzpicture}}
\end{textblock*}
\end{frame}
%%%%%%%%%%%%%%%%%%%%%%%%%%%%%%%%
\begin{frame}[plain,t]\frametitle{Exercises}

{\small 次の英文の(~~~~~~~~)内から動詞の正しい形を選びましょう}\\%
\hfill{\tiny 0231}\,{\scriptsize \myaudio{./audio/012_can_03.mp3}}


\begin{enumerate}
 \item John ( play / \alt<2->{\Circled[outer color=orange]{plays}}{plays} / playing ) baseball.
 \item Jane can ( plays / \alt<3->{\Circled[outer color=orange]{play}}{play} / playing ) the guitar.\hfill{\scriptsize guitar \textipa {/gIt\'A\textrhookschwa /} ギター}
 \item You can ( dancing / \alt<4->{\Circled[outer color=orange]{sing}}{sing} / sings  ) very well.\hfill{}{\scriptsize dance \textipa{/d\'\ae ns/} 踊る}
 \item They can ( \alt<5->{\Circled[outer color=orange]{swim}}{swim} / dances / singing ).
 \item We can( reads / writes /\alt<6->{\Circled[outer color=orange]{speak}}{speak} ) Japanese.
\end{enumerate}

\begin{textblock*}{0.4\linewidth}(350pt,150pt)
    % TiKZを使った図形の描画
    \visible<7->{\begin{tikzpicture}
        \duck[laughing,bowtie,
strawhat=brown!50!white,
ribbon=black,
think={\scriptsize \textbf{can} $+$ 原形},
bubblecolour=white!50!pink,
scale=1.1414]
    \end{tikzpicture}}
\end{textblock*}
\end{frame}
%%%%%%%%%%%%%%%%%%%%%%%%%%%%%%%
\section{canの否定}
%%%%%%%%%%%%%%%%%%%%%%%%%%
\begin{frame}[plain]{否定を表すnot}
 \Large

否定を表すことば: {\LARGE\bfseries not}\hspace{20pt}\textipa{/n\'At/}
\end{frame}
%%%%%%%%%%%%%%%%%%%%%%%%%%%%
\begin{frame}[plain]{〜できない}
  \Large

\begin{enumerate}
 \item<1-> I {\bfseries can} use a computer. \hfill{\scriptsize use \textipa{/j\'u:z/} 使う}\hfill\hfill{\scriptsize \textipa{/k@n/}}
% computer \textipa{/k@mpj\'u:t\textrhookschwa /} コンピューター}
 \item<3-> I \textcolor{Maroon}{\bfseries cannot} use a computer.\hfill{\scriptsize \textipa{/k\'\ae nAt/}}
 \item<5-> I \textcolor{Maroon}{\bfseries can't} use a computer.\hfill{\scriptsize \textipa{/k\'\ae nt/}}
\end{enumerate}

\vfill

\begin{block}<1->{Topics for Today}
\begin{itemize}\setbeamertemplate{items}[square]\small
 \item<1->  「〜できる」$\longrightarrow$\,\,\, {\bfseries can} $+$ 原形\hfill{\scriptsize \textipa{/k@n/}}
 \item<6->   「〜することができない」$\longrightarrow${\,\,\,}$\left\{\begin{tabular}{l}\text{\bfseries cannot}\\\text{\bfseries can't}\end{tabular}\right\} + \text{原形}$\hfill{\scriptsize \textipa{/k\'\ae nAt/} \textipa{/k\'\ae nt/}}
 \end{itemize}
     \end{block}

\hfill\visible<4->{{\scriptsize 2語のcan notはあまり使われません}}

\hfill{\tiny 0140}\,{\scriptsize \myaudio{./audio/012_can_04.mp3}}

\begin{textblock*}{0.4\linewidth}(210pt,190pt)
    % TiKZを使った図形の描画
    \visible<4->{\begin{tikzpicture}
        \duck[laughing,bowtie,glasses,
strawhat=brown!50!white,
ribbon=black,
think={\scriptsize cannot!},
speech={\scriptsize 1語で},
bubblecolour=white!50!pink,
scale=.75]
    \end{tikzpicture}}
\end{textblock*}
\end{frame}
%%%%%%%%%%%%%%%%%%%%%%%%%%%%
\begin{frame}[plain]\frametitle{Exercises}
 意味を考えましょう

\begin{enumerate}
 \item I {\bfseries can} skate very well, but I {\bfseries cannot} ski.\hfill{\scriptsize skate \textipa{/sk\'eIt/} スケートをする}
 \item Susan {\bfseries can} drive a car, but she {\bfseries can't} ride a bicycle.\hfill{\scriptsize ride \textipa{/r\'aId/} (自転車などに)乗る}
 \item We {\bfseries can} speak French, but we {\bfseries cannot} speak Japanese.
 \item They {\bfseries can} sing very well, but they {\bfseries can't} play the piano.
\end{enumerate}
\vfill


\begin{block}<2->{Topics for Today}
\begin{itemize}\setbeamertemplate{items}[square]\small
 \item  「〜できる」$\longrightarrow$\,\,\, {\bfseries can} $+$ 原形\hfill{\scriptsize \textipa{/k@n/}}
 \item 「〜できない」$\longrightarrow${\,\,\,}$\left\{\begin{tabular}{l}\text{\bfseries cannot}\\\text{\bfseries can't}\end{tabular}\right\} + \text{原形}$\hfill{\scriptsize \textipa{/k\'\ae nAt/} \textipa{/k\'\ae nt/}}
% \item 主語がなんであっても同じ
 \end{itemize}
     \end{block}%
\hfill\visible<2->{{\scriptsize \dbend\,\dbend\,\,\,can notはあまり使われません}}

\hfill{\tiny 0248}\,{\scriptsize \myaudio{./audio/012_can_05.mp3}}
\end{frame}
%%%%%%%%%%%%%%%%%%%%%%%%%%
\begin{frame}[plain]\frametitle{Exercises}

日本文を参考にして(~~~~~~~~)の語を並べかえ、英文を完成させましょう

\begin{enumerate}

 \item You ( can / the piano / play ) very well.
\hspace{2\zw}\begin{minipage}[t]{15\zw}{\scriptsize あなたはとてもじょうずにピアノを弾くことができます。}\end{minipage}
\\
\visible<2->{$\longrightarrow$\,\,\,You can play the piano very well.}
 \item  ( run / Naomi / can ) fast.\hspace{8.75\zw}{\scriptsize ナオミは速く走ることができます。}\\
\visible<3->{$\longrightarrow$\,\,\,Naomi can run fast.}
 \item ( French / can't / I / speak ).\hspace{7.5\zw}%
{\scriptsize わたしはフランス語を話せません。}\\
\visible<4->{$\longrightarrow$\,\,\,I can't speak French.}
 \item He ( a bicycle / ride / cannot ).\hspace{6.2\zw}{\scriptsize 彼は自転車に乗れません。}\\
\visible<5->{$\longrightarrow$\,\,\,He cannot ride a bicycle.}
\end{enumerate}

\hfill{\tiny 0207}\,{\scriptsize \myaudio{./audio/012_can_06.mp3}}
\end{frame}
%%%%%%%%%%%%%%%%%%%%%%%%%%%%%%
\section{canの疑問文}
%%%%%%%%%%%%%%%%%%%%%%%%%%%%%
\subsection{canの疑問文のつくり方}
%%%%%%%%%%%%%%%%%%%%%%%%%%%%%
\begin{frame}[plain]{〜できますか?}
 \Large

\myAnch{s-1}{olive}{You} \myAnch{aux-1}{orange}{can} play the guitar.\visible<4->{\raisebox{-10pt}{\scalebox{2}{\myDownwardPitch}}}\hfill\visible<5->{{\scriptsize \textipa{/k@n/}}}

\vspace{30pt}

\visible<2->{\myAnch{aux-2}{orange}{\textbf{Can}} \myAnch{s-2}{olive}{you} play the guitar\myAnch{question}{orange}{?}}\visible<4->{\scalebox{2}{\myRisingPitch}}

\hspace{190pt}\visible<4-> {\scriptsize ?と最後のイントネーションに注意}

\begin{tikzpicture}[remember picture, overlay]
 \visible<3->{\draw[line width=2pt,opacity=.5, olive, ->] (s-1.south) to[out=-90, in=90] (s-2.north);} 
 \visible<3->{\draw[line width=2pt, opacity=.5, orange, ->] (aux-1.south) to[out=-90, in=90] node[sloped,above,text=black,font=\tiny,pos=.55]{Canを先頭へ} (aux-2.north);}
\end{tikzpicture}

\vfill

\begin{block}<7->{Topic for Today}
\begin{itemize}\setbeamertemplate{items}[square]\small
 \item   「〜することができますか?」\\
\hspace{100pt}$\longrightarrow${\,\,\,}$\text{\bfseries Can} + \text{S} + \text{原形}$ \ldots{}\,\,? \textipa{/k\'\ae n/}
 \end{itemize}
     \end{block}

\vspace*{-8pt}

\hfill{\tiny 0110}\,{\scriptsize \myaudio{./audio/012_can_07.mp3}}

\begin{textblock*}{0.4\linewidth}(350pt,85pt)
    % TiKZを使った図形の描画
    \visible<6->{\begin{tikzpicture}
        \duck[laughing,bowtie,
strawhat=brown!50!white,
ribbon=black,
think={\tiny \begin{tabular}{l}
	      疑問文は\\\textipa{/k\'\ae n/}
	     \end{tabular}},
bubblecolour=white!50!pink,
scale=1.1414]
    \end{tikzpicture}}
\end{textblock*}
\end{frame}
%%%%%%%%%%%%%%%%%%%%%%%%%
\begin{frame}[plain]\frametitle{Exercises}

{\small 日本文を参考にして(~~~~~~~~)の語を並べかえ、英文を完成させましょう。
ただし先頭に来る単語は大文字で始めてください}

\begin{enumerate}
 \item  ( can / this English song / sing / you ) ?\hspace{2\zw}%
\begin{minipage}{15\zw}あなたはこの英語の歌を歌うことができますか。\raisebox{-5pt}{\rotatebox{0}{\scalebox{2.5}{\twemoji{musical notes}}}}\end{minipage}
\\
\visible<2->{$\longrightarrow$\,\,\,Can you sing this English song?}
 \item ( you / cook / can / pizza ) ?
\hspace{5.3\zw}あなたはピザをつくれますか。\raisebox{5pt}{\rotatebox{-45}{\scalebox{2.5}{\twemoji{pizza}}}}\\
\visible<3->{$\longrightarrow$\,\,\,Can you cook pizza?}%
\hfill{\scriptsize pizza \textipa{/p\'\i:ts@/}}
 \item ( a car / John / drive / can ) ?\hspace{4.8\zw}%
ジョンは車の運転ができますか。\raisebox{0pt}{\rotatebox{0}{\scalebox{2.5}{\twemoji{oncoming automobile}}}}\\
\visible<4->{$\longrightarrow$\,\,\,Can John drive a car?}%
\hfill{\scriptsize drive \textipa{/dr\'aIv/} (車など)を運転する}
\end{enumerate}

\hfill{\tiny 0139}\,{\scriptsize \myaudio{./audio/012_can_08.mp3}}
\end{frame}
%%%%%%%%%%%%%%%%%%%%%%%%%%
\subsection{canの疑問文への答え方}
\begin{frame}[plain]{Can you 〜? と聞かれたら}
 \Large

\visible<1->{Can you cook pizza?\hfill{\small \textipa{/k\'\ae n/}}}

\vspace{20pt}

\mbox{}\hspace{125pt}%
\visible<2->{$\left\{\begin{tabular}{ll}
         \visible<2->{\text{Yes, I can.}}&\visible<2->{\text{\small {\textipa{/k\'\ae n/}}}}\\
         \visible<3->{\text{No, I cannot.}}&\visible<3->{\text{\small {\textipa{/k\'\ae nAt/}}}}\\
         \visible<4->{\text{(}$=$ \text{No, I can't.)}}&\visible<4->{\text{\small {\textipa{/k\'\ae nt/}}}}
        \end{tabular}\right.$}

\hfill{\tiny 0142}\,{\scriptsize \myaudio{./audio/012_can_09.mp3}}

\begin{textblock*}{0.4\linewidth}(40pt,150pt)
    % TiKZを使った図形の描画
    \visible<5->{\begin{tikzpicture}
        \duck[laughing,bowtie,
strawhat=brown!50!white,
ribbon=black,
think={\tiny \begin{tabular}{l}
	      答えるときも\\\textipa{/\ae /}
	     \end{tabular}},
bubblecolour=white!50!pink,
scale=1.1414]
    \end{tikzpicture}}
\end{textblock*}
\end{frame}
%%%%%%%%%%%%%%%%%%%%%%%%%%%%%%%%%%
\begin{frame}[plain]{Exercises}
例にならって、つぎの質問に対する答えを「はい」と「いいえ」の2通りつくりましょう

\begin{tabular}{rlcll}
\visible<1->{例}& \visible<1->{Can you skate?}\hspace{2\zw}\scalebox{2}{\twemoji{ice skate}}& \visible<2->{$\rightarrow$}&\visible<3->{(1) Yes, I can.}&\visible<4->{(2) No, I can't.($= \text{cannot}$)}\\
\visible<1->{1}&\visible<1->{Can you read English?\hspace{10pt}\raisebox{0pt}{\bcbook}}&\visible<5->{$\rightarrow$}&\visible<6->{(1) Yes, I can.}&\visible<7->{(2) No, I can't.}\\
\visible<1->{2}&\visible<1->{Can they speak French?}&\visible<8->{$\rightarrow$}& \visible<9->{(1) Yes, they can.}&\visible<
10->{(2) No, they can't.}\\
\visible<1->{3}&\visible<1->{Can David and Bob sing well?}&\visible<11->{$\rightarrow$}&\visible<12->{(1) Yes, they can.}&\visible<13->{(2) No, they can't.}\\
\visible<1->{4}&\visible<1->{Can she drive a car?\hspace{10pt}\raisebox{-5pt}{\scalebox{2.5}{\twemoji{automobile}}}}&\visible<14->{$\rightarrow$}&\visible<15->{(1) Yes, she can.}&\visible<16->{(2) No, she can't.}
\end{tabular}

\vfill

\hfill{\tiny 0528}\,{\scriptsize \myaudio{./audio/012_can_10.mp3}}

\end{frame}
%%%%%%%%%%%%%%%%%%%%%%%%%%%%%%%%%%%%%%
\section{助動詞canのまとめ}
\begin{frame}[plain]{要点}
 
\begin{block}{助動詞can \textipa{/k@n/} \textipa{/k\'\ae n/}}
\begin{itemize}\setbeamertemplate{items}[square]\small
 \item<1-> S $+$ {\bfseries can} $+$ 原形 {\scriptsize (~することができる)}%
\hfill{}I {\bfseries can} cook. \textipa{/k@n/}

 \item<2-> 否定文のつくり方 $\text{S}+\left\{\begin{tabular}{l}\text{\bfseries cannot}\\\text{\bfseries can't}\end{tabular}\right\} + \text{原形}$\hfill{ }%
\hfill{}{\normalsize He {\bfseries cannot} play the guitar.}\textipa{/k\'\ae nAt/}\\[-5pt]
\hfill{}{\normalsize He {\bfseries can't} play the guitar.}\textipa{/k\'\ae nt/}\\
\hfill{\scriptsize \dbend\,\dbend\,\,\,can notはあまり使われません}\\[15pt]

 \item<3-> 疑問文のつくり方 {\bfseries Can} $+$ S $+$ 原形 \ldots\,\,?
\hfill{}{\normalsize {\bfseries Can} she sing well?\textipa{/k\'\ae n/}}
 \item<4-> 疑問文への答え方%
\hfill{}Yes, she {\bfseries can}. \textipa{/k\'\ae n/}\\
\hfill{}No, she {\bfseries cannot}.\textipa{/k\'\ae nAt/}\\
\hfill{}($=$ No, she {\bfseries can't}.)\textipa{/k\'\ae nt/}
\end{itemize}
\end{block}

\hfill{\tiny 0257}\,{\scriptsize \myaudio{./audio/012_can_11.mp3}}
\end{frame}
%%%%%%%%%%%%%%%%%%%%%%%%%%%%%%%%%%%%%%%
\section{Can I ~? / Can you ~?}
%%%%%%%%%%%%%%%%%%%%%%%%%%%%%%%%%%%%%%%
\begin{frame}[plain]{Can you ~?}
 
\begin{enumerate}
 \item<1-> \textbf{Can} you swim fast?
 \item<2-> \textbf{Can you} shut the door? \visible<4->{--- \textbf{Sure.}} 
\end{enumerate}

\begin{block}<3->{Topics for Today}
\begin{itemize}\setbeamertemplate{items}[square]\small
 \item<3->   \textbf{Can you ~?}が「~してくれますか」のように、相手になにか\kenten{依頼}するときにつかうことがあります
 \item<5-> 答えるときは
      $\left\{ \begin{tabular}{l}
		 \textbf{Sure.}\\
		 \textbf{Certainly.}\\
		 \textbf{All right.}\\
		 \textbf{Of course.}
		\end{tabular}\right\}$
などと答えます
 \end{itemize}%
     \end{block}
\hfill{\tiny 0142}\,{\scriptsize \myaudio{./audio/012_can_12.mp3}}

\end{frame}
%%%%%%%%%%%%%%%%%%%%%%%%%%%%%%%%%%%%%%
%%%%%%%%%%%%%%%%%%%%%%%%%%%%%%%%%%%%%%%
\begin{frame}[plain]{Can I ~?}
 Can I ~? は文字どおりには「わたしは~できますか」

\begin{enumerate}
 \item<2-> \textbf{Can I} use your pen? \visible<4->{--- \textbf{Sure.}} 
\end{enumerate}

\begin{block}<3->{Topics for Today}
\begin{itemize}\setbeamertemplate{items}[square]\small
 \item<3->   \textbf{Can I ~?}が「~していいですか」のように、相手の\kenten{許可}をもとめるときにつかうことがあります
 \item<5-> 答えるときは
      $\left\{ \begin{tabular}{l}
		 \textbf{Sure.}\\
		 \textbf{Certainly.}\\
		 \textbf{All right.}\\
		 \textbf{Of course.}
		\end{tabular}\right\}$
などと答えます
 \end{itemize}%
     \end{block}
\hfill{\tiny 0257}\,{\scriptsize \myaudio{./audio/012_can_13.mp3}}

\end{frame}
%%%%%%%%%%%%%%%%%%%%%%%%%%%%%%%%%%%%%%
\begin{frame}[plain]
 
\hfill{\tiny audio\_overview 1003}\,{\scriptsize \myaudio{./audio/overview/012_can_audio_overview.m4a}}
\end{frame}
%%%%%%%%%%%%%%%%%%%%%%%%%%%%%%%%%%%%%%%
\end{document}

