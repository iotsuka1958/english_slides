\documentclass[aspectratio=169]{beamer}
\usepackage[no-math,deluxe,haranoaji]{luatexja-preset}
\renewcommand{\kanjifamilydefault}{\gtdefault}
\renewcommand{\emph}[1]{{\upshape\bfseries #1}}
\usetheme{metropolis}
\metroset{block=fill}
\setbeamertemplate{navigation symbols}{}
\usecolortheme[rgb={0.7,0.2,0.2}]{structure}
%%%%%%%%%%%%%%%%%%%%%%%%%%%
\usepackage{media9}
%%%%%%%%%%%%%%%%%%%%%%%%%%%
%% さまざまなアイコン
%%%%%%%%%%%%%%%%%%%%%%%%%%%
\usepackage{fontawesome}
\usepackage{figchild}
\usepackage{twemojis}
\usepackage{utfsym}
\usepackage{bclogo}
\usepackage{marvosym}
\usepackage{fontmfizz}
\usepackage{pifont}
\usepackage{phaistos}
\usepackage{worldflags}
%%%%%%%%%%%%%%%%%%%%%%%%%%%
\usepackage{tikz}
\usetikzlibrary{backgrounds}
\usepackage{tcolorbox}
\usepackage{tikzpeople}
\usepackage{circledsteps}
\usepackage{xcolor}
\usepackage{amsmath}
%%%%%%%%%%%%%%%%%%%%%%%%%%%
%% 場合分け
\usepackage{cases}
%%%%%%%%%%%%%%%%%%%%%%%%%%%
% \myAnch{<名前>}{<色>}{<テキスト>}
% 指定のテキストを指定の色の四角枠で囲み, 指定の名前をもつTikZの
% ノードとして出力する. 図には remeber picture 属性を付けている
% ので外部から参照可能である.
\newcommand*{\myAnch}[3]{%
  \tikz[remember picture,baseline=(#1.base)]
    \node[draw,rectangle,#2] (#1) {\normalcolor #3};
}
%%%%%%%%%%%%%%%%%%%%%%%%%%%%
%% 音声リンク表示
\newcommand{\myaudio}[1]{\href{#1}{\faVolumeUp}}
%%%%%%%%%%%%%%%%%%%%%%%%%%%
% \myEmph コマンドの定義
%\newcommand{\myEmph}[3]{%
%    \textbf<#1>{\color<#1>{#2}{#3}}%
%}
\usepackage{xparse} % xparseパッケージの読み込み
\NewDocumentCommand{\myEmph}{O{} m m}{%
    \def\argOne{#1}%
    \ifx\argOne\empty
        \textbf{\color{#2}{#3}}% オプション引数が省略された場合
    \else
        \textbf<#1>{\color<#1>{#2}{#3}}% オプション引数が指定された場合
    \fi
}
%%%%%%%%%%%%%%%%%%%%%%%%%%%
%% 文末の上昇イントネーション記号 \myRisingPitch
%% 通常のイントネーション \myDownwardPitch
%% https://note.com/dan_oyama/n/n8be58e8797b2
%%%%%%%%%%%%%%%%%%%%%%%%%%%
\newcommand{\myRisingPitch}{
\begin{tikzpicture}[scale=0.3,baseline=0.3]
\draw[->,>=stealth] (0,0) to[bend right=45] (1,1);
\end{tikzpicture}
}
\newcommand{\myDownwardPitch}{
\begin{tikzpicture}[scale=0.3,baseline=0.3]
\draw[->,>=stealth] (0,1) to[bend left=45] (1,0);
\end{tikzpicture}
}
%%%%%%%%%%%%%%%%%%%%%%%%%%%
\title{English is fun.\,\,{}--- I can play the guitar. ---}
\author{}
\institute[]{}
\date[]

%%%%%%%%%%%%%%%%%%%%%%%%%%%%
%% TEXT
%%%%%%%%%%%%%%%%%%%%%%%%%%%%
\begin{document}
\begin{frame}[plain]
  \titlepage
\end{frame}

\section*{授業の流れ}
\begin{frame}[plain]
  \frametitle{授業の流れ}
  \tableofcontents
\end{frame}


\section{can}

\subsection{〜できる}
\begin{frame}[plain]\frametitle{〜できる}
 \Large

I play the guitar every day. \pause
\hspace{80pt}{\footnotesize every day: 毎日}

\pause

\vspace{15pt}

I \textcolor{orange}{can} play the guitar.

\pause

\vfill

\begin{exampleblock}{Topics for Today}
\pause
\begin{itemize}\small
 \item   「〜することができる」$\longrightarrow${\,\,\,}can $+$ 動詞
 \end{itemize}
     \end{exampleblock}

\pause

\mbox{}\hfill\visible{\myaudio{./audio/012_can_01.mp3}}

\end{frame}


\begin{frame}[plain]\frametitle{〜できる}

つぎの英文の意味を考えましょう。

\begin{enumerate}
 \item I can play the guitar.
 \item We can speak English.
 \item You can sing very well.
 \item He can swim.
 \item She can read French.
 \item They can dance very well.
\end{enumerate}

\pause

\vfill

\begin{exampleblock}{Topics for Today}
\pause
\begin{itemize}\small
 \item   主語がなんであってもcan $+$ 動詞
 \end{itemize}
     \end{exampleblock}

\pause

\mbox{}\hfill\visible{\myaudio{./audio/012_can_02.mp3}}
\end{frame}


\begin{frame}<1-7>[plain]\frametitle{Exercises}

次の英文の(~~~~~~~~)内から動詞の正しい形を選び、○で囲みましょう。

\begin{enumerate}
 \item John ( play / \alt<2->{\Circled[outer color=orange]{plays}}{plays} / playing ) baseball.
 \item Jane can ( \alt<3->{\Circled[outer color=orange]{play}}{play} / plays / playing ) the guitar.
 \item You can ( \alt<4->{\Circled[outer color=orange]{sing}}{sing} / sings / seeing ) very well.
 \item They can ( \alt<5->{\Circled[outer color=orange]{swim}}{swim} / swims / swimming ).
 \item We can( \alt<6->{\Circled[outer color=orange]{speak}}{speak} / speaks / reads ) Japanese.
\end{enumerate}
\pause

\vfill

\visible<7>{%
\mbox{}\hfill\visible{\myaudio{./audio/012_can_03.mp3}}
}

\end{frame}

\subsection{〜できない}
\begin{frame}[plain]{〜できない}
  \Large

I can use a computer. \pause
\hspace{80pt}{\footnotesize use: 使う computer: コンピューター}

\pause

\vspace{15pt}

I \textcolor{orange}{cannot} use a computer.\pause

I \textcolor{orange}{can't} use a computer.



\pause

\vfill

\begin{exampleblock}{Topics for Today}
\pause
\begin{itemize}\small
 \item   「〜することができない」$\longrightarrow${\,\,\,}$\left\{\begin{tabular}{l}\text{cannot}\\\text{can't}\end{tabular}\right\} + \text{動詞}$
 \end{itemize}
     \end{exampleblock}

\pause

\mbox{}\hfill{}\myaudio{./audio/012_can_04.mp3}
\end{frame}


\begin{frame}[plain]\frametitle{〜できる、できない}
 

\begin{enumerate}
 \item I can skate very well, but I cannot ski.
 \item Suzan can drive a car, but she can't ride a bicycle.
 \item We can speak French, but we cannot speak Japanese.
 \item They can sing very well, but they can't play the piano.
\end{enumerate}
\vfill

\pause

\begin{exampleblock}{Topics for Today}
\begin{itemize}\small
 \item  「〜できる」$\longrightarrow$\,\,\, can $+$ 動詞\hfill{}「〜できない」$\longrightarrow${\,\,\,}$\left\{\begin{tabular}{l}\text{cannot}\\\text{can't}\end{tabular}\right\} + \text{動詞}$ \pause
 \item 主語がなんであっても同じ
 \end{itemize}
     \end{exampleblock}

\pause

\mbox{}\hfill\visible<4>{\myaudio{./audio/012_can_05.mp3}}
\end{frame}



\begin{frame}<1-6>[plain]\frametitle{Exercises}

日本文を参考にして(~~~~~~~~)の語を並べかえ、英文を完成させましょう。

\begin{enumerate}

 \item You ( can / the piano / play ) very well.
\hspace{2\zw}\begin{minipage}{15\zw}あなたはとてもじょうずにピアノを弾くことができます。\end{minipage}
\\
\visible<2->{$\longrightarrow$\,\,\,You can play the piano very well.}
 \item  ( run / Naomi / can ) fast.\hspace{8.75\zw}ナオミは速く走ることができます。\\
\visible<3->{$\longrightarrow$\,\,\,Naomi can run fast.}
 \item ( French / can't / I / speak ).\hspace{7.5\zw}%
わたしはフランス語を話せません。\\
\visible<4->{$\longrightarrow$\,\,\,I can't speak French.}
 \item He ( a bicycle / ride / cannot ).\hspace{6.2\zw}彼は自転車に乗れません。\\
\visible<5->{$\longrightarrow$\,\,\,He cannot ride a bicycle.}
\end{enumerate}

\pause

\mbox{}\hfill\visible<6>{\myaudio{./audio/012_can_06.mp3}}
\end{frame}


\subsection{〜できますか}
\begin{frame}[plain,t]{〜できますか?}
 \Large

\pause


\myAnch{s-1}{orange}{You} \myAnch{aux-1}{olive}{can} play the guitar. \scalebox{2}{\myDownwardPitch}
\vspace{30pt}\pause

\myAnch{aux-2}{olive}{Can} \myAnch{s-2}{orange}{you} play the guitar\myAnch{question}{orange}{?}
\pause
\begin{tikzpicture}[remember picture, overlay]
 \draw[thick, orange, ->] (s-1.south) to[out=-90, in=90] (s-2.north); 
 \draw[thick, olive, ->] (aux-1.south) to[out=-90, in=90] (aux-2.north);
%\pause
%\visible<8->{\node at (0.5,0.25) {\scalebox{2}{\myRisingPitch}};}
\end{tikzpicture}

\vspace{-25pt}
\mbox{}\hspace{200pt}\visible<6->{\scalebox{2}{\myRisingPitch}}

\hspace{40pt}\visible<5->{\small 主語の前にCan}
\hspace{80pt}\visible<6-> {\small ?と最後のイントネーションに注意}

\vfill

\begin{exampleblock}<7->{Topics for Today}
\visible<8->{%
\begin{itemize}\small
 \item   「〜することができますか?」$\longrightarrow${\,\,\,}$\text{Can} + \text{主語} + \text{動詞}$ \ldots{}\,\,?
 \end{itemize}
}
     \end{exampleblock}

 \mbox{}\hfill\visible<9->{\myaudio{./audio/012_can_07.mp3}}
\end{frame}






\begin{frame}<1-4>[plain]\frametitle{Exercises}

日本文を参考にして(~~~~~~~~)の語を並べかえ、英文を完成させましょう。
ただし先頭に来る単語は大文字で始めてください。

\begin{enumerate}
 \item  ( can / this English song / sing / you ) ?\hspace{2\zw}%
\begin{minipage}{15\zw}あなたはこの英語の歌を歌うことができますか。\raisebox{-5pt}{\rotatebox{0}{\scalebox{2.5}{\twemoji{musical notes}}}}\end{minipage}
\\
\visible<2->{$\longrightarrow$\,\,\,Can you sing this English song?}
 \item ( you / cook / can / pizza ) ?
\hspace{5.3\zw}あなたはピザをつくれますか。\raisebox{5pt}{\rotatebox{-45}{\scalebox{2.5}{\twemoji{pizza}}}}\\
\visible<3->{$\longrightarrow$\,\,\,Can you cook pizza?}
 \item ( a car / John / drive / can ) ?\hspace{4.8\zw}%
ジョンは車の運転ができますか。\raisebox{0pt}{\rotatebox{0}{\scalebox{2.5}{\twemoji{oncoming automobile}}}}\\
\visible<3->{$\longrightarrow$\,\,\,Can John drive a car?}
\end{enumerate}

\pause

\mbox{}\hfill\visible<4>{\myaudio{./audio/012_can_08.mp3}}
\end{frame}

\subsection{Can you 〜? と聞かれたら}
\begin{frame}[plain]{Can you 〜? と聞かれたら}
 \Large

Can you cook pizza?

\vspace{20pt}
\pause

\mbox{}\hspace{100pt}$\left\{\begin{tabular}{l}
         \text{Yes, I can.}\\\pause
         \text{No, I cannot.}\\\pause
         \text{(}= \text{No, I can't.)}
        \end{tabular}\right.$


\end{frame}

\begin{frame}[plain]{Exercises}

\begin{enumerate}
 \item Can you ski?
 \item Can she drive a car?
 \item Can they speak French?
 \item Can David and Stephen sing well?
\end{enumerate}

\end{frame}

\end{document}

