\documentclass[12pt]{jlreq}
%%%%%%%%%%%%%%%%%%%%%%%%%%%%
%% 欧文TTF/OTFフォントを利用するにはfontspec.styをロードする必要あり
%% 和文TTF/OTFフォントを利用するにはluatexja-fontspec.styをロードする必要あり
%% luatexja-fontspec.styはfontspec.styをないぶてきにロードする
%% lualatex-ja-preset.sty は luatexja-fontspec.styをロードする
%% つまり次の1行でluatexja-fontspec.sty, fontspec.styも自動的にロードされる
\usepackage[no-math,deluxe,expert,haranoaji]{luatexja-preset}
%%%%
\usepackage{graphicx}
\usepackage{xcolor}
\usepackage{pxrubrica}
\usepackage[default]{fontsetup}
%%%% tabular環境の改良版
\usepackage{tabularray}
\UseTblrLibrary{booktabs}
%%%% ハイパーリンク
%%%% hyperref.sty は preamble の最後で読み込む
\usepackage{hyperref}
\usepackage{xurl}
\hypersetup{
  bookmarks=true,
  bookmarksnumbered=true,
  pdfauthor={iotsuka1958}
}
%%%%%%%%%%%%%%%%%%%%%%%%%%%%%
\usepackage{tikz}
\usetikzlibrary{arrows}
\usepackage{tcolorbox}
%%%%%%%%%%%%%%%%%%%%%%%%%%%%%
\usepackage{luatexja-otf}
\ltjsetparameter{jacharrange={-2}}
%%%%%%%%%%%%%%%%%%%%%%%%%%%%%
\usepackage{array}
\usepackage{cases}
\usepackage{marginnote}
\let\textipa\undefined
\usepackage{tipa}
%%%%%%%%%%%%%%%%%%%%%%%%%%%%
\usepackage{fontawesome5}
\usepackage{pifont}
\usepackage{marvosym}
%%%%%%%%%%%%%%%%%%%%%%%%%%%%
% my_check 環境の定義
\usepackage{amsfonts}
% my_check 環境の定義
\newenvironment{my_check}
  {\begin{itemize}
    \renewcommand\labelitemi{$\square\hspace{0.5em}$}} % 間隔を0.5emに設定
  {\end{itemize}}
%%%%%%%%%%%%%%%%%%%%%%
% カスタム列指定子を定義
\newcolumntype{C}[1]{>{\centering\arraybackslash}m{#1}}
\newcolumntype{L}[1]{>{\raggedright\arraybackslash}m{#1}}
%%%%%%%%%%%%%%%%%%%%%%
%% 生徒に作業を指示するときのコマンド
%%%%%%%%%%%%%%%%%%%%%%
\newcommand{\mySagyo}{%
\begin{minipage}[t]{.98\textwidth}
\mbox{}\hrulefill\mbox{}\par%
\mbox{}\hfill{}\raisebox{-5pt}{作業}\hfill\mbox{}\par%
\mbox{}\hrulefill\mbox{}
\end{minipage}%
\par%
\bigskip
}
%%%%%%%%%%%%%%%%%%%%%%%%%%
%% 授業のはじめのルーティーン
%%%%%%%%%%%%%%%5%%%%%%%%%
\newcommand{\myStartLesson}{%
\vspace*{5pt}%
\noindent{}{\Large\gtfamily 授業スタート!}
\begin{my_check}
\item マイクON
\item カメラON\hspace{40pt}{\LARGE \ComputerMouse}
\end{my_check}
みなさん、こんにちは。
エデュオプちばの英語の授業にようこそ。\par
さて
音声は届いていますか?
なにかトラブルがあったら、チャットで教えてください。
どうぞよろしくお願いします。\par
きょうも暑いですね。
体調管理にはじゅうぶん気をつけてください。
みなさんも熱中症にならないように、
適切な水分補給をお願いします。
授業中でも差し支えありません。
遠慮なく水分を補給しながら、
リラックスして参加してください。\par
それでは授業にはいります。
\begin{my_check}
\item カメラOFF\hspace{40pt}{\LARGE \ComputerMouse}
\end{my_check}
}
%%%%%%%%%%%%%%%%%%%%%%%%%%%%
%% マウスのアイコン
%%%%%%%%%%%%%%%%%%%%%%%%%%%
\newcommand{\myMouse}{%
  {\large \ComputerMouse}
}
%%%%%%%%%%%%%%%%%%%%%%%%%%%%%
\begin{document}
%%%%%%%%%%%%%%%%%%%%%%%%%%%%%
\section{Animal}
これからアルファベットを6つ順番に読みあげます。
聞こえたアルファベットを順番に小文字で書いてください。
するとある単語になります。
その意味を表す図を選んでください。
答えはa, b, c, dの記号でお願いします。


それでははじめます。\faVolumeUp\,(1\,min)

いかがですか。
わかった人は、チャットで記号を答えてください。

ではもう一度聞いてみましょう。\faVolumeUp\,(1\,min)

それでは、聞こえてきたアルファベットを順番に確認しましょう。{\large \ComputerMouse}

a
{\large \ComputerMouse}
n
{\large \ComputerMouse}
i
{\large \ComputerMouse}
m
{\large \ComputerMouse}
a
{\large \ComputerMouse}
l

animalという単語になりました。

animalは「動物」です。
ですから、4つの図の中から選ぶのなら(a)が正解。

それではanimalの発音練習をしましょう。
カタカナ読みで
「アニマル」と平坦にいうのではなく、
先頭のaを強く発音します。

先頭のaは、
エとアの中間の音です。
エの口の形でアといってみてください。
catの発音と同じです。
エニモゥ。

それでは発音練習です。
\faVolumeUp\,(38\,sec)

きょうのクイズ、発音練習でした。

\myMouse
%%%%%%%%%%%%%%%%%%%%%%%%%%%%%
\section{Book}
これからアルファベットを4つ順番に読みあげます。
聞こえたアルファベットを順番に小文字で書いてください。
するとある単語になります。
その意味を表す図を選んでください。
答えはa, b, c, dの記号でお願いします。


それでははじめます。\faVolumeUp\,(48\,sec)

いかがですか。
わかった人は、チャットで記号を答えてください。

ではもう一度聞いてみましょう。\faVolumeUp\,(48\,sec)

それでは、聞こえてきたアルファベットを順番に確認しましょう。{\large \ComputerMouse}

b
{\large \ComputerMouse}
o
{\large \ComputerMouse}
o
{\large \ComputerMouse}
k

bookという単語になりました。

bookは「本、書物、書籍」ですよね。
ですから、4つの図の中から選ぶのなら(d)が正解。

それでは発音練習です。bookの発音です。
「ブック」と平坦にいうのではなく、
先頭のbという音に続く「ウ」という音は、
おちょぼ口で前に出す感じで発音するとうまく発音できますよ。
ではやってみましょう。
\faVolumeUp\,(37\,sec)

きょうのクイズ、発音練習でした。
%%%%%%%%%%%%%%%%%%%%%%%%%%%%%%%%%%
\section{Candy}
これからアルファベットを5つ順番に読みあげます。
聞こえたアルファベットを順番に小文字で書いてください。
するとある単語になります。
その意味を表す図を選んでください。
答えはa, b, c, dの記号でお願いします。

それでははじめます。\faVolumeUp\,(48\,sec)

いかがですか。
わかった人は、チャットで記号を答えてください。

それでは、聞こえてきたアルファベットを順番に確認しましょう。{\large \ComputerMouse}

c\,\,
{\large \ComputerMouse}\,\,
a\,\,
{\large \ComputerMouse}\,\,
n\,\,
{\large \ComputerMouse}\,\,
d
{\large \ComputerMouse}\,\,
y

ではもう一度聞いてみましょう。\faVolumeUp\,(53\,sec)

c, a, n, d, yという単語になりました。

candyという単語になりましたね。

candyは「あめ、キャンディ」ですよね。
ですから、4つの図の中から選ぶのなら(c)が正解。

それではcandyの発音練習をしましょう。
カタカナ読みで
「キャンディ」と平坦にいうのではなく、
先頭のaを強く発音します。

aは、
エとアの中間の音です。
エの口の形でアといってみてください。
両ほほを左右に引っ張り、口を横にあけて「ア」と発音する。
catの発音と同じです。

あ、どこかでやったと思う人はいますか。

おとといの授業でanimalという単語を勉強しましたね4.
animalの発音と同じ音です。

それでは発音練習です。

\faVolumeUp\,(38\,sec)

きょうのクイズ、発音練習でした。
%%%%%%%%%%%%%%%%%%%%%%%%%%%%%
\section{Diamond}
これからアルファベットを7つ順番に読みあげます。
聞こえたアルファベットを順番に小文字で書いてください。
するとある単語になります。
その意味を表す図を選んでください。
答えはa, b, c, dの記号でお願いします。

それでははじめます。\faVolumeUp\,(1\,min\,06sec)

いかがですか。
わかった人は、チャットで記号を答えてください。


ではもう一度聞いてみましょう。\faVolumeUp\,(1\,min06\,sec)

それでは、聞こえてきたアルファベットを順番に確認しましょう。{\large \ComputerMouse}

{\large \ComputerMouse}

diamondという単語になりました。

diamond.
トランプのマークがありますね。

(b)が正解でした。

ところで日本語では「ダイヤ」のようにアではなくヤということもおおいのですが、
英語の綴りをもてもらうとわかるようにyというアルファベットは使われていませんよね。
ですから英語では「ダイヤ」とか「ダイヤモンド」とはいわないことに注意しましょう。
それでは発音練習です。diamondの発音です。
「ダイアモンド」と平坦にいうのではなく、
「ダイ」にアクセントを置いて発音しましょう。

ここでdiamondに含まれる「アイ」の発音に注目してください。
「アイ」と平坦にいうのではなく「アーイ」」となります。
そのことに注意して尾根性を聞いてください。
ではやってみましょう。
\faVolumeUp\,(38\,sec)

きょうのクイズ、発音練習でした。

{\large \ComputerMouse}

あ、もしかするとdiamondと聞いたときに、みなさんのアタマにまっさきにうかぶのはこれかもしれないですね。
ただ、きょうはトランプのマークだったので、とまどった人がいるかもしれませんね。
%%%%%%%%%%%%%%%%%%%%%%%%%5
\section{Eight}
これからアルファベットを5つ順番に読みあげます。
聞こえたアルファベットを順番に小文字で書いてください。
するとある単語になります。
その意味を表すものを選んでください。
答えはa, b, c, dの記号でお願いします。

それでははじめます。\marginnote{\tiny \faVolumeUp\,(54\,sec)}

いかがですか。
わかった人は、チャットで記号を答えてください。

それでは、聞こえてきたアルファベットを順番に確認しましょう。{\large \ComputerMouse}

e--i--g--h--t

5つのアルファベットを確認したところで、
もう一度聞いてみましょう。\marginnote{\tiny \faVolumeUp\,54\,sec}


{\large \ComputerMouse}

e--i--g--h--tという綴になりました。この単語はどう発音しますか。

そうです。「エイト」ですね。

「8」の意味です。

では選択肢のうちどれが世界でしょうか。

\begin{itemize}
 \item[(a)] サイコロの4と6の目がでています。あわせると、これは「10」ですね
 \item[(b)] これは、みてのとおり「9」です
 \item[(d)] (c)の前に(d)をみてみましょう。\\
「ダイア」の7ですね。前回学習した単語diamondを覚えていますか。平坦に「ダイアモンド」というのではなく、「{\gtfamily ダ}イアモンド」と発音することも確認しておきましょう
 \item[(c)] 数式が出ています。計算してみましょう。\\
14から2を弾いて12。この12に3をかけて36と思った人はいないでしょうね。それでは答がなくなってしまいます。かけ算やわり算をたし算やひき算より先に計算するのが約束でしたね。ですから、この問題では、14から2を引くのではなく、まず掛け算である$2\times{3}=6$を先に計算して、$14-6=8$とするのでしたね。8がでてきました。ではこれが正解です。
\end{itemize}

みなさん、いかがでしたか。

それでは、きょうの単語eightの発音練習です。

エとイではなく、連続するエ~ィという音です。

流れるように1つの音として読むことを意識してください。

また、この単語はもうひとつ注意してほしいことがあります。

綴のなかでg--hという2文字はまったく発音に関係ないことです。

このあたりがやっかいなところですが、そういうものだとおもって練習しましょう。


そのことに注意して音声を聞いてください。
ではやってみましょう。
\marginnote{\tiny \faVolumeUp\,(37\,sec)}

発音練習していただきましたが、書くときには、
発音しないけれど、きちんとghを書くことに注意しましょう。


きょうのクイズ、発音練習でした。
%%%%%%%%%%%%%%%%%%%%%%%%%%%%%%
\section{Fish}

これからアルファベットを4つ順番に読みあげます。
聞こえたアルファベットを順番に小文字で書いてください。
するとある単語になります。
その意味を表す図を選んでください。
答えはa, b, c, dの記号でお願いします。

それでははじめます。\faVolumeUp\,(48\,sec)

いかがですか。
わかった人は、チャットで記号を答えてください。

それでは、聞こえてきたアルファベットを順番に確認しましょう。{\large \ComputerMouse}

f\,\,
{\large \ComputerMouse}\,\,
i\,\,
{\large \ComputerMouse}\,\,
s\,\,
{\large \ComputerMouse}\,\,
h

ではもう一度聞いてみましょう。\faVolumeUp\,(48\,sec)

fishという単語になりました。

それではfishの発音練習をしましょう。
カタカナ読みで
「フィッシュ」と平坦にいうのではなく、
iを強く発音します。

それでは発音練習です。

\faVolumeUp\,(38\,sec)

きょうのクイズ、発音練習でした。
%%%%%%%%%%%%%%%%%%%%%%%%%%%%%%
\section{Ghost}

これからアルファベットを5つ順番に読みあげます。聞こえたアルファベッ
トを順番に小文字で書いてください。
うまく聞き取れると、ある単語になります。
その意味を表すものを選んでください

答えはa, b, c, dの記号でお願いします。

それでははじめます。\faVolumeUp\,(54\,sec)

いかがですか。
わかった人は、チャットで記号を答えてください。

それでは、聞こえてきたアルファベットを順番に確認しましょう。{\large \ComputerMouse}

g\,\,
{\large \ComputerMouse}\,\,
h\,\,
{\large \ComputerMouse}\,\,
o\,\,「オー」とのばすのではなく「オウ」オからウになめらかに変化させる気持ちです。
{\large \ComputerMouse}\,\,
s
\myMouse\,\,
t

ではもう一度聞いてみましょう。\marginnote{\faVolumeUp\,(54\,sec)}


ghostという単語になりました。

ghostは幽霊です。

aはアヒル。

cはホウキにのっています。魔女です。

dは星。

正解はbです。


発音は「ゴウスト」/\textipa{g\'oUst}/

「オー」と伸ばすのではありません。
「オ」から「ウ」になめらかに変化させる気持ちです。
最後は「ト」ではなくt.

それでは発音練習してみましょう。
\marginnote{\faVolumeUp\,(38\,sec)}

きょうの単語はghostでした。
%%%%%%%%%%%%%%%%%%%%%%%%%%%%%%
\newpage
\section{Horse}

これからアルファベットを5つ順番に読みあげます。聞こえたアルファベッ
トを順番に小文字で書いてください。
うまく聞き取れると、ある単語になります。
その意味を表すものを選んでください

答えはa, b, c, dの記号でお願いします。

それでははじめます。\faVolumeUp\,(54\,sec)

いかがですか。
わかった人は、チャットで記号を答えてください。

それでは、聞こえてきたアルファベットを順番に確認しましょう。{\large \ComputerMouse}

h\,\,
{\large \ComputerMouse}\,\,
o\,\,
{\large \ComputerMouse}\,\,
r\,\,「オー」とのばす音です。
{\large \ComputerMouse}\,\,
s
\myMouse\,\,
e

ではもう一度聞いてみましょう。\marginnote{\faVolumeUp\,(54\,sec)}


horseという単語になりました。

horseは馬です。

aはアヒル。

bは象。

cは月でしょうか。

正解はd。



発音は「ホース」/\textipa{h\textopeno\textlengthmark{}{r}s}/

「オー」と伸ばす音。

それでは発音練習してみましょう。
\marginnote{\faVolumeUp\,(38\,sec)}

きょうの単語はhorseでした。
%%%%%%%%%%%%%%%%%%%%%%%%%%%%%%
\newpage
\section{ice}
いつものようにToday's Quizです。

これからアルファベットを3つ順番に読みあげます。聞こえたアルファベッ
トを順番に小文字で書いてください。
うまく聞き取れると、ある単語になります。
その意味を表すものを選んでください

答えはa, b, c, dの記号でお願いします。

それでははじめます。\faVolumeUp\,(54\,sec)

いかがですか。
わかった人は、チャットで記号を答えてください。

それでは、聞こえてきたアルファベットを順番に確認しましょう。{\large \ComputerMouse}

i\,\,
{\large \ComputerMouse}\,\,
c\,\,
{\large \ComputerMouse}\,\,
e\,\,

ではもう一度聞いてみましょう。\marginnote{\faVolumeUp\,(54\,sec)}


iceという単語になりました。
iceはどういう意味でしょうか。
きょうの冒頭で、
わたしはこういいました。
「きのう、わたしはアイスを食べました。みなさんはアイス好きですか?」
そのときみなさんの頭に浮かんだのはなんでしたか。
いわゆる「アイスクリーム」を思い浮かべた人がほとんどだっと思うのですがいかがですか。

英語のiceは、基本「氷」

ですから正解はa。

bはhamburger。

cはcheeseのつもりです。

dは、日本語では「ホットケーキ」といいますが、英語ではpancakeといいます。


では正解はaでした。\textipa{/aIs/}

それでは発音練習してみましょう。
\marginnote{\faVolumeUp\,(38\,sec)}

ところで、これはなんですか。日本語では
単に「アイス」とか「アイスクリーム」「ソフトクリーム」とかいいますが、
英語では通常ice cream。
発音は\textipa{/aIs kri:m/}

きょうの単語はiceでした。
%%%%%%%%%%%%%%%%%%%%%%%%%%%%%%
\newpage
\section{jigsaw}

これからアルファベットを6つ順番に読みあげます。聞こえたアルファベッ
トを順番に小文字で書いてください。
うまく聞き取れると、ある単語になります。
その意味を表すものを選んでください

答えはa, b, c, dの記号でお願いします。

それでははじめます。\faVolumeUp\,(54\,sec)

いかがですか。
わかった人は、チャットで記号を答えてください。

それでは、聞こえてきたアルファベットを順番に確認しましょう。{\large \ComputerMouse}

j\,\,
{\large \ComputerMouse}\,\,
i\,\,
{\large \ComputerMouse}\,\,
g\,\,
{\large \ComputerMouse}\,\,
s\,\,
{\large \ComputerMouse}\,\,
a\,\,
\myMouse\,\,
w
ではもう一度聞いてみましょう。\marginnote{\faVolumeUp\,(54\,sec)}


jigsawという単語になりました。

jgsawはジグソーパズルのことです。

bはトランプ。
ただ英語ではcardsといいます。

cはchessのつもりです。
みなさnchessって知っていますか。
日本の将棋に似たボードゲームです。

dはジュース。トロピカルな感じですね。



正解はaでした。

それでは発音練習してみましょう。
\marginnote{\faVolumeUp\,(38\,sec)}

きょうの単語はjigsawでした。
%%%%%%%%%%%%%%%%%%%%%%%%%%%%%%
\newpage
\section{key}

これからアルファベットを3つ順番に読みあげます。聞こえたアルファベッ
トを順番にお手元のノートに小文字で書いてください。
うまく聞き取れると、ある単語になります。
その単語の意味を表すものを選んでください

答えはa, b, c, dの記号でお願いします。

それでははじめます。\faVolumeUp\,(54\,sec)

いかがですか。
わかった人は、チャットで記号を答えてください。

それでは、聞こえてきたアルファベットを順番に確認しましょう。{\large \ComputerMouse}

k\,\,
{\large \ComputerMouse}\,\,
e\,\,
{\large \ComputerMouse}\,\,
y
ではもう一度聞いてみましょう。\marginnote{\faVolumeUp\,(54\,sec)}

keyという単語になりました。

keyは鍵のことです。「車のキー」とかいうことがありますよね。

aは椰子の木。

bはカップ。

cはタコ。タコはタコでも8本足のタコではなく、空に飛ばして遊ぶタコ。

dが鍵。これが正解でした。


それでは発音練習してみましょう。

\textipa{/i:/}は、唇を横に引いて、口はあまり開けずに「イー」と発音しましょう。
\marginnote{\faVolumeUp\,(38\,sec)}

きょうの単語はkeyでした。
%%%%%%%%%%%%%%%%%%%%%%%%%%%%%%
\newpage
\section{Lemon}

これからアルファベットを5つ順番に読みあげます。聞こえたアルファベッ
トを順番に小文字で書いてください。
うまく聞き取れると、ある単語になります。
その意味を表すものを選んでください

答えはa, b, c, dの記号でお願いします。

それでははじめます。\faVolumeUp\,(54\,sec)

いかがですか。
わかった人は、チャットで記号を答えてください。

それでは、聞こえてきたアルファベットを順番に確認しましょう。{\large \ComputerMouse}

l\,\,
{\large \ComputerMouse}\,\,
e\,\,
{\large \ComputerMouse}\,\,
m
{\large \ComputerMouse}\,\,
o\,\,
{\large \ComputerMouse}\,\,
n

ではもう一度聞いてみましょう。\marginnote{\faVolumeUp\,(54\,sec)}

lemonという単語になりました。

lemonはもちろんレモンです。

aはレモン。これが正解でしょう。

bはオレンジ。

cはキャンディ。

dはバナナ。


それでは発音練習してみましょう。
\marginnote{\faVolumeUp\,(38\,sec)}

きょうの単語はlemonでした。
%%%%%%%%%%%%%%%%%%%%%%%%%%%%%%
\newpage
\section{milk}

これからアルファベットを3つ順番に読みあげます。聞こえたアルファベッ
トを順番に小文字で書いてください。
うまく聞き取れると、ある単語になります。
その意味を表すものを選んでください

答えはa, b, c, dの記号でお願いします。

それでははじめます。\faVolumeUp\,(54\,sec)

いかがですか。
わかった人は、チャットで記号を答えてください。

それでは、聞こえてきたアルファベットを順番に確認しましょう。{\large \ComputerMouse}

m\,\,
{\large \ComputerMouse}\,\,
i\,\,
{\large \ComputerMouse}\,\,
l
{\large \ComputerMouse}\,\,
k\,\,

ではもう一度聞いてみましょう。\marginnote{\faVolumeUp\,(54\,sec)}

milkという単語になりました。

milkはもちろん牛乳です。

aは何でしょう。タピオカドリンクでしょうか。

bは白い飲み物。これが正解でしょうか。

cはりんご。

dはスイカ。

どうやら正解はbでよさそうですね。

それでは発音練習してみましょう。
\marginnote{\faVolumeUp\,(38\,sec)}

きょうの単語はmilkでした。

{\large \ComputerMouse}\,\,
%%%%%%%%%%%%%%%%%%%%%%%%%%%%%%
\newpage
\section{notebook}

これからアルファベットを8つ順番に読みあげます。聞こえたアルファベッ
トを順番に小文字で書いてください。
うまく聞き取れると、ある単語になります。
その意味を表すものを選んでください

答えはa, b, c, dの記号でお願いします。

それでははじめます。\faVolumeUp\,(54\,sec)

いかがですか。
わかった人は、チャットで記号を答えてください。

それでは、聞こえてきたアルファベットを順番に確認しましょう。{\large \ComputerMouse}

n\,\,
{\large \ComputerMouse}\,\,
o\,\,
{\large \ComputerMouse}\,\,
t
{\large \ComputerMouse}\,\,
e\,\,m\,\,
{\large \ComputerMouse}\,\,
b\,\,
{\large \ComputerMouse}\,\,
o
{\large \ComputerMouse}\,\,
o\,\,
{\large \ComputerMouse}\,\,
k\,\,

ではもう一度聞いてみましょう。\marginnote{\faVolumeUp\,(54\,sec)}

notebookという単語になりました。

notebookはも日本語で言う「ノート」です。

aは何でしょう。はさみ。

bは白い飲み物。

cは万年筆

dがノートブック。本みたいにも見えますが、正解はこれしかなさそうですね。


それでは発音練習してみましょう。
\marginnote{\faVolumeUp\,(38\,sec)}

きょうの単語はnotebookでした。

{\large \ComputerMouse}\,\,
n
%%%%%%%%%%%%%%%%%%%%%%%%%%%%%%
\newpage
\section{onion}

これからアルファベットを5つ順番に読みあげます。聞こえたアルファベッ
トを順番に小文字で書いてください。
うまく聞き取れると、ある単語になります。
その意味を表すものを選んでください

答えはa, b, c, dの記号でお願いします。

それでははじめます。\faVolumeUp\,(54\,sec)

いかがですか。
わかった人は、チャットで記号を答えてください。

それでは、聞こえてきたアルファベットを順番に確認しましょう。{\large \ComputerMouse}

o\,\,
{\large \ComputerMouse}\,\,
n\,\,
{\large \ComputerMouse}\,\,
i
{\large \ComputerMouse}\,\,
o\,\,
{\large \ComputerMouse}\,\,
n\,\,

ではもう一度聞いてみましょう。\marginnote{\faVolumeUp\,(54\,sec)}

onionという単語になりました。

「タマネギ」という意味の単語です。

aはトマト。

bはブロッコリ。

cがタマネギっぽいですね。

dはナス。


正解はCです。

それでは発音練習してみましょう。

o--n--i--o--n.

これ、ローマ字読みすると「オニオン」となりますが、それでは英語としてはうまくないです。
英語の発音は\textipa{/2nj@n/}
\marginnote{\faVolumeUp\,(38\,sec)}

きょうの単語はonionでした。

{\large \ComputerMouse}\,\,
%%%%%%%%%%%%%%%%%%%%%%%%%%%%%%
\newpage
\section{peach}

これからアルファベットを5つ順番に読みあげます。聞こえたアルファベッ
トを順番に小文字で書いてください。
うまく聞き取れると、ある単語になります。
その意味を表すものを選んでください

答えはa, b, c, dの記号でお願いします。

それでははじめます。\faVolumeUp\,(54\,sec)

いかがですか。
わかった人は、チャットで記号を答えてください。

それでは、聞こえてきたアルファベットを順番に確認しましょう。{\large \ComputerMouse}

p\,\,
{\large \ComputerMouse}\,\,
e\,\,
{\large \ComputerMouse}\,\,
a
{\large \ComputerMouse}\,\,
c\,\,
{\large \ComputerMouse}\,\,
h\,\,

ではもう一度聞いてみましょう。\marginnote{\faVolumeUp\,(54\,sec)}

peachという単語になりました。

「もも」という意味の単語です。

aはいちご

bはオレンジ。

cはバナナ

dがもも。


正解はdです。

それでは発音練習してみましょう。

きょうの単語peachでした。

{\large \ComputerMouse}\,\,

%%%%%%%%%%%%%%%%%%%%%%%%%%%%%%
\newpage
\section{queen}

これからアルファベットを5つ順番に読みあげます。聞こえたアルファベッ
トを順番に小文字で書いてください。
うまく聞き取れると、ある単語になります。
その意味を表すものを選んでください

答えはa, b, c, dの記号でお願いします。

それでははじめます。\faVolumeUp\,(54\,sec)

いかがですか。
わかった人は、チャットで記号を答えてください。

それでは、聞こえてきたアルファベットを順番に確認しましょう。{\large \ComputerMouse}

q\,\,
{\large \ComputerMouse}\,\,
u\,\,
{\large \ComputerMouse}\,\,
e
{\large \ComputerMouse}\,\,
e\,\,
{\large \ComputerMouse}\,\,
n\,\,

ではもう一度聞いてみましょう。\marginnote{\faVolumeUp\,(55\,sec)}

queenという単語になりました。

「女王」という意味の単語です。

aはjack

bがqueen

cはking、王様です

dはjoker


正解はbです。

それではqueenの発音練習してみましょう。

カタカナだと「クイーン」といいますね。
クの後がイでいいますが、
英語では\textipa{/kw/}

ちょっとやってみましょう。
\textipa{/k/}
クのように「ウ」の音を入れてはいけませんよ。

そのあと唇を強くすぼめて「ウィーン」

「クイーン」ではなく「クウィーン」といったイメージ。

それでは練習しましょう。\marginnote{\faVolumeUp\,(34\,sec)}

queenの出だしと同じ発音の単語をもう一つ。

quizも\textipa{/kwiz/}

カタカナだと「クイズ」ですが「クウィズ」といった響きです。

please repeat after me. Wuiz. Quiz.

きょうの単語queenでした。

{\large \ComputerMouse}\,\,

%%%%%%%%%%%%%%%%%%%%%%%%%%%%%%
\newpage
\section{rainbow}

これからアルファベットを7つ順番に読みあげます。聞こえたアルファベッ
トを順番に小文字で書いてください。
うまく聞き取れると、ある単語になります。
その意味を表すものを選んでください

答えはa, b, c, dの記号でお願いします。

それでははじめます。\faVolumeUp\,(1\,min6\,sec)

いかがですか。
わかった人は、チャットで記号を答えてください。

それでは、聞こえてきたアルファベットを順番に確認しましょう。{\large \ComputerMouse}

r\,\,
{\large \ComputerMouse}\,\,
a\,\,
{\large \ComputerMouse}\,\,
i
{\large \ComputerMouse}\,\,
n\,\,
{\large \ComputerMouse}\,\,
b\,\,
{\large \ComputerMouse}\,\,
o\,\,
{\large \ComputerMouse}\,\,
w\,\,

ではもう一度聞いてみましょう。\faVolumeUp\,(1\,min6\,sec)

rainbowという単語になりました。

「虹」という意味の単語です。



bは雪。

cは太陽が雲に少し隠れています

dは雷雨


正解はAです。

それでは発音練習してみましょう。

日本語では「レ\kenten{イ}ンボー」のように「イ」を強く読むことがおおいとおもいますが、
英語のアクセントは「レ」にあります。
そのことくを意識するだけで格段とじょうずに発音できますよ。
また最後のbowは「ボー」とのばすのではなく「ボウ」とウをつけるのわわすrないでください。

では、この2つを意識して発音練習です。\marginnote{\faVolumeUp\,(38\,sec)}

きょうの単語rainbowでした。

{\large \ComputerMouse}\,\,


\paragraph{photo}

実はrainは「雨」でbowは「弓」という意味で、
rainbowとは「雨が空にかけた弓」というのがもともとの意味です。

一般的に、
朝に虹が見えた時はこれから雨が降るサイン。
夕方は雨上がりに虹が見える時が多いといわれています。
いずれにせよ、虹は雨と深い関係があるわけです。

rainbowとは「雨が空にかけた弓」というのがもともとの意味です。
ずいぶんと美しい弓ですよね。

%%%%%%%%%%%%%%%%%%%%%%%%%%%%%%
\newpage
\section{strawberry}

これからアルファベットを7つ順番に読みあげます。聞こえたアルファベッ
トを順番に小文字で書いてください。
うまく聞き取れると、ある単語になります。
その意味を表すものを選んでください

答えはa, b, c, dの記号でお願いします。

それでははじめます。\faVolumeUp\,(1\,min23\,sec)

いかがですか。
わかった人は、チャットで記号を答えてください。

それでは、聞こえてきたアルファベットを順番に確認しましょう。{\large \ComputerMouse}

s\,\,
{\large \ComputerMouse}\,\,
t\,\,
{\large \ComputerMouse}\,\,
r
{\large \ComputerMouse}\,\,
a\,\,
{\large \ComputerMouse}\,\,
w\,\,
{\large \ComputerMouse}\,\,
b\,\,
{\large \ComputerMouse}\,\,
e\,\,
{\large \ComputerMouse}\,\,
r\,\,
{\large \ComputerMouse}\,\,
r\,\,
{\large \ComputerMouse}\,\,
y\,\,
{\large \ComputerMouse}\,\,



ではもう一度聞いてみましょう。\faVolumeUp\,(1\,min23\,sec)

rainbowstrawberryという単語になりました。

「いちご」という意味の単語です。

ahかき氷

bはorange。

cはなんでしょうか。なにかの飲み物でしょう

dがいちご、これg正解です。


それでは発音練習してみましょう。

最初のstは完全に「息の音」となります。
ひそひそ話をする感じです。

そして、その後の「ロー」の部分を長めに発音しながら、ここにグッとアクセントを置きます。

その後、「ベ」から後ろは弱く発音します。

日本人の多くは「ストロベリー」というように「ロ」を短く発音しますが、実際の発音は長めに「ロー」となることに注意しましょう。

では、\marginnote{\faVolumeUp\,(40\,sec)}

きょうの単語strawberryでした。

{\large \ComputerMouse}\,\,


\paragraph{photo}


%%%%%%%%%%%%%%%%%%%%%%%%%%%%%%
\newpage
\section{tiger}

これからアルファベットを5つ順番に読みあげます。聞こえたアルファベッ
トを順番に小文字で書いてください。
うまく聞き取れると、ある単語になります。
その意味を表すものを選んでください

答えはa, b, c, dの記号でお願いします。

それでははじめます。\faVolumeUp\,(54\,sec)

いかがですか。
わかった人は、チャットで記号を答えてください。

それでは、聞こえてきたアルファベットを順番に確認しましょう。{\large \ComputerMouse}

t\,\,
{\large \ComputerMouse}\,\,
i\,\,
{\large \ComputerMouse}\,\,
g
{\large \ComputerMouse}\,\,
e\,\,
{\large \ComputerMouse}\,\,
r\,\,



ではもう一度聞いてみましょう。\faVolumeUp\,(54\,sec)

tigerという単語になりました。

「トラ」という意味の単語です。

aは「しまうま」

bが「トラ」です。これが正解。

cはなんでしょうか。「ライオン」です。

dはなんでしょう。わかる人はチャットでお願いします。
恐竜のつもり。T--Rexのつもりでした。


それでは発音練習してみましょう。

「アイ」のところは「ア」を強く発音すること、「ア」から「イ」になめらかにつなげることを意識しましょう。

カタカナ読みして平板にならないようにしましょう。

では、\marginnote{\faVolumeUp\,(38\,sec)}

きょうの単語tigerでした。

{\large \ComputerMouse}\,\,


\paragraph{photo}

%%%%%%%%%%%%%%%%%%%%%%%%%%%%%%%%%%
\newpage
\section{umbrella}

%%%%%%%%%%%%%%%%%%%%%%%%%%%%%%%%
\newpage
\section{vegetable}

vegetableは「野菜」

ポイントは3つです

\begin{itemize}
 \item 先頭のvはきのう学習した音ですね。
下唇に上の前歯を当ててこするようにして発音します。
有声音ですから声に出していうのでしたね
 \item アクセントは最初のeのところ。ここを強く読みます

 \item 語尾は「ブル」ではなく「ボー」という感じです
\end{itemize}


%%%%%%%%%%%%%%%%%%%%%%%%%%%%%%%%%%
\newpage
\section{watch}

\faVolumeUp\,\,(55\,sec)

先頭のwは口を丸めて突き出します。

2文字目のaの発音ですが、
日本語の「ア」に近い音ですが、日本語の「ア」と同じというわけではありません。

ためしに日本語の「ア」といってみてください。
口はどれくらい開いていますか。
おそらくそれほど開いていないのではないでしょうか。

きょうの単語のaの発音では、
\begin{enumerate}
 \item 口を大きく開けること(指3本)$\leftarrow$\,これが第1のポイントです
 \item 舌を下げる
 \item 欠伸をするときみたいに喉の奥から「あー」
\end{enumerate}

aはwhale、bはspoon,dはball, volleyball

\faVolumeUp\,\,(38\,sec)
%%%%%%%%%%%%%%%%%%%%%%%%%%%%%%%%%%
\newpage
\section{Xmas}

これChristmasと読みます。
Xを用いずにChristmasと表記することもあります。

正解はdです。クリスマスツリーです。
aはひな人形。桃の節句。bは門松、正月です。cはHalloweenのかぼちゃですね。
Jack-o'-Lantern(/ˌdʒæk.əˈlæn.tɚn/)

ところで、よくX'masという表記がされるのですが、これはまちがいですから、もしこういう表記を見たらまちがいですよと教えてあげるといいですね。

それではChristmasの発音です。/ˈkrɪs.məs/

日本語読みの「クリスマス」にならないように注意が必要です。

リのところにアクセントがありますから、そこをはっきり強くいいましょう

先頭の\textipa{/krI/}のところですが、日本語の「ク」にならないようにしましょう。
kの音の後に「う」という母音をはさまないことがかんじんです。

「クリ」ではなくって\textipa{/kr\textsci /}

いかがですか?

「栗ごはん」というときは「ク」と母音がはいってますよね「ク」に。

じゃあ、「しゃっくり」はどうですか。Shakkuriとはいってませんよね。

%%%%%%%%%%%%%%%%%%%%%%%%%%
\newpage
\section{yellow}

yellowはカタカナで「イエロー」といいますが、
最初の出だしの\textipa{/je/}のところに注目してください。

日本語のヤ行の音とおもっていいとおもうのですが、
日本語の「ヤ」は\textipa{/j/}という子音に「あ」が続いて「ヤ」となります。

「や」というときの口のかたちを確認してみましょう。
皆さんも今ご自分で「ヤ」といってみてください。
舌の動きがどうなっているか確認しましょう。

「ヤ」、「ヤ」。いかがですか。舌の真ん中あたりが口の中で上あごに近づいていませんか。

このように日本語のヤ行の音は、舌の真ん中あたりを上にもちあげ上あごの近づけて\textipa{/j]}という子音を出してから母音を続けます。\textipa{/j/}のあとに
それに母音の「あ」をつければ「や」。
それに母音の「う」をつければ「ゆ」。
それに母音の「お」をつければ「よ」。

それと同じ要領で
それに母音の「い」と母音の「え」をつけてみましょう。


現代の日本語では「ヤユヨ」にはこの子音が使われるのですが、つまり「アウオ」の段ではこの子音が使われるのですが、「イエ」の段ではこの子音が使われていません。


きょうの単語yellowは\textipa{/je}では、この子音\textipa{/j/}をしっかり発音してから「エ」といいます。
ここは「え」にヤ行の子音をいれてみてください。

yesterdayとかyesもこの音ですね。






\textipa{/ja/}, \textipa{/ji/}, \textipa{/ju/}, \textipa{/je/}, \textipa{/jo/}
\end{document}
