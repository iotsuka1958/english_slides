\documentclass[12pt]{jlreq}
%%%%%%%%%%%%%%%%%%%%%%%%%%%%
%% 欧文TTF/OTFフォントを利用するにはfontspec.styをロードする必要あり
%% 和文TTF/OTFフォントを利用するにはluatexja-fontspec.styをロードする必要あり
%% luatexja-fontspec.styはfontspec.styをないぶてきにロードする
%% lualatex-ja-preset.sty は luatexja-fontspec.styをロードする
%% つまり次の1行でluatexja-fontspec.sty, fontspec.styも自動的にロードされる
\usepackage[no-math,deluxe,expert,haranoaji]{luatexja-preset}
%%%%
\usepackage{graphicx}
\usepackage{xcolor}
\usepackage{pxrubrica}
\usepackage[default]{fontsetup}
%%%% tabular環境の改良版
\usepackage{tabularray}
\UseTblrLibrary{booktabs}
%%%% ハイパーリンク
%%%% hyperref.sty は preamble の最後で読み込む
\usepackage{hyperref}
\usepackage{xurl}
\hypersetup{
  bookmarks=true,
  bookmarksnumbered=true,
  pdfauthor={iotsuka1958}
}
%%%%%%%%%%%%%%%%%%%%%%%%%%%%%
\usepackage{tikz}
\usetikzlibrary{arrows}
\usepackage{tcolorbox}
%%%%%%%%%%%%%%%%%%%%%%%%%%%%%
\usepackage{luatexja-otf}
\ltjsetparameter{jacharrange={-2}}
%%%%%%%%%%%%%%%%%%%%%%%%%%%%%
\usepackage{array}
\usepackage{cases}
\usepackage{marginnote}
\let\textipa\undefined
\usepackage{tipa}
%%%%%%%%%%%%%%%%%%%%%%%%%%%%
\usepackage{fontawesome5}
\usepackage{pifont}
\usepackage{marvosym}
%%%%%%%%%%%%%%%%%%%%%%%%%%%%
% my_check 環境の定義
\usepackage{amsfonts}
% my_check 環境の定義
\newenvironment{my_check}
  {\begin{itemize}
    \renewcommand\labelitemi{$\square\hspace{0.5em}$}} % 間隔を0.5emに設定
  {\end{itemize}}
%%%%%%%%%%%%%%%%%%%%%%
% カスタム列指定子を定義
\newcolumntype{C}[1]{>{\centering\arraybackslash}m{#1}}
\newcolumntype{L}[1]{>{\raggedright\arraybackslash}m{#1}}
%%%%%%%%%%%%%%%%%%%%%%
%% 生徒に作業を指示するときのコマンド
%%%%%%%%%%%%%%%%%%%%%%
\newcommand{\mySagyo}{%
\begin{minipage}[t]{.98\textwidth}
\mbox{}\hrulefill\mbox{}\par%
\mbox{}\hfill{}\raisebox{-5pt}{作業}\hfill\mbox{}\par%
\mbox{}\hrulefill\mbox{}
\end{minipage}%
\par%
\bigskip
}
%%%%%%%%%%%%%%%%%%%%%%%%%%
%% 授業のはじめのルーティーン
%%%%%%%%%%%%%%%5%%%%%%%%%
\newcommand{\myStartLesson}{%
\vspace*{5pt}%
\noindent{}{\Large\gtfamily 授業スタート!}
\begin{my_check}
\item マイクON
\item カメラON\hspace{40pt}{\LARGE \ComputerMouse}
\end{my_check}
みなさん、こんにちは。
エデュオプちばの英語の授業にようこそ。\par
さて
音声は届いていますか?
なにかトラブルがあったら、チャットで教えてください。
どうぞよろしくお願いします。\par
きょうも暑いですね。
体調管理にはじゅうぶん気をつけてください。
みなさんも熱中症にならないように、
適切な水分補給をお願いします。
授業中でも差し支えありません。
遠慮なく水分を補給しながら、
リラックスして参加してください。\par
それでは授業にはいります。
\begin{my_check}
\item カメラOFF\hspace{40pt}{\LARGE \ComputerMouse}
\end{my_check}
}
%%%%%%%%%%%%%%%%%%%%%%%%%%%%
%% マウスのアイコン
%%%%%%%%%%%%%%%%%%%%%%%%%%%
\newcommand{\myMouse}{%
  {\large \ComputerMouse}
}
%%%%%%%%%%%%%%%%%%%%%%%%%%%%%
\begin{document}
%%%%%%%%%%%%%%%%%%%%%%%%%%%%%
\section{Animal}
これからアルファベットを6つ順番に読みあげます。
聞こえたアルファベットを順番に小文字で書いてください。
するとある単語になります。
その意味を表す図を選んでください。
答えはa, b, c, dの記号でお願いします。


それでははじめます。\faVolumeUp\,(1\,min)

いかがですか。
わかった人は、チャットで記号を答えてください。

ではもう一度聞いてみましょう。\faVolumeUp\,(1\,min)

それでは、聞こえてきたアルファベットを順番に確認しましょう。{\large \ComputerMouse}

a
{\large \ComputerMouse}
n
{\large \ComputerMouse}
i
{\large \ComputerMouse}
m
{\large \ComputerMouse}
a
{\large \ComputerMouse}
l

animalという単語になりました。

animalは「動物」です。
ですから、4つの図の中から選ぶのなら(a)が正解。

それではanimalの発音練習をしましょう。
カタカナ読みで
「アニマル」と平坦にいうのではなく、
先頭のaを強く発音します。

先頭のaは、
エとアの中間の音です。
エの口の形でアといってみてください。
catの発音と同じです。
エニモゥ。

それでは発音練習です。
\faVolumeUp\,(38\,sec)

きょうのクイズ、発音練習でした。

\myMouse
%%%%%%%%%%%%%%%%%%%%%%%%%%%%%
\section{Book}
これからアルファベットを4つ順番に読みあげます。
聞こえたアルファベットを順番に小文字で書いてください。
するとある単語になります。
その意味を表す図を選んでください。
答えはa, b, c, dの記号でお願いします。


それでははじめます。\faVolumeUp\,(48\,sec)

いかがですか。
わかった人は、チャットで記号を答えてください。

ではもう一度聞いてみましょう。\faVolumeUp\,(48\,sec)

それでは、聞こえてきたアルファベットを順番に確認しましょう。{\large \ComputerMouse}

b
{\large \ComputerMouse}
o
{\large \ComputerMouse}
o
{\large \ComputerMouse}
k

bookという単語になりました。

bookは「本、書物、書籍」ですよね。
ですから、4つの図の中から選ぶのなら(d)が正解。

それでは発音練習です。bookの発音です。
「ブック」と平坦にいうのではなく、
先頭のbという音に続く「ウ」という音は、
おちょぼ口で前に出す感じで発音するとうまく発音できますよ。
ではやってみましょう。
\faVolumeUp\,(37\,sec)

きょうのクイズ、発音練習でした。
%%%%%%%%%%%%%%%%%%%%%%%%%%%%%%%%%%
\section{Candy}
これからアルファベットを5つ順番に読みあげます。
聞こえたアルファベットを順番に小文字で書いてください。
するとある単語になります。
その意味を表す図を選んでください。
答えはa, b, c, dの記号でお願いします。

それでははじめます。\faVolumeUp\,(48\,sec)

いかがですか。
わかった人は、チャットで記号を答えてください。

それでは、聞こえてきたアルファベットを順番に確認しましょう。{\large \ComputerMouse}

c\,\,
{\large \ComputerMouse}\,\,
a\,\,
{\large \ComputerMouse}\,\,
n\,\,
{\large \ComputerMouse}\,\,
d
{\large \ComputerMouse}\,\,
y

ではもう一度聞いてみましょう。\faVolumeUp\,(53\,sec)

c, a, n, d, yという単語になりました。

candyという単語になりましたね。

candyは「あめ、キャンディ」ですよね。
ですから、4つの図の中から選ぶのなら(c)が正解。

それではcandyの発音練習をしましょう。
カタカナ読みで
「キャンディ」と平坦にいうのではなく、
先頭のaを強く発音します。

aは、
エとアの中間の音です。
エの口の形でアといってみてください。
両ほほを左右に引っ張り、口を横にあけて「ア」と発音する。
catの発音と同じです。

あ、どこかでやったと思う人はいますか。

おとといの授業でanimalという単語を勉強しましたね4.
animalの発音と同じ音です。

それでは発音練習です。

\faVolumeUp\,(38\,sec)

きょうのクイズ、発音練習でした。
%%%%%%%%%%%%%%%%%%%%%%%%%%%%%
\section{Diamond}
これからアルファベットを7つ順番に読みあげます。
聞こえたアルファベットを順番に小文字で書いてください。
するとある単語になります。
その意味を表す図を選んでください。
答えはa, b, c, dの記号でお願いします。

それでははじめます。\faVolumeUp\,(1\,min\,06sec)

いかがですか。
わかった人は、チャットで記号を答えてください。


ではもう一度聞いてみましょう。\faVolumeUp\,(1\,min06\,sec)

それでは、聞こえてきたアルファベットを順番に確認しましょう。{\large \ComputerMouse}

{\large \ComputerMouse}

diamondという単語になりました。

diamond.
トランプのマークがありますね。

(b)が正解でした。

ところで日本語では「ダイヤ」のようにアではなくヤということもおおいのですが、
英語の綴りをもてもらうとわかるようにyというアルファベットは使われていませんよね。
ですから英語では「ダイヤ」とか「ダイヤモンド」とはいわないことに注意しましょう。
それでは発音練習です。diamondの発音です。
「ダイアモンド」と平坦にいうのではなく、
「ダイ」にアクセントを置いて発音しましょう。

ここでdiamondに含まれる「アイ」の発音に注目してください。
「アイ」と平坦にいうのではなく「アーイ」」となります。
そのことに注意して尾根性を聞いてください。
ではやってみましょう。
\faVolumeUp\,(38\,sec)

きょうのクイズ、発音練習でした。

{\large \ComputerMouse}

あ、もしかするとdiamondと聞いたときに、みなさんのアタマにまっさきにうかぶのはこれかもしれないですね。
ただ、きょうはトランプのマークだったので、とまどった人がいるかもしれませんね。
%%%%%%%%%%%%%%%%%%%%%%%%%5
\section{Eight}
これからアルファベットを5つ順番に読みあげます。
聞こえたアルファベットを順番に小文字で書いてください。
するとある単語になります。
その意味を表すものを選んでください。
答えはa, b, c, dの記号でお願いします。

それでははじめます。\marginnote{\tiny \faVolumeUp\,(54\,sec)}

いかがですか。
わかった人は、チャットで記号を答えてください。

それでは、聞こえてきたアルファベットを順番に確認しましょう。{\large \ComputerMouse}

e--i--g--h--t

5つのアルファベットを確認したところで、
もう一度聞いてみましょう。\marginnote{\tiny \faVolumeUp\,54\,sec}


{\large \ComputerMouse}

e--i--g--h--tという綴になりました。この単語はどう発音しますか。

そうです。「エイト」ですね。

「8」の意味です。

では選択肢のうちどれが世界でしょうか。

\begin{itemize}
 \item[(a)] サイコロの4と6の目がでています。あわせると、これは「10」ですね
 \item[(b)] これは、みてのとおり「9」です
 \item[(d)] (c)の前に(d)をみてみましょう。\\
「ダイア」の7ですね。前回学習した単語diamondを覚えていますか。平坦に「ダイアモンド」というのではなく、「{\gtfamily ダ}イアモンド」と発音することも確認しておきましょう
 \item[(c)] 数式が出ています。計算してみましょう。\\
14から2を弾いて12。この12に3をかけて36と思った人はいないでしょうね。それでは答がなくなってしまいます。かけ算やわり算をたし算やひき算より先に計算するのが約束でしたね。ですから、この問題では、14から2を引くのではなく、まず掛け算である$2\times{3}=6$を先に計算して、$14-6=8$とするのでしたね。8がでてきました。ではこれが正解です。
\end{itemize}

みなさん、いかがでしたか。

それでは、きょうの単語eightの発音練習です。

エとイではなく、連続するエ~ィという音です。

流れるように1つの音として読むことを意識してください。

また、この単語はもうひとつ注意してほしいことがあります。

綴のなかでg--hという2文字はまったく発音に関係ないことです。

このあたりがやっかいなところですが、そういうものだとおもって練習しましょう。


そのことに注意して音声を聞いてください。
ではやってみましょう。
\marginnote{\tiny \faVolumeUp\,(37\,sec)}

発音練習していただきましたが、書くときには、
発音しないけれど、きちんとghを書くことに注意しましょう。


きょうのクイズ、発音練習でした。
%%%%%%%%%%%%%%%%%%%%%%%%%%%%%%
\section{Fish}

これからアルファベットを4つ順番に読みあげます。
聞こえたアルファベットを順番に小文字で書いてください。
するとある単語になります。
その意味を表す図を選んでください。
答えはa, b, c, dの記号でお願いします。

それでははじめます。\faVolumeUp\,(48\,sec)

いかがですか。
わかった人は、チャットで記号を答えてください。

それでは、聞こえてきたアルファベットを順番に確認しましょう。{\large \ComputerMouse}

f\,\,
{\large \ComputerMouse}\,\,
i\,\,
{\large \ComputerMouse}\,\,
s\,\,
{\large \ComputerMouse}\,\,
h

ではもう一度聞いてみましょう。\faVolumeUp\,(48\,sec)

fishという単語になりました。

それではfishの発音練習をしましょう。
カタカナ読みで
「フィッシュ」と平坦にいうのではなく、
iを強く発音します。

それでは発音練習です。

\faVolumeUp\,(38\,sec)

きょうのクイズ、発音練習でした。
%%%%%%%%%%%%%%%%%%%%%%%%%%%%%%
\section{Ghost}

これからアルファベットを5つ順番に読みあげます。聞こえたアルファベッ
トを順番に小文字で書いてください。
うまく聞き取れると、ある単語になります。
その意味を表すものを選んでください

答えはa, b, c, dの記号でお願いします。

それでははじめます。\faVolumeUp\,(54\,sec)

いかがですか。
わかった人は、チャットで記号を答えてください。

それでは、聞こえてきたアルファベットを順番に確認しましょう。{\large \ComputerMouse}

g\,\,
{\large \ComputerMouse}\,\,
h\,\,
{\large \ComputerMouse}\,\,
o\,\,「オー」とのばすのではなく「オウ」オからウになめらかに変化させる気持ちです。
{\large \ComputerMouse}\,\,
s
\myMouse\,\,
t

ではもう一度聞いてみましょう。\marginnote{\faVolumeUp\,(54\,sec)}


ghostという単語になりました。

ghostは幽霊です。

aはアヒル。

cはホウキにのっています。魔女です。

dは星。

正解はbです。


発音は「ゴウスト」/\textipa{g\'oUst}/

「オー」と伸ばすのではありません。
「オ」から「ウ」になめらかに変化させる気持ちです。
最後は「ト」ではなくt.

それでは発音練習してみましょう。
\marginnote{\faVolumeUp\,(38\,sec)}

きょうの単語はghostでした。






\end{document}
