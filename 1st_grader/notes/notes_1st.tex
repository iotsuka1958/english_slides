\documentclass[12pt]{jlreq}
%%%%%%%%%%%%%%%%%%%%%%%%%%%%
%% 欧文TTF/OTFフォントを利用するにはfontspec.styをロードする必要あり
%% 和文TTF/OTFフォントを利用するにはluatexja-fontspec.styをロードする必要あり
%% luatexja-fontspec.styはfontspec.styをないぶてきにロードする
%% lualatex-ja-preset.sty は luatexja-fontspec.styをロードする
%% つまり次の1行でluatexja-fontspec.sty, fontspec.styも自動的にロードされる
\usepackage[no-math,deluxe,expert,haranoaji]{luatexja-preset}
%%%%
\usepackage{graphicx}
\usepackage{xcolor}
\usepackage{pxrubrica}
\usepackage[default]{fontsetup}
%%%% tabular環境の改良版
\usepackage{tabularray}
\UseTblrLibrary{booktabs}
%%%% ハイパーリンク
%%%% hyperref.sty は preamble の最後で読み込む
\usepackage{hyperref}
\usepackage{xurl}
\hypersetup{
  bookmarks=true,
  bookmarksnumbered=true,
  pdfauthor={iotsuka1958}
}
%%%%%%%%%%%%%%%%%%%%%%%%%%%%%
\usepackage{tikz}
\usetikzlibrary{arrows}
\usepackage{tcolorbox}
%%%%%%%%%%%%%%%%%%%%%%%%%%%%%
\usepackage{luatexja-otf}
\ltjsetparameter{jacharrange={-2}}
%%%%%%%%%%%%%%%%%%%%%%%%%%%%%
\usepackage{array}
%%%%%%%%%%%%%%%%%%%%%%%%%%%%
\usepackage{fontawesome}
\usepackage{pifont}
\usepackage{marvosym}
%%%%%%%%%%%%%%%%%%%%%%%%%%%%
% my_check 環境の定義
\usepackage{amsfonts}
% my_check 環境の定義
\newenvironment{my_check}
  {\begin{itemize}
    \renewcommand\labelitemi{$\square\hspace{0.5em}$}} % 間隔を0.5emに設定
  {\end{itemize}}
%%%%%%%%%%%%%%%%%%%%%%
% カスタム列指定子を定義
\newcolumntype{C}[1]{>{\centering\arraybackslash}m{#1}}
\newcolumntype{L}[1]{>{\raggedright\arraybackslash}m{#1}}
%%%%%%%%%%%%%%%%%%%%%%%%%%%%%
\begin{document}
%%%%%%%%%%%%%%%%%%%%%%%%%%%%%
\thispagestyle{empty}
\title{EduOp Chiba English(1年生)}
\author{iotsuka}
\maketitle

\begin{tcolorbox}[title=いちばんはじめに]
\begin{itemize}
 \item usbメモリ:\url{/music.mp3}をMediaPlayerで開いて連続再生
 \item usbメモリ:\url{/english_slidesfiles/1st_grade/001_alphabet.pdf}をadobe readerで開いてzoomで共有(サウンドを共有にチェック)
 \item usbメモリ:\url{/english_slides/1st_grade/video/001_alphabet.mp4}をchromeのタブで開いておく
\end{itemize}
\end{tcolorbox}


\begin{tcolorbox}[title=授業の直前に]
\begin{itemize}
 \item 流している音楽music.mp3を止める
 \item zoomのマイク、カメラをON
\end{itemize}
\end{tcolorbox}

\tableofcontents
%%%%%%%%%%%%%%%%%%%%%%%
\newpage\setcounter{page}{1}
\section{alphabet}
\subsection{でだし}
みなさん、こんにちは。

エデュオプちばの英語の授業にようこそ。

先週は、オリエンテーション、本格的な授業に入る前の準備期間ということで
\begin{enumerate}
 \item 安心して学ぶ場にするために
 \item 授業の進め方や内容など
 \item 英語ってどんなことばなの
\end{enumerate}
について、お話しました。

きょうは3時間目です。
いよいよ本格的な授業にはいります。

とはいっても、なにも緊張する必要はありません。
リラックスして参加してください。

さて、
きょうの予定はごらんのとおりです。

アルファベットについて学習します。

もう知っているという人が多いと思うのですが、
聞き取りや発音の練習をまじえて進めていきます。

それでははじめます。

\begin{tcolorbox}[title=zoomの管理ボタンで]
\begin{itemize}
 \item カメラOFF
\end{itemize}
\end{tcolorbox}


\subsection{アルファベットとは}

さて、アルファベットという言葉は聞いたことがあるでしょう。

では、あらたまって、
「アルファベットとはなんでしょうか?」と聞かれたら、なんて答えればいいでしょうか。

英語のアルファベットは、英語を書くために使用される文字のセットです。

26種類あります。それぞれについて大文字と小文字があります。

\subsection{英語のアルファベットの文字数}

英語のアルファベットは、26種類で
大文字と小文字を合わせると、$26\times{}2$あるいは$26+26$どちらでも同じことですが、
合計52文字になりますね。

\subsubsection{日本語との比較}

ここで日本語について考えてみましょう。

ひらがなとカタカナ、それぞれ46文字ですから、かなだけでも$46\times{}2=92\text{文字}$ですから、
かなだけで英語のアルファベットよりも多いですね。

さらに日本語の場合はひらがなカタカナだけではありませんね。
漢字があります。
漢字の数は、まさに膨大です。ある漢和辞典には5万字が収録されているということですが、
まあ、そこまではいいませんが、わたしたちが日常的に使っている常用漢字は2,136文字です。

そうするとひらがな、カタカナに常用漢字を足し算すると
2,228文字となります。

英語のアルファベットの文字数は、日本の文字体系と比較するとかなり少ないということがわかります。


(スライドを進める。$English\colon{}52<2,228\colon{}Japanese$)。

\subsection{52 $<$ 2,228}
52文字をしっかりと身につけることで、英語の世界に飛び込んでいく準備がととのいます。

\subsection{アルファベット}

スライド再掲

\subsection{大文字と小文字の使い分け}

ところで大文字と小文字はどう使い分ければいいでしょう。

わたしたちのあいことばEnglish is fun.
「英語はおもしろい」ですが、
この英文で大文字は先頭のE.

約束の1です。「文の先頭は大文字」です。

My name is George Smiley.
わたしの名前はジョージ・スマイリーです。
My nameは「私の名前」、George Smileyは人の名前です。
先頭のMが大文字なのはいまいったとおり、文の先頭だからです。
GとSが大文字になっているのはなぜでしょう。

約束の2。
人の名前は大文字で始めるのが約束です。
みなさんも名前を英語で書くときは大文字で始めることになります。

Can I help you?は
「お手伝いしましょうか?」や「何かご用でしょうか?」くらいの意味。
細かなことはおいおい勉強するので、きょうはそういう決まり文句だとおもってくれればけっこうです。
ここでも先頭のCが大文字。文の先頭だからですね。
もうひとつ大文字がありますね。
Iが大文字です。
約束の3。「わたしは」の意味のIは、いつなんどきであっても大文字で書きます。




\subsection{mp4視聴}
では、動画を視聴してもらいます。
なお、とちゅうで英語が聞き取れない、意味がわからないと不安になる人がいるかもしれません。
でも心配する必要はありません。画面を見ながら、それぞれのアルファベットの発音に耳を傾けてください。
とちゅうで、オリエンテイションで話した
Please listen carefully.(よく聞いてください)というフレーズがでてきます。
では、注意して聞いてください。

\begin{tcolorbox}[title=zoomの管理ボタンで]
\begin{itemize}
 \item mp4のchromeタブを共有し、全画面にしてスタート。6\,min47\,sec)
 \item (動画が終わったら)pdfの画面を共有(サウンド共有になっていることをチェック)する
\end{itemize}
 
\end{tcolorbox}



\subsection{大文字}

まず、大文字です。

これからアルファベットの音に耳をよく傾けてもらい、
そのあと実際に発音してもらいます。

Please listen carefully.とPlease repeat after me.は覚えていますか。

それでは始めます。

(mp3を流す。2\,min42\,sec)

では、続いて小文字です。

\subsection{小文字}

さきほどと同じ要領で練習しましょう。


(mp3を流す。2\,min42\,sec)


\subsection{まとめ}

\begin{itemize}
 \item アルファベットは26種類です
 \item 大文字と小文字があります
       \begin{itemize}
	\item 大文字小文字を区別できるようになろう
        \item 大文字と小文字を書けるようになろう
       \end{itemize}
 \item 聞き取れるようになったら実際に発音してみよう
\end{itemize}

\subsection{アルファベットを書いてみよう}

では実際にアルファベットを書く練習をしましょう。
練習が重要です!


ではノートにアルファベットを書いてみましょう。まず大文字です。

\subsection{小文字を書く}

次は小文字を書く練習をしましょう。

あ、でもその前に
ここでとくにまぎらわしいものをいくつかあげますので、みなさんも注意するようにしてください。

\begin{itemize}
 \item  b/d(左右)
 \item  p/q (左右)
 \item h/n(縦棒の長さ)
 \item i/j(jは下に伸びて軽くカーブ)
 \item u/v(先端が丸みを帯びているかとがっているか)
\end{itemize}


\paragraph{bとd}

これはまちがいやすいところです。
どちらも輪っかの脇に縦に棒が伸びています。

左に棒があるのがb、
右に棒があるのがd.

棒は輪っかより上に伸びていますね。

\paragraph{pとq}
これもまちがいやすいです。


左に棒があるのがp、
右に棒があるのがq

棒は輪っかより下に伸びていますね。
\paragraph{pb,d,p,q}

ここでb, d, p,qを並べてみましょう。

棒があるのが輪っかの左右かどうか、
棒が伸びているのは上か下かによって、
4通りの文字があります。

まぎらわしいのですが、ここはきちんと区別できるようにしておくことがたいせつです。

書くときには左右と上下をしっかり意識しましょう。

\paragraph{h,n}

これは左の棒が長く上に突き出しているのがh。nは上に突き出していませんよね。
ここも書くときはhの左の棒を意識して上に突き出させるようにするといいでしょう。

\paragraph{iとj}
これもまぎらわしいですよね。

どちらも上に点があって、その下に棒があります。ちがいはイはまっすぐなのに、jは下の付き出すとともに丸みを帯びている、カーブしています。

jを書くときには、
意識して棒を下にお突き出させるとともにカーブさせるようにしましょう。

\paragraph{u,v}

あ、これもまぎらわしいですね。

ちがいはuは丸みを帯びているところ、
Vは尖っています。

書くときは、
uは丸みを、
Vはシャープさを意識しておくといいでしょう。

\subsection{ノートにアルファベット小文字を書く}

5分くらい?

\subsection{quiz}

これからアルファベットを読みあげます。

聞こえたアルファベットを順番に書いてください。

ぜんぶで10文字です。

音声は2回繰り返します。

(mp3を流す。1\,min34\,sec)

いかがでしたか。それでは答え合わせです。

EDUOP CHIBA

(時間が余っていたらもう1回mp3を流すことで調整)
\subsection{終わり}

それではきょうはこれで終わります。
%%%%%%%%%%%%%%%%%%%%%
\newpage
\section{主語と動詞}
\begin{my_check}
\item マイクON
\item カメラON
\end{my_check}

{\LARGE \ComputerMouse}

みなさん、こんにちは。
エデュオプちばの英語の時間にようこそ。
担当の大塚です。

先週まではZOOMで授業を配信していましたが、
今週からはGoogle Classroomを利用しMeetでお届けしています。

きのうからMeetでの配信がはじまっていますが、
英語の授業をMeetで配信するのは初めてです。
あの、音声はちゃんと届いていますか。
チャットでお答えくださるとありがたいです。

あ、だいじょうぶそうですね。
とちゅうでなにかトラブルがあったら、チャットで教えてください。
それでははじめます。

リラックスして参加してください。

さて、きょうは授業に入る前に、
クイズをやってもらいます。

{\large \ComputerMouse}

%%%%%%%%%%%%%%%%%%%%
\newpage
\subsection{Quiz}

これからアルファベットを6つ順番に読みあげます。
聞こえたアルファベットを順番に小文字で書いてください。
するとある単語になります。
その意味を表す図を選んでください。
答えはa, b, c, dの記号でお願いします。


それでははじめます。\faVolumeUp\,(1\,min)

いかがですか。
わかった人は、チャットで記号を答えてください。


ではもう一度聞いてみましょう。\faVolumeUp\,(1\,min)

それでは、聞こえてきたアルファベットを順番に確認しましょう。{\large \ComputerMouse}

a
{\large \ComputerMouse}
n
{\large \ComputerMouse}
i
{\large \ComputerMouse}
m
{\large \ComputerMouse}
a
{\large \ComputerMouse}
l

animalという単語になりました。

animalは「動物」です。
ですから、4つの図の中から選ぶのなら(a)が正解。

それではanimalの発音練習をしましょう。
カタカナ読みで
「アニマル」と平坦にいうのではなく、
先頭のaを強く発音します。

先頭のaは、
エとアの中間の音です。
エの口の形でアといってみてください。
catの発音と同じです。
エニモゥ。

それでは発音練習です。
\faVolumeUp\,(38\,sec)

きょうのクイズ、発音練習でした。

{\large \ComputerMouse}

%%%%%%%%%%%%%%%%%%%%%%%%%%%
\newpage
\subsection{授業の流れ}

これまでをふりかえると、
前回はアルファベットの勉強をしました。
きょうは、「主語と動詞」についてお話をします。
それでは、これからの授業の流れです。

{\large \ComputerMouse}

前回の授業の復習をかんたんにしてから、
主語と動詞について勉強していきます。

{\large \ComputerMouse}

\subsection{復習}

それでは復習です。

{\large \ComputerMouse}

(アルファベットの復習はアドリブで)

{\large \ComputerMouse}

\subsection{主語とは、動詞とは}
それでは、いよいよきょうの内容にはいりましょう。
きょうのテーマは「主語と動詞です」

皆さんは主語と動詞ということばを聞いたことがありますか。

はじめて聞く人もいるかもしれませんね。

{\large \ComputerMouse}

\subsection{ネコがミルクを飲む}

それでは、この写真を見てください。

この写真では、誰が何をしていますか。

そうです。

ネコがミルクを飲んでいます。

\paragraph{主語とは}
「誰が」に相当する部分のことを「主語」といいます。

ですから、この場合、
主語は「ねこ」です。

\paragraph{動詞}
何をしているかを表すのが「動詞」です。

この場合、ねこはミルクを飲んでいるのですから、
「飲んでいる」が動詞に当たります。

いま、ねこがミルクを飲んでいる写真を見てもらいました。

{\large \ComputerMouse}

\subsection{主語とは、動詞とは}


一般的に言うと、
「だれだれは〜する」という文でいうと、
「だれだれ」に相当する部分を主語というのです。

「何をしているのか」を表す部分が動詞です。

ここでは、主語に相当する部分を、べつに何色でもいいのですが、
わたしは主語を赤い枠でくくくってみます。
そして動詞を青い色でくくってみます。

{\large \ComputerMouse}

それでは4つ例文を出しました。
それぞれの文の主語と動詞はなにか考えてみてください。

さあ、ここで時間を取りますから、4つの例文をノートに移して、
主語と動詞を枠でくくってみてください。

作業が終わった人はリアクションお願いします。

\hrulefill

\hfill{}作業\hfill

\hrulefill


いかがでしょうか。
答え合わせをしましょう。

{\large \ComputerMouse}

(しばらくアドリブ)

むずかしいことはありませんよね。
これでみなさんは、主語と動詞について基本的なことを理解したことになります。
大きく一歩前進です。

ところで、先に行く前に動詞の部分に注目してください。
スライドで言えば青い枠でくくったところです。

「話す」、「食べる」、「行く」、「飲む」この4つの日本語を、
アルファベットで表記してみましょう。やはり、ノートに書いてみてください。

最初だけわたしがやってみます。

「話す」はアルファベットで表記すると
\ComputerMouse{}hanasuですよね。
では、「食べる」「行く」「飲む」をアルファベットで書いてみてください。

作業が終わったらリアクションをお願いします。

(様子を見ながら)
あ、みなさん終わったでしょうか。

では順番にいきましょう。

hanasu,taberu,
iku,nomu
となりますよね。

なにか気がつくことがありますか。

そうです、アルファベットで表記するとみんな`u'で終わっていますよね。
というわけで、最後のuは色を変えておきました。
これ、日本語の動詞の特徴です。

話す、食べる、行く、飲む、これらはアルファベットで書くとぜんぶ`u'でおわります、音の世界で言えば、語尾を伸ばすとみんあ「う」になりますよね。日本語の動詞の特徴として、アタマに入れておくといいですよ。

話が少しわきにそれました。話をもとにもどします。

「だれだれは〜する」とき、
「だれだれは」を主語、「〜する」の部分を動詞ということを、
最初は「ネコがミルクを飲んでいる写真」で確認し、次に日本語の文で確認しました。

さて、おぜんだてはととのいました。
英語の世界に話を移します。

{\large \ComputerMouse}
%%%%%%%%%%%%%%%
\subsection{だれだれは〜する}
では、まず

1. I have a car.

ノートに写してください(ちょっと時間を取る)

つぎに、
あ、これどういう意味ですか。
意味がわかる人は、わきに日本語で意味を書いてください。
わからない人は気にする必要はありません。これから説明しますから。

いかがですか。
確認しましょう。{\large \ComputerMouse}「わたしは車を持っている」という意味です。別に同じことですから「所有している」でもいいですね。
わからない人はまだ気にする必要はありません。
わからないから勉強しているのですから。


つぎに移ります。{\large \ComputerMouse}

We  go to work by bus.

まずノートに写しましょう。

あ、これも、
意味がわかる人は、わきに日本語で意味を書いてください。

{\large \ComputerMouse}確認しましょう。「われわれはバスで職場に行く」という意味です。別に同じことですから「バスで通勤している」でもいいですね。

つぎに移ります。

They speak English and Japanese.
これまでと同じように、まずノートに写して、
意味がわかる人は、わきに日本語で意味を書いてください。

確認しましょう。「彼らは英語と日本語を話す」という意味ですね。わたしたちもこうなるといいですよね。

つぎに移ります。

I drink coffee every morning.
まずノートに写して、
意味がわかる人は、わきに日本語で意味を書いてください。

確認しましょう。「わたしは毎朝コーヒーを飲みます」という意味です。

つぎに移ります。

They study at the library. 

手順はこれまでと同じですよ。
英文をノートに写して、わかる人は意味も書きましょう。

「彼らは図書館で勉強します」という意味です。

つぎに移ります。

We eat bread for breakfast.

英文をノートに写して、わかる人は意味も書きましょう。

「われわれは朝食にパンを食べる」という意味です。


では、ここで少し時間を取りますから、ノートに書き写すのがまにあわなかった人は、
ノートを整理してください。



(1分ほど取る)

いかがでしょうか。
ノートはとれましたか。

では、ここで音声に耳を傾けてもらい、そのあと実際にみなさんに発音してもらいます。
Please listen carefully.とPlease repeat after me.というフレーズに注意してください。

\faVolumeUp{}(3\,min 01\,sec)

Good job.(よくできました)といってくれましたね。

では、意味を確認した上で実際の音声を聞いていただき、発音の練習をしました。

\subsection{SVの確認}

さて、さきほど「主語と動詞」についてお話しました。

いま、スライドには6つの英文が示されていますね。

これらの英文の主語と動詞を確認することにしましょう。

ここで時間を取りますから、
6つの英文と主語と動詞がなにか考えてください。

さきほどノートに写してもらった英文を利用して、
主語と動詞を異なる色で囲んでください。
区別がつけば何色でもいいのですが、わたしは主語を赤で、動詞を青で囲むことにします。
区別がつけば何色でもいいです。枠の形も、四角でも円でも楕円でもなんでもいいですよ。

終わった人はリアクションをお願いします。

いかがですか。
それでは答え合わせです。

No.1から。
I have a car.

(ここからアドリブでNo.6まで)

いかがでしたか。

かんたんすぎてひょうしぬけした人もおおかったかもしれませんが、
なんでこんなにかんたんとおもえることに時間をかけて学習したのか、
やがてわかってもらえるとおもいます。

Topic for Todayは「きょうのテーマ」「本日の学習内容」「きょうの授業ポイント」
くらいの意味です。

「英文の骨格は主語と動詞です」としてまとめておきます。

このTopic for Todayもノートに写しましょう。

\subsection{単語の確認}

それでは、6つの例文について、
単語を確認していきましょう。

have
(しばらくアドリブ)

では、単語や表現の音をよく聞いてもらい、そのあと実際に発音してもらいます。
(5\,min41\,sec)


\subsection{主語と動詞の順番}

「わたしは英語を話します」という日本語は、わざわざ口に出していうかという問題はさておき、正しい日本語で、意味もまぎれることなく伝わります。

「英語をわたしは話します」はどうでしょうか。
まあこれでも伝わりますよね。

ではこれはどうでしょうか。
「英語を話します、わたしは」
これも伝わるでしょう。

けっきょく、日本語は主語と動詞の順番はかなり自由度が高いということになります。
順番は適当でも日本語としては伝わります。
もちろん、場面場面でたしょう不自然ということはあるかもしれませんが、
コミュニケーションの妨げになるようなことはないということです。


それではこんどはどうでしょうか。
「英語を話します」

この文では「わたしは」が消えてしまいました。
きょう勉強した言葉を用いていうと、主語が消えてしまっています。

でも、
たとえば会話の中で「あなたは日本語以外の言葉を話しますか」と聞かれた人が、
「英語を話します」と答えるのはきわめて自然な会話ですよね。
そうです、日本語は主語と動詞の順番が適当でも通じることがおおいですし、また前後関係によっては主語をいちいち表現しないでもすむのです。

いっぽう、英語の場合はどうでしょうか。

英語の場合、「わたしは英語を話します」を表すのは
I speak English.
これだけです。

ここにあげたようなのはぜんぶ英語としては失格なんですねえ。
(アドリブ)

きょうのポイントに追加します。
\begin{itemize}
 \item   英文の骨格は主語と動詞です
 \item   英語は語順がだいじです
 \item   英文にはかならず主語が必要
\end{itemize}

%%%%%%%%%%%%%%%%%%%%%%%%%%%%%%%%%%%%%
\subsection{Exercises}

では、ここでExercises(練習問題)です。

日本語を参考にして、それぞれの英文の主語と動詞を指摘してください。

\subsection{exercises}

では、理解度をみるための練習問題です。

(アドリブ)

それではみなさんさようなら。
また、次回お会いするのを楽しみにしています。

%%%%%%%%%%%%%%%%%%%%%%%%%%%%%%%%%%%%%%%%%%%%%%%%%%%%%%%%%%%%%%
\newpage

\section{be動詞}

\begin{my_check}
\item マイクON
\item カメラON
\end{my_check}

{\LARGE \ComputerMouse}

みなさん、こんにちは。
エデュオプちばの英語の授業にようこそ。

先週まではZOOMで授業を配信していましたが、
今週からはGoogle Classroomを利用しMeetでお届けしています。

きのうからMeetでの配信がはじまっていますが、
英語の授業をMeetで配信するのは初めてです。
あの、」音声はちゃんと届いていますか。
チャットでお答えくださるとありがたいです。

あ、だいじょうぶそうですね。
とちゅうでなにかトラブルがあったら、チャットで教えてください。
それでははじめます。

リラックスして参加してください。


さて、きょうは授業に入る前に、
クイズをやってもらいます。

{\large \ComputerMouse}


%%%%%%%%%%%%%%%%%%%%
\newpage
\subsection{Quiz}

これからアルファベットを4つ順番に読みあげます。
聞こえたアルファベットを順番に小文字で書いてください。
するとある単語になります。
その意味を表す図を選んでください。
答えはa, b, c, dの記号でお願いします。


それでははじめます。\faVolumeUp\,(48\,sec)

いかがですか。
わかった人は、チャットで記号を答えてください。


ではもう一度聞いてみましょう。\faVolumeUp\,(48\,sec)

それでは、聞こえてきたアルファベットを順番に確認しましょう。{\large \ComputerMouse}

b
{\large \ComputerMouse}
o
{\large \ComputerMouse}
o
{\large \ComputerMouse}
k

bookという単語になりました。

bookは「本、書物、書籍」ですよね。
ですから、4つの図の中から選ぶのなら(d)が正解。

それでは発音練習です。bookの発音です。
「ブック」と平坦にいうのではなく、
先頭のbという音に続く「ウ」という音は、
おちょぼ口で前に出す感じで発音するとうまく発音できますよ。
ではやってみましょう。
\faVolumeUp\,(37\,sec)

きょうのクイズ、発音練習でした。

{\large \ComputerMouse}

\subsection{授業の流れ}

それでは、これからの授業の流れです。

{\large \ComputerMouse}

前回の授業の復習をかんたんにしてから、
be動詞について順番に勉強していきます。

(2,3年生にとってはbe動詞の基本はすでに勉強済みだとはおもいますが、これからの授業を進めるに立ってここできちんとおさらいしておきましょう)

{\large \ComputerMouse}

%%%%%%%%%%%%%%%%%%%%%%%%%%%%%%%
\newpage
\subsection{復習}

前回は主語と動詞について勉強しましたね。

具体的には

{\large \ComputerMouse}

\begin{itemize}
 \item   英文の骨格は主語と動詞です
 \item   英語は語順がだいじです
 \item   英文にはかならず主語が必要だ
\end{itemize}
ということを勉強しましたね。

{\large \ComputerMouse}

\subsection{Exercises}

それでは確認のための練習問題です。

ぜんぶで8つの英文があります。
ここで時間をとりますから、
英文をノートに写して、
日本語を参考にしながら、それぞれの英文の主語と動詞を指摘してください。

終わった人は、リアクションで教えてください。

\hrulefill{}

\hfill{}作業\hfill{}

\hrulefill

(アドリブで答え合わせ)

\faVolumeUp\,(3\,min34\,sec)

{\large \ComputerMouse}

%%%%%%%%%%%%%%%%%%%%%%%%%%%%%%%%%%%%
\newpage
\subsection{be動詞とは}
それでは、新しい単元にはいりましょう。

「be動詞」とはなにかということを勉強します。

{\large \ComputerMouse}

では、まずこの6つの英文を見てください。

ここで時間をとるので、
英文をノートに写してください。
意味がわかる人は意味も書き加えましょう。

作業が終わった人は、リアクションで教えてください。

\hrulefill{}

\hfill{}作業\hfill{}

\hrulefill

では、順番に意味を確認していきましょう。

「わたしは生徒です」という意味の英文です。
(以下、順番に)

\begin{enumerate}
 \item I am a student.
 \item You are my friend.
 \item He is tall.
 \item She is kind.
 \item The sky is blue.
 \item They are my classmates.
\end{enumerate}

(意味の確認が終わったら)

では1. I am a student.に戻ります。
前回、英語の骨格は「主語と動詞」だといいました。
この文の主語はI、動詞は赤で示したamです。
この文では「I(私)$=$a student(生徒)」の関係であることに注意しましょう。
このようにamは特別な動詞で、イコールの意味を表します。

次の文はどうでしょう。
You are my friend.
動詞は赤で示したare。
このareという動詞も、You(あなた)$=$my friend(私の友人)ということを示しています。
つまりイコールの意味ですね。

次の
 He is tall.
動詞は赤で示したis。
やはりHe(彼)$=$tall(背が高い)ということを示しています。

次の
She is kind.
Shek(彼女)$=$kind(親切)というイコールの関係です。

次は
The sky is blue.
The sky(空)$=$blue(青い)というイコールの関係です。

They are my classmates.
They(彼ら)$=$my classmates(わたしのクラスメート)、
やhりイコールの関係です。

ここまでを振り返りますね。
ここで赤で示したam, are, is はぜんぶイコールの意味です。

この3つの動詞am, are, isをまとめてbe動詞といいます。
なぜbe動詞というのか不思議に思う人もいるとおもいますが、
しばらくのあいだは、そういうものだとおもっておいてください。

\paragraph{単語と表現} \mbox{}

(アドリブ)


\faVolumeUp\,(2\,min45\,sec)

%%%%%%%%%%%%%%%%%%%%%%%%%%%%%%%%%
\subsection{図}
am, are, is---これはみんななかまです。

まとめてbe動詞といいます。
%%%%%%%%%%%%%%%%%%%%%%%%%%%%%%%%%
\subsection{be動詞、どれ使う}

では、つぎにbe動詞の使い分けについて勉強します。

\end{document}

