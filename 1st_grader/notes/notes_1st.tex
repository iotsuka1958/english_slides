\DocumentMetadata{lang=ja-JP}
\documentclass[book,jafontscale=0.9247]{jlreq}
%%%%%%%%%%%%%%%%%%%%%%%%%%%%
%% 欧文TTF/OTFフォントを利用するにはfontspec.styをロードする必要あり
%% 和文TTF/OTFフォントを利用するにはluatexja-fontspec.styをロードする必要あり
%% luatexja-fontspec.styはfontspec.styをないぶてきにロードする
%% lualatex-ja-preset.sty は luatexja-fontspec.styをロードする
%% つまり次の1行でluatexja-fontspec.sty, fontspec.styも自動的にロードされる
\usepackage[no-math,deluxe,expert,haranoaji]{luatexja-preset}
%%%%
\usepackage{graphicx}
\usepackage{xcolor}
\usepackage{pxrubrica}
\usepackage[default]{fontsetup}
\let\textipa\relax
\usepackage{tipa, tipx}
%%%% tabular環境の改良版
\usepackage{tabularray}
\UseTblrLibrary{booktabs}
%%%% ハイパーリンク
%%%% hyperref.sty は preamble の最後で読み込む
\usepackage{hyperref}
\usepackage{xurl}
\hypersetup{
  bookmarks=true,
  bookmarksnumbered=true,
  pdfauthor={iotsuka1958}
}
%%%%%%%%%%%%%%%%%%%%%%%%%%%%%
\usepackage{tikz}
\usetikzlibrary{arrows}
\usepackage{tcolorbox}

%%%%%%%%%%%%%%%%%%%%%%%%%%%%%
\usepackage{luatexja-otf}
\ltjsetparameter{jacharrange={-2}}
%%%%%%%%%%%%%%%%%%%%%%%%%%%%%
\usepackage{array}
\usepackage{cases}
\usepackage{marginnote}
\usepackage{manfnt}
%%%%%%%%%%%%%%%%%%%%%%%%%%%%
\usepackage{fontawesome}
\usepackage{pifont}
\usepackage{marvosym}
%%%%%%%%%%%%%%%%%%%%%%%%%%%%
% my_check 環境の定義
\usepackage{amsfonts}
% my_check 環境の定義
\newenvironment{my_check}
  {\begin{itemize}
    \renewcommand\labelitemi{$\square\hspace{0.5em}$}} % 間隔を0.5emに設定
  {\end{itemize}}
%%%%%%%%%%%%%%%%%%%%%%
% カスタム列指定子を定義
\newcolumntype{C}[1]{>{\centering\arraybackslash}m{#1}}
\newcolumntype{L}[1]{>{\raggedright\arraybackslash}m{#1}}
%%%%%%%%%%%%%%%%%%%%%%
%% 生徒に作業を指示するときのコマンド
%%%%%%%%%%%%%%%%%%%%%%
\newcommand{\mySagyo}{%
\par%
\bigskip
では、ここで時間をとるので作業をしてください。
作業が終わったあ人はリアクションで教えてください\par%
\begin{minipage}[t]{.98\textwidth}
\mbox{}\hrulefill\mbox{}\par%
\mbox{}\hfill{}\raisebox{-.5\height}{作業}\hfill\mbox{}\par%
\mbox{}\hrulefill\mbox{}
\end{minipage}%
\par%
\bigskip%
では、まだ作業の途中の人もいるかもしれませんが、いったん筆をとめて
スライドに注目してください。%
\par%
\bigskip
}
%%%%%%%%%%%%%%%%%%%%%%%%%%
%% 授業のはじめのルーティーン
%%%%%%%%%%%%%%%5%%%%%%%%%
\newcommand{\myStartLesson}{%
\vspace*{5pt}%
\noindent{}{\Large\gtfamily 授業スタート!}
\begin{my_check}
\item マイクON
\item カメラON\hspace{40pt}{\LARGE \ComputerMouse}
\end{my_check}
みなさん、こんにちは。
エデュオプちばの英語の授業にようこそ。\par
さて
音声は届いていますか?
なにかトラブルがあったら、チャットで教えてください。
どうぞよろしくお願いします。\par
きょうも暑いですね。
体調管理にはじゅうぶん気をつけてください。
みなさんも熱中症にならないように、
適切な水分補給をお願いします。
授業中でも差し支えありません。
遠慮なく水分を補給しながら、
リラックスして参加してください。\par
それでは授業にはいります。
\begin{my_check}
\item カメラOFF\hspace{40pt}{\LARGE \ComputerMouse}
\end{my_check}
}
%%%%%%%%%%%%%%%%%%%%%%%%%%%%
%% マウスのアイコン
%%%%%%%%%%%%%%%%%%%%%%%%%%%
\newcommand{\myMouse}{%
  {\large \ComputerMouse}
}
%%%%%%%%%%%%%%%%%%%%%%%%%%%%%
\begin{document}
%%%%%%%%%%%%%%%%%%%%%%%%%%%%%
\thispagestyle{empty}
\title{EduOp Chiba English(1年生)}
\author{iotsuka}
\maketitle

\begin{tcolorbox}[title=いちばんはじめに]
\begin{itemize}
 \item usbメモリ:\url{/music.mp3}をMediaPlayerで開いて連続再生
 \item usbメモリ:\url{/english_slidesfiles/1st_grade/001_alphabet.pdf}をadobe readerで開いてzoomで共有(サウンドを共有にチェック)
 \item usbメモリ:\url{/english_slides/1st_grade/video/001_alphabet.mp4}をchromeのタブで開いておく
\end{itemize}
\end{tcolorbox}


\begin{tcolorbox}[title=授業の直前に]
\begin{itemize}
 \item 流している音楽music.mp3を止める
 \item zoomのマイク、カメラをON
\end{itemize}
\end{tcolorbox}

\tableofcontents
%%%%%%%%%%%%%%%%%%%%%%%
\newpage\setcounter{page}{1}
\chapter{alphabet}
\section{でだし}
みなさん、こんにちは。

エデュオプちばの英語の授業にようこそ。

先週は、オリエンテーション、本格的な授業に入る前の準備期間ということで
\begin{enumerate}
 \item 安心して学ぶ場にするために
 \item 授業の進め方や内容など
 \item 英語ってどんなことばなの
\end{enumerate}
について、お話しました。

きょうは3時間目です。
いよいよ本格的な授業にはいります。

とはいっても、なにも緊張する必要はありません。
リラックスして参加してください。

さて、
きょうの予定はごらんのとおりです。

アルファベットについて学習します。

もう知っているという人が多いと思うのですが、
聞き取りや発音の練習をまじえて進めていきます。

それでははじめます。

\begin{tcolorbox}[title=zoomの管理ボタンで]
\begin{itemize}
 \item カメラOFF
\end{itemize}
\end{tcolorbox}


\section{アルファベットとは}

さて、アルファベットという言葉は聞いたことがあるでしょう。

では、あらたまって、
「アルファベットとはなんでしょうか?」と聞かれたら、なんて答えればいいでしょうか。

英語のアルファベットは、英語を書くために使用される文字のセットです。

26種類あります。それぞれについて大文字と小文字があります。

\section{英語のアルファベットの文字数}

英語のアルファベットは、26種類で
大文字と小文字を合わせると、$26\times{}2$あるいは$26+26$どちらでも同じことですが、
合計52文字になりますね。

\subsection{日本語との比較}

ここで日本語について考えてみましょう。

ひらがなとカタカナ、それぞれ46文字ですから、かなだけでも$46\times{}2=92\text{文字}$ですから、
かなだけで英語のアルファベットよりも多いですね。

さらに日本語の場合はひらがなカタカナだけではありませんね。
漢字があります。
漢字の数は、まさに膨大です。ある漢和辞典には5万字が収録されているということですが、
まあ、そこまではいいませんが、わたしたちが日常的に使っている常用漢字は2,136文字です。

そうするとひらがな、カタカナに常用漢字を足し算すると
2,228文字となります。

英語のアルファベットの文字数は、日本の文字体系と比較するとかなり少ないということがわかります。


(スライドを進める。$English\colon{}52<2,228\colon{}Japanese$)。

\section{52 $<$ 2,228}
52文字をしっかりと身につけることで、英語の世界に飛び込んでいく準備がととのいます。

\section{アルファベット}

スライド再掲

\section{大文字と小文字の使い分け}

ところで大文字と小文字はどう使い分ければいいでしょう。

わたしたちのあいことばEnglish is fun.
「英語はおもしろい」ですが、
この英文で大文字は先頭のE.

約束の1です。「文の先頭は大文字」です。

My name is George Smiley.
わたしの名前はジョージ・スマイリーです。
My nameは「私の名前」、George Smileyは人の名前です。
先頭のMが大文字なのはいまいったとおり、文の先頭だからです。
GとSが大文字になっているのはなぜでしょう。

約束の2。
人の名前は大文字で始めるのが約束です。
みなさんも名前を英語で書くときは大文字で始めることになります。

Can I help you?は
「お手伝いしましょうか?」や「何かご用でしょうか?」くらいの意味。
細かなことはおいおい勉強するので、きょうはそういう決まり文句だとおもってくれればけっこうです。
ここでも先頭のCが大文字。文の先頭だからですね。
もうひとつ大文字がありますね。
Iが大文字です。
約束の3。「わたしは」の意味のIは、いつなんどきであっても大文字で書きます。




\section{mp4視聴}
では、動画を視聴してもらいます。
なお、とちゅうで英語が聞き取れない、意味がわからないと不安になる人がいるかもしれません。
でも心配する必要はありません。画面を見ながら、それぞれのアルファベットの発音に耳を傾けてください。
とちゅうで、オリエンテイションで話した
Please listen carefully.(よく聞いてください)というフレーズがでてきます。
では、注意して聞いてください。

\begin{tcolorbox}[title=zoomの管理ボタンで]
\begin{itemize}
 \item mp4のchromeタブを共有し、全画面にしてスタート。6\,min47\,sec)
 \item (動画が終わったら)pdfの画面を共有(サウンド共有になっていることをチェック)する
\end{itemize}
 
\end{tcolorbox}



\section{大文字}

まず、大文字です。

これからアルファベットの音に耳をよく傾けてもらい、
そのあと実際に発音してもらいます。

Please listen carefully.とPlease repeat after me.は覚えていますか。

それでは始めます。

(mp3を流す。2\,min42\,sec)

では、続いて小文字です。

\section{小文字}

さきほどと同じ要領で練習しましょう。


(mp3を流す。2\,min42\,sec)


\section{まとめ}

\begin{itemize}
 \item アルファベットは26種類です
 \item 大文字と小文字があります
       \begin{itemize}
	\item 大文字小文字を区別できるようになろう
        \item 大文字と小文字を書けるようになろう
       \end{itemize}
 \item 聞き取れるようになったら実際に発音してみよう
\end{itemize}

\subsection{アルファベットを書いてみよう}

では実際にアルファベットを書く練習をしましょう。
練習が重要です!


ではノートにアルファベットを書いてみましょう。まず大文字です。

\subsection{小文字を書く}

次は小文字を書く練習をしましょう。

あ、でもその前に
ここでとくにまぎらわしいものをいくつかあげますので、みなさんも注意するようにしてください。

\begin{itemize}
 \item  b/d(左右)
 \item  p/q (左右)
 \item h/n(縦棒の長さ)
 \item i/j(jは下に伸びて軽くカーブ)
 \item u/v(先端が丸みを帯びているかとがっているか)
\end{itemize}


\paragraph{bとd}

これはまちがいやすいところです。
どちらも輪っかの脇に縦に棒が伸びています。

左に棒があるのがb、
右に棒があるのがd.

棒は輪っかより上に伸びていますね。

\paragraph{pとq}
これもまちがいやすいです。


左に棒があるのがp、
右に棒があるのがq

棒は輪っかより下に伸びていますね。
\paragraph{pb,d,p,q}

ここでb, d, p,qを並べてみましょう。

棒があるのが輪っかの左右かどうか、
棒が伸びているのは上か下かによって、
4通りの文字があります。

まぎらわしいのですが、ここはきちんと区別できるようにしておくことがたいせつです。

書くときには左右と上下をしっかり意識しましょう。

\paragraph{h,n}

これは左の棒が長く上に突き出しているのがh。nは上に突き出していませんよね。
ここも書くときはhの左の棒を意識して上に突き出させるようにするといいでしょう。

\paragraph{iとj}
これもまぎらわしいですよね。

どちらも上に点があって、その下に棒があります。ちがいはイはまっすぐなのに、jは下の付き出すとともに丸みを帯びている、カーブしています。

jを書くときには、
意識して棒を下にお突き出させるとともにカーブさせるようにしましょう。

\paragraph{u,v}

あ、これもまぎらわしいですね。

ちがいはuは丸みを帯びているところ、
Vは尖っています。

書くときは、
uは丸みを、
Vはシャープさを意識しておくといいでしょう。

\section{ノートにアルファベット小文字を書く}

5分くらい?

\section{quiz}

これからアルファベットを読みあげます。

聞こえたアルファベットを順番に書いてください。

ぜんぶで10文字です。

音声は2回繰り返します。

(mp3を流す。1\,min34\,sec)

いかがでしたか。それでは答え合わせです。

EDUOP CHIBA

(時間が余っていたらもう1回mp3を流すことで調整)
\section{終わり}

それではきょうはこれで終わります。
%%%%%%%%%%%%%%%%%%%%%
\newpage
\chapter{主語と動詞}

\section{でだし}

\begin{my_check}
\item マイクON
\item カメラON
\end{my_check}

{\LARGE \ComputerMouse}

みなさん、こんにちは。
エデュオプちばの英語の時間にようこそ。
担当の大塚です。

先週まではZOOMで授業を配信していましたが、
今週からはGoogle Classroomを利用しMeetでお届けしています。

きのうからMeetでの配信がはじまっていますが、
英語の授業をMeetで配信するのは初めてです。
あの、音声はちゃんと届いていますか。
チャットでお答えくださるとありがたいです。

あ、だいじょうぶそうですね。
とちゅうでなにかトラブルがあったら、チャットで教えてください。
それでははじめます。

リラックスして参加してください。

さて、きょうは授業に入る前に、
クイズをやってもらいます。

{\large \ComputerMouse}

%%%%%%%%%%%%%%%%%%%%
\newpage
\section{Quiz}

これからアルファベットを6つ順番に読みあげます。
聞こえたアルファベットを順番に小文字で書いてください。
するとある単語になります。
その意味を表す図を選んでください。
答えはa, b, c, dの記号でお願いします。


それでははじめます。\faVolumeUp\,(1\,min)

いかがですか。
わかった人は、チャットで記号を答えてください。


ではもう一度聞いてみましょう。\faVolumeUp\,(1\,min)

それでは、聞こえてきたアルファベットを順番に確認しましょう。{\large \ComputerMouse}

a
{\large \ComputerMouse}
n
{\large \ComputerMouse}
i
{\large \ComputerMouse}
m
{\large \ComputerMouse}
a
{\large \ComputerMouse}
l

animalという単語になりました。

animalは「動物」です。
ですから、4つの図の中から選ぶのなら(a)が正解。

それではanimalの発音練習をしましょう。
カタカナ読みで
「アニマル」と平坦にいうのではなく、
先頭のaを強く発音します。

先頭のaは、
エとアの中間の音です。
エの口の形でアといってみてください。
catの発音と同じです。
エニモゥ。

それでは発音練習です。
\faVolumeUp\,(38\,sec)

きょうのクイズ、発音練習でした。

{\large \ComputerMouse}

%%%%%%%%%%%%%%%%%%%%%%%%%%%
\newpage
\section{授業の流れ}

これまでをふりかえると、
前回はアルファベットの勉強をしました。
きょうは、「主語と動詞」についてお話をします。
それでは、これからの授業の流れです。

{\large \ComputerMouse}

前回の授業の復習をかんたんにしてから、
主語と動詞について勉強していきます。

{\large \ComputerMouse}

\section{復習}

それでは復習です。

{\large \ComputerMouse}

(アルファベットの復習はアドリブで)

{\large \ComputerMouse}

\section{主語とは、動詞とは}
それでは、いよいよきょうの内容にはいりましょう。
きょうのテーマは「主語と動詞です」

皆さんは主語と動詞ということばを聞いたことがありますか。

はじめて聞く人もいるかもしれませんね。

{\large \ComputerMouse}

\subsection{ネコがミルクを飲む}

それでは、この写真を見てください。

この写真では、誰が何をしていますか。

そうです。

ネコがミルクを飲んでいます。

\paragraph{主語とは}
「誰が」に相当する部分のことを「主語」といいます。

ですから、この場合、
主語は「ねこ」です。

\paragraph{動詞}
何をしているかを表すのが「動詞」です。

この場合、ねこはミルクを飲んでいるのですから、
「飲んでいる」が動詞に当たります。

いま、ねこがミルクを飲んでいる写真を見てもらいました。

{\large \ComputerMouse}

\subsection{主語とは、動詞とは}


一般的に言うと、
「だれだれは〜する」という文でいうと、
「だれだれ」に相当する部分を主語というのです。

「何をしているのか」を表す部分が動詞です。

ここでは、主語に相当する部分を、べつに何色でもいいのですが、
わたしは主語を赤い枠でくくくってみます。
そして動詞を青い色でくくってみます。

{\large \ComputerMouse}

それでは4つ例文を出しました。
それぞれの文の主語と動詞はなにか考えてみてください。

さあ、ここで時間を取りますから、4つの例文をノートに移して、
主語と動詞を枠でくくってみてください。

作業が終わった人はリアクションお願いします。


\mySagyo


いかがでしょうか。
答え合わせをしましょう。

{\large \ComputerMouse}

(しばらくアドリブ)

むずかしいことはありませんよね。
これでみなさんは、主語と動詞について基本的なことを理解したことになります。
大きく一歩前進です。

ところで、先に行く前に動詞の部分に注目してください。
スライドで言えば青い枠でくくったところです。

「話す」、「食べる」、「行く」、「飲む」この4つの日本語を、
アルファベットで表記してみましょう。やはり、ノートに書いてみてください。

最初だけわたしがやってみます。

「話す」はアルファベットで表記すると
\ComputerMouse{}hanasuですよね。
では、「食べる」「行く」「飲む」をアルファベットで書いてみてください。

作業が終わったらリアクションをお願いします。

(様子を見ながら)
あ、みなさん終わったでしょうか。

では順番にいきましょう。

hanasu,taberu,
iku,nomu
となりますよね。

なにか気がつくことがありますか。

そうです、アルファベットで表記するとみんな`u'で終わっていますよね。
というわけで、最後のuは色を変えておきました。
これ、日本語の動詞の特徴です。

話す、食べる、行く、飲む、これらはアルファベットで書くとぜんぶ`u'でおわります、音の世界で言えば、語尾を伸ばすとみんあ「う」になりますよね。日本語の動詞の特徴として、アタマに入れておくといいですよ。

話が少しわきにそれました。話をもとにもどします。

「だれだれは〜する」とき、
「だれだれは」を主語、「〜する」の部分を動詞ということを、
最初は「ネコがミルクを飲んでいる写真」で確認し、次に日本語の文で確認しました。

さて、おぜんだてはととのいました。
英語の世界に話を移します。

{\large \ComputerMouse}
%%%%%%%%%%%%%%%
\section{だれだれは〜する}
では、まず

1. I have a car.

ノートに写してください(ちょっと時間を取る)

つぎに、
あ、これどういう意味ですか。
意味がわかる人は、わきに日本語で意味を書いてください。
わからない人は気にする必要はありません。これから説明しますから。

いかがですか。
確認しましょう。{\large \ComputerMouse}「わたしは車を持っている」という意味です。別に同じことですから「所有している」でもいいですね。
わからない人はまだ気にする必要はありません。
わからないから勉強しているのですから。


つぎに移ります。{\large \ComputerMouse}

We  go to work by bus.

まずノートに写しましょう。

あ、これも、
意味がわかる人は、わきに日本語で意味を書いてください。

{\large \ComputerMouse}確認しましょう。「われわれはバスで職場に行く」という意味です。別に同じことですから「バスで通勤している」でもいいですね。

つぎに移ります。

They speak English and Japanese.
これまでと同じように、まずノートに写して、
意味がわかる人は、わきに日本語で意味を書いてください。

確認しましょう。「彼らは英語と日本語を話す」という意味ですね。わたしたちもこうなるといいですよね。

つぎに移ります。

I drink coffee every morning.
まずノートに写して、
意味がわかる人は、わきに日本語で意味を書いてください。

確認しましょう。「わたしは毎朝コーヒーを飲みます」という意味です。

つぎに移ります。

They study at the library. 

手順はこれまでと同じですよ。
英文をノートに写して、わかる人は意味も書きましょう。

「彼らは図書館で勉強します」という意味です。

つぎに移ります。

We eat bread for breakfast.

英文をノートに写して、わかる人は意味も書きましょう。

「われわれは朝食にパンを食べる」という意味です。


では、ここで少し時間を取りますから、ノートに書き写すのがまにあわなかった人は、
ノートを整理してください。



(1分ほど取る)

いかがでしょうか。
ノートはとれましたか。
%%%%%%%%%%%%%%%%%%%%%%%%
\newpage
\begin{my_check}
\item マイクON
\item カメラON
\end{my_check}

{\LARGE \ComputerMouse}

みなさん、こんにちは。
エデュオプちばの英語の授業にようこそ。

きのうは、あいにくの天候でしたが、きょうは晴れ間がのぞいて、
これからかなり熱くなりそうです。

みんさんも熱中症にならないように、
適切な水分補給をお願いします。

さて、今週からはGoogle Classroomを利用しMeetでお届けしています。
音声は届いていますか?
まだまだ、こちらも不慣れな点があるとおもっています。
なにかトラブルがあったら、チャットで教えてくださるとありがたいです。
どうぞよろしくお願いします。

それでは、
リラックスして参加してください。
{\large \ComputerMouse}
%%%%%%%%%%%%%%%%%%%%
きのうに続いて「主語と動詞」について勉強します。

{\large \ComputerMouse}
%%%%%%%%%%%%%%%%%%%%
\subsection{Quiz}
でもその前にquizです。

これからアルファベットを4つ順番に読みあげます。
聞こえたアルファベットを順番に小文字で書いてください。
するとある単語になります。
その意味を表す図を選んでください。
答えはa, b, c, dの記号でお願いします。


それでははじめます。\faVolumeUp\,(48\,sec)

いかがですか。
わかった人は、チャットで記号を答えてください。


ではもう一度聞いてみましょう。\faVolumeUp\,(48\,sec)

それでは、聞こえてきたアルファベットを順番に確認しましょう。{\large \ComputerMouse}

b
{\large \ComputerMouse}
o
{\large \ComputerMouse}
o
{\large \ComputerMouse}
k

bookという単語になりました。

bookは「本、書物、書籍」ですよね。
ですから、4つの図の中から選ぶのなら(d)が正解。

それでは発音練習です。bookの発音です。
「ブック」と平坦にいうのではなく、
先頭のbという音に続く「ウ」という音は、
おちょぼ口で前に出す感じで発音するとうまく発音できますよ。
ではやってみましょう。
\faVolumeUp\,(37\,sec)

きょうのクイズ、発音練習でした。

{\large \ComputerMouse}

%%%%%%%%%%%%%%%%
\subsection{復習}
ではかんたんに、きのうのおさらいをしましょう。

まずねこがミルクを飲んでいる写真をみてもらいました。
だれがしているのか、それが主語。

何をしているのか、それが動詞。
この写真でいえば、「ねこ」が主語で「飲んでいる」が動詞だといいました。

{\large \ComputerMouse}

つぎに、
写真ではなくて文章で。まずは日本語で考えてもらいました。

「だれだれは〜する」というとき、「だれだれは」が主語で、「〜する」が動詞だといいました。

{\large \ComputerMouse}

つぎにこの6つの英文をみてもらいました。
それぞれの意味を確認しましたよね。

ねんのためもう一度確認しておきましょう。
(アドリブ)

では、それぞれの意味を確認したところで、音声に耳を傾けてもらい、そのあと実際にみなさんに発音してもらいます。
Please listen carefully.とPlease repeat after me.というフレーズに注意してください。





\faVolumeUp{}(3\,min 01\,sec)

Good job.(よくできました)といってくれましたね。


\subsection{SVの確認}

さて、さきほど「主語と動詞」についてお話しました。

いま、スライドには6つの英文が示されていますね。

これらの英文の主語と動詞を確認することにしましょう。

ここで時間を取りますから、
6つの英文と主語と動詞がなにか考えてください。

さきほどノートに写してもらった英文を利用して、
主語と動詞を異なる色で囲んでください。
区別がつけば何色でもいいのですが、わたしは主語を赤で、動詞を青で囲むことにします。
区別がつけば何色でもいいです。枠の形も、四角でも円でも楕円でもなんでもいいですよ。

終わった人はリアクションをお願いします。

いかがですか。
それでは答え合わせです。

No.1から。
I have a car.

(ここからアドリブでNo.6まで)

いかがでしたか。

かんたんすぎてひょうしぬけした人もおおかったかもしれませんが、
なんでこんなにかんたんとおもえることに時間をかけて学習したのか、
やがてわかってもらえるとおもいます。

Topic for Todayは「きょうのテーマ」「本日の学習内容」「きょうの授業ポイント」
くらいの意味です。

「英文の骨格は主語と動詞です」としてまとめておきます。

このTopic for Todayもノートに写しましょう。

%%%%%%%%%%%%%%%%%%%%%%%%%%%%%%%%%%%%%%%%%%%%%%%%%%%%%%%%%%%%%%%%%%
\newpage
\begin{my_check}
\item マイクON
\item カメラON
\end{my_check}

{\LARGE \ComputerMouse}

みなさん、こんにちは。
エデュオプちばの英語の授業にようこそ。

きょうも暑いですね。
体調管理にはじゅうぶん気をつけてください。
みんさんも熱中症にならないように、
適切な水分補給をお願いします。
授業中でも差し支えありません。

さて
音声は届いていますか?
なにかトラブルがあったら、チャットで教えてくださるとありがたいです。
どうぞよろしくお願いします。

それでは、
リラックスして参加してください。

きょうも、引き続き「主語と動詞」について勉強します。

{\large \ComputerMouse}



\section{quiz}
さて、きょうも、本題にはいるまえに、まず
クイズをやってもらいます。

{\large \ComputerMouse}

\subsection{Quiz}

これからアルファベットを5つ順番に読みあげます。
聞こえたアルファベットを順番に小文字で書いてください。
するとある単語になります。
その意味を表す図を選んでください。
答えはa, b, c, dの記号でお願いします。


それでははじめます。\faVolumeUp\,(48\,sec)

いかがですか。
わかった人は、チャットで記号を答えてください。

それでは、聞こえてきたアルファベットを順番に確認しましょう。{\large \ComputerMouse}

c\,\,
{\large \ComputerMouse}\,\,
a\,\,
{\large \ComputerMouse}\,\,
n\,\,
{\large \ComputerMouse}\,\,
d
{\large \ComputerMouse}\,\,
y

ではもう一度聞いてみましょう。\faVolumeUp\,(53\,sec)


c, a, n, d, yという単語になりました。

candyという単語になりましたね。

candyは「あめ、キャンディ」ですよね。
ですから、4つの図の中から選ぶのなら(c)が正解。

それではcandyの発音練習をしましょう。
カタカナ読みで
「キャンディ」と平坦にいうのではなく、
先頭のaを強く発音します。

aは、
エとアの中間の音です。
エの口の形でアといってみてください。
両ほほを左右に引っ張り、口を横にあけて「ア」と発音する。
catの発音と同じです。

あ、どこかでやったと思う人はいますか。

おとといの授業でanimalという単語を勉強しましたね4.
animalの発音と同じ音です。

それでは発音練習です。

\faVolumeUp\,(38\,sec)

きょうのクイズ、発音練習でした。

{\large \ComputerMouse}


あ、棒の先に飴がついていますが、これもcandyの一種ですね。

{\large \ComputerMouse}

\section{復習}


それでは、きのうの復習です。

6つの例文について、
それぞれ主語と動詞を確認しましたね。

Topic for Today(きょうのポイント)として
\begin{itemize}
 \item 英文の骨格は主語と動詞
\end{itemize}
とまとめておきました。

では、6つの英文の
単語を確認していきましょう。

have
(しばらくアドリブ)

では、単語や表現の音をよく聞いてもらい、そのあと実際に発音してもらいます。
(5\,min41\,sec)


\section{主語と動詞の順番}

「わたしは英語を話します」という日本語は、わざわざ口に出していうかという問題はさておき、正しい日本語で、意味もまぎれることなく伝わります。

「英語をわたしは話します」はどうでしょうか。
まあこれでも伝わりますよね。

ではこれはどうでしょうか。
「英語を話します、わたしは」
これも伝わるでしょう。

けっきょく、日本語は主語と動詞の順番はかなり自由度が高いということになります。
順番は適当でも日本語としては伝わります。
もちろん、場面場面でたしょう不自然ということはあるかもしれませんが、
コミュニケーションの妨げになるようなことはないということです。


それではこんどはどうでしょうか。
「英語を話します」

この文では「わたしは」が消えてしまいました。
きょう勉強した言葉を用いていうと、主語が消えてしまっています。

でも、
たとえば会話の中で「あなたは日本語以外の言葉を話しますか」と聞かれた人が、
「英語を話します」と答えるのはきわめて自然な会話ですよね。
そうです、日本語は主語と動詞の順番が適当でも通じることがおおいですし、また前後関係によっては主語をいちいち表現しないでもすむのです。

いっぽう、英語の場合はどうでしょうか。

英語の場合、「わたしは英語を話します」を表すのは
I speak English.
これだけです。

ここにあげたようなのはぜんぶ英語としては失格なんですねえ。
(アドリブ)

きょうのポイントに追加します。
\begin{itemize}
 \item   英文の骨格は主語と動詞です
 \item   英語は語順がだいじです
 \item   英文にはかならず主語が必要
\end{itemize}

%%%%%%%%%%%%%%%%%%%%%%%%%%%%%%%%%%%%%
\section{Exercises}

では、ここでExercises(練習問題)です。

ぜんぶで8つの英文があります。
ここで時間をとりますから、
英文をノートに写して、
日本語を参考にしながら、それぞれの英文の主語と動詞を指摘してください。

終わった人は、リアクションで教えてください。

\mySagyo

(アドリブで答え合わせ)

\faVolumeUp\,(3\,min34\,sec)

{\large \ComputerMouse}


それではみなさんさようなら。
また、次回お会いするのを楽しみにしています。

%%%%%%%%%%%%%%%%%%%%%%%%%%%%%%%%%%%%%%%%%%%%%%%%%%%%%%%%%%%%%%
\newpage


\chapter{be動詞}

%\mbox{}\hfill{\Large\gtfamily 2024-06-25 Tue}
%
%\mbox{}\hfill{\large\gtfamily 1年生14:10--14:40}
%
%
%\bigskip



%\begin{my_check}
%\item マイクON
%\item カメラON
%\end{my_check}
%
%{\LARGE \ComputerMouse}

%みなさん、こんにちは。
%エデュオプちばの英語の授業にようこそ。
%
%先週まではZOOMで授業を配信していましたが、
%今週からはGoogle Classroomを利用しMeetでお届けしています。
%
%きのうからMeetでの配信がはじまっていますが、
%英語の授業をMeetで配信するのは初めてです。
%あの、」音声はちゃんと届いていますか。
%チャットでお答えくださるとありがたいです。
%
%あ、だいじょうぶそうですね。
%とちゅうでなにかトラブルがあったら、チャットで教えてください。
%それでははじめます。
%
%リラックスして参加してください。

%\myStartLesson
%
%さて、きょうも授業に入る前に、
%クイズをやってもらいます。
%
%{\large \ComputerMouse}

%%%%%%%%%%%%%%%%%%%%%
%\newpage
%\section*{Quiz E}
%%きょうも、引き続き「be動詞」について勉強します。
%%
%%あ、でもその前に、いつものようにクイズです。
%
%これからアルファベットを5つ順番に読みあげます。
%聞こえたアルファベットを順番に小文字で書いてください。
%するとある単語になります。
%その意味を表すものを選んでください。
%答えはa, b, c, dの記号でお願いします。
%
%
%それでははじめます。\marginnote{\tiny \faVolumeUp\,(54\,sec)}
%
%いかがですか。
%わかった人は、チャットで記号を答えてください。
%
%
%
%それでは、聞こえてきたアルファベットを順番に確認しましょう。{\large \ComputerMouse}
%
%e--i--g--h--t
%
%5つのアルファベットを確認したところで、
%もう一度聞いてみましょう。\marginnote{\tiny \faVolumeUp\,54\,sec}
%
%
%{\large \ComputerMouse}
%
%e--i--g--h--tという綴になりました。この単語はどう発音しますか。
%
%そうです。「エイト」ですね。
%
%「8」の意味です。
%
%では選択肢のうちどれが世界でしょうか。
%
%\begin{itemize}
% \item[(a)] サイコロの4と6の目がでています。あわせると、これは「10」ですね
% \item[(b)] これは、みてのとおり「9」です
% \item[(d)] (c)の前に(d)をみてみましょう。\\
%「ダイア」の7ですね。前回学習した単語diamondを覚えていますか。平坦に「ダイアモンド」というのではなく、「{\gtfamily ダ}イアモンド」と発音することも確認しておきましょう
% \item[(c)] 数式が出ています。計算してみましょう。\\
%14から2を弾いて12。この12に3をかけて36と思った人はいないでしょうね。それでは答がなくなってしまいます。かけ算やわり算をたし算やひき算より先に計算するのが約束でしたね。ですから、この問題では、14から2を引くのではなく、まず掛け算である$2\times{3}=6$を先に計算して、$14-6=8$とするのでしたね。8がでてきました。ではこれが正解です。
%\end{itemize}
%
%みなさん、いかがでしたか。
%
%それでは、きょうの単語eightの発音練習です。
%
%エとイではなく、連続するエ~ィという音です。
%
%流れるように1つの音として読むことを意識してください。
%
%また、この単語はもうひとつ注意してほしいことがあります。
%
%綴のなかでg--hという2文字はまったく発音に関係ないことです。
%
%このあたりがやっかいなところですが、そういうものだとおもって練習しましょう。
%
%
%そのことに注意して音声を聞いてください。
%ではやってみましょう。
%\marginnote{\tiny \faVolumeUp\,(37\,sec)}
%
%発音練習していただきましたが、書くときには、
%発音しないけれど、きちんとghを書くことに注意しましょう。
%
%
%きょうのクイズ、発音練習でした。
%
%{\large \ComputerMouse}
%%%%%%%%%%%%%%%%%%%%%
%\newpage
%\section{授業の流れ}
%
%それでは、これからの授業の流れです。
%
%{\large \ComputerMouse}
%
%前回の授業の復習をかんたんにしてから、
%be動詞の基本的事項について順番に勉強していきます。
%
%{\large \ComputerMouse}

%%%%%%%%%%%%%%%%%%%%%%%%%%%%%%%
%\newpage
%\section*{復習}
%
%前回は主語と動詞について勉強しましたね。
%
%具体的には
%
%{\large \ComputerMouse}
%
%\begin{itemize}
% \item   英文の骨格は主語と動詞です
% \item   英語は語順がだいじです
% \item   英文にはかならず主語が必要だ
%\end{itemize}
%ということを勉強しましたね。
%
%
%日本語は語順については自由度が高くて、どんな順番でも通じることがおおいけれど、
%英語は語順が大事ですよ、語順が違うと通じないです---なんてことを勉強しました。
%
%それから日本語は、前後関係でわかればいちいち主語を明示しないこともよくあります。
%でも英語は必ず主語が必要で「わたしは英語を話します」の意味のI speak English.のIを省略することはできませんなんてことをいいましたよね。
%{\large \ComputerMouse}

%%%%%%%%%%%%%%%%%%%%%%%%%%%%%%%%%%%%%
%\subsection*{Exercises}
%
%では、ここでExercises(練習問題)です。
%
%ぜんぶで8つの英文があります。
%ここで時間をとりますから、
%英文をノートに写して、
%日本語を参考にしながら、それぞれの英文の主語と動詞を指摘してください。
%
%終わった人は、リアクションで教えてください。
%
%\mySagyo
%
%(アドリブで答え合わせ)
%
%\faVolumeUp\,(3\,min34\,sec)
%
%{\large \ComputerMouse}


%%%%%%%%%%%%%%%%%%%%%%%%%%%%%%%%%%%%
%\newpage
\section{be動詞とは}
それでは、新しい単元にはいりましょう。

「be動詞」とはなにかということを勉強します。

{\large \ComputerMouse}

では、まずこの6つの英文を見てください。

ここで時間をとるので、
英文をノートに写してください。
意味がわかる人は意味も書き加えましょう。

作業が終わった人は、リアクションで教えてください。

\mySagyo

では、順番に意味を確認していきましょう。

「わたしは生徒です」という意味の英文です。
(以下、順番に)


(意味の確認---ここはアドリブ)

\paragraph{例文1}
では1. I am a student.に戻ります。
前回、英語の骨格は「主語と動詞」だといいました。
この文の主語はI、動詞は赤で示したamです。
この文では「I(私)$=$a student(生徒)」の関係であることに注意しましょう。
このようにamは特別な動詞で、イコールの意味を表します。

\paragraph{例文2}
次の文はどうでしょう。
You are my friend.
動詞は赤で示したare。
このareという動詞も、You(あなた)$=$my friend(私の友人)ということを示しています。
つまりイコールの意味ですね。

\paragraph{例文3}
次の
 He is tall.
動詞は赤で示したis。
やはりHe(彼)$=$tall(背が高い)ということを示しています。

\paragraph{例文4}
次の
She is kind.
Shek(彼女)$=$kind(親切)というイコールの関係です。

次は
\paragraph{例文5}
The sky is blue.
The sky(空)$=$blue(青い)というイコールの関係です。

\paragraph{例文6}
They are my classmates.
They(彼ら)$=$my classmates(わたしのクラスメート)、
やhりイコールの関係です。

\bigskip

\subsection{Topics for Today}
ここまでを振り返りますね。
ここで赤で示したam, are, is はぜんぶイコールの意味です。


ポイントとしてTopics for Todayにまとめておきました。

\begin{enumerate}
 \item am, are, isはぜんぶなかまで、イコールの意味ですよ
 \item この3つをまとめてbe動詞というんですよ
\end{enumerate}
ということです。

それから、am動詞でもare動詞でもis動詞でもなく、なんでbe動詞なんていうのか、
不思議に思う人もいるとおもいますが、
しばらくのあいだは、そういうものだとおもっておいてください。
そのことについては、おいおい説明するので、
とうめんはそういうものなのだとおもっておいてください。

\paragraph{単語と表現} \mbox{}

(アドリブ)

では、ここで6つの英文の発音練習をしておきましょう。

\marginnote{\tiny \faVolumeUp\,(2\,min51\,sec)}

{\large \ComputerMouse}
%%%%%%%%%%%%%%%%%%%%%%%%%%%%%%%%%
\subsection{図}
さて、この図を見てください。

am\,\,\ComputerMouse\,\,are\,\,is---これはみんななかまです。

まとめてbe動詞というといいました。

さて、ここでひとつ問題があります。

am, is, are---この3つをどう使い分けるのかということです。

{\large \ComputerMouse}
%%%%%%%%%%%%%%%%%%%%%%%%%%%%%%%%%
\section{be動詞、どれ使う}

ということで、タイトルを「be動詞、どれ使う?」としました。

つぎにbe動詞---am, are, is---の使い分けについて勉強することにしましょう。

{\large \ComputerMouse}

\section{I am 〜}
{\large \ComputerMouse}

では、まずこの6つの英文を見てください。

いつものように
ここで時間をとるので、
英文をノートに写してください。
わかる人は意味も書き加えましょう。

作業が終わった人は、リアクションで教えてください。

\mySagyo

では、順番に意味を確認していきましょう。


\begin{enumerate}
 \item  I am a student.「わたしは生徒です」
 \item  I am tall.「わたしは背が高い」
 \item  I am 13 years old.「わたしは13歳です」
 \item  I am John.「わたしはジョンです」
 \item  I am happy.「わたしは幸せです」
 \item  I am from Tokyo.「わたしは東京の出身です」
\end{enumerate}

意味を大づかみに確認しました。

ポイントとして
\begin{itemize}
 \item amはbe動詞です
 \item amは必ずIとセットで使います。ですからI amをひとつのパターンとして覚えてしまいましょう。スライドの6つの例文ではI amを赤で目立つようにしておきました。
\end{itemize}


\paragraph{単語・表現の確認}
(アドリブ)

それでは発音練習です。

\faVolumeUp{}(2\,min50\,sec)
%%%%%%%%%%%%%%%%%%%%%%%%%%%%%%%%%%%%%%%%%%
\newpage
%%%%%%%%%%%%%%%%%%%%%%%%%%%%%%%%

%みなさん、こんにちは。
%エデュオプちばの英語の授業にようこそ。
%
%さて
%音声は届いていますか?
%なにかトラブルがあったら、チャットで教えてくださるとありがたいです。
%どうぞよろしくお願いします。
%
%
%
%きょうも暑いですね。
%体調管理にはじゅうぶん気をつけてください。
%みなさんも熱中症にならないように、
%適切な水分補給をお願いします。
%授業中でも差し支えありません。
%遠慮なく水分を補給しながら、
%リラックスして参加してください。
%
%それでは授業にはいります。
%きょうも、引き続き「主語と動詞」について勉強します。
%
%{\large \ComputerMouse}
%
%
%\begin{my_check}
%\item カメラOFF
%\end{my_check}
%
%{\LARGE \ComputerMouse}
%\section{Quiz}
%
%あ、でもその前に、クイズです。
%
%これからアルファベットを7つ順番に読みあげます。
%聞こえたアルファベットを順番に小文字で書いてください。
%するとある単語になります。
%その意味を表す図を選んでください。
%答えはa, b, c, dの記号でお願いします。
%
%
%それでははじめます。\faVolumeUp\,(1\,min\,06sec)
%
%いかがですか。
%わかった人は、チャットで記号を答えてください。
%
%
%ではもう一度聞いてみましょう。\faVolumeUp\,(1\,min06\,sec)
%
%それでは、聞こえてきたアルファベットを順番に確認しましょう。{\large \ComputerMouse}
%
%{\large \ComputerMouse}
%
%
%diamondという単語になりました。
%
%
%diamond.
%トランプのマークがありますね。
%
%(b)が正解でした。
%
%
%ところで日本語では「ダイヤ」のようにアではなくヤということもおおいのですが、
%英語の綴りをもてもらうとわかるようにyというアルファベットは使われていませんよね。
%ですから英語では「ダイヤ」とか「ダイヤモンド」とはいわないことに注意しましょう。
%それでは発音練習です。diamondの発音です。
%「ダイアモンド」と平坦にいうのではなく、
%「ダイ」にアクセントを置いて発音しましょう。
%
%ここでdiamondに含まれる「アイ」の発音に注目してください。
%「アイ」と平坦にいうのではなく「アーイ」」となります。
%そのことに注意して尾根性を聞いてください。
%ではやってみましょう。
%\faVolumeUp\,(38\,sec)
%
%きょうのクイズ、発音練習でした。
%
%{\large \ComputerMouse}
%
%あ、もしかするとdiamondと聞いたときに、みなさんのアタマにまっさきにうかぶのはこれかもしれないですね。
%ただ、きょうはトランプのマークだったので、とまどった人がいるkおしれませんね。
%
%もちろん、いまご覧いただいているのはdiamondです。
%{\large \ComputerMouse}

%%%%%%%%%%%%%%%%%%
%\newpage
%\section{復習}
%be動詞はam, are, isと3つあるけれど使い分けはどうしましょうかというのが課題でしたよね。
%
%第一段階としてamはI amでセット。パターンとして覚えましょうと言いました。
%
%I am を赤で目立たせましたよね。
%
%では発音の練習をしておきましょう。
%
%\faVolumeUp{}(2\,min50\,sec)

\newpage
\subsection{Exercises}
では、練習問題です。

John Smithさんは、Boston出身で年齢は 12 歳です。
John Smithさんになったつもりで自己紹介文を作成してみましょう。


ノートに書き込みましょう。
作業が終わった人は、リアクションで教えてください。

\mySagyo

あ、ところでBostonてどこにありますか。
アメリカ合衆国の東海岸ですよね。
Boston Red Soxの本拠地ですね。

では、順番に確認していきましょう。

\begin{itemize}
 \item I am John Smith.
 \item I am from Boston.
 \item I am twelve years old.
\end{itemize}

さて発音練習です。

\faVolumeUp\,(1\,min37\,sec)

みなさんもこの全体を一つのパターンとして、自分の自己紹介を作ってるといいですよ。


ここでもうひとつおつきあいください。

{\large \ComputerMouse}

I amは2つの単語ですが、この2つをまとめてI'mということもできます。
このように縮めたいいかたを短縮形といいます。

短く縮めた形。文字通りですね。

授業のポイントに1つ加えます。

短縮形の'をアポストロフィといいます。これからもこのことば、用語は使うので覚えておいてください。


では、短縮形のI'mの発音練習です。

きょうは
\begin{itemize}
 \item be動詞の使い分けとしてI amをひとつのパターンとして覚えよう、
 \item Iam をI'mといっても同じです。これを短縮形といいます、
 \item ちょんという記号をアポストロフィといいます
\end{itemize}
ということを勉強しました。


{\large \ComputerMouse}
%%%%%%%%%%%%%%%%%%%%%%%%%%%%%%%%%%%%%%%%%%%%%%%%%%%%%%%%%%%%%%%%%%%%
\newpage
%
%\mbox{}\hfill{\Large\gtfamily 2024-06-25 Tue}
%
%\mbox{}\hfill{\large\gtfamily 2年生10:20--10:50}
%
%\mbox{}\hfill{\large\gtfamily 3年生13:10--13:40}
%
%\bigskip
%
%\myStartLesson

%%%%%%%%%%%%%%%%%%%%
%\newpage
%\section{Quiz}
%きょうも、引き続き「be動詞」について勉強します。
%
%あ、でもその前に、いつものようにクイズです。
%
%これからアルファベットを5つ順番に読みあげます。
%聞こえたアルファベットを順番に小文字で書いてください。
%するとある単語になります。
%その意味を表すものを選んでください。
%答えはa, b, c, dの記号でお願いします。
%
%
%それでははじめます。\marginnote{\tiny \faVolumeUp\,(54\,sec)}
%
%いかがですか。
%わかった人は、チャットで記号を答えてください。
%
%
%
%それでは、聞こえてきたアルファベットを順番に確認しましょう。{\large \ComputerMouse}
%
%e--i--g--h--t
%
%5つのアルファベットを確認したところで、
%もう一度聞いてみましょう。\marginnote{\tiny \faVolumeUp\,54\,sec}
%
%
%{\large \ComputerMouse}
%
%e--i--g--h--tという綴になりました。この単語はどう発音しますか。
%
%そうです。「エイト」ですね。
%
%「8」の意味です。
%
%では選択肢のうちどれが正解でしょうか。
%
%\begin{itemize}
% \item[(a)] サイコロの4と6の目がでています。あわせると、これは「10」ですね
% \item[(b)] これは、みてのとおり「9」です
% \item[(d)] (c)の前に(d)をみてみましょう。\\
%「ダイア」の7ですね。前回学習した単語diamondを覚えていますか。平坦に「ダイアモンド」というのではなく、「{\gtfamily ダ}イアモンド」と発音することも確認しておきましょう
% \item[(c)] 数式が出ています。計算してみましょう。\\
%14から2を弾いて12。この12に3をかけて36と思った人はいないでしょうね。それでは答がなくなってしまいます。かけ算やわり算をたし算やひき算より先に計算するのが約束でしたね。ですから、この問題では、14から2を引くのではなく、まず掛け算である$2\times{3}=6$を先に計算して、$14-6=8$とするのでしたね。8がでてきました。ではこれが正解です。
%\end{itemize}
%
%みなさん、いかがでしたか。
%
%それでは、きょうの単語eightの発音練習です。
%
%エとイではなく、連続するエ~ィという音です。
%
%流れるように1つの音として読むことを意識してください。
%
%また、この単語はもうひとつ注意してほしいことがあります。
%
%綴のなかでg--hという2文字はまったく発音に関係ないことです。
%
%このあたりがやっかいなところですが、そういうものだとおもって練習しましょう。
%
%
%そのことに注意して音声を聞いてください。
%ではやってみましょう。
%\marginnote{\tiny \faVolumeUp\,(37\,sec)}
%
%発音練習していただきましたが、書くときには、発音しないghも書かないといけないことに注意しましょう。
%
%
%きょうのクイズ、発音練習でした。

%{\large \ComputerMouse}

%%%%%%%%%%%%%%%%%%%%%%%%
%\newpage
%\myStartLesson
%
%\subsection{Quiz}
%
%これからアルファベットを4つ順番に読みあげます。
%聞こえたアルファベットを順番に小文字で書いてください。
%するとある単語になります。
%その意味を表す図を選んでください。
%答えはa, b, c, dの記号でお願いします。
%
%
%それでははじめます。\faVolumeUp\,(48\,sec)
%
%いかがですか。
%わかった人は、チャットで記号を答えてください。
%
%それでは、聞こえてきたアルファベットを順番に確認しましょう。{\large \ComputerMouse}
%
%f\,\,
%{\large \ComputerMouse}\,\,
%i\,\,
%{\large \ComputerMouse}\,\,
%s\,\,
%{\large \ComputerMouse}\,\,
%h
%
%ではもう一度聞いてみましょう。\faVolumeUp\,(48\,sec)
%
%
%fishという単語になりました。
%
%
%
%それではfishの発音練習をしましょう。
%カタカナ読みで
%「フィッシュ」と平坦にいうのではなく、
%iを強く発音します。
%
%それでは発音練習です。
%
%\faVolumeUp\,(38\,sec)
%
%きょうのクイズ、発音練習でした。
%\myMouse
%あ、これなんという魚でしょうか。
%たくさんいますねえ。圧巻です。

%\section{復習}
%
%それでは前回のおさらいです。
%
%{\large \ComputerMouse}
%
%前回は、みなさんにJohn Smithさんになったつもりで、I amを用いた自己紹介文を作成してもらいました。
%
%あ、これはBostonの町並みですね。
%これはBostonの夜景。
%これは地下鉄の路線図。大都会ですね。
%
%さて、
%これが正解でしたね。
%
%
%
%I amは2つの単語ですが、この2つをまとめてI'mということもできます。
%I amといっても、縮めてI'mと言っても全く同じです。
%このように縮めたいいかたを短縮形といいます。
%
%短く縮めた形。文字通りですね。
%
%短縮形の'をア{\gtfamily ポ}ストロフィといいます。これからもこのことば、用語は使うので覚えておいてください。
%
%
%では、発音練習です。
%まず縮めないI amです。\marginnote{\tiny 1\,min37\,sec}
%
%短縮形のI'mの発音練習です。\marginnote{\tiny 1\,min36\,sec}

%%%%%%%%%%%%%%%%%%%%%%%%%%%%%
\newpage
\section{You are 〜}
{\large \ComputerMouse}

では、まずこの6つの英文を見てください。



いつものように
ここで時間をとるので、
英文をノートに写してください。
わかる人は意味も書き加えましょう。

作業が終わった人は、リアクションで教えてください。

\mySagyo

では、順番に意味を確認していきましょう。


\begin{enumerate}
 \item You are my friend.
 \item You are very kind.
 \item You are a good student.
 \item You are good at baseball.
 \item You are busy.
 \item You are from Chiba.
\end{enumerate}

(アドリブ)

意味の確認

前回は、この6つの例文でYou are(あなたは〜だ)という言い方を学習しました。


\subsection{まとめ}

I amとならんでYou areをひとつのパターンとして覚えましょう。

それでは、発音の練習です。
\marginnote{\tiny \faVolumeUp{}(2\,min26\,sec)}

では、単語の確認をしていきましょう。

(アドリブ)

%%%%%%%%%%%%%%%%%%%%%%
\newpage
\section{Exericises}

では練習問題です。あたえられた日本語の意味になるよう空所に適当な語を補いましょう。

ノートに写しながらやってください。
作業が終わった人はリアクションで教えてください。

\mySagyo

それでは答え合わせです。

no1.
「あなた」$=$kindです。You areが1つのパターンでしたね。

No2.
「あなた」$=$tallです。主語はYouですからareが正解ですね。

No3.
こんどはどうでしょう。
「あなた」$=$kindです。
困りました。You areをいれたいのですが、空所が1つしかありません。
1つの空所に無理やり2つの単語をいれるわけにはいきません。
どうすればいいでしょう。

あ、そういえば、I amをI'mと短く縮めるいいかたがあったっけと思った人は、
頭がいいなあとおもいます。

そうです、I amがI'mとなったように、You areも短く縮めるいいかたがあるのです。
You areを短く縮めるとYou'reとなります。

用語の復習です。短く縮めた形のことをなんといいましたか。
そう、「短縮形」でしたね。

ではとちゅうの'をなんといいましたっけ。
そう、「ア{\gtfamily ポ}ストロフィ」でしたね。

You areの短縮形はYou$+$apostrophy$+$reです。

発音はどうなるでしょうか。
You areは「ユーアー」ですが、You'reは「ユア」です。
綴だけでなく、発音も短めになりますね。


ポイントにつけくわえましょう。

\begin{itemize}
 \item areはbe動詞。You areはパターンとして覚えよう
 \item {\gtfamily $\text{You are}=\text{You're}$}
\end{itemize}

それでは、You areと短縮形のYou'reの違いに注意しながら発音練習しましょう。
\marginnote{\scriptsize \faVolumeUp(2\,min)}
%%%%%%%%%%%%%%%%%%%%%%%%%%%%%
\newpage

\subsection{I am, You are以外(1)}
{\large \ComputerMouse}
これまでI amとYou areをセットとして確認してきました。

こんどはI am, You are以外の場合について学習します。


では、まずこの6つの英文を見てください。

いつものように
ここで時間をとるので、
英文をノートに写してください。
わかる人は意味も書き加えましょう。

作業が終わった人は、リアクションで教えてください。

\mySagyo

では、順番に意味を確認していきましょう。


\begin{enumerate}
 \item This is my pencil.
 \item He is my classmate.
 \item She is a good singer.
 \item Your bike is new.
 \item George is busy.
 \item Jane is from France.
\end{enumerate}

(アドリブ)

意味の確認

まとめ

\begin{itemize}
 \item I, You 以外で1つ (It, This, That, The book ...)、1人(He, She, Jane, My father ...)が主語のときはisを使います
\end{itemize}

べつのいいかたをすると、主語が単数のときはisを使うということですね。

それでは、発音の練習です。
\marginnote{\tiny \faVolumeUp{}2\,min50\,sec}

では、単語の確認をしていきましょう。

(アドリブ)

%%%%%%%%%%%%%%%%%%%%%%%%%%%
\subsection{Exercises}

それでは練習問題です。

つぎの各2文が同じ意味になるよう空所に適当な語を補いましょう。
1.1と1.2を見比べるとvery kindの部分はまったく同じですからShe isを1つにまとめればいいことがわかりますね。
She isを1つにまとめてShe'sが正解です。

2つ目も同じ考え方。from Australiaは全く同じですからHe isをまとめてHe'sが正解。

3つ目はIt isをまとめてIt'sです。

4つ目も同じ考え方。That isをまとめてThat'sです。

以上をまとめて
\begin{itemize}
 \item He is\,($=\text{He's}$)\,\,\,\,/\,\,\,She is\,($=\text{She's}$)\,\,\,\,/\,\,\,It is\,($=\text{It's}$)\,\,\,\,/\,\,\,That is\,($=\text{That's}$)
\end{itemize} 
と整理しておきます。

では短縮形の発音練習です。
各組の英文のうち、短縮形を用いた2行目の例文の練習をします。
\marginnote{\tiny \faVolumeUp{}2\,min}
%%%%%%%%%%%%%%%%%%%%%%%%%%%%%
\newpage

\section{I am, You are以外(2)}
{\large \ComputerMouse}

I am, You are以外の場合として、主語が1つとか1人のときはisを使うことを確認しましたね。
1つとか1人とかいいましたが、別の言葉でいいかえると、主語が単数のときはisを使うということでした。

これから、
I am, You are以外の場合その2について学習します。%で、主語が2つ以上、2人以上の場合について考えることにしましょう。


では、まずこの6つの英文を見てください。

いつものように
ここで時間をとるので、
英文をノートに写してください。
わかる人は意味も書き加えましょう。

作業が終わった人は、リアクションで教えてください。

\mySagyo

では、順番に意味を確認していきましょう。

\begin{enumerate}
 \item These are my pencils.
 \item They are my classmates.
 \item They are kind.
 \item The flowers are beautiful.
 \item We are busy.
 \item Jane and George are from France.
\end{enumerate}

意味の確認(アドリブ)

\paragraph{まとめ}

さて、この6つの英文ではすべてareが使われています。
どう整理すればいいでしょうか。

つぎのように説明したらどうですか。

例文の1, 3のように主語がTheseならareを使います。例文2のように主語がTheyのときもareを使います。例文5のように主語がWeのときもareを使います。例文4のようにThe flowersが主語のときもareです。例文6のようにJane and Georgeが主語のときもareです。

いかがですか?
それではやってられませんよね。いつまでたってもきりがありません。とてもひとつひとつ覚えてなどいられません。

じつは英語には次のような約束があるのです。

\myMouse

\begin{itemize}
 \item I, You 以外で複数 (2 つ以上) のモノや人で始まるときは are を使います
\end{itemize}

いいかえると、
主語が複数のときはareを使うということ。
ただそれだけのこと。
この約束さえ、しっかり身につければ、いちいち個別に暗記する必要なんかないということですよね。
すばらしいとおもいませんか。

それでは、発音の練習です。
\marginnote{\tiny \faVolumeUp{}2\,min48\,sec}

では、単語の確認をしていきましょう。

(アドリブ)

\subsection{Exercises}

それでは練習問題です。

つぎの各2文が同じ意味になるよう空所に適当な語を補いましょう。
「わたしたちは親友だ」
1.1と1.2を見比べるとbest friendsの部分はまったく同じですからWe areを1つにまとめればいいことがわかりますね。
これまでI'mやYou'reなどで学習したように、短縮形の問題にたどりつきます。
We areを1つにまとめてWe'reが正解です。
短縮形を作るのにアポストロフィを用いるのもこれまでとまったく同じですよね。
1つの事柄をしっかり身につけることで、
どんなに応用がきくか、わかってもらえるとうれしくおもいます。

\bigskip

2つ目も同じ考え方。「彼らはイタリアの出身だ」from Italyは全く同じですからThey areをまとめてThey'reが正解。

\begin{itemize}
 \item We are\,($=\text{We're}$)\,\,\,\,/\,\,\,They are\,($=\text{They're}$)
\end{itemize}


では短縮形の発音練習です。
各組の英文のうち、短縮形を用いた2行目の例文の練習をします。
\marginnote{\tiny \faVolumeUp{}1\,min35\,sec}

%%%%%%%%%%%%%%%%%%%%%
\newpage

\section{まとめ}
これでbe動詞の基本が終わったことになります。
ここでこれまでの総まとめです。

\begin{numcases}{\text{ }}
 \text{\mbox{}\,\,{}Iではじまる}&$\longrightarrow$\,\,\,\,\,\,{}\text{am}\\
 \text{\mbox{}\,\,{}Youではじまる}&$\longrightarrow$\,\,\,\,\,\,{}\text{are}\\[5pt]
 \text{I, You以外のとき}\\
 \text{\mbox{}\,\,{}1つ、1人}&$\longrightarrow$\,\,\,\,\,\,{}\text{is}\\
 \text{\mbox{}\,\,{}2つ、2人以上}&$\longrightarrow$\,\,\,\,\,\,{}\text{are}
\end{numcases}

\paragraph{図}

\paragraph{Exercises}

最後にしあげの練習問題です。

\mySagyo

\begin{enumerate}
 \item 主語はJohn。単数形だからisが正解。「ジョンは野球の選手です」
 \item 主語がYouのときはYou areの一点張りでしたね。areが正解。「あなたは親切だ、やさしい」
 \item 主語のThis(これ)は単数形ですから、isが正解。「これはわたしのカップです」
 \item 主語のThey(彼ら)は複数形ですからareが正解。「彼らはわたしの両親です」
 \item 主語のWeは複数形ですからareが正解。「われわれは日本人です」
 \item 主語はIですからI amが正解。「わたしは空腹だ」

\end{enumerate}

いかがでしたか。全問正解だった人はリアクションしてください。


それでは発音練習です。
\marginnote{\tiny \faVolumeUp{}2\,min23\,sec}

%%%%%%%%%%%%%%%%%%%%%%%%%%%%%%%%%%%%
\chapter{一般動詞}

am, is, areの3つをまとめてbe動詞と呼びましたが、前回まではbe動詞について基本的なことを学習してきました。
きょうは「一般動詞」について学習します。「一般動詞」と急に言われてもわけわからないという人がいるとおもいます。その疑問はとうぜんの疑問です。
いまはまだわからなくてあたりまえです。これから学習します。\myMouse

\section{授業の流れ}

こんな予定で学習を進めます。

まずは「一般動詞とは」というところからはいります。

\section{一般動詞って?}

\subsection{意味の確認}

(ここはアドリブ)

\subsection{SVの確認}

では、意味を確認したところで、作業をしてもらいます。
6つの英文をノートに書いて、それぞれの主語と動詞を指摘してください。
はじめてください。
終わった人はリアクションしてください。

\mySagyo


\paragraph{ひとつひとつ確認}

(アドリブ)

主語が赤、動詞が青。
みなさん、かんたんにクリアできたものとおもいます。

\subsection{一般動詞の定義}
(動詞が黒くなったあと、青になる)
さて、ここで動詞に注目しtみましょう。

この青で示した動詞のことを、「一般動詞」といいます。

実は「be動詞以外の動詞を、ぜんぶひっくるめて一般動詞」といいます。
これが一般動詞の定義です。
なあんだ、いわれてみればかんたんですね。
「be動詞以外の動詞はなんでも一般動詞」なんです。

ところで、be動詞は主語に応じてam, are, isの使い分けこそありましたが、意味はなんでしたか?
覚えていますか。
そうです、be動詞は「イコール」の意味でしたよね。
「イコール」の意味しかありませんでした。

いっぽう一般動詞の意味はどうでしょう。
ここで例文1から6を見てください。
like, walk, speak, drink, study, play、ぜんぶ異なる意味ですよね。
一般動詞の意味はいろいろあります。
be動詞とはちがって、一般動詞はたくさんありますし、意味もさまざまです。
ちょっとやっかいだなとおもうでしょうが、
出てくるたびにひとつひとう覚えることが肝心です。

大変だなあ、と思う人もいるでしょうが、基本的な動詞はいろいろな場面で使われます。
何度も何度もでてきます。
いや、何度も何度も出てくるから基本的な動詞というのです。

あせることなくひとつひとつみについえていきましょう。

授業で用いた例文をノートに写してもらっていますが、
そのたびに何度も何度も実際に発音したり、自分の手で書くことによって身についていきますよ。

さあ、それでは実際の音声に耳をかたむけてください。そのあとで発音練習もしてもらいます。
\marginnote{\tiny \faVolumeUp\,2\,min53\,sec}

\paragraph{単語の確認}

(アドリブ)

\myMouse

\subsection{とりあえずのまとめ}

ここで、もう一度、整理してみます。

be動詞はイコールのいいしかないのに対して、
be動詞以外の一般動詞の意味はいろいろですね。

一般動詞の意味や発音や綴りはそのつど覚えることがだいじだといいましたね。

\myMouse

前のスライドでは表で整理しましたが、
図で整理するとこうなります。

みなさんも自分なりに整理するといいとおもいます。

\section{わたし、あなた、それ以外}

次に移ります。
「わたし、あなた、それ以外」というタイトルにしました。

なんのことなのか、よくわからないという人がいるかもしれませんが、
おつきあいください。

\subsection{人称}

一人称という用語を聞いたことがありますか。
一人称とは「話し手」つまり「自分」のこと。

英語でいえばiとWeだけです。

二人称は「聞き手」のこと。別の言葉でいえば、「眼の前にいるあなた」のことです。
英語でいえばYouだけです。

そして三人称というのがあります。心配しないでいいです。
これで人称は全てです。

三人称とは1人称2人称以外のすべてをひっくるめたものです。
I, we, you以外のすべて。
\myMouse
具体的にはどんなものが思い浮かびますか。

そういろいろありますね。

\subsection{練習問題}

では、練習問題です。

時間を取りますから英文をノートに写して、主語が何人称かかんがえてくださいv。

2\,min53\,sec

\section{動詞のかたち}
\subsection{主語によって動詞のかたちが定まる}

まず、前回の復習、おさらいをしておきましょう。

前回は「人称」について学習しました。
「人称」とはずいぶんもったいぶった用語ですが、
実はなんてことはありませんでした。
一人称二人称三人称とありますが
\begin{itemize}
 \item 一人称は話し手つまり自分のことで具体的にはIとYouだけ
 \item 二人称は聞き手つまり目の前にいる相手のこと。具体的にはYouだけ
 \item 三人称は一人称二人称以外のすべて。I, We, You 以外は何でも三人称
\end{itemize}
それだけのことでした。みなさんじゅうぶんに理解してくれたこととおもいます。

なぜ「人称」について学習したのか、もうすぐわかってもらえるとおもいます。

\myMouse

それでは、これからの授業の流れです。

きょうば「主語によって動詞の形が定まる」ことについて学習します。
これだけではまだピンとこないでしょうが、心配することはありません。
これから順を追って学習します。

\myMouse



最初の例文をみてください。
1 I like music.(わたしは音楽が好きだ)

2 We like music.主語がIからWeに変わりましたが「音楽が好きだ」という点では同じです。

3 You like

4 He likesあれ、sがついてます。このsが気になりますが、とりあえずここは無視して意味だけ確認しましょう。「彼は音楽が好きだ」

5 She likesこれもsがついてます。ただ「音楽が好きだ」という点では同じ

6 Paul likesこれにもsがついてます。Paulは男性の名前。やはり「音楽が好き」

7 They likeもとにもどりました。sはついていません。ただのlikeです。

意味はすべて「好きだ」の意味でした。

ではさきほど「」likesのsが気にはなるけどとりあえず無視する」といったのですが、
その点に戻りましょう。

全部「音楽が好きだ」という意味でしたが、
likeというときとlikesというときがありますね。

\myMouse

(赤が主語動詞が青で順番に表示。最後にぜんぶ黒に戻ったら)

どう考えたらいいでしょうか。

likeとlikesをどう使い分ければいのか考えてみましょう。

ここはフィーリングで、なんとなくlikeとかlikesとかやるのでしょうか。

もちろんそんなわけはありません。


お互いに意思をきちんと伝え合うことがことばの役割です。
なんとなくといった、あやふやなことで正確な意思伝達はできるはずはない。
ことばはきちんとしたきまりがあります。
という話はこれまでもしましたよね。

みなさんなりに予想がつきますか。

(時間があれば発言を促す)

\paragraph{Topics for Today}
それではきょうのポイントです。

主語が1人称2人称のときは、そのままの形を使います。

3人称のときは、それほど単純ではありません。

3人称のときは、単数形なら動詞はそのままの形ではなく、お尻にSをつけるのが約束です。
同じ3人称でも、複数形ならそのままの形でOKです。

ということは、ほとんどの場合、そのままの形でいいことになります。
ただ、主語が3人称単数のときは、そのままな形ではなく、お尻にsをつけるということになりますね。
これがきょうのポイントです。


前回、人称という用語を学習していただいた理由がわかってくれた人もいるとおもいます。

スライドをノートに移しておきましょう。
%%%%%%%%%%%%%%%%%%%
\newpage
 \section{3単現}
きょうは3単現ということを勉強します。
3単現って、ごろがいいですね。
何を意味するのでしょうか。
それはあとで勉強するので、まずは確認です。

主語がなんであるかによって、
動詞の形が変わりましたね。

ところで、前回学習したのはI like music. You like music.などすべて現在のことを表す文でしたね。そのことも踏まえて、整理するとこうなります。
「現在のことを表すのに主語が3人称で単数のときはsをつける」となります。

そして、このsのことを「3人称単数現在のs」といいます。

更に縮めて「3単現のs」といいます。

She plays the guitar.

He lives in Boston.

Naomi drinks coffee.

これらの動詞はすべて3単現のsがついていますね。

ところで、plays, livesは\textipa{/z/}、drinksは\textipa{/s/}。この発音の違いに気づいた人は鋭いと思います。

どういうときに「ズ」と発音し、どういうときに「ス」と発音するかについては、
あらためて4整理することとして、きょうのところは
日頃からどう発音するのか意識しておくことがだいじですというにとどめます。
\subsection{Exercises}


ではここで練習問題です。

(アドリブ)


\begin{enumerate}
 \item 空所のうしろのto schoo by busと内容がうまく合いそうなのは、「行く」の意味のgoですね。
主語は1人称ですから,そのままの形でgoが正解
 \item 空所の後ろのcoffeeと合いそうなのはdrinkしかなさそうです。主語はShe、つまり3人称単数ですから3単現のsをつけたdrinksが正解
 \item 空所のうしろのbreadから考えてeatを選びます。主語はTom、3人称単数ですからsをつけeatsとします
 \item 空所のうしろのin Bostonをいかせるのはliveだけ。主語はJennifer、三人称単数ですからlivesが正解
 \item 「英語を話す」の意味になるとかんがえます。主語は三人称複数ですから、動詞はそのままのかたち。speakが正解ですね。
\end{enumerate}

%%%%%%%%%%%%%%%%%%%
\newpage
\section{s以外の3単現}

話を先に進めます。

\begin{enumerate}
 \item She drinks coffee every morning.では、主語が三人称\\
主語はShe。3人称単数ですから、現在のことを表すには、drnkという動詞にsがつくのはわかりますね。
 \item He goes to school by bus. も主語がHe、つまり3人称単数ですからsがつくはずです。ところが、よく見るとgoes、sではなくesがついています。goの3単現在はgoesと覚えるしかありません。発音はgoesです
 \item どうようにteachも3単現はesをつけてteachesとなります。発音は「teachス」ではなくteaches\textipa{/iz/}
 \item washの3単現もesをつけてwashesです。washesも語尾の発音は\textipa{/iz}
 \item watchの3単現もesをつけてwatchesとなります。watchesも語尾の発音は\textipa{/iz}
 \item No. 6は、またちょっぴりやっかいです。haveの3単現はhavesではなくhasとなります。発音は\textipa{/z/}
\end{enumerate}

\paragraph{表の作成}

では、ここで表を作成してみます。

これまでに勉強したことをもとに、この表の3単現の列を完成させましょう。
発音の列は埋めなくていいですよ。
あとでいっしょに確認します。

\mySagyo

\begin{enumerate}
 \item playの3単現はplays.
 \item drinkの3多言はdrnks.
 \item goの3単現はgoes.
 \item teachの3単現はteaches.
 \item washの3単現はwashes.
 \item watchの3単現はwatches.
 \item haveの3単現はhas.
\end{enumerate}

綴りの確認をしました。

大原則は、「お尻にsをつける」でした。

例外としてesをつける動詞として、
go, teach, wash, watchを勉強しました。

またhaveは特別な動詞でhaveの3単現はhasと覚えるしかありません。

\paragraph{発音}

発音についてもかんたんに確認しておきましょう。

3単現のsは「ズ」と発音することがおおいです。

playsがその例です。

ただ、drinksのsは「ス」と発音します。

\paragraph{復習}

では、前回の復習、おさらいをしましょう。

3単現の原則は動詞の元の形、これを原形といいましたが、
原形のお尻にsをつけるでしたね。
1のplay, plays、弐のdrink, drinksは原則どおりでした。
これがあくまでも原則でした。


ただ、一部の動詞については例外的にお尻にesをつける3単現がありました。
具体的には3 go, goes, 4 teach, teaches, 5 wash, washes, 6 watch, watches
については例外として覚えようと言いました。

もうひとつhaveの3単現はhasという特別な形をしていて、これについても
覚えようと言いました。

もうひとつ、発音についてです。

3単現のsの発音は\textipa{/z/}と濁るときと, \textipa{/s/}と濁らない場合がありあるといいました。
この区別は重要ですが説明はもう少し待ってくださいといいました。
ただ\textipa{/z/}のように濁るほうが多いといいました。

あと、goesも\textipa{/z/}でした。
teaches, washes, watchesは\textipa{/iz/}となりましたね。これは覚えておきましょうといいました。

では、もう一度発音練習です。\marginnote{\tiny \faVolumeUp\,4\,min22\,sec}



\section{練習問題}

では、練習問題です。

\mySagyo

\begin{enumerate}
 \item 「飼っている」はhaveでいいでしょう。haveの3単現はhasでしたね
 \item 「教える」teachの3単現はteaches
 \item 「洗車する」とは「車を洗う」つまりwashの3単現washes
 \item 「行く」goの3単現はgoes
 \item 「見る」watchの3単現はwatches
 \item 「勉強する」studyの3単現はなんでしょう。正解はstudies。yをiに変えてesをつけます。
\end{enumerate}

\marginnote{\tiny \faVolumeUp\,3\,min6\,sec}

\paragraph{dangerous bend}

今日は新しい記号を紹介します。
スライドの右下を見てください。黄色い標識がありますね。
この標識は、「この先危険な曲がり角がありますよ。注意してください」という、
道路標識。


ただわたしたちの授業で使うときは次の意味です。
この記号が出てきたら、そこは少し難しい部分だと覚えてください。しっかりと集中して学ぶことが重要です。




\dbend\hspace{10pt}%
ちょっとむずかしい事項ですという標識
でも、しっかり身につけると実力アップにつながります

\bigskip

\dbend\,\dbend\hspace{10pt}%
かなりむずかしい事項ですという標識
いまはまだ理解できなくてもだいじょうぶ

この標識といいますか記号はこれからも使います。

皆さんも、この記号が出てきたときは、特に注意して学んでみてください。

あ、これは実際にさっきの標識が使われているところの写真です。
フランスの風景です。


では、先ほど作成した表に追加しておきましょう。\marginnote{\tiny \faVolumeUp\,5\,min41\,sec}

\section{まとめ}

それでは総まとめ、おさらいです。

この単元では、いろいろと学習しました。

\begin{enumerate}
 \item 動詞はbe動詞と一般動詞におおきく分類できること
 \item 1人称2人称3人称
 \item 主語が3人称で単数のときは動詞にsがつくこと
 \item いくつか注意するべき3単現の動詞があること
\end{enumerate}
を学習しました。

みなさんの頭にどの程度学習した事柄が浮かぶでしょうか。

この単元では、いろいろと重要な事項を学習しました。
余裕があったら、復習してください。

\chapter{Pronunciation}
\section{introduction}
これから発音の基礎について、かんたんに勉強します。

授業の流れは次のとおりです。

まず、母音と子音について学習し、
つぎに有声音と無声音について学習します。


\section{母音と子音}
まず「母音」です。

母音とは、日本語の「あいうえお」に近い音のことを言います。

「子音」とは、母音以外の音のことを言います。

\section{母音字と子音字}

「母音字」とは アルファベットで「母音」をあらわす文字: aeiou

「子音字」とは、残りのアルファベットのこと。
aeiou以外のすべてです。

\section{確認}

ややこしいのですが、
母音とか子音というのは、音についての話。

母音字・子音字というのは、字、つまり文字についての話です。

\section{実際の単語で確認}
それでは実際の単語で確認していきましょう。

\section{cat}

catという単語についてかんがえましょう。

cは子音字、aは母音字、tは子音字。

では音でいうとどうでしょうか。

Cは文字としては子音字ですから、音は子音になります。

aは母音字ですから、音は母音。

tは文字としては子音字ですから、音は子音になります。

実際の発音はどうなるでしょうか。

\textipa{/k/}
\textipa{/\ae /}
\textipa{/t/}

\section{bed}

catという単語についてかんがえましょう。

bは子音字、eは母音字、dは子音字。

では音でいうとどうでしょうか。

bは文字としては子音字ですから、音は子音になります。

eは母音字ですから、音は母音。

dは文字としては子音字ですから、音は子音になります。

実際の発音はどうなるでしょうか。

\textipa{/b/}
\textipa{/e/}
\textipa{/d/}

\section{sit}

sitという単語についてかんがえましょう。

sは子音字、iは母音字、tは子音字。

では音でいうとどうでしょうか。

sは文字としては子音字ですから、音は子音になります。

iは母音字ですから、音は母音。

tは文字としては子音字ですから、音は子音になります。

実際の発音はどうなるでしょうか。

\textipa{/s/}
\textipa{/I/}
\textipa{/t/}

\section{dog}

dogという単語についてかんがえましょう。

dは子音字、oは母音字、gは子音字。

では音でいうとどうでしょうか。

dは文字としては子音字ですから、音は子音になります。

oは母音字ですから、音は母音。

gは文字としては子音字ですから、音は子音になります。

実際の発音はどうなるでしょうか。

\textipa{/d}
\textipa{/\textscripta /}
\textipa{/g/}

\section{bus}

busという単語についてかんがえましょう。

bは子音字、uは母音字、sは子音字。

では音でいうとどうでしょうか。

bは文字としては子音字ですから、音は子音になります。

uは母音字ですから、音は母音。

sは文字としては子音字ですから、音は子音になります。

実際の発音はどうなるでしょうか。

\textipa{/b/}
\textipa{/\textturnv}
\textipa{/s/}

\textipa{[""Ekspl@"neIS@n]}
%宣言型マクロ \tipaencoding と共にショートカット文字を使用する.
{\tipaencoding [""Ekspl@"neIS@n]}
IPA 環境内でショートカット文字を使用する.
\begin{IPA}
  [""Ekspl@"neIS@n]
\end{IPA}
通常のテキスト内でマクロ名を指定する.
[\textsecstress\textepsilon kspl\textschwa
\textprimstress ne\textsci\textesh\textschwa n]
\end{document}

