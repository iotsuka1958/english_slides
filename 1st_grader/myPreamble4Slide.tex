%%%%%%%%%%%%%%%%%%%%%%%%%%%
%% さまざまなアイコン
%%%%%%%%%%%%%%%%%%%%%%%%%%%
%\usepackage{fontawesome}
\usepackage{fontawesome5}
\usepackage{figchild}
\usepackage{twemojis}
\usepackage{utfsym}
\usepackage{bclogo}
\usepackage{marvosym}
\usepackage{fontmfizz}
\usepackage{pifont}
\usepackage{phaistos}
\usepackage{worldflags}
\usepackage{jigsaw}
\usepackage{tikzlings}
\usepackage{tikzducks}
\usepackage{scsnowman}
\usepackage{epsdice}
\usepackage{halloweenmath}
\usepackage{svrsymbols}
\usepackage{countriesofeurope}
\usepackage{tipa}
\usepackage{manfnt}
%%%%%%%%%%%%%%%%%%%%%%%%%%%
\usepackage{tikz}
\usetikzlibrary{calc,patterns,decorations.pathmorphing,backgrounds}
\usepackage{tcolorbox}
\usepackage{tikzpeople}
\usepackage{circledsteps}
\usepackage{xcolor}
\usepackage{amsmath}
\usepackage{booktabs}
\usepackage{chronology}
\usepackage{signchart}
%%%%%%%%%%%%%%%%%%%%%%%%%%%
%% 場合分け
%%%%%%%%%%%%%%%%%%%%%%%%%%%
\usepackage{cases}
%%%%%%%%%%%%%%%%%%%%%%%%%%
\usepackage{pdfpages}
%%%%%%%%%%%%%%%%%%%%%%%%%%%
%% 音声リンク表示
\newcommand{\myaudio}[1]{\href{#1}{\faVolumeUp}}
%%%%%%%%%%%%%%%%%%%%%%%%%%
%% \myAnch{<名前>}{<色>}{<テキスト>}
%% 指定のテキストを指定の色の四角枠で囲み, 指定の名前をもつTikZの
%% ノードとして出力する. 図には remember picture 属性を付けている
%% ので外部から参照可能である.
\newcommand*{\myAnch}[3]{%
  \tikz[remember picture,baseline=(#1.base)]
    \node[draw,rectangle,line width=1pt,#2] (#1) {\normalcolor #3};
}
%%%%%%%%%%%%%%%%%%%%%%%%%%
%% \myEmph コマンドの定義
%%%%%%%%%%%%%%%%%%%%%%%%%%
%\newcommand{\myEmph}[3]{%
%    \textbf<#1>{\color<#1>{#2}{#3}}%
%}
\usepackage{xparse} % xparseパッケージの読み込み
\NewDocumentCommand{\myEmph}{O{} m m}{%
    \def\argOne{#1}%
    \ifx\argOne\empty
        \textbf{\color{#2}{#3}}% オプション引数が省略された場合
    \else
        \textbf<#1>{\color<#1>{#2}{#3}}% オプション引数が指定された場合
    \fi
}
%%%%%%%%%%%%%%%%%%%%%%%%%%%
%%%%%%%%%%%%%%%%%%%%%%%%%%%
%% 文末の上昇イントネーション記号 \myRisingPitch
%% 通常のイントネーション \myDownwardPitch
%% https://note.com/dan_oyama/n/n8be58e8797b2
%%%%%%%%%%%%%%%%%%%%%%%%%%%
\newcommand{\myRisingPitch}{
\begin{tikzpicture}[scale=0.3,baseline=0.3]
\draw[->,>=stealth] (0,0) to[bend right=45] (1,1);
\end{tikzpicture}
}
\newcommand{\myDownwardPitch}{
\begin{tikzpicture}[scale=0.3,baseline=0.3]
\draw[->,>=stealth] (0,1) to[bend left=45] (1,0);
\end{tikzpicture}
}
%%%%%%%%%%%%%%%%%%%%%%%%%%%%
%\AtBeginSection[%
%]{%
%  \begin{frame}[plain]\frametitle{授業の流れ}
%     \tableofcontents[currentsection]
%   \end{frame}%
%}
