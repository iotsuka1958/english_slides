%%%%%%%handoutオプションだとエラーになるTeXソースがある
%%%%%%%さてどうしたあものか
%%%%%%%handoutオプションをつけないと印刷するには枚数が増えすぎる
\documentclass[jafontscale=0.9247,dvipsnames,book,twoside]{jlreq}
\usepackage{geometry,notebeamer}
\geometry{margin = .75in}
\ModifyHeading{chapter}{label_format={Chapter~\thechapter}}
\begin{document}
\tableofcontents 

\newpage
\chapter{Orientation}
\includebeamer
[ color = Gray, ratio= .57, lines = 32, nup = 5,
  pages = -, lefthead = orientation 1, righthead = Page~\thepage
] { ../orientation/orientation_handout.pdf }

\includebeamer
[ color = Gray, ratio= .57, lines = 32, nup = 5,
  pages = -, lefthead = orientation 2, righthead = Page~\thepage
] { ../orientation/orientation_2.pdf }


\chapter{主語と動詞}
\includebeamer
[ color = Gray, ratio= .57, lines = 32, nup = 5,
  pages = -, lefthead = 1st\_grader/002\_sv, righthead = Page~\thepage
] { 002_sv_handout.pdf }

\chapter{be動詞と一般動詞}
\addcontentsline{toc}{section}{3.1\hspace{25pt}be動詞}
\includebeamer
[ color = Gray, ratio= .57, lines = 32, nup = 5,
  pages = -, lefthead = 1st\_grader/003\_be, righthead = Page~\thepage
] { 003_be_handout.pdf }

\addcontentsline{toc}{section}{3.2\hspace{25pt}一般動詞}
\includebeamer
[ color = Gray, ratio= .57, lines = 32, nup = 5,
  pages = -, lefthead = 1st\_grader/004\_verb, righthead = Page~\thepage
] { 004_verb_handout.pdf }


\chapter{目的語}
\includebeamer
[ color = Gray, ratio= .57, lines = 32, nup = 5,
  pages = -, lefthead = 1st\_grader/004\_verb\_object, righthead = Page~\thepage
] { 004_verb_object_handout.pdf }


\chapter{代名詞}
\includebeamer
[ color = Gray, ratio= .57, lines = 32, nup = 5,
  pages = -, lefthead = 1st\_grader/005\_pronoun, righthead = Page~\thepage
] { 005_pronoun_handout.pdf }

\chapter{単数と複数}
\includebeamer
[ color = Gray, ratio= .57, lines = 32, nup = 5,
  pages = -, lefthead = 1st\_grader/005\_singular\_plural, righthead = Page~\thepage
] { 005_singular_plural_handout.pdf }

\chapter{否定文}
\addcontentsline{toc}{section}{6.1\hspace{25pt}be動詞の否定}
\includebeamer
[ color = Gray, ratio= .57, lines = 32, nup = 5,
  pages = -, lefthead = 1st\_grader/006\_negative\_be, righthead = Page~\thepage
] { 006_negative_be_handout.pdf }

\addcontentsline{toc}{section}{6.2\hspace{25pt}一般動詞の否定}
\includebeamer
[ color = Gray, ratio= .57, lines = 32, nup = 5,
  pages = -, lefthead = 1st\_grader/007\_negative\_do, righthead = Page~\thepage
] { 007_negative_do_handout.pdf }

\chapter{疑問文}
\addcontentsline{toc}{section}{7.1\hspace{25pt}be動詞の疑問文}
\includebeamer
[ color = Gray, ratio= .57, lines = 32, nup = 5,
  pages = -, lefthead = 1st\\grader/008\_question\_be, righthead = Page~\thepage
] { 008_question_be_handout.pdf }

\addcontentsline{toc}{section}{7.2\hspace{25pt}be動詞の疑問文への答え方}
\includebeamer
[ color = Gray, ratio= .57, lines = 32, nup = 5,
  pages = -, lefthead = 1st\_grader/009\_answer\_be, righthead = Page~\thepage
] { 009_answer_be_handout.pdf }

\addcontentsline{toc}{section}{7.3\hspace{25pt}一般動詞の疑問文}
\includebeamer
[ color = Gray, ratio= .57, lines = 32, nup = 5,
  pages = -, lefthead = 1st\_grader/010\_question\_do, righthead = Page~\thepage
] { 010_question_do_handout.pdf }

\addcontentsline{toc}{section}{7.4\hspace{25pt}一般動詞の疑問文への答え方}
\includebeamer
[ color = Gray, ratio= .57, lines = 32, nup = 5,
  pages = -, lefthead = 1st\_grader/011\_answer\_do, righthead = Page~\thepage
] { 011_answer_do_handout.pdf }


\chapter{現在進行形}
\includebeamer
[ color = Gray, ratio= .57, lines = 32, nup = 5,
  pages = -, lefthead = 021\_is\_ing\_intro, righthead = Page~\thepage
] {021_is_ing_intro_handout.pdf }

\includebeamer
[ color = Gray, ratio= .57, lines = 32, nup = 5,
  pages = -, lefthead = 022\_is\_ing\_negative, righthead = Page~\thepage
] {022_is_ing_negative_handout.pdf }

\includebeamer
[ color = Gray, ratio= .57, lines = 32, nup = 5,
  pages = -, lefthead = 023\_is\_ing\_question, righthead = Page~\thepage
] {023_is_ing_question.pdf }

\chapter{過去形}

\addcontentsline{toc}{section}{8.1\hspace{15pt}be動詞の過去形}
\includebeamer
[ color = Gray, ratio= .57, lines = 32, nup = 5,
  pages = -, lefthead = 1st\\grader/024\_past\_be, righthead = Page~\thepage
] { 024_past_be_handout.pdf }

\addcontentsline{toc}{section}{8.2\hspace{25pt}一般動詞の過去形}
\includebeamer
[ color = Gray, ratio= .57, lines = 32, nup = 5,
  pages = -, lefthead = 1st\_grader/025\_past\_do, righthead = Page~\thepage
] { 025_past_do_handout.pdf }

\addcontentsline{toc}{section}{8.3\hspace{25pt}一般動詞の過去形の否定}
\includebeamer
[ color = Gray, ratio= .57, lines = 32, nup = 5,
  pages = -, lefthead = 1st\_grader/026\_past\_didnot, righthead = Page~\thepage
] { 026_past_didnot_handout.pdf }

\addcontentsline{toc}{section}{8.4\hspace{25pt}一般動詞の過去形の疑問文}
\includebeamer
[ color = Gray, ratio= .57, lines = 32, nup = 5,
  pages = -, lefthead = 1st\_grader/027\_past\_did\_you, righthead = Page~\thepage
] { 027_past_did_you_handout.pdf }

\chapter{過去進行形}
\includebeamer
[ color = Gray, ratio= .57, lines = 32, nup = 5,
  pages = -, lefthead = 2nd\_grader/005\_was\_ing\_intro, righthead = Page~\thepage
] { ../2nd_grader/005_was_ing_intro_handout.pdf }

\chapter{未来を表す表現}
\addcontentsline{toc}{section}{10.1\hspace{25pt}be going to--}
\includebeamer
[ color = Gray, ratio= .57, lines = 32, nup = 5,
  pages = -, lefthead = 2nd\_grader/011\_be\_going\_to, righthead = Page~\thepage
] { ../2nd_grader/011_be_going_to_handout.pdf }

\addcontentsline{toc}{section}{10.2\hspace{25pt}will}
\includebeamer
[ color = Gray, ratio= .57, lines = 32, nup = 5,
  pages = -, lefthead = 2nd\_grader/012\_will, righthead = Page~\thepage
] { ../2nd_grader/012_will_handout.pdf }

\chapter{助動詞}
\addcontentsline{toc}{section}{10.2\hspace{25pt}will}
\includebeamer
[ color = Gray, ratio= .57, lines = 32, nup = 5,
  pages = -, lefthead = /012\_can, righthead = Page~\thepage
] { ./012_can_handout.pdf }
\end{document}
