%%%%%%%handoutオプションだとエラーになるTeXソースがある
%%%%%%%さてどうしたあものか
%%%%%%%handoutオプションをつけないと印刷するには枚数が増えすぎる
\documentclass[jafontscale=0.9247,dvipsnames,book,twoside]{jlreq}
\usepackage[no-math,deluxe,haranoaji]{luatexja-preset}
\usepackage{geometry,notebeamer}
\usepackage{datetime}
\geometry{margin = .75in}
%\ModifyHeading{chapter}{label_format={Chapter~\thechapter}}
\usepackage{url}
\usepackage[bookmarks=true,bookmarksnumbered=true,
  pdftitle={slide_note},
  pdfauthor={iotsuka},
  pdfkeywords={englush; slide},
  pdflang=ja-JP
]{hyperref}
%%%%%%%%%%%%%%%%%%%%%%%%%%%%%%%%
\begin{document}
%%%%%%%%%%%%%%%%%%%%%%%%%%%%%%%%%%%%%%%%%%%
\thispagestyle{empty}
\begin{flushright}\tiny
\xxivtime/\mdyyyydate\today\\
\texttt{/\jobname}\\
Ver. 3.141592
\end{flushright}


%\maketitle
\mbox{}\vfill
\noindent{\bfseries\LARGE \mbox{}\hfill{}}\\[5pt]
\noindent{\bfseries\LARGE \mbox{}\hfill{}Slides for English Lessons}

\noindent\rule{\textwidth}{7pt}

\vfill
\vfill
\vfill

\vfill

\mbox{}\hfill{}{\Large%
\mbox{}\hfill{}iotsuka\\[10pt]
%\mbox{}\hfill{}令和3年7月%
}

\vfill

%%%%%%%%%%%%%%%%%%%%%%%%%%%%%%%%%%%%%%%
\cleardoublepage
\tableofcontents 
%%%%%%%%%%%%%%%%%%%%%%%%%%%%%%%%%
\newpage
\chapter{Orientation}
\addcontentsline{toc}{section}{1.1\hspace{25pt}授業の進め方}
\includebeamer
[ color = Gray, ratio= .57, lines = 32, nup = 5,
  pages = -, lhead = \url{orientation/orientation_1}, rhead = Page~\thepage
] { ../orientation/orientation_handout.pdf }

\addcontentsline{toc}{section}{1.2\hspace{25pt}英語ということば}
\includebeamer
[ color = Gray, ratio= .57, lines = 32, nup = 5,
  pages = -, lhead = \url{orientation/orientation_2}, rhead = Page~\thepage
] { ../orientation/orientation_2.pdf }

%%%%%%%%%%%%%%%%%%%%%%%%%%%%%%%%%
\chapter{Alphabet}
\includebeamer
[ color = Gray, ratio= .57, lines = 32, nup = 5,
  pages = -, lhead = \url{1st_grader/alphabet}, rhead = Page~\thepage
] { 001_alphabet_handout.pdf }
%%%%%%%%%%%%%%%%%%%%%%%%%%%%%%%%%
\chapter{主語と動詞}
\includebeamer
[ color = Gray, ratio= .57, lines = 32, nup = 5,
  pages = -, lhead = 1st\_grader/002\_sv, rhead = Page~\thepage
] { 002_sv_handout.pdf }

\chapter{be動詞と一般動詞}
\addcontentsline{toc}{section}{4.1\hspace{25pt}be動詞}
\includebeamer
[ color = Gray, ratio= .57, lines = 32, nup = 5,
  pages = -, lhead = \url{1st_grader/003_be}, rhead = Page~\thepage
] { 003_be_handout.pdf }

\addcontentsline{toc}{section}{4.2\hspace{25pt}一般動詞}
\includebeamer
[ color = Gray, ratio= .57, lines = 32, nup = 5,
  pages = -, lhead = \url{1st_grader/004_verb}, rhead = Page~\thepage
] { 004_verb_handout.pdf }


\chapter{目的語}
\includebeamer
[ color = Gray, ratio= .57, lines = 32, nup = 5,
  pages = -, lhead = \url{1st_grader/004_verb_object}, rhead = Page~\thepage
] { 004_verb_object_handout.pdf }


\chapter{代名詞}
\includebeamer
[ color = Gray, ratio= .57, lines = 32, nup = 5,
  pages = -, lhead = \url{1st_grader/005_pronoun}, rhead = Page~\thepage
] { 005_pronoun_handout.pdf }

\chapter{単数と複数}
\includebeamer
[ color = Gray, ratio= .57, lines = 32, nup = 5,
  pages = -, lhead = \url{1st_grader/005_singular_plural}, rhead = Page~\thepage
] { 005_singular_plural_handout.pdf }

\chapter{否定文}
\addcontentsline{toc}{section}{8.1\hspace{25pt}be動詞の否定}
\includebeamer
[ color = Gray, ratio= .57, lines = 32, nup = 5,
  pages = -, lhead = \url{1st_grader/006_negative_be}, rhead = Page~\thepage
] { 006_negative_be_handout.pdf }

\addcontentsline{toc}{section}{8.2\hspace{25pt}一般動詞の否定}
\includebeamer
[ color = Gray, ratio= .57, lines = 32, nup = 5,
  pages = -, lhead = \url{1st_grader/007_negative_do}, rhead = Page~\thepage
] { 007_negative_do_handout.pdf }

\chapter{疑問文}
\addcontentsline{toc}{section}{9.1\hspace{25pt}be動詞の疑問文}
\includebeamer
[ color = Gray, ratio= .57, lines = 32, nup = 5,
  pages = -, lhead = \url{1st_grader/008_question_be}, rhead = Page~\thepage
] { 008_question_be_handout.pdf }

\addcontentsline{toc}{section}{9.2\hspace{25pt}be動詞の疑問文への答え方}
\includebeamer
[ color = Gray, ratio= .57, lines = 32, nup = 5,
  pages = -, lhead = \url{1st_grader/009_answer_be}, rhead = Page~\thepage
] { 009_answer_be_handout.pdf }

\addcontentsline{toc}{section}{9.3\hspace{25pt}一般動詞の疑問文}
\includebeamer
[ color = Gray, ratio= .57, lines = 32, nup = 5,
  pages = -, lhead = \url{1st_grader/010_question_do}, rhead = Page~\thepage
] { 010_question_do_handout.pdf }

\addcontentsline{toc}{section}{9.4\hspace{25pt}一般動詞の疑問文への答え方}
\includebeamer
[ color = Gray, ratio= .57, lines = 32, nup = 5,
  pages = -, lhead = \url{1st_grader/011_answer_do}, rhead = Page~\thepage
] { 011_answer_do_handout.pdf }


\chapter{現在進行形}
\addcontentsline{toc}{section}{10.1\hspace{25pt}現在進行形とは}
\includebeamer
[ color = Gray, ratio= .57, lines = 32, nup = 5,
  pages = -, lhead = \url{021_is_ing_intro}, rhead = Page~\thepage
] {021_is_ing_intro_handout.pdf }

\addcontentsline{toc}{section}{10.2\hspace{25pt}現在進行形の否定}
\includebeamer
[ color = Gray, ratio= .57, lines = 32, nup = 5,
  pages = -, lhead = \url{022_is_ing_negative}, rhead = Page~\thepage
] {022_is_ing_negative_handout.pdf }

\addcontentsline{toc}{section}{10.3\hspace{25pt}現在進行形の疑問文}
\includebeamer
[ color = Gray, ratio= .57, lines = 32, nup = 5,
  pages = -, lhead = \url{023_is_ing_question}, rhead = Page~\thepage
] {023_is_ing_question.pdf }

\chapter{過去形}

\addcontentsline{toc}{section}{11.1\hspace{25pt}be動詞の過去形}
\includebeamer
[ color = Gray, ratio= .57, lines = 32, nup = 5,
  pages = -, lhead = \url{1st_grader/024_past_be}, rhead = Page~\thepage
] { 024_past_be_handout.pdf }

\addcontentsline{toc}{section}{11.2\hspace{25pt}一般動詞の過去形}
\includebeamer
[ color = Gray, ratio= .57, lines = 32, nup = 5,
  pages = -, lhead = \url{1st_grader/025_past_do}, rhead = Page~\thepage
] { 025_past_do_handout.pdf }

\addcontentsline{toc}{section}{11.3\hspace{25pt}一般動詞の過去形の否定}
\includebeamer
[ color = Gray, ratio= .57, lines = 32, nup = 5,
  pages = -, lhead = \url{1st_grader/026_past_didnot}, rhead = Page~\thepage
] { 026_past_didnot_handout.pdf }

\addcontentsline{toc}{section}{11.4\hspace{25pt}一般動詞の過去形の疑問文}
\includebeamer
[ color = Gray, ratio= .57, lines = 32, nup = 5,
  pages = -, lhead = \url{1st_grader/027_past_did_you}, rhead = Page~\thepage
] { 027_past_did_you_handout.pdf }

\chapter{過去進行形}
\includebeamer
[ color = Gray, ratio= .57, lines = 32, nup = 5,
  pages = -, lhead = \url{2nd_grader/005_was_ing_intro}, rhead = Page~\thepage
] { ../2nd_grader/005_was_ing_intro_handout.pdf }

\chapter{未来を表す表現}
\addcontentsline{toc}{section}{13.1\hspace{25pt}be going to--}
\includebeamer
[ color = Gray, ratio= .57, lines = 32, nup = 5,
  pages = -, lhead = \url{2nd_grader/011_be_going_to}, rhead = Page~\thepage
] { ../2nd_grader/011_be_going_to_handout.pdf }

\addcontentsline{toc}{section}{13.2\hspace{25pt}will}
\includebeamer
[ color = Gray, ratio= .57, lines = 32, nup = 5,
  pages = -, lhead = \url{2nd_grader/012_will}, rhead = Page~\thepage
] { ../2nd_grader/012_will_handout.pdf }
%%%%%%%%%%%%%%%%%%%%%%%%%%%%%%%%
\chapter{現在完了}

\addcontentsline{toc}{section}{14.1\hspace{25pt}現在完了の基本}
\includebeamer
[ color = Gray, ratio= .57, lines = 32, nup = 5,
  pages = -, lhead = \url{3rd_grader/011_have_pp_intro}, rhead = Page~\thepage
] { ../3rd_grader/011_have_pp_intro_handout.pdf }

\addcontentsline{toc}{section}{14.2\hspace{25pt}継続}
\includebeamer
[ color = Gray, ratio= .57, lines = 32, nup = 5,
  pages = -, lhead = \url{3rd_grader/012_have_pp_keizoku}, rhead = Page~\thepage
] { ../3rd_grader/012_have_pp_keizoku_handout.pdf }

\addcontentsline{toc}{section}{14.3\hspace{25pt}経験}
\includebeamer
[ color = Gray, ratio= .57, lines = 32, nup = 5,
  pages = -, lhead = \url{3rd_grader/013_have_pp_keiken}, rhead = Page~\thepage
] { ../3rd_grader/013_have_pp_keiken_handout.pdf }

\addcontentsline{toc}{section}{14.4\hspace{25pt}結果}
\includebeamer
[ color = Gray, ratio= .57, lines = 32, nup = 5,
  pages = -, lhead = \url{3rd_grader/014_have_pp_kekka}, rhead = Page~\thepage
] { ../3rd_grader/014_have_pp_kekka_handout.pdf }
%%%%%%%%%%%%%%%%%%%%%%%%%%%%%%%%
\chapter{命令文・感嘆文}

\addcontentsline{toc}{section}{15.1\hspace{25pt}命令文}
\includebeamer
[ color = Gray, ratio= .57, lines = 32, nup = 5,
  pages = -, lhead = \url{1st_grader/032_imperative}, rhead = Page~\thepage
] { ../1st_grader/032_imperative_handout.pdf }

\addcontentsline{toc}{section}{15.2\hspace{25pt}感嘆文}
\includebeamer
[ color = Gray, ratio= .57, lines = 32, nup = 5,
  pages = -, lhead = \url{1st_grader/033_exclamatory}, rhead = Page~\thepage
] { ../1st_grader/033_exclamatory_handout.pdf }
%%%%%%%%%%%%%%%%%%%%%%%%%%%%%%%%%
\chapter{There is 名詞}

\includebeamer
[ color = Gray, ratio= .57, lines = 32, nup = 5,
  pages = -, lhead = \url{2nd_grader/001_there_is}, rhead = Page~\thepage
] { ../2nd_grader/001_there_is_handout.pdf }
%%%%%%%%%%%%%%%%%%%%%%%%%%%%%%%%%
\chapter{品詞}

\includebeamer
[ color = Gray, ratio= .57, lines = 32, nup = 5,
  pages = -, lhead = \url{2nd_grader/020_part_of_speech}, rhead = Page~\thepage
] { ../2nd_grader/020_part_of_speech_handout.pdf }
%%%%%%%%%%%%%%%%%%%%%%%%%%%%%%%%%
\chapter{助動詞}
\addcontentsline{toc}{section}{18.1\hspace{25pt}can}
\includebeamer
[ color = Gray, ratio= .57, lines = 32, nup = 5,
  pages = -, lhead = \url{1st_grader/012_can}, rhead = Page~\thepage
] { ./012_can_handout.pdf }

\addcontentsline{toc}{section}{18.2\hspace{25pt}must}
\includebeamer
[ color = Gray, ratio= .57, lines = 32, nup = 5,
  pages = -, lhead = \url{2nd_grader/013_must}, rhead = Page~\thepage
] { ../2nd_grader/013_must_handout.pdf }

\addcontentsline{toc}{section}{18.3\hspace{25pt}have to}
\includebeamer
[ color = Gray, ratio= .57, lines = 32, nup = 5,
  pages = -, lhead = \url{2nd_grader/014_have_to}, rhead = Page~\thepage
] { ../2nd_grader/014_have_to_handout.pdf }

%%%%%%%%%%%%%%%%%%%%%%%%%%%%%%%
\chapter{疑問詞}

\addcontentsline{toc}{section}{19.1\hspace{25pt}who}
\includebeamer
[ color = Gray, ratio= .57, lines = 32, nup = 5,
  pages = -, lhead = \url{1st_grader/013_who}, rhead = Page~\thepage
] { ../1st_grader/013_who_handout.pdf }

\addcontentsline{toc}{section}{19.2\hspace{25pt}when}
\includebeamer
[ color = Gray, ratio= .57, lines = 32, nup = 5,
  pages = -, lhead = \url{1st_grader/014_when}, rhead = Page~\thepage
] { ../1st_grader/014_when_handout.pdf }

\addcontentsline{toc}{section}{19.3\hspace{25pt}where}
\includebeamer
[ color = Gray, ratio= .57, lines = 32, nup = 5,
  pages = -, lhead = \url{1st_grader/015_where}, rhead = Page~\thepage
] { ../1st_grader/015_where_handout.pdf }

\addcontentsline{toc}{section}{19.4\hspace{25pt}which}
\includebeamer
[ color = Gray, ratio= .57, lines = 32, nup = 5,
  pages = -, lhead = \url{1st_grader/016_which}, rhead = Page~\thepage
] { ../1st_grader/016_which_handout.pdf }

\addcontentsline{toc}{section}{19.5\hspace{25pt}how}
\includebeamer
[ color = Gray, ratio= .57, lines = 32, nup = 5,
  pages = -, lhead = \url{1st_grader/017_how}, rhead = Page~\thepage
] { ../1st_grader/017_how_handout.pdf }

\addcontentsline{toc}{section}{19.6\hspace{25pt}why}
\includebeamer
[ color = Gray, ratio= .57, lines = 32, nup = 5,
  pages = -, lhead = \url{1st_grader/018_why}, rhead = Page~\thepage
] { ../1st_grader/018_why_handout.pdf }

\addcontentsline{toc}{section}{19.7\hspace{25pt}what}
\includebeamer
[ color = Gray, ratio= .57, lines = 32, nup = 5,
  pages = -, lhead = \url{1st_grader/019_what}, rhead = Page~\thepage
] { ../1st_grader/019_what_handout.pdf }

\addcontentsline{toc}{section}{19.8\hspace{25pt}whose}
\includebeamer
[ color = Gray, ratio= .57, lines = 32, nup = 5,
  pages = -, lhead = \url{1st_grader/020_whose}, rhead = Page~\thepage
] { ../1st_grader/020_whose_handout.pdf }

\addcontentsline{toc}{section}{19.9\hspace{25pt}疑問詞のまとめ}
\includebeamer
[ color = Gray, ratio= .57, lines = 32, nup = 5,
  pages = -, lhead = \url{1st_grader/020a_wh}, rhead = Page~\thepage
] { ../1st_grader/020a_wh_handout.pdf }
%%%%%%%%%%%%%%%%%%%%%%%%%%%%%%%%%
\chapter{従属接続詞(副詞節)}
\addcontentsline{toc}{section}{20.1\hspace{25pt}when}
\includebeamer
[ color = Gray, ratio= .57, lines = 32, nup = 5,
  pages = -, lhead = \url{2nd_grader/021_when}, rhead = Page~\thepage
] { ../2nd_grader/021_when_handout.pdf }

\addcontentsline{toc}{section}{20.2\hspace{25pt}if}
\includebeamer
[ color = Gray, ratio= .57, lines = 32, nup = 5,
  pages = -, lhead = \url{2nd_grader/022_if}, rhead = Page~\thepage
] { ../2nd_grader/022_if_handout.pdf }

\addcontentsline{toc}{section}{20.3\hspace{25pt}because}
\includebeamer
[ color = Gray, ratio= .57, lines = 32, nup = 5,
  pages = -, lhead = \url{2nd_grader/023_because}, rhead = Page~\thepage
] { ../2nd_grader/023_because_handout.pdf }
%%%%%%%%%%%%%%%%%%%%%%%%%%%%%%%%
\chapter{文型}

\addcontentsline{toc}{section}{21.1\hspace{25pt}S$+$V$+$O$+$O}
\includebeamer
[ color = Gray, ratio= .57, lines = 32, nup = 5,
  pages = -, lhead = \url{3rd_grader/052_svoo}, rhead = Page~\thepage
] { ../3rd_grader/052_svoo_handout.pdf }


%%%%%%%%%%%%%%%%%%%%%%%%%%%%%%%%
\chapter{準動詞1(To不定詞)}

\addcontentsline{toc}{section}{22.1\hspace{25pt}不定詞とは}
\includebeamer
[ color = Gray, ratio= .57, lines = 32, nup = 5,
  pages = -, lhead = \url{2nd_grader/031_infinitive_intro}, rhead = Page~\thepage
] { ../2nd_grader/031_infinitive_intro_handout.pdf }

\addcontentsline{toc}{section}{22.2\hspace{25pt}名詞的用法}
\includebeamer
[ color = Gray, ratio= .57, lines = 32, nup = 5,
  pages = -, lhead = \url{2nd_grader/032_infinitive_noun}, rhead = Page~\thepage
] { ../2nd_grader/032_infinitive_noun_handout.pdf }

\addcontentsline{toc}{section}{22.3\hspace{25pt}形容詞的用法}
\includebeamer
[ color = Gray, ratio= .57, lines = 32, nup = 5,
  pages = -, lhead = \url{2nd_grader/033_infinitive_adj}, rhead = Page~\thepage
] { ../2nd_grader/033_infinitive_adj_handout.pdf }

\addcontentsline{toc}{section}{22.4\hspace{25pt}副詞的用法}
\includebeamer
[ color = Gray, ratio= .57, lines = 32, nup = 5,
  pages = -, lhead = \url{2nd_grader/034_infinitive_adv}, rhead = Page~\thepage
] { ../2nd_grader/034_infinitive_adv_handout.pdf }

\addcontentsline{toc}{section}{22.5\hspace{25pt}Wh $+$ to--}
\includebeamer
[ color = Gray, ratio= .57, lines = 32, nup = 5,
  pages = -, lhead = \url{2nd_grader/035_wh_to_do}, rhead = Page~\thepage
] { ../2nd_grader/035_wh_to_do_handout.pdf }

\addcontentsline{toc}{section}{22.6\hspace{25pt}It is \ldots\,\,for ~ to--}
\includebeamer
[ color = Gray, ratio= .57, lines = 32, nup = 5,
  pages = -, lhead = \url{3rd_grader/062_it_for_to}, rhead = Page~\thepage
] { ../3rd_grader/062_it_for_to_handout.pdf }

\addcontentsline{toc}{section}{22.7\hspace{25pt}would like to--}
\includebeamer
[ color = Gray, ratio= .57, lines = 32, nup = 5,
  pages = -, lhead = \url{1st_grader/041_would_like}, rhead = Page~\thepage
] { ../1st_grader/041_would_like_handout.pdf }

%%%%%%%%%%%%%%%%%%%%%%%%%
\chapter{準動詞2(動名詞)}

\addcontentsline{toc}{section}{23.1\hspace{25pt}動名詞とは}
\includebeamer
[ color = Gray, ratio= .57, lines = 32, nup = 5,
  pages = -, lhead = \url{2nd_grader/036_gerund}, rhead = Page~\thepage
] { ../2nd_grader/036_gerund_handout.pdf }
%%%%%%%%%%%%%%%%%%%%%%%%%%%%%%%
\chapter{準動詞3(分詞)}

\addcontentsline{toc}{section}{24.1\hspace{25pt}現在分詞}
\includebeamer
[ color = Gray, ratio= .57, lines = 32, nup = 5,
  pages = -, lhead = \url{3rd_grader/021_N_ing}, rhead = Page~\thepage
] { ../3rd_grader/021_N_ing_handout.pdf }

\addcontentsline{toc}{section}{24.2\hspace{25pt}過去分詞}
\includebeamer
[ color = Gray, ratio= .57, lines = 32, nup = 5,
  pages = -, lhead = \url{3rd_grader/022_N_pp}, rhead = Page~\thepage
] { ../3rd_grader/022_N_pp_handout.pdf }

%%%%%%%%%%%%%%%%%%%%%%%%%%%%%%%%
\chapter{比較}

\addcontentsline{toc}{section}{25.1\hspace{25pt}原級as ~ as}
\includebeamer
[ color = Gray, ratio= .57, lines = 32, nup = 5,
  pages = -, lhead = \url{2nd_grader/041_as_as}, rhead = Page~\thepage
] { ../2nd_grader/041_as_as_handout.pdf }

\addcontentsline{toc}{section}{25.2\hspace{25pt}比較級 ~er than \ldots}
\includebeamer
[ color = Gray, ratio= .57, lines = 32, nup = 5,
  pages = -, lhead = \url{2nd_grader/042_er}, rhead = Page~\thepage
] { ../2nd_grader/042_er_handout.pdf }

\addcontentsline{toc}{section}{25.3\hspace{25pt}最上級 the ~est}
\includebeamer
[ color = Gray, ratio= .57, lines = 32, nup = 5,
  pages = -, lhead = \url{2nd_grader/043_est}, rhead = Page~\thepage
] { ../2nd_grader/043_est_handout.pdf }

\addcontentsline{toc}{section}{25.4\hspace{25pt}more / most}
\includebeamer
[ color = Gray, ratio= .57, lines = 32, nup = 5,
  pages = -, lhead = \url{2nd_grader/044_more_most}, rhead = Page~\thepage
] { ../2nd_grader/044_more_most_handout.pdf }

\addcontentsline{toc}{section}{25.5\hspace{25pt}不規則変化する形容詞・副詞}
\includebeamer
[ color = Gray, ratio= .57, lines = 32, nup = 5,
  pages = -, lhead = \url{2nd_grader/045_better_best}, rhead = Page~\thepage
] { ../2nd_grader/045_better_best_handout.pdf }
%%%%%%%%%%%%%%%%%%%%%%%%%%%%%%%%%
\chapter{態}
\addcontentsline{toc}{section}{26.1\hspace{25pt}受け身とは}
\includebeamer
[ color = Gray, ratio= .57, lines = 32, nup = 5,
  pages = -, lhead = \url{2nd_grader/051_passive}, rhead = Page~\thepage
] { ../2nd_grader/051_passive_handout.pdf }
%%%%%%%%%%%%%%%%%%%%%%%%%%%%%%%%%
\chapter{名詞節}
\addcontentsline{toc}{section}{27.1\hspace{25pt}V $+$ that}
\includebeamer
[ color = Gray, ratio= .57, lines = 32, nup = 5,
  pages = -, lhead = \url{3rd_grader/054_V_that_clause}, rhead = Page~\thepage
] { ../3rd_grader/054_v_that_clause_handout.pdf }

\addcontentsline{toc}{section}{27.2\hspace{25pt}間接疑問文}
\includebeamer
[ color = Gray, ratio= .57, lines = 32, nup = 5,
  pages = -, lhead = \url{3rd_grader/055_indirect_question}, rhead = Page~\thepage
] { ../3rd_grader/055_indirect_question_handout.pdf }

%%%%%%%%%%%%%%%%%%%%%%%%%%%%%%%%%
\chapter{関係代名詞(形容詞節)}
\addcontentsline{toc}{section}{28.1\hspace{25pt}主格}
\includebeamer
[ color = Gray, ratio= .57, lines = 32, nup = 5,
  pages = -, lhead = \url{3rd_grader/031_N_wh_V}, rhead = Page~\thepage
] { ../3rd_grader/031_N_wh_V_handout.pdf }

\addcontentsline{toc}{section}{28.2\hspace{25pt}目的格}
\includebeamer
[ color = Gray, ratio= .57, lines = 32, nup = 5,
  pages = -, lhead = \url{3rd_grader/032_N_wh_SV}, rhead = Page~\thepage
] { ../3rd_grader/032_N_wh_SV_handout.pdf }
%%%%%%%%%%%%%%%%%%%%%%%%%%%%%%%%%
\chapter{仮定法}
\addcontentsline{toc}{section}{29.1\hspace{25pt}仮定法とは}
\includebeamer
[ color = Gray, ratio= .57, lines = 32, nup = 5,
  pages = -, lhead = \url{3rd_grader/041_mood_intro}, rhead = Page~\thepage
] { ../3rd_grader/041_mood_intro_handout.pdf }

\addcontentsline{toc}{section}{29.2\hspace{25pt}I wish $+$ 仮定法}
\includebeamer
[ color = Gray, ratio= .57, lines = 32, nup = 5,
  pages = -, lhead = \url{3rd_grader/042_mood_i_wish}, rhead = Page~\thepage
] { ../3rd_grader/042_mood_i_wish_handout.pdf }

\addcontentsline{toc}{section}{29.3\hspace{25pt}If S $+$ 仮定法}
\includebeamer
[ color = Gray, ratio= .57, lines = 32, nup = 5,
  pages = -, lhead = \url{3rd_grader/043_mood_if}, rhead = Page~\thepage
] { ../3rd_grader/043_mood_if_handout.pdf }
%%%%%%%%%%%%%%%%%%%%%%%%%%%%%%%%%
\chapter{miscellaneous}
\addcontentsline{toc}{section}{30.1\hspace{25pt}数}
\includebeamer
[ color = Gray, ratio= .57, lines = 32, nup = 5,
  pages = -, lhead = \url{misc/001_number}, rhead = Page~\thepage
] { ../misc/001_number_handout.pdf }

\addcontentsline{toc}{section}{30.2\hspace{25pt}月日・季節}
\includebeamer
[ color = Gray, ratio= .57, lines = 32, nup = 5,
  pages = -, lhead = \url{misc/002_day_month_season}, rhead = Page~\thepage
] { ../misc/002_day_month_season_handout.pdf }

\addcontentsline{toc}{section}{30.3\hspace{25pt}天候}
\includebeamer
[ color = Gray, ratio= .57, lines = 32, nup = 5,
  pages = -, lhead = \url{misc/003_weather}, rhead = Page~\thepage
] { ../misc/003_weather_handout.pdf }

\addcontentsline{toc}{section}{30.4\hspace{25pt}道案内}
\includebeamer
[ color = Gray, ratio= .57, lines = 32, nup = 5,
  pages = -, lhead = \url{misc/004_direction}, rhead = Page~\thepage
] { ../misc/004_direction_handout.pdf }

%%%%%%%%%%%%%%%%%%%%%%%%%%%%%%%%
\appendix
\chapter{Pronunciation}

\addcontentsline{toc}{section}{A.1\hspace{25pt}発音の基礎}
\includebeamer
[ color = Gray, ratio= .57, lines = 32, nup = 5,
  pages = -, lhead = \url{1st_grader/099_pronunciation}, rhead = Page~\thepage
] { 099_pronunciation_handout.pdf }

\addcontentsline{toc}{section}{A.2\hspace{25pt}子音}
\includebeamer
[ color = Gray, ratio= .57, lines = 32, nup = 5,
  pages = -, lhead = \url{pronunciation/pronunciation_consonant}, rhead = Page~\thepage
] { ../pronunciation/pronunciation_consonant_handout.pdf }

\addcontentsline{toc}{section}{A.3\hspace{25pt}母音}
\includebeamer
[ color = Gray, ratio= .57, lines = 32, nup = 5,
  pages = -, lhead = \url{pronunciation/pronunciation_vowel}, rhead = Page~\thepage
] { ../pronunciation/pronunciation_vowel_handout.pdf }
%%%%%%%%%%%%%%%%%%%%%%%%%%
\chapter{アルファベット聞き取りクイズ}
\includebeamer
[ color = Gray, ratio= .57, lines = 32, nup = 5,
  pages = -, lhead = \url{1st_grader/999quiz_handout}, rhead = Page~\thepage
] { 999quiz_handout.pdf }
%%%%%%%%%%%%%%%%%%%%%%%%%%%%%%%
\chapter{日本語の`ん'}

\includebeamer
[ color = Gray, ratio= .57, lines = 32, nup = 5,
  pages = -, lhead = \url{pronunciation/pronunciation_japanese}, rhead = Page~\thepage
] { ../pronunciation/pronunciation_japanese_handout.pdf }
\end{document}
