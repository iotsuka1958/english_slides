%%%%%%%handoutオプションだとエラーになるTeXソースがある
%%%%%%%さてどうしたあものか
%%%%%%%handoutオプションをつけないと印刷するには枚数が増えすぎる
\documentclass[jafontscale=0.9247,dvipsnames,book,twoside]{jlreq}
\usepackage{geometry,notebeamer}
\geometry{margin = .75in}
%\ModifyHeading{chapter}{label_format={Chapter~\thechapter}}
\usepackage{url}
%%%%%%%%%%%%%%%%%%%%%%%%%%%%%%%%
\begin{document}
\tableofcontents 
%%%%%%%%%%%%%%%%%%%%%%%%%%%%%%%%%
\newpage
\chapter{Orientation}
\includebeamer
[ color = Gray, ratio= .57, lines = 32, nup = 5,
  pages = -, lefthead = \url{orientation/orientation_1}, righthead = Page~\thepage
] { ../orientation/orientation_handout.pdf }

\includebeamer
[ color = Gray, ratio= .57, lines = 32, nup = 5,
  pages = -, lefthead = \url{orientation/orientation_2}, righthead = Page~\thepage
] { ../orientation/orientation_2.pdf }
%%%%%%%%%%%%%%%%%%%%%%%%%%%%%%%%%
\chapter{Alphabet}
\includebeamer
[ color = Gray, ratio= .57, lines = 32, nup = 5,
  pages = -, lefthead = \url{1st_grader/alphabet}, righthead = Page~\thepage
] { 001_alphabet_handout.pdf }
%%%%%%%%%%%%%%%%%%%%%%%%%%%%%%%%%
\chapter{主語と動詞}
\includebeamer
[ color = Gray, ratio= .57, lines = 32, nup = 5,
  pages = -, lefthead = 1st\_grader/002\_sv, righthead = Page~\thepage
] { 002_sv_handout.pdf }

\chapter{be動詞と一般動詞}
\addcontentsline{toc}{section}{3.1\hspace{25pt}be動詞}
\includebeamer
[ color = Gray, ratio= .57, lines = 32, nup = 5,
  pages = -, lefthead = \url{1st_grader/003_be}, righthead = Page~\thepage
] { 003_be_handout.pdf }

\addcontentsline{toc}{section}{3.2\hspace{25pt}一般動詞}
\includebeamer
[ color = Gray, ratio= .57, lines = 32, nup = 5,
  pages = -, lefthead = \url{1st_grader/004_verb}, righthead = Page~\thepage
] { 004_verb_handout.pdf }


\chapter{目的語}
\includebeamer
[ color = Gray, ratio= .57, lines = 32, nup = 5,
  pages = -, lefthead = \url{1st_grader/004_verb_object}, righthead = Page~\thepage
] { 004_verb_object_handout.pdf }


\chapter{代名詞}
\includebeamer
[ color = Gray, ratio= .57, lines = 32, nup = 5,
  pages = -, lefthead = \url{1st_grader/005_pronoun}, righthead = Page~\thepage
] { 005_pronoun_handout.pdf }

\chapter{単数と複数}
\includebeamer
[ color = Gray, ratio= .57, lines = 32, nup = 5,
  pages = -, lefthead = \url{1st_grader/005_singular_plural}, righthead = Page~\thepage
] { 005_singular_plural_handout.pdf }

\chapter{否定文}
\addcontentsline{toc}{section}{6.1\hspace{25pt}be動詞の否定}
\includebeamer
[ color = Gray, ratio= .57, lines = 32, nup = 5,
  pages = -, lefthead = \url{1st_grader/006_negative_be}, righthead = Page~\thepage
] { 006_negative_be_handout.pdf }

\addcontentsline{toc}{section}{6.2\hspace{25pt}一般動詞の否定}
\includebeamer
[ color = Gray, ratio= .57, lines = 32, nup = 5,
  pages = -, lefthead = \url{1st_grader/007_negative_do}, righthead = Page~\thepage
] { 007_negative_do_handout.pdf }

\chapter{疑問文}
\addcontentsline{toc}{section}{7.1\hspace{25pt}be動詞の疑問文}
\includebeamer
[ color = Gray, ratio= .57, lines = 32, nup = 5,
  pages = -, lefthead = \url{1st_grader/008_question_be}, righthead = Page~\thepage
] { 008_question_be_handout.pdf }

\addcontentsline{toc}{section}{7.2\hspace{25pt}be動詞の疑問文への答え方}
\includebeamer
[ color = Gray, ratio= .57, lines = 32, nup = 5,
  pages = -, lefthead = \url{1st_grader/009_answer_be}, righthead = Page~\thepage
] { 009_answer_be_handout.pdf }

\addcontentsline{toc}{section}{7.3\hspace{25pt}一般動詞の疑問文}
\includebeamer
[ color = Gray, ratio= .57, lines = 32, nup = 5,
  pages = -, lefthead = \url{1st_grader/010_question_do}, righthead = Page~\thepage
] { 010_question_do_handout.pdf }

\addcontentsline{toc}{section}{7.4\hspace{25pt}一般動詞の疑問文への答え方}
\includebeamer
[ color = Gray, ratio= .57, lines = 32, nup = 5,
  pages = -, lefthead = \url{1st_grader/011_answer_do}, righthead = Page~\thepage
] { 011_answer_do_handout.pdf }


\chapter{現在進行形}
\includebeamer
[ color = Gray, ratio= .57, lines = 32, nup = 5,
  pages = -, lefthead = \url{021_is_ing_intro}, righthead = Page~\thepage
] {021_is_ing_intro_handout.pdf }

\includebeamer
[ color = Gray, ratio= .57, lines = 32, nup = 5,
  pages = -, lefthead = \url{022_is_ing_negative}, righthead = Page~\thepage
] {022_is_ing_negative_handout.pdf }

\includebeamer
[ color = Gray, ratio= .57, lines = 32, nup = 5,
  pages = -, lefthead = \url{023_is_ing_question}, righthead = Page~\thepage
] {023_is_ing_question.pdf }

\chapter{過去形}

\addcontentsline{toc}{section}{8.1\hspace{15pt}be動詞の過去形}
\includebeamer
[ color = Gray, ratio= .57, lines = 32, nup = 5,
  pages = -, lefthead = \url{1st_grader/024_past_be}, righthead = Page~\thepage
] { 024_past_be_handout.pdf }

\addcontentsline{toc}{section}{8.2\hspace{25pt}一般動詞の過去形}
\includebeamer
[ color = Gray, ratio= .57, lines = 32, nup = 5,
  pages = -, lefthead = \url{1st_grader/025_past_do}, righthead = Page~\thepage
] { 025_past_do_handout.pdf }

\addcontentsline{toc}{section}{8.3\hspace{25pt}一般動詞の過去形の否定}
\includebeamer
[ color = Gray, ratio= .57, lines = 32, nup = 5,
  pages = -, lefthead = \url{1st_grader/026_past_didnot}, righthead = Page~\thepage
] { 026_past_didnot_handout.pdf }

\addcontentsline{toc}{section}{8.4\hspace{25pt}一般動詞の過去形の疑問文}
\includebeamer
[ color = Gray, ratio= .57, lines = 32, nup = 5,
  pages = -, lefthead = \url{1st_grader/027_past_did_you}, righthead = Page~\thepage
] { 027_past_did_you_handout.pdf }

\chapter{過去進行形}
\includebeamer
[ color = Gray, ratio= .57, lines = 32, nup = 5,
  pages = -, lefthead = \url{2nd_grader/005_was_ing_intro}, righthead = Page~\thepage
] { ../2nd_grader/005_was_ing_intro_handout.pdf }

\chapter{未来を表す表現}
\addcontentsline{toc}{section}{10.1\hspace{25pt}be going to--}
\includebeamer
[ color = Gray, ratio= .57, lines = 32, nup = 5,
  pages = -, lefthead = \url{2nd_grader/011_be_going_to}, righthead = Page~\thepage
] { ../2nd_grader/011_be_going_to_handout.pdf }

\addcontentsline{toc}{section}{10.2\hspace{25pt}will}
\includebeamer
[ color = Gray, ratio= .57, lines = 32, nup = 5,
  pages = -, lefthead = \url{2nd_grader/012_will}, righthead = Page~\thepage
] { ../2nd_grader/012_will_handout.pdf }
%%%%%%%%%%%%%%%%%%%%%%%%%%%%%%%%
\chapter{現在完了}

\includebeamer
[ color = Gray, ratio= .57, lines = 32, nup = 5,
  pages = -, lefthead = \url{3rd_grader/011_have_pp_intro}, righthead = Page~\thepage
] { ../3rd_grader/011_have_pp_intro_handout.pdf }

\includebeamer
[ color = Gray, ratio= .57, lines = 32, nup = 5,
  pages = -, lefthead = \url{3rd_grader/012_have_pp_keizoku}, righthead = Page~\thepage
] { ../3rd_grader/012_have_pp_keizoku_handout.pdf }

\includebeamer
[ color = Gray, ratio= .57, lines = 32, nup = 5,
  pages = -, lefthead = \url{3rd_grader/013_have_pp_keiken}, righthead = Page~\thepage
] { ../3rd_grader/013_have_pp_keiken_handout.pdf }

\includebeamer
[ color = Gray, ratio= .57, lines = 32, nup = 5,
  pages = -, lefthead = \url{3rd_grader/014_have_pp_kekka}, righthead = Page~\thepage
] { ../3rd_grader/014_have_pp_kekka_handout.pdf }
%%%%%%%%%%%%%%%%%%%%%%%%%%%%%%%%
\chapter{命令文・感嘆文}

\includebeamer
[ color = Gray, ratio= .57, lines = 32, nup = 5,
  pages = -, lefthead = \url{1st_grader/032_imperative}, righthead = Page~\thepage
] { ../1st_grader/032_imperative_handout.pdf }

\includebeamer
[ color = Gray, ratio= .57, lines = 32, nup = 5,
  pages = -, lefthead = \url{1st_grader/033_exclamator}, righthead = Page~\thepage
] { ../1st_grader/033_exclamatory_handout.pdf }
%%%%%%%%%%%%%%%%%%%%%%%%%%%%%%%%%
\chapter{There is 名詞}

\includebeamer
[ color = Gray, ratio= .57, lines = 32, nup = 5,
  pages = -, lefthead = \url{2nd_grader/001_there_is}, righthead = Page~\thepage
] { ../2nd_grader/001_there_is_handout.pdf }
%%%%%%%%%%%%%%%%%%%%%%%%%%%%%%%%%
\chapter{品詞}

\includebeamer
[ color = Gray, ratio= .57, lines = 32, nup = 5,
  pages = -, lefthead = \url{2nd_grader/020_part_of_speech}, righthead = Page~\thepage
] { ../2nd_grader/020_part_of_speech_handout.pdf }
%%%%%%%%%%%%%%%%%%%%%%%%%%%%%%%%%
\chapter{助動詞}
\addcontentsline{toc}{section}{10.1\hspace{25pt}can}
\includebeamer
[ color = Gray, ratio= .57, lines = 32, nup = 5,
  pages = -, lefthead = \url{1st_grader/012_can}, righthead = Page~\thepage
] { ./012_can_handout.pdf }

\addcontentsline{toc}{section}{10.2\hspace{25pt}must}
\includebeamer
[ color = Gray, ratio= .57, lines = 32, nup = 5,
  pages = -, lefthead = \url{2nd_grader/013_must}, righthead = Page~\thepage
] { ../2nd_grader/013_must_handout.pdf }

\addcontentsline{toc}{section}{10.3\hspace{25pt}have to}
\includebeamer
[ color = Gray, ratio= .57, lines = 32, nup = 5,
  pages = -, lefthead = \url{2nd_grader/014_have_to}, righthead = Page~\thepage
] { ../2nd_grader/014_have_to_handout.pdf }
%%%%%%%%%%%%%%%%%%%%%%%%%%%%%%%%%
\chapter{従属接続詞}
\includebeamer
[ color = Gray, ratio= .57, lines = 32, nup = 5,
  pages = -, lefthead = \url{2nd_grader/021_when}, righthead = Page~\thepage
] { ../2nd_grader/021_when_handout.pdf }

\includebeamer
[ color = Gray, ratio= .57, lines = 32, nup = 5,
  pages = -, lefthead = \url{2nd_grader/022_if}, righthead = Page~\thepage
] { ../2nd_grader/022_if_handout.pdf }

\includebeamer
[ color = Gray, ratio= .57, lines = 32, nup = 5,
  pages = -, lefthead = \url{2nd_grader/023_because}, righthead = Page~\thepage
] { ../2nd_grader/023_because_handout.pdf }
%%%%%%%%%%%%%%%%%%%%%%%%%%%%%%%
\chapter{疑問詞}

\includebeamer
[ color = Gray, ratio= .57, lines = 32, nup = 5,
  pages = -, lefthead = \url{1std_grader/013_who}, righthead = Page~\thepage
] { ../1st_grader/013_who_handout.pdf }

\includebeamer
[ color = Gray, ratio= .57, lines = 32, nup = 5,
  pages = -, lefthead = \url{1std_grader/014_when}, righthead = Page~\thepage
] { ../1st_grader/014_when_handout.pdf }

\includebeamer
[ color = Gray, ratio= .57, lines = 32, nup = 5,
  pages = -, lefthead = \url{1std_grader/015_where}, righthead = Page~\thepage
] { ../1st_grader/015_where_handout.pdf }

\includebeamer
[ color = Gray, ratio= .57, lines = 32, nup = 5,
  pages = -, lefthead = \url{1std_grader/016_which}, righthead = Page~\thepage
] { ../1st_grader/016_which_handout.pdf }

\includebeamer
[ color = Gray, ratio= .57, lines = 32, nup = 5,
  pages = -, lefthead = \url{1std_grader/017_how}, righthead = Page~\thepage
] { ../1st_grader/017_how_handout.pdf }

\includebeamer
[ color = Gray, ratio= .57, lines = 32, nup = 5,
  pages = -, lefthead = \url{1std_grader/018_why}, righthead = Page~\thepage
] { ../1st_grader/018_why_handout.pdf }

\includebeamer
[ color = Gray, ratio= .57, lines = 32, nup = 5,
  pages = -, lefthead = \url{1std_grader/019_what}, righthead = Page~\thepage
] { ../1st_grader/019_what_handout.pdf }

\includebeamer
[ color = Gray, ratio= .57, lines = 32, nup = 5,
  pages = -, lefthead = \url{1std_grader/020_whose}, righthead = Page~\thepage
] { ../1st_grader/020_whose_handout.pdf }
%%%%%%%%%%%%%%%%%%%%%%%%%%%%%%%%
\chapter{準動詞1(To不定詞)}

\includebeamer
[ color = Gray, ratio= .57, lines = 32, nup = 5,
  pages = -, lefthead = \url{2nd_grader/031_infinitive_intro}, righthead = Page~\thepage
] { ../2nd_grader/031_infinitive_intro_handout.pdf }

\includebeamer
[ color = Gray, ratio= .57, lines = 32, nup = 5,
  pages = -, lefthead = \url{2nd_grader/032_infinitive_noun}, righthead = Page~\thepage
] { ../2nd_grader/032_infinitive_noun_handout.pdf }

\includebeamer
[ color = Gray, ratio= .57, lines = 32, nup = 5,
  pages = -, lefthead = \url{2nd_grader/033_infinitive_adj}, righthead = Page~\thepage
] { ../2nd_grader/033_infinitive_adj_handout.pdf }

\includebeamer
[ color = Gray, ratio= .57, lines = 32, nup = 5,
  pages = -, lefthead = \url{2nd_grader/034_infinitive_adv}, righthead = Page~\thepage
] { ../2nd_grader/034_infinitive_adv_handout.pdf }

\includebeamer
[ color = Gray, ratio= .57, lines = 32, nup = 5,
  pages = -, lefthead = \url{2nd_grader/035_wh_to_do}, righthead = Page~\thepage
] { ../2nd_grader/035_wh_to_do_handout.pdf }

%%%%%%%%%%%%%%%%%%%%%%%%%
\chapter{準動詞2(動名詞)}

\includebeamer
[ color = Gray, ratio= .57, lines = 32, nup = 5,
  pages = -, lefthead = \url{2nd_grader/036_gerund}, righthead = Page~\thepage
] { ../2nd_grader/036_gerund_handout.pdf }
%%%%%%%%%%%%%%%%%%%%%%%%%%%%%%%
\chapter{準動詞3(分詞)}

\includebeamer
[ color = Gray, ratio= .57, lines = 32, nup = 5,
  pages = -, lefthead = \url{3rd_grader/021_N_ing}, righthead = Page~\thepage
] { ../3rd_grader/021_N_ing_handout.pdf }

\includebeamer
[ color = Gray, ratio= .57, lines = 32, nup = 5,
  pages = -, lefthead = \url{3rd_grader/022_N_pp}, righthead = Page~\thepage
] { ../3rd_grader/022_N_pp_handout.pdf }

%%%%%%%%%%%%%%%%%%%%%%%%%%%%%%%%
\chapter{比較}

\includebeamer
[ color = Gray, ratio= .57, lines = 32, nup = 5,
  pages = -, lefthead = \url{2nd_grader/041_as_as}, righthead = Page~\thepage
] { ../2nd_grader/041_as_as_handout.pdf }

\includebeamer
[ color = Gray, ratio= .57, lines = 32, nup = 5,
  pages = -, lefthead = \url{2nd_grader/042_er}, righthead = Page~\thepage
] { ../2nd_grader/042_er_handout.pdf }

\includebeamer
[ color = Gray, ratio= .57, lines = 32, nup = 5,
  pages = -, lefthead = \url{2nd_grader/043_est}, righthead = Page~\thepage
] { ../2nd_grader/043_est_handout.pdf }

\includebeamer
[ color = Gray, ratio= .57, lines = 32, nup = 5,
  pages = -, lefthead = \url{2nd_grader/044_more_most}, righthead = Page~\thepage
] { ../2nd_grader/044_more_most_handout.pdf }

\includebeamer
[ color = Gray, ratio= .57, lines = 32, nup = 5,
  pages = -, lefthead = \url{2nd_grader/045_better_best}, righthead = Page~\thepage
] { ../2nd_grader/045_better_best_handout.pdf }
%%%%%%%%%%%%%%%%%%%%%%%%%%%%%%%%%
\chapter{態}
\includebeamer
[ color = Gray, ratio= .57, lines = 32, nup = 5,
  pages = -, lefthead = \url{2nd_grader/051_passive}, righthead = Page~\thepage
] { ../2nd_grader/051_passive_handout.pdf }
%%%%%%%%%%%%%%%%%%%%%%%%%%%%%%%%%
\chapter{関係代名詞}
\includebeamer
[ color = Gray, ratio= .57, lines = 32, nup = 5,
  pages = -, lefthead = \url{3rd_grader/031_N_wh_V}, righthead = Page~\thepage
] { ../3rd_grader/031_N_wh_V_handout.pdf }

\includebeamer
[ color = Gray, ratio= .57, lines = 32, nup = 5,
  pages = -, lefthead = \url{3rd_grader/032_N_wh_SV}, righthead = Page~\thepage
] { ../3rd_grader/032_N_wh_SV_handout.pdf }
%%%%%%%%%%%%%%%%%%%%%%%%%%%%%%%%%
\chapter{仮定法}
\includebeamer
[ color = Gray, ratio= .57, lines = 32, nup = 5,
  pages = -, lefthead = \url{3rd_grader/041_mood_intro}, righthead = Page~\thepage
] { ../3rd_grader/041_mood_intro_handout.pdf }

\includebeamer
[ color = Gray, ratio= .57, lines = 32, nup = 5,
  pages = -, lefthead = \url{3rd_grader/042_mood_i_wish}, righthead = Page~\thepage
] { ../3rd_grader/042_mood_i_wish_handout.pdf }

\includebeamer
[ color = Gray, ratio= .57, lines = 32, nup = 5,
  pages = -, lefthead = \url{3rd_grader/043_mood_if}, righthead = Page~\thepage
] { ../3rd_grader/043_mood_if_handout.pdf }
%%%%%%%%%%%%%%%%%%%%%%%%%%%%%%%%%
\appendix
\chapter{Pronunciation}

\includebeamer
[ color = Gray, ratio= .57, lines = 32, nup = 5,
  pages = -, lefthead = \url{1st_grader/099_pronunciation}, righthead = Page~\thepage
] { 099_pronunciation_handout.pdf }

\includebeamer
[ color = Gray, ratio= .57, lines = 32, nup = 5,
  pages = -, lefthead = \url{pronunciation/pronunciation_consonant}, righthead = Page~\thepage
] { ../pronunciation/pronunciation_consonant_handout.pdf }

\includebeamer
[ color = Gray, ratio= .57, lines = 32, nup = 5,
  pages = -, lefthead = \url{pronunciation/pronunciation_vowel}, righthead = Page~\thepage
] { ../pronunciation/pronunciation_vowel_handout.pdf }
%%%%%%%%%%%%%%%%%%%%%%%%%%
\chapter{アルファベット聞き取りクイズ}
\includebeamer
[ color = Gray, ratio= .57, lines = 32, nup = 5,
  pages = -, lefthead = \url{1st_grader/999quiz_handout}, righthead = Page~\thepage
] { 999quiz_handout.pdf }
%%%%%%%%%%%%%%%%%%%%%%%%%%%%%%%
\chapter{日本語の`ん'}

\includebeamer
[ color = Gray, ratio= .57, lines = 32, nup = 5,
  pages = -, lefthead = \url{pronunciation/pronunciation_japanese}, righthead = Page~\thepage
] { ../pronunciation/pronunciation_japanese_handout.pdf }
\end{document}
