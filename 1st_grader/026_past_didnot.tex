\documentclass[aspectratio=169,xcolor={dvipsnames,table}]{beamer}
\usepackage[no-math,deluxe,haranoaji]{luatexja-preset}
\renewcommand{\kanjifamilydefault}{\gtdefault}
\renewcommand{\emph}[1]{{\upshape\bfseries #1}}
\usetheme{metropolis}
\metroset{block=fill}
\setbeamertemplate{navigation symbols}{}
\setbeamertemplate{blocks}[rounded][shadow=false]
\usecolortheme[rgb={0.7,0.2,0.2}]{structure}
%%%%%%%%%%%%%%%%%%%%%%%%%%%
\usepackage{media9}
%%%%%%%%%%%%%%%%%%%%%%%%%%%
%% さまざまなアイコン
%%%%%%%%%%%%%%%%%%%%%%%%%%%
\usepackage{fontawesome}
\usepackage{figchild}
\usepackage{twemojis}
\usepackage{utfsym}
\usepackage{bclogo}
\usepackage{marvosym}
\usepackage{fontmfizz}
\usepackage{pifont}
\usepackage{phaistos}
\usepackage{worldflags}
%%%%%%%%%%%%%%%%%%%%%%%%%%%
\usepackage{tikz}
\usetikzlibrary{backgrounds}
\usepackage{tcolorbox}
\usepackage{tikzpeople}
\usepackage{circledsteps}
\usepackage{xcolor}
\usepackage{amsmath}
\usepackage{booktabs}
%%%%%%%%%%%%%%%%%%%%%%%%%%%
%% 場合分け
\usepackage{cases}
%%%%%%%%%%%%%%%%%%%%%%%%%%%
% \myAnch{<名前>}{<色>}{<テキスト>}
% 指定のテキストを指定の色の四角枠で囲み, 指定の名前をもつTikZの
% ノードとして出力する. 図には remeber picture 属性を付けている
% ので外部から参照可能である.
\newcommand*{\myAnch}[3]{%
  \tikz[remember picture,baseline=(#1.base)]
    \node[draw,rectangle,#2] (#1) {\normalcolor #3};
}
%%%%%%%%%%%%%%%%%%%%%%%%%%%%
%% 音声リンク表示
\newcommand{\myaudio}[1]{\href{#1}{\faVolumeUp}}
%%%%%%%%%%%%%%%%%%%%%%%%%%%
% \myEmph コマンドの定義
%\newcommand{\myEmph}[3]{%
%    \textbf<#1>{\color<#1>{#2}{#3}}%
%}
\usepackage{xparse} % xparseパッケージの読み込み
\NewDocumentCommand{\myEmph}{O{} m m}{%
    \def\argOne{#1}%
    \ifx\argOne\empty
        \textbf{\color{#2}{#3}}% オプション引数が省略された場合
    \else
        \textbf<#1>{\color<#1>{#2}{#3}}% オプション引数が指定された場合
    \fi
}
%%%%%%%%%%%%%%%%%%%%%%%%%%%
%% 文末の上昇イントネーション記号 \myRisingPitch
%% 通常のイントネーション \myDownwardPitch
%% https://note.com/dan_oyama/n/n8be58e8797b2
%%%%%%%%%%%%%%%%%%%%%%%%%%%
\newcommand{\myRisingPitch}{
\begin{tikzpicture}[scale=0.3,baseline=0.3]
\draw[->,>=stealth] (0,0) to[bend right=45] (1,1);
\end{tikzpicture}
}
\newcommand{\myDownwardPitch}{
\begin{tikzpicture}[scale=0.3,baseline=0.3]
\draw[->,>=stealth] (0,1) to[bend left=45] (1,0);
\end{tikzpicture}
}
%%%%%%%%%%%%%%%%%%%%%%%%%%%
\title{English is fun.}
\subtitle{I didn't play tennis yesterday.}
\author{}
\institute[]{}
\date[]

%%%%%%%%%%%%%%%%%%%%%%%%%%%%
%% TEXT
%%%%%%%%%%%%%%%%%%%%%%%%%%%%
\begin{document}
\begin{frame}[plain]
  \titlepage
\end{frame}

\section*{授業の流れ}
\begin{frame}[plain]
  \frametitle{授業の流れ}
  \tableofcontents
\end{frame}

\section{一般動詞の過去形の否定}
\subsection{一般動詞の現在形の否定(復習)}
\begin{frame}<1-6>[plain]{一般動詞の現在形の否定文(復習)}
 
つぎの文を否定文にしましょう。

\begin{tabular}{rlcl}
1&\visible<1->{I live in Paris.}&\visible<1->{$\longrightarrow$}& \visible<2->{I do not($= \text{don't}$) live in Paris.}\\
2&\visible<1->{You live  in Paris.}& \visible<1->{$\longrightarrow$}& \visible<3->{You do not($= \text{don't}$) live in Paris.}\\
3&\visible<1->{They live in Paris.}&\visible<1->{$\longrightarrow$}& \visible<4->{They do not($= \text{don't}$) live in Paris.}\\
4&\visible<1->{He lives in Paris.}&\visible<1->{$\longrightarrow$}& \visible<5->{He does not($= \text{doesn't}$) live in Paris.}
\end{tabular}

\visible<6->{%
\begin{exampleblock}{Topics for Today}
一般動詞の現在形
\begin{itemize}
 \item  一般動詞の否定は\textcolor{Maroon}{\bfseries do not\,($=\text{don't)} + \text{動詞の原形}$}
 \item  主語が3人称単数のときは\textcolor{ForestGreen}{\bfseries does not\,($=\text{doesn't)} + \text{動詞の原形}$}
\end{itemize}
      \end{exampleblock}
}
\hfill\myaudio{./audio/026_past_didnot_01.mp3}
\end{frame}

\subsection{一般動詞の過去形の否定文のつくり方}
\begin{frame}[plain]{過去形の否定文}
 \Large

\begin{enumerate}
 \item \begin{enumerate}
	\item  \visible<1->{I live in Boston. $\rightarrow$} \visible<2->{I \myEmph[3-]{Maroon}{do not($=$ don't)} live in Boston.}
	\item  \visible<1->{He lives in Boston. $\rightarrow$} \visible<4->{He \myEmph[5-]{ForestGreen}{does not($=$ doesn't)} live in Boston.}\\
\mbox{}\hfill\visible<6->{\footnotesize 現在形の否定文は、主語によって\textcolor{Maroon}{\bfseries do not\,($=$ don't)}と\textcolor{ForestGreen}{\bfseries does not\,($=$ don't)}を使い分け}
       \end{enumerate}
 \item \begin{enumerate}
	\item  \visible<7->{I lived in Boston. $\rightarrow$} \visible<8->{I \myEmph[9-]{NavyBlue}{did not($=$ didn't)} live in Boston.}
	\item  \visible<7->{He lived in Boston. $\rightarrow$} \visible<10->{He \myEmph[11-]{NavyBlue}{did not($=$ didn't)} live in Boston.}\\
\mbox{}\hfill\visible<12->{\footnotesize 過去形の否定文は主語がなんであっても\textcolor{NavyBlue}{\bfseries did not\,($=$ didn't)}}
       \end{enumerate}
\end{enumerate}

\visible<13->{%
\begin{exampleblock}{Topics for Today}
\small
一般動詞の過去形
\begin{itemize}
 \item  一般動詞の否定は\textcolor{NavyBlue}{\bfseries did not\,($=\text{didn't)} + \text{動詞の原形}$}
 \item  主語がなんであっても同じです
\end{itemize}
      \end{exampleblock}
}
\hfill\myaudio{./audio/026_past_didnot_02a.mp3}\,1\,\,\,\myaudio{./audio/026_past_didnot_02a.mp3}\,2

\end{frame}

\begin{frame}[plain]{Exercises}
 
つぎの各文を否定文にしましょう。

\begin{enumerate}
 \item \visible<1->{We listened to the radio after dinner.}\\
\visible<2->{$\rightarrow$\,We didn't\,($=$ did not) listen to the radio after dinner.}
 \item \visible<1->{I played the piano last night.}\\
\visible<3->{$\rightarrow$\,I didn't\,($=$ did not) play the piano last night.}
 \item \visible<1->{He cooked this morning.}\\
\visible<4->{$\rightarrow$\,He didn't\,($=$ did not) cook this morning.}
 \item \visible<1->{She walked to work every day.}\\
\visible<5->{$\rightarrow$\,She didn't\,($=$ did not) walk to work every day.}
 \item \visible<1->{They lived in London two years ago.}\\
\visible<6->{$\rightarrow$\,They didn't\,($=$ did not) live in London two years ago.}
\end{enumerate}
\hfill\myaudio{./audio/026_past_didnot_03.mp3}

\end{frame}

\begin{frame}[plain]{Exercises}
 
つぎの各文を否定文にしましょう。

\begin{enumerate}
 \item \visible<1->{I studied math last night.}\hfill{$2x+1=5$}\\
\visible<2->{$\rightarrow$\,I didn't\,($=$ did not) study math last night.}
 \item \visible<1->{The baby cried on the bus yesterday.}\\
\visible<3->{$\rightarrow$\,The baby didn't\,($=$ did not) cry on the bus yesterday.}
 \item \visible<1->{The train stopped at every station two years ago.}\\
\visible<4->{$\rightarrow$\,The bus didn't\,($=$ did not) stop at every station two years ago.}
\end{enumerate}
\hfill\myaudio{./audio/026_past_didnot_04.mp3}
\end{frame}


\begin{frame}[plain]{--edで終わるが、ちょっぴり注意が必要な動詞}

\begin{center}
 
\rowcolors{2}{NavyBlue!50}{yellow!50}
\begin{tabular}{lll}\toprule
{\small 原形}&{\small (意味)}&{\small 過去形}\\\midrule
\visible<1->{study}&\visible<1->{{\small (勉強する)}}&\visible<2->{studied}\\
\visible<1->{cry}&\visible<1->{{\small (泣く)}}&\visible<2->{cried}\\
\visible<1->{try}&\visible<1->{{\small(試みる)}}&\visible<2->{tried}\\
\visible<1->{carry}&\visible<1->{{\small (運ぶ)}}&\visible<2->{carried}\\
\visible<1->{stop}&\visible<1->{{\small (止まる/止める)}}&\visible<3->{stopped}\\
\end{tabular}%
\end{center}
 
\visible<4>{%
\begin{exampleblock}{Topics for Today}\small
\begin{itemize}
 \item 最後のyをiに変えてedをつける動詞があります(結果的に--iedとなります)
 \item stopの過去形はpを重ねてedをつけます
\end{itemize}
\end{exampleblock}
}
\hfill\myaudio{./audio/026_past_didnot_05.mp3}

 \end{frame}

\begin{frame}<1-6>[plain]{Exercises}
 次の各文を否定文にします。(~~~~~~~~)に適当な語を補いましょう。

\begin{columns}
\begin{column}{.6\textwidth}
\begin{enumerate}
 \item \begin{enumerate}
	\item He went to London two years ago.
	\item He didn't~~(\alt<1-2>{~~\phantom{go}~~}{~~go~~}) to London two years ago.
       \end{enumerate}
 \item \begin{enumerate}
	\item She came to the party last night.
	\item She didn't~~(\alt<1-3>{~~\phantom{come}~~}{~~come~~}) to the party last night.
       \end{enumerate}
 \item \begin{enumerate}
	\item She ate bread this morning.
	\item She didn't~~(\alt<1-4>{~~\phantom{eat}~~}{~~eat~~}) bread this morning.
       \end{enumerate}
 \item \begin{enumerate}
	\item They had a meeting this afternoon.
	\item They didn't~~(\alt<1-5>{~~\phantom{have}~~}{~~have~~}) a meeting this afternoon.
       \end{enumerate}
\end{enumerate}
\end{column}
\begin{column}{.35\textwidth}
\visible<2->{\rowcolors{2}{NavyBlue!50}{yellow!50}
\begin{tabular}{lll}\toprule
{\small 原形}&{\small (意味)}&{\small 過去形}\\\midrule
{go}&{{\small (行く)}}&{went}\\
{come}&{{\small (来る)}}&{came}\\
{eat}&{{\small(食べる)}}&{ate}\\
{have}&{{\small (持つ)}}&{had}\\
\bottomrule
\end{tabular}}%
\end{column}
\end{columns}
\hfill\myaudio{./audio/026_past_didnot_06.mp3}

\end{frame}


\begin{frame}<1-6>[plain]{Exercises}
 次の各文を否定文にします。(~~~~~~~~)に適当な語を補いましょう。

\begin{columns}
\begin{column}{.6\textwidth}
\begin{enumerate}
 \item \begin{enumerate}
	\item He made a cake for her yesterday.
	\item He didn't~~(\alt<1-2>{~~\phantom{make}~~}{~~make~~}) a cake for her birthday.
       \end{enumerate}
 \item \begin{enumerate}
	\item We saw a great movie last week.
	\item We didn't~~(\alt<1-3>{~~\phantom{see}~~}{~~see~~}) a great movie last week.
       \end{enumerate}
 \item \begin{enumerate}
	\item She got a new car last month.
	\item She didn't~~(\alt<1-4>{~~\phantom{get}~~}{~~get~~}) a new car last month.
       \end{enumerate}
 \item \begin{enumerate}
	\item He spoke English then.
	\item He didn't~~(\alt<1-5>{~~\phantom{speak}~~}{~~speak~~}) English then.
       \end{enumerate}
\end{enumerate}
\end{column}
\begin{column}{.35\textwidth}
\visible<2->
{\rowcolors{2}{NavyBlue!50}{yellow!50}
\begin{tabular}{lll}\toprule
{\small 原形}&{\small (意味)}&{\small 過去形}\\\midrule
{make}&{{\small (作る)}}&{made}\\
{see}&{{\small (見る)}}&{saw}\\
{get}&{{\small (手に入れる)}}&{got}\\
{speak}&{{\small(話す)}}&{spoke}\\
%{take}&{{\small (取る)}}&{took}\\
%{write}&{{\small (書く)}}&{wrote}\\
\bottomrule
\end{tabular}}%

\end{column}
\end{columns}
\hfill\myaudio{./audio/026_past_didnot_07.mp3}

\end{frame}



\begin{frame}<1-6>[plain]{Exercises}
 次の各文を否定文にします。(~~~~~~~~)に適当な語を補いましょう。

\begin{columns}
\begin{column}{.6\textwidth}
\begin{enumerate}
 \item \begin{enumerate}
	\item I took a picture of my cat yesterday.
	\item I didn't~~(\alt<1-2>{~~\phantom{take}~~}{~~take~~}) a picture of my cat yesterday.
       \end{enumerate}
 \item \begin{enumerate}
	\item I wrote a letter to him last night.
	\item I didn't~~(\alt<1-3>{~~\phantom{write}~~}{~~write~~}) a letter to him last night.
       \end{enumerate}
\end{enumerate}
\end{column}
\begin{column}{.35\textwidth}
\visible<2->
{\rowcolors{2}{NavyBlue!50}{yellow!50}
\begin{tabular}{lll}\toprule
{\small 原形}&{\small (意味)}&{\small 過去形}\\\midrule
{take}&{{\small (取る)}}&{took}\\
{write}&{{\small (書く)}}&{wrote}\\
\bottomrule
\end{tabular}}%

\end{column}
\end{columns}
\hfill\myaudio{./audio/026_past_didnot_08.mp3}

\end{frame}


\begin{frame}[plain]{不規則動詞の表}
 
\begin{columns}
\begin{column}{.45\textwidth}
\raggedleft
\rowcolors{2}{NavyBlue!50}{yellow!50}
\begin{tabular}{lll}\toprule
{\small 原形}&{\small (意味)}&{\small 過去形}\\\midrule
\visible<1->{go}&\visible<2->{{\small (行く)}}&\visible<3->{went}\\
\visible<1->{come}&\visible<4->{{\small (来る)}}&\visible<5->{came}\\
\visible<1->{eat}&\visible<6->{{\small(食べる)}}&\visible<7->{ate}\\
\visible<1->{have}&\visible<8->{{\small (持つ)}}&\visible<9->{had}\\
\visible<1->{make}&\visible<10->{{\small (作る)}}&\visible<11->{made}\\
\bottomrule
\end{tabular}%
\end{column}
\begin{column}{.45\textwidth}
\raggedright
\rowcolors{2}{NavyBlue!50}{yellow!50}
\begin{tabular}{lll}\toprule
{\small 原形}&{\small (意味)}&{\small 過去形}\\\midrule
\visible<1->{see}&\visible<12->{{\small (見る)}}&\visible<13->{saw}\\
\visible<1->{get}&\visible<14->{{\small (来る)}}&\visible<15->{got}\\
\visible<1->{speak}&\visible<16->{{\small(食べる)}}&\visible<17->{spoke}\\
\visible<1->{take}&\visible<18->{{\small (持つ)}}&\visible<19->{took}\\
\visible<1->{write}&\visible<20->{{\small (作る)}}&\visible<21->{wrotegit status}\\
\bottomrule
\end{tabular}%
\end{column}
\end{columns}

\hfill\myaudio{./audio/026_past_didnot_09.mp3}

\end{frame}

\subsection{まとめ}
\begin{frame}[plain]{まとめ}
 \begin{exampleblock}{Topics for Today}
\small
一般動詞の過去形の否定
\begin{itemize}
 \item  一般動詞の否定は\textcolor{NavyBlue}{\bfseries did not\,($=\text{didn't)} + \text{動詞の原形}$}
 \item  主語がなんであっても同じです
\end{itemize}
      \end{exampleblock}
\end{frame}

\end{document}

