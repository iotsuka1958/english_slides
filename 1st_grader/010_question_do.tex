\documentclass[aspectratio=169]{beamer}
\usepackage[no-math,deluxe,haranoaji]{luatexja-preset}
\renewcommand{\kanjifamilydefault}{\gtdefault}
\renewcommand{\emph}[1]{{\upshape\bfseries #1}}
\usetheme{metropolis}
\metroset{block=fill}
\setbeamertemplate{navigation symbols}{}
\usecolortheme[rgb={0.7,0.2,0.2}]{structure}
%%%%%%%%%%%%%%%%%%%%%%%%%%%
\usepackage{media9}
%%%%%%%%%%%%%%%%%%%%%%%%%%%
%% さまざまなアイコン
%%%%%%%%%%%%%%%%%%%%%%%%%%%
\usepackage{fontawesome}
\usepackage{figchild}
\usepackage{twemojis}
\usepackage{utfsym}
\usepackage{bclogo}
\usepackage{marvosym}
\usepackage{fontmfizz}
\usepackage{pifont}
\usepackage{phaistos}
\usepackage{worldflags}
%%%%%%%%%%%%%%%%%%%%%%%%%%%
\usepackage{tikz}
\usetikzlibrary{backgrounds}
\usepackage{tcolorbox}
\usepackage{tikzpeople}
\usepackage{xcolor}
\usepackage{amsmath}
%%%%%%%%%%%%%%%%%%%%%%%%%%%
%% 場合分け
\usepackage{cases}
%%%%%%%%%%%%%%%%%%%%%%%%%%%
% \myAnch{<名前>}{<色>}{<テキスト>}
% 指定のテキストを指定の色の四角枠で囲み, 指定の名前をもつTikZの
% ノードとして出力する. 図には remeber picture 属性を付けている
% ので外部から参照可能である.
\newcommand*{\myAnch}[3]{%
  \tikz[remember picture,baseline=(#1.base)]
    \node[draw,rectangle,#2] (#1) {\normalcolor #3};
}
%%%%%%%%%%%%%%%%%%%%%%%%%%%%
%% 音声リンク表示
\newcommand{\myaudio}[1]{\href{#1}{\faVolumeUp}}
%%%%%%%%%%%%%%%%%%%%%%%%%%%
% \myEmph コマンドの定義
%\newcommand{\myEmph}[3]{%
%    \textbf<#1>{\color<#1>{#2}{#3}}%
%}
\usepackage{xparse} % xparseパッケージの読み込み
\NewDocumentCommand{\myEmph}{O{} m m}{%
    \def\argOne{#1}%
    \ifx\argOne\empty
        \textbf{\color{#2}{#3}}% オプション引数が省略された場合
    \else
        \textbf<#1>{\color<#1>{#2}{#3}}% オプション引数が指定された場合
    \fi
}
%%%%%%%%%%%%%%%%%%%%%%%%%%%
%% 文末の上昇イントネーション記号 \myRisingPitch
%% 通常のイントネーション \myDownwardPitch
%% https://note.com/dan_oyama/n/n8be58e8797b2
%%%%%%%%%%%%%%%%%%%%%%%%%%%
\newcommand{\myRisingPitch}{
\begin{tikzpicture}[scale=0.3,baseline=0.3]
\draw[->,>=stealth] (0,0) to[bend right=45] (1,1);
\end{tikzpicture}
}
\newcommand{\myDownwardPitch}{
\begin{tikzpicture}[scale=0.3,baseline=0.3]
\draw[->,>=stealth] (0,1) to[bend left=45] (1,0);
\end{tikzpicture}
}
%%%%%%%%%%%%%%%%%%%%%%%%%%%
\title{English is fun.\,\,{}--- Do you like baseball? ---}
\author{}
\institute[]{}
\date[]

%%%%%%%%%%%%%%%%%%%%%%%%%%%%
%% TEXT
%%%%%%%%%%%%%%%%%%%%%%%%%%%%
\begin{document}
\begin{frame}[plain]
  \titlepage
\end{frame}

\section*{授業の流れ}
\begin{frame}[plain]
  \frametitle{授業の流れ}
  \tableofcontents
\end{frame}


\section{一般動詞の疑問文}

\subsection{わたしは〜ですか}
\begin{frame}[plain]{わたしは〜しますか?}
 \Large

\mbox{}\hspace{35pt}I have  enough time. \pause
\hspace{80pt}{\footnotesize enough: じゅうぶんな}\pause\hspace{-170pt}\raisebox{-5pt}{\scalebox{1.4}{\myDownwardPitch}}
\pause

\vspace{15pt}

\myAnch{do}{orange}{Do} I have enough time \myAnch{question}{orange}{?}
\pause
\scalebox{1.4}{\myRisingPitch}

\pause

\mbox{}\hspace{50pt}\myAnch{txt1}{white}{先頭にDo}
\hspace{80pt}\myAnch{txt2}{white}{文末に`?'}


\begin{tikzpicture}[remember picture, overlay]
\draw[->, thick, orange] (txt1.west) to[out=180, in=-90] (do.south);
\draw[->, thick, orange] (txt2.west) to[out=180, in=-90] (question.south);
\end{tikzpicture}

\mbox{}\hfill\myaudio{./audio/010_question_do_01.mp3}

\end{frame}



\subsection{あなたは〜しますか}
\begin{frame}[plain]{あなたは〜しますか?}
 \Large

\mbox{}\hspace{35pt}You play the guitar. \pause
\hspace{80pt}{\footnotesize guitar: ギター}\pause\hspace{-140pt}\raisebox{-5pt}{\scalebox{1.4}{\myDownwardPitch}}
\pause

\vspace{15pt}

\myAnch{do}{orange}{Do} you play the guitar \myAnch{question}{orange}{?}
\pause
\scalebox{1.4}{\myRisingPitch}

\pause

\mbox{}\hspace{50pt}\myAnch{txt1}{white}{先頭にDo}
\hspace{80pt}\myAnch{txt2}{white}{文末に`?'}


\begin{tikzpicture}[remember picture, overlay]
\draw[->, thick, orange] (txt1.west) to[out=180, in=-90] (do.south);
\draw[->, thick, orange] (txt2.west) to[out=180, in=-90] (question.south);
\end{tikzpicture}

\mbox{}\hfill\myaudio{./audio/010_question_do_02.mp3}

\end{frame}




\subsection{彼らは〜しますか}
\begin{frame}[plain]{彼らは〜しますか?}
 \Large
\pause
\mbox{}\hspace{35pt}They speak French. \pause
\hspace{80pt}{\footnotesize speak: 話す French: フランス語}\pause\hspace{-215pt}\raisebox{-5pt}{\scalebox{1.4}{\myDownwardPitch}}
\pause

\vspace{15pt}

\myAnch{do}{orange}{Do} they speak French \myAnch{question}{orange}{?}
\pause
\scalebox{1.4}{\myRisingPitch}

\pause

\mbox{}\hspace{50pt}\myAnch{txt1}{white}{先頭にDo}
\hspace{80pt}\myAnch{txt2}{white}{文末に`?'}


\begin{tikzpicture}[remember picture, overlay]
\draw[->, thick, orange] (txt1.west) to[out=180, in=-90] (do.south);
\draw[->, thick, orange] (txt2.west) to[out=180, in=-90] (question.south);
\end{tikzpicture}

\vfill

\mbox{}\hfill\myaudio{./audio/010_question_do_03.mp3}

\end{frame}


\begin{frame}[plain]\frametitle{要点}
\Large
\begin{tabular}{rlcl}
 1& I have enough time.\pause{} &$\rightarrow$ &\myEmph[7,9-]{orange}{Do} I have enough time\myEmph[7,10-]{orange}{?}\pause{} \\
 2& You play the guitar.\pause{}&$\rightarrow$ &\myEmph[7,9-]{orange}{Do} you play the guitar\myEmph[7,10-]{orange}{?}\pause{} \\
 3& They speak French.\pause{}&$\rightarrow$ &\myEmph[7,9-]{orange}{Do} they speak French\myEmph[7,10-]{orange}{?}\pause
\end{tabular}
\pause

\vfill

\begin{exampleblock}{Topics for Today}
\pause
\begin{itemize}\small
 \item   先頭に `Do'を置く\pause
 \item   文末に`?'をつける(イントネーションは\myRisingPitch{})
\end{itemize}
     \end{exampleblock}

\vfill

\pause


\mbox{}\hfill\visible<11>{\myaudio{./audio/010_question_do_04.mp3}}

\end{frame}





\subsection{彼は〜しますか}
\begin{frame}[plain]{彼は〜しますか?}
 \Large

\pause


\mbox{}\hspace{50pt}\alt<2-3>{You}{He} sing\only<5->{\myAnch{T1}{orange}{s}} well. \only<4>{\ding{55}}
\hspace{85pt}\visible<3->{{\footnotesize well: じょうずに}}
\hspace{-150pt}%
\visible<7->{\raisebox{-5pt}{\scalebox{1.4}{\myDownwardPitch}}}

\hspace{100pt}%
\visible<6->{%
\myAnch{T2}{orange}{\small 三単現の`s'}
}

\vspace{30pt}

\onslide<8->{%
\myAnch{does}{orange}{Does} he sing well \myAnch{question}{orange}{?}
}
\onslide<9->{%
\scalebox{1.4}{\myRisingPitch}
}

\onslide<10->{%
\mbox{}\hspace{50pt}\myAnch{txt1}{white}{%
\small
\begin{tabular}[t]{@{}l}
先頭にDoes(Do\,\,\ding{55})\\
(Does he sing\textcolor{red}{s}\,\,\ding{55})
\end{tabular}
}
\hspace{60pt}\myAnch{txt2}{white}{\small 文末に`?'}

\begin{tikzpicture}[remember picture, overlay]
\draw[->, thick, orange] (txt1) to[out=180, in=-90] (does.south);
\draw[->, thick, orange] (txt2.west) to[out=180, in=-45] (question.south);
\end{tikzpicture}
}

\onslide<11->{%
\mbox{}\hfill\myaudio{./audio/010_question_do_05.mp3}
}
\end{frame}


\begin{frame}<1-10>[plain]\frametitle{Exercises}

つぎの文を疑問文にしましょう。

 \begin{enumerate}
  \item<1-> You like flowers.\hspace{59.7pt}
        \onslide<5->{$\longrightarrow$\,\,\,\,\, Do you like flowers?\hfill\scalebox{.75}{\bcfleur\bcfleur}}
  \item<1-> They live in Boston.\hspace{47.5pt}%
        \onslide<6->{$\longrightarrow$\,\,\,\,\, Do they live in Boston?\hfill\scalebox{.25}{\worldflag{US}}}
  \item<1-> She teaches science.\hspace{42pt}%
        \onslide<7->{$\longrightarrow$\,\,\,\,\, Does she teach science?\hfill\scalebox{1.75}{\twemoji{woman scientist}}}
  \item<1-> He has  a car.\hspace{80.5pt}%
        \onslide<8->{$\longrightarrow$\,\,\,\,\, Does he have a car?\hfill\faCar}
  \item<1-> Our teacher walks to school.
        \onslide<9->{$\longrightarrow$\,\,\,\,\, Does our teacher walk to school? \hfill\scalebox{.67}{\PHpedestrian}\,\,\scalebox{1.5}{\twemoji{school}}}
 \end{enumerate}

\begin{exampleblock}<2->{Topics for Today}
\begin{itemize}\small
 \item<3->   先頭に `Do'を置く\pause
 \item<4->   主語が三人称単数のときは`Does'
\end{itemize}
\end{exampleblock}
\vspace{-10pt}
% Embed the sound file
\onslide<10>{%
\mbox{}\hfill\myaudio{./audio/008_question_be_05.mp3}
}

\end{frame}


\end{document}

