\documentclass[aspectratio=169,xcolor={dvipsnames,table}]{beamer}
\usepackage[no-math,deluxe,haranoaji]{luatexja-preset}
\renewcommand{\kanjifamilydefault}{\gtdefault}
\renewcommand{\emph}[1]{{\upshape\bfseries #1}}
\usetheme{metropolis}
\metroset{block=fill}
\setbeamertemplate{navigation symbols}{}
\setbeamertemplate{blocks}[rounded][shadow=false]
\usecolortheme[rgb={0.7,0.2,0.2}]{structure}
%%%%%%%%%%%%%%%%%%%%%%%%%%%
\usepackage{media9}
%%%%%%%%%%%%%%%%%%%%%%%%%%%
%% さまざまなアイコン
%%%%%%%%%%%%%%%%%%%%%%%%%%%
\usepackage{fontawesome}
%\usepackage{figchild}
\usepackage{twemojis}
\usepackage{utfsym}
\usepackage{bclogo}
\usepackage{marvosym}
\usepackage{fontmfizz}
\usepackage{pifont}
\usepackage{phaistos}
\usepackage{worldflags}
\usepackage{manfnt}
%%%%%%%%%%%%%%%%%%%%%%%%%%%
\usepackage{tikz}
\usetikzlibrary{backgrounds}
\usepackage{tcolorbox}
\usepackage{tikzpeople}
\usepackage{xcolor}
\usepackage{amsmath}
\usepackage{circledsteps}
\usepackage{tipa}
%%%%%%%%%%%%%%%%%%%%%%%%%%%
%% 場合分け
\usepackage{cases}
%%%%%%%%%%%%%%%%%%%%%%%%%%%
% \myAnch{<名前>}{<色>}{<テキスト>}
% 指定のテキストを指定の色の四角枠で囲み, 指定の名前をもつTikZの
% ノードとして出力する. 図には remeber picture 属性を付けている
% ので外部から参照可能である.
\newcommand*{\myAnch}[3]{%
  \tikz[remember picture,baseline=(#1.base)]
    \node[draw,rectangle,#2] (#1) {\normalcolor #3};
}
%%%%%%%%%%%%%%%%%%%%%%%%%%%%
%% 音声リンク表示
\newcommand{\myaudio}[1]{\href{#1}{\faVolumeUp}}
%%%%%%%%%%%%%%%%%%%%%%%%%%%
% \myEmph コマンドの定義
%\newcommand{\myEmph}[3]{%
%    \textbf<#1>{\color<#1>{#2}{#3}}%
%}
\usepackage{xparse} % xparseパッケージの読み込み
\NewDocumentCommand{\myEmph}{O{} m m}{%
    \def\argOne{#1}%
    \ifx\argOne\empty
        \textbf{\color{#2}{#3}}% オプション引数が省略された場合
    \else
        \textbf<#1>{\color<#1>{#2}{#3}}% オプション引数が指定された場合
    \fi
}
%%%%%%%%%%%%%%%%%%%%%%%%%%%
%% 文末の上昇イントネーション記号 \myRisingPitch
%% 通常のイントネーション \myDownwardPitch
%% https://note.com/dan_oyama/n/n8be58e8797b2
%%%%%%%%%%%%%%%%%%%%%%%%%%%
\newcommand{\myRisingPitch}{
\begin{tikzpicture}[scale=0.3,baseline=0.3]
\draw[->,>=stealth] (0,0) to[bend right=45] (1,1);
\end{tikzpicture}
}
\newcommand{\myDownwardPitch}{
\begin{tikzpicture}[scale=0.3,baseline=0.3]
\draw[->,>=stealth] (0,1) to[bend left=45] (1,0);
\end{tikzpicture}
}
%%%%%%%%%%%%%%%%%%%%%%%%%%%
\title{English is fun.}
\subtitle{Do you like baseball?}
\author{}
\institute[]{}
\date[]

%%%%%%%%%%%%%%%%%%%%%%%%%%%%
%% TEXT
%%%%%%%%%%%%%%%%%%%%%%%%%%%%
\begin{document}
\begin{frame}[plain]
  \titlepage
\end{frame}

\section*{授業の流れ}
\begin{frame}[plain]
  \frametitle{授業の流れ}
  \tableofcontents
\end{frame}

%%%%%%%%%%%%%%%%%%%%%%%%%%%%%%%%
\section{一般動詞の疑問文}
\subsection{わたしは〜しますか}
%%%%%%%%%%%%%%%%%%%%%%%%%%%%%%%%
\begin{frame}[plain]{わたしは〜しますか?}
 \large

\visible<2->{1.\,\,\,\mbox{}\hspace{35pt}I have  enough time.}%
\visible<3->{\raisebox{-5pt}{\scalebox{1.4}{\myDownwardPitch}}}%
\hfill\visible<2->{{\scriptsize enough \textipa{/In\'2f/} じゅうぶんな}}


\vspace{15pt}

\visible<4->{2.\,\,\,\myAnch{do}{Maroon}{\bfseries Do} I have enough time \myAnch{question}{Maroon}{?}}
\visible<7->{\scalebox{1.4}{\myRisingPitch}}

\mbox{}\hspace{50pt}%
\visible<5->{\myAnch{txt1}{white}{先頭に\Circled[fill color = yellow!50]{\,Do\,}}}
\hspace{80pt}
\visible<6->{\myAnch{txt2}{white}{文末に\Circled[fill color = yellow!50]{?}}}

\begin{tikzpicture}[remember picture, overlay]
\visible<5->{\draw[->, thick, Maroon] (txt1.west) to[out=180, in=-90] (do.south);}
\visible<6->{\draw[->, thick, Maroon] (txt2.west) to[out=180, in=-90] (question.south);}
\end{tikzpicture}

\hfill{\tiny 0110}\,{\scriptsize \myaudio{./audio/010_question_do_01.mp3}}
\end{frame}
%%%%%%%%%%%%%%%%%%%%%%%%%%%%%%%%%
\subsection{あなたは〜しますか}
%%%%%%%%%%%%%%%%%%%%%%%%%%%%%%%%%
\begin{frame}[plain]{あなたは〜しますか?}
 \large

\visible<2->{1.\,\,\,\mbox{}\hspace{35pt}You play the guitar.}
\visible<3->{\raisebox{-5pt}{\scalebox{1.4}{\myDownwardPitch}}}
\hfill\visible<2->{{\scriptsize guitar \textipa{/gIt\'A\textrhookschwa /} ギター}}

\vspace{15pt}

\visible<4->{2.\,\,\,\myAnch{do}{Maroon}{\bfseries Do} you play the guitar \myAnch{question}{Maroon}{?}}
\visible<7->{\scalebox{1.4}{\myRisingPitch}}

\mbox{}\hspace{50pt}\visible<5->{\myAnch{txt1}{white}{先頭に\Circled[fill color = yellow!50]{\,Do\,}}}
\hspace{80pt}\visible<6->{\myAnch{txt2}{white}{文末に\Circled[fill color = yellow!50]{?}}}


\begin{tikzpicture}[remember picture, overlay]
\visible<5->{\draw[->, thick, Maroon] (txt1.west) to[out=180, in=-90] (do.south);}
\visible<6->{\draw[->, thick, Maroon] (txt2.west) to[out=180, in=-90] (question.south);}
\end{tikzpicture}

\hfill{\tiny 0110}\,{\scriptsize \myaudio{./audio/010_question_do_02.mp3}}

\end{frame}
%%%%%%%%%%%%%%%%%%%%%%%%%%%%%%
\subsection{彼らは〜しますか}
%%%%%%%%%%%%%%%%%%%%%%%%%%%%%
\begin{frame}[plain]{彼らは〜しますか?}
 \large

\mbox{}\hspace{35pt}%
\visible<2->{They speak French.}
\visible<3->{\raisebox{-5pt}{\scalebox{1.4}{\myDownwardPitch}}}
\hfill\visible<2->{{\scriptsize speak \textipa{/sp\'\i :k/} 話す French \textipa{/fr\'entS/} フランス語}}

\vspace{15pt}

\visible<4->{\myAnch{do}{Maroon}{\bfseries Do} they speak French \myAnch{question}{Maroon}{?}}
\visible<7->{\scalebox{1.4}{\myRisingPitch}}

\mbox{}\hspace{50pt}\visible<5->{\myAnch{txt1}{white}{先頭に\Circled[fill color = yellow!50]{\,Do\,}}}
\hspace{80pt}\visible<6->{\myAnch{txt2}{white}{文末に\Circled[fill color = yellow!50]{?}}}


\begin{tikzpicture}[remember picture, overlay]
\visible<5->{\draw[->, thick, Maroon] (txt1.west) to[out=180, in=-90] (do.south);}
\visible<6->{\draw[->, thick, Maroon] (txt2.west) to[out=180, in=-90] (question.south);}
\end{tikzpicture}

\vfill

\mbox{}\hfill{\tiny 0109}\,{\scriptsize \myaudio{./audio/010_question_do_03.mp3}}

\end{frame}

%%%%%%%%%%%%%%%%%%%%%%%
\begin{frame}[plain]\frametitle{Exercises}

つぎの各文を疑問文にしましょう

\begin{tabular}{rlcl}
 1& I have enough time.&$\rightarrow$ &\visible<2->{\myEmph[2-]{Maroon}{Do} I have enough time\myEmph[2-]{Maroon}{?}}\\
 2& You play the guitar.&$\rightarrow$ &\visible<3->{\myEmph[3-]{Maroon}{Do} you play the guitar\myEmph[3-]{Maroon}{?}}\\
 3& They speak French.&$\rightarrow$ &\visible<4->{\myEmph[4-]{Maroon}{Do} they speak French\myEmph[4-]{Maroon}{?}}
\end{tabular}

\vfill

\begin{block}<5->{Topics for Today}\small
一般動詞の疑問文のつくり方
\begin{itemize}\setbeamertemplate{items}[square]\small
 \item<6->   先頭に \Circled[fill color = yellow!50]{\,Do\,}を置く
 \item<7->   文末に\Circled[fill color = yellow!50]{?}をつける
 \item<8-> イントネーションは\myRisingPitch{}\\[10pt]
	  \mbox{}\hfill\visible<9->{{\bfseries $\text{S}+\text{V\,\ldots\,.}\longrightarrow\text{Do}+\text{S}+\text{V($=$原形)}$\,\ldots\,? \myRisingPitch}}\hfill\mbox{}
\end{itemize}
     \end{block}

\vfill

\mbox{}\hfill{\tiny 0212}{\scriptsize \myaudio{./audio/010_question_do_04.mp3}}

\end{frame}
%%%%%%%%%%%%%%%%%%%%%%%%%%%%%%%%%%%%%%%%%%%%
\subsection{彼は〜しますか}
%%%%%%%%%%%%%%%%%%%%%%%%%%%%%%%%%%%%%%%%%%%%
\begin{frame}[plain]{彼は〜しますか?}
 \large

1.\,\,\,\mbox{}\hspace{50pt}\alt<2-3>{You}{He} sing\only<5->{\myAnch{T1}{Maroon}{s}} well. \only<4>{\ding{55}}
\visible<7->{\raisebox{-5pt}{\scalebox{1.4}{\myDownwardPitch}}}
\hfill\visible<3->{{\scriptsize well \textipa{/w\'el/} じょうずに}}

\hspace{120pt}%
\visible<6->{%
\myAnch{T2}{Maroon}{\scriptsize 3単現の\Circled[fill color=yellow!50]{s}}
}

\vspace{5pt}

\onslide<8->{%
2.\,\,\,\myAnch{does}{Maroon}{\bfseries Does} he sing well \myAnch{question}{Maroon}{?}}
\onslide<9->{%
\scalebox{1.4}{\myRisingPitch}}
\hfill\onslide<8->{{\scriptsize does \textipa{/d\'2z/}}}

\onslide<10->{%
\mbox{}\hspace{50pt}\myAnch{txt1}{white}{%
\small
\begin{tabular}[t]{@{}l}
先頭に\Circled[fill color = yellow!50]{\,Does\,}\,(Do\,\,\ding{55})\\
(*Do\textcolor{Maroon}{es} he sing\textcolor{Maroon}{s} well?\,\,\dbend)
\end{tabular}
}
\hspace{60pt}\visible<11->{\myAnch{txt2}{white}{\small 文末に\Circled[fill color = yellow!50]{?}}}

\begin{tikzpicture}[remember picture, overlay]
\draw[->, thick, Maroon] (txt1) to[out=180, in=-90] (does.south);
\visible<11->{\draw[->, thick, Maroon] (txt2.west) to[out=180, in=-45] (question.south);}
\end{tikzpicture}
}

\vspace*{-10pt}

\begin{block}<12->{主語が3人称単数のとき}\small
\begin{itemize}\setbeamertemplate{items}[square]
\small
 \item<12->   先頭に \Circled[fill color=yellow!50]{\,Does\,}\,を置く\hfill{}続く動詞は原形です cf. He does not live in Boston.\\
\hfill{}\visible<13->{*He do\textcolor{Maroon}{es} not live\textcolor{Maroon}{s} in Boston.}
 \item<14->   文末に\,\Circled[fill color=yellow!50]{?}\,をつける / イントネーションは\myRisingPitch{}\\
         \mbox{}\hspace{80pt}{\bfseries $\text{S}+\text{V\,\ldots\,.}\longrightarrow\text{\Circled[fill color = yellow!50]{\,Does\,}}+\text{S}+\text{V($=$原形)}$\,\ldots\,?} 
\end{itemize}
     \end{block}

\vspace*{-8pt}

\hfill{\tiny 0108}\,{\scriptsize \myaudio{./audio/010_question_do_05.mp3}}
\end{frame}
%%%%%%%%%%%%%%%%
\begin{frame}[plain]\frametitle{Exercises}

つぎの文を疑問文にしましょう

 \begin{enumerate}
  \item<1-> You like flowers.\hspace{59.7pt}
        \onslide<2->{$\longrightarrow$\,\,\,\,\, Do you like flowers?\hfill\scalebox{.75}{\bcfleur\bcfleur}}
  \item<1-> They live in Boston.\hspace{47.5pt}%
        \onslide<3->{$\longrightarrow$\,\,\,\,\, Do they live in Boston?}
  \item<1-> She teaches science.\hspace{42pt}%
        \onslide<4->{$\longrightarrow$\,\,\,\,\, Does she teach science?\hfill\scalebox{1.75}{\twemoji{woman scientist}}}
  \item<1-> He has  a car.\hspace{80.5pt}%
        \onslide<5->{$\longrightarrow$\,\,\,\,\, Does he have a car?\hfill\faCar}
  \item<1-> Our teacher walks to school.
        \onslide<6->{$\longrightarrow$\,\,\,\,\, Does our teacher walk to school? \hfill\scalebox{.67}{\PHpedestrian}\,\,\scalebox{1.5}{\twemoji{school}}}
 \end{enumerate}

\hfill{\tiny 0329}\,{\scriptsize \myaudio{./audio/010_question_do_06.mp3}}

\worldflag{GB}
\end{frame}
%%%%%%%%%%%%%%%%%%%%%
\section{まとめ}%
%%%%%%%%%%%%%%%%%%%%%
%%%%%%%%%%%%%%%%%%%%%%%%%%%%%%%
\begin{frame}[plain]{まとめ}
 
\begin{block}{一般動詞の疑問文のつくり方}\small
\begin{itemize}\setbeamertemplate{items}[square]
 \item   先頭にDoまたはDoes / 続く動詞は原形です
 \item 主語(S)が3人称単数のときは\,\Circled[fill color=white]{\,\,Does\,\,}\,(それ以外はすべて\,\Circled[fill color=white]{\,\,Do\,\,}\,)\\
	 \Circled[fill color = white]{\,\,Do\,\,}\,$+$\,S\,$+$\,V{\scriptsize ($=$\,原形)} \ldots\,\,\,?\hfill{}Do you like sushi?\\
	 \Circled[fill color = white]{\,\,Does\,\,}\,$+$\,S\,$+$\,V{\scriptsize ($=$\,原形)} \ldots\,\,\,?\hfill{}Does she play the guitar?\\
\hfill{}*Does she play\Circled{\textbf{s}} the guitar?
 \item  文末に\,\Circled[fill color=white]{?}\,をつける / イントネーションは\myRisingPitch{}
% \item   文末に`?'をつける
\end{itemize}
     \end{block}

\begin{block}{Pronunciation}
\mbox{}\hfill{}do \textipa{/d\'u:/}\hspace{30pt}does \textipa{/d\'2z/}\hfill\mbox{}
\end{block}
\end{frame}
%%%%%%%%%%%%%%%%%%%%%%%%%%%%%
\begin{frame}[plain]{まとめ}
\begin{block}{一般動詞の疑問文のつくり方}
$\left\{
\begin{array}{l}
 \textbf{Do}\\
 \textbf{Does}
\end{array}
\right\} + \textbf{S} + \textbf{原形}$ \ldots \textbf{?}
\end{block}

\begin{enumerate}
 \item \begin{enumerate}
	\item They play baseball on Sundays.
	\item Do they play baseball on Sundays?
       \end{enumerate}
 \item \begin{enumerate}
	\item She plays tennis on Saturdays.
	\item Does she play tennis on Saturdays? 
	\item *Does she plays tennis on Saturdays? 
       \end{enumerate}
\end{enumerate}

\end{frame}
%%%%%%%%%%%%%%%%%%%%%%%%%%%%%%%%%%%%%%%%%
\end{document}

