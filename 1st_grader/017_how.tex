\documentclass[aspectratio=169,xcolor={dvipsnames,table}]{beamer}
\usepackage[no-math,deluxe,haranoaji]{luatexja-preset}
\renewcommand{\kanjifamilydefault}{\gtdefault}
\renewcommand{\emph}[1]{{\upshape\bfseries #1}}
\usetheme{metropolis}
\metroset{block=fill}
\setbeamertemplate{navigation symbols}{}
\setbeamertemplate{blocks}[rounded][shadow=false]
\usecolortheme[rgb={0.7,0.2,0.2}]{structure}
%%%%%%%%%%%%%%%%%%%%%%%%%%%
%%%%%%%%%%%%%%%%%%%%%%%%%%%
%% さまざまなアイコン
%%%%%%%%%%%%%%%%%%%%%%%%%%%
%\usepackage{fontawesome}
\usepackage{fontawesome5}
\usepackage{figchild}
\usepackage{twemojis}
\usepackage{utfsym}
\usepackage{bclogo}
\usepackage{marvosym}
\usepackage{fontmfizz}
\usepackage{pifont}
\usepackage{phaistos}
\usepackage{worldflags}
\usepackage{jigsaw}
\usepackage{tikzlings}
\usepackage{tikzducks}
\usepackage{scsnowman}
\usepackage{epsdice}
\usepackage{halloweenmath}
\usepackage{svrsymbols}
\usepackage{countriesofeurope}
\usepackage{tipa}
\usepackage{manfnt}
%%%%%%%%%%%%%%%%%%%%%%%%%%%
\usepackage{tikz}
\usetikzlibrary{calc,patterns,decorations.pathmorphing,backgrounds}
\usepackage{tcolorbox}
\usepackage{tikzpeople}
\usepackage{circledsteps}
\usepackage{xcolor}
\usepackage{amsmath}
\usepackage{booktabs}
\usepackage{chronology}
\usepackage{signchart}
%%%%%%%%%%%%%%%%%%%%%%%%%%%
%% 場合分け
%%%%%%%%%%%%%%%%%%%%%%%%%%%
\usepackage{cases}
%%%%%%%%%%%%%%%%%%%%%%%%%%
\usepackage{pdfpages}
%%%%%%%%%%%%%%%%%%%%%%%%%%%
%% 音声リンク表示
\newcommand{\myaudio}[1]{\href{#1}{\faVolumeUp}}
%%%%%%%%%%%%%%%%%%%%%%%%%%
%% \myAnch{<名前>}{<色>}{<テキスト>}
%% 指定のテキストを指定の色の四角枠で囲み, 指定の名前をもつTikZの
%% ノードとして出力する. 図には remember picture 属性を付けている
%% ので外部から参照可能である.
\newcommand*{\myAnch}[3]{%
  \tikz[remember picture,baseline=(#1.base)]
    \node[draw,rectangle,line width=1pt,#2] (#1) {\normalcolor #3};
}
%%%%%%%%%%%%%%%%%%%%%%%%%%
%% \myEmph コマンドの定義
%%%%%%%%%%%%%%%%%%%%%%%%%%
%\newcommand{\myEmph}[3]{%
%    \textbf<#1>{\color<#1>{#2}{#3}}%
%}
\usepackage{xparse} % xparseパッケージの読み込み
\NewDocumentCommand{\myEmph}{O{} m m}{%
    \def\argOne{#1}%
    \ifx\argOne\empty
        \textbf{\color{#2}{#3}}% オプション引数が省略された場合
    \else
        \textbf<#1>{\color<#1>{#2}{#3}}% オプション引数が指定された場合
    \fi
}
%%%%%%%%%%%%%%%%%%%%%%%%%%%
%%%%%%%%%%%%%%%%%%%%%%%%%%%
%% 文末の上昇イントネーション記号 \myRisingPitch
%% 通常のイントネーション \myDownwardPitch
%% https://note.com/dan_oyama/n/n8be58e8797b2
%%%%%%%%%%%%%%%%%%%%%%%%%%%
\newcommand{\myRisingPitch}{
\begin{tikzpicture}[scale=0.3,baseline=0.3]
\draw[->,>=stealth] (0,0) to[bend right=45] (1,1);
\end{tikzpicture}
}
\newcommand{\myDownwardPitch}{
\begin{tikzpicture}[scale=0.3,baseline=0.3]
\draw[->,>=stealth] (0,1) to[bend left=45] (1,0);
\end{tikzpicture}
}
%%%%%%%%%%%%%%%%%%%%%%%%%%%%
%\AtBeginSection[%
%]{%
%  \begin{frame}[plain]\frametitle{授業の流れ}
%     \tableofcontents[currentsection]
%   \end{frame}%
%}

\usepackage{pxrubrica}
\usepackage{highlightx}
%%%%%%%%%%%%%%%%%%%%%%%%%%%
\title{English is fun.}
\subtitle{How are you?}
\author{}
\institute[]{}
\date[]

%%%%%%%%%%%%%%%%%%%%%%%%%%%%
%% TEXT
%%%%%%%%%%%%%%%%%%%%%%%%%%%%
\begin{document}
\begin{frame}[plain]
  \titlepage
\end{frame}

\section*{授業の流れ}
\begin{frame}[plain]
  \frametitle{授業の流れ}
  \tableofcontents
\end{frame}

%%%%%%%%%%%%%%%%%%%%%%%%%%%%%%%
\section{How \textipa{/h\'aU/}}
\subsection{How are you?: be動詞のとき}
\begin{frame}[plain]{どんな状態で: be動詞と}
\large

\mbox{}\hspace{50pt}Your father is  \alt<3->{\myAnch{FOCUS}{orange}{fine}}{\myAnch{focus}{white}{fine}}.

\pause

\vspace{7pt}

\mbox{}\hfill{}cf. \myEmph[6-]{Maroon}{Is your father} fine?%

\vspace{-5pt}

\hfill{\scriptsize Yes / Noで答える疑問文}

\pause

\visible<4->{\myAnch{wh}{orange}{How} \myEmph[6-]{Maroon}{is your father} \myAnch{question}{orange}{?}}
\visible<6->{\scalebox{1.4}{\myDownwardPitch}}

\visible<5->{%
\begin{tikzpicture}[remember picture, overlay]
\draw[->, line width=3pt,opacity=.5, orange] (focus.south) to[out=-90, in=90] node[sloped,above,text=black,font=\tiny,pos=.667]{Howに置き換えて先頭へ}  node[sloped,below,text=black,font=\tiny,pos=.5]{後ろは疑問文の語順} (wh.north);
\end{tikzpicture}
}

\begin{block}<7->{Topics for Today}
\begin{itemize}\setbeamertemplate{items}[square]\small
 \item 「どんな状態で」と聞くとき $\rightarrow$\, \textbf{how} \textipa{/h\'aU/}
 \item {\bfseries How}の後ろは疑問文の語順
\hfill$\HighlightFormula{\text{How} + \text{be動詞} + \text{S ?}}$\\
\hfill{\scriptsize \textbf{How} is the weather?}
 \item   文末に`?'をつける / イントネーションは\myDownwardPitch{}
 \item 短縮形{\bfseries How is} $\longrightarrow$ {\bfseries How's} \textipa{/h\'aUz/}\hfill{\scriptsize \textbf{How's} the weather?}
\end{itemize}
     \end{block}

\vspace{-10pt}
\hfill{\tiny 0133}\,{\scriptsize \myaudio{./audio/017_how_01.mp3}}
\end{frame}
%%%%%%%%%%%%%%%%%%%%%%%%%%%%%%%%%%%%%%%%%%%
\begin{frame}[plain]{どんな状態で: be動詞と}
\large
\mbox{}\hspace{46pt}Your parents are  \alt<3->{\myAnch{FOCUS_2}{orange}{fine}}{\myAnch{focus_2}{white}{fine}}.

\pause


\vspace{7pt}

\mbox{}\hfill{}cf. \myEmph[6-]{Maroon}{Are your parents} fine?%

\vspace{-5pt}

\hfill{\scriptsize Yes / Noで答える疑問文}

\visible<4->{\myAnch{wh_2}{orange}{How} \myEmph[6-]{Maroon}{are your parents} \myAnch{question}{orange}{?}}
\visible<6->{\scalebox{1.4}{\myDownwardPitch}}

\visible<5->{%
\begin{tikzpicture}[remember picture, overlay]
\draw[->, line width=3pt,opacity=.5, orange] (focus_2.south) to[out=-90, in=90] node[sloped,above,text=black,font=\tiny,pos=.667]{Howに置き換えて先頭へ} node[sloped,below,text=black,font=\tiny,pos=.5]{後ろは疑問文の語順} (wh_2.north);
\end{tikzpicture}
}

\vspace{-8pt}

\begin{block}<7->{Topics for Today}\small
\begin{itemize}\setbeamertemplate{items}[square]\small
 \item 「どんな状態で」と聞くとき $\rightarrow$\, \textbf{how} \textipa{/h\'aU/}
 \item {\bfseries How}の後ろは疑問文の語順
\hfill$\HighlightFormula{\text{How} + \text{be動詞} + \text{S ?}}$\\
\hfill{\scriptsize \textbf{How} are you?}
%$\longrightarrow$\,\,\,{\bfseries How is} S($=$ 単数形)\,? / {\bfseries How are} S($=$ 複数形)\,?
 \item   文末に`?'をつける / イントネーションは\myDownwardPitch{}
 \item 短縮形 {\bfseries How are} $\longrightarrow$ {\bfseries How're} \textipa{/h\'aU\textrhookschwa /}
\end{itemize}
     \end{block}

\vspace{-12pt}

\hfill{\tiny 0135}\,{\scriptsize \myaudio{./audio/017_how_02.mp3}}%

\end{frame}
%%%%%%%%%%%%%%%%%%%%%%%%%%%%%%%%%%%%%
\begin{frame}[plain]{Exercises 1}

 {\small 日本語の意味になるように、語句を並べ替えましょう。先頭にくる語は大文字で書き始めてください}
\begin{enumerate}
 \item ( is / mother / how / your )
{\small あなたのお母さんはいかがですか?}\\
\visible<2->{How is your mother?}\\
\visible<3->{--- She is very well, thank you.{\small おかげさまでとても元気です}}\\
 \item  ( how / you / are ){\small お元気ですか?}\\ 
\visible<4->{How are you?}
\visible<5->{--- Good.{\small 元気ですよ}}\\
 \item ( is / in / how / the weather / Boston )
{\small ボストンの天気はどうですか?}\\
\visible<6->{How is the weather in Boston?}
\visible<7->{---\begin{tabular}[t]{l@{\,\,\,}lll}
	      It's& sunny. &{\small (晴れた)}&{$<$ sun} \\
                  & \visible<8->{cloudy.}&\visible<8->{\small (曇った)}&\visible<8->{$< $ cloud}\\
                  & \visible<9->{rainy.}&\visible<9->{\small (雨降りの)}&\visible<9->{$<$ rain}\\
                  & \visible<10->{snowy.}&\visible<10->{\small (雪の降る)}&\visible<10->{$<$ snow}
	     \end{tabular}}
\end{enumerate}
\hfill{\tiny 0219}\,{\scriptsize \myaudio{./audio/017_how_03.mp3}}
\end{frame}
%%%%%%%%%%%%%%%%%%%%%%%%%%%%%%%%%%%%%
\subsection{How does she go to work?: 一般動詞のとき}
\begin{frame}[plain]\frametitle{どうやって、どんな方法で: 一般動詞と}
%一般動詞のとき
\large

\mbox{}\hspace{55pt}%
She goes to work \alt<4->{\myAnch{FOCUS_3}{orange}{by bus}}{\myAnch{focus_3}{white}{by bus}}.\hfill{\scriptsize by ~によって(交通手段)}

\mbox{}\hfill%
{\small cf. \myEmph[6-]{Maroon}{Does she go} to work by bus?}

\vspace{-5pt}

\hfill\visible<3->{{\scriptsize Yes/Noで答える疑問文}}

\visible<5->{\myAnch{WH_3}{orange}{\bfseries How} \myEmph[6-]{Maroon}{does she go} to work\myAnch{question2}{orange}{?}}

\visible<5->{%
\begin{tikzpicture}[remember picture, overlay]
 \draw[line width=3pt,opacity=.5,orange, ->] (focus_3.south) to[out=-90, in=90] node[sloped,above,text=black,font=\tiny,pos=.667]{Howに置き換えて先頭へ}  node[sloped,below,text=black,font=\tiny,pos=.5]{後ろは疑問文の語順}(WH_3.north);
\end{tikzpicture}
}

\begin{block}<7->{Topics for Today}\small
\begin{itemize}\setbeamertemplate{items}[square]\small
 \item<7-> 疑問詞{\bfseries how}が「どうやって、どんな方法で」の意味になることがあります
 \item<8-> {\bfseries How}を先頭に置いて、疑問文のかたちを続ける\\
\hfill{}How do you  〜? / How does she 〜? 
 \item<9->   文末に`?'をつける / イントネーションは\myDownwardPitch{}
\end{itemize}
     \end{block}

\vspace{-10pt}
\hfill{\tiny 0136}\,{\scriptsize \myaudio{./audio/017_how_04.mp3}}

\end{frame}
%%%%%%%%%%%%%%%%%%%%%%%%%%%
\begin{frame}[plain]{Exercises 2}
 日本語の意味になるように、空所に適当な語を補いましょう
\begin{enumerate}
 \item {\small その映画について、あなたはどう感じていますか --- すばらしい}\\
       (~~\alt<4->{How}{\phantom{How}}~~)  (~~\alt<4->{do}{\phantom{do}}~~)  you feel about the movie? --- It is great.
 \item {\small あなたは冬休みをどうやってすごしますか --- いつもは祖母のところに行きます}\\
       (~~\alt<5->{How}{\phantom{How}}~~)  do (~~\alt<5->{you}{\phantom{you}}~~)  spend the winter vacation?\\
 --- I usually visit my grandmother.%
\hfill{\scriptsize usually \textipa{/j\'u:Zu@li/} ふつうは、いつもは}
 \item {\small 彼はどうやって公園に行きますか --- 歩いて行きます}\\
       (~~\alt<6->{How}{\phantom{How}}~~)  (~~\alt<6->{does}{\phantom{does}}~~)  he go to the park? --- He goes there on  (~~\alt<7->{foot}{\phantom{foot}}~~) .
\end{enumerate}

\mbox{}\hfill\visible<8->{{\scriptsize on foot: 徒歩で}}\\
\hfill \visible<9->{{\scriptsize by car / by bus / by train / by air}}

\hfill{\tiny 0243}\,{\scriptsize \myaudio{./audio/017_how_05.mp3}}

\end{frame}
%%%%%%%%%%%%%%%%%%%%%%%%%%%%%%
\subsection{\fbox{How $+$ X} どれくらい(程度)}
%%%%%%%%%%%%%%%%%%%%%%%%%%%%%%%%
\begin{frame}[plain]{How $+$ X どれくらい(程度)}
 \begin{enumerate}
  \item<1-> \fbox{How} are you?\hfill{\scriptsize どんな状態で}
  \item<2-> \fbox{How} do you go to work?\hfill{\scriptsize どんな方法で}
  \item<3-> \fbox{How old} are you?\hfill{\scriptsize どれくらい}\\
\hfill\visible<4->{{\scriptsize どのくらい年をとっているのか $\longrightarrow$\,何歳か}}
 \end{enumerate}

\bigskip

\begin{block}<5->{Topics for Today}
 \begin{itemize}\setbeamertemplate{items}[square]\small
  \item {\bfseries How}が単独ではなく\,\,\fcolorbox{black}{white}{\bfseries How $+$ X\,}\,のように他の語とセットで使われることがあります
  \item この{\bfseries How}は「どのくらい~」と\kenten{程度}をたずねる疑問詞です
\end{itemize}
\end{block}
\end{frame}
%%%%%%%%%%%%%%%%%%%%%%%%%%%%%%%%
\begin{frame}[plain]{How $+$ X}
 \begin{enumerate}
  \item \myEmph[2-]{Maroon}{\bfseries How old} is the building? --- It is 80 years old.%
\hfill{}\visible<2->{{\small \fbox{How\raisebox{5pt}{\scalebox{1.5}[1]{$\curvearrowright$}}old}}}
  \item \myEmph[3-]{Maroon}{\bfseries How tall} is the tower? --- It is fifty meters.%
\hfill{}\visible<3->{{\small {\fbox{How\raisebox{5pt}{\scalebox{1.5}[1]{$\curvearrowright$}}tall}}}}
  \item \myEmph[4-]{Maroon}{\bfseries How long} is the bridge? --- It is 100 meters.
\hfill{}\visible<4->{{\small {\fbox{How\raisebox{5pt}{\scalebox{1.5}[1]{$\curvearrowright$}}long}}}}
  \item \myEmph[5-]{Maroon}{\bfseries How many} dogs do you have? --- I have two.
\hfill{}\visible<5->{{\small {\fbox{How\raisebox{5pt}{\scalebox{1.5}[1]{$\curvearrowright$}}many\raisebox{5pt}{\scalebox{1.5}[1]{$\curvearrowright$}}dogs}}}}
  \item \myEmph[6-]{Maroon}{\bfseries How much} is the book? --- It is 10 dollars.
\hfill{}\visible<6->{{\small {\fbox{How\raisebox{5pt}{\scalebox{1.5}[1]{$\curvearrowright$}}much}}}}
 \end{enumerate}

\begin{block}<7->{Topics for Today}\small
\pause
\begin{itemize}\setbeamertemplate{items}[square]\small
 \item 「どのくらい」と`程度'をたずねるとき$\longrightarrow$\,\,\,\fcolorbox{black}{white}{\bfseries How $+$ X\,}\,\,\,\ldots{}\,\,\,?
 \item \fcolorbox{black}{white}{\bfseries How $+$ X\,}\,\,の後は疑問文と同じ語順になります\hfill{\bfseries How old} is the building\\
\hfill{}{\bfseries How many} comics do you have?
 \item   {\bfseries How much}は「いくら」と`値段'をたずねるときの決まり文句です
\end{itemize}
     \end{block}


\hfill{\tiny 0327}\,{\scriptsize\myaudio{./audio/017_how_06.mp3}}

\end{frame}
%%%%%%%%%%%%%%%%%%%%%%%%%%%%%%
\begin{frame}[plain]{How $+$ X ...\,\,\,?}
 
 \begin{center}
 \rowcolors{1}{NavyBlue!50}{yellow!40}
\begin{tabular}{ll}\toprule
年齢・古さ& How \visible<2->{old \ldots{}\, ?}\\
身長・高さ& How \visible<3->{tall \ldots{}\, ?}\\
長さ&How \visible<4->{long \ldots{}\, ?}\\
数&How \visible<5->{many 名詞(複数形) \ldots{}\, ?}\\
値段&How \visible<6->{much \ldots{}\, ?}\\
頻度&How \visible<7->{often \ldots{}\, ?}\\
\bottomrule
\end{tabular}
\end{center}

\hfill\visible<7->{\scriptsize often \textipa{/\'O:f(t)n/} しばしば}

\hfill\visible<8->{\textbf{How often} do you studay math? --- Every day.}
\end{frame}
%%%%%%%%%%%%%%%%%%
\begin{frame}[plain]{Exercises 3}
 
 {\small 次の対話の ( ) に入れるのにもっとも適切なものをア~エから
それぞれ1つずつ選びましょう}
\begin{columns}
\begin{column}{.65\textwidth}
\begin{enumerate}
 \item A: How many brothers do you have?\\
B: \alt<2->{( Two brothers. ) イ}{( \phantom{Two brothers.} )}
\item A: How much is the notebook? \\
B:  \alt<3->{( 100 yen. ) エ}{(\phantom{ 100 yen. })} 
 \item  A: How many hours do you study every night?\\ 
B:  \alt<4->{( About two hours. ) ウ}{(\phantom{ About two hours. })} 
  \item A: How old is your brother?\\
B:  \alt<5->{( Fifteen years old. ) ア}{(\phantom{Fifteen years old. })}
\end{enumerate} 
\end{column}
\begin{column}{.35\textwidth}
\begin{tcolorbox}
 
ア Fifteen years old.\\
イ Two brothers.\\
ウ About two hours.\\
エ 100 yen.
\end{tcolorbox}
\hfill{\tiny 0257}\,{\scriptsize \myaudio{./audio/017_how_07.mp3}}
\end{column}
\end{columns}
\end{frame}
%%%%%%%%%%%%%%%%%%%%%%%%%%%%%%%%%%%%%%%
\section{疑問詞How \textipa{/h\'aU/} のまとめ}
%%%%%%%%%%%%%%%%%%%%%%%
\begin{frame}[plain]{まとめ}
 \begin{block}{be動詞のとき}\small
\begin{itemize}\setbeamertemplate{items}[square]\small
 \item 「〜はどうですか、どんな状態ですか」と聞くとき
({\bfseries How}の後ろは疑問文の語順)\\
$\longrightarrow$\,\,\,{\bfseries How is} 〜? / {\bfseries How are} 〜?
 \item   文末に`?'をつける(イントネーションは\myDownwardPitch{}\,\,)
 \item 短縮形 {\bfseries How is} $\longrightarrow$ {\bfseries How's} / {\bfseries How are} $\longrightarrow$ {\bfseries How're}
\end{itemize}
     \end{block}

\begin{block}{一般動詞のとき}\small
\begin{itemize}\setbeamertemplate{items}[square]\small
 \item 疑問詞{\bfseries how}が「どんな風に、どうやって、どんな方法で」の意味になることがあります
 \item `{\bfseries How}'を先頭に置いて、疑問文のかたちを続ける\,\,How do you  〜? / How does she 〜? 
 \item   文末に`?'をつける(イントネーションは\myDownwardPitch{}\,\,)
\end{itemize}
     \end{block}
\end{frame}

\begin{frame}[plain]{まとめ}
 \begin{block}{How $+$ X}\small
\begin{itemize}\setbeamertemplate{items}[square]\small
 \item 「どのくらい」と`程度'をたずねるとき$\longrightarrow$\,\,\,$\text{\bfseries How} + \text{形容詞}$\,\ldots{}\,\,\,? / $\text{\bfseries How} + \text{副詞}$\,\ldots{}\,\,\,?
 \item \fcolorbox{black}{white}{\bfseries How $+$ X\,}\,\,の後は疑問文と同じ語順になります\hfill{\bfseries How old} is the building?\\
\hfill{}{\bfseries How many} comics do you have?
 \item   {\bfseries How much}は「いくら」と`値段'をたずねるときの決まり文句です
\end{itemize}
     \end{block}

 \begin{center}
 \rowcolors{1}{NavyBlue!50}{yellow!40}
\begin{tabular}{ll}\toprule
年齢・古さ& How old \ldots{}\, ?\\
身長・高さ& How tall \ldots{}\, ?\\
長さ&How long \ldots{}\, ?\\
数&How many 名詞 \ldots{}\, ?\\
値段&How much \ldots{}\, ?\\
頻度&How often \ldots{}\, ?\\
\bottomrule
\end{tabular}
\end{center}
\end{frame}
\end{document}

