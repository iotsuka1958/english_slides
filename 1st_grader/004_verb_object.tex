\documentclass[aspectratio=169,xcolor={dvipsnames,table}]{beamer}
\usepackage[no-math,deluxe,haranoaji]{luatexja-preset}
\renewcommand{\kanjifamilydefault}{\gtdefault}
\renewcommand{\emph}[1]{{\upshape\bfseries #1}}
\usetheme{metropolis}
\metroset{block=fill}
\setbeamertemplate{navigation symbols}{}
\setbeamertemplate{blocks}[rounded][shadow=false]
\usecolortheme[rgb={0.7,0.2,0.2}]{structure}
%%%%%%%%%%%%%%%%%%%%%%%%%%
%% Change alert block colors
%%% 1- Block title (background and text)
\setbeamercolor{block title alerted}{fg=mDarkTeal, bg=mLightBrown!45!yellow!45}
\setbeamercolor{block title example}{fg=magenta!10!black, bg=mLightGreen!60}
%%% 2- Block body (background)
\setbeamercolor{block body alerted}{bg=mLightBrown!25}
\setbeamercolor{block body example}{bg=mLightGreen!15}
%%%%%%%%%%%%%%%%%%%%%%%%%%%
%%%%%%%%%%%%%%%%%%%%%%%%%%%
%% さまざまなアイコン
%%%%%%%%%%%%%%%%%%%%%%%%%%%
%\usepackage{fontawesome}
\usepackage{fontawesome5}
\usepackage{figchild}
\usepackage{twemojis}
\usepackage{utfsym}
\usepackage{bclogo}
\usepackage{marvosym}
\usepackage{fontmfizz}
\usepackage{pifont}
\usepackage{phaistos}
\usepackage{worldflags}
\usepackage{jigsaw}
\usepackage{tikzlings}
\usepackage{tikzducks}
\usepackage{scsnowman}
\usepackage{epsdice}
\usepackage{halloweenmath}
\usepackage{svrsymbols}
\usepackage{countriesofeurope}
\usepackage{tipa}
\usepackage{manfnt}
%%%%%%%%%%%%%%%%%%%%%%%%%%%
\usepackage{tikz}
\usetikzlibrary{calc,patterns,decorations.pathmorphing,backgrounds}
\usepackage{tcolorbox}
\usepackage{tikzpeople}
\usepackage{circledsteps}
\usepackage{xcolor}
\usepackage{amsmath}
\usepackage{booktabs}
\usepackage{chronology}
\usepackage{signchart}
%%%%%%%%%%%%%%%%%%%%%%%%%%%
%% 場合分け
%%%%%%%%%%%%%%%%%%%%%%%%%%%
\usepackage{cases}
%%%%%%%%%%%%%%%%%%%%%%%%%%
\usepackage{pdfpages}
%%%%%%%%%%%%%%%%%%%%%%%%%%%
%% 音声リンク表示
\newcommand{\myaudio}[1]{\href{#1}{\faVolumeUp}}
%%%%%%%%%%%%%%%%%%%%%%%%%%
%% \myAnch{<名前>}{<色>}{<テキスト>}
%% 指定のテキストを指定の色の四角枠で囲み, 指定の名前をもつTikZの
%% ノードとして出力する. 図には remember picture 属性を付けている
%% ので外部から参照可能である.
\newcommand*{\myAnch}[3]{%
  \tikz[remember picture,baseline=(#1.base)]
    \node[draw,rectangle,line width=1pt,#2] (#1) {\normalcolor #3};
}
%%%%%%%%%%%%%%%%%%%%%%%%%%
%% \myEmph コマンドの定義
%%%%%%%%%%%%%%%%%%%%%%%%%%
%\newcommand{\myEmph}[3]{%
%    \textbf<#1>{\color<#1>{#2}{#3}}%
%}
\usepackage{xparse} % xparseパッケージの読み込み
\NewDocumentCommand{\myEmph}{O{} m m}{%
    \def\argOne{#1}%
    \ifx\argOne\empty
        \textbf{\color{#2}{#3}}% オプション引数が省略された場合
    \else
        \textbf<#1>{\color<#1>{#2}{#3}}% オプション引数が指定された場合
    \fi
}
%%%%%%%%%%%%%%%%%%%%%%%%%%%
%%%%%%%%%%%%%%%%%%%%%%%%%%%
%% 文末の上昇イントネーション記号 \myRisingPitch
%% 通常のイントネーション \myDownwardPitch
%% https://note.com/dan_oyama/n/n8be58e8797b2
%%%%%%%%%%%%%%%%%%%%%%%%%%%
\newcommand{\myRisingPitch}{
\begin{tikzpicture}[scale=0.3,baseline=0.3]
\draw[->,>=stealth] (0,0) to[bend right=45] (1,1);
\end{tikzpicture}
}
\newcommand{\myDownwardPitch}{
\begin{tikzpicture}[scale=0.3,baseline=0.3]
\draw[->,>=stealth] (0,1) to[bend left=45] (1,0);
\end{tikzpicture}
}
%%%%%%%%%%%%%%%%%%%%%%%%%%%%
%\AtBeginSection[%
%]{%
%  \begin{frame}[plain]\frametitle{授業の流れ}
%     \tableofcontents[currentsection]
%   \end{frame}%
%}

\usepackage{pxrubrica}
\usepackage{lua-ul}
%%%%%%%%%%%%%%%%%%%%%%%%%%%
\title{English is fun.}
\subtitle{She drinks tea  every morning.}
\author{}
\institute[]{}
\date[]

%%%%%%%%%%%%%%%%%%%%%%%%%%%%
%% TEXT
%%%%%%%%%%%%%%%%%%%%%%%%%%%%
\begin{document}

\begin{frame}[plain]
  \titlepage
\end{frame}

\section*{授業の流れ}
\begin{frame}[plain]
  \frametitle{授業の流れ}
  \tableofcontents
\end{frame}

\section{目的語}
\subsection{目的語}
%%%%%%%%%%%%%%%%%%%
\begin{frame}[plain]\frametitle{目的語}

\begin{enumerate}\small
     \item \alt<2->{\Circled[outer color=Maroon]{Birds}}{Birds} \alt<3->{\Circled[outer color=NavyBlue]{sing}}{sing}. {\footnotesize 鳥は歌います。}\hfill\visible<27->{SV}
    \item \alt<4->{\Circled[outer color=Maroon]{Dogs}}{Dogs} \alt<5->{\Circled[outer color=NavyBlue]{swim}}{swim}. {\footnotesize 犬は泳ぎます。}\hfill\visible<28->{SV}
    \item \alt<6->{\Circled[outer color=Maroon]{You}}{You} \alt<7->{\Circled[outer color=NavyBlue]{have}}{have} \alt<19->{\fbox{a nice car}}{a nice car}. {\footnotesize  あなたはいい車\alt<20->{\kenten{を}}{を}持っている。}\hfill\visible<29->{SVO}
 \item \alt<8->{\Circled[outer color=Maroon]{I}}{I} \alt<9->{\Circled[outer color=NavyBlue]{play}}{play} \alt<21->{\fbox{the piano}}{the piano}. {\footnotesize わたしはピアノ\alt<21->{\kenten{を}}{を}弾きます。}\hfill\visible<30->{SVO}
    \item \alt<10->{\Circled[outer color=Maroon]{They}}{They} \alt<11->{\Circled[outer color=NavyBlue]{watch}}{watch} \alt<22->{\fbox{TV}}{TV}.  {\footnotesize 彼らはテレビ\alt<22->{\kenten{を}}{を}見ます。}\hfill\visible<31->{SVO}
    \item \alt<12->{\Circled[outer color=Maroon]{We}}{We} \alt<13->{\Circled[outer color=NavyBlue]{study}}{study} \alt<23->{\fbox{English}}{English}. {\footnotesize わたしたちは英語\alt<23->{\kenten{を}}{を}勉強します。}\hfill\visible<32->{SVO}
 \item \alt<14->{\Circled[outer color=Maroon]{I}}{I} \alt<15->{\Circled[outer color=NavyBlue]{write}}{write} \alt<24->{\fbox{a letter}}{a letter}. {\footnotesize わたしは手紙\alt<24->{\kenten{を}}{を}書きます。}\hfill\visible<33->{SVO}
     \item \alt<16->{\Circled[outer color=Maroon]{They}}{They} \alt<17->{\Circled[outer color=NavyBlue]{like}}{like} \alt<25->{\fbox{music}}{music}.  {\footnotesize 彼らは音楽が好きです。}\,\,\visible<25->{\dbend}\hfill\visible<34->{SVO}
\end{enumerate}

\vspace{-10pt}

\begin{columns}[t]
\begin{column}{.55\textwidth}
 \begin{alertblock}<18->{Topics for Today}\small
\begin{itemize}\setbeamertemplate{items}[square]
 \item<18-> 英文の骨格は\kenten{主語}と\kenten{述語動詞}でしたね
 \item<20-> 「~を」に当たる語を\kenten{目的語}といいます\\
\hfill{\scriptsize 目的語は必ずあるわけではありません}\\
%\hfill\visible<25->{{\scriptsize 「を」でないこともあります}}

\end{itemize}
 \end{alertblock}
%
\vspace*{-5pt}
{\tiny 0323}\,{\scriptsize \myaudio{./audio/004_verb_object_1.mp3}}

\end{column} 
\begin{column}{.44\textwidth}
 \begin{block}<26->{新しい記号を覚えましょう}\small
%   \textbullet\,\,新しい記号を覚えましょう
       \begin{description}
	\item[主語:] S\hfill{\underLine{s}ubjectの頭文字} 
	\item[動詞:] V\hfill{\underLine{v}erbの頭文字}
	\item[目的語:] O \hfill{\underLine{o}bjectの頭文字}
       \end{description}
      \end{block}
\end{column} 
\end{columns}

\end{frame}
%%%%%%%%%%%%%%%%%%%%%%%%%%%%%%
\begin{frame}<1-23>[plain,label=is_verb]\frametitle{Exercises}

{\small つぎの各文の\Circled[outer color=Maroon]{主語}と\Circled[outer color=NavyBlue]{述語動詞}を確認してください。目的語があるなら目的語を\,\fbox{  }\,で囲みましょう。さらに各文の構造をS, V, Oの記号で表してください}%

 % \setbeamercovered{transparent}
  \begin{enumerate}
   \item<1-> \alt<2->{\Circled[outer color=Maroon]{Birds}}{Birds} \alt<2->{\Circled[outer color=NavyBlue]{sing}}{sing}.\hfill\visible<14->{SV}
   \item<1-> \alt<3->{\Circled[outer color=Maroon]{Birds}}{Birds} \alt<3->{\Circled[outer color=NavyBlue]{sing}}{sing} \alt<22->{\Circled[fill color=gray, outer color=gray, inner color=white]{\,in the morning\,}}{\,in the morning\,}.\hfill\visible<15->{SV}
   \item<1-> \alt<4->{\Circled[outer color=Maroon]{He}}{He} \alt<4->{\Circled[outer color=NavyBlue]{swims}}{swims} \alt<22->{\Circled[fill color=gray, outer color=gray, inner color=white]{\,well\,}}{\,well\,}.\hfill\visible<16->{SV}
   \item<1-> \alt<5->{\Circled[outer color=Maroon]{They}}{They} \alt<5->{\Circled[outer color=NavyBlue]{swim}}{swim} \alt<22->{\Circled[fill color=gray, outer color=gray, inner color=white]{\,in the river\,}}{\,in the river\,}.\hfill\visible<17->{SV}
   \item<1-> \alt<6->{\Circled[outer color=Maroon]{George}}{George} \alt<6->{\Circled[outer color=NavyBlue]{speaks}}{speaks} \alt<10->{\fbox{English}}{English}.\hfill\visible<18->{SVO}
   \item<1-> \alt<7->{\Circled[outer color=Maroon]{Janis}}{Janis} \alt<7->{\Circled[outer color=NavyBlue]{speaks}}{speaks} \alt<11->{\fbox{French}}{French} \alt<22->{\Circled[fill color=gray, outer color=gray, inner color=white]{\,well\,}}{\,well\,}.\hfill\visible<19->{SVO}
   \item<1-> \alt<8->{\Circled[outer color=Maroon]{I}}{I} \alt<8->{\Circled[outer color=NavyBlue]{drink}}{drink} \alt<12->{\fbox{tea}}{tea} \alt<22->{\Circled[fill color=gray, outer color=gray, inner color=white]{\,every morning\,}}{\,every morning\,}.\hfill\visible<20->{SVO}
   \item<1-> \alt<9->{\Circled[outer color=Maroon]{Jimmy}}{Jimmy} \alt<9->{\Circled[outer color=NavyBlue]{plays}}{plays} \alt<13->{\fbox{the guitar}}{the guitar} \alt<22->{\Circled[fill color=gray, outer color=gray, inner color=white]{\,in his room\,}}{\,in his room\,}.\hfill\visible<21->{SVO}
  \end{enumerate}

{\tiny 0335}\,{\scriptsize \myaudio{./audio/004_verb_object_2.mp3}}
\hfill\visible<23->{%
\Circled[fill color = gray, outer color=gray, inner color=white]{  }\,の語句は、文に$+\alpha$の意味をつけくわえています
}
\end{frame}
%%%%%%%%%%%%%%%%%%%%%%%%%%%%%%%
\section{$+\alpha$の意味を追加しよう}
\subsection{副詞}
%%%%%%%%%%%%%%%%%%%%%%%%%%%%%%%%%%
\begin{frame}[plain]{副詞}
 \begin{enumerate}
  \item Dogs swim  \alt<2->{\Circled[fill color=gray, outer color=gray, inner color=white]{\bfseries well}}{well}.\hfill{\tiny 0311}\,{\scriptsize \myaudio{./audio/004_verb_object_3.mp3}}
  \item She speaks \alt<2->{\Circled[fill color=gray, outer color=gray, inner color=white]{\bfseries fast}}{fast}.\hfill{\scriptsize fast \textipa{/f\'\ae st/} 速く}
  \item He eats lunch \alt<2->{\Circled[fill color=gray, outer color=gray, inner color=white]{\bfseries slowly}}{slowly}.
  \item I finish my homework \alt<2->{\Circled[fill color=gray, outer color=gray, inner color=white]{\bfseries today}}{today}.
  \item I call my friend \alt<2->{\Circled[fill color=gray, outer color=gray, inner color=white]{\bfseries now}}{now}.
  \item She plays the piano \alt<2->{\Circled[fill color=gray, outer color=gray, inner color=white]{\bfseries here}}{here}.\hfill{\scriptsize here \textipa{/h\'I\textrhookschwa /} ここで}
  \item We play tennis \alt<2->{\Circled[fill color=gray, outer color=gray, inner color=white]{\bfseries there}}{there}. \hfill{\scriptsize there \textipa{/D\'e\textrhookschwa /} あそこで}
\end{enumerate}

\visible<3->{%
\begin{block}{Topics for Today}
\Circled[fill color = gray, outer color=gray, inner color=white]{\,\,副詞\,\,}\,は、文に$+\alpha$の意味をつけくわえます\\
\hfill{\scriptsize かならず必要なわけではありませんが、あると文意が鮮明になります}
\begin{description}
 \item[様態:] well, fast, slowly \ldots
 \item[とき:] now, today, yesterday, tomorrow \ldots
 \item[場所:] here, there \ldots\hfill{\scriptsize 副詞はいろいろあります。ちょっとずつ覚えましょう}
\end{description}
     \end{block}
}
\end{frame}
%%%%%%%%%%%%%%%%%%%%%%%%%%%%%%%
\subsection{$\text{前置詞}+\text{名詞}$}
\begin{frame}[plain]{\fbox{$\text{前置詞}+\text{名詞}$}}
\hfill{\tiny 0234}\,{\scriptsize \myaudio{./audio/004_verb_object_4.mp3}}
 \begin{enumerate}
  \item<1-> He plays tennis.\hfill{SVO}
  \item<2-> He plays tennis \Circled[fill color = gray, outer color=gray, inner color=white]{\,\,with his friends\,\,}.%
\hfill{\scriptsize with: ~といっしょに}
  \item<3-> He plays tennis \Circled[fill color = gray, outer color=gray, inner color=white]{\,\,in the park\,\,}.%
\hfill{\scriptsize in ~の中で}
  \item<4-> He plays tennis \Circled[fill color = gray, outer color=gray, inner color=white]{\,\,after school\,\,}.%
\hfill{\scriptsize after ~の後に \textipa{/\'\ae ft\textrhookschwa /}}
  \item<5-> He plays tennis \Circled[fill color = gray, outer color=gray, inner color=white]{\,\,with his friends\,\,} \Circled[fill color = gray, outer color=gray, inner color=white]{\,\,in the park\,\,} \Circled[fill color = gray, outer color=gray, inner color=white]{\,\,after school\,\,}.

 \end{enumerate}

\visible<6->{%
\begin{block}{Topics for Today}
with, in, after などを\kenten{前置詞}といいます%
\hfill{}「名詞の\kenten{前}に\kenten{置}く\kenten{詞}」
\begin{itemize}\setbeamertemplate{items}[square]\small
 \item<7-> 裏返していうと「前置詞の\kenten{後}にはかならず\kenten{名詞}がくる」ということです
 \item<8-> \Circled[fill color = gray, outer color=gray, inner color=white]{\,\,$\text{前置詞}+\text{名詞}$\,\,}\,は、文に$+\alpha$の意味をつけくわえます\\
 \hfill\visible<9->{{\scriptsize 前置詞はいろいろあります。少しずつ覚えましょう}}
\end{itemize}
     \end{block}
}
\end{frame}
%%%%%%%%%%%%%%%%%%%
\begin{frame}[plain]{Exercises}

{\small 日本語の意味になるよう(~~~~~)内の語句を並べ替えましょう。なお、1語不足しているので、解答作成に当たっては不足している1語を補ってください}
 \begin{enumerate}
  \item {\small 彼女は昼食のあと、コーヒーを飲む。}\hfill%
	She ( lunch / coffee / drinks ) .\\
	\visible<2->{She drinks coffee \myEmph[2-]{Maroon}{after} lunch.}
  \item {\small 彼は夕食の前に本を読む。}\hfill%
	He ( dinner / a book / reads ) .\\
	\visible<3->{He reads a book \myEmph[2-]{Maroon}{before} dinner.}
  \item {\small 放課後、彼らは川で泳ぐ。}\hfill%
	They ( after / the river / swim ) school.\\
	\visible<4->{They swim \myEmph[2-]{Maroon}{in} the river after school.}
  \item {\small わたしは両親といっしょにボストンに住んでいる。}\\
	\hfill{}I ( parents / live /  my ) in Boston.\\
	\visible<5->{I live \myEmph[2-]{Maroon}{with} my parents in Boston.}
%  \item わたしは朝食前に犬と散歩する。
%	(  my dog / walk  / with / I ) breakfast.\\
%	\visible<6->{I walk with my dog \myEmph[2-]{Maroon}{before} breakfast.} 
\end{enumerate}
\hfill{\tiny 0207}\,{\scriptsize \myaudio{./audio/004_verb_object_5.mp3}}
\end{frame}
%%%%%%%%%%%%%%%%%%%%%%%%%%%%%%%%
\section{まとめ}
\begin{frame}[plain]{学習したこと}
\begin{block}<1->{まとめ1}
\begin{itemize}\setbeamertemplate{items}[square]\small
 \item<1-> 英文の骨格は\kenten{主語}と\kenten{述語動詞}
 \item<2-> \kenten{目的語}(「~を」に当たる語)が必要なことがあります
 \item<3-> 新しい記号
        \begin{description}
	\item[主語:] S\hfill{\underLine{s}ubjectの頭文字} 
	\item[動詞:] V\hfill{\underLine{v}erbの頭文字}
	\item[目的語:] O \hfill{\underLine{o}bjectの頭文字}
       \end{description}
\end{itemize}
 \end{block}
%%%%%%%%%%%
\begin{block}<4->{まとめ2}\small
\visible<4->{文に$+\alpha$の意味をつけくわえる語句\hfill{\scriptsize かならず必要な要素ではないが、あると\kenten{文意が鮮明}になります}}
\begin{itemize}\setbeamertemplate{items}[square]\small
 \item<5-> \Circled[fill color = white]{\,\,副詞\,\,}%
\hspace{60pt}{\scriptsize 様態:}well, fast, slowly\hspace{15pt}{\scriptsize とき:}now, today\hspace{15pt}{\scriptsize 場所:}here, there \ldots
\item<6-> \Circled[fill color = white]{\,\,$\text{前置詞}+\text{名詞}$\,\,}%
\hspace{20pt}{in, on, with, before, after \ldots}\hfill{\scriptsize  (前置詞とは「名詞の\kenten{前}に\kenten{置}く\kenten{詞}」)}
\end{itemize}

\hfill\visible<7->{{\scriptsize 副詞や前置詞はいろいろあります。少しずつ覚えていきましょう}}
 \end{block}
\end{frame}
\end{document}
