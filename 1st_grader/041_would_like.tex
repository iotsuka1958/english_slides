\documentclass[aspectratio=169,xcolor={dvipsnames,table}]{beamer}
\usepackage[no-math,deluxe,haranoaji]{luatexja-preset}
\renewcommand{\kanjifamilydefault}{\gtdefault}
\renewcommand{\emph}[1]{{\upshape\bfseries #1}}
\usetheme{metropolis}
\metroset{block=fill}
\setbeamertemplate{navigation symbols}{}
\setbeamertemplate{blocks}[rounded][shadow=false]
\usecolortheme[rgb={0.7,0.2,0.2}]{structure}
%%%%%%%%%%%%%%%%%%%%%%%%%%
%% Change alert block colors
%%% 1- Block title (background and text)
\setbeamercolor{block title alerted}{fg=mDarkTeal, bg=mLightBrown!45!yellow!45}
\setbeamercolor{block title example}{fg=magenta!10!black, bg=mLightGreen!60}
%%% 2- Block body (background)
\setbeamercolor{block body alerted}{bg=mLightBrown!25}
\setbeamercolor{block body example}{bg=mLightGreen!15}
%%%%%%%%%%%%%%%%%%%%%%%%%%%
%%%%%%%%%%%%%%%%%%%%%%%%%%%
%% さまざまなアイコン
%%%%%%%%%%%%%%%%%%%%%%%%%%%
%\usepackage{fontawesome}
\usepackage{fontawesome5}
\usepackage{figchild}
\usepackage{twemojis}
\usepackage{utfsym}
\usepackage{bclogo}
\usepackage{marvosym}
\usepackage{fontmfizz}
\usepackage{pifont}
\usepackage{phaistos}
\usepackage{worldflags}
\usepackage{jigsaw}
\usepackage{tikzlings}
\usepackage{tikzducks}
\usepackage{scsnowman}
\usepackage{epsdice}
\usepackage{halloweenmath}
\usepackage{svrsymbols}
\usepackage{countriesofeurope}
\usepackage{tipa}
\usepackage{manfnt}
%%%%%%%%%%%%%%%%%%%%%%%%%%%
\usepackage{tikz}
\usetikzlibrary{calc,patterns,decorations.pathmorphing,backgrounds}
\usepackage{tcolorbox}
\usepackage{tikzpeople}
\usepackage{circledsteps}
\usepackage{xcolor}
\usepackage{amsmath}
\usepackage{booktabs}
\usepackage{chronology}
\usepackage{signchart}
%%%%%%%%%%%%%%%%%%%%%%%%%%%
%% 場合分け
%%%%%%%%%%%%%%%%%%%%%%%%%%%
\usepackage{cases}
%%%%%%%%%%%%%%%%%%%%%%%%%%
\usepackage{pdfpages}
%%%%%%%%%%%%%%%%%%%%%%%%%%%
%% 音声リンク表示
\newcommand{\myaudio}[1]{\href{#1}{\faVolumeUp}}
%%%%%%%%%%%%%%%%%%%%%%%%%%
%% \myAnch{<名前>}{<色>}{<テキスト>}
%% 指定のテキストを指定の色の四角枠で囲み, 指定の名前をもつTikZの
%% ノードとして出力する. 図には remember picture 属性を付けている
%% ので外部から参照可能である.
\newcommand*{\myAnch}[3]{%
  \tikz[remember picture,baseline=(#1.base)]
    \node[draw,rectangle,line width=1pt,#2] (#1) {\normalcolor #3};
}
%%%%%%%%%%%%%%%%%%%%%%%%%%
%% \myEmph コマンドの定義
%%%%%%%%%%%%%%%%%%%%%%%%%%
%\newcommand{\myEmph}[3]{%
%    \textbf<#1>{\color<#1>{#2}{#3}}%
%}
\usepackage{xparse} % xparseパッケージの読み込み
\NewDocumentCommand{\myEmph}{O{} m m}{%
    \def\argOne{#1}%
    \ifx\argOne\empty
        \textbf{\color{#2}{#3}}% オプション引数が省略された場合
    \else
        \textbf<#1>{\color<#1>{#2}{#3}}% オプション引数が指定された場合
    \fi
}
%%%%%%%%%%%%%%%%%%%%%%%%%%%
%%%%%%%%%%%%%%%%%%%%%%%%%%%
%% 文末の上昇イントネーション記号 \myRisingPitch
%% 通常のイントネーション \myDownwardPitch
%% https://note.com/dan_oyama/n/n8be58e8797b2
%%%%%%%%%%%%%%%%%%%%%%%%%%%
\newcommand{\myRisingPitch}{
\begin{tikzpicture}[scale=0.3,baseline=0.3]
\draw[->,>=stealth] (0,0) to[bend right=45] (1,1);
\end{tikzpicture}
}
\newcommand{\myDownwardPitch}{
\begin{tikzpicture}[scale=0.3,baseline=0.3]
\draw[->,>=stealth] (0,1) to[bend left=45] (1,0);
\end{tikzpicture}
}
%%%%%%%%%%%%%%%%%%%%%%%%%%%%
%\AtBeginSection[%
%]{%
%  \begin{frame}[plain]\frametitle{授業の流れ}
%     \tableofcontents[currentsection]
%   \end{frame}%
%}

\usepackage{pxrubrica}
%%%%%%%%%%%%%%%%%%%%%%%%%%%
\title{English is fun.}
\subtitle{お茶をいかがですか}
\author{}
\institute[]{}
\date[]

%%%%%%%%%%%%%%%%%%%%%%%%%%%%
%% TEXT
%%%%%%%%%%%%%%%%%%%%%%%%%%%%
\begin{document}

\begin{frame}[plain]
  \titlepage
\end{frame}

%\section*{授業の流れ}
%\begin{frame}[plain]
%  \frametitle{授業の流れ}
%  \tableofcontents
%\end{frame}

%%%%%%%%%%%%%%%%%%%%%%%%%%%%%%%%%%%%%%
\begin{frame}[plain]{ほしい}
\large
 \begin{enumerate}
  \item I \textbf{want} some tea.
  \item I \textbf{would like} some tea.
  \item I{\bfseries 'd like} some tea.
 \end{enumerate}

\begin{block}{Topics for Today}\small
\textbf{would like}は\textbf{want}のていねいな表現です
\begin{itemize}\setbeamertemplate{items}[square]\small
 \item I \textbf{want} $+$ 名詞
 \item I \textbf{would like} $+$ 名詞
 \item I\textbf{'d like} $+$ 名詞
       \end{itemize}
\end{block}

\hfill{\scriptsize \myaudio{./audio/041_would_like_01.mp3}}
\end{frame}
%%%%%%%%%%%%%%%%%%%%%%%%%%%%%%%%%%%%
\begin{frame}[plain]{Exercises}
次の英文をI would likeまたはI'd likeをもちいて、 ていねいな表現にしてみましょう
 \begin{enumerate}
  \item I want some water.\\
\visible<2->{I would like some water.}\\
\visible<3->{I'd like some water.}
  \item I want a sandwich.\\
\visible<4->{I would like a sandwich.}\\
\visible<5->{I'd like a sandwich.}
  \item I want some flowers for my birthday.\\
\visible<6->{I would like some flowers for my birthday.}\\
\visible<7->{I'd like some flowers for my birthday.}
 \end{enumerate}

\hfill{\scriptsize \myaudio{./audio/041_would_like_02.mp3}}

\end{frame}
%%%%%%%%%%%%%%%%%%%%%%%%%%%%%%%%%%%%
%%%%%%%%%%%%%%%%%%%%%%%%%%%%%%%%%%%%%%
\begin{frame}[plain]{~したい}
\large
 \begin{enumerate}
  \item I \textbf{want to} eat a hamburger.
  \item I \textbf{would like to} eat hamburger.
  \item I{\bfseries 'd like to} eat hamburger.
 \end{enumerate}

\begin{block}{Topics for Today}\small
\textbf{would like to} $+$ 原形は\textbf{want to} $+$原形のていねいな表現です
\begin{itemize}\setbeamertemplate{items}[square]\small
 \item I \textbf{want to} $+$ 原形
 \item I \textbf{would like to} $+$ 原形
 \item I\textbf{'d like to} $+$ 原形
       \end{itemize}
\end{block}

\hfill{\scriptsize \myaudio{./audio/041_would_like_03.mp3}}

\end{frame}
%%%%%%%%%%%%%%%%%%%%%%%%%%%%%%%%%%%%
\begin{frame}[plain]{Exercises}
次の英文をI would likeまたはI'd likeをもちいて、 ていねいな表現にしてみましょう
 \begin{enumerate}
  \item I want to visit London.\\
\visible<2->{I would like to visit London.}\\
\visible<3->{I'd like to visit London.}
  \item I want to read a book.\\
\visible<4->{I would like to read a book.}\\
\visible<5->{I'd like to read a book.}
  \item I want to play tennis.\\
\visible<6->{I would like to play tennis.}\\
\visible<7->{I'd like to play tennis.}
 \end{enumerate}
\hfill{\scriptsize \myaudio{./audio/041_would_like_04.mp3}}

\end{frame}
%%%%%%%%%%%%%%%%%%%%%%%%%%%%%%%%%%%%
\begin{frame}[plain]{~がほしいですか、~はいかがですか}
\large
 \begin{enumerate}
  \item Do you \textbf{want} some coffee?
  \item \textbf{Would} you \textbf{like} some coffee? 
 \end{enumerate}

\begin{block}{Topics for Today}\small
\textbf{Would you like \ldots ?}は\textbf{Do you want \ldots ?}のていねいな表現です
\begin{itemize}\setbeamertemplate{items}[square]\small
 \item \textbf{Would you like} $+$ 名詞 \ldots ?
       \end{itemize}
\end{block}
\hfill{\scriptsize \myaudio{./audio/041_would_like_05.mp3}}

\end{frame}
%%%%%%%%%%%%%%%%%%%%%%%%%%%%
\begin{frame}[plain]{Exercises}
次の英文を would you likeをもちいて、 ていねいな表現にしてみましょう
 \begin{enumerate}
  \item Do you want a pizza?\\
\visible<2->{Would you like a pizza?}\\
  \item Do you want some juice?\\
\visible<3->{Would you like some juice?}\\
  \item What do you want for dinner?\\
\visible<4->{What would you like for dinner?}\\
 \end{enumerate}

\hfill{\scriptsize \myaudio{./audio/041_would_like_06.mp3}}

\end{frame}
%%%%%%%%%%%%%%%%%%%%%%%%%%%%%%%%%%%%
%%%%%%%%%%%%%%%%%%%%%%%%%%%%%%%%%%%%%%
\begin{frame}[plain]{~したいですか?}
\large
 \begin{enumerate}
  \item \textbf{Do} you \textbf{want to} watch amovie?
  \item \textbf{Would} you \textbf{like to} watch a movie?
 \end{enumerate}

\begin{block}{Topics for Today}\small
\textbf{Would you like to} $+$ 原形 \ldots ?は\textbf{Do you want to} $+$原形\ldots ?のていねいな表現です
\begin{itemize}\setbeamertemplate{items}[square]\small
 \item \textbf{Do} you \textbf{want to} $+$ 原形 \ldots ?
 \item \textbf{Would} you \textbf{like to} $+$ 原形 \ldots ?
       \end{itemize}
\end{block}
\hfill{\scriptsize \myaudio{./audio/041_would_like_07.mp3}}

\end{frame}
%%%%%%%%%%%%%%%%%%%%%%%%%%%%%%%%%%%%
\begin{frame}[plain]{Exercises}
日本語の意味になるよう(~~~~~~)内の語句を並べ替えてください。先頭に来る語は大文字で始めてましょう。なお[$+1$]とある場合は1語補ってください

 \begin{enumerate}
  \item わたしたちといっしょに動物園に行きたいですか。(行きませんか)\\
( go / like / us / you / to / to / with / would / the zoo ) ?\\
\visible<2->{Would you like to go to the zoo with us?}
  \item あなたはなにをお飲みになりますか。\\
( like / drink / you / to / would / what ) ?\\
\visible<3->{What would you like to drink?}
  \item パーティーにきたいですか。(きませんか)\\
( you / the party / come / like / to / would ) ? [$+1$]\\
\visible<4->{Would you like to come to the party?}
\end{enumerate}
\hfill{\scriptsize \myaudio{./audio/041_would_like_08.mp3}}

\end{frame}
%%%%%%%%%%%%%%%%%%%%%%%%%%%%%%%%%%
\begin{frame}[plain]{まとめ}
 \begin{block}{Topics for Today}\small
\textbf{would like}は\textbf{want}のていねいな表現です
\begin{itemize}\setbeamertemplate{items}[square]\small
 \item I \textbf{want} $+$ 名詞
 \item I \textbf{would like} $+$ 名詞
 \item I\textbf{'d like} $+$ 名詞
       \end{itemize}
\end{block}

\begin{block}{Topics for Today}\small
\textbf{would like to} $+$ 原形は\textbf{want to} $+$原形のていねいな表現です
\begin{itemize}\setbeamertemplate{items}[square]\small
 \item I \textbf{want to} $+$ 原形
 \item I \textbf{would like to} $+$ 原形
 \item I\textbf{'d like to} $+$ 原形
       \end{itemize}
\end{block}
\end{frame}
%%%%%%%%%%%%%%%%%%%%%%%%%%%
%%%%%%%%%%%%%%%%%%%%%%%%%%%%%%%%%%
\begin{frame}[plain]{まとめ}
\begin{block}{Topics for Today}\small
\textbf{Would you like \ldots ?}は\textbf{Do you want \ldots ?}のていねいな表現です
\begin{itemize}\setbeamertemplate{items}[square]\small
 \item \textbf{Would you like} $+$ 名詞 \ldots ?
       \end{itemize}
\end{block}

\begin{block}{Topics for Today}\small
\textbf{Would you like to} $+$ 原形 \ldots ?は\textbf{Do you want to} $+$原形\ldots ?のていねいな表現です
\begin{itemize}\setbeamertemplate{items}[square]\small
 \item \textbf{Do} you \textbf{want to} $+$ 原形 \ldots ?
 \item \textbf{Would} you \textbf{like to} $+$ 原形 \ldots ?
       \end{itemize}
\end{block}
\end{frame}
\end{document}
