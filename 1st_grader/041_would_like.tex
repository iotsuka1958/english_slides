\documentclass[aspectratio=169,xcolor={dvipsnames,table}]{beamer}
\usepackage[no-math,deluxe,haranoaji]{luatexja-preset}
\renewcommand{\kanjifamilydefault}{\gtdefault}
\renewcommand{\emph}[1]{{\upshape\bfseries #1}}
\usetheme{metropolis}
\metroset{block=fill}
\setbeamertemplate{navigation symbols}{}
\setbeamertemplate{blocks}[rounded][shadow=false]
\usecolortheme[rgb={0.7,0.2,0.2}]{structure}
%%%%%%%%%%%%%%%%%%%%%%%%%%
%% Change alert block colors
%%% 1- Block title (background and text)
\setbeamercolor{block title alerted}{fg=mDarkTeal, bg=mLightBrown!45!yellow!45}
\setbeamercolor{block title example}{fg=magenta!10!black, bg=mLightGreen!60}
%%% 2- Block body (background)
\setbeamercolor{block body alerted}{bg=mLightBrown!25}
\setbeamercolor{block body example}{bg=mLightGreen!15}
%%%%%%%%%%%%%%%%%%%%%%%%%%%
%%%%%%%%%%%%%%%%%%%%%%%%%%%
%% さまざまなアイコン
%%%%%%%%%%%%%%%%%%%%%%%%%%%
%\usepackage{fontawesome}
\usepackage{fontawesome5}
\usepackage{figchild}
\usepackage{twemojis}
\usepackage{utfsym}
\usepackage{bclogo}
\usepackage{marvosym}
\usepackage{fontmfizz}
\usepackage{pifont}
\usepackage{phaistos}
\usepackage{worldflags}
\usepackage{jigsaw}
\usepackage{tikzlings}
\usepackage{tikzducks}
\usepackage{scsnowman}
\usepackage{epsdice}
\usepackage{halloweenmath}
\usepackage{svrsymbols}
\usepackage{countriesofeurope}
\usepackage{tipa}
%%%%%%%%%%%%%%%%%%%%%%%%%%%
\usepackage{tikz}
\usetikzlibrary{calc,patterns,decorations.pathmorphing,backgrounds}
\usepackage{tcolorbox}
\usepackage{tikzpeople}
\usepackage{circledsteps}
\usepackage{xcolor}
\usepackage{amsmath}
\usepackage{booktabs}
\usepackage{chronology}
\usepackage{signchart}
%%%%%%%%%%%%%%%%%%%%%%%%%%%
%% 場合分け
%%%%%%%%%%%%%%%%%%%%%%%%%%%
\usepackage{cases}
%%%%%%%%%%%%%%%%%%%%%%%%%%
\usepackage{pdfpages}
%%%%%%%%%%%%%%%%%%%%%%%%%%%
%% 音声リンク表示
\newcommand{\myaudio}[1]{\href{#1}{\faVolumeUp}}
%%%%%%%%%%%%%%%%%%%%%%%%%%
%% \myAnch{<名前>}{<色>}{<テキスト>}
%% 指定のテキストを指定の色の四角枠で囲み, 指定の名前をもつTikZの
%% ノードとして出力する. 図には remember picture 属性を付けている
%% ので外部から参照可能である.
\newcommand*{\myAnch}[3]{%
  \tikz[remember picture,baseline=(#1.base)]
    \node[draw,rectangle,line width=1pt,#2] (#1) {\normalcolor #3};
}
%%%%%%%%%%%%%%%%%%%%%%%%%%
%% \myEmph コマンドの定義
%%%%%%%%%%%%%%%%%%%%%%%%%%
%\newcommand{\myEmph}[3]{%
%    \textbf<#1>{\color<#1>{#2}{#3}}%
%}
\usepackage{xparse} % xparseパッケージの読み込み
\NewDocumentCommand{\myEmph}{O{} m m}{%
    \def\argOne{#1}%
    \ifx\argOne\empty
        \textbf{\color{#2}{#3}}% オプション引数が省略された場合
    \else
        \textbf<#1>{\color<#1>{#2}{#3}}% オプション引数が指定された場合
    \fi
}
%%%%%%%%%%%%%%%%%%%%%%%%%%%
%%%%%%%%%%%%%%%%%%%%%%%%%%%
%% 文末の上昇イントネーション記号 \myRisingPitch
%% 通常のイントネーション \myDownwardPitch
%% https://note.com/dan_oyama/n/n8be58e8797b2
%%%%%%%%%%%%%%%%%%%%%%%%%%%
\newcommand{\myRisingPitch}{
\begin{tikzpicture}[scale=0.3,baseline=0.3]
\draw[->,>=stealth] (0,0) to[bend right=45] (1,1);
\end{tikzpicture}
}
\newcommand{\myDownwardPitch}{
\begin{tikzpicture}[scale=0.3,baseline=0.3]
\draw[->,>=stealth] (0,1) to[bend left=45] (1,0);
\end{tikzpicture}
}
%%%%%%%%%%%%%%%%%%%%%%%%%%%%
%\AtBeginSection[%
%]{%
%  \begin{frame}[plain]\frametitle{授業の流れ}
%     \tableofcontents[currentsection]
%   \end{frame}%
%}

\usepackage{pxrubrica}
%%%%%%%%%%%%%%%%%%%%%%%%%%%
%%% リエゾン
\newcommand{\liaison}[1][black]{% 引数1を受け取る。デフォルト値はblack
  \tikz[baseline=0pt] \draw[draw=#1,line width=.6pt] (0,0) to[bend right=20] (0.2,0);%
}
%%%%%%%%%%%%%%%%%%%%%%%%%%%
\title{English is fun.}
\subtitle{お茶をいかがですか}
\author{}
\institute[]{}
\date[]

%%%%%%%%%%%%%%%%%%%%%%%%%%%%
%% TEXT
%%%%%%%%%%%%%%%%%%%%%%%%%%%%
\begin{document}

\begin{frame}[plain]
  \titlepage
\end{frame}

%\section*{授業の流れ}
%\begin{frame}[plain]
%  \frametitle{授業の流れ}
%  \tableofcontents
%\end{frame}

%%%%%%%%%%%%%%%%%%%%%%%%%%%%%%%%%%%%%%
\section{want / would like}
%%%%%%%%%%%%%%%%%%%%%%%%%%%%%%%%%%%%%%
\begin{frame}[plain]{~がほしい}
\large
 \begin{enumerate}
  \item<1-> \textbf{I want} some tea.\hfill{\scriptsize \textipa{/w\'Ant/}}
  \item<2-> \textbf{I would like} some tea.\hfill{\scriptsize \textipa{/w@d laIk/}}
  \item<3-> {\bfseries I'd like} some tea.
 \end{enumerate}

\begin{block}<4->{Topics for Today}\small
\textbf{I would like} $+$ 名詞\,\ldots\,\,.\,\,は\textbf{I want} $+$ 名詞\,\ldots\,\,.\,\,のていねいな表現です
\begin{itemize}\setbeamertemplate{items}[square]\small
 \item \textbf{I want} $+$ 名詞\,\ldots\,\,.
 \item \textbf{I would like} $+$ 名詞\,\ldots\,\,.
 \item \textbf{I'd like} $+$ 名詞\,\ldots\,\,.
       \end{itemize}
\end{block}

\hfill{\tiny 0133}\,{\scriptsize \myaudio{./audio/041_would_like_01.mp3}}
\end{frame}
%%%%%%%%%%%%%%%%%%%%%%%%%%%%%%%%%%%%
\begin{frame}[plain]{Exercises}
次の英文をI would likeまたはI'd likeをもちいて、 ていねいな表現にしてみましょう
 \begin{enumerate}
  \item I want some water.\\
\visible<2->{$\longrightarrow$ \textbf{I would like} some water.}\\
\visible<3->{$\longrightarrow$ \textbf{I'd like} some water.}
  \item I want a sandwich.\hfill{\scriptsize sandwich \textipa{/s\'\ae n(d)wItS/}}\\
\visible<4->{$\longrightarrow$ \textbf{I would like} a sandwich.}\\
\visible<5->{$\longrightarrow$ \textbf{I'd like} a sandwich.}
  \item I want some flowers for my birthday.\\
\visible<6->{$\longrightarrow$ \textbf{I would like} some flowers for my birthday.}\\
\visible<7->{$\longrightarrow$ \textbf{I'd like} some flowers for my birthday.}
 \end{enumerate}

\hfill{\tiny 0332}\,{\scriptsize \myaudio{./audio/041_would_like_02.mp3}}

\end{frame}
%%%%%%%%%%%%%%%%%%%%%%%%%%%%%%%%%%%%
\section{Do you want \ldots ? / Would you like \ldots ?}
%%%%%%%%%%%%%%%%%%%%%%%%%%%%%%%%%%%%
\begin{frame}[plain]{~がほしいですか、~はいかがですか}
\large
 \begin{enumerate}
  \item<1-> \textbf{Do you want} some coffee?\hfill{\scriptsize cf. I want some coffee.}
  \item<2-> \textbf{Would you like} some coffee? \hfill{\scriptsize cf. I would like some coffee.}
 \end{enumerate}

\bigskip

\begin{block}<3->{Topics for Today}\small
\textbf{Would you like \ldots\,?}\,\,は\textbf{Do you want \ldots\,?}\,\,のていねいな表現です
\begin{itemize}\setbeamertemplate{items}[square]\small
 \item \textbf{Do you want} $+$ 名詞 \ldots\,?
 \item \textbf{Would you like} $+$ 名詞 \ldots\,?
       \end{itemize}
\end{block}
\hfill{\tiny 0110}\,{\scriptsize \myaudio{./audio/041_would_like_05.mp3}}

\end{frame}
%%%%%%%%%%%%%%%%%%%%%%%%%%%%
\begin{frame}[plain]{Exercises}
次の英文を would you likeをもちいて、 ていねいな表現にしてみましょう
 \begin{enumerate}
  \item Do you want a pizza?\\
\visible<2->{$\longrightarrow$ \textbf{Would you like} a pizza?}\\
  \item Do you want some juice?\\
\visible<3->{$\longrightarrow$ \textbf{Would you like} some juice?}\\
  \item What do you want for dinner?\\
\visible<4->{$\longrightarrow$ What \textbf{would you like} for dinner?}\\
 \end{enumerate}

\hfill{\tiny 0222}\,{\scriptsize \myaudio{./audio/041_would_like_06.mp3}}

\end{frame}
%%%%%%%%%%%%%%%%%%%%%%%%%%%%%%%%%%%%
\section{want to --- / would like to ---}
%%%%%%%%%%%%%%%%%%%%%%%%%%%%%%%%%%%%%%
%%%%%%%%%%%%%%%%%%%%%%%%%%%%%
\begin{frame}[plain]{want to ---}
 \begin{enumerate}
  \item[0.]<1-> I \textbf{want} \fbox{\,\,\,\,\,\,\,\,\,\,{$x$}\,\,\,\,\,\,\,\,\,\,}\,\,.\hfill{\scriptsize $\fbox{\,\,\,\,X\,\,\,\,} = \text{O}$}
  \item<2-> I \textbf{want} some cookies.\hfill{\scriptsize $\fbox{\,\,\,\,X\,\,\,\,} = \text{some cookies} = \text{O}$}
  \item<3-> I \textbf{want to} make some cookies.\hfill{\scriptsize $\fbox{\,\,\,\,X\,\,\,\,} = \text{to make some cookies} = \text{O}$}
 \end{enumerate}

\vfill

\begin{block}<4->{}
\begin{itemize}\setbeamertemplate{items}[square]\small
 \item wantには名詞が続く場合と\,\,\Circled[fill color = white]{\,\,to$+$原形\,\,}\,\,が続く場合があります
 \item want to ---「~すること」を欲する $\longrightarrow$ ~したい
\end{itemize}
\end{block}

\hfill{\tiny 0222}\,{\scriptsize \myaudio{./audio/041_would_like_1a.mp3}}

\end{frame}
%%%%%%%%%%%%%%%%%%%%%%%%%%%%%
%%%%%%%%%%%%%%%%%%%%%%%%%%%%%
\begin{frame}[plain]{would like to ---}
 \begin{enumerate}
  \item[0.]<1-> I \textbf{would like} \fbox{\,\,\,\,\,\,\,\,\,\,{$x$}\,\,\,\,\,\,\,\,\,\,}\,\,.\hfill{\scriptsize $\fbox{\,\,\,\,X\,\,\,\,} = \text{O}$}
  \item<2-> I \textbf{would like} some milk.\hfill{\scriptsize $\fbox{\,\,\,\,X\,\,\,\,} = \text{some mlk} = \text{O}$}
  \item<3-> I \textbf{would like to} drink some milk.\hfill{\scriptsize $\fbox{\,\,\,\,X\,\,\,\,} = \text{to drink some milk} = \text{O}$}
  \item<4-> I\textbf{'d like to} drink some milk.\hfill{\scriptsize $\fbox{\,\,\,\,X\,\,\,\,} = \text{to drink some milk} = \text{O}$}
 \end{enumerate}

\vfill

\begin{block}<5->{}
\begin{itemize}\setbeamertemplate{items}[square]\small
 \item would likeにも名詞が続く場合と\,\,\Circled[fill color = white]{\,\,to$+$原形\,\,}\,\,が続く場合があります
 \item would like to --- はwant to --- よりもていねいなな表現
\end{itemize}
\end{block}
\hfill{\tiny 0222}\,{\scriptsize \myaudio{./audio/041_would_like_1b.mp3}}

\end{frame}
%%%%%%%%%%%%%%%%%%%%%%%%%%%%%
\begin{frame}[plain]{~したい}
\large
 \begin{enumerate}
  \item<1-> \textbf{I want to} eat a hamburger.\hfill{\scriptsize hamburger \textipa{/h\'\ae mb\textrhookschwa :g\textrhookschwa/}}
  \item<2-> \textbf{I would like to} eat a hamburger.\hfill{\scriptsize \textipa{/w@d laIk t@/}}
  \item<3-> {\bfseries I'd like to} eat a hamburger.
 \end{enumerate}

\begin{block}<4->{Topics for Today}\small
\textbf{I want to} ---\,\,は「~したい」という意味

\textbf{I would like to}  ---\,\ldots\,\,.\,\,は\textbf{I want to} ---\,\ldots\,\,.\,\,のていねいな表現
\begin{itemize}\setbeamertemplate{items}[square]\small
 \item \textbf{I want to}  ---\,\ldots\,\,.
 \item \textbf{I would like to}  ---\,\ldots\,\,.
 \item \textbf{I'd like to}  ---\,\ldots\,\,.
       \end{itemize}
\end{block}

\hfill{\tiny 0137}\,{\scriptsize \myaudio{./audio/041_would_like_03.mp3}}

\end{frame}
%%%%%%%%%%%%%%%%%%%%%%%%%%%%%%%%%%%%
\begin{frame}[plain]{Exercises}
次の英文をI would likeまたはI'd likeをもちいて、 ていねいな表現にしてみましょう
 \begin{enumerate}
  \item I want to visit London.\hfill{\scriptsize London \textipa{/l\'\textturnv nd@n/}}\\
\visible<2->{$\longrightarrow$ \textbf{I would like to} visit London.}\\
\visible<3->{$\longrightarrow$ \textbf{I'd like to} visit London.}
  \item I want to read a book.\\
\visible<4->{$\longrightarrow$ \textbf{I would like to} read a book.}\\
\visible<5->{$\longrightarrow$ \textbf{I'd like to} read a book.}
  \item I want to play tennis.\\
\visible<6->{$\longrightarrow$ \textbf{I would like to} play tennis.}\\
\visible<7->{$\longrightarrow$ \textbf{I'd like to} play tennis.}
 \end{enumerate}
\hfill{\tiny 0325}\,{\scriptsize \myaudio{./audio/041_would_like_04.mp3}}

\end{frame}
%%%%%%%%%%%%%%%%%%%%%%%%%%%%%%%%%%%%%%
\begin{frame}[plain]{~したいですか、~しませんか}
\large
 \begin{enumerate}
  \item \textbf{Do you want to} watch a movie?
  \item \textbf{Would you like to} watch a movie?
 \end{enumerate}

\begin{block}{Topics for Today}\small
\textbf{Would you like to} $+$ 原形\,\,\ldots\,?\,\,は\,\,\textbf{Do you want to} $+$原形\,\,\ldots\,?\,\,のていねいな表現です
\begin{itemize}\setbeamertemplate{items}[square]\small
 \item \textbf{Do you want to} $+$ 原形 \ldots\,?
 \item \textbf{Would you like to} $+$ 原形 \ldots\,?
       \end{itemize}
\end{block}
\hfill{\tiny 0111}\,{\scriptsize \myaudio{./audio/041_would_like_07.mp3}}

\end{frame}
%%%%%%%%%%%%%%%%%%%%%%%%%%%%%%%%%%%%
\begin{frame}[plain]{Exercises}
日本語の意味になるよう(~~~~~~)内の語句を並べ替えてください。先頭に来る語は大文字で始めてましょう。なお[$+1$]とある場合は1語補ってください

 \begin{enumerate}
  \item わたしたちといっしょに動物園に行きたいですか。(行きませんか)\\
( go / like / you / to / to / with / would / the zoo ) us?\\
\visible<2->{\textbf{Would you like to} go to the zoo with us?}
  \item あなたはなにをお飲みになりますか。\\
( like / drink / you / to / would / what ) ?\\
\visible<3->{What \textbf{would you like to} drink?}
  \item パーティーにきたいですか。(きませんか)\\
( you / the party / come / like / to / would ) ? [$+1$]\\
\visible<4->{\textbf{Would you like to}} \visible<5->{come to the party?}
\end{enumerate}
\hfill{\tiny 0136}\,{\scriptsize \myaudio{./audio/041_would_like_08.mp3}}

\end{frame}
%%%%%%%%%%%%%%%%%%%%%%%%%%%%%%%%%%
%%%%%%%%%%%%%%%%%%%%%%
\section{聞いてみよう、読んでみよう{\tiny 0044}\,{\scriptsize \myaudio{./audio/041_would_like_reading.mp3}}}
%%%%%%%%%%%%%%%%%%%%%%
\begin{frame}[plain]
 
\includegraphics[width=1.01\textwidth]{./images/nanobanana-output/041_would_like_reading.png}

\vspace{-15pt}

\hfill{\tiny 0044}\,{\scriptsize \myaudio{./audio/041_would_like_reading.mp3}}

\end{frame}
%%%%%%%%%%%%%%%%%%%
%%%%%%%%%%%%%%%%%%%%%%%%%%%
\begin{frame}[plain]{Exercises}
 \begin{tcolorbox}[colframe=NavyBlue!20,
  colback=NavyBlue!10!white,
  colbacktitle=NavyBlue!40!white,
  coltitle=black, fonttitle=\bfseries,
  title=A Day at the Park\mbox{}\hfill{\tiny 0044}\,{\scriptsize \myaudio{./audio/041_would_like_reading.mp3}},
before upper={\setlength{\parindent}{1.25em}}
]
I would like to go to the park this Saturday. The weather will be nice. I can see many people there. Some are walking their dogs. Some are playing football. There are children on the swings. I would like to play tennis with my friends. After that, we will eat ice cream. I hope it will be a wonderful day. Do you want to go to the park with me?
\end{tcolorbox}
\visible<2->{\small 次の各文が本文の内容とあっていればT,そうでなければFと答えましょう}
\vspace{-5pt}
\begin{enumerate}\setlength{\itemsep}{-2pt}
 \item<2-> The person wants to go to the park on Saturday.\hfill\visible<3->{T}
 \item<2-> The person plans to play football at the park.\hfill\visible<4->{F}
 \item<2-> The person will eat pizza after playing.\hfill\visible<5->{F}
 \item<2-> There are children playing on the swings.\hfill\visible<6->{T}
 \item<2-> The person invites you to go to the park.\hfill\visible<7->{T}

\end{enumerate}

\end{frame}
%%%%%%%%%%%%%%%%%%%%%%%%%%%%%%%%%%
\begin{frame}[plain]{まとめ}
 \begin{block}{~がほしい}\small
\textbf{I would like} $+$ 名詞\,\ldots\,\,.\,\,は\,\,\textbf{I want} $+$ 名詞\,\ldots\,\,.\,\,のていねいな表現です
\begin{itemize}\setbeamertemplate{items}[square]\small
 \item \textbf{I want} $+$ 名詞\,\ldots\,\,.\hfill{\scriptsize \textbf{I want} some tea.}
 \item \textbf{I would like} $+$ 名詞\,\ldots\,\,.\hfill{\scriptsize \textbf{I would like} some tea.}
 \item \textbf{I'd like} $+$ 名詞\,\ldots\,\,.\hfill{\scriptsize \textbf{I'd like} some tea.}
       \end{itemize}
\end{block}

\begin{block}{~したい}\small
\textbf{I would like to} $+$ 原形\,\ldots\,\,.\,\,は\,\,\textbf{I want to} $+$原形\,\ldots\,\,.\,\,のていねいな表現です
\begin{itemize}\setbeamertemplate{items}[square]\small
 \item \textbf{I want to} $+$ 原形\,\ldots\,\,.\hfill{\scriptsize \textbf{I want to} have some tea.}
 \item \textbf{I would like to} $+$ 原形\,\ldots\,\,.\hfill{\scriptsize \textbf{I would like to} have some tea.}
 \item \textbf{I'd like to} $+$ 原形\,\ldots\,\,.\hfill{\scriptsize \textbf{I'd like to} have some tea.}
       \end{itemize}
\end{block}
\end{frame}
%%%%%%%%%%%%%%%%%%%%%%%%%%%%%%
\section{まとめ}
%%%%%%%%%%%%%%%%%%%%%%%%%%%%%
%%%%%%%%%%%%%%%%%%%%%%%%%%%%%%%%%%
\begin{frame}[plain]{まとめ}
\begin{block}{~がほしいですか、~はいかがですか}\small
\textbf{Would you like} $+$ 名詞\,\ldots\,\,?\,\,は\,\,\textbf{Do you want} $+$ 名詞\,\ldots\,\,?\,\,のていねいな表現です
\begin{itemize}\setbeamertemplate{items}[square]\small
 \item \textbf{Do you want} $+$ 名詞 \ldots\,?\hfill{\scriptsize \textbf{Do you want} some tea?}
 \item \textbf{Would you like} $+$ 名詞 \ldots\,?\hfill{\scriptsize \textbf{Would you like} some tea?}
       \end{itemize}
\end{block}

\begin{block}{~したいですか、~しませんか}\small
\textbf{Would you like to} $+$ 原形\,\ldots\,?\,\,は\,\,\textbf{Do you want to} $+$原形\,\ldots\,?\,\,のていねいな表現です
\begin{itemize}\setbeamertemplate{items}[square]\small
 \item \textbf{Do you want to} $+$ 原形\,\ldots\,?\hfill{\scriptsize \textbf{Do you want to} eat lunch together?}
 \item \textbf{Would you like to} $+$ 原形\,\ldots\,?\hfill{\scriptsize \textbf{Would you like to} eat lunch together?}
       \end{itemize}
\end{block}
\end{frame}
%%%%%%%%%%%%%%%%%%%%%%%%%%%%%
\section{want toの発音について}
%%%%%%%%%%%%%%%%%%%%%%%%%%%%%
\begin{frame}[plain]{want toの発音}
 \begin{enumerate}
  \item I \textbf{want to} see you.
  \item I \textbf{want to} play soccer.
  \item I \textbf{want to} go to the park.
  \item I \textbf{want to} read this book.
  \item I \textbf{want to} drink some water.
 \end{enumerate}

 \begin{block}{want to}\small
%want\tikz[baseline] \draw[line width=.7pt, color=Maroon] (0,0) .. controls (0.1,-0.08) and (0.3,-0.08) .. (0.4,0);to


want\liaison[Maroon]to

  \begin{itemize}\setbeamertemplate{items}[square]\small
   \item \textipa{/wAn@/}と発音されることがあります
  \end{itemize}
 \end{block}


\hfill{\tiny 0222}\,{\scriptsize \myaudio{./audio/041_would_like_09.mp3}}

\end{frame}
%%%%%%%%%%%%%%%%%%%%%%%%%%%%%%
%%%%%%%%%%%%%%%%%%%%%%%%%%%%%%%
\begin{frame}[plain]
 
\includegraphics[width=1.01\textwidth]{./infographic/041_would_like_infographic.png}
\end{frame}
%%%%%%%%%%%%%%%%%%%
%%%%%%%%%%%%%%%%%%%%%%%%%%%%%%
\begin{frame}
 
\hfill{\tiny 1343}\,{\scriptsize \myaudio{./audio/overview_041_would_like_audio_overview.mp3}}

\end{frame}

\end{document}
