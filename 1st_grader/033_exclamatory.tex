\documentclass[aspectratio=169,xcolor={dvipsnames,table}]{beamer}
\usepackage[no-math,deluxe,haranoaji]{luatexja-preset}
\renewcommand{\kanjifamilydefault}{\gtdefault}
\renewcommand{\emph}[1]{{\upshape\bfseries #1}}
\usetheme{metropolis}
\metroset{block=fill}
\setbeamertemplate{navigation symbols}{}
\setbeamertemplate{blocks}[rounded][shadow=false]
\usecolortheme[rgb={0.7,0.2,0.2}]{structure}
%%%%%%%%%%%%%%%%%%%%%%%%%%
%% Change alert block colors
%%% 1- Block title (background and text)
\setbeamercolor{block title alerted}{fg=mDarkTeal, bg=mLightBrown!45!yellow!45}
\setbeamercolor{block title example}{fg=magenta!10!black, bg=mLightGreen!70}
%%% 2- Block body (background)
\setbeamercolor{block body alerted}{bg=mLightBrown!25}
\setbeamercolor{block body example}{bg=mLightGreen!15}
%%%%%%%%%%%%%%%%%%%%%%%%%%%
%%%%%%%%%%%%%%%%%%%%%%%%%%%
%% さまざまなアイコン
%%%%%%%%%%%%%%%%%%%%%%%%%%%
%\usepackage{fontawesome}
\usepackage{fontawesome5}
\usepackage{figchild}
\usepackage{twemojis}
\usepackage{utfsym}
\usepackage{bclogo}
\usepackage{marvosym}
\usepackage{fontmfizz}
\usepackage{pifont}
\usepackage{phaistos}
\usepackage{worldflags}
\usepackage{jigsaw}
\usepackage{tikzlings}
\usepackage{tikzducks}
\usepackage{scsnowman}
\usepackage{epsdice}
\usepackage{halloweenmath}
\usepackage{svrsymbols}
\usepackage{countriesofeurope}
\usepackage{tipa}
\usepackage{manfnt}
%%%%%%%%%%%%%%%%%%%%%%%%%%%
\usepackage{tikz}
\usetikzlibrary{calc,patterns,decorations.pathmorphing,backgrounds}
\usepackage{tcolorbox}
\usepackage{tikzpeople}
\usepackage{circledsteps}
\usepackage{xcolor}
\usepackage{amsmath}
\usepackage{booktabs}
\usepackage{chronology}
\usepackage{signchart}
%%%%%%%%%%%%%%%%%%%%%%%%%%%
%% 場合分け
%%%%%%%%%%%%%%%%%%%%%%%%%%%
\usepackage{cases}
%%%%%%%%%%%%%%%%%%%%%%%%%%
\usepackage{pdfpages}
%%%%%%%%%%%%%%%%%%%%%%%%%%%
%% 音声リンク表示
\newcommand{\myaudio}[1]{\href{#1}{\faVolumeUp}}
%%%%%%%%%%%%%%%%%%%%%%%%%%
%% \myAnch{<名前>}{<色>}{<テキスト>}
%% 指定のテキストを指定の色の四角枠で囲み, 指定の名前をもつTikZの
%% ノードとして出力する. 図には remember picture 属性を付けている
%% ので外部から参照可能である.
\newcommand*{\myAnch}[3]{%
  \tikz[remember picture,baseline=(#1.base)]
    \node[draw,rectangle,line width=1pt,#2] (#1) {\normalcolor #3};
}
%%%%%%%%%%%%%%%%%%%%%%%%%%
%% \myEmph コマンドの定義
%%%%%%%%%%%%%%%%%%%%%%%%%%
%\newcommand{\myEmph}[3]{%
%    \textbf<#1>{\color<#1>{#2}{#3}}%
%}
\usepackage{xparse} % xparseパッケージの読み込み
\NewDocumentCommand{\myEmph}{O{} m m}{%
    \def\argOne{#1}%
    \ifx\argOne\empty
        \textbf{\color{#2}{#3}}% オプション引数が省略された場合
    \else
        \textbf<#1>{\color<#1>{#2}{#3}}% オプション引数が指定された場合
    \fi
}
%%%%%%%%%%%%%%%%%%%%%%%%%%%
%%%%%%%%%%%%%%%%%%%%%%%%%%%
%% 文末の上昇イントネーション記号 \myRisingPitch
%% 通常のイントネーション \myDownwardPitch
%% https://note.com/dan_oyama/n/n8be58e8797b2
%%%%%%%%%%%%%%%%%%%%%%%%%%%
\newcommand{\myRisingPitch}{
\begin{tikzpicture}[scale=0.3,baseline=0.3]
\draw[->,>=stealth] (0,0) to[bend right=45] (1,1);
\end{tikzpicture}
}
\newcommand{\myDownwardPitch}{
\begin{tikzpicture}[scale=0.3,baseline=0.3]
\draw[->,>=stealth] (0,1) to[bend left=45] (1,0);
\end{tikzpicture}
}
%%%%%%%%%%%%%%%%%%%%%%%%%%%%
%\AtBeginSection[%
%]{%
%  \begin{frame}[plain]\frametitle{授業の流れ}
%     \tableofcontents[currentsection]
%   \end{frame}%
%}

\usepackage{luatexja-otf}
%%%%%%%%%%%%%%%%%%%%%%%%%%%
\title{English is fun.}
\subtitle{What a beautiful flower!}
\author{}
\institute[]{}
\date[]

%%%%%%%%%%%%%%%%%%%%%%%%%%%%
%% TEXT
%%%%%%%%%%%%%%%%%%%%%%%%%%%%
\begin{document}

\begin{frame}[plain]
  \titlepage
\end{frame}

\section*{授業の流れ}
\begin{frame}[plain]
  \frametitle{授業の流れ}
  \tableofcontents
\end{frame}

\section{感嘆文}
\subsection{なんて~でしょう}
%%%%%%%%%%%%%%%%%%%%%%%%%%%%%%%%%%%%%%%%%%%%%
\begin{frame}[plain]{What \ldots !}

\begin{columns}[t]
%%%%%%%%%%%%%%%%%%%%%%%%%
 \begin{column}{.47\textwidth}  
 \begin{enumerate}
  \item<1-> This is \myAnch{a1}{NavyBlue}{a beautiful flower}.\\
       \mbox{}
  \item<2-> \myAnch{a2}{NavyBlue}{\myEmph[2-]{Maroon}{What} a beautiful flower} this is!
  \item<3-> What a beautiful flower!
 \end{enumerate}
 \end{column}
%%%%%%%%%%%%%%%%%%%%%%
 \begin{column}{.48\textwidth}  
 \begin{enumerate}\setcounter{enumi}{3}
  \item<5-> They have \myAnch{b1}{NavyBlue}{nice cars}.\\
       \mbox{}
  \item<6-> \myAnch{b2}{NavyBlue}{\myEmph[2-]{Maroon}{What} nice cars} they have!
  \item<7-> What nice cars!
 \end{enumerate}
 \end{column}
%%%%%%%%%%%%%%%%%%%%%%%
\end{columns}
 

\begin{tikzpicture}[remember picture, overlay]
\visible<2->{\draw[->, thick, NavyBlue] (a1.south) to[out=-90, in=90] (a2.north);}
\visible<6->{\draw[->, thick, NavyBlue] (b1.south) to[out=-90, in=90] (b2.north);}
\end{tikzpicture}


\visible<4->{%
\begin{exampleblock}{Topics for Today}
\pause
\begin{itemize}\small
 \item<4-> 「なんと~な\ldots{}でしょう」というとき$\longrightarrow$\,\,\,%
\Circled[fill color = white]{ What\,\,(a)\,\,形容詞 $+$ 名詞 } $+$ S $+$ V !
 \item<4->   文末に`!'をつける
\end{itemize}
     \end{exampleblock}
}
\hfill\myaudio{./audio/033_exclamatory_01.mp3}
\end{frame}
%%%%%%%%%%%%%%%%%%%%%
\begin{frame}[plain]{Exercises}
日本語の意味になるよう空所に適当な単語を補いましょう
 \begin{enumerate}
  \item うわあ、大きな家。\\
	(~~\alt<2->{What}{\phantom{What}}~~) a big house!
  \item なんておもしろい映画なんだ。\\
	(~~\alt<3->{What}{\phantom{What}}~~) (~~~\alt<3->{an}{\phantom{an}}~~~) interesting movie this is!\hfill{\scriptsize interesting: おもしろい}
  \item うわあ、なんて大きな犬を飼っているんでしょう。\\
	(~~\alt<4->{What}{\phantom{What}}~~) (~~~~\alt<4->{a}{\phantom{a}}~~~~) big (~~\alt<4->{dog}{\phantom{dog}}~~) you have!
 \end{enumerate}
\hfill\myaudio{./audio/033_exclamatory_02.mp3}

\end{frame}
%%%%%%%%%%%%%%%%%%%%%%%%%%%%%%%%%%%%%%%%%%%%%
\begin{frame}[plain]{How \ldots !}

\begin{columns}[t]
%%%%%%%%%%%%%%%%%%%%%%%%%
 \begin{column}{.47\textwidth}  
 \begin{enumerate}
  \item<1-> This flower is \myAnch{c1}{NavyBlue}{beautiful}.\\
       \mbox{}
  \item<2-> \myAnch{c2}{NavyBlue}{\myEmph[2-]{Maroon}{How} beautiful} this flower is!
  \item<3-> How beautiful!
 \end{enumerate}
 \end{column}
%%%%%%%%%%%%%%%%%%%%%%
 \begin{column}{.48\textwidth}  
 \begin{enumerate}\setcounter{enumi}{3}
  \item<4-> She runs \myAnch{d1}{NavyBlue}{fast}.\\
       \mbox{}
  \item<5-> \myAnch{d2}{NavyBlue}{\myEmph[2-]{Maroon}{How} fast} she runs!
  \item<6-> How fast!
 \end{enumerate}
 \end{column}
%%%%%%%%%%%%%%%%%%%%%%%
\end{columns}
 

\begin{tikzpicture}[remember picture, overlay]
\visible<2->{\draw[->, thick, NavyBlue] (c1.south) to[out=-90, in=90] (c2.north);}
\visible<5->{\draw[->, thick, NavyBlue] (d1.south) to[out=-90, in=90] (d2.north);}
\end{tikzpicture}


\visible<7->{%
\begin{exampleblock}{Topics for Today}
\pause
\begin{itemize}\small
 \item<7-> 「なんて~でしょう」というとき$\longrightarrow$\,\,\,%
\Circled[fill color = white]{\,\,\,How 
      $\left\{\begin{array}{l}
       形容詞\\
       副詞
      \end{array}\right\}$\,\,\,\,\,}\,\,\,$+$ S $+$ V !
 \item<7->   文末に`!'をつける
\end{itemize}
     \end{exampleblock}
}
\hfill\myaudio{./audio/033_exclamatory_03.mp3}

\end{frame}
%%%%%%%%%%%%%%%%%%%%%%%%
\begin{frame}[plain]{Exercises}
日本語の意味になるよう(~~~~~~)内の単語を並べ替えましょう。先頭に来る語は大文字で始めてください\hfill\myaudio{./audio/033_exclamatory_04.mp3}

 \begin{enumerate}
  \item あなたはなんて親切なんでしょう。\\
	( are / how / you / kind ) !\\
	\visible<2->{How kind you are!}
  \item わたしたちはなんて幸運なんでしょう。\\
	( lucky / are / how / we ) !\\
	\visible<3->{How lucky we are!}
  \item あの山はなんてきれいなのでしょう。\\
	( is / that / beautiful / mountain / how ) !\\
	\visible<4->{How beautiful that mountain is!}
  \item 彼女はなんて早口なんでしょう。\\
	( speaks / fast / how / she ) !\\
	\visible<5->{How fast she speaks!}
 \end{enumerate}

\end{frame}
%%%%%%%%%%%%%%%%%%%%%
\begin{frame}[plain]{Exercises}
日本語の意味になるよう(~~~~~~)内の語句を並べ替えましょう。先頭に来る語は大文字で書き始めてください。なお $[ +1 ]$とある場合は不足している1語を補ってください。$[ -1 ]$とある場合は余計な1語が含まれています\hfill\myaudio{./audio/033_exclamatory_05.mp3}
 \
 \begin{enumerate}
  \item これはなんてかわいい猫なんでしょう。\\
	( cat / this / a / is / cute / what ) !\\
	\visible<2->{What a cute cat this is!}
  \item この本はなんておもいしろいのでしょう。\\
	( this / how / book / is / intereting / an ) $[ -1 ]$\\
	\visible<3->{How intereting this book is! (✕ an)}
  \item 彼はなんて古い自転車を持っているんだろう。\\
	( bike / has / what / old / he ) ! $[ +1 ]$\\
	\visible<4->{What an old bike he has!}
 \end{enumerate}
\end{frame}
%%%%%%%%%%%%%%%%%%%%%%%%%%%
\begin{frame}[plain]{Exercises}
日本語の意味になるよう空所に適語を補いましょう\hfill\myaudio{./audio/033_exclamatory_06.mp3}
\begin{enumerate}
 \item  なんて運がいいんだろう。\\
	(~~\alt<2->{How}{\phantom{How}}~~) lucky!
 \item まあ、きれい。\\
       (~~\alt<3->{How}{\phantom{How}}~~) pretty!
 \item まあ、きれいなドレス。\\
       (~~\alt<4->{What}{\phantom{What}}~~) a pretty dress!
 \item うわあ、大きい。\\
       (~~\alt<5->{How}{\phantom{How}}~~) big!
 \item うわあ、大きな家。\\
       (~~\alt<6->{What}{\phantom{How}}~~) a big house!
 \item まあ、きれいな花。\\
       (~~\alt<7->{What}{\phantom{How}}~~) beautiful fowers!
\end{enumerate}
\end{frame}
%%%%%%%%%%%%%%%%%%%%%%%%%%%%%%
\begin{frame}[plain]{Exercises}
A, B\,\,2人の対話がなりたつように、(~~~~~~~~)に適する文を選択肢から選びましょう。\\
ひとつ余計なものが含まれています\hfill\myaudio{./audio/033_exclamatory_07.mp3}

\bigskip

 \begin{columns}[b]
   \begin{column}[T]{.5\textwidth}
    \begin{tabular}{rp{.9\textwidth}}
     A:& Look at this picture.\\
     B:& (~~\alt<2->{Wow, what a beautiful mountain.}%
	 {\phantom{Wow, a beautiful mountain.}}~~)\\
     A:& You see, it's not a mountain, but a rock.\\
     B:& A rock?\,\,\,\,\,\,(~~~~\alt<3->{What a big rock!}%
	 {\phantom{What a big rock!}}~~~~)\\
     A:& Yes, and it changes its color through a day.\\
     B:& (~~~~\alt<4->{How strange!}{\phantom{How strange!}}~~~)~~~But I want to see it someday.\\
    \end{tabular}
   \end{column}
%%%%%%%%%%%%%%%
    \begin{column}[T]{.45\textwidth}
    \begin{tcolorbox}[title=選択肢]
     \CID{7555} What a big rock!\\
     \CID{7556} How small!\\
     \CID{7557} How strange!\\
     \CID{7558} Wow, a beautiful mountain.
    \end{tcolorbox}

     \visible<2->{正解:\,\,\CID{7558}}
     \visible<3->{\,\,\CID{7555}}
     \visible<4->{\,\,\CID{7557}}
    \end{column}
 \end{columns}
\end{frame}
%%%%%%%%%%%%%%%%%%%%%%%%%%%%%
\end{document}


