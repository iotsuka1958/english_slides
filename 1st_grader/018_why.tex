\documentclass[aspectratio=169,xcolor={dvipsnames,table}]{beamer}
\usepackage[no-math,deluxe,haranoaji]{luatexja-preset}
\renewcommand{\kanjifamilydefault}{\gtdefault}
\renewcommand{\emph}[1]{{\upshape\bfseries #1}}
\usetheme{metropolis}
\metroset{block=fill}
\setbeamertemplate{navigation symbols}{}
\setbeamertemplate{blocks}[rounded][shadow=false]
\usecolortheme[rgb={0.7,0.2,0.2}]{structure}
%%%%%%%%%%%%%%%%%%%%%%%%%%%
%%%%%%%%%%%%%%%%%%%%%%%%%%%
%% さまざまなアイコン
%%%%%%%%%%%%%%%%%%%%%%%%%%%
%\usepackage{fontawesome}
\usepackage{fontawesome5}
\usepackage{figchild}
\usepackage{twemojis}
\usepackage{utfsym}
\usepackage{bclogo}
\usepackage{marvosym}
\usepackage{fontmfizz}
\usepackage{pifont}
\usepackage{phaistos}
\usepackage{worldflags}
\usepackage{jigsaw}
\usepackage{tikzlings}
\usepackage{tikzducks}
\usepackage{scsnowman}
\usepackage{epsdice}
\usepackage{halloweenmath}
\usepackage{svrsymbols}
\usepackage{countriesofeurope}
\usepackage{tipa}
\usepackage{manfnt}
%%%%%%%%%%%%%%%%%%%%%%%%%%%
\usepackage{tikz}
\usetikzlibrary{calc,patterns,decorations.pathmorphing,backgrounds}
\usepackage{tcolorbox}
\usepackage{tikzpeople}
\usepackage{circledsteps}
\usepackage{xcolor}
\usepackage{amsmath}
\usepackage{booktabs}
\usepackage{chronology}
\usepackage{signchart}
%%%%%%%%%%%%%%%%%%%%%%%%%%%
%% 場合分け
%%%%%%%%%%%%%%%%%%%%%%%%%%%
\usepackage{cases}
%%%%%%%%%%%%%%%%%%%%%%%%%%
\usepackage{pdfpages}
%%%%%%%%%%%%%%%%%%%%%%%%%%%
%% 音声リンク表示
\newcommand{\myaudio}[1]{\href{#1}{\faVolumeUp}}
%%%%%%%%%%%%%%%%%%%%%%%%%%
%% \myAnch{<名前>}{<色>}{<テキスト>}
%% 指定のテキストを指定の色の四角枠で囲み, 指定の名前をもつTikZの
%% ノードとして出力する. 図には remember picture 属性を付けている
%% ので外部から参照可能である.
\newcommand*{\myAnch}[3]{%
  \tikz[remember picture,baseline=(#1.base)]
    \node[draw,rectangle,line width=1pt,#2] (#1) {\normalcolor #3};
}
%%%%%%%%%%%%%%%%%%%%%%%%%%
%% \myEmph コマンドの定義
%%%%%%%%%%%%%%%%%%%%%%%%%%
%\newcommand{\myEmph}[3]{%
%    \textbf<#1>{\color<#1>{#2}{#3}}%
%}
\usepackage{xparse} % xparseパッケージの読み込み
\NewDocumentCommand{\myEmph}{O{} m m}{%
    \def\argOne{#1}%
    \ifx\argOne\empty
        \textbf{\color{#2}{#3}}% オプション引数が省略された場合
    \else
        \textbf<#1>{\color<#1>{#2}{#3}}% オプション引数が指定された場合
    \fi
}
%%%%%%%%%%%%%%%%%%%%%%%%%%%
%%%%%%%%%%%%%%%%%%%%%%%%%%%
%% 文末の上昇イントネーション記号 \myRisingPitch
%% 通常のイントネーション \myDownwardPitch
%% https://note.com/dan_oyama/n/n8be58e8797b2
%%%%%%%%%%%%%%%%%%%%%%%%%%%
\newcommand{\myRisingPitch}{
\begin{tikzpicture}[scale=0.3,baseline=0.3]
\draw[->,>=stealth] (0,0) to[bend right=45] (1,1);
\end{tikzpicture}
}
\newcommand{\myDownwardPitch}{
\begin{tikzpicture}[scale=0.3,baseline=0.3]
\draw[->,>=stealth] (0,1) to[bend left=45] (1,0);
\end{tikzpicture}
}
%%%%%%%%%%%%%%%%%%%%%%%%%%%%
%\AtBeginSection[%
%]{%
%  \begin{frame}[plain]\frametitle{授業の流れ}
%     \tableofcontents[currentsection]
%   \end{frame}%
%}

\usepackage{pxrubrica}
%%%%%%%%%%%%%%%%%%%%%%%%%%%
\title{English is fun.}
\subtitle{Why is she happy?}
\author{}
\institute[]{}
\date[]

%%%%%%%%%%%%%%%%%%%%%%%%%%%%
%% TEXT
%%%%%%%%%%%%%%%%%%%%%%%%%%%%
\begin{document}
\begin{frame}[plain]
  \titlepage
\end{frame}

\section*{授業の流れ}
\begin{frame}[plain]
  \frametitle{授業の流れ}
  \tableofcontents
\end{frame}

%%%%%%%%%%%%%%%%%%%%%%%%%%%%%%%
\section{Why}
\subsection{理由や原因をたずねるとき}
%%%%%%%%%%%%%%%%%%%%
\begin{frame}[plain]{なぜ} \Large

\mbox{}\hspace{42pt} She is happy \alt<3->{\myAnch{FOCUS}{orange}{for some reason}}{\myAnch{focus}{white}{for some reason}}.
\hfill{\scriptsize for some reason: なんらかの理由で}

\pause

\vspace{7pt}

\mbox{}\hfill{}{\small cf. \myEmph[7-]{Maroon}{Is she} happy for some reason?}%

\vspace{-5pt}

\hfill{\small YesまたはNoで答える疑問文}

\pause

\visible<4->{\myAnch{wh}{orange}{\bfseries Why} \myEmph[7-]{Maroon}{\bfseries is she} happy \myAnch{question}{orange}{?}}
\visible<6->{\scalebox{1.4}{\myDownwardPitch}}

\pause

\visible<5->{%
\begin{tikzpicture}[remember picture, overlay]
\draw[->, line width=3pt,opacity=.5, orange] (focus.south) to[out=-90, in=90]  node[sloped,above,text=black,font=\tiny,pos=.667]{先頭へ} (wh.north);
\end{tikzpicture}
}

\visible<7->{%
\begin{exampleblock}{Topics for Today}
\pause
\begin{itemize}\setbeamertemplate{items}[square]\small
 \item 「なぜ、どうして」と\kenten{理由}・\kenten{原因}を聞くとき$\longrightarrow$\,\,\,疑問詞{\bfseries why} \textipa{/w\'aI/}\hfill{\scriptsize \myaudio{./audio/018_why_01.mp3}}
 \item {\bfseries Why}は文の先頭
 \item {\bfseries Why}の後は疑問文の語順
 \item   文末に`?'をつける(イントネーションは\myDownwardPitch{}\,\,)
\end{itemize}
     \end{exampleblock}
}

\end{frame}
%%%%%%%%%%%%%%%%%%%%%%%%%%%%%%%%
\begin{frame}[plain]{なぜ} \Large

\mbox{}\hspace{42pt} She likes jazz \alt<3->{\myAnch{FOCUS}{orange}{for some reason}}{\myAnch{focus}{white}{for some reason}}.
\hfill{\scriptsize for some reason: なんらかの理由で}

\pause

\vspace{7pt}

\mbox{}\hfill{}{\small cf. \myEmph[7-]{Maroon}{Does  she like} jazz for some reason?}%

\vspace{-5pt}

\hfill{\small YesまたはNoで答える疑問文}

\pause

\visible<4->{\myAnch{wh}{orange}{\bfseries Why} \myEmph[7-]{Maroon}{\bfseries does she like} jazz \myAnch{question}{orange}{?}}
\visible<6->{\scalebox{1.4}{\myDownwardPitch}}

\pause

\visible<5->{%
\begin{tikzpicture}[remember picture, overlay]
\draw[->, line width=3pt,opacity=.5, orange] (focus.south) to[out=-90, in=90]  node[sloped,above,text=black,font=\tiny,pos=.667]{先頭へ} (wh.north);
\end{tikzpicture}
}

\visible<7->{%
\begin{exampleblock}{Topics for Today}
\pause
\begin{itemize}\setbeamertemplate{items}[square]\small
 \item 「なぜ、どうして」と\kenten{理由}・\kenten{原因}を聞くとき$\longrightarrow$\,\,\,疑問詞{\bfseries why} \textipa{/w\'aI/}\hfill{\scriptsize \myaudio{./audio/018_why_01b.mp3}}
 \item {\bfseries Why}は文の先頭
 \item {\bfseries Why}の後は疑問文の語順
 \item   文末に`?'をつける(イントネーションは\myDownwardPitch{}\,\,)
\end{itemize}
     \end{exampleblock}
}

\end{frame}
%%%%%%%%%%%%%%%%%%%%%%%%%%%%%%%%
\begin{frame}[plain]{Exercises 1}
日本語の意味になるようにカッコ内の語句を並べ替えてください。ただし解答にあたっては1語補ってください。また先頭の単語は大文字で始めてください
 \begin{enumerate}
  \item ( is / sad / he ) ? どうして彼は悲しいのですか?\\
\visible<2->{Why is he sad?}
  \item ( blue / the sky/ why ) ? なぜ空は青いのですか?\\
\visible<3->{Why is the sky blue?}
  \item ( pasta / you / like / why )? どうしてあなたはパスタが好きなのですか?\\
\visible<4->{Why do you like pasta?}
  \item ( English / do / you / why ) ? なぜあなたは英語を勉強するのですか?\\
\visible<5->{Why do you study English?}
 \end{enumerate}

\hfill\myaudio{./audio/018_why_02.mp3}
\end{frame}
%%%%%%%%%%%%%%%%%%%%%%%%%%
\section{Whyで聞かれたときの答え方}
\begin{frame}[plain]{Whyで聞かれたときの答え方}
 \begin{enumerate}
  \item<1-> \myEmph[1-]{Maroon}{Why} is he sad? --- \myEmph[2-]{Maroon}{Because} \myEmph[3-]{NavyBlue}{it is} rainy.
  \item<4-> \myEmph[4-]{Maroon}{Why} do you like pasta? --- \myEmph[5-]{Maroon}{Because} \myEmph[6-]{NavyBlue}{it is} delicious.%
\hfill{\scriptsize delicious \textipa{/dIl\'IS@s/} おいしい}
  \item<7-> \myEmph[7-]{Maroon}{Why} do you study English? --- \myEmph[8-]{Maroon}{Because} \myEmph[9-]{NavyBlue}{I like} English songs.
 \end{enumerate}

\bigskip

\visible<10->{%
\begin{exampleblock}{Topics for Today}
\pause
\begin{itemize}\setbeamertemplate{items}[square]\small
 \item {\bfseries Why} \ldots\,?で質問されたら{\bfseries Because}をつかって答えます\hfill{\small \textipa{/bIk\'O:z/}}
 \item {\bfseries Because}の後には\,\,\Circled[fill color=white]{\,\,$\text{S} + \text{V}$\,\,}\,\,が続きます
\end{itemize}
     \end{exampleblock}
}

\hfill{\scriptsize \myaudio{./audio/018_why_03.mp3}}
\end{frame}
\section{まとめ}
%%%%%%%%%%%%%%%%%%%%%%%%%
\begin{frame}[plain]{まとめ}
 
\begin{exampleblock}{Topics for Today}
\begin{itemize}\setbeamertemplate{items}[square]\small
 \item 「なぜ、どうして」と\kenten{理由}・\kenten{原因}を聞くとき$\longrightarrow$\,\,\,疑問詞{\bfseries why} \textipa{/w\'aI/}
 \item {\bfseries Why}は文の先頭
 \item {\bfseries Why}の後は疑問文の語順
 \item   文末に`?'をつける(イントネーションは\myDownwardPitch{}\,\,)\\
\hfill{}{\scriptsize {\bfseries Why} is she happy?}\\
\hfill{}{\scriptsize {\bfseries Why} does she like jazz?}
\end{itemize}
     \end{exampleblock}


\begin{exampleblock}{Topic for Today}
\begin{itemize}\setbeamertemplate{items}[square]\small
 \item {\bfseries Why} \ldots\,?で質問されたら{\bfseries Because}をつかって答えます\hfill{\scriptsize \textipa{/bIk\'O:z/}}
 \item {\bfseries Because}の後には\,\,\Circled[fill color=white]{\,\,$\text{S} + \text{V}$\,\,}\,\,が続きます

\hfill{}{\scriptsize {\bfseries Why} is he sad? --- {\bfseries Because} it is rainy.}\\
\hfill{}{\scriptsize {\bfseries Why} do you study English? --- {\bfseries Because} I like English songs.}
\end{itemize}
     \end{exampleblock}
\hfill{\scriptsize \myaudio{./audio/018_why_04.mp3}}
\end{frame}
\end{document}
