\documentclass[aspectratio=169,xcolor={dvipsnames,table}]{beamer}
\usepackage[no-math,deluxe,haranoaji]{luatexja-preset}
\renewcommand{\kanjifamilydefault}{\gtdefault}
\renewcommand{\emph}[1]{{\upshape\bfseries #1}}
\usetheme{metropolis}
\metroset{block=fill}
\setbeamertemplate{navigation symbols}{}
\setbeamertemplate{blocks}[rounded][shadow=false]
\usecolortheme[rgb={0.7,0.2,0.2}]{structure}
%%%%%%%%%%%%%%%%%%%%%%%%%%%
%%%%%%%%%%%%%%%%%%%%%%%%%%%
%% さまざまなアイコン
%%%%%%%%%%%%%%%%%%%%%%%%%%%
%\usepackage{fontawesome}
\usepackage{fontawesome5}
\usepackage{figchild}
\usepackage{twemojis}
\usepackage{utfsym}
\usepackage{bclogo}
\usepackage{marvosym}
\usepackage{fontmfizz}
\usepackage{pifont}
\usepackage{phaistos}
\usepackage{worldflags}
\usepackage{jigsaw}
\usepackage{tikzlings}
\usepackage{tikzducks}
\usepackage{scsnowman}
\usepackage{epsdice}
\usepackage{halloweenmath}
\usepackage{svrsymbols}
\usepackage{countriesofeurope}
\usepackage{tipa}
\usepackage{manfnt}
%%%%%%%%%%%%%%%%%%%%%%%%%%%
\usepackage{tikz}
\usetikzlibrary{calc,patterns,decorations.pathmorphing,backgrounds}
\usepackage{tcolorbox}
\usepackage{tikzpeople}
\usepackage{circledsteps}
\usepackage{xcolor}
\usepackage{amsmath}
\usepackage{booktabs}
\usepackage{chronology}
\usepackage{signchart}
%%%%%%%%%%%%%%%%%%%%%%%%%%%
%% 場合分け
%%%%%%%%%%%%%%%%%%%%%%%%%%%
\usepackage{cases}
%%%%%%%%%%%%%%%%%%%%%%%%%%
\usepackage{pdfpages}
%%%%%%%%%%%%%%%%%%%%%%%%%%%
%% 音声リンク表示
\newcommand{\myaudio}[1]{\href{#1}{\faVolumeUp}}
%%%%%%%%%%%%%%%%%%%%%%%%%%
%% \myAnch{<名前>}{<色>}{<テキスト>}
%% 指定のテキストを指定の色の四角枠で囲み, 指定の名前をもつTikZの
%% ノードとして出力する. 図には remember picture 属性を付けている
%% ので外部から参照可能である.
\newcommand*{\myAnch}[3]{%
  \tikz[remember picture,baseline=(#1.base)]
    \node[draw,rectangle,line width=1pt,#2] (#1) {\normalcolor #3};
}
%%%%%%%%%%%%%%%%%%%%%%%%%%
%% \myEmph コマンドの定義
%%%%%%%%%%%%%%%%%%%%%%%%%%
%\newcommand{\myEmph}[3]{%
%    \textbf<#1>{\color<#1>{#2}{#3}}%
%}
\usepackage{xparse} % xparseパッケージの読み込み
\NewDocumentCommand{\myEmph}{O{} m m}{%
    \def\argOne{#1}%
    \ifx\argOne\empty
        \textbf{\color{#2}{#3}}% オプション引数が省略された場合
    \else
        \textbf<#1>{\color<#1>{#2}{#3}}% オプション引数が指定された場合
    \fi
}
%%%%%%%%%%%%%%%%%%%%%%%%%%%
%%%%%%%%%%%%%%%%%%%%%%%%%%%
%% 文末の上昇イントネーション記号 \myRisingPitch
%% 通常のイントネーション \myDownwardPitch
%% https://note.com/dan_oyama/n/n8be58e8797b2
%%%%%%%%%%%%%%%%%%%%%%%%%%%
\newcommand{\myRisingPitch}{
\begin{tikzpicture}[scale=0.3,baseline=0.3]
\draw[->,>=stealth] (0,0) to[bend right=45] (1,1);
\end{tikzpicture}
}
\newcommand{\myDownwardPitch}{
\begin{tikzpicture}[scale=0.3,baseline=0.3]
\draw[->,>=stealth] (0,1) to[bend left=45] (1,0);
\end{tikzpicture}
}
%%%%%%%%%%%%%%%%%%%%%%%%%%%%
%\AtBeginSection[%
%]{%
%  \begin{frame}[plain]\frametitle{授業の流れ}
%     \tableofcontents[currentsection]
%   \end{frame}%
%}

%%%%%%%%%%%%%%%%%%%%%%%%%%%
\title{English is fun.}
\subtitle{Why is she happy?}
\author{}
\institute[]{}
\date[]

%%%%%%%%%%%%%%%%%%%%%%%%%%%%
%% TEXT
%%%%%%%%%%%%%%%%%%%%%%%%%%%%
\begin{document}
\begin{frame}[plain]
  \titlepage
\end{frame}

\section*{授業の流れ}
\begin{frame}[plain]
  \frametitle{授業の流れ}
  \tableofcontents
\end{frame}

%%%%%%%%%%%%%%%%%%%%%%%%%%%%%%%
\section{Why}
\subsection{Why is she happy?}
\begin{frame}[plain]{Why is she happy?} \Large

\mbox{}\hspace{42pt} She is happy \alt<3->{\myAnch{FOCUS}{orange}{for some reason}}{\myAnch{focus}{white}{for some reason}}.
\hfill{\scriptsize for some reason: なんらかの理由で}

\pause


\vspace{7pt}

\mbox{}\hfill{}{\small cf. \myEmph[7-]{Maroon}{Is she} happy for some reason?}%

\vspace{-5pt}

\hfill{\small YesまたはNoで答える疑問文}

\pause

\visible<4->{\myAnch{wh}{orange}{Why} \myEmph[7-]{Maroon}{is she} happy \myAnch{question}{orange}{?}}
\visible<6->{\scalebox{1.4}{\myDownwardPitch}}

\pause

%\mbox{}\hspace{30pt}\myAnch{txt1}{white}{\small 先頭にWho}

\visible<5->{%
\begin{tikzpicture}[remember picture, overlay]
\draw[->, thick, orange] (focus.south) to[out=-90, in=90] (wh.north);
\end{tikzpicture}
}

\visible<7->{%
\begin{exampleblock}{Topics for Today}
\pause
\begin{itemize}\small
 \item 「なぜ」と聞くとき$\longrightarrow$\,\,\,Why
 \item Whyの後は疑問文の語順
 \item   文末に`?'をつける(イントネーションは\myDownwardPitch{}\,\,)
\end{itemize}
     \end{exampleblock}
}
\hfill\myaudio{./audio/018_why_01.mp3}

\end{frame}
%%%%%%%%%%%%%%%%%%%%%%%%%%%%%%%%
\begin{frame}[plain]{Exercises 1}
日本語の意味になるようにカッコ内の語句を並べ替えてください。ただし解答にあたっては1語補ってください。また先頭の単語は大文字で始めてください
 \begin{enumerate}
  \item ( is / sad / he ) ? なぜ彼は悲しいのですか?\\
\visible<2->{Why is he sad?}
  \item ( blue / the sky/ why ) ? なぜ空は青いのですか?\\
\visible<3->{Why is the sky blue?}
  \item ( pasta / you / like / why )? なぜあなたはパスタが好きですか?\\
\visible<4->{Why do you like pasta?}
  \item ( English / do / you / why ) ? なぜあなたは英語を勉強するのですか?\\
\visible<5->{Why do you study English?}
 \end{enumerate}

\hfill\myaudio{./audio/018_why_02.mp3}
\end{frame}
%%%%%%%%%%%%%%%%%%%%%%%%%%
\begin{frame}[plain]{Whyで聞かれたときの答え方}
 \begin{enumerate}
  \item \myEmph[3-]{Maroon}{Why} is he sad? --- \myEmph[3-]{Maroon}{Because} \myEmph[3-]{NavyBlue}{it is} rainy.
  \item \myEmph[3-]{Maroon}{Why} do you like pasta? --- \myEmph[3-]{Maroon}{Because} \myEmph[3-]{NavyBlue}{it is} delicious.
  \item \myEmph[3-]{Maroon}{Why} do you study English. --- \myEmph[3-]{Maroon}{Because} \myEmph[3-]{NavyBlue}{I like} English songs.
 \end{enumerate}

\visible<2->{%
\begin{exampleblock}{Topic for Today}
\pause
\begin{itemize}\small
 \item Why \ldots\,?で質問されたらBecauseをつかって答えます
 \item Becauseの後には$\text{S} + \text{V}$が続きます
\end{itemize}
     \end{exampleblock}
}

\hfill\myaudio{./audio/018_why_03.mp3}
\end{frame}
\end{document}
