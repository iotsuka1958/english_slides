\documentclass[aspectratio=169,xcolor={dvipsnames,table}]{beamer}
\usepackage[no-math,deluxe,haranoaji]{luatexja-preset}
\renewcommand{\kanjifamilydefault}{\gtdefault}
\renewcommand{\emph}[1]{{\upshape\bfseries #1}}
\usetheme{metropolis}
\metroset{block=fill}
\setbeamertemplate{navigation symbols}{}
\setbeamertemplate{blocks}[rounded][shadow=false]
\usecolortheme[rgb={0.7,0.2,0.2}]{structure}
%%%%%%%%%%%%%%%%%%%%%%%%%%%
\usepackage{media9}
%%%%%%%%%%%%%%%%%%%%%%%%%%%
%% さまざまなアイコン
%%%%%%%%%%%%%%%%%%%%%%%%%%%
\usepackage{fontawesome}
\usepackage{figchild}
\usepackage{twemojis}
\usepackage{utfsym}
\usepackage{bclogo}
\usepackage{marvosym}
\usepackage{fontmfizz}
\usepackage{pifont}
\usepackage{phaistos}
\usepackage{worldflags}
%%%%%%%%%%%%%%%%%%%%%%%%%%%
\usepackage{tikz}
\usetikzlibrary{backgrounds}
\usepackage{tcolorbox}
\usepackage{tikzpeople}
\usepackage{tikzducks}
\usepackage{circledsteps}
\usepackage{xcolor}
\usepackage{amsmath}
\usepackage{booktabs}
\usepackage{tipa}
\usepackage{manfnt}
\usepackage{pxrubrica}
%%%%%%%%%%%%%%%%%%%%%%%%%%%
\newcommand*\myCrossedOut[2]{%
  \tikz[baseline=(T.base)]
    \node[draw=#1, thick, shape=cross out, decorate,
      inner sep=2pt, outer sep=0pt,
      decoration={random steps, segment length=2pt, amplitude=0.4pt}]
      (T) {#2};}
%%%%%%%%%%%%%%%%%%%%%%%%%%%
%% 場合分け
\usepackage{cases}
%%%%%%%%%%%%%%%%%%%%%%%%%%%
% \myAnch{<名前>}{<色>}{<テキスト>}
% 指定のテキストを指定の色の四角枠で囲み, 指定の名前をもつTikZの
% ノードとして出力する. 図には remeber picture 属性を付けている
% ので外部から参照可能である.
\newcommand*{\myAnch}[3]{%
  \tikz[remember picture,baseline=(#1.base)]
    \node[draw,rectangle,#2] (#1) {\normalcolor #3};
}
%%%%%%%%%%%%%%%%%%%%%%%%%%%%
%% 音声リンク表示
\newcommand{\myaudio}[1]{\href{#1}{\faVolumeUp}}
%%%%%%%%%%%%%%%%%%%%%%%%%%%
% \myEmph コマンドの定義
%\newcommand{\myEmph}[3]{%
%    \textbf<#1>{\color<#1>{#2}{#3}}%
%}
\usepackage{xparse} % xparseパッケージの読み込み
\NewDocumentCommand{\myEmph}{O{} m m}{%
    \def\argOne{#1}%
    \ifx\argOne\empty
        \textbf{\color{#2}{#3}}% オプション引数が省略された場合
    \else
        \textbf<#1>{\color<#1>{#2}{#3}}% オプション引数が指定された場合
    \fi
}
%%%%%%%%%%%%%%%%%%%%%%%%%%%
%% 文末の上昇イントネーション記号 \myRisingPitch
%% 通常のイントネーション \myDownwardPitch
%% https://note.com/dan_oyama/n/n8be58e8797b2
%%%%%%%%%%%%%%%%%%%%%%%%%%%
\newcommand{\myRisingPitch}{
\begin{tikzpicture}[scale=0.3,baseline=0.3]
\draw[->,>=stealth] (0,0) to[bend right=45] (1,1);
\end{tikzpicture}
}
\newcommand{\myDownwardPitch}{
\begin{tikzpicture}[scale=0.3,baseline=0.3]
\draw[->,>=stealth] (0,1) to[bend left=45] (1,0);
\end{tikzpicture}
}
%%%%%%%%%%%%%%%%%%%%%%%%%%%
\title{English is fun.}
\subtitle{I played tennis yesterday.}
\author{}
\institute[]{}
\date[]

%%%%%%%%%%%%%%%%%%%%%%%%%%%%
%% TEXT
%%%%%%%%%%%%%%%%%%%%%%%%%%%%
\begin{document}
\begin{frame}[plain]
  \titlepage
\end{frame}

\section*{授業の流れ}
\begin{frame}[plain]\small
  \frametitle{授業の流れ}
  \tableofcontents
\end{frame}


\section{復習}

%%%%%%%%%%%%%%%%%%%%%%%%%%%%%%%%%%%%%%%%%%%%%%%%%%%%
\subsection{be動詞と一般動詞}
\begin{frame}[plain]\frametitle{be動詞と一般動詞(復習)}
\begin{tikzpicture}

% グリッドを描画(5mm刻み)
%\draw[step=1cm, gray!20, very thin] (-6,0) grid (6,5);

\node[rectangle, rounded corners, draw=black, fill=white] (verb) at (-3,5) {動詞};

% 左側のノード(be動詞)
\node[circle, draw=black, line width=1pt, fill=yellow!30, minimum size=2cm] (be) at (-2.5,2.5) {be動詞{\footnotesize ($=$)}};

% 右側のノード(一般動詞)
\node[rectangle, draw=black, line width=1pt, fill=pink!30, minimum width=4cm, minimum height=1.5cm, inner sep=20pt] (general) at (4,2.5) {一般動詞\,\,{}$\left\{\begin{array}{ll}
\text{go}&\text{walk}\\
\text{have}&\text{like}\\
\text{speak}&\text{study}\\
\text{eat}& \text{drink}\\
\text{play}&\text{ほかにもたくさん}\\
\end{array}\right.$};


% 大きな長方形を描画して2つのノードを包む
\begin{scope}[on background layer]
\draw[draw=black!50, fill=blue!10, rounded corners, line width=1.5pt]
    ([xshift=-25pt,yshift=50pt]be.north west) rectangle ([xshift=8pt,yshift=-12pt]general.south east);
\end{scope}

\end{tikzpicture}
\end{frame}
%%%%%%%%%%%%%%%%%%%%%%%%%%%%%%
\subsection{人称}
\begin{frame}[plain,label=ninsyo]\frametitle{人称(復習)}

\begin{block}{人称とは}
\begin{description}
\item[1人称] 話し手(自分)のこと\pause{}\,\,{} $\longrightarrow$ Iとweだけ\pause
\item[2人称] 聞き手(相手)のこと\pause{}\,\,{} $\longrightarrow$ youだけ\pause
\item[3人称] 1人称、2人称以外\pause{}\hspace{15pt} $\longrightarrow$%
 \begin{tabular}[t]{@{\,\,}l}
he\\\pause
she\\\pause
it\\\pause
they\\\pause
the teacher\\\pause
this pencil\\\pause
the cats\\\pause
\ldots
 \end{tabular}
\end{description}
\end{block}
\end{frame}
%%%%%%%%%%%%%%%%%%%%%%%
\subsection{一般動詞の現在形}
\begin{frame}[plain]{一般動詞の現在形(復習)}
\begin{enumerate}
   \item<1-> \myEmph[9,17]{Maroon}{I} \myEmph[9,17]{NavyBlue}{like} music. \onslide*<2>{\footnotesize  like: 好き、好む music: 音楽}
   \item<3-> \myEmph[10,17]{Maroon}{We} \myEmph[10,17]{NavyBlue}{like} music.
   \item<4-> \myEmph[11,17]{Maroon}{You} \myEmph[11,17]{NavyBlue}{like} music.
   \item<5-> \myEmph[12,18]{Maroon}{He} \myEmph[12,18]{NavyBlue}{likes} music.
   \item<6-> \myEmph[13,18]{Maroon}{She} \myEmph[13,18]{NavyBlue}{likes} music.
   \item<7-> \myEmph[14,18]{Maroon}{Paul} \myEmph[14,18]{NavyBlue}{likes} music.
   \item<8-> \myEmph[15,19]{Maroon}{They} \myEmph[15,19]{NavyBlue}{like} music.
  \end{enumerate}
\begin{exampleblock}<17->{Topics for Today}
\begin{itemize}\setbeamertemplate{items}[square]\small
 \item<1-> \myEmph[17]{Maroon}{主語が1人称、2人称のとき、動詞はそのまま}
 \item<2-> \myEmph[18]{Maroon}{主語が3人称で単数のとき、---sとなります}\myEmph[19]{Maroon}{(複数ならそのままです)}
\end{itemize}
      \end{exampleblock}
\end{frame}
%%%%%%%%%%%%%%%%%%%%%%%%%%%%%%%%%%%%%

\begin{frame}<1-3>[plain]\frametitle{3単現のs}
\Large

\myEmph[2]{orange}{現在}を表す文では、
主語が\myEmph[2]{orange}{3人称}で\myEmph[2]{orange}{単数}のとき\\動詞の最後に\,\myEmph[2]{orange}{\Circled{\,s\,}}\,をつけます

\pause

\mbox{}\hfill$\longrightarrow{}$これを\myAnch{T1}{orange}{3単現の`s'}といいます\pause

She play\myAnch{T2}{orange}{s} \,\,{}the guitar.

% ノード間を結ぶ矢印を別のTikZ環境で描く
\begin{tikzpicture}[remember picture,overlay]
\draw[->,orange,line width=2pt,opacity=.5] (T1.south)to[out=-120,in=-30](T2.south east);
\end{tikzpicture}
\end{frame}
%%%%%%%%%%%%%%%%%%%%%%%%%%%%%%%%%%
\begin{frame}<1-10>[plain]\frametitle{Exercises}
{\small つぎの空所に適当な動詞を選んで補い、\kenten{現在}のことを表す英文をつくってください。なお、必要があれば、形を変えてください}

\begin{columns}
\begin{column}{.5\textwidth}
 % \setbeamercovered{transparent}
  \begin{enumerate}
   \item I (\onslide<2,8,9->{\textcolor{orange}{\bfseries ~have~}}) two cats.
   \item We (\onslide<3,8,9->{\textcolor{orange}{\bfseries ~go~}}) to school by bus.
   \item  She (\onslide<4,8,9->{\textcolor{orange}{\bfseries ~drinks~}}) coffee every morning.
   \item Tom (\onslide<5,8,9->{\textcolor{orange}{\bfseries ~eats~}}) bread for breakfast.
   \item Jennifer (\onslide<6,8,9->{\textcolor{orange}{\bfseries ~lives~}}) in Boston.
   \item They (\onslide<7,8,9->{\textcolor{orange}{\bfseries ~speak~}}) English.
  \end{enumerate}
\end{column}
\begin{column}{.45\textwidth}
\begin{tcolorbox}[title={\small この中から選んでください}]
live, go, have, speak, drink, eat
\end{tcolorbox}
\end{column}
\end{columns}

\vfill

\only<4,8,9->{{\small 3\,\,}}\only<5,8,9->{{\small 4\,\,}}\visible<4,5,8,9>{{\small \textcolor{orange}{\bfseries has}も正解です}}

\visible<10->{{\scriptsize \textdbend}\,\,{\small \textbf{have}は\textbf{eat}, \textbf{drink}の意味で使うことがあります}}

% Embed the sound file
\hfill{\tiny 0224}\,\,{\scriptsize \myaudio{./audio/004_verb_03.mp3}}
\end{frame}
%%%%%%%%%%%%%%%%%%%%%%%%%%%%%%%%%%%%%%%
\section{一般動詞の過去形}

\subsection{過去形とは}
%%%%%%%%%%%%%%%%%%%%%%%%%%%%%%%%%%%%
\begin{frame}[plain]{一般動詞の過去形}
 \large

\begin{columns}
\begin{column}<1->{.48\textwidth}
\begin{enumerate}
 \item 現在のこと
       \begin{enumerate}
	\item I play tennis every day.
	\item We play tennis every day.
	\item You play tennis every day.
	\item He play\textcolor{Maroon}{\bfseries s} tennis every day.
	\item They play tennis every day.
       \end{enumerate}
\end{enumerate}

{\small 現在形: playとplaysの使い分け}
\end{column}
\begin{column}<2->{.48\textwidth}
\begin{enumerate}
\setcounter{enumi}{1} \item 過去のこと
        \begin{enumerate}
	\item I \textcolor{ForestGreen}{\bfseries played} tennis every day.
	\item We \textcolor{ForestGreen}{\bfseries played} tennis every day.
	\item You \textcolor{ForestGreen}{\bfseries played} tennis every day.
	\item He \textcolor{ForestGreen}{\bfseries played} tennis every day.
	\item They \textcolor{ForestGreen}{\bfseries played} tennis every day.
       \end{enumerate}
\end{enumerate}

{\small 過去形: ぜんぶplayed}
\end{column}
\end{columns}

\vfill

\begin{block}<3->{Topics for Today}\small
\begin{itemize}\setbeamertemplate{items}[square]
 \item 一般動詞の過去形は、主語にかかわらず同じ
 \item 多くの動詞は、原形にedをつけることで過去形になります
\end{itemize}
\end{block}

\hfill{\tiny 0351}\,{\scriptsize \myaudio{./audio/025_past_do_01.mp3}}
\end{frame}
%%%%%%%%%%%%%%%%%%%%%%%%%%%%%%%%%%%
\subsection{過去形のつくり方}
\subsubsection{基本}
%%%%%%%%%%%%%%%%%%%%%%%%%%%%%%%%%%
\begin{frame}[plain]{過去形のつくり方--- 基本 ---}
 
\begin{enumerate}
 \item<1-> \begin{enumerate}
	\item We listen to the radio after dinner.\hfill{\scriptsize listen to ~を聞く\,\textipa{/l\'Isn/}}
	\item We listen\textcolor{ForestGreen}{\bfseries ed} to the radio after dinner.
       \end{enumerate}
 \item<2-> \begin{enumerate}
	\item I play the piano every day.
	\item I play\textcolor{ForestGreen}{\bfseries ed} the piano last night.
       \end{enumerate}
 \item<3-> \begin{enumerate}
	\item He cooks every morning.
	\item He cook\textcolor{ForestGreen}{\bfseries ed} this morning.
       \end{enumerate}

 \item<4-> \begin{enumerate}
	\item She walks to work every day.\hfill{}{\scriptsize work: 職場}
	\item She walk\textcolor{ForestGreen}{\bfseries ed} to work every day.
       \end{enumerate}
\end{enumerate}

\begin{block}<5->{Topics for Today}\small
\begin{itemize}\setbeamertemplate{items}[square]
 \item 一般動詞の過去形は、主語にかかわらず同じ
 \item 多くの動詞は、原形にedをつけることで過去形になります
\end{itemize}
\end{block}
\hfill{\tiny 0324}\,{\scriptsize \myaudio{./audio/025_past_do_02.mp3}}

\end{frame}
%%%%%%%%%%%%%%%%%%%%%%%%%%%
\begin{frame}<-17>[plain,label=ed]{edをつける過去形}
 {\small 原形にedをつけて過去形をつくる動詞の表を完成させましょう}

\begin{center}
 
\rowcolors{2}{NavyBlue!20}{yellow!50}
\begin{tabular}{llll}\toprule
{\small 原形}&{\small (意味)}&{\small 過去形}&\visible<17->{{\small 発音}}\\\midrule
\visible<1->{listen}&\visible<2->{{\small (聞く)}}&\visible<3->{liste\myEmph[19-]{Maroon}{n}ed}&\visible<17->{\textipa{/d/}}\\
\visible<1->{play}&\visible<4->{{\small (演奏する、競技をする)}}&\visible<5->{pl\myEmph[19-]{Maroon}{ay}ed}&\visible<17->{\textipa{/d/}}\\
\visible<1->{cook}&\visible<6->{{\small(調理する)}}&\visible<7->{coo\myEmph[20-]{ForestGreen}{k}ed}&\visible<17->{\textipa{/t/}}\\
\visible<1->{walk}&\visible<8->{{\small (歩く)}}&\visible<9->{wal\myEmph[20-]{ForestGreen}{k}ed}&\visible<17->{\textipa{/t/}}\\
\visible<1->{watch}&\visible<10->{{\small (見る)}}&\visible<11->{wa\myEmph[20-]{ForestGreen}{tch}ed}&\visible<17->{\textipa{/t/}}\\
\visible<1->{enjoy}&\visible<12->{{\small (楽しむ)}}&\visible<13->{enj\myEmph[19-]{Maroon}{oy}ed}&\visible<17->{\textipa{/d/}}\\
\visible<1->{learn}&\visible<14->{{\small (学ぶ)}}&\visible<15->{lear\myEmph[19-]{Maroon}{n}ed}&\visible<17->{\textipa{/d/}}\\\bottomrule
\end{tabular}%
\end{center}

\visible<16->{{\scriptsize 多くの動詞は、原形に--edをつけると過去形になります}}

\vspace{-5pt}

\visible<18->{{\scriptsize 動詞の原形の最後の音が有声音なら\textipa{/d/}、無声音なら\textipa{/t/}\,\dbend}}

\vspace{-15pt}

\hfill{\tiny 0406}\,{\scriptsize \myaudio{./audio/025_past_do_03.mp3}}

\end{frame}
%%%%%%%%%%%%%%%%%%%%%%%%%%%%%%%%%%
\begin{frame}[plain]{有声音と無声音}
\large
 \begin{description}
  \item[有声音: ] \onslide<1->{声帯がふるえる音}
                  \onslide<3->{%
                  \begin{itemize}\setbeamertemplate{items}[circle]\small
		   \item \onslide<3->{すべての母音}
		   \item \onslide<4->{一部の子音}\hspace{25pt}\onslide<5->{{\small たとえば \textipa{/b/}\,\,\,\textipa{/d/}\,\,\,\textipa{/g/}\,\,\,\textipa{/v/}\,\,\,\textipa{/z/}}}
		  \end{itemize}}
  \item[無声音: ] \onslide<2->{声帯がふるえない音}
                  \onslide<6->{%
                  \begin{itemize}\setbeamertemplate{items}[circle]\small
		   \item 一部の子音\hspace{25pt}\onslide<7->{{\small たとえば \textipa{/p/}\,\,\,\textipa{/t/}\,\,\,\textipa{/k/}\,\,\,\textipa{/f/}\,\,\,\textipa{/s/}}}
		  \end{itemize}}
 \end{description}
\end{frame}
%%%%%%%%%%%%%%%%%%%%%%%%%%%%%%%%%%
\againframe<17->[plain]{ed}
%%%%%%%%%%%%%%%%%%%%%%%%%%%%%%%%%
\begin{frame}[plain]{Exercises}
日本語の意味になるよう(~~~~~~~~~~)に下の枠から適当な動詞を選び補いましょう。必要に応じて、動詞の形は変化させてください 

\begin{enumerate}
	\item We~~(\visible<3->{~~listened~~}) to the radio \myEmph[2-]{NavyBlue}{last night}. {\scriptsize わたしたちは昨晩ラジオを聞いた}        
 \item I~~(\visible<5->{~~played~~}) the piano \myEmph[4-]{NavyBlue}{yesterday}.{\scriptsize わたしは昨日ピアノを弾いた}
	\item He~~(\visible<7->{~~cooked~~}) \myEmph[6-]{NavyBlue}{yesterday}.{\scriptsize 彼はきのう料理した}
	\item She~~(\visible<9->{~~walked~~}) to work \myEmph[8-]{NavyBlue}{last week}.{\scriptsize 先週彼女は職場に歩いて行った}
	\item They~~(\visible<11->{~~watched~~}) TV \myEmph[10-]{NavyBlue}{last night}.{\scriptsize 彼らは昨晩テレビを見た}
\end{enumerate}

\begin{tcolorbox}[title=この中から選んでください]
\centering
cook~~~~~/~~~~~listen~~~~~/~~~~~play~~~~~/~~~~~watch~~~~~/~~~~~walk
\end{tcolorbox}
\hfill{\tiny 0230}\,{\scriptsize \myaudio{./audio/025_past_do_04.mp3}}

\end{frame}
%%%%%%%%%%%%%%%%%%%%%%%%%%%
%subsubsection{過去形のつくり方-- eで終わる動詞 --}
\begin{frame}[plain,t]{過去形のつくり方 --- eで終わる動詞 ---}
 
\begin{enumerate}
 \item \begin{enumerate}
	\item We live in New York.\pause
	\item We \textcolor{ForestGreen}{\bfseries live}\textcolor{Maroon}{\bfseries d} in New York.\pause
       \end{enumerate}
 \item \begin{enumerate}
	\item She loves her cat.\pause
	\item She  \textcolor{ForestGreen}{\bfseries love}\textcolor{Maroon}{\bfseries d} her cat.\pause
       \end{enumerate} \item \begin{enumerate}
	\item I like tennis.\pause
	\item I  \textcolor{ForestGreen}{\bfseries like}\textcolor{Maroon}{\bfseries d} tennis.\pause
       \end{enumerate}
 \item \begin{enumerate}
	\item I use my phone every day.%
\hfill{\scriptsize phone \textipa{/f\'oUn/} 電話}
\pause
	\item I  \textcolor{ForestGreen}{\bfseries use}\textcolor{Maroon}{\bfseries d} the map.\hfill{{{\scriptsize map \textipa{/m\'\ae p/} 地図}}}\pause
       \end{enumerate}
\end{enumerate}

\begin{block}{Topic for Today}\small
\begin{itemize}\setbeamertemplate{items}[square]
 \item --eで終わる動詞は、原形にdだけつけます\\
\hfill{}{\scriptsize 結果的に--edで終わります}
\end{itemize}
\end{block}

\hfill{\tiny 0253}\,{\scriptsize \myaudio{./audio/025_past_do_05.mp3}}

\vspace{-.85\textheight}

\hfill
\pause
{\begin{tikzpicture}
\duck[tshirt,
jacket=gray,
bowtie,
cap=Maroon,
%crazyhair,
speech={\tiny dだけつける},
laughing,
signpost=\scalebox{0.45}{
\parbox{1.7cm}{
結果的に\\
\textbf{ed}}},
]
\end{tikzpicture}}


\end{frame}
%%%%%%%%%%%%%%%%%%%%%%%%%%%%%%%%
\begin{frame}[plain]{--eで終わる動詞の過去形}
 --eで終わる動詞の表を完成させましょう

\begin{center}
 
\rowcolors{2}{NavyBlue!50}{yellow!50}
\begin{tabular}{llll}\toprule
{\small 原形}&{\small (意味)}&{\small 過去形}&\visible<16->{{\small 発音}}\\\midrule
%\visible<1->{live}&\visible<2->{{\small (住む)}}&\visible<3->{lived}&\visible<16->{\textipa{/d/}}\\
\visible<1->{love}&\visible<2->{{\small (愛する)}}&\visible<3->{loved}&\visible<16->{\textipa{/d/}}\\
\visible<1->{like}&\visible<4->{{\small(好きだ)}}&\visible<5->{liked}&\visible<16->{\textipa{/t/}}\\
\visible<1->{use}&\visible<6->{{\small (使う)}}&\visible<7->{used}&\visible<16->{\textipa{/d/}}\\
\visible<1->{move}&\visible<8->{{\small (動かす)}}&\visible<9->{moved}&\visible<16->{\textipa{/d/}}\\
%\visible<1->{decide}&\visible<10->{{\small (決定する)}}&\visible<11->{decided}&\visible<18->{\textipa{/Id/}}\\
%\visible<1->{invite}&\visible<12->{{\small (招待する)}}&\visible<13->{invited}&\visible<18->{\textipa{/Id/}}\\\bottomrule
\end{tabular}%
\end{center}
 
\visible<14->{{\scriptsize --eで終わる動詞の過去形は--dだけつけます}}
\visible<15->{{\scriptsize (結果的に--edで終わることになります)}}

\visible<17->{{\scriptsize 動詞の最後の音が有声音なら\textipa{/d/}、無声音なら\textipa{/t/}}}

%\visible<19->{{\scriptsize ただし\textipa{/d/}または\textipa{/t/}のときは、\textipa{/Id/}となります}}%
\hfill{\tiny 0246}\,{\scriptsize \myaudio{./audio/025_past_do_06.mp3}}

\end{frame}
%%%%%%%%%%%%%%%%%%%%%%%%%%%%%
\begin{frame}[plain]{Exercises}

{\small つぎの(~~~~~~~~~~)に下の枠から適当な動詞を選び、過去形にして補いましょう}

\begin{enumerate}
	\item We~~(\visible<2->{~~lived~~}) in London two years ago.  {\small 2年前私たちはロンドンに住んでいた。}\\%
\hfill{\scriptsize London \textipa{/l\'\textturnv nd@n/}}    
 \item She~~(\visible<3->{~~loved~~}) her cat.{\small 彼女は飼い猫を愛していた}。
	\item He~~(\visible<4->{~~used~~}) the map.{\small 彼は地図を使った。}
%	\item I~~(\visible<5->{~~liked~~}) tennis.私はテニスが好きだった。
\end{enumerate}

\bigskip

\begin{tcolorbox}[title=この中から選んでください]
\centering
use~~~~~/~~~~~love~~~~~/~~~~~live
\end{tcolorbox}
\hfill{\tiny 0136}\,{\scriptsize\myaudio{./audio/025_past_do_07.mp3}}
\end{frame}
%%%%%%%%%%%%%%%%%%%%%%%
%subsubsection{--edで終わるが、ちょっぴり注意が必要な動詞}
\begin{frame}[plain]{--edで終わるが、ちょっぴり注意が必要な動詞}
\begin{enumerate}
 \item \begin{enumerate}
	\item<1-> I stud\textcolor{ForestGreen}{\bfseries y} math every day.
	\item<2-> I stud\textcolor{Maroon}{\bfseries ied} math.\hfill{}{{\scriptsize cf. studyの3単現: studies}}
       \end{enumerate}
 \item<1-> \begin{enumerate}
	\item<1-> Babies cr\textcolor{ForestGreen}{\bfseries y}.\hfill{}{{\scriptsize  babies: baby(赤ちゃん)の複数形}}
	\item<3-> The baby cr\textcolor{Maroon}{\bfseries ied} on the bus.
       \end{enumerate}
 \item<1-> \begin{enumerate}
	\item<1-> The train stops at every station.\hfill{}{{\scriptsize every $+$ 単数形: すべての~ \textipa{/\'evri/}}}
%\hfill{}{{\scriptsize train: 列車}}
	\item<4-> The train sto\textcolor{NavyBlue}{\bfseries pp}\textcolor{Maroon}{\bfseries ed} at every station.
       \end{enumerate}
\end{enumerate}

\begin{block}<5->{Topics for Today}\small
\begin{itemize}\setbeamertemplate{items}[square]
 \item studyとcryは最後のyをiに変えてedをつけます(結果的に--iedとなります)\\%
       \hfill{}{\scriptsize study\,$\rightarrow$\, studied\,\myCrossedOut{Maroon!50}{studyed}}\\
       \hfill{}{\scriptsize cry\,$\rightarrow$\, cried\,\myCrossedOut{Maroon!50}{cryed}}
 \item stopの過去形はpを重ねてedをつけます\\
       \hfill{}{\scriptsize stop\,$\rightarrow$\,stopped\,\myCrossedOut{Maroon!50}{stoped}}
\end{itemize}
\end{block}

\vspace{-10pt}

\hfill{\tiny 0247}\,{\scriptsize \myaudio{./audio/025_past_do_08.mp3}}

\end{frame}
%%%%%%%%%%%%%%%%%%%%%%%%%%%%%%%%%%%%%%
\begin{frame}[plain]{--edで終わるが、ちょっぴり注意が必要な動詞}
つぎの動詞の表を完成させましょう

\begin{center}
 
\rowcolors{2}{NavyBlue!50}{yellow!50}
\begin{tabular}{llll}\toprule
{\small 原形}&{\small (意味)}&{\small 過去形}&\visible<13->{\small 発音}\\\midrule
\visible<1->{study}&\visible<2->{{\small (勉強する)}}&\visible<3->{studied}&\visible<13->{\textipa{/d/}}\\
\visible<1->{cry}&\visible<4->{{\small (泣く)}}&\visible<5->{cried}&\visible<13->{\textipa{/d/}}\\
\visible<1->{try}&\visible<6->{{\small(試みる)}}&\visible<7->{tried}&\visible<13->{\textipa{/d/}}\\
\visible<1->{carry}&\visible<8->{{\small (運ぶ)}}&\visible<9->{carried}&\visible<13->{\textipa{/d/}}\\
\visible<1->{stop}&\visible<10->{{\small (止まる/止める)}}&\visible<11->{stopped}&\visible<13->{\textipa{/t/}}\\\bottomrule
\end{tabular}%
\end{center}
 
\visible<12->{{\scriptsize これらの動詞もすべて--edで終わっていることに注意}}

\visible<14->{{\scriptsize 動詞の最後の音が有声音なら\textipa{/d/}、無声音なら\textipa{/t/}}}
\hfill{\tiny 0304}\,{\scriptsize \myaudio{./audio/025_past_do_09.mp3}}

 \end{frame}
%%%%%%%%%%%%%%%%%%%%%%%%%%%%%%%%%%
\begin{frame}[plain]{Exercises}
つぎの(~~~~~~~~~~)に下の枠から適当な動詞を選び、過去形にして補いましょう。 

\begin{enumerate}
	\item I~~(\visible<2->{~~studied~~}) math last night.         
 \item The baby~~(\visible<3->{~~cried~~}) on the bus yesterday.
	\item The train (\visible<4->{~~stopped~~}) at every station two years ago.
\end{enumerate}

\begin{tcolorbox}[title=この中から選んでください]
\centering
cry (泣く)~~~~~/~~~~~stop (止まる)~~~~~/~~~~~study (勉強する)
\end{tcolorbox}
\hfill{\tiny 0148}\,{\scriptsize \myaudio{./audio/025_past_do_10.mp3}}

\end{frame}
%%%%%%%%%%%%%%%%%%%%%%%%%%%
\begin{frame}[plain]{ここに気をつけよう}
 
\dbend\,\,\,\,\,yで終わる動詞のときは

\begin{enumerate}
 \item<1-> \begin{enumerate}
	\item<1-> I study math every day.
	\item<2-> I studied math last night.
       \end{enumerate}
 \item<1-> \begin{enumerate}
	\item<1-> I play tennis on Sundays.
	\item<3-> I played tennis yesterday.
       \end{enumerate}
\end{enumerate}

\begin{block}<4->{yで終わる動詞の過去形}\small
 \begin{enumerate}
  \item<5-> \Circled[fill color = white]{子音字} $+$ y $\longrightarrow$ yをiに変えてedをつける%
\hfill{}stu\Circled[fill color=white]{d}y $\rightarrow$ stud\textcolor{NavyBlue}{\bfseries ied}\\
\hfill{}c\Circled[fill color=white]{r}y $\rightarrow$ cr\textcolor{NavyBlue}{\bfseries ied}\\
\hfill{}t\Circled[fill color=white]{r}y $\rightarrow$ tr\textcolor{NavyBlue}{\bfseries ied}\\
\hfill{}car\Circled[fill color=white]{r}y $\rightarrow$ carr\textcolor{NavyBlue}{\bfseries ied}
  \item<6-> \Circled[fill color = white]{母音字} $+$ y $\longrightarrow$ edをつける\hfill{}pl\Circled[fill color=white]{a}y $\rightarrow$ play\textcolor{Maroon}{\bfseries ed}\\
\hfill{}enj\Circled[fill color=white]{o}y $\rightarrow$ enjoy\textcolor{Maroon}{\bfseries ed}
 \end{enumerate}
\end{block}
\end{frame}
%%%%%%%%%%%%%%%%%%%%%%%%%%%
\subsubsection{不規則動詞}
\begin{frame}<1-11>[plain]{不規則動詞}
 次の各文の意味を考えましょう

 \begin{enumerate}
  \item He \alt<1-2>{went}{\textcolor{Maroon}{\bfseries went}} to London \alt<1>{two years ago}{\textcolor{NavyBlue}{\bfseries two years ago}}.\hfill{}{\small \textbf{went}: goの過去形\textipa{/w\'ent/}}
  \item She \alt<1-4>{came}{\textcolor{Maroon}{\bfseries came}} to the party \alt<1-3>{last night}{\textcolor{NavyBlue}{\bfseries last night}}.\hfill{}{\small \textbf{came}: comeの過去形\textipa{/k\'eIm/}}
  \item She \alt<1-6>{ate}{\textcolor{Maroon}{\bfseries ate}} bread \alt<1-5>{this morning}{\textcolor{NavyBlue}{\bfseries this morning}}.\hfill{}{\small \textbf{ate}: eatの過去形\textipa{/\'eIt/}}
  \item They \alt<1-8>{had}{\textcolor{Maroon}{\bfseries had}} a meeting \alt<1-7>{this afternoon}{\textcolor{NavyBlue}{\bfseries this afternoon}}.\hfill{}{\small \textbf{had}: haveの過去形\textipa{/h\'\ae d/}}
  \item He \myEmph[11-]{Maroon}{made} a cake for her \myEmph[10-]{NavyBlue}{yesterday}. \hfill{}{\small \textbf{made}: makeの過去形\textipa{/m\'eId/}}
\end{enumerate}
\hfill{\tiny 0231}\,{\scriptsize \myaudio{./audio/025_past_do_11.mp3}}

\end{frame}
%%%%%%%%%%%%%%%%%%%%%%%%%%%%%
\begin{frame}<1-11>[plain]{不規則動詞}
 次の各文の意味を考えましょう

 \begin{enumerate}%\setcounter{enumi}{5}
   \item We  \myEmph[3-]{Maroon}{saw} a great movie  \myEmph[2-]{NavyBlue}{last week}.\hfill{}{\small \textbf{saw}: seeの過去形\textipa{/s\'O:/}}
  \item She \myEmph[5-]{Maroon}{got} a new car  \myEmph[4-]{NavyBlue}{last month}.\hfill{}{\small \textbf{got}: getの過去形\textipa{/g\'At/}}
  \item He \myEmph[7-]{Maroon}{spoke} English  \myEmph[6-]{NavyBlue}{then}.\hfill{}{\small \textbf{spoke}: speakの過去形\textipa{/sp\'oUk/}}
  \item I \myEmph[9-]{Maroon}{took} a picture of my cat  \myEmph[8-]{NavyBlue}{yesterday}.\hfill{}{\small \textbf{took}: takeの過去形\textipa{/t\'Uk/}}
  \item I \myEmph[11-]{Maroon}{wrote} a letter to him  \myEmph[10-]{NavyBlue}{last night}.\hfill{}{\small \textbf{wrote}: writeの過去形\textipa{/r\'oUt/}}
\end{enumerate}
\hfill{\tiny 0236}\,{\scriptsize \myaudio{./audio/025_past_do_12.mp3}}

\end{frame}
%%%%%%%%%%%%%%%%%%%%%%%%%%%%%%%%%%%%
\begin{frame}[plain]{不規則動詞の表1}

\dbend
 
\begin{center}

\rowcolors{2}{NavyBlue!50}{yellow!50}
\begin{tabular}{llll}\toprule
{\small 原形}&{\small (意味)}&{\small 過去形}&{\small 発音}\\\midrule
\visible<1->{go}&\visible<2->{{\small (行く)}}&\visible<3->{went}&\visible<4->{\textipa{/w\'ent/}}\\
\visible<1->{come}&\visible<5->{{\small (来る)}}&\visible<6->{came}&\visible<7->{\textipa{/k\'eIm/}}\\
\visible<1->{eat}&\visible<8->{{\small(食べる)}}&\visible<9->{ate}&\visible<10->{\textipa{/\'eIt/}}\\
\visible<1->{have}&\visible<11->{{\small (持つ)}}&\visible<12->{had}&\visible<13->{\textipa{/h\'\ae d/}}\\
\visible<1->{make}&\visible<14->{{\small (作る)}}&\visible<15->{made}&\visible<16->{\textipa{/m\'eId/}}\\
\bottomrule
\end{tabular}%
\end{center}

{{\scriptsize これらの動詞は不規則な変化をします}}
\hfill{\tiny 0300}\,{\scriptsize \myaudio{./audio/025_past_do_13.mp3}
}
\end{frame}
%%%%%%%%%%%%%%%%%%%%%%%%%%%%%
\begin{frame}[plain]{不規則動詞の表2}

\dbend
 
\begin{center}
\rowcolors{2}{NavyBlue!50}{yellow!50}
\begin{tabular}{llll}\toprule
{\small 原形}&{\small (意味)}&{\small 過去形}&{\small 発音}\\\midrule
\visible<1->{see}&\visible<2->{{\small (見る)}}&\visible<3->{saw}&\visible<4->{\textipa{/s\'O:/}}\\
\visible<1->{get}&\visible<5->{{\small (手に入れる)}}&\visible<6->{got}&\visible<7->{\textipa{/g\'At/}}\\
\visible<1->{speak}&\visible<8->{{\small(話す)}}&\visible<9->{spoke}&\visible<10->{\textipa{/sp\'oUk/}}\\
\visible<1->{take}&\visible<11->{{\small (取る)}}&\visible<12->{took}&\visible<13->{\textipa{/t\'Uk/}}\\
\visible<1->{write}&\visible<14->{{\small (書く)}}&\visible<15->{wrote}&\visible<16->{\textipa{/r\'oUt/}}\\
\bottomrule
\end{tabular}%
\end{center}

{{\scriptsize これらの動詞は不規則な変化をします}}
\hfill{\tiny 0300}\,{\scriptsize \myaudio{./audio/025_past_do_14.mp3}}

\end{frame}
%%%%%%%%%%%%%%%%%%%%%%%%%%%%%
\begin{frame}<21>[plain,label=tableofirregularverb]{不規則動詞の表(1$+$2)}
 
\dbend

\begin{columns}
\begin{column}{.45\textwidth}
\raggedleft
\rowcolors{2}{NavyBlue!50}{yellow!50}
\begin{tabular}{lll}\toprule
{\small 原形}&{\small (意味)}&{\small 過去形}\\\midrule
\visible<1->{go}&\visible<2->{{\small (行く)}}&\visible<3->{went}\\
\visible<1->{come}&\visible<4->{{\small (来る)}}&\visible<5->{came}\\
\visible<1->{eat}&\visible<6->{{\small(食べる)}}&\visible<7->{ate}\\
\visible<1->{have}&\visible<8->{{\small (持つ)}}&\visible<9->{had}\\
\visible<1->{make}&\visible<10->{{\small (作る)}}&\visible<11->{made}\\
\bottomrule
\end{tabular}%
\end{column}
\begin{column}{.45\textwidth}
\raggedright
\rowcolors{2}{NavyBlue!50}{yellow!50}
\begin{tabular}{lll}\toprule
{\small 原形}&{\small (意味)}&{\small 過去形}\\\midrule
\visible<1->{see}&\visible<12->{{\small (見る)}}&\visible<13->{saw}\\
\visible<1->{get}&\visible<14->{{\small (手に入れる)}}&\visible<15->{got}\\
\visible<1->{speak}&\visible<16->{{\small(話す)}}&\visible<17->{spoke}\\
\visible<1->{take}&\visible<18->{{\small (取る)}}&\visible<19->{took}\\
\visible<1->{write}&\visible<20->{{\small (書く)}}&\visible<21->{wrote}\\
\bottomrule
\end{tabular}%
\end{column}
\end{columns}

\end{frame}
%%%%%%%%%%%%%%%%%%%%%%%%%%
\begin{frame}[plain]{Exercises}
あたえられた日本語の意味になるよう、カッコ内の語句を並べ替えましょう。ただし動詞は適当な形にしてください。なお、先頭の語は大文字で始めてください


\begin{enumerate}
 \item 彼は二年前にロンドンに行きました。
(  go / to / London / he )~~two years ago.\\
\visible<2->{He went to London two years ago.}
 \item 彼女は昨夜パーティーに来ました。\hspace{1\zw}
( come / the  party / to / she )~~last night.\\
\visible<3->{She came to the party last night.}
 \item 彼女は今朝パンを食べました。\hspace{2.65\zw}
( bread / eat /  she  )~~this morning.\\
\visible<4->{She ate bread this morning.}
 \item 彼らは今日の午後会議をしました。\hspace{.75\zw}
( have / they / a meeting )~~this afternoon.\\
\visible<5->{They had a meeting this afternoon.}
\end{enumerate}
\hfill{\tiny 0206}\,{\scriptsize \myaudio{./audio/025_past_do_15.mp3}}

\end{frame}
%%%%%%%%%%%%%%%%%%%%%%%%%%%%%%%%
\begin{frame}[plain]{Exercises}
あたえられた日本語の意味になるよう、カッコ内の語句を並べ替えましょう。ただし動詞は適当な形にしてください。なお、先頭の語は大文字で始めてください

\begin{enumerate}
 \item きのう彼は彼女のためにケーキを作った。\\
\mbox{}\hfill{}( a cake / make / for / he / her )~~yesterday.\\
\visible<2->{He made a cake for her yesterday.}
 \item 私たちは先週すばらしい映画を見た。
\hfill{}( see / we / a great movie )~~last week.\\
\visible<3->{We saw a great movie last week.}
 \item 先月、彼女は新しい車を手に入れた。
( get / she / car  / new / a )~~last month.\\
\visible<4->{She got a new car last month.}

\end{enumerate}
\hfill{\tiny 0144}\,{\scriptsize \myaudio{./audio/025_past_do_16.mp3}}

\end{frame}
%%%%%%%%%%%%%%%%%%%%%%%%%%%%%%%%%%
\begin{frame}[plain]{Exercises}
あたえられた日本語の意味になるよう、カッコ内の語句を並べ替えましょう。ただし動詞は適当な形にしてください。なお、先頭の語は大文字で始めてください


\begin{enumerate}
 \item 彼はそのとき英語を話しました。%
\hfill{}( speak / English  / he )~~then.\\
\visible<2->{He spoke English then.}
 \item 私はきのう猫の写真を撮りました。\hfill{\scriptsize A of B: BのA}\\
\mbox{}\hfill{}( a picture / I / take / of / my cat ) yesterday.\\
\visible<3->{I took a picture of my cat yesterday.}
 \item 昨夜、私は彼に手紙を書きました。
( a / to / write / I / letter / him ) last night.\\
\visible<4->{I wrote a letter to him last night.}
\end{enumerate}
\hfill{\tiny 0141}\,{\scriptsize \myaudio{./audio/025_past_do_17.mp3}}

\end{frame}
%%%%%%%%%%%%%%%%%%%%%%%%%%%%%%%
\section{まとめ}
\begin{frame}[plain]{一般動詞の過去形のまとめ}
 \begin{block}{過去のことは過去形で表します}\small
\pause
\begin{enumerate}
 \item 一般動詞の過去形は主語がなんであっても同じ\hfill{}{\scriptsize cf. be動詞の過去形 / 一般動詞の現在形}\pause
 \item 過去形のつくり方の原則$\longrightarrow$ 原形にedをつける\\\pause
でも、つぎの動詞に注意\pause
       \begin{enumerate}
	\item --eで終わる動詞はdだけつける\hfill{}lik\Circled[fill color=white]{e} $\rightarrow$ like\textcolor{OliveGreen}{\bfseries d} / lov\Circled[fill color=white]{e} $\rightarrow$ love\textcolor{OliveGreen}{\bfseries d}\pause
	\item --yで終わる動詞\pause
\begin{itemize}\setbeamertemplate{items}[circle]
	         \item \Circled[fill color = white]{\,子音字\,} $+$ y $\longrightarrow$ yをiに変えてedをつける%
\hfill{}stu\Circled[fill color=white]{d}y $\rightarrow$ stud\textcolor{NavyBlue}{\bfseries ied}\\
\hfill{}c\Circled[fill color=white]{r}y $\rightarrow$ cr\textcolor{NavyBlue}{\bfseries ied}\\
\hfill{}t\Circled[fill color=white]{r}y $\rightarrow$ tr\textcolor{NavyBlue}{\bfseries ied}\\
\hfill{}car\Circled[fill color=white]{r}y $\rightarrow$ carr\textcolor{NavyBlue}{\bfseries ied}\pause
  \item \Circled[fill color = white]{\,母音字\,} $+$ y $\longrightarrow$ edをつける\hfill{}pl\Circled[fill color=white]{a}y $\rightarrow$ play\textcolor{Maroon}{\bfseries ed}\\
\hfill{}enj\Circled[fill color=white]{o}y $\rightarrow$ enjoy\textcolor{Maroon}{\bfseries ed}\pause
	      \end{itemize}
	\item stopは最後のpを重ねてed\hfill{}stop $\longrightarrow$ sto\textcolor{BurntOrange}{\bfseries pped}
       \end{enumerate}\pause
 \item go, come, eat, have, makeなど不規則な変化をする動詞もあります(不規則動詞)
\end{enumerate}

\end{block}

\end{frame}
%%%%%%%%%%%%%%%%%%%%%%
\againframe<21>[plain]{tableofirregularverb}
\end{document}

