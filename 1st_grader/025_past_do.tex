\documentclass[aspectratio=169,xcolor={dvipsnames,table}]{beamer}
\usepackage[no-math,deluxe,haranoaji]{luatexja-preset}
\renewcommand{\kanjifamilydefault}{\gtdefault}
\renewcommand{\emph}[1]{{\upshape\bfseries #1}}
\usetheme{metropolis}
\metroset{block=fill}
\setbeamertemplate{navigation symbols}{}
\usecolortheme[rgb={0.7,0.2,0.2}]{structure}
%%%%%%%%%%%%%%%%%%%%%%%%%%%
\usepackage{media9}
%%%%%%%%%%%%%%%%%%%%%%%%%%%
%% さまざまなアイコン
%%%%%%%%%%%%%%%%%%%%%%%%%%%
\usepackage{fontawesome}
\usepackage{figchild}
\usepackage{twemojis}
\usepackage{utfsym}
\usepackage{bclogo}
\usepackage{marvosym}
\usepackage{fontmfizz}
\usepackage{pifont}
\usepackage{phaistos}
\usepackage{worldflags}
%%%%%%%%%%%%%%%%%%%%%%%%%%%
\usepackage{tikz}
\usetikzlibrary{backgrounds}
\usepackage{tcolorbox}
\usepackage{tikzpeople}
\usepackage{circledsteps}
\usepackage{xcolor}
\usepackage{amsmath}
\usepackage{booktabs}
%%%%%%%%%%%%%%%%%%%%%%%%%%%
%% 場合分け
\usepackage{cases}
%%%%%%%%%%%%%%%%%%%%%%%%%%%
% \myAnch{<名前>}{<色>}{<テキスト>}
% 指定のテキストを指定の色の四角枠で囲み, 指定の名前をもつTikZの
% ノードとして出力する. 図には remeber picture 属性を付けている
% ので外部から参照可能である.
\newcommand*{\myAnch}[3]{%
  \tikz[remember picture,baseline=(#1.base)]
    \node[draw,rectangle,#2] (#1) {\normalcolor #3};
}
%%%%%%%%%%%%%%%%%%%%%%%%%%%%
%% 音声リンク表示
\newcommand{\myaudio}[1]{\href{#1}{\faVolumeUp}}
%%%%%%%%%%%%%%%%%%%%%%%%%%%
% \myEmph コマンドの定義
%\newcommand{\myEmph}[3]{%
%    \textbf<#1>{\color<#1>{#2}{#3}}%
%}
\usepackage{xparse} % xparseパッケージの読み込み
\NewDocumentCommand{\myEmph}{O{} m m}{%
    \def\argOne{#1}%
    \ifx\argOne\empty
        \textbf{\color{#2}{#3}}% オプション引数が省略された場合
    \else
        \textbf<#1>{\color<#1>{#2}{#3}}% オプション引数が指定された場合
    \fi
}
%%%%%%%%%%%%%%%%%%%%%%%%%%%
%% 文末の上昇イントネーション記号 \myRisingPitch
%% 通常のイントネーション \myDownwardPitch
%% https://note.com/dan_oyama/n/n8be58e8797b2
%%%%%%%%%%%%%%%%%%%%%%%%%%%
\newcommand{\myRisingPitch}{
\begin{tikzpicture}[scale=0.3,baseline=0.3]
\draw[->,>=stealth] (0,0) to[bend right=45] (1,1);
\end{tikzpicture}
}
\newcommand{\myDownwardPitch}{
\begin{tikzpicture}[scale=0.3,baseline=0.3]
\draw[->,>=stealth] (0,1) to[bend left=45] (1,0);
\end{tikzpicture}
}
%%%%%%%%%%%%%%%%%%%%%%%%%%%
\title{English is fun.\,\,{}--- I played tennis yesterday. ---}
\author{}
\institute[]{}
\date[]

%%%%%%%%%%%%%%%%%%%%%%%%%%%%
%% TEXT
%%%%%%%%%%%%%%%%%%%%%%%%%%%%
\begin{document}
\begin{frame}[plain]
  \titlepage
\end{frame}

\section*{授業の流れ}
\begin{frame}[plain]
  \frametitle{授業の流れ}
  \tableofcontents
\end{frame}

\section{一般動詞の過去形}

\subsection{現在形(復習)}


\begin{frame}[plain]\frametitle{be動詞と一般動詞}
\begin{tikzpicture}

% グリッドを描画(5mm刻み)
%\draw[step=1cm, gray!20, very thin] (-6,0) grid (6,5);

\node[rectangle, rounded corners, draw=black, fill=white] (verb) at (-3,5) {動詞};



% 左側のノード(be動詞)
\node[circle, draw=black, line width=1pt, fill=yellow!30, minimum size=2cm] (be) at (-2.5,2.5) {be動詞{\footnotesize ($=$)}};

% 右側のノード(一般動詞)
\node[rectangle, draw=black, line width=1pt, fill=pink!30, minimum width=4cm, minimum height=1.5cm, inner sep=20pt] (general) at (4,2.5) {一般動詞\,\,{}$\left\{\begin{array}{ll}
\text{go}&\text{walk}\\
\text{have}&\text{like}\\
\text{speak}&\text{study}\\
\text{eat}& \text{drink}\\
\text{play}&\text{ほかにもたくさん}\\
\end{array}\right.$};


% 大きな長方形を描画して2つのノードを包む
\begin{scope}[on background layer]
\draw[draw=black!50, fill=blue!10, rounded corners, line width=1.5pt]
    ([xshift=-25pt,yshift=50pt]be.north west) rectangle ([xshift=8pt,yshift=-12pt]general.south east);
\end{scope}

\end{tikzpicture}
\end{frame}


\begin{frame}<1-9>[plain]\frametitle{Exercises}
つぎの空所に適当な動詞を選んで補って、現在のことを表す英文をつくってください。なお、必要があれば、形を変えてください

\begin{columns}
\begin{column}{.5\textwidth}
 % \setbeamercovered{transparent}
  \begin{enumerate}
   \item I (\onslide<2,8,9>{\textcolor{orange}{~have~}}) two cats.
   \item We (\onslide<3,8,9>{\textcolor{orange}{~go~}}) to school by bus.
   \item  She (\onslide<4,8,9>{\textcolor{orange}{~drinks~}}) coffee every morning.
   \item Tom (\onslide<5,8,9>{\textcolor{orange}{~eats~}}) bread for breakfast.
   \item Jennifer (\onslide<6,8,9>{\textcolor{orange}{~lives~}}) in Boston.
   \item They (\onslide<7,8,9>{\textcolor{orange}{~speak~}}) English.
  \end{enumerate}
\end{column}
\begin{column}{.45\textwidth}
\begin{tcolorbox}[title=この中から選んでください]
live, go, have, speak, drink, eat
\end{tcolorbox}
\end{column}
\end{columns}

% Embed the sound file
\onslide<9>{%
\myaudio{audio/004_verb_03.mp3}\,\,{}Listen carefully.(注意して聞いてください)

}
\end{frame}

\begin{frame}[plain]{一般動詞の現在形(復習)}
\begin{enumerate}
   \item<1-> \myEmph[9,17]{red}{I} \myEmph[9,17]{blue}{like} music. \onslide*<2>{\footnotesize  like: 好き、好む music: 音楽}
   \item<3-> \myEmph[10,17]{red}{We} \myEmph[10,17]{blue}{like} music.
   \item<4-> \myEmph[11,17]{red}{You} \myEmph[11,17]{blue}{like} music.
   \item<5-> \myEmph[12,18]{red}{He} \myEmph[12,18]{blue}{likes} music.
   \item<6-> \myEmph[13,18]{red}{She} \myEmph[13,18]{blue}{likes} music.
   \item<7-> \myEmph[14,18]{red}{Paul} \myEmph[14,18]{blue}{likes} music.
   \item<8-> \myEmph[15,19]{red}{They} \myEmph[15,19]{blue}{like} music.
  \end{enumerate}
\begin{exampleblock}<17->{Topics for Today}
\begin{itemize}
 \item<1-> \myEmph[17]{orange}{主語が1人称、2人称のとき、動詞はそのまま}
 \item<2-> \myEmph[18]{orange}{主語が3人称で単数のとき、---sとなります}\myEmph[19]{orange}{(複数ならそのままです)}
\end{itemize}
      \end{exampleblock}
\end{frame}


\begin{frame}<1-3>[plain]\frametitle{3単現のs}
\Large

\myEmph[2]{orange}{現在}を表す文では、
主語が\myEmph[2]{orange}{3人称}で\myEmph[2]{orange}{単数}のとき\\動詞の最後に\myEmph[2]{orange}{--s}をつけます

\pause

\mbox{}\hfill$\longrightarrow{}$これを\myAnch{T1}{orange}{3単現の`s'}といいます\pause

She play\myAnch{T2}{orange}{s} \,\,{}the guitar.

% ノード間を結ぶ矢印を別のTikZ環境で描く
\begin{tikzpicture}[remember picture,overlay]
\draw[->,orange] (T1.south)to[out=-120,in=-30](T2.south);
\end{tikzpicture}
\end{frame}

\subsection{過去形とは}
\begin{frame}[plain]{一般動詞の過去形}
 \Large

\begin{columns}
\begin{column}{.48\textwidth}
\begin{enumerate}
 \item 現在のこと
       \begin{enumerate}
	\item I play tennis every day.
	\item We play tennis every day.
	\item You play tennis every day.
	\item He play\textcolor{Maroon}{\bfseries s} tennis every day.
	\item They play tennis every day.
       \end{enumerate}
\end{enumerate}

{\small 現在形: playとplaysの使い分け}
\end{column}
\begin{column}{.48\textwidth}
\begin{enumerate}
\setcounter{enumi}{1} \item 過去のこと
        \begin{enumerate}
	\item I \textcolor{ForestGreen}{played} tennis every day.
	\item We \textcolor{ForestGreen}{played} tennis every day.
	\item You \textcolor{ForestGreen}{played} tennis every day.
	\item He \textcolor{ForestGreen}{played} tennis every day.
	\item They \textcolor{ForestGreen}{played} tennis every day.
       \end{enumerate}
\end{enumerate}

{\small 過去形: ぜんぶplayed}
\end{column}
\end{columns}

\begin{exampleblock}{Topics for Today}\small
\begin{itemize}
 \item 一般動詞の過去形は、主語にかかわらず同じ形
\end{itemize}
\end{exampleblock}
\end{frame}

\subsection{過去形のつくり方}
\subsubsection{基本 --ed}
\begin{frame}[plain]{過去形のつくり方--基本--}
 
\begin{enumerate}
 \item \begin{enumerate}
	\item We listen to the radio after dinner.
	\item We listen\textcolor{ForestGreen}{ed} to the radio after dinner.
       \end{enumerate}
 \item \begin{enumerate}
	\item I play the piano every day.
	\item I play\textcolor{ForestGreen}{ed} the piano last night.
       \end{enumerate}
 \item \begin{enumerate}
	\item He cooks every morning.
	\item He cook\textcolor{ForestGreen}{ed} this morning.
       \end{enumerate}

 \item \begin{enumerate}
	\item She walks to work every day.\hfill{}{\small work: 職場}
	\item She walk\textcolor{ForestGreen}{ed} to work every day.
       \end{enumerate}
\end{enumerate}

\begin{exampleblock}{Topics for Today}\small
\begin{itemize}
 \item 多くの動詞は、原形にedをつけることで過去形になります
\end{itemize}
\end{exampleblock}
\end{frame}

\begin{frame}[plain]{edをつける過去形}
 原形にedをつけて過去形をつくる動詞の表を完成させましょう。

\begin{center}
 
\rowcolors{2}{NavyBlue!50}{yellow!50}
\begin{tabular}{lll}\toprule
{\small 原形}&{\small (意味)}&{\small 過去形}\\\midrule
\visible<1->{listen}&\visible<2->{{\small (聴く)}}&\visible<3->{listened}\\
\visible<1->{play}&\visible<4->{{\small (演奏する、競技をする)}}&\visible<5->{played}\\
\visible<1->{cook}&\visible<6->{{\small(調理する)}}&\visible<7->{cooked}\\
\visible<1->{walk}&\visible<8->{{\small (歩く)}}&\visible<9->{walked}\\
\visible<1->{watch}&\visible<10->{{\small (見る)}}&\visible<11->{watched}\\
\visible<1->{enjoy}&\visible<12->{{\small (楽しむ)}}&\visible<13->{enjoyed}\\
\visible<1->{learn}&\visible<14->{{\small (学ぶ)}}&\visible<15->{learned}\\\bottomrule
\end{tabular}%
\end{center}

\visible<15->{多くの動詞が--edをつけると過去形になります}

\end{frame}

\begin{frame}[plain]{Exercises}
つぎの(~~~~~~~~~~)に下の枠から適当な動詞を選び補いましょう。必要に応じて、動詞の形は変化させてください。 

\begin{enumerate}
	\item We~~(\visible<3->{~~listened~~}) to the radio \myEmph[2-]{NavyBlue}{last night}.         
 \item I~~(\visible<5->{~~played~~}) the piano \myEmph[4-]{NavyBlue}{yesterday}.
	\item He~~(\visible<7->{~~cooked~~}) \myEmph[6-]{NavyBlue}{yesterday}.
	\item She~~(\visible<9->{~~walked~~}) to work \myEmph[8-]{NavyBlue}{last week}.\hfill{}{\small work: 職場}
	\item They~~(\visible<11->{~~watched~~}) TV \myEmph[10-]{NavyBlue}{last night}.
\end{enumerate}

\begin{tcolorbox}[title=この中から選んでください]
\centering
cook~~~~~/~~~~~listen~~~~~/~~~~~play~~~~~/~~~~~watch~~~~~/~~~~~walk
\end{tcolorbox}

\end{frame}


\subsubsection{過去形のつくり方--eで終わる動詞--}
\begin{frame}[plain]{過去形のつくり方--eで終わる動詞--}
 
\begin{enumerate}
 \item \begin{enumerate}
	\item We live in New York.
	\item We \textcolor{ForestGreen}{\bfseries live}\textcolor{Maroon}{\bfseries d} in New York.
       \end{enumerate}
 \item \begin{enumerate}
	\item She loves her cat.\hfill{}{\small work: 職場}
	\item She  \textcolor{ForestGreen}{\bfseries love}\textcolor{Maroon}{\bfseries d} her cat.
       \end{enumerate} \item \begin{enumerate}
	\item I like tennis.
	\item I  \textcolor{ForestGreen}{\bfseries like}\textcolor{Maroon}{\bfseries d} tennis.
       \end{enumerate}
 \item \begin{enumerate}
	\item I use my phone every day.
	\item I  \textcolor{ForestGreen}{\bfseries use}\textcolor{Maroon}{\bfseries d} the map.\hfill{{\small map: 地図}}
       \end{enumerate}


\end{enumerate}

\begin{exampleblock}{Topic for Today}\small
\begin{itemize}
 \item --eで終わる動詞は、原形にdだけつけます(結果的に--edで終わります)
\end{itemize}
\end{exampleblock}
\end{frame}


\begin{frame}[plain]{--eで終わる動詞の過去形}
 --eで終わる動詞の表を完成させましょう。

\begin{center}
 
\rowcolors{2}{NavyBlue!50}{yellow!50}
\begin{tabular}{lll}\toprule
{\small 原形}&{\small (意味)}&{\small 過去形}\\\midrule
\visible<1->{live}&\visible<2->{{\small (住む)}}&\visible<3->{lived}\\
\visible<1->{love}&\visible<4->{{\small (愛する)}}&\visible<5->{loved}\\
\visible<1->{like}&\visible<6->{{\small(好きだ)}}&\visible<7->{liked}\\
\visible<1->{use}&\visible<8->{{\small (使う)}}&\visible<9->{used}\\
\visible<1->{move}&\visible<10->{{\small (動かす)}}&\visible<11->{moved}\\
\visible<1->{invite}&\visible<12->{{\small (招待する)}}&\visible<13->{invited}\\
\end{tabular}%
\end{center}
 
\visible<13->{--eで終わる動詞の過去形は--dだけつけます}

 \visible<14->{結果的に--edで終わることになります}
\end{frame}


\begin{frame}[plain]{Exercises}
つぎの(~~~~~~~~~~)に下の枠から適当な動詞を選び、過去形にして補いましょう。 

\begin{enumerate}
	\item We~~(\visible<2->{~~lived~~}) in London two years ago.  2年前私たちはロンドンに住んでいた。       
 \item She~~(\visible<3->{~~loved~~}) her cat.彼女は飼い猫を愛していた。
	\item He~~(\visible<4->{~~used~~}) the map.彼は地図を使った。
%	\item I~~(\visible<5->{~~liked~~}) tennis.私はテニスが好きだった。
\end{enumerate}

\begin{tcolorbox}[title=この中から選んでください]
\centering
use~~~~~/~~~~~love~~~~~/~~~~~live
\end{tcolorbox}

\end{frame}


\subsubsection{--edで終わるが、ちょっぴり注意が必要な動詞}
\begin{frame}[plain]{--edで終わるが、ちょっぴり注意が必要な動詞}
 
\begin{enumerate}
 \item \begin{enumerate}
	\item I stud\textcolor{ForestGreen}{\bfseries y} math every day.
	\item I stud\textcolor{Maroon}{\bfseries ied} math..
       \end{enumerate}
 \item \begin{enumerate}
	\item Babies cr\textcolor{ForestGreen}{\bfseries y}.\hfill{}{{\small babies: baby(赤ちゃん)の複数形}}
	\item The baby cr\textcolor{Maroon}{\bfseries ied} on the bus.
       \end{enumerate}
 \item \begin{enumerate}
	\item The train stops at every satation.\hfill{}{{\small train: 列車}}
	\item The train sto\textcolor{NavyBlue}{\bfseries pp}\textcolor{Maroon}{\bfseries ed} at every station.
       \end{enumerate}
\end{enumerate}

\begin{exampleblock}{Topics for Today}\small
\begin{itemize}
 \item 最後のyをiに変えてedをつける動詞があります(結果的に--iedとなります)
 \item stopの過去形はpを重ねてedをつけます
\end{itemize}
\end{exampleblock}
\end{frame}

\begin{frame}[plain]{--edで終わるが、ちょっぴり注意が必要な動詞}
つぎの動詞の表を完成させましょう。

\begin{center}
 
\rowcolors{2}{NavyBlue!50}{yellow!50}
\begin{tabular}{lll}\toprule
{\small 原形}&{\small (意味)}&{\small 過去形}\\\midrule
\visible<1->{study}&\visible<2->{{\small (勉強する)}}&\visible<3->{studied}\\
\visible<1->{cry}&\visible<4->{{\small (泣く)}}&\visible<5->{cried}\\
\visible<1->{try}&\visible<6->{{\small(試みる)}}&\visible<7->{tried}\\
\visible<1->{carry}&\visible<8->{{\small (運ぶ)}}&\visible<9->{carried}\\
\visible<1->{stop}&\visible<10->{{\small (止まる/止める)}}&\visible<11->{stopped}\\
\end{tabular}%
\end{center}
 
\visible<12->{これらの動詞もすべて--edで終わっていることに注意}

 \end{frame}


\begin{frame}[plain]{Exercises}
つぎの(~~~~~~~~~~)に下の枠から適当な動詞を選び、過去形にして補いましょう。 

\begin{enumerate}
	\item I~~(\visible<2->{~~studied~~}) math last night.         
 \item The baby~~(\visible<3->{~~cried~~}) on the bus yesterday.
	\item The train (\visible<4->{~~stopped~~}) at every station two years ago.
\end{enumerate}

\begin{tcolorbox}[title=この中から選んでください]
\centering
cry (泣く)~~~~~/~~~~~stop (止まる)~~~~~/~~~~~study (勉強する)
\end{tcolorbox}

\end{frame}


\subsection{不規則動詞}
\begin{frame}<1-12>[plain]{不規則動詞}
 次の各文の意味を考えましょう。



 \begin{enumerate}
  \item He \alt<1-2>{went}{\textcolor{Maroon}{\bfseries went}} to London \alt<1>{two years ago}{\textcolor{NavyBlue}{\bfseries two years ago}}.\hfill{}{\small (went: goの過去形)}
  \item She \alt<1-4>{came}{\textcolor{Maroon}{\bfseries came}} to the party \alt<1-3>{last night}{\textcolor{NavyBlue}{\bfseries last night}}.\hfill{}{\small (came: comeの複数形)}
  \item She \alt<1-6>{ate}{\textcolor{Maroon}{\bfseries ate}} bread \alt<1-5>{this morning}{\textcolor{NavyBlue}{\bfseries this morning}}.\hfill{}{\small (ate: eatの過去形)}
  \item They \alt<1-8>{had}{\textcolor{Maroon}{\bfseries had}} a meeting \alt<1-7>{this afternoon}{\textcolor{NavyBlue}{\bfseries this afternoon}}.\hfill{}{\small (had: haveの過去形)}
  \item He \myEmph[11-]{Maroon}{made} a cake for her \myEmph[10-]{NavyBlue}{yesterday}. \hfill{}{\small (made: makeの過去形)}
\end{enumerate}

\end{frame}


\begin{frame}<1-12>[plain]{不規則動詞}
 次の各文の意味を考えましょう。



 \begin{enumerate}\setcounter{enumi}{5}
   \item We  \myEmph[3-]{Maroon}{saw} a great movie  \myEmph[2-]{NavyBlue}{last week}.\hfill{}{\small (saw: seeの過去形)}
  \item She \myEmph[5-]{Maroon}{got} a new car  \myEmph[4-]{NavyBlue}{last month}.\hfill{}{\small (got: getの過去形)}
  \item He \myEmph[7-]{Maroon}{spoke} English  \myEmph[6-]{NavyBlue}{then}.\hfill{}{\small (spoke: speakの過去形)}
  \item I \myEmph[9-]{Maroon}{took} a picture of my cat  \myEmph[8-]{NavyBlue}{yesterday}.\hfill{}{\small (took: takeの過去形)}
  \item I \myEmph[11-]{Maroon}{wrote} a letter to him  \myEmph[10-]{NavyBlue}{last night}.\hfill{}{\small (write: wroteの過去形)}
\end{enumerate}

\end{frame}

\begin{frame}[plain]{不規則動詞の表1}
 
\begin{center}
 
\rowcolors{2}{NavyBlue!50}{yellow!50}
\begin{tabular}{lll}\toprule
{\small 原形}&{\small (意味)}&{\small 過去形}\\\midrule
\visible<1->{go}&\visible<2->{{\small (行く)}}&\visible<3->{went}\\
\visible<1->{come}&\visible<4->{{\small (来る)}}&\visible<5->{came}\\
\visible<1->{eat}&\visible<6->{{\small(食べる)}}&\visible<7->{ate}\\
\visible<1->{have}&\visible<8->{{\small (持つ)}}&\visible<9->{had}\\
\visible<1->{make}&\visible<10->{{\small (作る)}}&\visible<11->{made}\\
\bottomrule
\end{tabular}%
\end{center}

これらの動詞の過去形は不規則なかたちをしています
\end{frame}


\begin{frame}[plain]{不規則動詞の表2}
 
\begin{center}
 
\rowcolors{2}{NavyBlue!50}{yellow!50}
\begin{tabular}{lll}\toprule
{\small 原形}&{\small (意味)}&{\small 過去形}\\\midrule
\visible<1->{see}&\visible<2->{{\small (見る)}}&\visible<3->{saw}\\
\visible<1->{get}&\visible<4->{{\small (手に入れる)}}&\visible<5->{got}\\
\visible<1->{speak}&\visible<6->{{\small(話す)}}&\visible<7->{spoke}\\
\visible<1->{take}&\visible<8->{{\small (取る)}}&\visible<9->{took}\\
\visible<1->{write}&\visible<10->{{\small (書く)}}&\visible<11->{wrote}\\
\bottomrule
\end{tabular}%
\end{center}

これらの動詞の過去形は不規則なかたちをしています
\end{frame}


\begin{frame}[plain]{不規則動詞の表(1$+$2)}
 
\begin{columns}
\begin{column}{.45\textwidth}
\raggedleft
\rowcolors{2}{NavyBlue!50}{yellow!50}
\begin{tabular}{lll}\toprule
{\small 原形}&{\small (意味)}&{\small 過去形}\\\midrule
\visible<1->{go}&\visible<2->{{\small (行く)}}&\visible<3->{went}\\
\visible<1->{come}&\visible<4->{{\small (来る)}}&\visible<5->{came}\\
\visible<1->{eat}&\visible<6->{{\small(食べる)}}&\visible<7->{ate}\\
\visible<1->{have}&\visible<8->{{\small (持つ)}}&\visible<9->{had}\\
\visible<1->{make}&\visible<10->{{\small (作る)}}&\visible<11->{made}\\
\bottomrule
\end{tabular}%
\end{column}
\begin{column}{.45\textwidth}
\raggedright
\rowcolors{2}{NavyBlue!50}{yellow!50}
\begin{tabular}{lll}\toprule
{\small 原形}&{\small (意味)}&{\small 過去形}\\\midrule
\visible<1->{see}&\visible<12->{{\small (見る)}}&\visible<13->{saw}\\
\visible<1->{get}&\visible<14->{{\small (手に入れる)}}&\visible<15->{got}\\
\visible<1->{speak}&\visible<16->{{\small(話す)}}&\visible<17->{spoke}\\
\visible<1->{take}&\visible<18->{{\small (取る)}}&\visible<19->{took}\\
\visible<1->{write}&\visible<20->{{\small (書く)}}&\visible<21->{wrote}\\
\bottomrule
\end{tabular}%
\end{column}
\end{columns}



\end{frame}

\begin{frame}[plain]{Exercises}
あたえられた日本語の意味になるよう、カッコ内の語句を並べ替えましょう。なお、先頭の語は大文字で始めてください。 


\begin{enumerate}
 \item 彼は二年前にロンドンに行きました。
(  went / to / London / he )~~two years ago.\\
\visible<2->{He went to London two years ago.}
 \item 彼女は昨夜パーティーに来ました。\hspace{1\zw}
( came / the  party / to / she )~~last night.\\
\visible<3->{She came to the party last night.}
 \item 彼女は今朝パンを食べました。\hspace{2.65\zw}
( bread / ate /  she  )~~this morning.\\
\visible<4->{She ate bread this morning.}
 \item 彼らは今日の午後会議をしました。\hspace{1\zw}
( had / they / a meeting )~~this afternoon.\\
\visible<5->{They had a meeting this afternoon.}
\end{enumerate}
\end{frame}


\begin{frame}[plain]{Exercises}
あたえられた日本語の意味になるよう、カッコ内の語句を並べ替えましょう。なお、先頭の語は大文字で始めてください。 


\begin{enumerate}
 \item きのう彼は彼女のためにケーキを作った。\\
\mbox{}\hfill{}( a cake / made / for her / he ) yesterday.\\
\visible<2->{He made a cake for her yesterday.}
 \item 私たちは先週すばらしい映画を見た。
\hfill{}( saw / we / a great movie ) last week.\\
\visible<3->{We saw a great movie last week.}
 \item 先月、彼女は新しい車を手に入れました。
( got / she / a new car ) last month.\\
\visible<4->{She got a new car last month.}

\end{enumerate}
\end{frame}


\begin{frame}[plain]{Exercises}
あたえられた日本語の意味になるよう、カッコ内の語句を並べ替えましょう。なお、先頭の語は大文字で始めてください。 


\begin{enumerate}
 \item 彼はそのとき英語を話しました。%
\hfill{}( spoke / English  / he )~~then.\\
\visible<2->{He spoke English then.}
 \item 私はきのう猫の写真を撮りました。\\
\mbox{}\hfill{}( a picture / I / took / of / my cat ) yesterday.\\
\visible<3->{I took a picture of my cat yesterday.}
 \item 昨夜、私は彼に手紙を書きました。
( a / to / wrote / I / letter / him ) last night.\\
\visible<4->{I wrote a letter to him last night.}
\end{enumerate}
\end{frame}

\begin{frame}[plain]{まとめ}
 \begin{exampleblock}{Topics for Today}\small
\begin{enumerate}
 \item 過去のことは過去形で表します
       \begin{enumerate}
	\item 過去形は主語がなんであっても同じです
       \end{enumerate}
 \item 過去形のつくり方の原則$\longrightarrow$ 原形に--edをつける\\
でも、つぎの動詞に注意
       \begin{enumerate}
	\item --eで終わる動詞は--dだけつける
	\item study, cryはyをiに変えて--ed。 例:$\longrightarrow$ studies cry$\longrightarrow$ cried
	\item stopは最後のpを重ねて--ed $\longrightarrow$ stopped
       \end{enumerate}
 \item go, come, eat, haveなど不規則な変化をする動詞もあります
\end{enumerate}


\end{exampleblock}

\end{frame}

\end{document}

