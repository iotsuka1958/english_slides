\documentclass[aspectratio=169,xcolor={dvipsnames,table}]{beamer}
\usepackage[no-math,deluxe,haranoaji]{luatexja-preset}
\renewcommand{\kanjifamilydefault}{\gtdefault}
\renewcommand{\emph}[1]{{\upshape\bfseries #1}}
\usetheme{metropolis}
\metroset{block=fill}
\setbeamertemplate{navigation symbols}{}
\usecolortheme[rgb={0.7,0.2,0.2}]{structure}
%%%%%%%%%%%%%%%%%%%%%%%%%%%
\usepackage{media9}
%%%%%%%%%%%%%%%%%%%%%%%%%%%
%% さまざまなアイコン
%%%%%%%%%%%%%%%%%%%%%%%%%%%
\usepackage{fontawesome}
\usepackage{figchild}
\usepackage{twemojis}
\usepackage{utfsym}
\usepackage{bclogo}
\usepackage{marvosym}
\usepackage{fontmfizz}
\usepackage{pifont}
\usepackage{phaistos}
\usepackage{worldflags}
%%%%%%%%%%%%%%%%%%%%%%%%%%%
\usepackage{tikz}
\usetikzlibrary{backgrounds}
\usepackage{tcolorbox}
\usepackage{circledsteps}
\usepackage{xcolor}
\usepackage{amsmath}
%%%%%%%%%%%%%%%%%%%%%%%%%%%
%% 場合分け
\usepackage{cases}
%%%%%%%%%%%%%%%%%%%%%%%%%%%
% \myAnch{<名前>}{<色>}{<テキスト>}
% 指定のテキストを指定の色の四角枠で囲み, 指定の名前をもつTikZの
% ノードとして出力する. 図には remeber picture 属性を付けている
% ので外部から参照可能である.
\newcommand*{\myAnch}[3]{%
  \tikz[remember picture,baseline=(#1.base)]
    \node[draw,rectangle,#2] (#1) {\normalcolor #3};
}
%%%%%%%%%%%%%%%%%%%%%%%%%%%%
%% 音声リンク表示
\newcommand{\myaudio}[1]{\href{#1}{\faVolumeUp}}
%%%%%%%%%%%%%%%%%%%%%%%%%%%
% \myEmph コマンドの定義
%\newcommand{\myEmph}[3]{%
%    \textbf<#1>{\color<#1>{#2}{#3}}%
%}
\usepackage{xparse} % xparseパッケージの読み込み
\NewDocumentCommand{\myEmph}{O{} m m}{%
    \def\argOne{#1}%
    \ifx\argOne\empty
        \textbf{\color{#2}{#3}}% オプション引数が省略された場合
    \else
        \textbf<#1>{\color<#1>{#2}{#3}}% オプション引数が指定された場合
    \fi
}
%%%%%%%%%%%%%%%%%%%%%%%%%%%
%% 文末の上昇イントネーション記号 \myRisingPitch
%% 通常のイントネーション \myDownwardPitch
%% https://note.com/dan_oyama/n/n8be58e8797b2
%%%%%%%%%%%%%%%%%%%%%%%%%%%
\newcommand{\myRisingPitch}{
\begin{tikzpicture}[scale=0.3,baseline=0.3]
\draw[->,>=stealth] (0,0) to[bend right=45] (1,1);
\end{tikzpicture}
}
\newcommand{\myDownwardPitch}{
\begin{tikzpicture}[scale=0.3,baseline=0.3]
\draw[->,>=stealth] (0,1) to[bend left=45] (1,0);
\end{tikzpicture}
}
%%%%%%%%%%%%%%%%%%%%%%%%%%%
\title{English is fun.\,\,{}--- Is she studying? ---}
\author{}
\institute[]{}
\date[]

%%%%%%%%%%%%%%%%%%%%%%%%%%%%
%% TEXT
%%%%%%%%%%%%%%%%%%%%%%%%%%%%
\begin{document}


\begin{frame}[plain]
  \titlepage
\end{frame}


\section*{授業の流れ}
\begin{frame}[plain]
  \frametitle{授業の流れ}
  \tableofcontents
\end{frame}



\section{be動詞の疑問文}

%\subsection{復習}
\begin{frame}[plain]{be動詞の疑問文}

つぎの文を疑問文にしましょう

 \begin{enumerate}
  \item<1-> I am late for the meeting.
        \onslide<5->{$\longrightarrow$ Am I late for the meeting?}
  \item<1-> You are hungry.
        \onslide<6->{$\longrightarrow$ Are you hungry?}
 \item<1-> They are happy.
        \onslide<7->{$\longrightarrow$ Are they happy?}
  \item<1-> She is in the kitchen.
        \onslide<8->{$\longrightarrow$ Is she in the kitchen?}
  \item<1-> Our teacher is busy.
        \onslide<9->{$\longrightarrow$ Is our teacher busy? }
 \end{enumerate}

\begin{exampleblock}<2->{Topics for Today}
be動詞の疑問文のつくり方
\begin{itemize}
 \item<3->  be動詞を先頭にする(主語はその後ろにきます)
 \item<4-> 文末に`?'をつける   イントネーションは\myRisingPitch
\end{itemize}
      \end{exampleblock}
\vspace{-10pt}
% Embed the sound file

\myaudio{./audio/023_is_ing_question_00.mp3}


\end{frame}


\section{現在進行形の疑問文}
\subsection{現在進行形の疑問文のつくり方}
\begin{frame}[plain]{現在進行形の疑問文}
\Large

\visible<1->{They \alt<1-3>{\myAnch{be1}{white}{are}}{\myAnch{BE1}{orange}{are}} studying.}\hfill\visible<2->{{\small study:勉強する}}

\vspace{15pt}

\visible<3->{\alt<1-3>{\myAnch{be2}{white}{Are}}{\myAnch{BE2}{orange}{Are}} they studying?} 

\visible<4->{%
\begin{tikzpicture}[remember picture,overlay]
 \draw[thick,orange,->] (be1.south) to[out=-90, in=90] (be2.north);
\end{tikzpicture}
}

\normalsize

\begin{exampleblock}<5->{Topics for Today}
現在進行形の疑問文のつくり方
\begin{itemize}
 \item<3->  be動詞を先頭にする(主語はその後ろにきます)
 \item<4-> 文末に`?'をつける   イントネーションは\myRisingPitch
\end{itemize}
      \end{exampleblock}

\mbox{}\hfill{\myaudio{./audio/023_is_ing_question_01.mp3}}
\end{frame}


\begin{frame}<1-5>[plain]\frametitle{Exercises}

日本文を参考にして(~~~~~~~~)の語を並べかえ、英文を完成させましょう。
ただし先頭に来る単語は大文字で始めてください。



\begin{enumerate}
 \item  ( watchin/  you  / TV / are ) now?\hspace{3\zw}%
あなたはいまテレビを見ていますか。\fcTelevision{0.025}{blue}{.5}\\
\visible<2->{$\longrightarrow$\,\,\,Are you watching TV now?}
 \item ( she / working/ is ) now ?
\hspace{5.3\zw}彼女はいま働いていますか。\\
\visible<3->{$\longrightarrow$\,\,\,Is she working now?}
 \item ( sleeping / John / is ) now?\hspace{4.8\zw}%
ジョンはいま寝ていますか。\raisebox{-10pt}{\scalebox{3}{\twemoji{sleepy face}}}\\
\visible<4->{$\longrightarrow$\,\,\,Is John sleeping now?}
\end{enumerate}

\mbox{}\hfill\visible<5>{\myaudio{./audio/023_is_ing_question_02.mp3}}
\end{frame}


\begin{frame}<1-6>[plain]\frametitle{Exercises}

次の英文の(~~~~~~~~)内から動詞の正しい形を選び、○で囲みましょう。

\begin{enumerate}
 \item ( Do / Does / \alt<2->{\Circled[outer color=orange]{Is}}{Is} / is ) she reading a book?.\hspace{10pt}\raisebox{0pt}{\bcbook}
 \item ( Do / Does / \alt<3->{\Circled[outer color=orange]{Are}}{Are} / is ) you waiting for the bus?\hspace{10pt}\scalebox{2.5}{\twemoji{bus stop}\twemoji{bus}}
 \item Are you ( \alt<4->{\Circled[outer color=orange]{listening}}{listening} / listen / playing ) to music now?
  \item ( Do / Does / \alt<5->{\Circled[outer color=orange]{Are}}{Are} / Is ) George and Eric singing in the hall now?\hspace{10pt}\raisebox{-5pt}{\rotatebox{0}{\scalebox{2.5}{\twemoji{musical notes}}}}
\end{enumerate}
\pause

\vfill


\mbox{}\hfill\visible{\myaudio{./audio/023_is_ing_question_03.mp3}}


\end{frame}

\subsection{疑問文への答え方}
\begin{frame}[plain]{疑問文への答え方}

\Large

\begin{columns}
\begin{column}{.425\textwidth}
\visible<1->{{\small be動詞の疑問文のとき}}

\visible<2->{Is he busy?}

\pause
\visible<3->{
\mbox{}\hfill$\left\{\begin{tabular}{l}
         \text{Yes, he is.}\\
         \visible<4->{\text{No, he is not.}}\\
         \visible<5->{\text{(}= \text{No, he's not.)}}\\
         \visible<6->{\text{(}= \text{No, he isn't.)}}
        \end{tabular}\right.$
}
\end{column}
\begin{column}{.425\textwidth}
\visible<7->{{\small 現在進行形のとき}}

\visible<8->{Is he studying?}

\visible<9->{%
\mbox{}\hfill$\left\{\begin{tabular}{l}
         \text{Yes, he is.}\\
         \visible<10->{\text{No, he is not.}}\\
         \visible<11->{\text{(}= \text{No, he's not.)}}\\
         \visible<12->{\text{(}= \text{No, he isn't.)}}
        \end{tabular}\right.$
}
\end{column}
\end{columns}

\vspace{-75pt}

\visible<13->{\mbox{}\hfill\textcolor{Orange}{\bfseries \begin{tabular}{c}
			   $\equiv$\\[-5pt]
                           {\footnotesize (同じ)}
			  \end{tabular}}\hfill\mbox{}}

\vspace{25pt}

\visible<14->{%
\begin{exampleblock}{Topic for Today\,\,\,{\textcolor{black}{\mdseries ---現在進行形の疑問文への答え方---}}}
\small

\begin{itemize}
 \item  be動詞の疑問文への答え方と同じ
\end{itemize}
      \end{exampleblock}
}
\vspace{-5pt}

\mbox{}\hfill{\myaudio{./audio/023_is_ing_question_04.mp3}}
\end{frame}


\begin{frame}[plain]{Exercises}
例にならって、質問に対する答えを「はい」と「いいえ」の2通りつくりましょう。

\begin{tabular}{rl@{\,\,\,}c@{\,\,\,}l@{\,\,\,}l}
\visible<1->{例}& \visible<1->{Is she eating lunch now?}\hspace{.5\zw}\fchamburger{0.05}{black}{.5}\fcFrenchFries{0.05}{black}{.5}& \visible<2->{$\rightarrow$}&\visible<3->{(1) Yes, she is.}&\visible<4->{(2) No, she is not.~~~\,}%
\footnote{\visible<5->{No, she's not.やNo, she isn't.も\scalebox{1.4}{\twemoji{OK button}}}}\\
\visible<1->{1}&\visible<1->{Are they studying in the library?\hspace{5pt}\raisebox{0pt}{\bcbook}}&\visible<6->{$\rightarrow$}&\visible<7->{(1) Yes, they are.}&\visible<8->{(2) No, they are not.}%
\footnote{\visible<9->{No, No they're not.やNo, they aren't.も\scalebox{1.4}{\twemoji{OK button}}}}\\
\visible<1->{2}&\visible<1->{Are you enjoying your vacation?}&\visible<10->{$\rightarrow$}& \visible<11->{(1) Yes, I am }&\visible<
12->{(2) No, I am not.~~~~\,\,\,\,}%
\footnote{\visible<13->{No, I'm not.も\scalebox{1.4}{\twemoji{OK button}}}\hspace{2\zw}\visible<14->{でもNo, I amn't.とはいいません。\scalebox{.5}{\bcbombe}}}\\
\visible<1->{3}&\visible<1->{Is he sleeping in his room?}&\visible<15->{$\rightarrow$}&\visible<16->{(1) Yes, he is.}&\visible<17->{(2) No, he is not.}\\
\visible<1->{4}&\visible<1->{Are Jack and Betty watching TV?\hspace{5pt}\raisebox{-5pt}{\fcTelevision{0.025}{blue}{.5}}}&\visible<18->{$\rightarrow$}&\visible<19->{(1) Yes, they are.}&\visible<20->{(2) No, they are not.}
\end{tabular}

\vfill

\mbox{}\hfill{\myaudio{./audio/023_is_ing_05.mp3}}
\end{frame}

%\subsection{What are you --ing?}
\begin{frame}[plain]{What are you --ing?}
\Large

\mbox{}\hspace{80pt}%
You are studying English now.


\pause

\mbox{}\hspace{55pt}%
Are you studying \alt<4->{\myAnch{FOCUS2}{orange}{English}}{\myAnch{focus2}{white}{English}} now?
\hspace{10pt}\visible<3->{{\normalsize Yes/Noで答える疑問文}}

\visible<5->{\myAnch{WH2}{orange}{What} are you studying now?}%
\hspace{2\zw}\visible<7->{{\small あなたはいまなにを勉強していますか。}}

\vspace{-5pt}

\visible<8->{\hspace{103pt}drinking}

\vspace{-5pt}

\visible<9->{\hspace{103pt}eating}

\vspace{-5pt}

\visible<10->{\hspace{103pt}{{\small ほかにもいろいろな動詞の--ing}}}


\visible<11->{%
\begin{exampleblock}{Topic for Today\,\,\,{\textcolor{black}{\mdseries ---What are you --ing?---}}}
\small

\begin{itemize}
 \item  動詞を変えることでいろいろな意味を表せます%
\end{itemize}
      \end{exampleblock}
}

\visible<6->{%
\begin{tikzpicture}[remember picture, overlay]
 \draw[thick, orange, ->] (focus2.south) to[out=-165, in=15] (WH2.north east);
\end{tikzpicture}
}

\vspace{-30pt}

\mbox{}\hfill{\myaudio{./audio/023_is_ing_06.mp3}}

\end{frame}
%%%%%%%%%%%%%%%%%%%%%%%
\begin{frame}[plain]{Exercises}
AとBとの対話になるように空所に適語を補いましょう。
\begin{enumerate}
 \item A: (\visible<2->{~~\textcolor{Orange}{What}~~}) are you studying?\\
       B: I am studying math.%
\hfill%
\visible<6->{$\text{study}+\text{ing}\rightarrow{}\text{studying}$}
 \item A: What are you watching?\\
       B: I (\visible<3->{~~\textcolor{Orange}{am}~~}) watching TV.%
\hfill%
\visible<7->{$\text{watch}+\text{ing}\rightarrow{}\text{watching}$}
 \item A: (\visible<4->{~~\textcolor{Orange}{What}~~}) is he making?\\
       B: He is making a cake now.%
\hfill%
\visible<8->{$\text{mak\textcolor{red}{\bfseries e}}+\text{ing}\rightarrow{}\text{making}$}

\item A: What  (\visible<5->{~~\textcolor{Orange}{are}~~})  you writing?\\
       B: I am wriitng  a letter.%
\hfill%
\visible<9->{$\text{writ\textcolor{red}{\bfseries e}}+\text{ing}\rightarrow{}\text{writing}$}
 \end{enumerate} 

\hfill%
\visible<10>{%
つづりが`e'で終わる動詞の--ing形に注意しよう}

\mbox{}\hfill{\myaudio{./audio/023_is_ing_07.mp3}}

\end{frame}


\begin{frame}[plain]{Exercises}
(~~~~)内の動詞を用いて、あたえられた日本語の意味になるよう英文を作りましょう。
なお、(~~~~)内の動詞は、必要があれば変化させてください。

\bigskip

\begin{tabular}{rll}
 1&あなたは何を食べているのですか。(eat) &\\
 &\visible<2->{What are you eating?} & \\
 2&彼女は何を飲んでいますか。(drink) &\\
 &\visible<3->{What is she drinking?} & \\
3&彼は何を作っていますか。(make) &\\
 &\visible<4->{What is he making?} & \\
4&彼らは何を書いていますか。(write) & \\
 &\visible<5->{What are they writing?} & \\
\end{tabular}

\mbox{}\hfill{\myaudio{./audio/023_is_ing_08.mp3}}

\end{frame}



\begin{frame}[plain]{What are you doing?}
\Large

What are you \begin{tabular}[t]{l@{\,}}
	      studying\\
              \visible<2->{drinking}\\
              \visible<3->{making}\\
              \visible<4->{eating}\\
              \multicolumn{1}{c}{\visible<5->{$\downarrow$}}\\
              \,\,\,\visible<6->{doing}
	     \end{tabular}
? 

\begin{exampleblock}<7->{Topic for Today}
\small

\begin{itemize}
 \item  What are you doing?
は「あなたは何をしていますか」という決まり文句として覚えよう
\end{itemize}
      \end{exampleblock}

\mbox{}\hfill{\myaudio{./audio/023_is_ing_09.mp3}}

\end{frame}



\begin{frame}[plain]{Exercises}
 あたえられた日本語の意味になるよう英文を作りましょう。

\begin{tabular}{rll}
 1&あなたは何をしていますか。 & \\
 &\visible<2->{What are you doing?} & \\
 2&彼女は何をしていますか。 & \\
 &\visible<3->{What is she doing?} & \\
3&彼は何をしていますか。 & \\
 &\visible<4->{What is he doing?} & \\
4&彼らは何をしていますか。 & \\
 &\visible<5->{What are they doing?} & \\
\end{tabular}

\mbox{}\hfill{\myaudio{./audio/023_is_ing_10.mp3}}

\end{frame}


\begin{frame}[plain]{Exercises}
Aの質問に対するBの答えがあたえられた内容になるよう英文を書きましょう。
\begin{enumerate}
 \item A: What are you doing?\hspace{2\zw}{\small [数学の勉強をしています]}\hfill$2(a+b)=2a+2b$\\
       B: \visible<2->{I am studying math.}%
\item A: What is he doing?\hspace{2\zw}{\small [夕食を作っています]}\hspace{2\zw}{}\raisebox{-5pt}{\scalebox{2}{\twemoji{cooking}\twemoji{spaghetti}\twemoji{carrot}\twemoji{leafy green}\twemoji{tomato}\twemoji{meat on bone}}}\\
       B: \visible<3->{He is cooking dinner}.%
 \item A: What are they doing?\hspace{2\zw}{\small [川(the river)で泳いでいます]}\hfill\raisebox{-5pt}{\scalebox{2}{\twemoji{man swimming}}}\\
       B: \visible<4->{They are swimming in the river.}%
\hfill%
\visible<6->{$\text{swi\textcolor{red}{m}}+\text{ing}\rightarrow{}\text{swi\textcolor{red}{mm}ing}}$
 \item A: What is she doing?\hspace{2\zw}{\small [公園(the park)で走っています]}\hfill\scalebox{2.5}{\twemoji{woman running}}\\
       B: \visible<5->{She is running in the park.}%
\hfill%
\visible<7->{$\text{ru\textcolor{red}{n}}+\text{ing}\rightarrow{}\text{ru\textcolor{red}{nn}ing}}$

 \end{enumerate} 

\mbox{}\hfill{\myaudio{./audio/023_is_ing_11.mp3}}

\end{frame}

%\subsection{Who is --ing?}
\begin{frame}[plain]{Who is --ing?}
\Large

\visible<1->{\alt<1>{\myAnch{jimmy}{white}{Jimmy}}{\myAnch{JIMMY}{orange}{Jimmy}} is playing the guitar.}\hspace{20pt}\scalebox{6}{\twemoji{guitar}}
\vspace{20pt}

\visible<3->{\alt<1-2>{\myAnch{who}{white}{Who}}{\myAnch{WHO}{orange}{Who}} is playing the guitar\alt<1-2>{\myAnch{qm}{white}{?}}{\myAnch{QM}{orange}{?}}}
\visible<5->{{\small だれがギターを演奏していますか}}

\begin{exampleblock}<6->{Topics for Today}
\small

\begin{itemize}
 \item 「だれが〜していますか」はWho is --ing?
 \item Jimmy is.のように答えます\mbox{}\hfill{\myaudio{./audio/023_is_ing_12.mp3}}

\end{itemize}
      \end{exampleblock}

\smallskip


\visible<4->{%
\begin{tikzpicture}[remember picture, overlay]
\draw[thick,orange,->] (jimmy.south) to (who.north);
\end{tikzpicture}
}
\end{frame}


\begin{frame}[plain]{Exercises}
 あたえられた日本語の意味になるよう英文を作りましょう。

\begin{tabular}{rll}
 1&だれがその歌を歌っていますか。\scalebox{2.5}{\twemoji{woman singer}} & \\
 &\visible<2->{Who is singing the song?} & \\
2&だれがピアノを演奏していますか。\scalebox{2}{\twemoji{musical score}\,\,\twemoji{musical keyboard}} & \\
 &\visible<3->{Who is playing the guitar?} & \\
 2&だれがテレビ(TV)を見ていますか。\scalebox{2.5}{\twemoji{television}} & \\
 &\visible<4->{Who is watching TV?} & \\
4&だれがあそこに座っていますか。 \raisebox{-5pt}{\fcChairA{0.05}{black}{.2}} &{\small (あそこに: over there)} \\
 &\visible<5->{Who is sitting over there?} & $\text{si\textcolor{red}{t}}+\text{ing}\rightarrow{}\text{si\textcolor{red}{tt}ing}$\\
\end{tabular}

\mbox{}\hfill{\myaudio{./audio/023_is_ing_13.mp3}}

\end{frame}

\begin{frame}[plain]{hoge}
 \fcChairA{0.05}{black}{.2}
\end{frame}

\end{document}
