\documentclass[aspectratio=169,xcolor={dvipsnames,table}]{beamer}
\usepackage[no-math,deluxe,haranoaji]{luatexja-preset}
\renewcommand{\kanjifamilydefault}{\gtdefault}
\renewcommand{\emph}[1]{{\upshape\bfseries #1}}
\usetheme{metropolis}
\metroset{block=fill}
\setbeamertemplate{navigation symbols}{}
\setbeamertemplate{blocks}[rounded][shadow=false]
\usecolortheme[rgb={0.7,0.2,0.2}]{structure}
%%%%%%%%%%%%%%%%%%%%%%%%%%%
\usepackage{media9}
%%%%%%%%%%%%%%%%%%%%%%%%%%%
%% さまざまなアイコン
%%%%%%%%%%%%%%%%%%%%%%%%%%%
\usepackage{fontawesome}
%\usepackage{figchild}
\usepackage{twemojis}
\usepackage{utfsym}
\usepackage{bclogo}
\usepackage{marvosym}
\usepackage{fontmfizz}
\usepackage{pifont}
\usepackage{phaistos}
\usepackage{worldflags}
%%%%%%%%%%%%%%%%%%%%%%%%%%%
\usepackage{tikz}
\usetikzlibrary{backgrounds}
\usepackage{tcolorbox}
\usepackage{circledsteps}
\usepackage{xcolor}
\usepackage{amsmath}
\usepackage{pxrubrica}
\usepackage{tipa}
%%%%%%%%%%%%%%%%%%%%%%%%%%%
%% 場合分け
\usepackage{cases}
%%%%%%%%%%%%%%%%%%%%%%%%%%%
% \myAnch{<名前>}{<色>}{<テキスト>}
% 指定のテキストを指定の色の四角枠で囲み, 指定の名前をもつTikZの
% ノードとして出力する. 図には remeber picture 属性を付けている
% ので外部から参照可能である.
\newcommand*{\myAnch}[3]{%
  \tikz[remember picture,baseline=(#1.base)]
    \node[draw,rectangle,#2] (#1) {\normalcolor #3};
}
%%%%%%%%%%%%%%%%%%%%%%%%%%%%
%% 音声リンク表示
\newcommand{\myaudio}[1]{\href{#1}{\faVolumeUp}}
%%%%%%%%%%%%%%%%%%%%%%%%%%%
% \myEmph コマンドの定義
%\newcommand{\myEmph}[3]{%
%    \textbf<#1>{\color<#1>{#2}{#3}}%
%}
\usepackage{xparse} % xparseパッケージの読み込み
\NewDocumentCommand{\myEmph}{O{} m m}{%
    \def\argOne{#1}%
    \ifx\argOne\empty
        \textbf{\color{#2}{#3}}% オプション引数が省略された場合
    \else
        \textbf<#1>{\color<#1>{#2}{#3}}% オプション引数が指定された場合
    \fi
}
%%%%%%%%%%%%%%%%%%%%%%%%%%%
%% 文末の上昇イントネーション記号 \myRisingPitch
%% 通常のイントネーション \myDownwardPitch
%% https://note.com/dan_oyama/n/n8be58e8797b2
%%%%%%%%%%%%%%%%%%%%%%%%%%%
\newcommand{\myRisingPitch}{
\begin{tikzpicture}[scale=0.3,baseline=0.3]
\draw[->,>=stealth] (0,0) to[bend right=45] (1,1);
\end{tikzpicture}
}
\newcommand{\myDownwardPitch}{
\begin{tikzpicture}[scale=0.3,baseline=0.3]
\draw[->,>=stealth] (0,1) to[bend left=45] (1,0);
\end{tikzpicture}
}
%%%%%%%%%%%%%%%%%%%%%%%%%%%
\title{English is fun.}
\subtitle{Is she tired? Is she studying?}
\author{}
\institute[]{}
\date[]

%%%%%%%%%%%%%%%%%%%%%%%%%%%%
%% TEXT
%%%%%%%%%%%%%%%%%%%%%%%%%%%%
\begin{document}


\begin{frame}[plain]
  \titlepage
\end{frame}

\section*{授業の流れ}
\begin{frame}[plain]
  \frametitle{授業の流れ}
  \tableofcontents
\end{frame}
%%%%%%%%%%%%%%%%%%%%%%%%%%%%%%
\section{be動詞の疑問文(復習)}
%%%%%%%%%%%%%%%%%%%%%%%%%%%
%%%%%%%%%%%%%%%%%%%%%%%%%%%%%%%
\begin{frame}[plain,t]{be動詞の疑問文}
 \large

疑問文のつくり方%
\hfill{\tiny 0057}\,{\scriptsize \myaudio{./audio/008_question_be_02.mp3}}


\vspace{10pt}

\myAnch{s-1}{orange}{You} \myAnch{be-1}{olive}{are} happy.\scalebox{2}{\myDownwardPitch}
\vspace{20pt}

\visible<2->{\myAnch{be-2}{olive}{Are} \myAnch{s-2}{orange}{you} happy\myAnch{question}{orange}{?}}\visible<5->{\scalebox{2}{\myRisingPitch}}

\visible<4->{\small be動詞を先頭に}\hspace{52pt}\visible<5->{\small ?と最後のイントネーションに注意}

\visible<3->{\begin{tikzpicture}[remember picture, overlay]
%\draw [help lines] (0,0) grid (10,4);%(0,0)から(10,4)までの"細線の方眼"
 \draw[line width=2pt, opacity=.5, orange, ->] (s-1.south) to[out=-90, in=90] (s-2.north);
 \draw[line width=2pt, opacity=.5, olive, ->] (be-1.south) to[out=-90, in=90] (be-2.north);
\end{tikzpicture}}

\begin{block}<6->{Topics for Today}\small
\begin{itemize}\setbeamertemplate{items}[square]
 \item \ \textbf{be}動詞を先頭にする(主語はその後ろ)\\
      \hfill\Circled[fill color = white]{\,\,$\text{S}+\text{be動詞}\,\,$\,\,\ldots\,\,\,}\,\,.\,\,$\longrightarrow$\,\,\,\,\Circled[fill color = white]{\,\,$\text{be動詞}+\text{S}\,\,$\,\,\ldots\,\,\,}\,\,?
 \item 文末に`?'をつける   イントネーションは\myRisingPitch
\end{itemize}
\end{block}
\end{frame}
%%%%%%%%%%%%%%%%%%%%%%%%%%%%%%%
%%%%%%%%%%%%%%%%%%%%%%%%%
\begin{frame}[plain]\frametitle{Exercises}
つぎの各文を疑問文にしましょう%
\hfill{\tiny 0340}\,{\scriptsize\myaudio{./audio/008_question_be_05.mp3}}

 \begin{enumerate}
  \item I am late for the meeting.
        \onslide<2->{$\longrightarrow$ Am I late for the meeting?}%
	\hfill{\scriptsize be late for ~に遅れる}
  \item You are hungry.
        \onslide<3->{$\longrightarrow$ Are you hungry?}%
	\hfill{\scriptsize hungry \textipa{/h\'2Ngri/} 空腹の}
  \item They are happy.
        \onslide<4->{$\longrightarrow$ Are they happy?}
  \item She is in the kitchen.
        \onslide<5->{$\longrightarrow$ Is she in the kitchen?}%
	\hfill{\scriptsize kitchen \textipa{/k\'ItS@n/} 台所、キッチン}
  \item Our teacher is busy.
        \onslide<6->{$\longrightarrow$ Is our teacher busy?}%
	\hfill{\scriptsize busy \textipa{/b\'Izi/} 忙しい}
 \end{enumerate}

\begin{block}{Topics for Today}\small
be動詞の疑問文のつくり方
\begin{itemize}\setbeamertemplate{items}[square]
 \item\ \textbf{be}動詞を先頭にする(主語はその後ろ)\\
      \hfill\Circled[fill color = white]{\,\,$\text{S}+\text{be動詞}\,\,$\,\,\ldots\,\,\,}\,\,.\,\,$\longrightarrow$\,\,\,\,\Circled[fill color = white]{\,\,$\text{be動詞}+\text{S}\,\,$\,\,\ldots\,\,\,}\,\,?
 \item 文末に`?'をつける   イントネーションは\myRisingPitch
\end{itemize}
      \end{block}
\end{frame}
%%%%%%%%%%%%%%%%%%%%%%%%%%%%%%%
\section{現在進行形の疑問文}
%%%%%%%%%%%%%%%%%%%%%%%%%%%%%%%
%\subsection{現在進行形の疑問文のつくり方}
%%%%%%%%%%%%%%%%%%%%%%%%%%%%%%
\begin{frame}[plain]{現在進行形の疑問文}
\large

\visible<1->{They \alt<1>{\myAnch{be1}{white}{are}}{\myAnch{BE1}{orange}{are}} studying.}\hfill\visible<1->{{\scriptsize study \textipa{/st\'2di/} 勉強する}}

\vspace{15pt}

\visible<2->{\alt<1>{\myAnch{be2}{white}{Are}}{\myAnch{BE2}{orange}{Are}} they studying?} 
%
\begin{tikzpicture}[remember picture,overlay]
 \visible<2->{\draw[thick,orange,->] (be1.south) to[out=-90, in=90] (be2.north);}
\end{tikzpicture}

\begin{block}<3->{Topics for Today}\small
現在進行形の疑問文のつくり方
\begin{itemize}\setbeamertemplate{items}[square]
 \item  be動詞を先頭にする(主語はその後ろ)\\
\hfill\Circled[fill color = white]{\,\,$\text{S}+\text{be動詞}+\text{---ing}\,\,$\,\,\ldots\,\,\,}\,\,.\,\,$\longrightarrow$\,\,\,\,\Circled[fill color = white]{\,\,$\text{be動詞}+\text{S}+\text{---ing}\,\,$\,\,\ldots\,\,\,}\,\,? 
 \item 文末に`?'をつける   イントネーションは\myRisingPitch
\end{itemize}
\end{block}

\hfill{}\visible<4->{{\scriptsize be動詞の疑問文のときと同じです}}

\mbox{}\hfill{\tiny 0108}\,{\scriptsize \myaudio{./audio/023_is_ing_question_01.mp3}}
\end{frame}
%%%%%%%%%%%%%%%%%%%%%%%%%%%%%%%%%%%%%%%
\begin{frame}<1-4>[plain]\frametitle{Exercises}

{\small 日本文を参考にして(~~~~~~~~)の語を並べかえ、英文を完成させましょう。
ただし先頭に来る単語は大文字で始めてください}

\begin{enumerate}
 \item  ( watching /  you  / TV / are ) now?\hfill%
あなたはいまテレビを見ていますか。\\%\fcTelevision{0.025}{blue}{.5}\\
\visible<2->{$\longrightarrow$\,\,\,Are you watching TV now?}
 \item ( she / working/ is ) now ?
\hspace{5.3\zw}彼女はいま働いていますか。\\
\visible<3->{$\longrightarrow$\,\,\,Is she working now?}
 \item ( sleeping / John / is ) now?\hspace{4.8\zw}%
ジョンはいま寝ていますか。\raisebox{-10pt}{\scalebox{3}{\twemoji{sleepy face}}}\\
\visible<4->{$\longrightarrow$\,\,\,Is John sleeping now?}
\end{enumerate}

\mbox{}\hfill{\tiny 0134}\,{\scriptsize \myaudio{./audio/023_is_ing_question_02.mp3}}
\end{frame}
%%%%%%%%%%%%%%%%%%%%%%%%%%%%%%%%%%%%%%
\begin{frame}<1-7>[plain]\frametitle{Exercises}

{\small 次の英文の(~~~~~~~~)内から動詞の正しい形を選び、○で囲みましょう}

\begin{enumerate}
 \item ( Do / Does / \alt<2->{\Circled[outer color=orange]{Is}}{Is} ) she reading a book?\hspace{10pt}\raisebox{0pt}{\bcbook}
 \item ( Do / Does / \alt<3->{\Circled[outer color=orange]{Are}}{Are} ) you waiting for the bus?\hspace{10pt}\scalebox{2.5}{\twemoji{bus stop}\twemoji{bus}}\hfill{\scriptsize wait for ~を待つ}
 \item Are you ( playing / \alt<4->{\Circled[outer color=orange]{listening}}{listening} / listen ) \myEmph[4-]{Maroon}{to} music now?\hfill{\scriptsize listen to ~を聞く}\\
\hfill\visible<5->{{\scriptsize cf. Are you playing music now?}}
  \item ( Do / Does / \alt<6->{\Circled[outer color=orange]{Are}}{Are} / Is ) George and Eric singing in the hall now?\hspace{10pt}\raisebox{-5pt}{\rotatebox{0}{\scalebox{2.5}{\twemoji{musical notes}}}}\\
\hfill\visible<7->{\scriptsize cf. Are \textbf{they} singing in the hall now?}
\end{enumerate}
\pause

\vfill

\mbox{}\hfill{\tiny 0200}\,{\scriptsize \myaudio{./audio/023_is_ing_question_03.mp3}}
\end{frame}
%%%%%%%%%%%%%%%%%%%%%%%%%%%%%%%%%%
\section{現在進行形の疑問文への答え方}
%%%%%%%%%%%%%%%%%%%%%%%%%%%%%%%%%%
\begin{frame}[plain]{疑問文への答え方}

\large

\begin{columns}
\begin{column}{.425\textwidth}
\visible<1->{{\small be動詞の疑問文のとき(復習)}}

\visible<2->{Is he busy?}

\pause
\visible<3->{
\mbox{}\hfill$\left\{\begin{tabular}{l}
         \text{Yes, he is.}\\
         \visible<4->{\text{No, \textcolor<5>{Maroon}{\textbf<5>{he}} \textcolor<5,6>{Maroon}{\textbf<5,6>{is}} \textcolor<6>{Maroon}{\textbf<6>{not}}.}}\\
         \visible<5->{\text{(}= \text{No, \textcolor<5>{Maroon}{\textbf<5>{he's}} not.)}}\\
         \visible<6->{\text{(}= \text{No, he \textcolor<6>{Maroon}{\textbf<6>{isn't}}.)}}
        \end{tabular}\right.$
}
\end{column}
\begin{column}{.425\textwidth}
\visible<7->{{\small 現在進行形の疑問文のとき}}

\visible<8->{Is he studying?}

\visible<9->{%
\mbox{}\hfill$\left\{\begin{tabular}{l}
         \text{Yes, he is.}\\
         \visible<10->{\text{No, \textcolor<11>{Maroon}{\textbf<11>{he}} \textcolor<11,12>{Maroon}{\textbf<11,12>{is}} \textcolor<12>{Maroon}{\textbf<12>{not}}.}}\\
         \visible<11->{\text{(}= \text{No, \textcolor<11>{Maroon}{\textbf<11>{he's}} not.)}}\\
         \visible<12->{\text{(}= \text{No, he \textcolor<12>{Maroon}{\textbf<12>{isn't}}.)}}
        \end{tabular}\right.$
}
\end{column}
\end{columns}

\vspace{-50pt}

\visible<13->{\mbox{}\hfill\textcolor{Orange}{\bfseries \begin{tabular}{c}
			   $\equiv$\\[-5pt]
                           {\footnotesize (同じ)}
			  \end{tabular}}\hfill\mbox{}}

\vspace{15pt}

\begin{block}<14->{Topics for Today\,\,\,}\small

\begin{itemize}\setbeamertemplate{items}[square]
 \item 「be動詞の疑問文」への答え方と「現在進行形の疑問文」への答え方は同じ
 \item 「いいえ」のときは\kenten{短縮形}が使われることがあります
         \begin{enumerate}\setbeamertemplate{items}[circle]
	  \item \Circled[fill color = white]{\,\,$\text{S}+\text{be動詞}$\,\,}\,\,がまとまる場合\hfill{}{\scriptsize I'm not, you're not, she's not, it's not,  they're not}
	  \item \Circled[fill color = white]{\,\,$\text{be動詞}+\text{not}$\,\,}\,\,がまとまる場合\hfill{}{\scriptsize aren't, isn't\,\,(でもamn'tはない)}
	 \end{enumerate}
\end{itemize}
      \end{block}

\vspace{-15pt}

\mbox{}\hfill{\tiny 0321}\,{\scriptsize \myaudio{./audio/023_is_ing_question_04.mp3}}
\end{frame}

%%%%%%%%%%%%%%%%%%%%%%%%%%%%%%%%%%%%%%%%%%%%%%
\begin{frame}[plain]{Exercises}

{\small 例にならって、質問に対する答えを「はい」と「いいえ」の2通りつくりましょう。短縮形も可}

\noindent\begin{tabular}{@{}rl@{\,\,\,}c@{\,\,\,}l@{\,\,\,}l}
\visible<1->{例}& \visible<1->{Is she eating lunch now?}%\hspace{.5\zw}\fchamburger{0.05}{black}{.5}\fcFrenchFries{0.05}{black}{.5}
& \visible<2->{$\rightarrow$}&\visible<3->{(1) Yes, she is.}&\visible<4->{(2) No, she is not.~~~\,}%
\footnote{\visible<5->{No, she's not.やNo, she isn't.も\scalebox{1.4}{\twemoji{OK button}}}}\\
\visible<1->{1}&\visible<1->{Are they studying in the library?\hspace{5pt}\raisebox{0pt}{\bcbook}}&\visible<6->{$\rightarrow$}&\visible<7->{(1) Yes, they are.}&\visible<8->{(2) No, they are not.}%
\footnote{\visible<9->{No, they're not.やNo, they aren't.も\scalebox{1.4}{\twemoji{OK button}}}}\\
\visible<1->{2}&\visible<1->{Are you enjoying your vacation?}&\visible<10->{$\rightarrow$}& \visible<11->{(1) Yes, I am.}&
\visible<12->{(2) No, I am not.~~~~\,\,\,\,}%
\footnote{\visible<13->{No, I'm not.も\scalebox{1.4}{\twemoji{OK button}}}\hspace{2\zw}\visible<14->{でもNo, I amn't.とはいいません。\scalebox{.5}{\bcbombe}}}\\
\visible<1->{3}&\visible<1->{Is he sleeping in his room?}&\visible<15->{$\rightarrow$}&\visible<16->{(1) Yes, he is.}&\visible<17->{(2) No, he is not.}\\
\visible<1->{4}&\visible<1->{Are Jack and Betty watching TV?}%\hspace{5pt}\raisebox{-5pt}{\fcTelevision{0.025}{blue}{.5}}
&\visible<18->{$\rightarrow$}&\visible<19->{(1) Yes, they are.}&\visible<20->{(2) No, they are not.}
\end{tabular}

\vfill

\mbox{}\hfill{\tiny 0401}\,{\scriptsize \myaudio{./audio/023_is_ing_question_05.mp3}}
\end{frame}
%%%%%%%%%%%%%%%%%%%%%%%%%%%%%%%%%%%%%%%
%%%%%%%%%%%%%%%%%%%%%%%%%%%%%%%%%%%%%%
\section{現在進行形のまとめ}
\begin{frame}[plain]{要点}
 
\begin{block}{現在進行形$=$\,\,\Circled[fill color=white]{\,be動詞の現在形 $+$ --ing\,}}
\begin{itemize}\setbeamertemplate{items}[square]\small
 \item<2-> 現在進行形(〜している)$\longrightarrow$いまこの瞬間に起こっていること%
\hfill{She {\bfseries is sleeping} now.}
 \item<3-> 否定文のつくり方: be動詞の直後にnot
\hfill{}{She {\bfseries is} \Circled[fill color=white]{\,not\,} {\bfseries sleeping} now.}
 \item<4-> 疑問文のつくり方: be動詞を先頭に
\hfill{}{{\bfseries Is} she {\bfseries sleeping} now?}\\
\hfill{}Yes, she is.\\
\hfill{}No, She is not.\\
\hfill{}($=$ No, she's not.)\\
\hfill{}($=$ No, she isn't.)
\end{itemize}
\end{block}
\hfill\visible<5->{{\scriptsize 否定文のつくり方・疑問文のつくり方・疑問文への答え方---みんなbe動詞の時と同じ}}
\end{frame}
%%%%%%%%%%%%%%%%%%%%%%%%%%%%%%%%%%%%%%%
%\subsection{What are you --ing?}
\begin{frame}[plain]{What are you --ing?}
\Large

You are studying \alt<4->{\myAnch{FOCUS2}{orange}{English}}{\myAnch{focus2}{white}{English}} now.

\hfill\visible<2->{{\small Are you studying English now?}}\\[-5pt]
\hfill\visible<2->{{\scriptsize Yes/Noで答える疑問文}}

\visible<5->{\myAnch{WH2}{orange}{What} are you studying now?}%
\hfill\visible<7->{{\scriptsize あなたはいまなにを勉強していますか。}}

\vspace{-8pt}

\visible<8->{\hspace{103pt}drinking}%
\hfill\visible<8->{{\scriptsize あなたはいまなにを飲んでいますか。}}

\vspace{-8pt}

\visible<9->{\hspace{103pt}eating}%
\hfill\visible<9->{{\scriptsize あなたはいまなにを食べていますか。}}

\vspace{-8pt}

\visible<10->{\hspace{103pt}{{\scriptsize ほかにもいろいろな動詞の--ing}}}

\begin{block}<11->{Topic for Today}
\small

\begin{itemize}\setbeamertemplate{items}[square]\small
 \item  動詞を変えることでいろいろな意味を表せます%
\end{itemize}
      \end{block}

\begin{tikzpicture}[remember picture, overlay]
 \visible<6->{\draw[line width=3pt, opacity=.5, orange, ->] (focus2.south) to[out=-165, in=15] node[sloped,above,text=black,font=\tiny,pos=.55]{whatに置き換えて先頭へ} node[sloped,below,text=black,font=\tiny,pos=.5]{後は疑問文の語順} (WH2.north);}
\end{tikzpicture}

\vspace{-30pt}

\mbox{}\hfill{\tiny 0226}\,{\scriptsize \myaudio{./audio/023_is_ing_06.mp3}}

\end{frame}
%%%%%%%%%%%%%%%%%%%%%%%
\begin{frame}[plain]{Exercises}

{\small AとBとの対話になるように空所に適語を補いましょう}
\begin{enumerate}
 \item A: (\visible<2->{~~\textcolor{Orange}{What}~~}) are you studying?\\
       B: I am studying math.%
\hfill%
\visible<6->{$\text{study}+\text{ing}\rightarrow{}\text{studying}$}
 \item A: What are you watching?\\
       B: I (\visible<3->{~~\textcolor{Orange}{am}~~}) watching TV.%
\hfill%
\visible<7->{$\text{watch}+\text{ing}\rightarrow{}\text{watching}$}
 \item A: (\visible<4->{~~\textcolor{Orange}{What}~~}) is he making?\\
       B: He is making a cake now.%
\hfill%
\visible<8->{$\text{mak\textcolor{red}{\bfseries e}}+\text{ing}\rightarrow{}\text{making}$}

\item A: What  (\visible<5->{~~\textcolor{Orange}{are}~~})  you writing?\\
       B: I am wriitng  a letter.%
\hfill%
\visible<9->{$\text{writ\textcolor{red}{\bfseries e}}+\text{ing}\rightarrow{}\text{writing}$}
 \end{enumerate} 

\hfill%
\visible<10>{%
つづりが`e'で終わる動詞の--ing形に注意しよう}

\mbox{}\hfill{\tiny 0248}\,{\scriptsize \myaudio{./audio/023_is_ing_07.mp3}}

\end{frame}
%%%%%%%%%%%%%%%%%%%%%%%%%%%%%%%%%
\begin{frame}[plain]{Exercises}

{\small (~~~~)内の動詞を用いて、あたえられた日本語の意味になるよう英文を作りましょう。なお、動詞は必要があれば変化させてください}

\bigskip

\begin{tabular}{rll}
 1&あなたは何を食べているのですか。(eat) &\\
 &\visible<2->{What are you eating?} & \\
 2&彼女は何を飲んでいますか。(drink) &\\
 &\visible<3->{What is she drinking?} & \\
3&彼は何を作っていますか。(make) &\\
 &\visible<4->{What is he making?} & \\
4&彼らは何を書いていますか。(write) & \\
 &\visible<5->{What are they writing?} & \\
\end{tabular}

\mbox{}\hfill{\tiny 0152}\,{\scriptsize \myaudio{./audio/023_is_ing_08.mp3}}
\end{frame}
%%%%%%%%%%%%%%%%%%%%%%%%%%%%%%%%%
\begin{frame}[plain]{What are you doing?}
\Large

What are you \begin{tabular}[t]{l@{\,}}
	      studying\\
              \visible<2->{drinking}\\
              \visible<3->{making}\\
              \visible<4->{eating}\\
              \multicolumn{1}{c}{\visible<5->{$\downarrow$}}\\
              \,\,\,\visible<6->{doing}
	     \end{tabular}
? 

\begin{block}<7->{Topic for Today}\small
\begin{itemize}\setbeamertemplate{items}[square]\small
 \item  What are you doing?
は「あなたは何をしていますか」という決まり文句として覚えよう
\end{itemize}
\end{block}

\mbox{}\hfill{\tiny 0211}\,{\scriptsize \myaudio{./audio/023_is_ing_09.mp3}}
\end{frame}
%%%%%%%%%%%%%%%%%%%%%%%%%%%
\begin{frame}[plain]{Exercises}

{\small あたえられた日本語の意味になるよう英文を作りましょう}

\begin{tabular}{rll}
 1&あなたは何をしていますか。 & \\
 &\visible<2->{What are you doing?} & \\
 2&彼女は何をしていますか。 & \\
 &\visible<3->{What is she doing?} & \\
3&彼は何をしていますか。 & \\
 &\visible<4->{What is he doing?} & \\
4&彼らは何をしていますか。 & \\
 &\visible<5->{What are they doing?} & \\
\end{tabular}

\mbox{}\hfill{\tiny 0150}\,{\scriptsize \myaudio{./audio/023_is_ing_10.mp3}}

\end{frame}
%%%%%%%%%%%%%%%%%%%%%%%%%%%%%%
\begin{frame}[plain]{Exercises}

{\small Aの質問に対するBの答えがあたえられた内容になるようBのセリフを英語で書きましょう}
\begin{enumerate}
 \item A: What are you doing?\hspace{2\zw}{\small [私は数学の勉強をしています]}\hfill$2(a+b)=2a+2b$\\
       B: \visible<2->{I am studying math.}%
\item A: What is he doing?\hspace{2\zw}{\small [彼は夕食を作っています]}\hfill\raisebox{-5pt}{\scalebox{2}{\twemoji{cooking}\twemoji{spaghetti}\twemoji{carrot}\twemoji{leafy green}\twemoji{tomato}\twemoji{meat on bone}}}\\
       B: \visible<3->{He is cooking dinner}.%
 \item A: What are they doing?\hspace{2\zw}{\small [彼らは川(the river)で泳いでいます]}\hfill\raisebox{-5pt}{\scalebox{2}{\twemoji{man swimming}}}\\
       B: \visible<4->{They are swimming in the river.}%
\hfill%
\visible<6->{$\text{swi\textcolor{red}{m}}+\text{ing}\rightarrow{}\text{swi\textcolor{red}{mm}ing}}$
 \item A: What is she doing?\hspace{2\zw}{\small [彼女は公園(the park)で走っています]}\hfill\scalebox{2.5}{\twemoji{woman running}}\\
       B: \visible<5->{She is running in the park.}%
\hfill%
\visible<7->{$\text{ru\textcolor{red}{n}}+\text{ing}\rightarrow{}\text{ru\textcolor{red}{nn}ing}}$

 \end{enumerate} 

\mbox{}\hfill{\tiny 0247}\,{\scriptsize \myaudio{./audio/023_is_ing_11.mp3}}
\end{frame}
%%%%%%%%%%%%%%%%%%%%%%%%%%%%%%%%%%
%\subsection{Who is --ing?}
\begin{frame}[plain]{Who is --ing?}
\Large

\visible<1->{\alt<1>{\myAnch{jimmy}{white}{Jimmy}}{\myAnch{JIMMY}{orange}{Jimmy}} is playing the guitar.}\hspace{20pt}\scalebox{6}{\twemoji{guitar}}
\vspace{20pt}

\visible<3->{\alt<1-2>{\myAnch{who}{white}{Who}}{\myAnch{WHO}{orange}{Who}} is playing the guitar\alt<1-2>{\myAnch{qm}{white}{?}}{\myAnch{QM}{orange}{?}}}
\visible<5->{{\small だれがギターを演奏していますか}}

\begin{block}<6->{Topics for Today}
\small

\begin{itemize}\setbeamertemplate{items}[square]\small
 \item 「だれが〜していますか」はWho is --ing?
 \item Jimmy is.のように答えます\mbox{}\hfill{\tiny 0113}\,{\scriptsize \myaudio{./audio/023_is_ing_12.mp3}}

\end{itemize}
      \end{block}

ls\smallskip

\begin{tikzpicture}[remember picture, overlay]
\visible<4->{\draw[thick,orange,->] (jimmy.south) to (who.north);}
\end{tikzpicture}
\end{frame}
%%%%%%%%%%%%%%%%%%%%%%%%%%%%%
\begin{frame}[plain]{Exercises}

{\small あたえられた日本語の意味になるよう英文を作りましょう}

\begin{tabular}{rll}
 1&だれがその歌を歌っていますか。\scalebox{2.5}{\twemoji{woman singer}} & \\
 &\visible<2->{Who is singing the song?} & \\
2&だれがピアノを演奏していますか。\scalebox{2}{\twemoji{musical score}\,\,\twemoji{musical keyboard}} & \\
 &\visible<3->{Who is playing the guitar?} & \\
 2&だれがテレビ(TV)を見ていますか。\scalebox{2.5}{\twemoji{television}} & \\
 &\visible<4->{Who is watching TV?} & \\
4&だれがあそこに座っていますか。% \raisebox{-5pt}{\fcChairA{0.005}{black}{.02}} 
&{\small (あそこに: over there)} \\
 &\visible<5->{Who is sitting over there?} & $\text{si\textcolor{red}{t}}+\text{ing}\rightarrow{}\text{si\textcolor{red}{tt}ing}$\\
\end{tabular}

\mbox{}\hfill{\tiny 0157}\,{\scriptsize \myaudio{./audio/023_is_ing_13.mp3}}
\end{frame}
%%%%%%%%%%%%%%%%%%%%%%%%%%%%%%%%%%%
\end{document}
