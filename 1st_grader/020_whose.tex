\documentclass[aspectratio=169,xcolor={dvipsnames,table}]{beamer}
\usepackage[no-math,deluxe,haranoaji]{luatexja-preset}
\renewcommand{\kanjifamilydefault}{\gtdefault}
\renewcommand{\emph}[1]{{\upshape\bfseries #1}}
\usetheme{metropolis}
\metroset{block=fill}
\setbeamertemplate{navigation symbols}{}
\setbeamertemplate{blocks}[rounded][shadow=false]
\usecolortheme[rgb={0.7,0.2,0.2}]{structure}
%%%%%%%%%%%%%%%%%%%%%%%%%%%
%%%%%%%%%%%%%%%%%%%%%%%%%%%
%% さまざまなアイコン
%%%%%%%%%%%%%%%%%%%%%%%%%%%
%\usepackage{fontawesome}
\usepackage{fontawesome5}
\usepackage{figchild}
\usepackage{twemojis}
\usepackage{utfsym}
\usepackage{bclogo}
\usepackage{marvosym}
\usepackage{fontmfizz}
\usepackage{pifont}
\usepackage{phaistos}
\usepackage{worldflags}
\usepackage{jigsaw}
\usepackage{tikzlings}
\usepackage{tikzducks}
\usepackage{scsnowman}
\usepackage{epsdice}
\usepackage{halloweenmath}
\usepackage{svrsymbols}
\usepackage{countriesofeurope}
\usepackage{tipa}
\usepackage{manfnt}
%%%%%%%%%%%%%%%%%%%%%%%%%%%
\usepackage{tikz}
\usetikzlibrary{calc,patterns,decorations.pathmorphing,backgrounds}
\usepackage{tcolorbox}
\usepackage{tikzpeople}
\usepackage{circledsteps}
\usepackage{xcolor}
\usepackage{amsmath}
\usepackage{booktabs}
\usepackage{chronology}
\usepackage{signchart}
%%%%%%%%%%%%%%%%%%%%%%%%%%%
%% 場合分け
%%%%%%%%%%%%%%%%%%%%%%%%%%%
\usepackage{cases}
%%%%%%%%%%%%%%%%%%%%%%%%%%
\usepackage{pdfpages}
%%%%%%%%%%%%%%%%%%%%%%%%%%%
%% 音声リンク表示
\newcommand{\myaudio}[1]{\href{#1}{\faVolumeUp}}
%%%%%%%%%%%%%%%%%%%%%%%%%%
%% \myAnch{<名前>}{<色>}{<テキスト>}
%% 指定のテキストを指定の色の四角枠で囲み, 指定の名前をもつTikZの
%% ノードとして出力する. 図には remember picture 属性を付けている
%% ので外部から参照可能である.
\newcommand*{\myAnch}[3]{%
  \tikz[remember picture,baseline=(#1.base)]
    \node[draw,rectangle,line width=1pt,#2] (#1) {\normalcolor #3};
}
%%%%%%%%%%%%%%%%%%%%%%%%%%
%% \myEmph コマンドの定義
%%%%%%%%%%%%%%%%%%%%%%%%%%
%\newcommand{\myEmph}[3]{%
%    \textbf<#1>{\color<#1>{#2}{#3}}%
%}
\usepackage{xparse} % xparseパッケージの読み込み
\NewDocumentCommand{\myEmph}{O{} m m}{%
    \def\argOne{#1}%
    \ifx\argOne\empty
        \textbf{\color{#2}{#3}}% オプション引数が省略された場合
    \else
        \textbf<#1>{\color<#1>{#2}{#3}}% オプション引数が指定された場合
    \fi
}
%%%%%%%%%%%%%%%%%%%%%%%%%%%
%%%%%%%%%%%%%%%%%%%%%%%%%%%
%% 文末の上昇イントネーション記号 \myRisingPitch
%% 通常のイントネーション \myDownwardPitch
%% https://note.com/dan_oyama/n/n8be58e8797b2
%%%%%%%%%%%%%%%%%%%%%%%%%%%
\newcommand{\myRisingPitch}{
\begin{tikzpicture}[scale=0.3,baseline=0.3]
\draw[->,>=stealth] (0,0) to[bend right=45] (1,1);
\end{tikzpicture}
}
\newcommand{\myDownwardPitch}{
\begin{tikzpicture}[scale=0.3,baseline=0.3]
\draw[->,>=stealth] (0,1) to[bend left=45] (1,0);
\end{tikzpicture}
}
%%%%%%%%%%%%%%%%%%%%%%%%%%%%
%\AtBeginSection[%
%]{%
%  \begin{frame}[plain]\frametitle{授業の流れ}
%     \tableofcontents[currentsection]
%   \end{frame}%
%}

\usepackage{pxrubrica}
%%%%%%%%%%%%%%%%%%%%%%%%%%%
\title{English is fun.}
\subtitle{Whose bag is this?}
\author{}
\institute[]{}
\date[]

%%%%%%%%%%%%%%%%%%%%%%%%%%%%
%% TEXT
%%%%%%%%%%%%%%%%%%%%%%%%%%%%
\begin{document}
\begin{frame}[plain]
  \titlepage
\end{frame}

\section*{授業の流れ}
\begin{frame}[plain]
  \frametitle{授業の流れ}
  \tableofcontents
\end{frame}
%%%%%%%%%%%%%%%%%%%%%%%%%%%%%%
\section{所有代名詞}
%%%%%%%%%%%%%%%%%%%%%%%%%%%%%%%
\begin{frame}[plain]{所有代名詞}
 
\begin{enumerate}
 \item<1-> \begin{enumerate}
	\item<1-> Her hair is blond and my hair is black.%
\hfill{}{\scriptsize blond \textipa{/bl\'And/} 金髪の}
	\item<2-> Her hair is blond and \myEmph{Maroon}{mine} is black. \visible<3->{($\text{mine} = \text{my hair}$)}
       \end{enumerate}
 \item<4-> Is this bike \myEmph{Maroon}{yours} or \myEmph{Maroon}{his}? \visible<5-> {($\text{yours} = \text{your bike}$, $\text{his} = \text{his bike}$)}
 \item<6-> The dog is not \myEmph{Maroon}{ours}. \visible<7->{($\text{ours} = \text{our dog}$)}
\end{enumerate}

\vfill

\begin{block}<8->{Topics for Today}
\pause
\begin{itemize}\setbeamertemplate{items}[square]\small
 \item 同じ名詞の繰り返しを避けて\\
mine(わたしのもの)、yours(あなたのもの)、his(彼のもの)、hers(彼女のもの)\\
などということがあります
 \item mine, yoursなどを\kenten{所有代名詞}といいます
\item 所有代名詞 $=$ \Circled[fill color=white]{\,所有格 $+$ 名詞\,}
\end{itemize}
     \end{block}

\hfill{\tiny 0156}\,{\scriptsize \myaudio{./audio/020_whose_01.mp3}}

\end{frame}
%%%%%%%%%%%%%%%%%%%%%%%%%%%%%%%
\begin{frame}[plain,label=table]{〜のもの(所有代名詞)}
 
 \begin{center}
 \rowcolors{1}{NavyBlue!40}{yellow!40}
\begin{tabular}{llll}\toprule
\rowcolor{white}&主格&所有格&所有代名詞\\\midrule
1人称&I&my&\visible<2->{mine \textipa{/m\'aIn/}}\\
&we&our&\visible<3->{ours \textipa{/\'aU\textrhookschwa z/}}\\
2人称&you&your&\visible<4->{yours \textipa{/j\'U\textrhookschwa z/}}\\
 3人称&he&his&\visible<5->{his \textipa{/h\'Iz/}}\\
&she&her&\visible<6->{hers \textipa{/h\'\textrhookschwa :z/}}\\
&they&their&\visible<7->{theirs \textipa{/D\'e\textrhookschwa z/}}\\
\bottomrule
\end{tabular}
\end{center}

\hfill{}\visible<8->{mine以外はすべてsで終わっていますね}

\hfill{\tiny 0336}\,{\scriptsize \myaudio{./audio/020_whose_01b.mp3}}

\end{frame}
%%%%%%%%%%%%%%%%%%%%%%%%%%%%%%%
\begin{frame}[plain]{ジェニファーのもの、わたしのおじのもの}
 
\begin{enumerate}
 \item This car is \myEmph{Maroon}{Jennifer's}.
 \item These books are \myEmph{Maroon}{my uncle's}.
\end{enumerate}

\begin{block}<2->{Topic for Today}
\pause
\begin{itemize}\setbeamertemplate{items}[square]\small
 \item 「ジェニファーのもの」「私のおじのもの」などの場合は、
\Circled[fill color=white]{\,\,名詞 's\,\,}\,とします
\end{itemize}
     \end{block}

\hfill{\tiny 0114}\,{\scriptsize \myaudio{./audio/020_whose_02.mp3}}

\end{frame}
%%%%%%%%%%%%%%%%%%%%%%%%%%%%%%
\begin{frame}[plain]{Exercises}
(~~~~~~~~)内の語を「〜のもの」を表すように変えてください
 \begin{enumerate}
  \item Is this cup ( you ) ?\hfill\visible<2->{\makebox[40pt][l]{yours}\hspace{180pt}\mbox{}}
  \item This dictionary is ( she )\hfill\visible<3->{\makebox[40pt][l]{hers}\hspace{180pt}\mbox{}}
  \item Is this car ( your mother ) ?\hfill\visible<4->{\makebox[40pt][l]{your mother's}\hspace{180pt}\mbox{}}
  \item These cats are ( Billy ).\hfill\visible<5->{\makebox[40pt][l]{Billy's}\hspace{180pt}\mbox{}}
 \end{enumerate}

\hfill{\tiny 0157}\,{\scriptsize \myaudio{./audio/020_whose_03.mp3}}

\end{frame}
%%%%%%%%%%%%%%%%%%%%%%%%%%%%%%%
\section{whose \textipa{/h\'u:z/}}
\subsection{Whose bag is this?}
%%%%%%%%%%%%%%%%%%%%%%%%%%%%%
\begin{frame}[plain]{だれの} \Large

\mbox{}\hspace{55pt}This is \alt<3->{\myAnch{FOCUS}{orange}{her bag}}{\myAnch{focus}{white}{her bag}}.

\pause


\vspace{7pt}

\mbox{}\hfill{}{\small cf. \myEmph[6-]{Maroon}{Is this} her bag?}%

\vspace{-5pt}

\hfill{\small Yes / Noで答える疑問文}

\pause

\visible<4->{\myAnch{wh}{orange}{Whose bag} \myEmph[6-]{Maroon}{is this} \myAnch{question}{orange}{?}}
\visible<6->{\scalebox{1.4}{\myDownwardPitch}}

\pause

%\mbox{}\hspace{30pt}\myAnch{txt1}{white}{\small 先頭にWho}

\visible<5->{%
\begin{tikzpicture}[remember picture, overlay]
\draw[->,line width=3pt, opacity=.5, orange] (focus.south) to[out=-90, in=90] node[sloped,above,text=black,font=\tiny,pos=.55]{\Circled{whose $+$名詞}にして先頭へ} node[sloped,below,text=black,font=\tiny,pos=.55]{後は疑問文の語順} (wh.north);
\end{tikzpicture}
}

\begin{block}<7->{Topics for Today}
\pause
\begin{itemize}\setbeamertemplate{items}[square]\small
 \item 「だれの?」と聞くとき$\longrightarrow$\,\,\,Whose 〜? \textipa{/h\'u:z/}
 \item  \fcolorbox{black}{white}{$\text{Whose} + \text{名詞}$}\,\,がひとかたまりで先頭にきます / 後ろは疑問文の語順
\end{itemize}
     \end{block}

\vspace{-10pt}
\hfill{\tiny 0131}\,{\scriptsize \myaudio{./audio/020_whose_04.mp3}}

\end{frame}
%%%%%%%%%%%%%%%%%%%%%
%%%%%%%%%%%%%%%%%%%%%%%%%%%%%
\subsection{Whose movies do you like?}
\begin{frame}[plain]{だれの} \Large

\mbox{}\hspace{55pt}You like \alt<3->{\myAnch{FOCUS}{orange}{Spielberg's movies}}{\myAnch{focus}{white}{Spielberg's movies}}.%
\hfill{}{\scriptsize Spielberg \textipa{/sp\'\i:lb\textrhookschwa :g/} 米国の映画監督}

\pause

\vspace{7pt}

\mbox{}\hfill{}{\small cf. \myEmph[6-]{Maroon}{Do you like} Spielberg's movies?}%

\vspace{-5pt}

\hfill{\small Yes / Noで答える疑問文}

\pause

\visible<4->{\myAnch{wh}{orange}{Whose movies} \myEmph[6-]{Maroon}{do you like} \myAnch{question}{orange}{?}}
\visible<6->{\scalebox{1.4}{\myDownwardPitch}}

\pause

%\mbox{}\hspace{30pt}\myAnch{txt1}{white}{\small 先頭にWho}

\visible<5->{%
\begin{tikzpicture}[remember picture, overlay]
\draw[->,line width=3pt, opacity=.5, orange] (focus.south) to[out=-90, in=90]node[sloped,above,text=black,font=\tiny,pos=.55]{\Circled{whose $+$名詞}にして先頭へ} node[sloped,below,text=black,font=\tiny,pos=.55]{後は疑問文の語順} (wh.north);
\end{tikzpicture}
}

\begin{block}{Topics for Today}
\pause
\begin{itemize}\setbeamertemplate{items}[square]\small
 \item 「だれの~を」と聞くとき$\longrightarrow$\,\,\,Whose 〜? \textipa{/h\'u:z/}
 \item  \fcolorbox{black}{white}{$\text{Whose} + \text{名詞}$}\,\,がひとかたまりで先頭にきます / 後ろは疑問文の語順
\end{itemize}
     \end{block}

\vspace{-10pt}
\hfill{\tiny 0138}\,{\scriptsize \myaudio{./audio/020_whose_04b.mp3}}

\end{frame}
%%%%%%%%%%%%%%%%%%%%%
\begin{frame}[plain]{Exercises}

対話になるよう(~~~~~~~~)内の語を並べ替えましょう。先頭に来る語は大文字ではじめてください
 \begin{enumerate}
  \item ( this / is / whose bike ) ? --- It's Bob's.\\
	\visible<2->{Whose bike is this? --- It's Bob's.}
  \item ( are / whose keys / these ) ? --- They are hers.\\
	\visible<3->{Whose keys are these? --- They are hers.}
  \item ( cat / whose / it / is ) ? --- It's my children's.\\
	\visible<4->{Whose cat is it? --- It's my children's.}
  \item ( this / cell phone / whose / is ) ? --- It's mine.%
	\hfill{\scriptsize cell phone: 携帯電話}\\
	\visible<5->{Whose cell phone is this? --- It's mine.}
 \end{enumerate}

\hfill{\tiny 0315}\,{\scriptsize \myaudio{./audio/020_whose_05.mp3}}
\end{frame}
%%%%%%%%%%%%%%%%%%%%%%%%%%
\begin{frame}<1-5>[plain]{Exercises}

あたえられた日本語の意味になるよう空所に適語を補いましょう

\begin{enumerate}
 \item あなたはだれのジャケットを着ていますか\hfill
(\alt<1>{~~\phantom{Whose}~~}{~~Whose~~}) jacket do you wear?
 \item あなたはだれの自転車に乗っていますか\hfill
Whose bike (\alt<1-2>{~~\phantom{do}~~}{~~do~~}) you ride?
 \item 彼女はだれの車を使っていますか\hfill
Whose car (\alt<1-3>{~~\phantom{does}~~}{~~does~~}) (\alt<1-3>{~~\phantom{she}~~}{~~she~~})  use?
 \item あなたはだれの歌が好きですか\hfill
(\alt<1-4>{~~\phantom{Whose}~~}{~~Whose~~}) songs (\alt<1-4>{~~\phantom{do}~~}{~~do~~}) you (\alt<1-4>{~~\phantom{like}~~}{~~like~~})?
\end{enumerate}

\hfill{\tiny 0156}\,{\scriptsize \myaudio{./audio/020_whose_05b.mp3}}

\end{frame}
%%%%%%%%%%%%%%%%%%%%%
\section{疑問詞whoseのまとめ}
\begin{frame}[plain]{まとめ}
 \begin{block}{所有代名詞 $=$ 所有格 $+$ 名詞}
\begin{itemize}\setbeamertemplate{items}[square]\small
 \item 同じ名詞の繰り返しを避けて\textbf{mine}(わたしのもの)、\textbf{yours}(あなたのもの)などということがあります\hfill{\scriptsize Is this bike {\bfseries yours} or {\bfseries his}?}
 \item \textbf{mine, yours}などを\kenten{所有代名詞}といいます
 \item 「ジェニファーのもの」「私の母のもの」などの場合は、
\Circled[fill color=white]{\,\,名詞 's\,\,}\,とします\\
\hfill{\scriptsize This car is {\bfseries Jennifer's}.}\\
\hfill{\scriptsize These books are {\bfseries my mother's}.}
\end{itemize}
     \end{block}
\pause
\begin{block}{疑問詞whose \textipa{/h\'u:z/}}
\begin{itemize}\setbeamertemplate{items}[square]\small
 \item 「だれの~?」と聞くとき$\longrightarrow$\,\,\,{\bfseries Whose} 〜?
 \item  \fcolorbox{black}{white}{$\text{{\bfseries Whose}} + \text{名詞}$}\,\,がひとかたまりで先頭にきます / 後ろは疑問文の語順
 \item {\bfseries whose} \textipa{/h\'u:z/}\hfill{\scriptsize {\bfseries Whose} bag is this?}\\\hfill{\scriptsize {\bfseries Whose} movies do you like?}
\end{itemize}
     \end{block}
\hfill{\tiny 0221}\,{\scriptsize \myaudio{./audio/020_whose_06.mp3}}

\end{frame}
%%%%%%%%%%%%%%%%%%%%%%%
\againframe<8>[plain]{table}
\end{document}
