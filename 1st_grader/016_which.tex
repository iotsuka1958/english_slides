\documentclass[aspectratio=169]{beamer}
\usepackage[no-math,deluxe,haranoaji]{luatexja-preset}
\renewcommand{\kanjifamilydefault}{\gtdefault}
\renewcommand{\emph}[1]{{\upshape\bfseries #1}}
\usetheme{metropolis}
\metroset{block=fill}
\setbeamertemplate{navigation symbols}{}
\usecolortheme[rgb={0.7,0.2,0.2}]{structure}
%%%%%%%%%%%%%%%%%%%%%%%%%%%
\usepackage{media9}
%%%%%%%%%%%%%%%%%%%%%%%%%%%
%% さまざまなアイコン
%%%%%%%%%%%%%%%%%%%%%%%%%%%
\usepackage{fontawesome}
\usepackage{figchild}
\usepackage{twemojis}
\usepackage{utfsym}
\usepackage{bclogo}
\usepackage{marvosym}
\usepackage{fontmfizz}
\usepackage{pifont}
\usepackage{phaistos}
\usepackage{worldflags}
%%%%%%%%%%%%%%%%%%%%%%%%%%%
\usepackage{tikz}
\usetikzlibrary{backgrounds}
\usepackage{tcolorbox}
\usepackage{tikzpeople}
\usepackage{xcolor}
\usepackage{amsmath}
%%%%%%%%%%%%%%%%%%%%%%%%%%%
%% 場合分け
\usepackage{cases}
%%%%%%%%%%%%%%%%%%%%%%%%%%%
% \myAnch{<名前>}{<色>}{<テキスト>}
% 指定のテキストを指定の色の四角枠で囲み, 指定の名前をもつTikZの
% ノードとして出力する. 図には remeber picture 属性を付けている
% ので外部から参照可能である.
\newcommand*{\myAnch}[3]{%
  \tikz[remember picture,baseline=(#1.base)]
    \node[draw,rectangle,#2] (#1) {\normalcolor #3};
}
%%%%%%%%%%%%%%%%%%%%%%%%%%%%
%% 音声リンク表示
\newcommand{\myaudio}[1]{\href{#1}{\faVolumeUp}}
%%%%%%%%%%%%%%%%%%%%%%%%%%%
% \myEmph コマンドの定義
%\newcommand{\myEmph}[3]{%
%    \textbf<#1>{\color<#1>{#2}{#3}}%
%}
\usepackage{xparse} % xparseパッケージの読み込み
\NewDocumentCommand{\myEmph}{O{} m m}{%
    \def\argOne{#1}%
    \ifx\argOne\empty
        \textbf{\color{#2}{#3}}% オプション引数が省略された場合
    \else
        \textbf<#1>{\color<#1>{#2}{#3}}% オプション引数が指定された場合
    \fi
}
%%%%%%%%%%%%%%%%%%%%%%%%%%%
%% 文末の上昇イントネーション記号 \myRisingPitch
%% 通常のイントネーション \myDownwardPitch
%% https://note.com/dan_oyama/n/n8be58e8797b2
%%%%%%%%%%%%%%%%%%%%%%%%%%%
\newcommand{\myRisingPitch}{
\begin{tikzpicture}[scale=0.3,baseline=0.3]
\draw[->,>=stealth] (0,0) to[bend right=45] (1,1);
\end{tikzpicture}
}
\newcommand{\myDownwardPitch}{
\begin{tikzpicture}[scale=0.3,baseline=0.3]
\draw[->,>=stealth] (0,1) to[bend left=45] (1,0);
\end{tikzpicture}
}
%%%%%%%%%%%%%%%%%%%%%%%%%%%
\title{English is fun.\,\,{}--- Where does Jane live? ---}
\author{}
\institute[]{}
\date[]

%%%%%%%%%%%%%%%%%%%%%%%%%%%%
%% TEXT
%%%%%%%%%%%%%%%%%%%%%%%%%%%%
\begin{document}

\begin{frame}[plain]
 oyoyo
\end{frame}

\begin{frame}[plain]
  \titlepage
\end{frame}


\section*{授業の流れ}
\begin{frame}[plain]
  \frametitle{授業の流れ}
  \tableofcontents
\end{frame}



\section{Which}

\subsection{Which is your favorite color?: be動詞のとき}
\begin{frame}[plain]{Which is your favorite color?}
 \Large

be動詞のとき

\mbox{}\hspace{57pt}Your favorite color is pink.\hspace{50pt}{\small favorite: お気に入りの}

\pause


\vspace{7pt}

\mbox{}\hspace{55pt}Is your favorite color \alt<3->{\myAnch{FOCUS}{orange}{pink}}{\myAnch{focus}{white}{pink}}?%
\hspace{20pt}{\normalsize YesまたはNoで答える疑問文}

\vspace{7pt}

\pause

\visible<4->{\myAnch{wh}{orange}{Which} is your favorite color \myAnch{question}{orange}{?}}
\visible<6->{\scalebox{1.4}{\myDownwardPitch}}

\pause


\visible<5->{%
\begin{tikzpicture}[remember picture, overlay]
\draw[->, thick, orange] (focus.south) to[out=-165, in=30] (wh.north);
\end{tikzpicture}
}

\visible<7->{%
\begin{exampleblock}{Topics for Today}
\pause
\begin{itemize}\small
 \item 「どれ、どちら」と聞くとき$\longrightarrow$\,\,\,which is 〜 ?
 \item   文末に`?'をつける(イントネーションは\myDownwardPitch{}\,\,)
\end{itemize}
     \end{exampleblock}
}
\visible<8->{%
\mbox{}\hfill\myaudio{./audio/015_which_01.mp3}
}
\end{frame}

\begin{frame}[plain]\frametitle{Which is your favorite color, pink or blue?}
\Large
 
\begin{enumerate}
\visible<1->{ \item Which is your favorite color?} \hfill\visible<2->{{\small どれ}}
\visible<3->{ \item \textcolor{orange}{\bfseries Which} is your favorite color, \colorbox{pink}{pink}  \textcolor{orange}{\bfseries or} \colorbox{blue!50}{blue} ?}\hfill\visible<4->{{\small AとBのどちら}}
\end{enumerate}

\vspace{20pt}

\visible<5->{%
\begin{exampleblock}{Topic for Today}
\pause
\begin{itemize}
 \item 「AとBのどちら」と聞くとき$\longrightarrow$\,\,\,\textcolor{orange}{\bfseries Which}  is 〜, \textcolor{orange}{\bfseries A or B} ?
\end{itemize}
     \end{exampleblock}
}
\visible<6->{%
\mbox{}\hfill\myaudio{./audio/015_which_02.mp3}
}
\end{frame}

\begin{frame}[plain]{Exercises}
つぎの文の意味を考えましょう。

\begin{enumerate}
 \item Which is Ken's bag?
 \item Which is your pencil?
 \item Which is your favorite drink, tea or coffee?

\end{enumerate} 

\pause

\mbox{}\hfill\myaudio{./audio/015_which_03.mp3}
\end{frame}


\subsection{Which do you like?: 一般動詞のとき}
\begin{frame}[plain]\frametitle{Which season do you like?}
\Large
一般動詞のとき

\pause

\mbox{}\hspace{75pt}%
You like summer.


\pause

\mbox{}\hspace{55pt}%
Do you like \alt<5->{\myAnch{FOCUS2}{orange}{summer}}{\myAnch{focus2}{white}{summer}}?
\hspace{10pt}\visible<4->{{\normalsize Yes/Noで答える疑問文}}

\visible<6->{\myAnch{WH2}{orange}{Which} do you like, spring \textcolor{orange}{\bfseries or} summer?}\hfill\visible<7->{{\small AとBのどちら}}

\visible<8->{%
\begin{tikzpicture}[remember picture, overlay]
 \draw[thick, orange, ->] (focus2.south) to[out=-165, in=15] (WH2.north east);
\end{tikzpicture}
}

\visible<9->{%
\begin{exampleblock}{Topics for Today}
\pause
\begin{itemize}\small
 \item `Which'を先頭に置いて、疑問文のかたちを続ける\,\,Which do you  〜?
 \item   文末に`?'をつける(イントネーションは\myDownwardPitch{}\,\,)
\end{itemize}
     \end{exampleblock}
}
\visible<10->{%
\mbox{}\hfill\myaudio{./audio/015_which_04.mp3}
}
\end{frame}

\subsection{Exercises}
\begin{frame}[plain]{Exercises}
つぎの文の意味を考えましょう。

\begin{tabular}{rll}
1& Which does he need, a pen or a pencil? &  {\small need: 必要とする} \\
2&Which does she eat for breakfast, cereal or toast? &   {\small cereal: シリアル}\\
3& Which do you like, Englsi or math? &   {\small math: 数学}\\
4& Which do they speak, English or Spanish?&  {\small Spanish: スペイン語} \\
\end{tabular}

\visible<2->{%
\mbox{}\hfill\myaudio{./audio/016_which_05.mp3}
}
\end{frame}


\begin{frame}[plain]{Exercises}
 あたえられた状況に合うように、(~~~~~~~) に適語を入れなさい。

\begin{enumerate}
 \item どれが相手の自転車かをたずねる場合\\
\alt<2->{\textcolor{orange}{Which}}{(\hspace{30pt})} is your bike?
 \item ピザとパスタのどちらが好きかをたずねる場合\\
\alt<3->{\textcolor{orange}{Which}}{(\hspace{30pt})} do you like, pizza \alt<3->{\textcolor{orange}{or}}{(\hspace{30pt})} pasta?
 \item 猫と犬のどちらを飼いたいかをたずねる場合\\
\alt<4->{\textcolor{orange}{Which}}{(\hspace{30pt})} do you want, a cat \alt<4->{\textcolor{orange}{or}}{(\hspace{30pt})} a dog?
\end{enumerate}

\visible<5->{%
\mbox{}\hfill\myaudio{./audio/016_which_06.mp3}
}
\end{frame}

\begin{frame}[plain]{$\text{Which} + \text{名詞}$}
\begin{enumerate}
 \visible<1->{\item \myAnch{pron}{orange}{Which} do you like, tea or coffee?}
 \visible<2->{\item \myAnch{adj}{orange}{which\raisebox{5pt}{$\curvearrowright$}drink} do you like, tea or coffee?}\\ {\small どちらの飲みもの}
\end{enumerate}

\visible<3->{%
\begin{exampleblock}{Topics for Today}
\pause
\begin{itemize}\small
 \item \fbox{$\text{Which} + \text{名詞}$}\,\,がひとかたまりになって「どちらの〜」の意味で使うことがあります
 \item \fbox{$\text{Which} + \text{名詞}$}\,\,も文の先頭にきます
\end{itemize}
     \end{exampleblock}
}
\visible<4->{%
\mbox{}\hfill\myaudio{./audio/015_which_07.mp3}
}
\end{frame}


\begin{frame}[plain]{Exercises}
つぎの文の意味を考えましょう。

\begin{enumerate}
 \item \visible<1->{Which color do you like, red or blue?}\hfill\visible<2->{color: 色}
 \item \visible<1->{Which subject  do you like, music or art?}\hfill\visible<3->{subject: 科目}
 \item \visible<1->{Which language do they speak, English or French?}\hfill\visible<4->{language: 言語}
 \item \visible<1->{Which animal does she have, a cat or a dog?}\hfill\visible<5->{animal: 動物}
\end{enumerate} 

\visible<6->{%
\mbox{}\hfill\myaudio{./audio/015_which_08.mp3}
}
\end{frame}

\end{document}
