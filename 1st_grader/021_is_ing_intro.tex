\documentclass[aspectratio=169,xcolor={dvipsnames,table}]{beamer}
\usepackage[no-math,deluxe,haranoaji]{luatexja-preset}
\renewcommand{\kanjifamilydefault}{\gtdefault}
\renewcommand{\emph}[1]{{\upshape\bfseries #1}}
\usetheme{metropolis}
\metroset{block=fill}
\setbeamertemplate{navigation symbols}{}
\setbeamertemplate{blocks}[rounded][shadow=false]
\usecolortheme[rgb={0.7,0.2,0.2}]{structure}
%%%%%%%%%%%%%%%%%%%%%%%%%%
%% Change alert block colors
%%% 1- Block title (background and text)
\setbeamercolor{block title alerted}{fg=mDarkTeal, bg=mLightBrown!45!yellow!45}
\setbeamercolor{block title example}{fg=magenta!10!black, bg=mLightGreen!70}
%%% 2- Block body (background)
\setbeamercolor{block body alerted}{bg=mLightBrown!25}
\setbeamercolor{block body example}{bg=mLightGreen!15}
%%%%%%%%%%%%%%%%%%%%%%%%%%%
%%%%%%%%%%%%%%%%%%%%%%%%%%%
\usepackage{media9}
%%%%%%%%%%%%%%%%%%%%%%%%%%%
%% さまざまなアイコン
%%%%%%%%%%%%%%%%%%%%%%%%%%%
\usepackage{fontawesome}
\usepackage{figchild}
\usepackage{twemojis}
\usepackage{utfsym}
\usepackage{bclogo}
\usepackage{marvosym}
\usepackage{fontmfizz}
\usepackage{pifont}
\usepackage{phaistos}
\usepackage{worldflags}
\usepackage{tabularray}
\usepackage{booktabs}
\usepackage{circledsteps}
\usepackage{manfnt}
\usepackage{tipa}
%%%%%%%%%%%%%%%%%%%%%%%%%%%%
%% さまざまなアイコン
%%%%%%%%%%%%%%%%%%%%%%%%%%%
%\usepackage{fontawesome}
\usepackage{fontawesome5}
\usepackage{figchild}
\usepackage{twemojis}
\usepackage{utfsym}
\usepackage{bclogo}
\usepackage{marvosym}
\usepackage{fontmfizz}
\usepackage{pifont}
\usepackage{phaistos}
\usepackage{worldflags}
\usepackage{jigsaw}
\usepackage{tikzlings}
\usepackage{tikzducks}
\usepackage{scsnowman}
\usepackage{epsdice}
\usepackage{halloweenmath}
\usepackage{svrsymbols}
\usepackage{countriesofeurope}
\usepackage{tipa}
\usepackage{manfnt}
%%%%%%%%%%%%%%%%%%%%%%%%%%%
\usepackage{tikz}
\usetikzlibrary{calc,patterns,decorations.pathmorphing,backgrounds}
\usepackage{tcolorbox}
\usepackage{tikzpeople}
\usepackage{circledsteps}
\usepackage{xcolor}
\usepackage{amsmath}
\usepackage{booktabs}
\usepackage{chronology}
\usepackage{signchart}
%%%%%%%%%%%%%%%%%%%%%%%%%%%
%% 場合分け
%%%%%%%%%%%%%%%%%%%%%%%%%%%
\usepackage{cases}
%%%%%%%%%%%%%%%%%%%%%%%%%%
\usepackage{pdfpages}
%%%%%%%%%%%%%%%%%%%%%%%%%%%
%% 音声リンク表示
\newcommand{\myaudio}[1]{\href{#1}{\faVolumeUp}}
%%%%%%%%%%%%%%%%%%%%%%%%%%
%% \myAnch{<名前>}{<色>}{<テキスト>}
%% 指定のテキストを指定の色の四角枠で囲み, 指定の名前をもつTikZの
%% ノードとして出力する. 図には remember picture 属性を付けている
%% ので外部から参照可能である.
\newcommand*{\myAnch}[3]{%
  \tikz[remember picture,baseline=(#1.base)]
    \node[draw,rectangle,line width=1pt,#2] (#1) {\normalcolor #3};
}
%%%%%%%%%%%%%%%%%%%%%%%%%%
%% \myEmph コマンドの定義
%%%%%%%%%%%%%%%%%%%%%%%%%%
%\newcommand{\myEmph}[3]{%
%    \textbf<#1>{\color<#1>{#2}{#3}}%
%}
\usepackage{xparse} % xparseパッケージの読み込み
\NewDocumentCommand{\myEmph}{O{} m m}{%
    \def\argOne{#1}%
    \ifx\argOne\empty
        \textbf{\color{#2}{#3}}% オプション引数が省略された場合
    \else
        \textbf<#1>{\color<#1>{#2}{#3}}% オプション引数が指定された場合
    \fi
}
%%%%%%%%%%%%%%%%%%%%%%%%%%%
%%%%%%%%%%%%%%%%%%%%%%%%%%%
%% 文末の上昇イントネーション記号 \myRisingPitch
%% 通常のイントネーション \myDownwardPitch
%% https://note.com/dan_oyama/n/n8be58e8797b2
%%%%%%%%%%%%%%%%%%%%%%%%%%%
\newcommand{\myRisingPitch}{
\begin{tikzpicture}[scale=0.3,baseline=0.3]
\draw[->,>=stealth] (0,0) to[bend right=45] (1,1);
\end{tikzpicture}
}
\newcommand{\myDownwardPitch}{
\begin{tikzpicture}[scale=0.3,baseline=0.3]
\draw[->,>=stealth] (0,1) to[bend left=45] (1,0);
\end{tikzpicture}
}
%%%%%%%%%%%%%%%%%%%%%%%%%%%%
%\AtBeginSection[%
%]{%
%  \begin{frame}[plain]\frametitle{授業の流れ}
%     \tableofcontents[currentsection]
%   \end{frame}%
%}

%%%%%%%%%%%%%%%%%%%%%%%%%%%
\usepackage{tikz}
\usetikzlibrary{backgrounds}
\usepackage{tcolorbox}
\usepackage{xcolor}
\usepackage{amsmath}
%%%%%%%%%%%%%%%%%%%%%%%%%%%
%% 場合分け
\usepackage{cases}
%%%%%%%%%%%%%%%%%%%%%%%%%%%
% \myAnch{<名前>}{<色>}{<テキスト>}
% 指定のテキストを指定の色の四角枠で囲み, 指定の名前をもつTikZの
% ノードとして出力する. 図には remeber picture 属性を付けている
% ので外部から参照可能である.
\newcommand*{\myAnch}[3]{%
  \tikz[remember picture,baseline=(#1.base)]
    \node[draw,rectangle,#2] (#1) {\normalcolor #3};
}
%%%%%%%%%%%%%%%%%%%%%%%%%%%%
%% 音声リンク表示
\newcommand{\myaudio}[1]{\href{#1}{\faVolumeUp}}
%%%%%%%%%%%%%%%%%%%%%%%%%%%
% \myEmph コマンドの定義
%\newcommand{\myEmph}[3]{%
%    \textbf<#1>{\color<#1>{#2}{#3}}%
%}
\usepackage{xparse} % xparseパッケージの読み込み
\NewDocumentCommand{\myEmph}{O{} m m}{%
    \def\argOne{#1}%
    \ifx\argOne\empty
        \textbf{\color{#2}{#3}}% オプション引数が省略された場合
    \else
        \textbf<#1>{\color<#1>{#2}{#3}}% オプション引数が指定された場合
    \fi
}
%%%%%%%%%%%%%%%%%%%%%%%%%%%
%% 文末の上昇イントネーション記号 \myRisingPitch
%% 通常のイントネーション \myDownwardPitch
%% https://note.com/dan_oyama/n/n8be58e8797b2
%%%%%%%%%%%%%%%%%%%%%%%%%%%
\newcommand{\myRisingPitch}{
\begin{tikzpicture}[scale=0.3,baseline=0.3]
\draw[->,>=stealth] (0,0) to[bend right=45] (1,1);
\end{tikzpicture}
}
\newcommand{\myDownwardPitch}{
\begin{tikzpicture}[scale=0.3,baseline=0.3]
\draw[->,>=stealth] (0,1) to[bend left=45] (1,0);
\end{tikzpicture}
}
%%%%%%%%%%%%%%%%%%%%%%%%%%%
\title{English is fun.}
\subtitle{He is reading a book now.}
\author{}
\institute[]{}
\date[]

%%%%%%%%%%%%%%%%%%%%%%%%%%%%
%% TEXT
%%%%%%%%%%%%%%%%%%%%%%%%%%%%
\begin{document}


\begin{frame}[plain]
  \titlepage
\end{frame}


\section*{授業の流れ}
\begin{frame}[plain]
  \frametitle{授業の流れ}
  \tableofcontents
\end{frame}



\section{現在進行形}

\subsection{現在進行形とは}
\begin{frame}[plain]{現在進行形とは}
 \Large


\visible<1->{\textcolor{Maroon}{1.}\,\,They usually play baseball on Sundays.\hfill{\scriptsize usually: いつもは}}

\vspace{7pt}

\visible<3->{\textcolor{Maroon}{2.}\,\,They \Circled[fill color=white]{\bfseries\,\,\textcolor{Maroon}{are} play\textcolor{Maroon}{ing}\,\,} baseball in the park now.\hfill{\scriptsize now: 今、たった今}}

\mbox{}\hspace{70pt}%
\visible<4->{{\small 
$\left\{\begin{tabular}[c]{l}
         am\\
         are\\
         is
       \end{tabular}\right\} + \text{--ing}$
}
\,\,\,\,$\longrightarrow$\,\,\,\,\Circled[fill color = white]{\,\,$\text{be動詞}+\text{---ing}$\,\,}\,\,\,現在進行形
}

\vspace{10pt}

\visible<2->{%
\begin{exampleblock}{Topics for Today}
\pause
\begin{itemize}\small
 \item<2-> 現在形$\longrightarrow$現在の習慣的な行為
 \item<5-> 現在進行形$\longrightarrow$今この瞬間に起こっていること\hfill{}「~している」\,\,\,\,
\end{itemize}
     \end{exampleblock}
}
\visible<1->{%
\mbox{}\hfill\myaudio{./audio/021_is_ing_intro_01.mp3}
}
\end{frame}
%%%%%%%%%%%%%%%%%%%%%%%%%%%%%%%%%%%%%%%%%
\subsubsection{Exercises}
\begin{frame}[plain]{Exercises}
あたえられた日本文の意味になるように、空所に適切な単語を補いましょう

\begin{enumerate}
 \item わたしは、いま読書をしています。\\
I \alt<2->{(\,\,\textcolor{Maroon}{\bfseries am}\,\,)}{(\phantom{\,\,am\,\,})} reading a book now.\hfill\visible<3->{{\scriptsize cf.\,\,\,I read a book after dinner.}}
 \item 彼女は、いま歌を歌っています。\\
She \alt<4->{(\,\,\textcolor{Maroon}{\bfseries is}\,\,)}{(\phantom{\,\,is\,\,})} singing a song now.\hfill\visible<5->{{\scriptsize cf.\,\,\,She sings a song every day.}}
 \item わたしたちは、いま宿題をしています。\\
We \alt<6->{(\,\,\textcolor{Maroon}{\bfseries are}\,\,)}{(\phantom{\,\,are\,\,})} doing homework now.\hfill\visible<7->{{\scriptsize cf.\,\,\,We do homework before dinner.}}
 \item わたしたちのネコは、いまベッドの上で寝ています。\\
Our cat \alt<8->{(\,\,\textcolor{Maroon}{\bfseries is}\,\,)}{(\phantom{\,\,is\,\,})} sleeping on the bed now.\hfill\visible<9->{{\scriptsize cf.\,\,\,Our cat usually sleeps on the sofa.}}
\end{enumerate} 


\visible<1->{%
\mbox{}\hfill\myaudio{./audio/021_is_ing_intro_02.mp3}
}
\end{frame}
%%%%%%%%%%%%%%%%%%%%%%%%%%%%
\begin{frame}[plain]{Exercises}
あたえられた日本文の意味になるように、空所に適切な単語を補いましょう。

\begin{enumerate}
 \item わたしは、いま読書をしています。\\
I am \alt<2->{(\,\,\textcolor{Maroon}{\bfseries reading}\,\,)}{(\phantom{\,\,reading\,\,})} a book now.
 \item 彼女は、いま歌を歌っています。\\
She is \alt<3->{(\,\,\textcolor{Maroon}{\bfseries singing}\,\,)}{(\phantom{\,\,singing\,\,})} a song now.
 \item わたしたちは、いま宿題をしています。\\
We are \alt<4->{(\,\,\textcolor{Maroon}{\bfseries doing}\,\,)}{(\phantom{\,\,doing\,\,})}  homework now.
 \item わたしたちのネコは、いまベッドの上で寝ています。\\
Our cat is \alt<5->{(\,\,\textcolor{Maroon}{\bfseries sleeping}\,\,)}{(\phantom{\,\,sleeping\,\,})} on the bed now.
\end{enumerate} 

\visible<6->{%
\mbox{}\hfill\myaudio{./audio/021_is_ing_intro_02.mp3}
}
\end{frame}
%%%%%%%%%%%%%%%%%%%%%%%%%%%%%%%%%%%%%%%%%%%%%
\subsection{---ingの作り方}
\begin{frame}[plain,label=how2makeIng]{--ingのつくりかた}
 
\begin{center}
% \rowcolors{2}{NavyBlue!50}{yellow!40}
%\begin{tblr}{
 % colspec = {l l l l}, % 列の仕様
 % row{5-6} = {bg=yellow!50}, % 1行目の背景色を黄色に設定
 % row{7-8} = {bg=NavyBlue!40} % 3行目の背景色を薄い灰色に設定
%}
\begin{tabular}{rlll}
 \toprule
&{\small 原形}&{\small ---ing}\\\midrule
1&\visible<1->{eat}&\visible<2->{{\small eating}}&\visible<3->{{原形に---ingをつけるだけ(原則)}}\\
2&\visible<1->{play}&\visible<4->{{\small playing}}&\\
3&\visible<1->{go}&\visible<5->{{\small going}}&\\
4&\visible<1->{read}&\visible<6->{{\small reading}}&\\
\rowcolor{yellow!40}5&\visible<1->{make}&\visible<7->{{\small making}}&\visible<10->{eをとって---ing}\\
\rowcolor{yellow!40}6&\visible<1->{write}&\visible<8->{{\small writing}}&\\
\rowcolor{yellow!40}7&\visible<1->{use}&\visible<9->{{\small using}}&\\
\rowcolor{NavyBlue!40}8&\visible<1->{swim}&\visible<11->{{\small swimming}}&\visible<13->{最後の1文字を重ねて---ing}\\
\rowcolor{NavyBlue!40}9&\visible<1->{run}&\visible<12->{{\small running}}&\visible<14->{\scriptsize ほかにもsit $\rightarrow$ sitting \ldots}\\
%\rowcolor{NavyBlue!40}10&\visible<1->{sit}&\visible<14->{{\small sitting}}&\\
\bottomrule
\end{tabular}
%\end{tblr}%
\end{center}

\vspace{-13pt}

\hfill\visible<15->{\scriptsize \dbend\,\,\dbend\,\,最後の1文字を重ねるのは\,\,\Circled{\,\,$\text{短い母音}+\text{子音字}$\,\,}\,\,で終わる動詞のときです}%
\mbox{}\hfill\myaudio{./audio/021_is_ing_intro_03.mp3}

\end{frame}
%%%%%%%%%%%%%%%%%%%%%%%%%%%%
\begin{frame}[plain]{最後の1文字を重ねるとき}
 \LARGE

sit / eat

 どちらもtというアルファベットで終わっているが\ldots

\bigskip

\begin{enumerate}
 \item sit\,\,\,\,\,\, \textipa{/s\'It/} 短い母音\textipa{/I/} $\longrightarrow$ sitting
 \item eat\,\,\, \textipa{/\'\i:t/}\, 長い母音\textipa{/i:/} $\longrightarrow$\, eating
\end{enumerate}

\bigskip

\hfill\visible{\scriptsize \dbend\,\,\dbend\,\,最後の1文字を重ねるのは\,\,\Circled{\,\,$\text{短い母音}+\text{子音字}$\,\,}\,\,で終わる動詞のときです}%
\end{frame}

%%%%%%%%%%%%%%%%%%%%%%%%%%%%%
\subsubsection{Exercises}
\begin{frame}[plain]{Exercises}
(~~~~~)内の動詞を適当な形に変えましょう。また各文の意味を書いてください
\begin{enumerate}
 \item She is (~~\alt<2->{\textcolor{Maroon}{\bfseries washing}}{wash}~~) her car now.
 \item They are (~~\alt<3->{\textcolor{Maroon}{\bfseries studying}}{study}~~) Spanish now. 
 \item He is  (~~\alt<4->{\textcolor{Maroon}{\bfseries making}}{make}~~) a delicious dinner for us.\hfill{\small delicious: おいしい}
 \item I am (~~\alt<5->{\textcolor{Maroon}{\bfseries writing}}{write}~~) a letter to my friend.
 \item They are (~~\alt<6->{\textcolor{Maroon}{\bfseries swimming}}{swim}~~) in the pool now.
 \item The dog is (~~\alt<7->{\textcolor{Maroon}{\bfseries running}}{run}~~) in the yard now.\hfill{\small yard: 庭}
\end{enumerate} 


\pause

\mbox{}\hfill\myaudio{./audio/021_is_ing_intro_04.mp3}
\end{frame}
%%%%%%%%%%%%%%%%%%%%%%%%%%%%%
\section{「動作」をあらわす動詞と「状態」をあらわす動詞}
\begin{frame}[plain]{「動作」と「状態」}
\large
 \begin{enumerate}
  \item<1-> He \myEmph[1-]{Maroon}{plays} the piano. (動作)\hspace{20pt}\visible<4->{$\rightarrow$\,\,He \myEmph[4-]{Maroon}{is playing} the piano.}
  \item<2-> He \myEmph[2-]{NavyBlue}{knows} Helen. (状態)\hspace{34pt}\visible<5->{$\rightarrow$\,\,*He \myEmph[5-]{NavyBlue}{is knowing} Helen.}\\
\hfill\visible<5->{{\scriptsize *: 「まちがった英文です」という記号}}
 \end{enumerate}

\begin{block}<3->{「動作」をあらわす動詞と「状態」をあらわす動詞}
\begin{itemize}
 \visible<3->{\item 「動作」をあらわす動詞の例: eat, drink, run, walk, playなど}
 \visible<3->{\item 「状態」をあらわす動詞の例: know, like, haveなど}
 \visible<6->{\item \dbend\,\,「状態」を表す動詞は進行形にしません\\
\mbox{}\hfill{\scriptsize \dbend{}: 「ちょっと難しいが、しっかり理解すると実力アップ」という記号}\\\mbox{}}
\begin{itemize}
 \item<7-> I like tea. (* I am liking tea.)
 \item<8-> He has a car. (*He is having a car.)
\end{itemize}

\end{itemize}


\end{block}
\end{frame}
%%%%%%%%%%%%%%%%%%%%
%%%%%%%%%%%%%%%%%%%%%%%%%%%%%%%%
\section{まとめ}
\begin{frame}[plain]{学習したこと}
\begin{block}<1->{まとめ1}
\begin{itemize}
 \visible<1->{\item 現在進行形\,\,\Circled[fill color=white]{\,\,$\text{be動詞}+\text{---ing}$\,\,}}\hfill\visible<2->{今この瞬間に起こっていること }
 \visible<2->{\item ---ingのつくり方
\begin{itemize}
 \item 原形にing(原則)
 \item 語尾のeをとって---ing\hfill{}make\,$\rightarrow$\, making, write\,$\rightarrow$\, writing, use\,$\rightarrow$\,using
 \item 最後の1文字を重ねて---ing\hfill{}swim\,$\rightarrow$\, swimming, run\,$\rightarrow$\, running, sit\,$\rightarrow$\,sitting
\end{itemize}}
\end{itemize}
 \end{block}
%%%%%%%%%%%
%%%%%%%%%%%
\begin{block}<3->{まとめ2}
\visible<4->{「動作」を表す動詞と「状態」をあらわす動詞があります}
\begin{itemize}
 \visible<5->{\item ほとんどは「動作」をあらわす動詞\hfill{}eat, drink,  run, walk, play \ldots}
 \visible<6->{\item 「状態」を表す動詞\hfill{}know, like, have \ldots}\\
\hfill{}\visible<7->{$\longrightarrow$\,\,「状態」をあらわす動詞は「現在進行形」になりません}\\
\hfill\visible<8->{{I know Tom.}}\\%
\hfill\visible<8->{{*I am knowing Tom.}}%
\end{itemize}
 \end{block}
\end{frame}
%%%%%%%%%%%%%%%%%%%%%%%%%%%%%%%%
\againframe<15>{how2makeIng}
\end{document}
