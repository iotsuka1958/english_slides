\documentclass[aspectratio=169]{beamer}
\usepackage[no-math,deluxe,haranoaji]{luatexja-preset}
\renewcommand{\kanjifamilydefault}{\gtdefault}
\renewcommand{\emph}[1]{{\upshape\bfseries #1}}
\usetheme{metropolis}
\metroset{block=fill}
\setbeamertemplate{navigation symbols}{}
\usecolortheme[rgb={0.7,0.2,0.2}]{structure}
%%%%%%%%%%%%%%%%%%%%%%%%%%%
%% さまざまなアイコン
%%%%%%%%%%%%%%%%%%%%%%%%%%%
\usepackage{fontawesome}
%%%%%%%%%%%%%%%%%%%%%%%%%%%
\usepackage{tikz}
%%%%%%%%%%%%%%%%%%%%%%%%%%%
%% 場合分け
\usepackage{cases}
%%%%%%%%%%%%%%%%%%%%%%%%%%%
%% 音声リンク表示
\newcommand{\myaudio}[1]{\href{#1}{\faVolumeUp}}
%%%%%%%%%%%%%%%%%%%%%%%%%%%
\title{English is fun.\,\,{}---be動詞を学びます---}
\author{}
\institute[]{}
\date[]

%%%%%%%%%%%%%%%%%%%%%%%%%%%%
%% TEXT
%%%%%%%%%%%%%%%%%%%%%%%%%%%%
\begin{document}
\begin{frame}[plain]
  \titlepage
\end{frame}

\section*{目次}
\begin{frame}[plain]
  \frametitle{授業の流れ}
  \tableofcontents
\end{frame}

\section{主語と動詞}
\begin{frame}<1-34>[plain]\frametitle{Aは〜する.}
 % \setbeamercovered{transparent}
  \begin{enumerate}
   \item<1-> \textbf<14>{\color<14>{red}{I}} \textbf<15>{\color<15>{blue}{like}} music. \onslide*<2>{わたしは音楽が好きです。}\onslide*<28>{\footnotesize like: 好き、好む music: 音楽}
   \item<3-> \textbf<16>{\color<16>{red}{We}} \textbf<17>{\color<17>{blue}{go}} to school by bus. \onslide*<4>{われわれはバスで通学します。}\onslide*<29>{\footnotesize by bus: バスで}
   \item<5-> \textbf<18>{\color<18>{red}{They}} \textbf<19>{\color<19>{blue}{speak}} English and Japanese. \onslide*<6>{彼らは英語と日本語を話す。}\onslide*<30>{\footnotesize speak: 話す}
   \item<7-> \textbf<20>{\color<20>{red}{I}} \textbf<21>{\color<21>{blue}{drink}} coffee every morning. \onslide*<8>{わたしは毎朝コーヒーを飲みます。}\onslide*<31>{\footnotesize drink: 飲む coffee: コーヒー every morning: 毎朝}
   \item<9-> \textbf<22>{\color<22>{red}{They}} \textbf<23>{\color<23>{blue}{study}} at the library. \onslide*<10>{彼らは図書館で勉強します。}\onslide*<32>{\footnotesize study: 勉強する at the library: 図書館で}
   \item<11-> \textbf<24>{\color<24>{red}{We}} \textbf<25>{\color<25>{blue}{eat}} bread for breakfast. \onslide*<12>{われわれは朝食にパンを食べる。}\onslide*<33>{\footnotesize eat: 食べる bread: パン for breakfast: 朝食に}
  \end{enumerate}

\bigskip

\begin{exampleblock}<27->{Topics for Today}
  英文の骨格は主語と動詞です
     \end{exampleblock}


% Embed the sound file
\onslide<34>{%
\myaudio{./audio/002_be_02.mp3}\,\,{}Listen carefully.(注意して聞いてください)
}

\end{frame}

\section{主語と動詞の順番}
\begin{frame}[plain]\frametitle{Aは〜する.}
 % \setbeamercovered{transparent}

\begin{columns}
\begin{column}[t]{.45\textwidth}
\begin{block}{日本語}
わたしは日本語を話します。\pause

日本語をわたしは話します。\pause

わたしは話します、日本語を。\pause

日本語を話します。
\end{block}
\end{column}
\pause
\begin{column}[t]{.45\textwidth}
\begin{block}{英語}
I speak Japanese.
\end{block}
\end{column}
\end{columns}


\bigskip
\pause
\begin{exampleblock}{Topics for Today}
\begin{itemize}
 \item   英文の骨格は主語と動詞です
 \item   英語は語順がだいじです
\end{itemize}
     \end{exampleblock}
\end{frame}



\end{document}
