\documentclass[aspectratio=169,xcolor={dvipsnames,table}]{beamer}
\usepackage[no-math,deluxe,haranoaji]{luatexja-preset}
\renewcommand{\kanjifamilydefault}{\gtdefault}
\renewcommand{\emph}[1]{{\upshape\bfseries #1}}
\usetheme{metropolis}
\metroset{block=fill}
\setbeamertemplate{navigation symbols}{}
\usecolortheme[rgb={0.7,0.2,0.2}]{structure}
%%%%%%%%%%%%%%%%%%%%%%%%%%%
%% さまざまなアイコン
%%%%%%%%%%%%%%%%%%%%%%%%%%%
\usepackage{fontawesome}
%%%%%%%%%%%%%%%%%%%%%%%%%%%
\usepackage{tikz}
\usepackage{xcolor}
\usepackage{circledsteps}
\usetikzlibrary{backgrounds}
\usepackage{tcolorbox}
\usepackage{tikzpeople}
%%%%%%%%%%%%%%%%%%%%%%%%%%%
%% 場合分け
\usepackage{cases}
%%%%%%%%%%%%%%%%%%%%%%%%%%%
% \myAnch{<名前>}{<色>}{<テキスト>}
% 指定のテキストを指定の色の四角枠で囲み, 指定の名前をもつTikZの
% ノードとして出力する. 図には remeber picture 属性を付けている
% ので外部から参照可能である.
\newcommand*{\myAnch}[3]{%
  \tikz[remember picture,baseline=(#1.base)]
    \node[draw,rectangle,#2] (#1) {\normalcolor #3};
}
%%%%%%%%%%%%%%%%%%%%%%%%%%%%
%%%%%%%%%%%%%%%%%%%%%%%%%%%
%% 音声リンク表示
\newcommand{\myaudio}[1]{\href{#1}{\faVolumeUp}}
%%%%%%%%%%%%%%%%%%%%%%%%%%%
% \myEmph コマンドの定義
\newcommand{\myEmph}[3]{%
    \textbf<#1>{\color<#1>{#2}{#3}}%
}
%%%%%%%%%%%%%%%%%%%%%%%%%%%
\title{English is fun.\,\,{}---主語と動詞---}
\author{}
\institute[]{}
\date[]

%%%%%%%%%%%%%%%%%%%%%%%%%%%%
%% TEXT
%%%%%%%%%%%%%%%%%%%%%%%%%%%%
\begin{document}
\begin{frame}[plain]
  \titlepage
\end{frame}

\section*{目次}
\begin{frame}[plain]
  \frametitle{授業の流れ}
  \tableofcontents
\end{frame}

\section{主語と動詞}

\begin{frame}<1-5>[plain]{「主語」とは、「動詞」とは}
\large

「だれだれは〜する」

\bigskip

\begin{enumerate}
 \item \alt<1>{\myAnch{s_1}{white}{わたしは}}{\myAnch{S_1}{Maroon}{わたしは}}英語を\alt<1>{\myAnch{v_1}{white}{話す}}{\myAnch{V_1}{NavyBlue}{話す}}。
 \item \alt<1-2>{\myAnch{s_2}{white}{あなたは}}{\myAnch{S_2}{Maroon}{あなたは}}寿司を\alt<1-2>{\myAnch{v_2}{white}{食べる}}{\myAnch{V_2}{NavyBlue}{食べる}}。
 \item \alt<1-3>{\myAnch{s_3}{white}{われわれは}}{\myAnch{S_3}{Maroon}{われわれは}}バスで職場に\alt<1-3>{\myAnch{v_3}{white}{行く}}{\myAnch{V_3}{NavyBlue}{行く}}。
 \item 毎朝、\alt<1-4>{\myAnch{s_4}{white}{彼らは}}{\myAnch{S_4}{Maroon}{彼らは}}コーヒーを\alt<1-4>{\myAnch{v_4}{white}{飲む}}{\myAnch{V_4}{NavyBlue}{飲む}}。
\end{enumerate}


\mbox{}\hfill \myAnch{S}{Maroon}{主語}\myAnch{V}{NavyBlue}{動詞}

 
\end{frame}


\begin{frame}<1-34>[plain]\frametitle{Aは〜する.}
 % \setbeamercovered{transparent}
  \begin{enumerate}
   \item<1-> \textbf<14>{\color<14>{red}{I}} \textbf<15>{\color<15>{blue}{like}} music. \onslide*<2>{わたしは音楽が好きです。}\hfill\onslide*<28>{\footnotesize like: 好き、好む music: 音楽}
   \item<3-> \textbf<16>{\color<16>{red}{We}} \textbf<17>{\color<17>{blue}{go}} to work by bus. \onslide*<4>{われわれはバスで職場に行く。}\hfill\onslide*<29>{\footnotesize work: 職場 by bus: バスで}
   \item<5-> \textbf<18>{\color<18>{red}{They}} \textbf<19>{\color<19>{blue}{speak}} English and Japanese. \onslide*<6>{彼らは英語と日本語を話す。}\hfill\onslide*<30>{\footnotesize speak: 話す}
   \item<7-> \textbf<20>{\color<20>{red}{I}} \textbf<21>{\color<21>{blue}{drink}} coffee every morning. \onslide*<8>{わたしは毎朝コーヒーを飲みます。}\hfill\onslide*<31>{\footnotesize drink: 飲む coffee: コーヒー every morning: 毎朝}
   \item<9-> \textbf<22>{\color<22>{red}{They}} \textbf<23>{\color<23>{blue}{study}} at the library. \onslide*<10>{彼らは図書館で勉強します。}\hfill\onslide*<32>{\footnotesize study: 勉強する at the library: 図書館で}
   \item<11-> \textbf<24>{\color<24>{red}{We}} \textbf<25>{\color<25>{blue}{eat}} bread for breakfast. \onslide*<12>{われわれは朝食にパンを食べる。}\hfill\onslide*<33>{\footnotesize eat: 食べる bread: パン for breakfast: 朝食に}
  \end{enumerate}

\bigskip

\begin{exampleblock}<27->{Topic for Today}
  英文の骨格は主語と動詞です
     \end{exampleblock}


% Embed the sound file
\onslide<34>{%
\myaudio{./audio/003_sv_01.mp3}\,\,{}Listen carefully.(注意して聞いてください)
}

\end{frame}

\section{主語と動詞の順番}
\begin{frame}[plain]\frametitle{Aは〜する.}
 % \setbeamercovered{transparent}

\begin{columns}
\begin{column}[t]{.45\textwidth}
\begin{block}{日本語}
わたしは日本語を話します。\pause

日本語をわたしは話します。\pause

わたしは話します、日本語を。\pause

日本語を話します。
\end{block}
\end{column}
\pause
\begin{column}[t]{.45\textwidth}
\begin{block}{英語}
I speak Japanese.
\end{block}
\end{column}
\end{columns}


\bigskip
\pause
\begin{exampleblock}{Topics for Today}
\begin{itemize}
 \item   英文の骨格は主語と動詞です
 \item   英語は語順がだいじです
\end{itemize}
     \end{exampleblock}
\end{frame}

\begin{frame}[plain]\frametitle{Exercises}
日本語を参考にして、主語と動詞を指摘してください。
\begin{enumerate}
    \item \alt<2->{\Circled[outer color=orange]{They}}{They} read a book. 彼らは本を読みます。
 \item \alt<2->{\Circled[outer color=orange]{I}}{I} play the piano. わたしはピアノを弾きます。
    \item \alt<2->{\Circled[outer color=orange]{They}}{They} watch TV. 彼らはテレビを見ます。
    \item \alt<2->{\Circled[outer color=orange]{We}}{We} study English. 私たちは英語を勉強します。
 \item \alt<2->{\Circled[outer color=orange]{I}}{I} write a letter. 私は手紙を書きます。
     \item \alt<2->{\Circled[outer color=orange]{Birds}}{Birds} sing. 鳥は歌います。
    \item \alt<2->{\Circled[outer color=orange]{My parents}}{My parents} swim well. 私の両親は上手に泳ぎます。
     \item \alt<2->{\Circled[outer color=orange]{The children}}{The children} play soccer. 子供たちはサッカーをします。
\end{enumerate}


\end{frame}

\end{document}
