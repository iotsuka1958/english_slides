\documentclass[aspectratio=169,xcolor={dvipsnames,table}]{beamer}
\usepackage[no-math,deluxe,expert,haranoaji]{luatexja-preset}
\usepackage{luatexja-otf}
\renewcommand{\kanjifamilydefault}{\gtdefault}
\renewcommand{\emph}[1]{{\upshape\bfseries #1}}
\usetheme{metropolis}
\metroset{block=fill}
\setbeamertemplate{navigation symbols}{}
\usecolortheme[rgb={0.7,0.2,0.2}]{structure}
%%%%%%%%%%%%%%%%%%%%%%%%%%%
\usepackage{media9}
%%%%%%%%%%%%%%%%%%%%%%%%%%%
%% さまざまなアイコン
%%%%%%%%%%%%%%%%%%%%%%%%%%%
\usepackage{fontawesome}
\usepackage{figchild}
\usepackage{twemojis}
\usepackage{utfsym}
\usepackage{bclogo}
\usepackage{marvosym}
\usepackage{fontmfizz}
\usepackage{pifont}
\usepackage{phaistos}
\usepackage{worldflags}
\usepackage{jigsaw}
%%%%%%%%%%%%%%%%%%%%%%%%%%%
\usepackage{tikz}
\usetikzlibrary{backgrounds}
\usepackage{tcolorbox}
\usepackage{tikzpeople}
\usepackage{circledsteps}
\usepackage{xcolor}
\usepackage{amsmath}
\usepackage{booktabs}
\usepackage{chronology}
\usepackage{signchart}
%%%%%%%%%%%%%%%%%%%%%%%%%%%
%% 場合分け
\usepackage{cases}
%%%%%%%%%%%%%%%%%%%%%%%%%%%
% \myAnch{<名前>}{<色>}{<テキスト>}
% 指定のテキストを指定の色の四角枠で囲み, 指定の名前をもつTikZの
% ノードとして出力する. 図には remeber picture 属性を付けている
% ので外部から参照可能である.
\newcommand*{\myAnch}[3]{%
  \tikz[remember picture,baseline=(#1.base)]
    \node[draw,rectangle,#2] (#1) {\normalcolor #3};
}
%%%%%%%%%%%%%%%%%%%%%%%%%%%%
%% 音声リンク表示
\newcommand{\myaudio}[1]{\href{#1}{\faVolumeUp}}
%%%%%%%%%%%%%%%%%%%%%%%%%%%
% \myEmph コマンドの定義
%\newcommand{\myEmph}[3]{%
%    \textbf<#1>{\color<#1>{#2}{#3}}%
%}
\usepackage{xparse} % xparseパッケージの読み込み
\NewDocumentCommand{\myEmph}{O{} m m}{%
    \def\argOne{#1}%
    \ifx\argOne\empty
        \textbf{\color{#2}{#3}}% オプション引数が省略された場合
    \else
        \textbf<#1>{\color<#1>{#2}{#3}}% オプション引数が指定された場合
    \fi
}
%%%%%%%%%%%%%%%%%%%%%%%%%%%
%% 文末の上昇イントネーション記号 \myRisingPitch
%% 通常のイントネーション \myDownwardPitch
%% https://note.com/dan_oyama/n/n8be58e8797b2
%%%%%%%%%%%%%%%%%%%%%%%%%%%
\newcommand{\myRisingPitch}{
\begin{tikzpicture}[scale=0.3,baseline=0.3]
\draw[->,>=stealth] (0,0) to[bend right=45] (1,1);
\end{tikzpicture}
}
\newcommand{\myDownwardPitch}{
\begin{tikzpicture}[scale=0.3,baseline=0.3]
\draw[->,>=stealth] (0,1) to[bend left=45] (1,0);
\end{tikzpicture}
}
%%%%%%%%%%%%%%%%%%%%%%%%%%%
\title{English is fun.\,\,{}---ABCDEFG ... XYZ ---}
  \author{}
\institute[]{}
\date[]

%%%%%%%%%%%%%%%%%%%%%%%%%%%%
%% TEXT
%%%%%%%%%%%%%%%%%%%%%%%%%%%%
\begin{document}

\begin{frame}{}
\phantomsection\label{section}
\thispagestyle{empty}
\Large

\raggedright

予定の時刻になったらはじまります

\textbullet  音声を流しています

\textbullet  聞こえていますか 

\vfill

\raggedleft

The lesson will begin at the scheduled time.

\vspace{-6pt}

We are playing audio.

\vspace{-6pt}

Can you hear it?
\end{frame}

\begin{frame}
\phantomsection\label{section-1}
\thispagestyle{empty}
\titlepage
\end{frame}


\section*{授業の流れ}
\begin{frame}[plain]
  \frametitle{授業の流れ}
  \tableofcontents
\end{frame}

\section{alphabet}
\subsection{alphabet}
\begin{frame}[plain]{アルファベット}
\Large


日本語は\pause

\begin{itemize}
 \item ひらがな・カタカナ
 \item 漢字
\end{itemize}

\pause
いっぽう英語は\pause
\begin{itemize}
 \item アルファベットは26種類
 \item 大文字と小文字があります
\end{itemize}

\pause
ということは、ぜんぶで
$26\times2=52$


\end{frame}



\begin{frame}[plain]{アルファベット}
\Huge
\begin{rmfamily}\bfseries
\setbeamercovered{transparent}
\begin{tabular}{cccccccccc}
\onslide<2,28->{A}&
\onslide<3,28->{B}&
\onslide<4,28->{C}&
\onslide<5,28->{D}&
\onslide<6,28->{E}&
\onslide<7,28->{F}&
\onslide<8,28->{G}&
\onslide<9,28->{H}&
\onslide<10,28->{I}&
\onslide<11,28->{J} \\
\onslide<12,28->{K}&
\onslide<13,28->{L}&
\onslide<14,28->{M}&
\onslide<15,28->{N}&
\onslide<16,28->{O}&
\onslide<17,28->{P}&
\onslide<18,28->{Q}&
\onslide<19,28->{R}&
\onslide<20,28->{S}&
\onslide<21,28->{T}\\
\onslide<22,28->{U}&
\onslide<23,28->{V}&
\onslide<24,28->{W}&
\onslide<25,28->{X}&
\onslide<26,28->{Y}&
\onslide<27,28->{Z}&
 & & &  \\
\end{tabular}
\end{rmfamily}


 \begin{exampleblock}<29->{Topics for Today}
\small
\begin{itemize}
 \item  アルファベットは26文字です
\item 何度も練習しましょう
\end{itemize}
      \end{exampleblock}

\mbox{}\hfill\visible{\myaudio{./audio/001_alphabet_01.mp3}}
\end{frame}


\begin{frame}[plain]{アルファベット}
\Huge

\begin{rmfamily}\bfseries
\setbeamercovered{transparent}
\begin{tabular}{cccccccccc}
\onslide<2,28->{a}&
\onslide<3,28->{b}&
\onslide<4,28->{c}&
\onslide<5,28->{d}&
\onslide<6,28->{e}&
\onslide<7,28->{f}&
\onslide<8,28->{g}&
\onslide<9,28->{h}&
\onslide<10,28->{i}&
\onslide<11,28->{j} \\
\onslide<12,28->{k}&
\onslide<13,28->{l}&
\onslide<14,28->{m}&
\onslide<15,28->{n}&
\onslide<16,28->{o}&
\onslide<17,28->{p}&
\onslide<18,28->{q}&
\onslide<19,28->{r}&
\onslide<20,28->{s}&
\onslide<21,28->{t}\\
\onslide<22,28->{u}&
\onslide<23,28->{v}&
\onslide<24,28->{w}&
\onslide<25,28->{x}&
\onslide<26,28->{y}&
\onslide<27,28->{z}&
 & & &  \\
\end{tabular}
\end{rmfamily}


 \begin{exampleblock}<29->{Topics for Today}
\small
\begin{itemize}
 \item  アルファベットは26文字です
\item 何度も練習しましょう
\end{itemize}
      \end{exampleblock}
\end{frame}



\end{document}
