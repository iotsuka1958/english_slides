\documentclass[aspectratio=169,xcolor={dvipsnames,table}]{beamer}
\usepackage[no-math,deluxe,haranoaji]{luatexja-preset}
\renewcommand{\kanjifamilydefault}{\gtdefault}
\renewcommand{\emph}[1]{{\upshape\bfseries #1}}
\usetheme{metropolis}
\metroset{block=fill}
\setbeamertemplate{navigation symbols}{}
\setbeamertemplate{blocks}[rounded][shadow=false]
\usecolortheme[rgb={0.7,0.2,0.2}]{structure}
%%%%%%%%%%%%%%%%%%%%%%%%%%
%% Change alert block colors
%%% 1- Block title (background and text)
\setbeamercolor{block title alerted}{fg=mDarkTeal, bg=mLightBrown!45!yellow!45}
\setbeamercolor{block title example}{fg=magenta!10!black, bg=mLightGreen!70}
%%% 2- Block body (background)
\setbeamercolor{block body alerted}{bg=mLightBrown!25}
\setbeamercolor{block body example}{bg=mLightGreen!15}
%%%%%%%%%%%%%%%%%%%%%%%%%%%
%%%%%%%%%%%%%%%%%%%%%%%%%%%
%% さまざまなアイコン
%%%%%%%%%%%%%%%%%%%%%%%%%%%
%\usepackage{fontawesome}
\usepackage{fontawesome5}
\usepackage{figchild}
\usepackage{twemojis}
\usepackage{utfsym}
\usepackage{bclogo}
\usepackage{marvosym}
\usepackage{fontmfizz}
\usepackage{pifont}
\usepackage{phaistos}
\usepackage{worldflags}
\usepackage{jigsaw}
\usepackage{tikzlings}
\usepackage{tikzducks}
\usepackage{scsnowman}
\usepackage{epsdice}
\usepackage{halloweenmath}
\usepackage{svrsymbols}
\usepackage{countriesofeurope}
\usepackage{tipa}
\usepackage{manfnt}
%%%%%%%%%%%%%%%%%%%%%%%%%%%
\usepackage{tikz}
\usetikzlibrary{calc,patterns,decorations.pathmorphing,backgrounds}
\usepackage{tcolorbox}
\usepackage{tikzpeople}
\usepackage{circledsteps}
\usepackage{xcolor}
\usepackage{amsmath}
\usepackage{booktabs}
\usepackage{chronology}
\usepackage{signchart}
%%%%%%%%%%%%%%%%%%%%%%%%%%%
%% 場合分け
%%%%%%%%%%%%%%%%%%%%%%%%%%%
\usepackage{cases}
%%%%%%%%%%%%%%%%%%%%%%%%%%
\usepackage{pdfpages}
%%%%%%%%%%%%%%%%%%%%%%%%%%%
%% 音声リンク表示
\newcommand{\myaudio}[1]{\href{#1}{\faVolumeUp}}
%%%%%%%%%%%%%%%%%%%%%%%%%%
%% \myAnch{<名前>}{<色>}{<テキスト>}
%% 指定のテキストを指定の色の四角枠で囲み, 指定の名前をもつTikZの
%% ノードとして出力する. 図には remember picture 属性を付けている
%% ので外部から参照可能である.
\newcommand*{\myAnch}[3]{%
  \tikz[remember picture,baseline=(#1.base)]
    \node[draw,rectangle,line width=1pt,#2] (#1) {\normalcolor #3};
}
%%%%%%%%%%%%%%%%%%%%%%%%%%
%% \myEmph コマンドの定義
%%%%%%%%%%%%%%%%%%%%%%%%%%
%\newcommand{\myEmph}[3]{%
%    \textbf<#1>{\color<#1>{#2}{#3}}%
%}
\usepackage{xparse} % xparseパッケージの読み込み
\NewDocumentCommand{\myEmph}{O{} m m}{%
    \def\argOne{#1}%
    \ifx\argOne\empty
        \textbf{\color{#2}{#3}}% オプション引数が省略された場合
    \else
        \textbf<#1>{\color<#1>{#2}{#3}}% オプション引数が指定された場合
    \fi
}
%%%%%%%%%%%%%%%%%%%%%%%%%%%
%%%%%%%%%%%%%%%%%%%%%%%%%%%
%% 文末の上昇イントネーション記号 \myRisingPitch
%% 通常のイントネーション \myDownwardPitch
%% https://note.com/dan_oyama/n/n8be58e8797b2
%%%%%%%%%%%%%%%%%%%%%%%%%%%
\newcommand{\myRisingPitch}{
\begin{tikzpicture}[scale=0.3,baseline=0.3]
\draw[->,>=stealth] (0,0) to[bend right=45] (1,1);
\end{tikzpicture}
}
\newcommand{\myDownwardPitch}{
\begin{tikzpicture}[scale=0.3,baseline=0.3]
\draw[->,>=stealth] (0,1) to[bend left=45] (1,0);
\end{tikzpicture}
}
%%%%%%%%%%%%%%%%%%%%%%%%%%%%
%\AtBeginSection[%
%]{%
%  \begin{frame}[plain]\frametitle{授業の流れ}
%     \tableofcontents[currentsection]
%   \end{frame}%
%}

%%%%%%%%%%%%%%%%%%%%%%%%%%%
\title{English is fun.}
\subtitle{Open the door.
}
\author{}
\institute[]{}
\date[]

%%%%%%%%%%%%%%%%%%%%%%%%%%%%
%% TEXT
%%%%%%%%%%%%%%%%%%%%%%%%%%%%
\begin{document}

\begin{frame}[plain]
  \titlepage
\end{frame}

\section*{授業の流れ}
\begin{frame}[plain]
  \frametitle{授業の流れ}
  \tableofcontents
\end{frame}

\section{命令文}
\subsection{〜しなさい}
%%%%%%%%%%%%%%%%%%%%%%%%%%%%%%%%%%%%%%%%%%%%%
\begin{frame}[plain]{~しなさい}
 \large


\begin{enumerate}
\item<2-> 一般動詞のとき
      \begin{enumerate}
	\item<3-> \myEmph[3-]{NavyBlue}{Come} here.
	\item<4-> \myEmph[4-]{NavyBlue}{Stand} up.
	\item<5-> \myEmph[5-]{NavyBlue}{Have} a good day.
      \end{enumerate}
 \item<6-> be動詞のとき
       \begin{enumerate}
	\item<7-> \myEmph[7-]{Maroon}{Be} quiet.  
	\item<8-> \myEmph[8-]{Maroon}{Be} careful.
	\item<9-> \myEmph[9-]{Maroon}{Be} nice to others.
       \end{enumerate}\end{enumerate}

\visible<10->{%
\begin{exampleblock}{Topic for Today}
\begin{itemize}\small
 \item 「~しなさい」と命令を表すには、\Circled[fill color=white]{ 動詞の原形 }\,ではじめる
 \end{itemize}
     \end{exampleblock}
}
\end{frame}
%%%%%%%%%%%%%%%%%%%%%%%%%%%%%%%%%%%%%%%%%%%%%%
\begin{frame}[plain]{Exercises}
 \begin{enumerate}
  \item ( Standing / Stands / \alt<2->{\Circled[outer color=BurntOrange]{Stand}}{~~Stand~~} ) up, please.\hfill{\scriptsize please: どうぞ}\\
  \item Please ( coming / comes / \alt<3->{\Circled[outer color=BurntOrange]{come}}{~~come~~} ) here.\hfill\visible<4->{Come here, please.}
  \item ( Am / \alt<5->{\Circled[outer color=BurntOrange]{Be}}{~~Be~~} / Is ) careful.
  \item ( Looking / \alt<6->{\Circled[outer color=BurntOrange]{Look}}{~~Look~~} / Look ) at the blackboard.\hfill{\scriptsize blackboard: 黒板}
  \item ( Cleaning / Cleans / \alt<7->{\Circled[outer color=BurntOrange]{Clean}}{~~Clean~~} ) your room right now. \hfill{\scriptsize right now: いますぐ}
 \end{enumerate}
\visible<8->{%
\begin{exampleblock}{Topics for Today}
\begin{itemize}\small
 \item 「~しなさい」と命令を表すには、\Circled[fill color=white]{ 動詞の原形 }\,ではじめる
 \item Please $+$ 命令文. $\Longleftrightarrow$ 命令文 $+$ , please. 
 \end{itemize}
     \end{exampleblock}
}
\end{frame}
%%%%%%%%%%%%%%%%%%%%%%%%%%%%%%%%%%%%%%%%%%%%%%%%
\begin{frame}[plain]{Exercises}
 \begin{enumerate}
  \item あなたの名前を書いてください。\\
	Please ( name / your / write ).\\
	\visible<2->{Please write your name.}
  \item ここで走ってはいけません。\\
	( run / don't / here ), please.\\
	\visible<3->{Don't run here, please.}
  \item 試合を始めなさい。\\
	( game / the/ begin ).\\
	\visible<4->{Begin the game.}
  \item お年寄りには親切にしなさい。\\
	( kind / to / be ) old people.\\
	\visible<5->{Be kind to old people.}
 \end{enumerate}
\end{frame}
%%%%%%%%%%%%%%%%%%%%%%%%%%%%%%%%%%%%%%%%%%%%%%%%%
\begin{frame}[plain]{~するな}
\begin{enumerate}
 \item \begin{enumerate}
	\item<1-> \myEmph[1-]{NavyBlue}{Run}.
	\item<2-> \myEmph[2-]{Maroon}{Don't} \myEmph[2-]{NavyBlue}{run}.
       \end{enumerate} 
 \item \begin{enumerate}
	\item<3-> \myEmph[3-]{NavyBlue}{Be} quiet.
	\item<4-> \myEmph[4-]{Maroon}{Don't} \myEmph[4-]{NavyBlue}{be} noisy.\hfill\visible<4->{\scriptsize noisy: うるさい}
       \end{enumerate}
\end{enumerate}
\visible<5->{%
\begin{exampleblock}{Topic for Today}
\begin{itemize}\small
 \item 「~するな」と否定の命令を表すには、Don't $+$ \Circled[fill color=white]{ 動詞の原形 }
 \end{itemize}
     \end{exampleblock}
}
\end{frame}
%%%%%%%%%%%%%%%%%%%%%%%%%%
\begin{frame}[plain]{Exercises}
日本語の意味になるよう(~~~~~~)内の語句を並べ替えましょう。先頭に来る語は大文字で書き始めてください。なお $[ +1 ]$とある場合は不足している1語を補ってください。$[ -1 ]$とある場合は余計な1語が含まれています。
 \begin{enumerate}
  \item ここでは携帯電話を使わないでください。%
	\hfill{\scriptsize cell phone: 携帯電話}\\
	( your cell phone / here / use ) $[ +1 ]$\\
	\visible<2->{Don't use your cell phone here.}
  \item 遅刻しないで。\\
	( be / don't / late / aren't ) $[ -1 ]$\\
	\visible<3->{Don't be late. (✕ aren't)}
  \item 走らないでください。\\
	( not / please / run / don't ) $[ -1 ]$\\
	\visible<4->{Please don't run.またはDon't run, please. (✕not)}
 \end{enumerate}
\end{frame}
%%%%%%%%%%%%%%%%%%%%%%%%%%%%%%%%
\begin{frame}[plain]{~しましょう}
 \begin{enumerate}
  \item<1-> \myEmph[1-]{Maroon}{Let's} \myEmph[1-]{NavyBlue}{dance}.
  \item<2-> \myEmph[2-]{Maroon}{Let's} \myEmph[2-]{NavyBlue}{begin}.
  \item<3-> \myEmph[3-]{Maroon}{Let's} \myEmph[3-]{NavyBlue}{eat} lunch. 
  \item<4-> \myEmph[4-]{Maroon}{Let's} \myEmph[4-]{NavyBlue}{go} to the park.
  \item<5-> \myEmph[4-]{Maroon}{Let's} \myEmph[4-]{NavyBlue}{play} basketball after school.
 \end{enumerate}

\visible<6->{%
\begin{exampleblock}{Topic for Today}
\begin{itemize}\small
 \item 「~しましょう」と相手を誘うときは、Let's $+$ \Circled[fill color=white]{ 動詞の原形 }
 \end{itemize}
     \end{exampleblock}
}
\end{frame}
%%%%%%%%%%%%%%%%%%%%%%%%%%%%
\begin{frame}[plain]{Exercises}
つぎの対話の意味を考えましょう

\begin{tabular}{rp{.9\textwidth}}
John:& Hi, Jane. Do you like sports?\\
Jane:& Not really, but I enjoy reading. How about you?\footnote{How about you?: あなたはどうですか}\\
John:& I like tennis. It's my favorite sport. Let's play tennis this weekend!\\
Jane:& That sounds fun\footnote{That sounds fun.:おもしろそうですね}, but I also want to read a new book. Let's read together after playing.\\
John:& Great idea! Let's meet at the park at 10 a.m.\\
Jane:& Perfect! I'll bring the book.
\end{tabular}

\end{frame}
\end{document}
