\documentclass[aspectratio=169,xcolor={dvipsnames,table}]{beamer}
\usepackage[no-math,deluxe,haranoaji]{luatexja-preset}
\renewcommand{\kanjifamilydefault}{\gtdefault}
\renewcommand{\emph}[1]{{\upshape\bfseries #1}}
\usetheme{metropolis}
\metroset{block=fill}
\setbeamertemplate{navigation symbols}{}
\usecolortheme[rgb={0.7,0.2,0.2}]{structure}
%%%%%%%%%%%%%%%%%%%%%%%%%%%
\usepackage{media9}
%%%%%%%%%%%%%%%%%%%%%%%%%%%
%% さまざまなアイコン
%%%%%%%%%%%%%%%%%%%%%%%%%%%
\usepackage{fontawesome}
\usepackage{figchild}
\usepackage{twemojis}
\usepackage{utfsym}
\usepackage{bclogo}
\usepackage{marvosym}
\usepackage{fontmfizz}
\usepackage{pifont}
\usepackage{phaistos}
\usepackage{worldflags}
%%%%%%%%%%%%%%%%%%%%%%%%%%%
\usepackage{tikz}
\usetikzlibrary{backgrounds}
\usepackage{tcolorbox}
\usepackage{xcolor}
\usepackage{amsmath}
%%%%%%%%%%%%%%%%%%%%%%%%%%%
%% 場合分け
\usepackage{cases}
%%%%%%%%%%%%%%%%%%%%%%%%%%%
% \myAnch{<名前>}{<色>}{<テキスト>}
% 指定のテキストを指定の色の四角枠で囲み, 指定の名前をもつTikZの
% ノードとして出力する. 図には remeber picture 属性を付けている
% ので外部から参照可能である.
\newcommand*{\myAnch}[3]{%
  \tikz[remember picture,baseline=(#1.base)]
    \node[draw,rectangle,#2] (#1) {\normalcolor #3};
}
%%%%%%%%%%%%%%%%%%%%%%%%%%%%
%% 音声リンク表示
\newcommand{\myaudio}[1]{\href{#1}{\faVolumeUp}}
%%%%%%%%%%%%%%%%%%%%%%%%%%%
% \myEmph コマンドの定義
%\newcommand{\myEmph}[3]{%
%    \textbf<#1>{\color<#1>{#2}{#3}}%
%}
\usepackage{xparse} % xparseパッケージの読み込み
\NewDocumentCommand{\myEmph}{O{} m m}{%
    \def\argOne{#1}%
    \ifx\argOne\empty
        \textbf{\color{#2}{#3}}% オプション引数が省略された場合
    \else
        \textbf<#1>{\color<#1>{#2}{#3}}% オプション引数が指定された場合
    \fi
}
%%%%%%%%%%%%%%%%%%%%%%%%%%%
%% 文末の上昇イントネーション記号 \myRisingPitch
%% 通常のイントネーション \myDownwardPitch
%% https://note.com/dan_oyama/n/n8be58e8797b2
%%%%%%%%%%%%%%%%%%%%%%%%%%%
\newcommand{\myRisingPitch}{
\begin{tikzpicture}[scale=0.3,baseline=0.3]
\draw[->,>=stealth] (0,0) to[bend right=45] (1,1);
\end{tikzpicture}
}
\newcommand{\myDownwardPitch}{
\begin{tikzpicture}[scale=0.3,baseline=0.3]
\draw[->,>=stealth] (0,1) to[bend left=45] (1,0);
\end{tikzpicture}
}
%%%%%%%%%%%%%%%%%%%%%%%%%%%
\title{English is fun.\,\,{}--- He is not  reading a book now. ---}
\author{}
\institute[]{}
\date[]

%%%%%%%%%%%%%%%%%%%%%%%%%%%%
%% TEXT
%%%%%%%%%%%%%%%%%%%%%%%%%%%%
\begin{document}


\begin{frame}[plain]
  \titlepage
\end{frame}


\section*{授業の流れ}
\begin{frame}[plain]
  \frametitle{授業の流れ}
  \tableofcontents
\end{frame}



\section{現在進行形の否定}

\subsection{復習}
\begin{frame}[plain]{復習}

つぎの各文を否定文にしてください。


 % \setbeamercovered{transparent}
  \begin{enumerate}
   \item \visible<1->{His mother is a teacher.}\\
         \visible<2->{His mother \textcolor{orange}{is not} a teacher.} 
         \,\,\,\visible<3->{His mother \textcolor{orange}{isn't} a teacher.}%
         \hfill{}\visible<4->{($\text{is not} = \text{isn't}$)}
   \item \visible<1->{The room is clean.}\\
         \visible<5->{The room \textcolor{orange}{is not} clean.}
         \,\,\,\visible<6->{The room \textcolor{orange}{isn't} clean.}%
   \item \visible<1->{You are busy.}\\
         \visible<7->{You \textcolor{orange}{are not} busy.}
         \,\,\,\visible<8->{You \textcolor{orange}{aren't} busy.}%
          \hfill{}\visible<9->{($\text{are not} = \text{aren't}$)}
   \item \visible<1->{They are students.}\\
         \visible<10->{They \textcolor{orange}{are not} students.} 
         \,\,\,\visible<11->{They \textcolor{orange}{aren't} students.}%
         
   \item \visible<1->{I am a doctor.}\\
         \visible<12->{I \textcolor{orange}{am not} a doctor.}
         \,\,\,\visible<13->{*I \textcolor{olive}{amn't} a doctor.}\,\,\,{}%
         \visible<14->{$\longleftarrow$こうはいいません}
  \end{enumerate}

% Embed the sound file
\visible<15->{%
\myaudio{audio/021_is_ing_negative_01.mp3}
}
\end{frame}


\begin{frame}[plain]\frametitle{復習}
%
       \begin{exampleblock}{Topics for Today}
\begin{itemize}
 \item  be動詞の否定は、$\text{be動詞} + \text{not}$
 \item  縮めて aren't($=\text{are not}$)、isn't($=\text{is not}$) ということもあります\\
\mbox{}\hfill{}これを「短縮形」といいます
% \item  She\textcolor{orange}{'s not} ということもあります
\end{itemize}
      \end{exampleblock}

\end{frame}


\begin{frame}[plain]{現在進行形の否定}
\Large

\visible<1->{They are studying now.}\hfill\visible<2->{{\small study:勉強する}}

\visible<3->{They are \textcolor{orange}{not} studying now.} 

\end{frame}

\begin{frame}[plain]{Exercises}
あたえられた日本文の意味になるように、空所に適切な単語を補いましょう。

\begin{enumerate}
 \item わたしは、いま読書をしていません。\\
I \alt<2->{(\,\,am\,\,)~~(\,\,not\,\,)}{(\phantom{\,\,am\,\,})~~(\phantom{\,\,not\,\,})} reading a book now.
 \item 彼女は、いま歌を歌っていません。\\
She \alt<3->{(\,\,is\,\,)~~(\,\,not\,\,)}{(\phantom{\,\,is\,\,})~~(\phantom{\,\,not\,\,})} singing a song now.
 \item わたしたちは、いま宿題をしていません。\\
We \alt<4->{(\,\,are\,\,)~~(\,\,not\,\,)}{(\phantom{\,\,are\,\,})~~(\phantom{\,\,not\,\,})} doing homework now.
 \item わたしたちのネコは、いまベッドの上で寝ていません。\\
Our cat \alt<5->{(\,\,isn't\,\,)}{(\phantom{\,\,isn't\,\,})} sleeping on the bed now.
 \item わたしたちは、いま数学の勉強をしてていません。\\
We \alt<6->{(\,\,aren't\,\,)}{(\phantom{\,\,aren't\,\,})} studying math now.
\end{enumerate} 


\visible<6->{%
\mbox{}\hfill\myaudio{./audio/021_is_ing_negative_02.mp3}
}
\end{frame}

 \begin{frame}[plain]{Exercises}
日本文を参考にして (~~~~~~) 内の語を並べかえ、英文を完成させましょう。  

\vspace{-5pt}

\begin{enumerate}
 \item \visible<1->{{\small ジョンは、いまサッカーをしていません。}}\\\visible<1->{John ( playing / not / is ) soccer now.}
\visible<2->{$\longrightarrow$\,\,\,\,John \textcolor{orange}{is not playing} soccer now.}
 \item \visible<1->{{\small 彼らは、いまプールで泳いでいません。}}\\
\visible<1->{They ( not / swimming / are) in the pool now.}\\
\visible<3->{$\longrightarrow$\,\,\,\,They \textcolor{orange}{are not swimming} in the pool now.}
 \item \visible<1->{{\small その犬は、いま庭を走っていません。}}\\
\visible<1->{The dog ( running / not / is ) in the yard.}\\
\visible<4->{$\longrightarrow$\,\,\,\,The dog \textcolor{orange}{is not running} in the yard.}
 \item \visible<1->{{\small 彼女は、いま手紙を書いているのではありません。}}\\
\visible<1->{She ( a letter / not / writing / is ) now.}
\visible<5->{$\longrightarrow$\,\,\,\,She \textcolor{orange}{is not  writing a letter} now.}
 \item \visible<1->{{\small 彼はいま、ケーキを作っているのではありません。}}\\
\visible<1->{He (making / not / is ) cake now.}
\visible<6->{$\longrightarrow$\,\,\,\,He \textcolor{orange}{is not making} cake now.}
\end{enumerate}


 \end{frame}





\end{document}
