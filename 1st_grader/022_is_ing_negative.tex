\documentclass[aspectratio=169,xcolor={dvipsnames,table}]{beamer}
\usepackage[no-math,deluxe,haranoaji]{luatexja-preset}
\renewcommand{\kanjifamilydefault}{\gtdefault}
\renewcommand{\emph}[1]{{\upshape\bfseries #1}}
\usetheme{metropolis}
\metroset{block=fill}
\setbeamertemplate{navigation symbols}{}
\setbeamertemplate{blocks}[rounded][shadow=false]
\usecolortheme[rgb={0.7,0.2,0.2}]{structure}
%%%%%%%%%%%%%%%%%%%%%%%%%%%
\usepackage{media9}
%%%%%%%%%%%%%%%%%%%%%%%%%%%
%% さまざまなアイコン
%%%%%%%%%%%%%%%%%%%%%%%%%%%
\usepackage{fontawesome}
%\usepackage{figchild}
\usepackage{twemojis}
\usepackage{utfsym}
\usepackage{bclogo}
\usepackage{marvosym}
\usepackage{fontmfizz}
\usepackage{pifont}
\usepackage{phaistos}
%\usepackage{worldflags}
\usepackage{circledsteps}
\usepackage{tipa,manfnt}
%%%%%%%%%%%%%%%%%%%%%%%%%%%
\usepackage{tikz}
%\usetikzlibrary{backgrounds}
\usepackage{tcolorbox}
\usepackage{tikzducks}
\usepackage{xcolor}
\usepackage{amsmath}
\usepackage{tabularray}
\UseTblrLibrary{booktabs}
%%%%%%%%%%%%%%%%%%%%%%%%%%
\usepackage[absolute,overlay]{textpos}  %% 任意の位置に図を配置
%\usepackage[colorgrid,gridunit=pt,texcoord]{eso-pic} %%場所決めのための格子
%%%%%%%%%%%%%%%%%%%%%%%%%%%
%% 場合分け
\usepackage{cases}
%%%%%%%%%%%%%%%%%%%%%%%%%%%
% \myAnch{<名前>}{<色>}{<テキスト>}
% 指定のテキストを指定の色の四角枠で囲み, 指定の名前をもつTikZの
% ノードとして出力する. 図には remeber picture 属性を付けている
% ので外部から参照可能である.
\newcommand*{\myAnch}[3]{%
  \tikz[remember picture,baseline=(#1.base)]
    \node[draw,rectangle,#2] (#1) {\normalcolor #3};
}
%%%%%%%%%%%%%%%%%%%%%%%%%%%%
%% 音声リンク表示
\newcommand{\myaudio}[1]{\href{#1}{\faVolumeUp}}
%%%%%%%%%%%%%%%%%%%%%%%%%%%
% \myEmph コマンドの定義
%\newcommand{\myEmph}[3]{%
%    \textbf<#1>{\color<#1>{#2}{#3}}%
%}
\usepackage{xparse} % xparseパッケージの読み込み
\NewDocumentCommand{\myEmph}{O{} m m}{%
    \def\argOne{#1}%
    \ifx\argOne\empty
        \textbf{\color{#2}{#3}}% オプション引数が省略された場合
    \else
        \textbf<#1>{\color<#1>{#2}{#3}}% オプション引数が指定された場合
    \fi
}
%%%%%%%%%%%%%%%%%%%%%%%%%%%
%% 文末の上昇イントネーション記号 \myRisingPitch
%% 通常のイントネーション \myDownwardPitch
%% https://note.com/dan_oyama/n/n8be58e8797b2
%%%%%%%%%%%%%%%%%%%%%%%%%%%
\newcommand{\myRisingPitch}{
\begin{tikzpicture}[scale=0.3,baseline=0.3]
\draw[->,>=stealth] (0,0) to[bend right=45] (1,1);
\end{tikzpicture}
}
\newcommand{\myDownwardPitch}{
\begin{tikzpicture}[scale=0.3,baseline=0.3]
\draw[->,>=stealth] (0,1) to[bend left=45] (1,0);
\end{tikzpicture}
}
%%%%%%%%%%%%%%%%%%%%%%%%%%%
\title{English is fun.}
\subtitle{She is not reading a book.}
\author{}
\institute[]{}
\date[]

%%%%%%%%%%%%%%%%%%%%%%%%%%%%
%% TEXT
%%%%%%%%%%%%%%%%%%%%%%%%%%%%
\begin{document}
%%%%%%%%%%%%%%%%%%%%%%%%%%%%
\begin{frame}[plain]
  \titlepage
\end{frame}
%%%%%%%%%%%%%%%%%%%%%%%%%%%%
\section*{授業の流れ}
%%%%%%%%%%%%%%%%%%%%%%%%%%%%
\begin{frame}[plain]
  \frametitle{授業の流れ}
  \tableofcontents
\end{frame}
%%%%%%%%%%%%%%%%%%%%%%%%%%
\section{否定を表すnot}
%%%%%%%%%%%%%%%%%%%%%%%%%%
\begin{frame}[plain]{否定を表すnot}
 \Large

否定を表すことば: {\LARGE\bfseries not}\hspace{20pt}\textipa{/n\'At/}
\end{frame}
%%%%%%%%%%%%%%%%%%%%%%%%%%
\section{be動詞の否定(復習)}
%%%%%%%%%%%%%%%%%%%%%%%%%%
\begin{frame}[plain]{be動詞の否定(復習)}
つぎの各文をnotを用いて否定文にしてください(短縮形も可)

 % \setbeamercovered{transparent}
  \begin{enumerate}
   \item His mother is a teacher.\\
         \visible<2->{\,\,\,\,\,$\longrightarrow$\,His mother \textcolor{Maroon}{\bfseries is not} a teacher.}\\
	 \hfill\visible<3->{His mother\textbf{'s} \textcolor{Maroon}{\bfseries not} a teacher. / His mother \textcolor{Maroon}{\bfseries isn't} a teacher.}
   \item The room is clean.\\
         \visible<4->{\,\,\,\,\,$\longrightarrow$\,The room \textcolor{Maroon}{\bfseries is not} clean.}
	 \visible<5->{/ The room\textbf{'s} \textcolor{Maroon}{\bfseries not} clean. / The room \textcolor{Maroon}{\bfseries isn't} clean.}
   \item You are busy.\\
         \visible<6->{\,\,\,\,\,$\longrightarrow$\,You \textcolor{Maroon}{\bfseries are not} busy}.
	 \visible<7->{/ You\textbf{'re} \textcolor{Maroon}{\bfseries not} busy. / You \textcolor{Maroon}{\bfseries aren't} busy.}
   \item They are students.\\
         \visible<8->{\,\,\,\,\,$\longrightarrow$\,They \textcolor{Maroon}{\bfseries are not} students.}
	 \visible<9->{/ They\textbf{'re} \textcolor{Maroon}{\bfseries not} students. / They \textcolor{Maroon}{\bfseries aren't} students.}
   \item I am a doctor.\\
	 \visible<10->{\,\,\,\,\,$\longrightarrow$\,I \textcolor{Maroon}{\bfseries am not} a doctor.}
	 \visible<11->{/ I\textbf{'m} \textcolor{Maroon}{\bfseries not} a doctor.}
	 \visible<12->{/ *I \textcolor{NavyBlue}{amn't} a doctor.}
  \end{enumerate}

% Embed the sound file
\hfill{\tiny 0447}\,{\scriptsize \myaudio{audio/022_is_ing_negative_00.mp3}}
\end{frame}
%%%%%%%%%%%%%%%%
\begin{frame}[plain]\frametitle{be動詞の否定(復習)}
\begin{block}<1->{要点}
\begin{itemize}\setbeamertemplate{items}[square]
 \item<1->  be動詞の否定は\,\,\Circled[fill color = white]{\,\,$\text{be動詞} + \text{\textcolor{Maroon}{\bfseries not}}$\,\,}
 \item<2-> 短縮形が用いられることもあります
       \begin{enumerate}\setbeamertemplate{items}[circle]
	\item<3-> \temporal<4,5>{She is not}{\Circled{She is} not}{She \Circled{is not}} busy.
	\item<5-> \textbf{She's} not busy.
	\item<7-> She \textbf{isn't} busy.
       \end{enumerate}
\end{itemize}
\end{block}
\end{frame}
%%%%%%%%%%%%%%%%%%%%%%%
\begin{frame}[plain]{短縮形の一覧表}
 
% --- プリアンブルで必要なもの ---
% \usepackage{tabularray}
% \UseTblrLibrary{booktabs}
% \documentclass{beamer} などで色の定義も必要です

\centering
\begin{tblr}{
  colspec = {lll}, % 列の指定:3列とも左揃え
  row{odd} = {bg=NavyBlue!40},
  row{even} = {bg=yellow!40},
  row{1-2} = {font=\small, bg=white}, % 1行目と2行目のフォントを \small に
}
\toprule
% --- 表のヘッダー ---
& \SetCell[c=2]{c} 短縮形 & \\ % 2列を結合(Colspan=2)して中央揃え(center)
& パターンA & パターンB \\
\midrule
% --- 表の本体 ---
\visible<1->{I am not X.}   & \visible<2->{{I'm not x.}}   & \visible<10->{{*I amn't X.}} \\
\visible<1->{You are not X.}  & \visible<3->{{You're not x.}}  & \visible<11->{{You aren't X.}} \\
\visible<1->{He is not X.}    & \visible<4->{{He's not x.}}    & \visible<12->{{He isn't X.}} \\
\visible<1->{She is not X.}   & \visible<5->{{She's not x.}}   & \visible<13->{{She isn't X.}} \\
\visible<1->{It is not X.}    & \visible<6->{{It's not x.}}    & \visible<14->{{It isn't X.}} \\
\visible<1->{That is not X.}  & \visible<7->{{That's not x.}}  & \visible<15->{{That isn't X.}} \\
\visible<1->{We are not X.}   & \visible<8->{{We're not x.}}   & \visible<16->{{We aren't X.}} \\
\visible<1->{They are not X.} & \visible<9->{{They're not x.}} & \visible<17->{{They aren't X.}} \\
\bottomrule
\end{tblr}
\end{frame}
%%%%%%%%%%%%%%%%%%%%%%%%%%%%%%%%%%%
\section{現在進行形の否定}
%%%%%%%%%%%%%%%%%%%%%%%
\begin{frame}[plain]{現在進行形の否定}
\Large

\begin{enumerate}
 \item<2-> They {\itshape\bfseries are studying} now.\hfill{\scriptsize study \textipa{/st\'2di/} 勉強する}
 \item<3-> They {\itshape\bfseries are} \textcolor{Maroon}{\bfseries not} {\itshape\bfseries studying} now.\\
 \item<5-> They\textcolor{Maroon}{\bfseries 're not} {\itshape studying} now.
 \item<6-> They \textcolor{Maroon}{\bfseries aren't} {\itshape studying} now.
\end{enumerate}

\vfill

\hfill{\tiny 0159}\,{\scriptsize \myaudio{audio/022_is_ing_negative_01.mp3}}

\begin{block}<4->{Topic for Today}\small
\begin{itemize}\setbeamertemplate{items}[square]
 \item  現在進行形\,\,\Circled[fill color = white]{\,\,$\text{be動詞}+\text{---ing}$\,\,}\,\,の否定$\longrightarrow$\,\,%
\Circled[fill color = white]{\,\,$\text{be動詞} + \text{\textcolor{Maroon}{\bfseries not}} + \text{---ing}$\,\,}
 \visible<7->{\item 短縮形が使われることもあります}
\end{itemize}
\end{block}
\hfill{}\onslide<7->{\scriptsize be動詞の直後に\textcolor{Maroon}{\bfseries not}がくるのは、be動詞の否定と同じです\hspace{1\zw}}
\end{frame}
%%%%%%%%%%%%%%%%%%%%%%%%%
\begin{frame}[plain]{Exercises}
あたえられた日本文の意味になるように、空所に適切な単語を補いましょう

\begin{enumerate}
 \item わたしは、いま読書をしていません。\\
I \alt<2->{(\,\,\textcolor{BurntOrange}{\bfseries am}\,\,)~~(\,\,\textcolor{BurntOrange}{\bfseries not}\,\,)}{(\phantom{\,\,am\,\,})~~(\phantom{\,\,not\,\,})} reading a book now.
 \item 彼女は、いま歌を歌っていません。\\
She \alt<3->{(\,\,\textcolor{BurntOrange}{\bfseries is}\,\,)~~(\,\,\textcolor{BurntOrange}{\bfseries not}\,\,)}{(\phantom{\,\,is\,\,})~~(\phantom{\,\,not\,\,})} singing a song now.
 \item わたしたちは、いま宿題をしていません。\\
We \alt<4->{(\,\,\textcolor{BurntOrange}{\bfseries are}\,\,)~~(\,\,\textcolor{BurntOrange}{\bfseries not}\,\,)}{(\phantom{\,\,are\,\,})~~(\phantom{\,\,not\,\,})} doing homework now.\hfill{\scriptsize do homework: 宿題をする}
 \item わたしたちのネコは、いまベッドの上で寝ていません。\\
Our cat \alt<5->{(\,\,\textcolor{BurntOrange}{\bfseries isn't}\,\,)}{(\phantom{\,\,isn't\,\,})} sleeping on the bed now.
 \item わたしたちは、いま数学の勉強をしていません。\\
We \alt<6->{(\,\,\textcolor{BurntOrange}{\bfseries aren't}\,\,)}{(\phantom{\,\,aren't\,\,})} studying math now.
\end{enumerate} 

\hfill{\tiny 0233}\,{\scriptsize \myaudio{./audio/022_is_ing_negative_02.mp3}}
\end{frame}
%%%%%%%%%%%%%%%%%%%%%%%%%%%
 \begin{frame}[plain]{Exercises}
日本文を参考にして (~~~~~~) 内の語を並べかえ、英文を完成させましょう%
\hfill{\tiny 0237}\,{\scriptsize \myaudio{./audio/022_is_ing_negative_03.mp3}}

\vspace{-5pt}

\begin{enumerate}
 \item \visible<1->{{\small ジョンは、いまサッカーをしていません。}}\\\visible<1->{John ( playing / not / is ) soccer now.}\hfill{\scriptsize soccer \textipa{/s\'Ak\textrhookschwa /} サッカー}\\
\visible<2->{$\longrightarrow$\,\,\,\,John \textcolor{orange}{\bfseries is not playing} soccer now.}
 \item \visible<1->{{\small 彼らは、いまプールで泳いでいません。}}\\
\visible<1->{They ( not / swimming / are ) in the pool now.}\\
\visible<3->{$\longrightarrow$\,\,\,\,They \textcolor{orange}{\bfseries are not swimming} in the pool now.}
 \item \visible<1->{{\small その犬は、庭を走っていません。}}\\
\visible<1->{The dog ( running / not / is ) in the yard.}\\
\visible<4->{$\longrightarrow$\,\,\,\,The dog \textcolor{orange}{\bfseries is not running} in the yard.}
 \item \visible<1->{{\small 彼女は、いま手紙を書いていません。}}\\
\visible<1->{She ( a letter / not / writing / is ) now.}
\visible<5->{$\longrightarrow$\,\,\,\,She \textcolor{orange}{\bfseries is not  writing a letter} now.}
 \item \visible<1->{{\small 彼は、いまケーキを作っていません。}}\\
\visible<1->{He ( making / not / is ) cake now.}
\visible<6->{$\longrightarrow$\,\,\,\,He \textcolor{orange}{\bfseries is not making} cake now.}
\end{enumerate}
 \end{frame}
%%%%%%%%%%%%%%%%%%%%%%%%%
\section{現在進行形の否定のまとめ}
%%%%%%%%%%%%%%%%%%%%%%%%
\begin{frame}[plain,t]{まとめ}
 
\begin{block}<1->{be動詞の否定}\small
\begin{itemize}\setbeamertemplate{items}[square]
 \item<1->  be動詞の否定は\,\,\Circled[fill color = white]{\,\,$\text{be動詞} + \text{\textcolor{Maroon}{\bfseries not}}$\,\,}
 \item<2-> 短縮形が用いられることもあります
       \begin{enumerate}\setbeamertemplate{items}[circle]
	\item<3-> \temporal<4,5>{She is not}{\Circled{She is} not}{She \Circled{is not}} busy.
	\item<5-> \textbf{She's} not busy.
	\item<7-> She \textbf{isn't} busy.
       \end{enumerate}
\end{itemize}
\end{block}

\begin{block}<8->{現在進行形の否定}\small
\begin{itemize}\setbeamertemplate{items}[square]
 \item<8->  現在進行形\,\,\Circled[fill color = white]{\,\,$\text{be動詞}+\text{---ing}$\,\,}\,\,の否定$\longrightarrow$\,\,%
\Circled[fill color = white]{\,\,$\text{be動詞} + \text{\textcolor{Maroon}{\bfseries not}} + \text{---ing}$\,\,}
 \item<9-> 短縮形が用いられることもあります
       \begin{enumerate}\setbeamertemplate{items}[circle]
	\item<10-> \temporal<11,12>{They are not}{\Circled{They are} not}{They \Circled{are not}} playing the guitar.
	\item<12-> \textbf{They're} not playing the guitar.
	\item<14-> They \textbf{aren't} playing the guitar.
       \end{enumerate}
%\begin{enumerate}\setbeamertemplate{items}[circle]
% \item They {\itshape are playing} the guitar.
% \item They {\itshape are} \textcolor{Maroon}{\bfseries not} {\itshape playing} the guitar.
% \item They\textcolor{Maroon}{\bfseries 're not} {\itshape playing} the guitar.
% \item They \textcolor{Maroon}{\bfseries aren't} {\itshape playing} the guitar.
%\end{enumerate}
\end{itemize}
\end{block}


\end{frame}
%%%%%%%%%%%%%%%%%%%%%%%%%%%
\begin{frame}[plain,t]
\vspace*{50pt}

be動詞の否定も現在進行形の否定も、どちらもbe動詞の直後に\textcolor{Maroon}{\bfseries not}がきます!

\vspace{20pt}

% この場合は (340pt, 200pt) の位置に 0.4\linewidth の幅のブロックができる.
\begin{textblock*}{0.4\linewidth}(340pt,100pt)
    % TiKZを使った図形の描画
    \begin{tikzpicture}
        \duck[laughing,bowtie,
strawhat=brown!50!white,
ribbon=black,
think={\scriptsize おなじ!},
bubblecolour=white!50!pink,
scale=1.1414]
    \end{tikzpicture}
\end{textblock*}

\end{frame}
\end{document}
