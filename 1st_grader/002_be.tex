\documentclass[aspectratio=169]{beamer}
\usepackage[no-math,deluxe,haranoaji]{luatexja-preset}
\renewcommand{\kanjifamilydefault}{\gtdefault}
\renewcommand{\emph}[1]{{\upshape\bfseries #1}}
\usetheme{metropolis}
\usetheme{metropolis}
\metroset{block=fill}
%%%%%%%%%%%%%%%%%%%%%%%%%%
\setbeamertemplate{navigation symbols}{}
\usecolortheme[rgb={0.7,0.2,0.2}]{structure}
%%%%%%%%%%%%%%%%%%%%%%%%%%%
%% さまざまなアイコン
%%%%%%%%%%%%%%%%%%%%%%%%%%%
\usepackage{fontawesome}
%%%%%%%%%%%%%%%%%%%%%%%%%%%
\usepackage{tikz}
%%%%%%%%%%%%%%%%%%%%%%%%%%%
%% 場合分け
\usepackage{cases}
%%%%%%%%%%%%%%%%%%%%%%%%%%%
%% 音声リンク表示
\newcommand{\myaudio}[1]{\href{#1}{\faVolumeUp}}
%%%%%%%%%%%%%%%%%%%%%%%%%%%
\title{English is fun.\,\,{}---be動詞を学びます---}
\author{}
\institute[]{}
\date[]

%%%%%%%%%%%%%%%%%%%%%%%%%%%%
%% TEXT
%%%%%%%%%%%%%%%%%%%%%%%%%%%%
\begin{document}
\begin{frame}[plain]
  \titlepage
\end{frame}

\section*{目次}
\begin{frame}[plain]
  \frametitle{授業の流れ}
  \tableofcontents
\end{frame}

\section{「AはBだ」という表現}
\begin{frame}<1-27>[plain]\frametitle{「AはBだ」という表現}
 % \setbeamercovered{transparent}
  \begin{enumerate}
   \item<1-> I \textbf<14-20>{\color<14-20>{red}{am}} a student. \onslide*<2>{わたしは生徒です。}\onslide*<14-20>{(I $=$ a student)}\onslide*<21-26>{\footnotesize  a:1つの、1人の student: 生徒、学生}
   \item<3-> You \textbf<15-20>{\color<15-20>{red}{are}} my friend. \onslide*<4>{あなたはわたしのともだちです。}\onslide*<15-20>{(You $=$ my friend)}\onslide*<22-26>{\footnotesize  my: わたしの friend: ともだち}
   \item<5-> He \textbf<16-20>{\color<16-20>{red}{is}} tall. \onslide*<6>{彼は背が高い。}\onslide*<16-20>{(He $=$ tall)}\onslide*<23-26>{\footnotesize  tall: 背が高い}
   \item<7-> She \textbf<17-20>{\color<17-20>{red}{is}} kind. \onslide*<8>{彼女は親切だ。}\onslide*<17-20>{(She $=$ kind)}\onslide*<24-26>{\footnotesize  kind: 親切な}
   \item<9-> The sky \textbf<18-20>{\color<18-20>{red}{is}} blue. \onslide*<10>{空は青い。}\onslide*<18-20>{(The sky $=$ blue)}\onslide*<25-26>{\footnotesize  the sky: 空 blue: 青い}
   \item<11-> They \textbf<19-20>{\color<19-20>{red}{are}} my classmates. \onslide*<12>{彼らはわたしのクラスメートです。}\onslide*<19-20>{(They $=$ my classmates)}\onslide*<26>{\footnotesize  classmates: (2人以上の)クラスメート}
  \end{enumerate}

\bigskip

\begin{exampleblock}<20->{Topics for Today}
\begin{itemize}
 \item am, are, isはイコール($=$)の意味
 \item まとめてbe動詞といいます
\end{itemize}
\end{exampleblock}


% Embed the sound file
\onslide<27>{%
\myaudio{audio/002_be_01.mp3}\,\,{}Listen carefully.(注意して聞いてください)

}
\end{frame}

\begin{frame}[plain]\frametitle{am, are, is --- みんな、なかまです}
 \centering
\begin{tikzpicture}
% 補助グリッドを描画
%\draw[step=1cm, gray!20, very thin] (-6,-2) grid (6,6);
% ノードの定義と配置

5\node[circle, draw=black, fill=pink!30,minimum size=15mm, line width=1pt] (B) at (-5,0) {\LARGE am};\pause
\node[circle, draw=black, fill=blue!30, minimum size=15mm, line width=1pt] (C) at (0,5) {\LARGE are};\pause
\node[circle, draw=black, fill=green!30, minimum size=15mm, line width=1pt] (D) at (5,0) {\LARGE is};\pause
\node[circle, draw=black, fill=yellow!30, minimum size=20mm, line width=1pt] (A) at (0,0) {\LARGE be動詞};\pause
% ノード間の線の描画
\draw[-latex, line width=1.5pt] (A) -- node[above] {} (B);\pause
\draw[-latex, line width=1.5pt] (A) -- node[sloped, above] {} node[sloped, below] {} (C);\pause
\draw[-latex, line width=1.5pt] (A) -- node[above] {} (D);
\end{tikzpicture}
\end{frame}

\begin{frame}[plain]{be動詞、どれ使う?}
 \centering\Large

be動詞$\left\{\begin{tabular}[c]{l}
       am\\are\\is\end{tabular}\right\}$の使い分けを学習しよう
\end{frame}

\section{I am 〜}
\begin{frame}<1-20>[plain]\frametitle{I am 〜.}
 % \setbeamercovered{transparent}
  \begin{enumerate}
   \item<1-> \textbf{\color{red}{I am}} a student. \onslide*<2>{わたしは生徒です。}\onslide*<15-20>{\footnotesize  student: 生徒、学生}
   \item<3-> \textbf{\color{red}{I am}} tall. \onslide*<4>{わたしは背が高い。}\onslide*<16-20>{\footnotesize  tall: 背が高い}
   \item<5-> \textbf{\color{red}{I am}} 13 years old. \onslide*<6>{わたしは13歳です。}\onslide*<17-20>{\footnotesize  〜 years old: 〜歳だ 〜には数字がはいります}
   \item<7-> \textbf{\color{red}{I am}} John. \onslide*<8>{わたしはジョンです。}
   \item<9-> \textbf{\color{red}{I am}} happy. \onslide*<10>{わたしは幸せです。}\onslide*<18-20>{\footnotesize  happy: 幸せだ}
   \item<11-> \textbf{\color{red}{I am}} from Tokyo. \onslide*<12>{わたしは東京の出身です。}\onslide*<19>{\footnotesize  from 〜: 〜の出身だ}
  \end{enumerate}

\bigskip

\begin{exampleblock}<14->{Tpics for Today}
\begin{itemize}
 \item   amはbe動詞
 \item I am 〜.をひとつのパターンとして覚えよう
\end{itemize}
     \end{exampleblock}

% Embed the sound file
\onslide<20>{%
\myaudio{./audio/002_be_02.mp3}\,\,{}Listen carefully.(注意して聞いてください)
}

\end{frame}


\section{You are 〜}
\begin{frame}<1-21>[plain]\frametitle{You are 〜.}
 % \setbeamercovered{transparent}
  \begin{enumerate}
   \item<1-> \textbf{\color{red}{You are}} my friend. \onslide*<2>{あなたはわたしのともだちです。}\onslide*<15-20>{\footnotesize  friend: ともだち、友人}
   \item<3-> \textbf{\color{red}{You are}} very kind. \onslide*<4>{あなたはとても親切だ。}\onslide*<16-20>{\footnotesize  kind: 親切だ}
   \item<5-> \textbf{\color{red}{You are}} a good student. \onslide*<6>{あなたはいい生徒です。}\onslide*<17-20>{\footnotesize  student: 生徒}
   \item<7-> \textbf{\color{red}{You are}} good at baseball. \onslide*<8>{あなたは野球がうまい。}\onslide*<18-20>{\footnotesize  good at 〜: 〜がうまい、得意だ}
   \item<9-> \textbf{\color{red}{You are}} busy. \onslide*<10>{あなたは忙しい。}\onslide*<19-20>{\footnotesize  happy: 幸せだ}
   \item<11-> \textbf{\color{red}{You are}} from Chiba. \onslide*<12>{あなたは千葉の出身です。}\onslide*<20>{\footnotesize  from 〜: 〜の出身だ}
  \end{enumerate}

\bigskip

\begin{exampleblock}<14->{Topics for Today}
\begin{itemize}
 \item areはbe動詞
 \item You are 〜.をひとつのパターンとして覚えよう
\end{itemize}
     \end{exampleblock}

% Embed the sound file
\onslide<21>{%
\myaudio{audio/002_be_03.mp3}\,\,{}Listen carefully.(注意して聞いてください)

}

\end{frame}

\section{I am, You are以外の場合(1)}
\begin{frame}<1-21>[plain]\frametitle{I am, You are以外の場合(1)}
 % \setbeamercovered{transparent}
  \begin{enumerate}
   \item<1-> This \textbf{\color{red}{is}} my pencil. \onslide*<2>{これはわたしの鉛筆です。}\onslide*<15-20>{\footnotesize this: これ pencil: 鉛筆}
   \item<3-> He \textbf{\color{red}{is}} my classmate. \onslide*<4>{彼はわたしのクラスメートです。}\onslide*<16-20>{\footnotesize   classmate: クラスメート、級友}
   \item<5-> She \textbf{\color{red}{is}} a good singer. \onslide*<6>{彼女は歌がうまい。}\onslide*<17-20>{\footnotesize  singer: 歌い手、歌手}
   \item<7-> Your bike \textbf{\color{red}{is}} new. \onslide*<8>{あなたの自転車は新しい。}\onslide*<18-20>{\footnotesize  bike: 自転車 new: 新しい}
   \item<9-> George \textbf{\color{red}{is}} busy. \onslide*<10>{ジョージは忙しい。}\onslide*<19-20>{\footnotesize  busy: 忙しい}
   \item<11-> Jane \textbf{\color{red}{is}} from France. \onslide*<12>{ジェーンはフランスの出身です。}\onslide*<20>{\footnotesize  from 〜: 〜の出身だ France: フランス}
  \end{enumerate}

\bigskip

\begin{exampleblock}<14->{Topics for Today}
\begin{itemize}
 \item I, You 以外で1つ(This, That, The book \ldots{})、1人(He, She, George, Jane \ldots{})で始まるときはisを使います
\end{itemize}  
     \end{exampleblock}


% Embed the sound file
\onslide<21>{%
\myaudio{audio/002_be_04.mp3}\,\,{}Listen carefully.(注意して聞いてください)
}

\end{frame}


\section{I am, You are以外の場合(2)}

\begin{frame}<1-21>[plain]\frametitle{I am, You are以外の場合(2)}
 % \setbeamercovered{transparent}
  \begin{enumerate}
   \item<1-> These \textbf{\color{red}{are}} my pencils. \onslide*<2>{これらはわたしの鉛筆です。}\onslide*<15-20>{\footnotesize  these: これら pencil: 鉛筆}
   \item<3-> They \textbf{\color{red}{are}} my classmates. \onslide*<4>{彼らはわたしのクラスメートです。}\onslide*<16-20>{\footnotesize  they: 彼ら classmate: クラスメート、級友}
   \item<5-> They \textbf{\color{red}{are}} kind. \onslide*<6>{彼らは親切だ。}\onslide*<17-20>{\footnotesize  kind: 親切な}
   \item<7-> The flowers \textbf{\color{red}{are}} beautiful. \onslide*<8>{その花は美しい。}\onslide*<18-20>{\footnotesize  flower: 花 beautifuk: 美しい}
   \item<9-> We \textbf{\color{red}{are}} busy. \onslide*<10>{わたしたちは忙しい。}\onslide*<19-20>{\footnotesize  we: わたしたち busy: 忙しい}
   \item<11-> Jane and George \textbf{\color{red}{are}} from France. \onslide*<12>{ジェーンとジョージはフランスの出身です。}\onslide*<20>{\footnotesize  from 〜: 〜の出身だ France: フランス}
  \end{enumerate}

\bigskip

\begin{exampleblock}<14->{Topics for Today}
\begin{itemize}
 \item   I, You 以外で複数(2つ以上)のモノや人で始まるときはareを使います
\end{itemize}
     \end{exampleblock}

% Embed the sound file
\onslide<21>{%
\myaudio{audio/002_be_05.mp3}\,\,{}Listen carefully.(注意して聞いてください)
}
\end{frame}

\section{まとめ}
\begin{frame}[plain]\frametitle{まとめ}

\begin{block}{be動詞の使い分け}

{\Large
\begin{numcases}{\text{ }}
 \text{\mbox{}\,\,{}Iではじまる}&$\longrightarrow$\,\,\,\,\,\,{}\text{am}\\
 \text{\mbox{}\,\,{}Youではじまる}&$\longrightarrow$\,\,\,\,\,\,{}\text{are}\\
 \text{\mbox{}\,\,{}1つ、1人}&$\longrightarrow$\,\,\,\,\,\,{}\text{is}\\
 \text{\mbox{}\,\,{}2つ、2人以上}&$\longrightarrow$\,\,\,\,\,\,{}\text{are}
\end{numcases}
}
\end{block}
\end{frame}

\begin{frame}[plain]\frametitle{まとめ}
 \centering
\begin{tikzpicture}
% 補助グリッドを描画
%\draw[step=1cm, gray!20, very thin] (-6,-2) grid (6,6);
% ノードの定義と配置
\node[circle, draw=black, fill=yellow!30, minimum size=20mm, line width=1pt] (A) at (0,0) {\LARGE be動詞};\pause
\node[circle, draw=black, fill=pink!30,minimum size=15mm, line width=1pt] (B) at (-6,0) {\LARGE am};\pause
\node[circle, draw=black, fill=blue!30, minimum size=15mm, line width=1pt] (C) at (0,5) {\LARGE are};\pause
\node[circle, draw=black, fill=green!30, minimum size=15mm, line width=1pt] (D) at (6,0) {\LARGE is};\pause

% ノード間の線の描画
\draw[-latex, line width=1.5pt] (A) -- node[above] {Iのとき} (B);\pause
\draw[-latex, line width=1.5pt] (A) -- node[sloped, above] {Youのとき}  node[sloped, below] {複数なら} (C);\pause
\draw[-latex, line width=1.5pt] (A) -- node[above] {1つ、1人なら} node[below] {} (D);
\end{tikzpicture}
\end{frame}

\end{document}
