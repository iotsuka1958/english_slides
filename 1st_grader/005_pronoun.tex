\documentclass[aspectratio=169,xcolor={dvipsnames,table}]{beamer}
\usepackage[no-math,deluxe,haranoaji]{luatexja-preset}
\renewcommand{\kanjifamilydefault}{\gtdefault}
\renewcommand{\emph}[1]{{\upshape\bfseries #1}}
\usetheme{metropolis}
\metroset{block=fill}
\setbeamertemplate{navigation symbols}{}
\setbeamertemplate{blocks}[rounded][shadow=false]
\usecolortheme[rgb={0.7,0.2,0.2}]{structure}
%%%%%%%%%%%%%%%%%%%%%%%%%%
%% Change alert block colors
%%% 1- Block title (background and text)
\setbeamercolor{block title alerted}{fg=mDarkTeal, bg=mLightBrown!45!yellow!45}
\setbeamercolor{block title example}{fg=magenta!10!black, bg=mLightGreen!60}
%%% 2- Block body (background)
\setbeamercolor{block body alerted}{bg=mLightBrown!25}
\setbeamercolor{block body example}{bg=mLightGreen!15}
%%%%%%%%%%%%%%%%%%%%%%%%%%%
%%%%%%%%%%%%%%%%%%%%%%%%%%%
%% さまざまなアイコン
%%%%%%%%%%%%%%%%%%%%%%%%%%%
%\usepackage{fontawesome}
\usepackage{fontawesome5}
\usepackage{figchild}
\usepackage{twemojis}
\usepackage{utfsym}
\usepackage{bclogo}
\usepackage{marvosym}
\usepackage{fontmfizz}
\usepackage{pifont}
\usepackage{phaistos}
\usepackage{worldflags}
\usepackage{jigsaw}
\usepackage{tikzlings}
\usepackage{tikzducks}
\usepackage{scsnowman}
\usepackage{epsdice}
\usepackage{halloweenmath}
\usepackage{svrsymbols}
\usepackage{countriesofeurope}
\usepackage{tipa}
\usepackage{manfnt}
%%%%%%%%%%%%%%%%%%%%%%%%%%%
\usepackage{tikz}
\usetikzlibrary{calc,patterns,decorations.pathmorphing,backgrounds}
\usepackage{tcolorbox}
\usepackage{tikzpeople}
\usepackage{circledsteps}
\usepackage{xcolor}
\usepackage{amsmath}
\usepackage{booktabs}
\usepackage{chronology}
\usepackage{signchart}
%%%%%%%%%%%%%%%%%%%%%%%%%%%
%% 場合分け
%%%%%%%%%%%%%%%%%%%%%%%%%%%
\usepackage{cases}
%%%%%%%%%%%%%%%%%%%%%%%%%%
\usepackage{pdfpages}
%%%%%%%%%%%%%%%%%%%%%%%%%%%
%% 音声リンク表示
\newcommand{\myaudio}[1]{\href{#1}{\faVolumeUp}}
%%%%%%%%%%%%%%%%%%%%%%%%%%
%% \myAnch{<名前>}{<色>}{<テキスト>}
%% 指定のテキストを指定の色の四角枠で囲み, 指定の名前をもつTikZの
%% ノードとして出力する. 図には remember picture 属性を付けている
%% ので外部から参照可能である.
\newcommand*{\myAnch}[3]{%
  \tikz[remember picture,baseline=(#1.base)]
    \node[draw,rectangle,line width=1pt,#2] (#1) {\normalcolor #3};
}
%%%%%%%%%%%%%%%%%%%%%%%%%%
%% \myEmph コマンドの定義
%%%%%%%%%%%%%%%%%%%%%%%%%%
%\newcommand{\myEmph}[3]{%
%    \textbf<#1>{\color<#1>{#2}{#3}}%
%}
\usepackage{xparse} % xparseパッケージの読み込み
\NewDocumentCommand{\myEmph}{O{} m m}{%
    \def\argOne{#1}%
    \ifx\argOne\empty
        \textbf{\color{#2}{#3}}% オプション引数が省略された場合
    \else
        \textbf<#1>{\color<#1>{#2}{#3}}% オプション引数が指定された場合
    \fi
}
%%%%%%%%%%%%%%%%%%%%%%%%%%%
%%%%%%%%%%%%%%%%%%%%%%%%%%%
%% 文末の上昇イントネーション記号 \myRisingPitch
%% 通常のイントネーション \myDownwardPitch
%% https://note.com/dan_oyama/n/n8be58e8797b2
%%%%%%%%%%%%%%%%%%%%%%%%%%%
\newcommand{\myRisingPitch}{
\begin{tikzpicture}[scale=0.3,baseline=0.3]
\draw[->,>=stealth] (0,0) to[bend right=45] (1,1);
\end{tikzpicture}
}
\newcommand{\myDownwardPitch}{
\begin{tikzpicture}[scale=0.3,baseline=0.3]
\draw[->,>=stealth] (0,1) to[bend left=45] (1,0);
\end{tikzpicture}
}
%%%%%%%%%%%%%%%%%%%%%%%%%%%%
%\AtBeginSection[%
%]{%
%  \begin{frame}[plain]\frametitle{授業の流れ}
%     \tableofcontents[currentsection]
%   \end{frame}%
%}

\usepackage{pxrubrica}
%%%%%%%%%%%%%%%%%%%%%%%%%%%
\title{English is fun.}
\subtitle{David is a singer. He is famous.}
\author{}
\institute[]{}
\date[]

%%%%%%%%%%%%%%%%%%%%%%%%%%%%
%% TEXT
%%%%%%%%%%%%%%%%%%%%%%%%%%%%
\begin{document}

\begin{frame}[plain]
  \titlepage
\end{frame}

\section*{授業の流れ}
\begin{frame}[plain]
  \frametitle{授業の流れ}
  \tableofcontents
\end{frame}

\section{代名詞}
\subsection{代名詞とは}
%%%%%%%%%%%%%%%%%%%
\begin{frame}[plain]\frametitle{代名詞}
\begin{enumerate}
 \item \begin{enumerate}
	\item<1-> \myEmph[1-]{Maroon}{Billy} is a singer.\,\,\,\,\,\myEmph[1-]{Maroon}{Billy} sings well.\,\,\,\,\,I like \myEmph[1-]{Maroon}{Billy}.%
\hfill{\scriptsize singer \textipa{/s\'IN\textrhookschwa /} 歌手}
	\item<2-> \myEmph[1-]{Maroon}{Billy} is a singer.\,\,\,\,\,\Circled[fill color = yellow!20]{\,He\,} sings well.\,\,\,\,\,I like \Circled[fill color=NavyBlue!20]{\,him\,}.
       \end{enumerate}
 \item \begin{enumerate}
	\item<1-> \myEmph[1-]{Maroon}{Diana} is a pianist.\,\,\,\,\,\myEmph[1-]{Maroon}{Diana} plays the piano well.\,\,\,\,\,I like \myEmph[1-]{Maroon}{Diana}.\\%
\hfill{\scriptsize pianist \textipa{/pi\'\ae nIst/} ピアニスト\hspace{1\zw}piano \textipa{/pi\'\ae noU/} ピアノ}
	\item<3-> \myEmph[1-]{Maroon}{Diana} is a pianist.\,\,\,\,\,\Circled[fill color=yellow!20]{\,She\,} plays the piano well.\,\,\,\,\,I like \Circled[fill color=NavyBlue!20]{\,her\,}. 
       \end{enumerate}
\end{enumerate}

\vfill

\begin{block}<4->{Topics for Today}
\textbf{he / him},\hspace{8pt}\textbf{she / her} などを\kenten{代名詞}といいます%
\hfill{}{\scriptsize 「\kenten{名詞}の\kenten{代}わり」ということですね}
\begin{itemize}\setbeamertemplate{items}[circle]
 \item<5-> \textbf{he, she}\,$\longrightarrow$\,主語(~は)になります\hfill\makebox[30pt][l]{\kenten{主格}の代名詞}\hspace{80pt}\mbox{}
 \item<6-> \textbf{him, her}\,$\longrightarrow$\,目的語(~を)になります\hfill\makebox[30pt][l]{\kenten{目的格}の代名詞}\hspace{80pt}\mbox{}
\end{itemize}
     \end{block}

\hfill{\tiny 0204}\,{\scriptsize \myaudio{./audio/005_pronoun_1.mp3}}
\end{frame}
%%%%%%%%%%%%%%%%%%
\subsection{主格・目的格}
%%%%%%%%%%%%%%%%%%
\begin{frame}<1-15>[plain,t]{一覧表}
 
\begin{tblr}{%
colspec={X[.6]X[.8]X[.8]X[.8]XX[.8]X[.8]},
row{4} = {bg = yellow!50},
cell{4}{1} = {bg = white},
% 表の最上と最下に太さ 0.08em の横罫線
hline{1} = {2-7}{ 0.08em },
hline{Z} = { 0.08em },
cell{1}{2} = {r =1, c = 3}{halign = c},%
cell{1}{5} = {r =1, c = 3}{halign = c},%
cell{5}{5} = {bg = NavyBlue!30},%
cell{6}{5} = {bg = NavyBlue!30},%
cell{7}{5} = {bg = NavyBlue!30},
cell{7}{2} = {bg = Maroon!20},%
cell{7}{3} = {bg = Maroon!20},%
cell{7}{4} = {bg = Maroon!20},%
cell{6}{2} = {bg = Maroon!10},%
cell{6}{3} = {bg = Maroon!10},%
cell{6}{4} = {bg = Maroon!10},%
cell{5}{2} = {bg = Maroon!5},%
cell{5}{3} = {bg = Maroon!5},%
cell{5}{4} = {bg = Maroon!5},%
cell{5}{6} = {r =3, c = 1}{halign = l, valign = m, bg = NavyBlue!30},%
cell{5}{7} = {r =3, c = 1}{halign = l, valign = m, bg = NavyBlue!30},
%cell{1}{3} = {r = 1, c = 5}{bg = gray6, fg = white, font = { \bfseries\sffamily\gtfamily }, cmd = {} },
%row{1} = { halign = c, bg = gray6, fg = white, font = { \bfseries\sffamily\gtfamily }, cmd = {} },
hline{3} = {1-7}{ 0.04em },
}
 &単数&&&複数&& \\
 & &主格 &目的格 & &主格 &目的格 \\
1人称&私 & \onslide<2->{\myEmph[2]{Maroon}{I}}&\onslide<3->{\myEmph[3]{Maroon}{me}} &私たち &\onslide<4->{\myEmph[4]{Maroon}{we}} &\onslide<5->{\myEmph[5]{Maroon}{us}} \\
2人称&あなた & \onslide<6->{\myEmph[6]{Maroon}{you}}&\onslide<6->{\myEmph[6]{Maroon}{you}} &あなたたち &\onslide<6->{\myEmph[6]{Maroon}{you}} &\onslide<6->{\myEmph[6]{Maroon}{you}} \\
3人称&彼 & \onslide<7->{\myEmph[7]{Maroon}{he}}& \onslide<8->{\myEmph[8]{Maroon}{him}}&彼ら &\onslide<13->{\myEmph[13]{Maroon}{they}} & \onslide<14->{\myEmph[14]{Maroon}{them}}\\
 &彼女 & \onslide<9->{\myEmph[9]{Maroon}{she}}& \onslide<10->{\myEmph[10]{Maroon}{her}}&彼女たち & &\\
 &それ & \onslide<11->{\myEmph[11]{Maroon}{it}}& \onslide<12->{\myEmph[12]{Maroon}{it}}&それら & & \\
\end{tblr}

\hfill{\scriptsize あとで表を拡張するので、主格と目的格の列の間に十分な間隔をあけておいてください}

\vspace{20pt}

\Large
\only<2>{\Circled{\,\,I\,\,} {\scriptsize ($=\text{S}$)}\,\,\,\,like sushi.}
\only<3>{The man knows \Circled{\,\,me\,\,} {\scriptsize ($=\text{O}$)}\,.}

\only<4>{\hspace*{220pt}\Circled{\,\,We\,\,}  {\scriptsize ($=\text{S}$)}\,\,\,\,like music.}

\only<5>{\hspace*{215pt}The woman knows \Circled{\,\,us\,\,} {\scriptsize ($=\text{O}$)}\,.}
\only<6>{\Circled[fill color=yellow!50]{\,\,You\,\,}  {\scriptsize ($=\text{S}$)}\,\,\,\,are kind. She loves \Circled[fill color=yellow!50]{\,\,you\,\,} {\scriptsize ($=\text{O}$)}\,.}
\only<7>{\Circled[fill color=Maroon!5]{\,\,He\,\,} {\scriptsize ($=\text{S}$)}\,\,\,\,likes music.}
\only<8>{The woman  knows \Circled[fill color=Maroon!5]{\,\,him\,\,} {\scriptsize ($=\text{O}$)}\,.}
\only<9>{\Circled[fill color=Maroon!10]{\,\,She\,\,} {\scriptsize ($=\text{S}$)}\,\,\,\,loves the cat.}
\only<10>{The man loves \Circled[fill color=Maroon!10]{\,\,her\,\,} {\scriptsize ($=\text{O}$)}\,.}
\only<11>{\Circled[fill color=Maroon!20]{\,\,It\,\,} {\scriptsize ($=\text{S}$)}\,\,\,\,is beautiful.}
\only<12>{The dog likes \Circled[fill color=Maroon!20]{\,\,it\,\,} {\scriptsize ($=\text{O}$)}\,.}
\only<13>{\hspace*{220pt}\Circled[fill color=NavyBlue!30]{\,\,They\,\,} {\scriptsize ($=\text{S}$)}\,\,\,\,are beautiful.}
\only<14>{\hspace*{220pt}My friends like \Circled[fill color=NavyBlue!30]{\,\,them\,\,} {\scriptsize ($=\text{O}$)}\,.}

\end{frame}
%%%%%%%%%%%%%%%%%%
\begin{frame}[plain]{Exercises}

{\small 正しいほうを〇で囲みましょう}\hfill{\tiny 0238}\,{\scriptsize \myaudio{./audio/005_pronoun_2.mp3}}
 \begin{enumerate}
  \item ( \alt<2->{\Circled{She}}{She} / Her ) speaks English. {\small 彼女は英語を話します。}
  \item (  \alt<3->{\Circled{He}}{He} / Him ) swims in the river. {\small 彼は川で泳ぐ。}
  \item  I know ( she /  \alt<4->{\Circled{her}}{her} ). {\small わたしは彼女のことを知っています。}
  \item She likes ( he /  \alt<5->{\Circled{him}}{him} ). {\small 彼女は彼が好きです。}
  \item I need ( they /  \alt<6->{\Circled{them}}{them} ). {\small わたしは彼らを必要としている。}
  \item He plays tennis with ( we /  \alt<7->{\Circled{us}}{us} ). {\small 彼はわたしたちといっしょにテニスをする。}
 \end{enumerate}


%\begin{block}<8->{Topic for Today}
%\dbend\,\,前置詞のあとも\kenten{目的格}をつかいます\\
%\hfill{\scriptsize \dbend{}: 前置詞のあとにくる語を「前置詞の目的語」といいます}
% \begin{enumerate}
%  \item<9-> I play tennis with my friends.
%  \item<10-> I play tennis with \textcolor{Maroon}{\textbf{them}}. (*I play tennis with \textcolor{NavyBlue}{\textbf{they}}.)\\
%\hfill{\scriptsize *:「まちがった英文です」という記号}
% \end{enumerate}
%     \end{block}

\end{frame}
%%%%%%%%%%%%%%%%%%%%%%%
\begin{frame}[plain]{前置詞の目的語}
 
\begin{enumerate}
 \item<1-> \begin{enumerate}
       \item<1-> I know George.\hfill{\scriptsize George $=$ 目的語}
       \item<2-> I know him. \hfill{\scriptsize him $=$ 目的語}
       \end{enumerate}
 \item<3-> \begin{enumerate}
       \item<3-> I play tennis with George.
	\item<4-> I play tennis with him.\hspace{20pt}\myCrossedOut{Maroon}{with he}
      \end{enumerate}
\end{enumerate}


\begin{block}<5->{Topic for Today}\small
\begin{itemize}\setbeamertemplate{items}[circle]
 \item 前置詞のあとも\kenten{目的格}をつかいます
       \begin{enumerate}
	\item I play tennis with my friends.
	\item I play tennis with \textcolor{Maroon}{\textbf{them}}.\hspace{30pt}\myCrossedOut{Maroon}{with they}
       \end{enumerate}
 \item 前置詞のあとにくる語を「前置詞の目的語」といいます
\end{itemize}
     \end{block}
\end{frame}
%%%%%%%%%%%%%%%%%%%%%%%%
\subsection{所有格}
\begin{frame}[plain]{所有格}
 \begin{enumerate}
  \item<1-> This is \myEmph[1-]{Maroon}{my} car. That is \myEmph[1-]{Maroon}{your} bike.
  \item<4-> *This is my a car.
  \item<4-> *This is a my car.\hfill{\scriptsize * 「まちがった英文です」という記号}
 \end{enumerate}

\begin{block}<2->{Topics for Today}


\begin{itemize}\setbeamertemplate{items}[circle]
 \item myはIが変化したかたちで「わたしの~」の意味

yourはyouが変化したかたちで「あなたの~」の意味
 \item<2-> myやyourを\kenten{所有格}の代名詞といいます
 \item<3-> \Circled[fill color=white]{\,\,所有格 $+$ 名詞\,\,}\,がセットになります
 \item[]<5-> \mbox{}\hfill\dbend\,\,my a carとかa my carとはいいません\hspace{.5\zw}\mbox{}\\
\mbox{}
%\hfill{}my booksはOKです
\end{itemize}
     \end{block}
\end{frame}
%%%%%%%%%%%%%%%%%%%%%%%%%%%
\begin{frame}<1-11>[plain,label=ichiran]{代名詞 $=$ 名詞の代わり}
 

\bigskip

\begin{tblr}{%
colspec={XXXXXXX},
row{4} = {bg = yellow!50},
cell{4}{1} = {bg = white},
% 表の最上と最下に太さ 0.08em の横罫線
hline{1} = {2-7}{ 0.08em },
hline{Z} = { 0.08em },
cell{1}{2} = {r =1, c = 3}{halign = c},%
cell{1}{5} = {r =1, c = 3}{halign = c},%
%cell{5}{5} = {bg = NavyBlue!30},%
%cell{6}{5} = {bg = NavyBlue!30},%
%cell{7}{5} = {bg = NavyBlue!30},
cell{7}{2} = {bg = Maroon!20},%
cell{7}{3} = {bg = Maroon!20},%
cell{7}{4} = {bg = Maroon!20},%
cell{6}{2} = {bg = Maroon!10},%
cell{6}{3} = {bg = Maroon!10},%
cell{6}{4} = {bg = Maroon!10},%
cell{5}{2} = {bg = Maroon!5},%
cell{5}{3} = {bg = Maroon!5},%
cell{5}{4} = {bg = Maroon!5},%
cell{5}{5} = {r =3, c = 1}{halign = l, valign = m, bg = NavyBlue!30},%
cell{5}{6} = {r =3, c = 1}{halign = l, valign = m, bg = NavyBlue!30},%
cell{5}{7} = {r =3, c = 1}{halign = l, valign = m, bg = NavyBlue!30},
%cell{1}{3} = {r = 1, c = 5}{bg = gray6, fg = white, font = { \bfseries\sffamily\gtfamily }, cmd = {} },
%row{1} = { halign = c, bg = gray6, fg = white, font = { \bfseries\sffamily\gtfamily }, cmd = {} },
hline{3} = {1-7}{ 0.04em },
}
 &単数&&&複数&& \\
 & 主格 &所有格 &目的格 &主格 &所有格&目的格 \\
1人称&\onslide<2->{I}&\onslide<3->{my} &\onslide<2->{me} &\onslide<2->{we} &\onslide<4->{our}&\onslide<2->{us} \\
2人称& \onslide<2->{you}&\onslide<5->{your} &\onslide<2->{you}&\onslide<2->{you} &\onslide<6->{your}&\onslide<2->{you} \\
3人称&\onslide<2->{he}& \onslide<7->{his}&\onslide<2->{him} &\onslide<2->{they} & \onslide<10->{their}&\onslide<2->{them}\\
 & \onslide<2->{she}& \onslide<8->{her}&\onslide<2->{her} & &&\\
 &\onslide<2->{it}& \onslide<9->{its}&\onslide<2->{it} & && \\
\end{tblr}

%\bigskip

\only<3>{This is \Circled{\,my book\,}.}
\hspace*{230pt}\only<4>{He is \Circled{\,our father\,}.}

\only<5,6>{\Circled[fill color=yellow!50]{\,Your cat\,} is cute.}

\only<7>{\Circled[fill color=Maroon!5]{\,His bike\,} is new.}

\only<8>{This is \Circled[fill color=Maroon!10]{\,her car\,}.}

\only<9>{The robot moves \Circled[fill color=Maroon!20]{\,its arm\,}.\hspace{10pt}cf.\,\,\,Tom moves \Circled[fill color=Maroon!5]{\,his arm\,}.\hfill{}arm: 腕}

\hspace*{230pt}\only<10>{\Circled[fill color=NavyBlue!30]{\,Their house\,} is big.}

\hfill\onslide<11>{なんども口ずさめば、自然に身につきますよ}
\end{frame}
%%%%%%%%%%%%%%%%%%%%%%%%%%
\begin{frame}[plain]{Exercises}

{\small (~~~~~)内の代名詞を適当な形にしましょう}

\begin{columns}[t]
 \begin{column}{.45\textwidth}
   \begin{enumerate}
 % \item I love ( you ).
  \item That is ( he ) book.\hfill\visible<2->{\textcolor{Maroon}{\bfseries his}}
  \item I see ( they ) every day.\hfill\visible<3->{\textcolor{NavyBlue}{\bfseries them}}
  \item ( you ) brother is kind.\hfill\visible<4->{\textcolor{Maroon}{\bfseries Your}}
  \item I know ( she ) name.\hfill\visible<5->{\textcolor{Maroon}{\bfseries her}}
  \item ( I ) favorite subject is math.\hfill\visible<6->{\textcolor{Maroon}{\bfseries My}}
  \item She helps ( I ).\hfill\visible<7->{\textcolor{NavyBlue}{\bfseries me}}
 \end{enumerate}
 \end{column}
%%%%%%%%%%%%
\begin{column}{.45\textwidth}
  \begin{enumerate}\setcounter{enumi}{6}
  \item They know ( we ).\hfill\visible<8->{\textcolor{NavyBlue}{\bfseries us}}
  \item I like ( she ) very much.\hfill\visible<9->{\textcolor{NavyBlue}{\bfseries her}}
  \item This is ( we ) house.\hfill\visible<10->{\textcolor{Maroon}{\bfseries our}}
  \item ( they ) school is very old.\hfill\visible<11->{\textcolor{Maroon}{\bfseries Their}}
  \item She knows ( he ) well.\hfill\visible<12->{\textcolor{NavyBlue}{\bfseries him}}
  \item I play tennis with ( he ).\hfill\visible<13->{\textcolor{NavyBlue}{\bfseries him}}
 \end{enumerate}
\end{column}
\end{columns}

%\begin{block}<14->{Topic for Today}
%\dbend\,\,前置詞のあとも\kenten{目的格}をつかいます\hfill{\tiny 0503}\,{\scriptsize \myaudio{./audio/005_pronoun_3.mp3}}
% \begin{enumerate}
%  \item<14-> I listen to my parents.\hfill{\scriptsize listen to:~耳を貸す}
%  \item<15-> I listen to \textbf{them}.
% \end{enumerate}
%     \end{block}
\end{frame}
%%%%%%%%%%%%%%%%%%%%%%%%%%%%%%%%%
\begin{frame}[plain]{it'sとits}
\Large
 \begin{enumerate}
  \item The robot moves \textcolor{Maroon}{\bfseries its} arm.\\
\hfill{\scriptsize robot \textipa{/r\'oUbAt/} ロボット\hspace{1\zw}move \textipa{/m\'u:v/} 動かす\hspace{1\zw}arm \textipa{/\'A\textrhookschwa m/} 腕}
  \item \textcolor{NavyBlue}{\bfseries It's} a nice car.
 \end{enumerate}
\normalsize

\vfill

\begin{block}{Topics for Today}
 \begin{itemize}\setbeamertemplate{items}[circle]
  \item itsはitの所有格
  \item {it's}は{it is}の短縮形
 \end{itemize}

\hfill{}itsとit'sの発音は同じです \textipa{/Its/}\hspace*{1\zw}
     \end{block}

\end{frame}
%%%%%%%%%%%%%%%%%%%%%%%%%%%%%%%%
\begin{frame}[plain]{Exercises}
 正しいほうを選びましょう

 \begin{enumerate}
  \item That is a nice car.\,\,\,\,\,( \alt<2->{\Circled{Its}}{Its} / It's ) owner is Mr. Johnson.\hfill{\scriptsize owner \textipa{/\'oUn\textrhookschwa /} 所有者}
  \item That is a nice car.\,\,\,\,\,(  Its / \alt<3->{\Circled{It's}}{It's} ) very expensive.\hfill{\scriptsize expensive \textipa{/Iksp\'ensIv/} 高価な}
 \end{enumerate}

\vfill

\hfill{\tiny 0121}\,{\scriptsize \myaudio{./audio/005_pronoun_4.mp3}}
\end{frame}

%%%%%%%%%%%%%%%%%%%%%%%%%%%%%%%%
\section{まとめ}%
\begin{frame}[plain]{この単元で学習したこと}
\begin{block}{まとめ}

 \begin{itemize}
  \item 代名詞\,\,$=$\,\,\kenten{名詞}の\kenten{代}わり
  \item 主格 / 所有格 / 目的格%
\hspace{30pt}%
\begin{tabular}[t]{lll}
 主格&〜は&主語になります \\
 所有格&〜の&  \Circled[fill color=white]{\,\,所有格 $+$ 名詞\,\,}\,がセットです\\
 目的格&〜を&目的語になります \\
 & &前置詞の後にくるときも \\

\end{tabular}
 \end{itemize}

     \end{block}
\end{frame}
%%%%%%%%%%%%%%%%%%%%%%%%%%%%%%%%
\againframe<11>[plain]{ichiran}
\end{document}
