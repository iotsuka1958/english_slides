\documentclass[aspectratio=169,xcolor={dvipsnames,table}]{beamer}
\usepackage[no-math,deluxe,haranoaji]{luatexja-preset}
\renewcommand{\kanjifamilydefault}{\gtdefault}
\renewcommand{\emph}[1]{{\upshape\bfseries #1}}
\usetheme{metropolis}
\metroset{block=fill}
\setbeamertemplate{navigation symbols}{}
\setbeamertemplate{blocks}[rounded][shadow=false]
\usecolortheme[rgb={0.7,0.2,0.2}]{structure}
%%%%%%%%%%%%%%%%%%%%%%%%%%%
%%%%%%%%%%%%%%%%%%%%%%%%%%%
%% さまざまなアイコン
%%%%%%%%%%%%%%%%%%%%%%%%%%%
%\usepackage{fontawesome}
\usepackage{fontawesome5}
\usepackage{figchild}
\usepackage{twemojis}
\usepackage{utfsym}
\usepackage{bclogo}
\usepackage{marvosym}
\usepackage{fontmfizz}
\usepackage{pifont}
\usepackage{phaistos}
\usepackage{worldflags}
\usepackage{jigsaw}
\usepackage{tikzlings}
\usepackage{tikzducks}
\usepackage{scsnowman}
\usepackage{epsdice}
\usepackage{halloweenmath}
\usepackage{svrsymbols}
\usepackage{countriesofeurope}
\usepackage{tipa}
\usepackage{manfnt}
%%%%%%%%%%%%%%%%%%%%%%%%%%%
\usepackage{tikz}
\usetikzlibrary{calc,patterns,decorations.pathmorphing,backgrounds}
\usepackage{tcolorbox}
\usepackage{tikzpeople}
\usepackage{circledsteps}
\usepackage{xcolor}
\usepackage{amsmath}
\usepackage{booktabs}
\usepackage{chronology}
\usepackage{signchart}
%%%%%%%%%%%%%%%%%%%%%%%%%%%
%% 場合分け
%%%%%%%%%%%%%%%%%%%%%%%%%%%
\usepackage{cases}
%%%%%%%%%%%%%%%%%%%%%%%%%%
\usepackage{pdfpages}
%%%%%%%%%%%%%%%%%%%%%%%%%%%
%% 音声リンク表示
\newcommand{\myaudio}[1]{\href{#1}{\faVolumeUp}}
%%%%%%%%%%%%%%%%%%%%%%%%%%
%% \myAnch{<名前>}{<色>}{<テキスト>}
%% 指定のテキストを指定の色の四角枠で囲み, 指定の名前をもつTikZの
%% ノードとして出力する. 図には remember picture 属性を付けている
%% ので外部から参照可能である.
\newcommand*{\myAnch}[3]{%
  \tikz[remember picture,baseline=(#1.base)]
    \node[draw,rectangle,line width=1pt,#2] (#1) {\normalcolor #3};
}
%%%%%%%%%%%%%%%%%%%%%%%%%%
%% \myEmph コマンドの定義
%%%%%%%%%%%%%%%%%%%%%%%%%%
%\newcommand{\myEmph}[3]{%
%    \textbf<#1>{\color<#1>{#2}{#3}}%
%}
\usepackage{xparse} % xparseパッケージの読み込み
\NewDocumentCommand{\myEmph}{O{} m m}{%
    \def\argOne{#1}%
    \ifx\argOne\empty
        \textbf{\color{#2}{#3}}% オプション引数が省略された場合
    \else
        \textbf<#1>{\color<#1>{#2}{#3}}% オプション引数が指定された場合
    \fi
}
%%%%%%%%%%%%%%%%%%%%%%%%%%%
%%%%%%%%%%%%%%%%%%%%%%%%%%%
%% 文末の上昇イントネーション記号 \myRisingPitch
%% 通常のイントネーション \myDownwardPitch
%% https://note.com/dan_oyama/n/n8be58e8797b2
%%%%%%%%%%%%%%%%%%%%%%%%%%%
\newcommand{\myRisingPitch}{
\begin{tikzpicture}[scale=0.3,baseline=0.3]
\draw[->,>=stealth] (0,0) to[bend right=45] (1,1);
\end{tikzpicture}
}
\newcommand{\myDownwardPitch}{
\begin{tikzpicture}[scale=0.3,baseline=0.3]
\draw[->,>=stealth] (0,1) to[bend left=45] (1,0);
\end{tikzpicture}
}
%%%%%%%%%%%%%%%%%%%%%%%%%%%%
%\AtBeginSection[%
%]{%
%  \begin{frame}[plain]\frametitle{授業の流れ}
%     \tableofcontents[currentsection]
%   \end{frame}%
%}

%%%%%%%%%%%%%%%%%%%%%%%%%%%
\title{English is fun.}
\subtitle{What is this?}
\author{}
\institute[]{}
\date[]

%%%%%%%%%%%%%%%%%%%%%%%%%%%%
%% TEXT
%%%%%%%%%%%%%%%%%%%%%%%%%%%%
\begin{document}
\begin{frame}[plain]
  \titlepage
\end{frame}

\section*{授業の流れ}
\begin{frame}[plain]
  \frametitle{授業の流れ}
  \tableofcontents
\end{frame}

%%%%%%%%%%%%%%%%%%%%%%%%%%%%%%%
\section{What 〜?}
\subsection{What is this?}
\begin{frame}[plain]{Whhat is this?} \Large

be動詞のとき

\mbox{}\hspace{55pt}This is \alt<3->{\myAnch{FOCUS}{orange}{a laptop}}{\myAnch{focus}{white}{a laptop}}.

\pause


\vspace{7pt}

\mbox{}\hfill{}cf. \myEmph[6-]{Maroon}{Is this} a laptop?%

\vspace{-5pt}

\hfill{\small YesまたはNoで答える疑問文}

\pause

\visible<4->{\myAnch{wh}{orange}{What} \myEmph[6-]{Maroon}{is this} \myAnch{question}{orange}{?}}
\visible<6->{\scalebox{1.4}{\myDownwardPitch}}

\pause

%\mbox{}\hspace{30pt}\myAnch{txt1}{white}{\small 先頭にWho}

\visible<5->{%
\begin{tikzpicture}[remember picture, overlay]
\draw[->, thick, orange] (focus.south) to[out=-90, in=90] (wh.north);
\end{tikzpicture}
}

\visible<7->{%
\begin{exampleblock}{Topics for Today}
\pause
\begin{itemize}\small
 \item 「〜は何ですか」と聞くとき$\longrightarrow$\,\,\,What is 〜?
 \item   文末に`?'をつける(イントネーションは\myDownwardPitch{}\,\,)
\end{itemize}
     \end{exampleblock}
}

\vspace{-10pt}
\visible<1->{%
\mbox{}\hfill\myaudio{./audio/019_what_01.mp3}
}
\end{frame}
%%%%%%%%%%%%%%%%%%%%%%%%%%%%%%%%%%%%%%%%%%%%%%
\subsection{What do you 〜?}
\begin{frame}[plain]\frametitle{What do you want for dinner?}
\Large
一般動詞のとき

\pause

\mbox{}\hspace{55pt}%
You want \alt<4->{\myAnch{FOCUS2}{orange}{pizza}}{\myAnch{focus2}{white}{pizza }} for dinner.

\pause

\mbox{}\hfill%
{\small cf. \myEmph[6-]{Maroon}{Do you want} pizza for dinner?}

\vspace{-5pt}

\hfill\visible<3->{{\small Yes/Noで答える疑問文}}

\visible<5->{\myAnch{WH2}{orange}{What} \myEmph[6-]{Maroon}{do you want} for dinner\myAnch{question2}{orange}{?}}

\visible<5->{%
\begin{tikzpicture}[remember picture, overlay]
 \draw[thick, orange, ->] (focus2.south) to[out=-90, in=90] (WH2.north);
\end{tikzpicture}
}

\visible<7->{%
\begin{exampleblock}{Topics for Today}
\pause
\begin{itemize}\small
 \item `What'を先頭に置いて、疑問文のかたちを続ける\,\,What do you  〜? / What does he 〜?
 \item   文末に`?'をつける(イントネーションは\myDownwardPitch{}\,\,)
\end{itemize}
     \end{exampleblock}
}
\mbox{}\hfill\myaudio{./audio/019_what_02.mp3}

\end{frame}
%%%%%%%%%%%%%%%%%%%%%%%%%%%
\begin{frame}[plain]{Exeicises}
 

(~~~~~) 内の語句を並べかえ、対話を完成させましょう。なお、先頭に来る語は大文字ではじめてください
\begin{enumerate}
 \item ( his name / is / what ) ? --- His name is John.\\
\visible<2->{What is his name?}
 \item ( favorite sport / is / what / your) ? --- I love baseball.\\
\visible<3->{{What is your favorite sport?}}
 \item (morning / you / drink / do / what / every ) ? --- I drink milk.\\
\visible<4->{What do  you drink every morning?}
 \item ( does / study / what / she ) every day? --- She studies English .\\
\visible<5->{What does she study every day?}\end{enumerate}

\mbox{}\hfill\myaudio{./audio/019_what_03.mp3}

\end{frame}
%%%%%%%%%%%%%%%%%%%%%%%%%%%%%%%%
\begin{frame}[plain]{$\text{What} + \text{名詞}$}
\begin{enumerate}
 \visible<1->{\item \myAnch{adj}{orange}{What\raisebox{5pt}{$\curvearrowright$}time} is it? --- It is ten forty.%\hfill{}{\small cf. It is ten forty.}\\
\\ {\small 何時}}
 \visible<2->{\item \myAnch{adj}{orange}{What\raisebox{5pt}{$\curvearrowright$}sport} do you like? --- I like baseball.\\ {\small 何のスポーツ}}
\end{enumerate}

\visible<3->{%
\begin{exampleblock}{Topics for Today}
\pause
\begin{itemize}\small
 \item \fbox{$\text{What} + \text{名詞}$}\,\,がひとかたまりになって「何の〜」の意味で使うことがあります
 \item \fbox{$\text{What} + \text{名詞}$}\,\,も文の先頭にきます
\end{itemize}
     \end{exampleblock}

\mbox{}\hfill\myaudio{./audio/019_what_04.mp3}

}

\end{frame}
%%%%%%%%%%%%%%%%%%%
\begin{frame}[plain]{Exercises}
 対話になるようになるよう(~~~~~~)内の語句を並べ替えましょう。先頭にくる語は
大文字で始めてください
\begin{enumerate}
 \item ( watch / what / you / movies / do ) ? --- I watch action movies.\\
\visible<2->{What movies do you watch?}
 \item ( books / you / read / what / do ) ? --- I read mystery novels.\\
\visible<3->{What books do you read?}
 \item ( hobbies / you / have / do / what ) ?\\
\mbox{}\hfill{}--- I have hobbies like painting and reading.\hfill{}{\small like 〜: 〜のような}\\
\visible<4->{What hobbies do you have?}
 \item ( is / day / what / it )  today? --- It's Saturday.\\
\visible<5->{What day is it today?}\hfill{}\visible<6->{{\small \dbend\,What day of the week is it today?}}
\end{enumerate}
\mbox{}\hfill\myaudio{./audio/019_what_05.mp3}

\end{frame}
%%%%%%%%%%%%%%%%%%%
\end{document}
