\documentclass[aspectratio=169,xcolor={dvipsnames,table}]{beamer}
\usepackage[no-math,deluxe,haranoaji]{luatexja-preset}
\renewcommand{\kanjifamilydefault}{\gtdefault}
\renewcommand{\emph}[1]{{\upshape\bfseries #1}}
\usetheme{metropolis}
\metroset{block=fill}
%%%%%%%%%%%%%%%%%%%%%%%%%%
\setbeamertemplate{navigation symbols}{}
\usecolortheme[rgb={0.7,0.2,0.2}]{structure}
%%%%%%%%%%%%%%%%%%%%%%%%%%
%% Change alert block colors
%%% 1- Block title (background and text)
\setbeamercolor{block title alerted}{fg=mDarkTeal, bg=mLightBrown!45!yellow!45}
\setbeamercolor{block title example}{fg=magenta!10!black, bg=mLightGreen!70}
%%% 2- Block body (background)
\setbeamercolor{block body alerted}{bg=mLightBrown!25}
\setbeamercolor{block body example}{bg=mLightGreen!15}
%%%%%%%%%%%%%%%%%%%%%%%%%%%
%%%%%%%%%%%%%%%%%%%%%%%%%%%
%% さまざまなアイコン
%%%%%%%%%%%%%%%%%%%%%%%%%%%
%\usepackage{fontawesome}
\usepackage{fontawesome5}
\usepackage{figchild}
\usepackage{twemojis}
\usepackage{utfsym}
\usepackage{bclogo}
\usepackage{marvosym}
\usepackage{fontmfizz}
\usepackage{pifont}
\usepackage{phaistos}
\usepackage{worldflags}
\usepackage{jigsaw}
\usepackage{tikzlings}
\usepackage{tikzducks}
\usepackage{scsnowman}
\usepackage{epsdice}
\usepackage{halloweenmath}
\usepackage{svrsymbols}
\usepackage{countriesofeurope}
\usepackage{tipa}
\usepackage{manfnt}
%%%%%%%%%%%%%%%%%%%%%%%%%%%
\usepackage{tikz}
\usetikzlibrary{calc,patterns,decorations.pathmorphing,backgrounds}
\usepackage{tcolorbox}
\usepackage{tikzpeople}
\usepackage{circledsteps}
\usepackage{xcolor}
\usepackage{amsmath}
\usepackage{booktabs}
\usepackage{chronology}
\usepackage{signchart}
%%%%%%%%%%%%%%%%%%%%%%%%%%%
%% 場合分け
%%%%%%%%%%%%%%%%%%%%%%%%%%%
\usepackage{cases}
%%%%%%%%%%%%%%%%%%%%%%%%%%
\usepackage{pdfpages}
%%%%%%%%%%%%%%%%%%%%%%%%%%%
%% 音声リンク表示
\newcommand{\myaudio}[1]{\href{#1}{\faVolumeUp}}
%%%%%%%%%%%%%%%%%%%%%%%%%%
%% \myAnch{<名前>}{<色>}{<テキスト>}
%% 指定のテキストを指定の色の四角枠で囲み, 指定の名前をもつTikZの
%% ノードとして出力する. 図には remember picture 属性を付けている
%% ので外部から参照可能である.
\newcommand*{\myAnch}[3]{%
  \tikz[remember picture,baseline=(#1.base)]
    \node[draw,rectangle,line width=1pt,#2] (#1) {\normalcolor #3};
}
%%%%%%%%%%%%%%%%%%%%%%%%%%
%% \myEmph コマンドの定義
%%%%%%%%%%%%%%%%%%%%%%%%%%
%\newcommand{\myEmph}[3]{%
%    \textbf<#1>{\color<#1>{#2}{#3}}%
%}
\usepackage{xparse} % xparseパッケージの読み込み
\NewDocumentCommand{\myEmph}{O{} m m}{%
    \def\argOne{#1}%
    \ifx\argOne\empty
        \textbf{\color{#2}{#3}}% オプション引数が省略された場合
    \else
        \textbf<#1>{\color<#1>{#2}{#3}}% オプション引数が指定された場合
    \fi
}
%%%%%%%%%%%%%%%%%%%%%%%%%%%
%%%%%%%%%%%%%%%%%%%%%%%%%%%
%% 文末の上昇イントネーション記号 \myRisingPitch
%% 通常のイントネーション \myDownwardPitch
%% https://note.com/dan_oyama/n/n8be58e8797b2
%%%%%%%%%%%%%%%%%%%%%%%%%%%
\newcommand{\myRisingPitch}{
\begin{tikzpicture}[scale=0.3,baseline=0.3]
\draw[->,>=stealth] (0,0) to[bend right=45] (1,1);
\end{tikzpicture}
}
\newcommand{\myDownwardPitch}{
\begin{tikzpicture}[scale=0.3,baseline=0.3]
\draw[->,>=stealth] (0,1) to[bend left=45] (1,0);
\end{tikzpicture}
}
%%%%%%%%%%%%%%%%%%%%%%%%%%%%
%\AtBeginSection[%
%]{%
%  \begin{frame}[plain]\frametitle{授業の流れ}
%     \tableofcontents[currentsection]
%   \end{frame}%
%}

%%%%%%%%%%%%%%%%%%%%%%%%%%
%%%%%%%%%%%%%%%%%%%%%%%%%%%
\AtBeginSection[%
]{%
  \begin{frame}[plain]\frametitle{授業の流れ}
     \tableofcontents[currentsection]
   \end{frame}%
}
%%%%%%%%%%%%%%%%%%%%%%%%%%%%
\title{English is fun.}
\subtitle{発音--母音と子音--}
\author{}
\institute[]{}
\date[]

%%%%%%%%%%%%%%%%%%%%%%%%%%%%
%% TEXT
%%%%%%%%%%%%%%%%%%%%%%%%%%%%
\begin{document}
%%%%%%%%%%%%%%%%%%%%%%%%%%%%%%
\begin{frame}[label=title]
%\phantomsection\label{section-1}
\thispagestyle{empty}
\titlepage
\end{frame}
%%%%%%%%%%%%%%%%%%%%%%%%%%%%%%
\section*{授業の流れ}
\begin{frame}[plain]
  \frametitle{授業の流れ}
  \tableofcontents
\end{frame}
%%%%%%%%%%%%%%%%%%%%%%%%%%%%%%
\section{母音と子音}
\begin{frame}[plain]{母音と子音}
\Large
\begin{description}
 \item[母音: ] \onslide<1->{日本語の「あ、い、う、え、お」に近い音} 
 \item[子音: ] \onslide<2->{それ以外の音}
\end{description}
\end{frame}
%%%%%%%%%%%%%%%%%%%%%%%%%%%%%
\begin{frame}[plain]{母音字と子音字}
\Large
\begin{description}
 \item[母音字: ] \onslide<1->{アルファベットで「母音」をあらわす文字: }%
               \onslide<2->{aeiou}
 \item[子音字: ] \onslide<3->{残りのアルファベット}
\end{description}

\bigskip

\Huge\centering
\begin{tabular}{cccccccccc}
\onslide<4->{\myEmph[5-]{BurntOrange}{a}}&
\onslide<4->{b}&
\onslide<4->{c}&
\onslide<4->{d}&
\onslide<4->{\myEmph[5-]{BurntOrange}{e}}&
\onslide<4->{f}&
\onslide<4->{g}&
\onslide<4->{h}&
\onslide<4->{\myEmph[5-]{BurntOrange}{i}}&
\onslide<4->{j} \\
\onslide<4->{k}&
\onslide<4->{l}&
\onslide<4->{m}&
\onslide<4->{n}&
\onslide<4->{\myEmph[5-]{BurntOrange}{o}}&
\onslide<4->{p}&
\onslide<4->{q}&
\onslide<4->{r}&
\onslide<4->{s}&
\onslide<4->{t}\\
\onslide<4->{\myEmph[5-]{BurntOrange}{u}}&
\onslide<4->{v}&
\onslide<4->{w}&
\onslide<4->{x}&
\onslide<4->{y}&
\onslide<4->{z}&
 & & &  \\
\end{tabular}
\end{frame}
%%%%%%%%%%%%%%%%%%%%%%%%%%%%%
\begin{frame}[plain]{確認}
 \Large
\begin{description}[        ]
 \item[母音・子音: ] 音 
 \item[母音字・子音字: ] 文字 
\end{description}
\end{frame}

%%%%%%%%%%%%%%%%%%%%%%%%%%%%%%
%% 背景色を黒に変更
\setbeamercolor{background canvas}{bg=black}
\begin{frame}
\centering
  \textcolor{white}{\LARGE\bfseries 実際の単語で確認しよう}
\end{frame}
\setbeamercolor{background canvas}{bg=}
%%%%%%%%%%%%%%%%%%%%%%%%%%%%%%%%%%%%%%%%%%%
\begin{frame}[plain]{cat}
\begin{columns}[t]
%%%%%%%%%%%%%%%%%%%%%%%%%%%%%%%
\begin{column}<1->{.3\textwidth}
  
 \Huge\centering

c\myEmph[2-]{BurntOrange}{a}t
\end{column}
%%%%%%%%%%%%%%%
\begin{column}<3->{.65\textwidth}
 \large
 \begin{tabular}[t]{cccc}
 \toprule
\onslide<3->{綴}&\onslide<3->{字}&\onslide<3->{音}&\onslide<3->{発音記号}\\\midrule
 \onslide<3->{c}&\onslide<4->{子音字}&\onslide<7->{子音} &\onslide<10->{\textipa{/k/}} \\
 \myEmph[3-]{BurntOrange}{a}&\onslide<5->{母音字}& \onslide<8->{母音} &\onslide<11->{\textipa{/\ae /}}\\
 \onslide<3->{t}&\onslide<6->{子音字}& \onslide<9->{子音} &\onslide<12->{\textipa{/t/}}\\
 \bottomrule
\end{tabular}
\end{column}
%%%%%%%%%%%%%%%%%%%%%%%%%%%%%%%%%%%
\end{columns}
\end{frame}
%%%%%%%%%%%%%%%%%%%%%%%%%%%%%%%%%%%%%%%%%%%
\begin{frame}[plain]{bed}
\begin{columns}[t]
%%%%%%%%%%%%%%%%%%%%%%%%%%%%%%%
\begin{column}<1->{.3\textwidth}
  
 \Huge\centering

b\myEmph[2-]{BurntOrange}{e}d
\end{column}
%%%%%%%%%%%%%%%
\begin{column}<3->{.65\textwidth}
 \large
 \begin{tabular}[t]{cccc}
 \toprule
\onslide<3->{綴}&\onslide<3->{字}&\onslide<3->{音}&\onslide<3->{発音記号}\\\midrule
 \onslide<3->{b}&\onslide<4->{子音字}&\onslide<7->{子音} &\onslide<10->{\textipa{/b/}} \\
 \myEmph[3-]{BurntOrange}{e}&\onslide<5->{母音字}& \onslide<8->{母音} &\onslide<11->{\textipa{/e/}}\\
 \onslide<3->{d}&\onslide<6->{子音字}& \onslide<9->{子音} &\onslide<12->{\textipa{/d/}}\\
 \bottomrule
\end{tabular}
\end{column}
%%%%%%%%%%%%%%%%%%%%%%%%%%%%%%%%%%%

\end{columns}
\end{frame}
%%%%%%%%%%%%%%%%%%%%%%%%%%%%%%%%%%%%%%%%
%%%%%%%%%%%%%%%%%%%%%%%%%%%%%%%%%%%%%%%%%%%
\begin{frame}[plain]{sit}
\begin{columns}[t]
%%%%%%%%%%%%%%%%%%%%%%%%%%%%%%%
\begin{column}<1->{.3\textwidth}
  
 \Huge\centering

s\myEmph[2-]{BurntOrange}{i}t
\end{column}
%%%%%%%%%%%%%%%
\begin{column}<3->{.65\textwidth}
 \large
 \begin{tabular}[t]{cccc}
 \toprule
\onslide<3->{綴}&\onslide<3->{字}&\onslide<3->{音}&\onslide<3->{発音記号}\\\midrule
 \onslide<3->{s}&\onslide<4->{子音字}&\onslide<7->{子音} &\onslide<10->{\textipa{/s/}} \\
 \myEmph[3-]{BurntOrange}{i}&\onslide<5->{母音字}& \onslide<8->{母音} &\onslide<11->{\textipa{/\textsci /}}\\
 \onslide<3->{t}&\onslide<6->{子音字}& \onslide<9->{子音} &\onslide<12->{\textipa{/t/}}\\
 \bottomrule
\end{tabular}
\end{column}
%%%%%%%%%%%%%%%%%%%%%%%%%%%%%%%%%%%
\end{columns}
\end{frame}
%%%%%%%%%%%%%%%%%%%%%%%%%%%%%%%%%%%%%%%%%%%
\begin{frame}[plain]{}
\begin{columns}[t]
%%%%%%%%%%%%%%%%%%%%%%%%%%%%%%%
\begin{column}<1->{.3\textwidth}
  
 \Huge\centering

d\myEmph[2-]{BurntOrange}{o}g
\end{column}
%%%%%%%%%%%%%%%
\begin{column}<3->{.65\textwidth}
 \large
 \begin{tabular}[t]{cccc}
 \toprule
\onslide<3->{綴}&\onslide<3->{字}&\onslide<3->{音}&\onslide<3->{発音記号}\\\midrule
 \onslide<3->{d}&\onslide<4->{子音字}&\onslide<7->{子音} &\onslide<10->{\textipa{/d/}} \\
 \myEmph[3-]{BurntOrange}{o}&\onslide<5->{母音字}& \onslide<8->{母音} &\onslide<11->{\textipa{/\textscripta/}}\\
 \onslide<3->{d}&\onslide<6->{子音字}& \onslide<9->{子音} &\onslide<12->{\textipa{/g/}}\\
 \bottomrule
\end{tabular}
\end{column}
%%%%%%%%%%%%%%%%%%%%%%%%%%%%%%%%%%%
\end{columns}
\end{frame}
%%%%%%%%%%%%%%%%%%%%%%%%%%%%%%%%%%%%%%%%%%%
\begin{frame}[plain]{}
\begin{columns}[t]
%%%%%%%%%%%%%%%%%%%%%%%%%%%%%%%
\begin{column}<1->{.3\textwidth}
  
 \Huge\centering

b\myEmph[2-]{BurntOrange}{u}s
\end{column}
%%%%%%%%%%%%%%%
\begin{column}<3->{.65\textwidth}
 \large
 \begin{tabular}[t]{cccc}
 \toprule
\onslide<3->{綴}&\onslide<3->{字}&\onslide<3->{音}&\onslide<3->{発音記号}\\\midrule
 \onslide<3->{b}&\onslide<4->{子音字}&\onslide<7->{子音} &\onslide<10->{\textipa{/b/}} \\
 \myEmph[3-]{BurntOrange}{u}&\onslide<5->{母音字}& \onslide<8->{母音} &\onslide<11->{\textipa{/\textturnv /}}\\
 \onslide<3->{s}&\onslide<6->{子音字}& \onslide<9->{子音} &\onslide<12->{\textipa{/s/}}\\
 \bottomrule
\end{tabular}
\end{column}
%%%%%%%%%%%%%%%%%%%%%%%%%%%%%%%%%%%
\end{columns}
\end{frame}
%%%%%%%%%%%%%%%%%%%%%%%%%%%%%%%%%%%%%%%%%%%%
\begin{frame}[plain]{発音練習}
 \LARGE

\begin{enumerate}
 \item cat\hfill\makebox[30pt][l]{\textipa{/k\ae t/}}\hspace{300pt}\mbox{}
 \item bed\hfill\makebox[30pt][l]{\textipa{/bed/}}\hspace{300pt}\mbox{}
 \item sit\hfill\makebox[30pt][l]{\textipa{/s\textsci t/}}\hspace{300pt}\mbox{}
 \item dog\hfill\makebox[30pt][l]{\textipa{/d\textscripta g/}}\hspace{300pt}\mbox{}
 \item bus\hfill\makebox[30pt][l]{\textipa{/b\textturnv s/}}\hspace{300pt}\mbox{}
\end{enumerate}

\myaudio{./audio/099_pronunciation_01.mp3}
\end{frame}
%%%%%%%%%%%%%%%%%%%%%%%%%%%%%%%%
\begin{frame}[plain]{確認}
\Large
 \begin{enumerate}
  \item 音
        \begin{enumerate}
	 \item 母音とはなんですか
	 \item 子音とはなんですか
	\end{enumerate}
  \item 文字
        \begin{enumerate}
	 \item 母音字とはなんですか
	 \item 子音字とはなんですか
	\end{enumerate}
 \end{enumerate}
\end{frame}
%%%%%%%%%%%%%%%%%%%
\section{有声音と無声音}
\begin{frame}[plain]{有声音と無声音}
\Large
 \begin{description}
  \item[有声音: ] \onslide<1->{声帯がふるえる音}
                  \onslide<3->{%
                  \begin{itemize}\large
		   \item \onslide<3->{すべての母音}
		   \item \onslide<4->{一部の子音}\hspace{25pt}\onslide<5->{{\small たとえば \textipa{/b/}\,\,\,\textipa{/d/}\,\,\,\textipa{/g/}\,\,\,\textipa{/v/}\,\,\,\textipa{/z/}}}
		  \end{itemize}}
  \item[無声音: ] \onslide<2->{声帯がふるえない音}
                  \onslide<6->{%
                  \begin{itemize}\large
		   \item 一部の子音\hspace{25pt}\onslide<7->{{\small たとえば \textipa{/p/}\,\,\,\textipa{/t/}\,\,\,\textipa{/k/}\,\,\,\textipa{/f/}\,\,\,\textipa{/s/}}}
		  \end{itemize}}
 \end{description}
\end{frame}
%%%%%%%%%%%%%%%%%%%%%%%%%
\begin{frame}[plain]{有声音と無声音の区別法}
 \Large
\begin{description}
 \item[方法1:] \onslide<2->{喉に手をあてる}\,\,\,\,\,\,\,\onslide<3->{$\longrightarrow$ ふるえたら「有声音」}
 \item[方法2:] \onslide<4->{耳を両手でふさぐ}\,\onslide<5->{$\longrightarrow$ 反響したら「有声音」}
\end{description}
\end{frame}
%%%%%%%%%%%%%%%%%%%%%%%%%%
\begin{frame}[plain]{有声音と無声音のペア}
\Large 
 \begin{enumerate}
  \item \onslide<2->{\textipa{/b/}と\textipa{/p/}}\hfill%
\onslide<3->{\makebox[35pt][l]{\underline{b}ig} /\hspace{8pt}\makebox[40pt][l]{\underline{p}ig}}\hspace{180pt}\mbox{}
  \item \onslide<4->{\textipa{/d/}と\textipa{/t/}}\hfill%
\onslide<5->{\makebox[35pt][l]{ba\underline{d}} /\hspace{8pt}\makebox[40pt][l]{ba\underline{t}}}\hspace{180pt}\mbox{}
  \item  \onslide<6->{\textipa{/g/}と\textipa{/k/}}\hfill%
\onslide<7->{\makebox[35pt][l]{\underline{g}ap} /\hspace{8pt}\makebox[40pt][l]{\underline{c}ap}}\hspace{180pt}\mbox{}
  \item  \onslide<8->{\textipa{/v/}と\textipa{/f/}}\hfill%
\onslide<9->{\makebox[35pt][l]{\underline{v}ase} /\hspace{8pt}\makebox[40pt][l]{\underline{f}ace}}\hspace{180pt}\mbox{}
  \item  \onslide<10->{\textipa{/z/}と\textipa{/s/}}\hfill%
\onslide<11->{\makebox[35pt][l]{\underline{z}oo} /\hspace{8pt}\makebox[40pt][l]{\underline{S}ue}}\hspace{180pt}\mbox{}
 \end{enumerate}

\bigskip

\normalsize
\mbox{}\hfill\myaudio{./audio/099_pronunciation_02.mp3}

\end{frame}
\end{document}
