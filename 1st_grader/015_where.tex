\documentclass[aspectratio=169]{beamer}
\usepackage[no-math,deluxe,haranoaji]{luatexja-preset}
\renewcommand{\kanjifamilydefault}{\gtdefault}
\renewcommand{\emph}[1]{{\upshape\bfseries #1}}
\usetheme{metropolis}
\metroset{block=fill}
\setbeamertemplate{navigation symbols}{}
\usecolortheme[rgb={0.7,0.2,0.2}]{structure}
%%%%%%%%%%%%%%%%%%%%%%%%%%%
\usepackage{media9}
%%%%%%%%%%%%%%%%%%%%%%%%%%%
%% さまざまなアイコン
%%%%%%%%%%%%%%%%%%%%%%%%%%%
\usepackage{fontawesome}
\usepackage{figchild}
\usepackage{twemojis}
\usepackage{utfsym}
\usepackage{bclogo}
\usepackage{marvosym}
\usepackage{fontmfizz}
\usepackage{pifont}
\usepackage{phaistos}
\usepackage{worldflags}
%%%%%%%%%%%%%%%%%%%%%%%%%%%
\usepackage{tikz}
\usetikzlibrary{backgrounds}
\usepackage{tcolorbox}
\usepackage{tikzpeople}
\usepackage{xcolor}
\usepackage{amsmath}
%%%%%%%%%%%%%%%%%%%%%%%%%%%
%% 場合分け
\usepackage{cases}
%%%%%%%%%%%%%%%%%%%%%%%%%%%
% \myAnch{<名前>}{<色>}{<テキスト>}
% 指定のテキストを指定の色の四角枠で囲み, 指定の名前をもつTikZの
% ノードとして出力する. 図には remeber picture 属性を付けている
% ので外部から参照可能である.
\newcommand*{\myAnch}[3]{%
  \tikz[remember picture,baseline=(#1.base)]
    \node[draw,rectangle,#2] (#1) {\normalcolor #3};
}
%%%%%%%%%%%%%%%%%%%%%%%%%%%%
%% 音声リンク表示
\newcommand{\myaudio}[1]{\href{#1}{\faVolumeUp}}
%%%%%%%%%%%%%%%%%%%%%%%%%%%
% \myEmph コマンドの定義
%\newcommand{\myEmph}[3]{%
%    \textbf<#1>{\color<#1>{#2}{#3}}%
%}
\usepackage{xparse} % xparseパッケージの読み込み
\NewDocumentCommand{\myEmph}{O{} m m}{%
    \def\argOne{#1}%
    \ifx\argOne\empty
        \textbf{\color{#2}{#3}}% オプション引数が省略された場合
    \else
        \textbf<#1>{\color<#1>{#2}{#3}}% オプション引数が指定された場合
    \fi
}
%%%%%%%%%%%%%%%%%%%%%%%%%%%
%% 文末の上昇イントネーション記号 \myRisingPitch
%% 通常のイントネーション \myDownwardPitch
%% https://note.com/dan_oyama/n/n8be58e8797b2
%%%%%%%%%%%%%%%%%%%%%%%%%%%
\newcommand{\myRisingPitch}{
\begin{tikzpicture}[scale=0.3,baseline=0.3]
\draw[->,>=stealth] (0,0) to[bend right=45] (1,1);
\end{tikzpicture}
}
\newcommand{\myDownwardPitch}{
\begin{tikzpicture}[scale=0.3,baseline=0.3]
\draw[->,>=stealth] (0,1) to[bend left=45] (1,0);
\end{tikzpicture}
}
%%%%%%%%%%%%%%%%%%%%%%%%%%%
\title{English is fun.\,\,{}--- Where does Jane live? ---}
\author{}
\institute[]{}
\date[]

%%%%%%%%%%%%%%%%%%%%%%%%%%%%
%% TEXT
%%%%%%%%%%%%%%%%%%%%%%%%%%%%
\begin{document}
\begin{frame}[plain]
  \titlepage
\end{frame}

\section*{授業の流れ}
\begin{frame}[plain]
  \frametitle{授業の流れ}
  \tableofcontents
\end{frame}


\section{When}

\subsection{Where is your key?: be動詞のとき}
\begin{frame}[plain]{When is your birthday?}
 \Large

be動詞のとき

\mbox{}\hspace{57pt}Your key is on the desk.

\pause


\vspace{7pt}

\mbox{}\hspace{55pt}Is your key \alt<3->{\myAnch{FOCUS}{orange}{on the desk}}{\myAnch{focus}{white}{on the desk}}?%
\hspace{20pt}{\normalsize YesまたはNoで答える疑問文}

\vspace{7pt}

\pause

\visible<4->{\myAnch{wh}{orange}{Where} is your key \myAnch{question}{orange}{?}}
\visible<6->{\scalebox{1.4}{\myDownwardPitch}}

\pause

%\mbox{}\hspace{30pt}\myAnch{txt1}{white}{\small 先頭にWho}

\visible<5->{%
\begin{tikzpicture}[remember picture, overlay]
\draw[->, thick, orange] (focus.south) to[out=-165, in=30] (wh.north);
\end{tikzpicture}
}

\visible<7->{%
\begin{exampleblock}{Topics for Today}
\pause
\begin{itemize}\small
 \item 「〜はどこ」と聞くとき$\longrightarrow$\,\,\,Where is 〜?
 \item   文末に`?'をつける(イントネーションは\myDownwardPitch{}\,\,)
\end{itemize}
     \end{exampleblock}
}
\visible<8->{%
\mbox{}\hfill\myaudio{./audio/014_where_01.mp3}
}
\end{frame}

\subsection{Where do they live?: 一般動詞のとき}
\begin{frame}[plain]\frametitle{Where do they live?}
\Large
一般動詞のとき

\pause

\mbox{}\hspace{75pt}%
They lives in Sydney.

\pause

\mbox{}\hspace{55pt}%
Do they live \alt<5->{\myAnch{FOCUS2}{orange}{in Sydney}}{\myAnch{focus2}{white}{in Sydney}}?
\hspace{10pt}\visible<4->{{\normalsize Yes/Noで答える疑問文}}

\visible<6->{\myAnch{WH2}{orange}{Where} do they live?}

\visible<7->{%
\begin{tikzpicture}[remember picture, overlay]
 \draw[thick, orange, ->] (focus2.south) to[out=-165, in=15] (WH2.north east);
\end{tikzpicture}
}

\visible<8->{%
\begin{exampleblock}{Topics for Today}
\pause
\begin{itemize}\small
 \item `When'を先頭に置いて、疑問文のかたちを続ける\,\,When do you  〜?
 \item   文末に`?'をつける(イントネーションは\myDownwardPitch{}\,\,)
\end{itemize}
     \end{exampleblock}
}
\visible<9->{%
\mbox{}\hfill\myaudio{./audio/014_when_02.mp3}
}
\end{frame}

\subsection{Exercises}
\begin{frame}[plain]{Exercises}
つぎの文の意味を考えましょう。

\begin{tabular}{rll}
1& Where is my phone? &  {\small phone: 電話} \\
2& Where are your shoes?&   {\small shoe:くつ}\\
3& Where do they go for vacation? &   {\small vacation: バケーション}\\
4& Where does the cat sleep?&  {\small sleep: 眠る} \\
\end{tabular}

\visible<2->{%
\mbox{}\hfill\myaudio{./audio/014_where_03.mp3}
}
\end{frame}

\begin{frame}[plain]{Exercises}
 次の質問に対する答えとしてもっとも適切なものを,下のア~ウの中から選びましょう。

\begin{enumerate}
 \item A: Where is Mr. Brown from?\\
B: \alt<2->{He's from the USA.}{(\hspace{100pt})}
 \item A: Where does Naomi live?\\
B: \alt<3->{She lives in Narita.}{(\hspace{100pt})}
 \item A: Where does this bus go?\\
B: \alt<4->{It goes to Chiba Station.}{(\hspace{100pt})}
 \item A: Where does Tom study?\\
B: \alt<5->{He studies in the library.}{(\hspace{100pt})}
 \item A: Where is Mary now?\\
B:\alt<6->{She's in the kitchen now.}{(\hspace{100pt})}
\end{enumerate}
\
\begin{tcolorbox}
\centering
ア She lives in Narita.
イ He studies in the library.
ウ He's from the USA.
エ It goes to Chiba Station.
オ She's in the kitchen now.
\end{tcolorbox}

\visible<8->{%
\mbox{}\hfill\myaudio{./audio/014_when_04.mp3}
}

\end{frame}


\begin{frame}[plain]{Exercises}
 (~~~~~~~~) 内の語句を並べかえ、AとBの対話を完成させましょう。なお、先頭の語は大文字で始めてください。

\begin{enumerate}
 \item A: ( Children's Day / when / is ) ? 
\hspace{137.4pt}B: It's May 5th.\\
\phantom{A: }\visible<2->{When is Children's Day?}
 \item A: ( birthday / your / father's / when / is ) ?
\hspace{83.5pt}B: It's July 2nd.\\
\phantom{A: }\visible<3->{When is your father's birthday?}
 \item A: ( to school / does / Mr. Brown / come / when ) ?
\hspace{47pt}B: Around 7:00.\\
\phantom{A: }\visible<4->{When does Mr. Brown come to school?}
 \item A: ( does / the violin / when / play / your brother ) ?
\hspace{39.7pt}B: After dinner.\\
\phantom{A: }\visible<5->{When does your brother play the violin?}
\end{enumerate}
\visible<6->{%
\mbox{}\hfill\myaudio{./audio/014_when_05.mp3}
}
\end{frame}
\end{document}
