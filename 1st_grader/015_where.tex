\documentclass[aspectratio=169,xcolor={dvipsnames,table}]{beamer}
\usepackage[no-math,deluxe,haranoaji]{luatexja-preset}
\renewcommand{\kanjifamilydefault}{\gtdefault}
\renewcommand{\emph}[1]{{\upshape\bfseries #1}}
\usetheme{metropolis}
\metroset{block=fill}
\setbeamertemplate{navigation symbols}{}
\setbeamertemplate{blocks}[rounded][shadow=false]
\usecolortheme[rgb={0.7,0.2,0.2}]{structure}
%%%%%%%%%%%%%%%%%%%%%%%%%%
%%%%%%%%%%%%%%%%%%%%%%%%%%%
%% さまざまなアイコン
%%%%%%%%%%%%%%%%%%%%%%%%%%%
%\usepackage{fontawesome}
\usepackage{fontawesome5}
\usepackage{figchild}
\usepackage{twemojis}
\usepackage{utfsym}
\usepackage{bclogo}
\usepackage{marvosym}
\usepackage{fontmfizz}
\usepackage{pifont}
\usepackage{phaistos}
\usepackage{worldflags}
\usepackage{jigsaw}
\usepackage{tikzlings}
\usepackage{tikzducks}
\usepackage{scsnowman}
\usepackage{epsdice}
\usepackage{halloweenmath}
\usepackage{svrsymbols}
\usepackage{countriesofeurope}
\usepackage{tipa}
\usepackage{manfnt}
%%%%%%%%%%%%%%%%%%%%%%%%%%%
\usepackage{tikz}
\usetikzlibrary{calc,patterns,decorations.pathmorphing,backgrounds}
\usepackage{tcolorbox}
\usepackage{tikzpeople}
\usepackage{circledsteps}
\usepackage{xcolor}
\usepackage{amsmath}
\usepackage{booktabs}
\usepackage{chronology}
\usepackage{signchart}
%%%%%%%%%%%%%%%%%%%%%%%%%%%
%% 場合分け
%%%%%%%%%%%%%%%%%%%%%%%%%%%
\usepackage{cases}
%%%%%%%%%%%%%%%%%%%%%%%%%%
\usepackage{pdfpages}
%%%%%%%%%%%%%%%%%%%%%%%%%%%
%% 音声リンク表示
\newcommand{\myaudio}[1]{\href{#1}{\faVolumeUp}}
%%%%%%%%%%%%%%%%%%%%%%%%%%
%% \myAnch{<名前>}{<色>}{<テキスト>}
%% 指定のテキストを指定の色の四角枠で囲み, 指定の名前をもつTikZの
%% ノードとして出力する. 図には remember picture 属性を付けている
%% ので外部から参照可能である.
\newcommand*{\myAnch}[3]{%
  \tikz[remember picture,baseline=(#1.base)]
    \node[draw,rectangle,line width=1pt,#2] (#1) {\normalcolor #3};
}
%%%%%%%%%%%%%%%%%%%%%%%%%%
%% \myEmph コマンドの定義
%%%%%%%%%%%%%%%%%%%%%%%%%%
%\newcommand{\myEmph}[3]{%
%    \textbf<#1>{\color<#1>{#2}{#3}}%
%}
\usepackage{xparse} % xparseパッケージの読み込み
\NewDocumentCommand{\myEmph}{O{} m m}{%
    \def\argOne{#1}%
    \ifx\argOne\empty
        \textbf{\color{#2}{#3}}% オプション引数が省略された場合
    \else
        \textbf<#1>{\color<#1>{#2}{#3}}% オプション引数が指定された場合
    \fi
}
%%%%%%%%%%%%%%%%%%%%%%%%%%%
%%%%%%%%%%%%%%%%%%%%%%%%%%%
%% 文末の上昇イントネーション記号 \myRisingPitch
%% 通常のイントネーション \myDownwardPitch
%% https://note.com/dan_oyama/n/n8be58e8797b2
%%%%%%%%%%%%%%%%%%%%%%%%%%%
\newcommand{\myRisingPitch}{
\begin{tikzpicture}[scale=0.3,baseline=0.3]
\draw[->,>=stealth] (0,0) to[bend right=45] (1,1);
\end{tikzpicture}
}
\newcommand{\myDownwardPitch}{
\begin{tikzpicture}[scale=0.3,baseline=0.3]
\draw[->,>=stealth] (0,1) to[bend left=45] (1,0);
\end{tikzpicture}
}
%%%%%%%%%%%%%%%%%%%%%%%%%%%%
%\AtBeginSection[%
%]{%
%  \begin{frame}[plain]\frametitle{授業の流れ}
%     \tableofcontents[currentsection]
%   \end{frame}%
%}

%%%%%%%%%%%%%%%%%%%%%%%%%%%
\title{English is fun.}
\subtitle{Where does Jane live?}
\author{}
\institute[]{}
\date[]

%%%%%%%%%%%%%%%%%%%%%%%%%%%%
%% TEXT
%%%%%%%%%%%%%%%%%%%%%%%%%%%%
\begin{document}
\begin{frame}[plain]
  \titlepage
\end{frame}

\section*{授業の流れ}
\begin{frame}[plain]
  \frametitle{授業の流れ}
  \tableofcontents
\end{frame}


\section{where \textipa{/w\'e\textrhookschwa /}}

\subsection{Where is your key?: be動詞のとき}
\begin{frame}[plain]{Where is your key?}
 \Large

be動詞のとき

\mbox{}\hspace{10pt}Your key is  \alt<4->{\myAnch{FOCUS}{orange}{on the desk}}{\myAnch{focus}{white}{on the desk}}.
\pause
\hfill{}cf. \myEmph[6-]{Maroon}{Is your key} on the desk?\pause%

\hfill{\normalsize Yes/Noで答える疑問文}

\vspace{7pt}

\pause

\visible<5->{\myAnch{wh}{orange}{Where} \myEmph[6-]{Maroon}{is your key} \myAnch{question}{orange}{?}}
\visible<7->{\scalebox{1.4}{\myDownwardPitch}}

\pause

%\mbox{}\hspace{30pt}\myAnch{txt1}{white}{\small 先頭にWho}

\visible<5->{%
\begin{tikzpicture}[remember picture, overlay]
\draw[->, thick, orange] (focus.south) to[out=-90, in=90] (wh.north);
\end{tikzpicture}
}

\visible<8->{%
\begin{exampleblock}{Topics for Today}
\pause
\begin{itemize}\setbeamertemplate{items}[square]\small
 \item 「どこ」と聞くときは{\bfseries where}を先頭に
 \item {\bfseries Where}に続く部分は疑問文の語順
 \item   文末に`?'をつける(イントネーションは\myDownwardPitch{}\,\,)
\end{itemize}
     \end{exampleblock}
}
\mbox{}\hfill\myaudio{./audio/015_where_01.mp3}

\end{frame}

\subsection{Where do they live?: 一般動詞のとき}
\begin{frame}[plain]\frametitle{Where do they live?}
\Large
一般動詞のとき

They live \alt<4->{\myAnch{FOCUS2}{orange}{in Sydney}}{\myAnch{focus2}{white}{in Sydney}}.
\pause
\hfill{}%
cf. \myEmph[6-]{Maroon}{Do they live} in Sydney?\pause

\hfill{\normalsize Yes/Noで答える疑問文}

\visible<5->{\myAnch{WH2}{orange}{Where}  \myEmph[6-]{Maroon}{do they live}\myAnch{question}{orange}{?}}

\visible<5->{%
\begin{tikzpicture}[remember picture, overlay]
 \draw[thick, orange, ->] (focus2.south) to[out=-165, in=15] (WH2.north east);
\end{tikzpicture}
}

\visible<7->{%
\begin{exampleblock}{Topics for Today}
\pause
\begin{itemize}\setbeamertemplate{items}[square]\small
 \item 「どこ」と聞くときは{\bfseries where}を先頭に
 \item {\bfseries Where}に続く部分は疑問文の語順
 \item   文末に`?'をつける(イントネーションは\myDownwardPitch{}\,\,)
\end{itemize}
     \end{exampleblock}
}
\mbox{}\hfill\myaudio{./audio/015_where_02.mp3}

\end{frame}

\subsection{Exercises}
\begin{frame}[plain]{Exercises}
つぎの文の意味を考えましょう。

\begin{tabular}{rll}
1& Where is my phone? &  {\small phone: 電話} \\
2& Where are your shoes?&   {\small shoe:くつ}\\
3& Where do they go for vacation? &   {\small vacation: バケーション}\\
4& Where does the cat sleep?&  {\small sleep: 眠る} \\
\end{tabular}

\mbox{}\hfill\myaudio{./audio/015_where_03.mp3}

\end{frame}

\begin{frame}[plain]{Exercises}
 次の質問に対する答えとしてもっとも適切なものを、右のア~オの中から選びましょう。

\begin{columns}[t]
 \begin{column}{.475\textwidth}
\begin{enumerate}
 \item A: Where is Mr. Brown from?\\
B: \alt<2->{He's from the USA. ウ}{(\hspace{100pt})}
 \item A: Where does Naomi live?\\
B: \alt<3->{She lives in Narita. ア}{(\hspace{100pt})}
 \item A: Where does this bus go?\\
B: \alt<4->{It goes to Chiba Station. エ}{(\hspace{100pt})}
 \item A: Where does Tom study?\\
B: \alt<5->{He studies in the library. イ}{(\hspace{100pt})}
 \item A: Where is Mary now?\\
B: \alt<6->{She's in the kitchen now. オ}{(\hspace{100pt})}
\end{enumerate}  
 \end{column}
\begin{column}{.475\textwidth}
\begin{tcolorbox}
ア She lives in Narita.\\
イ He studies in the library.\\
ウ He's from the USA.\\
エ It goes to Chiba Station.\\
オ She's in the kitchen now.
\end{tcolorbox} 
\mbox{}\hfill\myaudio{./audio/015_where_04.mp3}
\end{column}
\end{columns}
\end{frame}
%%%%%%%%%%%%%%%%%%%%%%%
\end{document}
