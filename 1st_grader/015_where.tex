\documentclass[aspectratio=169,xcolor={dvipsnames,table}]{beamer}
\usepackage[no-math,deluxe,haranoaji]{luatexja-preset}
\renewcommand{\kanjifamilydefault}{\gtdefault}
\renewcommand{\emph}[1]{{\upshape\bfseries #1}}
\usetheme{metropolis}
\metroset{block=fill}
\setbeamertemplate{navigation symbols}{}
\setbeamertemplate{blocks}[rounded][shadow=false]
\usecolortheme[rgb={0.7,0.2,0.2}]{structure}

%%%%%%%%%%%%%%%%%%%%%%%%%%
%%%%%%%%%%%%%%%%%%%%%%%%%%%
%% さまざまなアイコン
%%%%%%%%%%%%%%%%%%%%%%%%%%%
%\usepackage{fontawesome}
\usepackage{fontawesome5}
\usepackage{figchild}
\usepackage{twemojis}
\usepackage{utfsym}
\usepackage{bclogo}
\usepackage{marvosym}
\usepackage{fontmfizz}
\usepackage{pifont}
\usepackage{phaistos}
\usepackage{worldflags}
\usepackage{jigsaw}
\usepackage{tikzlings}
\usepackage{tikzducks}
\usepackage{scsnowman}
\usepackage{epsdice}
\usepackage{halloweenmath}
\usepackage{svrsymbols}
\usepackage{countriesofeurope}
\usepackage{tipa}
%%%%%%%%%%%%%%%%%%%%%%%%%%%
\usepackage{tikz}
\usetikzlibrary{calc,patterns,decorations.pathmorphing,backgrounds}
\usepackage{tcolorbox}
\usepackage{tikzpeople}
\usepackage{circledsteps}
\usepackage{xcolor}
\usepackage{amsmath}
\usepackage{booktabs}
\usepackage{chronology}
\usepackage{signchart}
%%%%%%%%%%%%%%%%%%%%%%%%%%%
%% 場合分け
%%%%%%%%%%%%%%%%%%%%%%%%%%%
\usepackage{cases}
%%%%%%%%%%%%%%%%%%%%%%%%%%
\usepackage{pdfpages}
%%%%%%%%%%%%%%%%%%%%%%%%%%%
%% 音声リンク表示
\newcommand{\myaudio}[1]{\href{#1}{\faVolumeUp}}
%%%%%%%%%%%%%%%%%%%%%%%%%%
%% \myAnch{<名前>}{<色>}{<テキスト>}
%% 指定のテキストを指定の色の四角枠で囲み, 指定の名前をもつTikZの
%% ノードとして出力する. 図には remember picture 属性を付けている
%% ので外部から参照可能である.
\newcommand*{\myAnch}[3]{%
  \tikz[remember picture,baseline=(#1.base)]
    \node[draw,rectangle,line width=1pt,#2] (#1) {\normalcolor #3};
}
%%%%%%%%%%%%%%%%%%%%%%%%%%
%% \myEmph コマンドの定義
%%%%%%%%%%%%%%%%%%%%%%%%%%
%\newcommand{\myEmph}[3]{%
%    \textbf<#1>{\color<#1>{#2}{#3}}%
%}
\usepackage{xparse} % xparseパッケージの読み込み
\NewDocumentCommand{\myEmph}{O{} m m}{%
    \def\argOne{#1}%
    \ifx\argOne\empty
        \textbf{\color{#2}{#3}}% オプション引数が省略された場合
    \else
        \textbf<#1>{\color<#1>{#2}{#3}}% オプション引数が指定された場合
    \fi
}
%%%%%%%%%%%%%%%%%%%%%%%%%%%
%%%%%%%%%%%%%%%%%%%%%%%%%%%
%% 文末の上昇イントネーション記号 \myRisingPitch
%% 通常のイントネーション \myDownwardPitch
%% https://note.com/dan_oyama/n/n8be58e8797b2
%%%%%%%%%%%%%%%%%%%%%%%%%%%
\newcommand{\myRisingPitch}{
\begin{tikzpicture}[scale=0.3,baseline=0.3]
\draw[->,>=stealth] (0,0) to[bend right=45] (1,1);
\end{tikzpicture}
}
\newcommand{\myDownwardPitch}{
\begin{tikzpicture}[scale=0.3,baseline=0.3]
\draw[->,>=stealth] (0,1) to[bend left=45] (1,0);
\end{tikzpicture}
}
%%%%%%%%%%%%%%%%%%%%%%%%%%%%
%\AtBeginSection[%
%]{%
%  \begin{frame}[plain]\frametitle{授業の流れ}
%     \tableofcontents[currentsection]
%   \end{frame}%
%}

\usepackage{highlightx}
\usepackage{lua-ul}
\usepackage{pxrubrica}
%%%%%%%%%%%%%%%%%%%%%%%%%%%
\title{English is fun.}
\subtitle{Where does Jane live?}
\author{}
\institute[]{}
\date[]

%%%%%%%%%%%%%%%%%%%%%%%%%%%%
%% TEXT
%%%%%%%%%%%%%%%%%%%%%%%%%%%%
\begin{document}
\begin{frame}[plain]
  \titlepage
\end{frame}

\section*{授業の流れ}
\begin{frame}[plain]
  \frametitle{授業の流れ}
  \tableofcontents
\end{frame}


\section{where \textipa{/w\'e\textrhookschwa /}}

\subsection{Where is your key?: be動詞のとき}
\begin{frame}[plain]{Where is your key?}
 %\Large

be動詞のとき

\mbox{}\hspace{40pt}Your key is  \alt<4->{\myAnch{FOCUS}{orange}{on the desk}}{\myAnch{focus}{white}{on the desk}}.
\pause
\hfill{}cf. \myEmph[6-]{Maroon}{Is your key} on the desk?\pause%

\hfill{\normalsize Yes/Noで答える疑問文}

\vspace{7pt}

\visible<5->{\myAnch{wh}{orange}{Where} \myEmph[6-]{Maroon}{is your key} ?}%\myAnch{question}{orange}{?}}
\visible<7->{\scalebox{1.4}{\myDownwardPitch}}

\pause

%\mbox{}\hspace{30pt}\myAnch{txt1}{white}{\small 先頭にWho}

\visible<5->{%
\begin{tikzpicture}[remember picture, overlay]
\draw[->, line width=3pt, opacity =.5,orange] (focus.south) to[out=-90, in=90]node[sloped,above,text=black,font=\tiny,pos=.6]{Whereに置き換えて先頭へ} node[sloped,below,text=black,font=\tiny,pos=.4]{その後は疑問文の語順} (wh.north);
\end{tikzpicture}
}

\begin{block}<8->{Topics for Today}
\pause
\begin{itemize}\setbeamertemplate{items}[square]\small
 \item 「どこ」と聞くときは疑問詞の{\bfseries where}を先頭に
 \item {\bfseries Where}に続く部分はyes/noで答える疑問文の語順
 \item   文末に`?'をつける / イントネーションは\myDownwardPitch
\end{itemize}
     \end{block}

\mbox{}\hfill{\tiny 0132}\,{\scriptsize \myaudio{./audio/015_where_01.mp3}}

\end{frame}

\subsection{Where do they live?: 一般動詞のとき}
\begin{frame}[plain]\frametitle{Where do they live?}
%\Large
一般動詞のとき

\hspace*{55pt}They live \alt<4->{\myAnch{FOCUS2}{orange}{in Sydney}}{\myAnch{focus2}{white}{in Sydney}}.\,\,\,\,\,{\scriptsize \textipa{/s\'Idni/}}
\pause
\hfill{}%
cf. \myEmph[6-]{Maroon}{Do they live} in Sydney?\pause

\hfill{\scriptsize Yes/Noで答える疑問文}

\visible<5->{\myAnch{WH2}{orange}{Where}  \myEmph[6-]{Maroon}{do they live} ?}%\myAnch{question}{orange}{?}}

\visible<5->{%
\begin{tikzpicture}[remember picture, overlay]
 \draw[line width=3pt, opacity=.5, orange, ->] (focus2.south) to[out=-90, in=90] node[sloped,above,text=black,font=\tiny,pos=.6]{Whereに置き換えて先頭へ} node[sloped,below,text=black,font=\tiny,pos=.5]{その後は疑問文の語順} (WH2.north);
\end{tikzpicture}
}

\begin{block}<7->{Topics for Today}\small
\begin{itemize}\setbeamertemplate{items}[square]\small
 \item 「どこ」と聞くときは疑問詞{\bfseries where}を先頭に
 \item {\bfseries Where}に続く部分はyes/noで答える疑問文の語順
 \item   文末に`?'をつける / イントネーションは\myDownwardPitch{}
\end{itemize}
     \end{block}

\mbox{}\hfill{\tiny 0131}\,{\scriptsize \myaudio{./audio/015_where_02.mp3}}

\end{frame}

\subsection{Exercises}
\begin{frame}[plain]{Exercises}

{\small つぎの文の意味を考えましょう}

\begin{tabular}{rll}
1& Where is my phone? &  {\scriptsize phone \textipa{/f\'oUn/}: 電話} \\
2& Where are your shoes?&   {\scriptsize shoe \textipa{/S\'u:/} くつ}\\
3& Where do they go for vacation? &   {\scriptsize vacation \textipa{/veIk\'eIS@n/} バケーション}\\
4& Where does the cat sleep?&  {\scriptsize sleep \textipa{/sl\'\i :p/} 眠る} \\
\end{tabular}

\mbox{}\hfill{\tiny 0155}\,{\scriptsize \myaudio{./audio/015_where_03.mp3}}
\bigskip

\visible<2->{{\small なぜ1はWhere \Circled{\textbf{\,is\,}} \ldots\,?\,で、2はWhere \Circled{\textbf{are}} \ldots\,?\,なのでしょうか}}

\begin{itemize}\setbeamertemplate{items}[circle]
 \item<3-> My phone is on the desk.
 \item<4-> Your shoes are in the box.
\end{itemize}

\end{frame}
%%%%%%%%%%%%%%%%%%%%%%%%%%%%%%%%%%
\begin{frame}[plain]{Exercises}

 {\small 次の質問に対する答えとしてもっとも適切なものを、右のア~オの中から選びましょう}

\begin{columns}[t]
 \begin{column}{.475\textwidth}
\begin{enumerate}
 \item A: Where is Mr. Brown from?\\
B: \alt<2->{He's from the USA. ウ}{(\hspace{100pt})}
 \item A: Where does Naomi live?\\
B: \alt<3->{She lives in Narita. ア}{(\hspace{100pt})}
 \item A: Where does this bus go?\\
B: \alt<4->{It goes to Chiba Station. エ}{(\hspace{100pt})}
 \item A: Where does Tom study?\\
B: \alt<5->{He studies in the library. イ}{(\hspace{100pt})}
 \item A: Where is Mary now?\\
B: \alt<6->{She's in the kitchen now. オ}{(\hspace{100pt})}
\end{enumerate}  
 \end{column}
\begin{column}{.475\textwidth}
\begin{tcolorbox}
ア She lives in Narita.\\
イ He studies in the library.\\
ウ He's from the USA.\\
エ It goes to Chiba Station.\\
オ She's in the kitchen now.
\end{tcolorbox} 

\hfill{\scriptsize the USA: the \underLine{U}nited \underLine{S}tates of \underLine{A}merica}

\end{column}
\end{columns}

\mbox{}\hfill{\tiny 0336}\,{\scriptsize \myaudio{./audio/015_where_04.mp3}}

\end{frame}
%%%%%%%%%%%%%%%%%%%%%%
\section{聞いてみよう、読んでみよう{\tiny 0056}\,{\scriptsize \myaudio{./audio/015_where_reading.mp3}}}
%%%%%%%%%%%%%%%%%%%%%%
\begin{frame}[plain]
 
\includegraphics[width=1.01\textwidth]{./images/nanobanana-output/015_where_reading.png}

\vspace{-15pt}

\hfill{\tiny 0056}\,{\scriptsize \myaudio{./audio/015_where_reading.mp3}}

\end{frame}
%%%%%%%%%%%%%%%%%%%
%%%%%%%%%%%%%%%%%%%%%%
\begin{frame}[plain,t]{Exercises}

\begin{tcolorbox}[colframe=ForestGreen,
  colback=ForestGreen!10!white,
  colbacktitle=ForestGreen!40!white,
  coltitle=black, %fonttitle=\bfseries,
before upper={\setlength{\parindent}{1.25em}},
 title=Ken's Busy Morning\mbox{}\hfill{\tiny 0056}\,{\scriptsize \myaudio{./audio/015_where_reading.mp3}}
]
It is Sunday morning.
Ken is very busy.
He cannot find his things.
He asks his mother, ``Where is my red cap?''
She says, ``It is on the chair.''
Next, Ken asks his father, ``Where are my shoes?''
His father says, ``They are by the door.''
Ken has a small dog named Pochi.
``Where does Pochi sleep?''
Ken asks. ``He sleeps under the table,'' says his sister.
``Where do you go now?'' asks his mother.
``I go to the park with Pochi!'' says Ken.

\end{tcolorbox}

\visible<2->{\small 次の各文が本文の内容とあっていればT,そうでなければFと答えましょう}
\vspace{-5pt}
\begin{enumerate}\setlength{\itemsep}{-2pt}
 \item<2-> The cap is under the table.\hfill\visible<3->{F}
 \item<2-> The shoes are by the door.\hfill\visible<4->{T}
 \item<2-> Pochi sleeps on the chair.\hfill\visible<5->{F}
 \item<2-> Ken goes to the park with his dog.\hfill\visible<6->{T}
\end{enumerate}

\end{frame}
%%%%%%%%%%%%%%%%%%%%%%%%%%%%%%%%%
%%%%%%%%%%%%%%%%%%%%%%%%%%%%%%%%%%%%%
\begin{frame}[plain]{ask}
\large
ask \textipa{/\'\ae sk/}\hspace{20pt}
\visible<2->{三単現はasks \textipa{/\'\ae sks/}}

\begin{enumerate}
 \item<3-> The mother asks, ``Can you help me?''\hfill{}S$+$V,\,\,``\,\fbox{    ?}\,''
 \item<4-> ``Can you help me?'' the mother asks.\hfill{}``\,\fbox{    ?}\,''\,\,\,S$+$V\,.
 \item<5-> ``Can you help me?'' asks the mother.\hfill{}``\,\fbox{    ?}\,''\,\,\,V$+$S\,.
\end{enumerate}

\vspace{20pt}

\small

\hfill{}\visible<3->{``\,\fbox{    ?}\,''\,$=$ 引用した実際の疑問文}

\hfill{}\visible<3->{前後の `` と '' のことを\kenten{引用符}(quotation mark)といいます}
\end{frame}
%%%%%%%%%%%%%%%%%%%%%
\begin{frame}[plain]{say}
\large

say \textipa{/s\'eI/}\hspace{20pt}\visible<2->{三単現はsays \textipa{/s\'ez/}}

 \begin{enumerate}
  \item<3-> The boy says, ``I'm hungry.''\hfill{}S$+$V,\,\,``\,\fbox{    .}\,''
  \item<4-> ``I'm hungry,'' the boy says.\hfill{}``\,\fbox{    ,}\,''\,\,\,S$+$V\,.
  \item<5-> ``I'm hungry,'' says the boy.\hfill{}``\,\fbox{    ,}\,''\,\,\,V$+$S\,.
 \end{enumerate}
\vspace{20pt}

\small

\hfill{}\visible<3->{``\,\fbox{    ?}\,''\,$=$ 引用した実際の文}

\hfill{}\visible<3->{前後の `` と '' のことを\kenten{引用符}(quotation mark)といいます}
\end{frame}
%%%%%%%%%%%%%%%%%%%%%%%%%%%%%%%%%
\begin{frame}[plain]{大意}
 
\begin{tcolorbox}[title=ケンの忙しい朝]
  日曜日の朝です。
ケンはとても忙しくしています。
彼は自分のものを見つけられません。 
彼は「わたしの赤い帽子はどこですか?」とお母さんにたずねます。
彼女は「それはイスの上にありますよ」と言います。
次に、ケンはお父さんに「わたしの靴はどこですか?」とたずねます。
お父さんは「ドアのそばにあるよ」と言います。
ケンはポチという名前の小さな犬を飼っています。
「ポチはどこで眠りますか?」とケンはたずねます。
「テーブルの下で眠っているよ」とお姉さんが言います。
「あなたは今からどこへ行くの?」とお母さんがたずねます。
「ポチと一緒に公園へ行くんだ!」とケンは言います。
\end{tcolorbox}
\end{frame}
%%%%%%%%%%%%%%%%%%%%%%%%%%%%%%%%
%%%%%%%%%%%%%%%%%%%%%%%
\section{疑問詞whereのまとめ}
%%%%%%%%%%%%%%%%%%%%%%%%%%%%%%%%
\begin{frame}[plain]{まとめ}
 \begin{block}{Where ~?\hspace{10pt}\textipa{/w\'e\textrhookschwa /}}
\begin{description}[    ]
 \item[be動詞]<2-> %Where is 単数形?\,\,\,\,\,/\,\,\,\,\,Where are 複数形?
$\HighlightFormula{\text{Where} \left\{
\begin{array}{l}
 \text{is / are}\\
% \text{was / were}\\
\end{array}
\right\} + \text{S \ldots\,\,\, ?}}$

\begin{enumerate}
 \item<3-> Where is your car?
 \item<4-> Where are my glasses?
\mbox{}
\end{enumerate}

 \item[一般動詞]<5-> $\HighlightFormula{\text{Where}\,\,\,\left\{ \begin{tabular}{l}
	  do / does\\
%	  did\\
%	  will\\
	 \end{tabular}\right\}%
\,\,+ \text{S} + \text{原形 \ldots\,\,\,?}}$

\begin{enumerate}\setcounter{enumi}{2}
       \item<6-> Where do you eat lunch?
       \item<7-> Where does she live?
%       \item<8-> Where did they play baseball?
%       \item<9-> Where will we meet?
      \end{enumerate}
\end{description}
  
 \end{block}

\mbox{}\hfill{\tiny 0236}\,{\scriptsize \myaudio{./audio/015_where_05.mp3}}
\end{frame}
%%%%%%%%%%%%%%%%%%%%%%%%%%%%%%%
\begin{frame}[plain]{}
\includegraphics[width=1.05\textwidth]{./infographic/015_where_infographic.png}
\end{frame}
%%%%%%%%%%%%%%%%%%%%%%%%%%%%%%%
%%%%%%%%%%%%%%%%%%%
\begin{frame}[plain]

 {\tiny audio\_overview 0919}\,{\scriptsize \myaudio{./audio/overview/015_where_audio_overview.m4a}}
\end{frame}
%%%%%%%%%%%%%%%%%%%%%%%%%%
\end{document}
