\documentclass[aspectratio=169]{beamer}
\usepackage[no-math,deluxe,haranoaji]{luatexja-preset}
\renewcommand{\kanjifamilydefault}{\gtdefault}
\renewcommand{\emph}[1]{{\upshape\bfseries #1}}
\usetheme{metropolis}
\metroset{block=fill}
\setbeamertemplate{navigation symbols}{}
\usecolortheme[rgb={0.7,0.2,0.2}]{structure}
%%%%%%%%%%%%%%%%%%%%%%%%%%%
\usepackage{media9}
%%%%%%%%%%%%%%%%%%%%%%%%%%%
%% さまざまなアイコン
%%%%%%%%%%%%%%%%%%%%%%%%%%%
\usepackage{fontawesome}
\usepackage{figchild}
\usepackage{twemojis}
\usepackage{utfsym}
\usepackage{bclogo}
\usepackage{marvosym}
\usepackage{fontmfizz}
%%%%%%%%%%%%%%%%%%%%%%%%%%%
\usepackage{tikz}
\usetikzlibrary{backgrounds}
\usepackage{tcolorbox}
\usepackage{tikzpeople}
\usepackage{xcolor}
\usepackage{amsmath}
%%%%%%%%%%%%%%%%%%%%%%%%%%%
%% 場合分け
\usepackage{cases}
%%%%%%%%%%%%%%%%%%%%%%%%%%%
% \myAnch{<名前>}{<色>}{<テキスト>}
% 指定のテキストを指定の色の四角枠で囲み, 指定の名前をもつTikZの
% ノードとして出力する. 図には remeber picture 属性を付けている
% ので外部から参照可能である.
\newcommand*{\myAnch}[3]{%
  \tikz[remember picture,baseline=(#1.base)]
    \node[draw,rectangle,#2] (#1) {\normalcolor #3};
}
%%%%%%%%%%%%%%%%%%%%%%%%%%%%
%% 音声リンク表示
\newcommand{\myaudio}[1]{\href{#1}{\faVolumeUp}}
%%%%%%%%%%%%%%%%%%%%%%%%%%%
% \myEmph コマンドの定義
%\newcommand{\myEmph}[3]{%
%    \textbf<#1>{\color<#1>{#2}{#3}}%
%}
\usepackage{xparse} % xparseパッケージの読み込み
\NewDocumentCommand{\myEmph}{O{} m m}{%
    \def\argOne{#1}%
    \ifx\argOne\empty
        \textbf{\color{#2}{#3}}% オプション引数が省略された場合
    \else
        \textbf<#1>{\color<#1>{#2}{#3}}% オプション引数が指定された場合
    \fi
}
%%%%%%%%%%%%%%%%%%%%%%%%%%%
%% 文末の上昇イントネーション記号 \myRisingPitch
%% 通常のイントネーション \myDownwardPitch
%% https://note.com/dan_oyama/n/n8be58e8797b2
%%%%%%%%%%%%%%%%%%%%%%%%%%%
\newcommand{\myRisingPitch}{
\begin{tikzpicture}[scale=0.3,baseline=0.3]
\draw[->,>=stealth] (0,0) to[bend right=45] (1,1);
\end{tikzpicture}
}
\newcommand{\myDownwardPitch}{
\begin{tikzpicture}[scale=0.3,baseline=0.3]
\draw[->,>=stealth] (0,1) to[bend left=45] (1,0);
\end{tikzpicture}
}
%%%%%%%%%%%%%%%%%%%%%%%%%%%
\title{English is fun.\,\,{}--- Are you hungry? ---}
\author{}
\institute[]{}
\date[]

%%%%%%%%%%%%%%%%%%%%%%%%%%%%
%% TEXT
%%%%%%%%%%%%%%%%%%%%%%%%%%%%
\begin{document}
\begin{frame}[plain]
  \titlepage
\end{frame}

\section*{授業の流れ}
\begin{frame}[plain]
  \frametitle{授業の流れ}
  \tableofcontents
\end{frame}


\section{be動詞の疑問文}

\subsection{わたしは〜ですか}
\begin{frame}[plain]{わたしは〜ですか?}
 \Large

疑問文のつくり方($\text{主語} = \text{I}$)\pause
\vspace{20pt}

\myAnch{s-1}{orange}{I} \myAnch{be-1}{olive}{am} late for the movie.
\vspace{35pt}\pause

\myAnch{be-2}{olive}{Am} \myAnch{s-2}{orange}{I} late for the movie\myAnch{question}{orange}{?}
\pause
\begin{tikzpicture}[remember picture, overlay]
 \draw[thick, orange, ->] (s-1.south) to[out=-90, in=90] (s-2.north); \pause
 \draw[thick, olive, ->] (be-1.south) to[out=-90, in=90] (be-2.north);
\pause
\node at (-4.5,-1) {逆にするだけ};
\pause
\node at (0.5,0.25) {\scalebox{2}{\myRisingPitch}};
\pause
\node at (3,-1) {?と最後のイントネーションに注意};
\end{tikzpicture}

\end{frame}


\subsection{あなたは〜ですか}
\begin{frame}[plain]{あなたは〜ですか?}
 \Large

疑問文のつくり方($\text{主語} = \text{You}$)\pause

\vspace{20pt}

\myAnch{s-1}{orange}{You} \myAnch{be-1}{olive}{are} happy.
\vspace{35pt}\pause

\myAnch{be-2}{olive}{Are} \myAnch{s-2}{orange}{you} happy\myAnch{question}{orange}{?}
\pause
\begin{tikzpicture}[remember picture, overlay]
 \draw[thick, orange, ->] (s-1.south) to[out=-90, in=90] (s-2.north); \pause
 \draw[thick, olive, ->] (be-1.south) to[out=-90, in=90] (be-2.north);
\pause
\node at (-4,-1) {逆にするだけ};
\pause
\node at (0.5,0.25) {\scalebox{2}{\myRisingPitch}};
\pause
\node at (3,-1) {?と最後のイントネーションに注意};
\end{tikzpicture}

\end{frame}



\subsection{彼女は〜ですか}
\begin{frame}<1-10>[plain]{Xは〜ですか?}
 \Large

疑問文のつくり方($\text{主語} = \text{\temporal<9>{She}{He}{Our teacher}}$)\pause

\vspace{20pt}

\myAnch{s-1}{orange}{\temporal<9>{She}{He}{Our teacher}} \myAnch{be-1}{olive}{is} from Australia.
\vspace{35pt}\pause

\myAnch{be-2}{olive}{Is} \myAnch{s-2}{orange}{\temporal<9>{she}{he}{our teacher}} from Australia\myAnch{question}{orange}{?}
\pause
\begin{tikzpicture}[remember picture, overlay]
 \draw[thick, orange, ->] (s-1.south) to[out=-90, in=90] (s-2.north); \pause
 \draw[thick, olive, ->] (be-1.south) to[out=-90, in=90] (be-2.north);
\pause
\onslide<6-8>{\node at (-4.5,-1) {逆にするだけ};}
\pause
\node at (0.5,0.25) {\scalebox{2}{\myRisingPitch}};
\pause
\onslide<8>{\node at (3,-1) {?と最後のイントネーションに注意};}
\end{tikzpicture}
\end{frame}

\begin{frame}<1-10>[plain]\frametitle{Exercises}

つぎの文を疑問文にしましょう。

 \begin{enumerate}
  \item<1-> I am late for the meeting.
        \onslide<5->{$\longrightarrow$ Am I late for the meeting?}
  \item<1-> You are hungry.
        \onslide<6->{$\longrightarrow$ Are you hungry?}
 \item<1-> They are happy.
        \onslide<7->{$\longrightarrow$ Are they happy?}
  \item<1-> She is in the kitchen.
        \onslide<8->{$\longrightarrow$ Is she in the kitchen?}
  \item<1-> Our teacher is busy.
        \onslide<9->{$\longrightarrow$ Is our teacher busy? }
 \end{enumerate}

\begin{exampleblock}<2->{Topics for Today}
be動詞の疑問文のつくり方
\begin{itemize}
 \item<3->  主語とbe動詞を逆の順番にする
 \item<4-> 最後に`?'をつける
\end{itemize}
      \end{exampleblock}

% Embed the sound file
\onslide<10>{%
\myaudio{./audio/008_question_be_01.mp3}\,\,{}Listen carefully.(注意して聞いてください)
}

\end{frame}


\subsection{疑問文への答え方}
 \begin{frame}[plain]{Are you 〜 ? と聞かれたら}
 \Large

疑問文にたいする答え方
\vspace{10pt}

\pause

Are you from New York?

\pause

$\left\{\begin{array}{l}
         \text{Yes, I am.}\\\pause
         \text{No, I am not.}
        \end{array}\right.$

\pause

\mbox{}\hfill{}{\small No, I am not. はNo, I'm not.ともいいます}
\end{frame}

\begin{frame}[plain]{Is he  〜 ? / Is she 〜 ? と聞かれたら}
 \Large

疑問文にたいする答え方
\vspace{10pt}

\pause

\begin{columns}
\begin{column}{.475\textwidth}
Is he a baseball player?

\pause

\mbox{}\hspace{40pt}$\left\{\begin{array}{l}
         \text{Yes, he is.}\\\pause
         \text{No, he is not.}
        \end{array}\right.$

\pause

\mbox{}\hfill{}{\footnotesize No, he is not. はNo, he isn't.ともいいます}
\end{column}
\pause
\begin{column}{.475\textwidth}
Is she a good singer?

\pause

\mbox{}\hspace{40pt}$\left\{\begin{array}{l}
         \text{Yes, she is.}\\\pause
         \text{No, she is not.}
        \end{array}\right.$

\pause

\mbox{}\hfill{}{\footnotesize No, she is not. はNo, she isn't.ともいいます}


\end{column}
\end{columns}

\end{frame}

\begin{frame}[plain]{Is John  〜 ? / Is Emily 〜 ? と聞かれたら}
 \Large

疑問文にたいする答え方
\vspace{10pt}

\pause

\begin{columns}
\begin{column}{.49\textwidth}
Is \myAnch{john}{orange}{John} a teacher?

\pause

\vspace{10pt}

\mbox{}\hfill$\left\{\begin{tabular}{l}
         Yes, \myAnch{he1}{orange}{he} is.\\\pause
         No, \myAnch{he2}{orange}{he} is not($=\text{isn't}$).
        \end{tabular}\right.$

\pause

%\mbox{}\hfill{}{\footnotesize }

\begin{tikzpicture}[remember picture,overlay]
 \draw[->,thick,orange] (john.south) to[out=-30,in=160] (he1.north west);
 \draw[->,thick,orange] (john.south) to[out=-110, in=155] (he2.west);
\end{tikzpicture}

\end{column}
\pause
\begin{column}{.49\textwidth}
Is \myAnch{emily}{orange}{Emily} from Sydne?

\pause

\vspace{10pt}

\mbox{}\hfill$\left\{\begin{tabular}{l}
         Yes, \myAnch{she1}{orange}{she} is.\\\pause
         No, \myAnch{she2}{orange}{she} is not($=\text{isn't}$).
        \end{tabular}\right.$

\begin{tikzpicture}[remember picture,overlay]
 \draw[->,thick,orange] (emily.south) to[out=-30,in=160] (she1.north west);
 \draw[->,thick,orange] (emily.south) to[out=-165, in=165] (she2.west);
\end{tikzpicture}

\end{column}
\end{columns}
\end{frame}

\begin{frame}[plain]{Is this  〜 ? / Is that 〜 ? と聞かれたら}
 \Large

疑問文にたいする答え方
\vspace{10pt}

\pause

\begin{columns}
\begin{column}{.49\textwidth}
Is \myAnch{this}{orange}{this} your book?

\pause

\vspace{10pt}

\mbox{}\hfill$\left\{\begin{tabular}{l}
         Yes, \myAnch{it1}{orange}{it} is.\\\pause
         No, \myAnch{it2}{orange}{it} is not($=\text{isn't}$).
        \end{tabular}\right.$

\pause

%\mbox{}\hfill{}{\footnotesize }

\begin{tikzpicture}[remember picture,overlay]
 \draw[->,thick,orange] (this.south) to[out=-30,in=160] (it1.north west);
 \draw[->,thick,orange] (this.south) to[out=-110, in=165] (it2.west);
\end{tikzpicture}

\end{column}
\pause
\begin{column}{.49\textwidth}
Is \myAnch{that}{orange}{that} your house?

\pause

\vspace{10pt}

\mbox{}\hfill$\left\{\begin{tabular}{l}
         Yes, \myAnch{it3}{orange}{it} is.\\\pause
         No, \myAnch{it4}{orange}{it} is not($=\text{isn't}$).
        \end{tabular}\right.$

\begin{tikzpicture}[remember picture,overlay]
 \draw[->,thick,orange] (that.south) to[out=-30,in=160] (it3.north west);
 \draw[->,thick,orange] (that.south) to[out=-165, in=165] (it4.west);
\end{tikzpicture}

\end{column}
\end{columns}
\end{frame}

\end{document}

