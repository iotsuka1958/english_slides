\documentclass[aspectratio=169]{beamer}
\usepackage[no-math,deluxe,haranoaji]{luatexja-preset}
\renewcommand{\kanjifamilydefault}{\gtdefault}
\renewcommand{\emph}[1]{{\upshape\bfseries #1}}
\usetheme{metropolis}
\metroset{block=fill}
\setbeamertemplate{navigation symbols}{}
\usecolortheme[rgb={0.7,0.2,0.2}]{structure}
%%%%%%%%%%%%%%%%%%%%%%%%%%%
\usepackage{media9}
%%%%%%%%%%%%%%%%%%%%%%%%%%%
%% さまざまなアイコン
%%%%%%%%%%%%%%%%%%%%%%%%%%%
\usepackage{fontawesome}
\usepackage{figchild}
\usepackage{twemojis}
\usepackage{utfsym}
\usepackage{bclogo}
\usepackage{marvosym}
\usepackage{fontmfizz}
%%%%%%%%%%%%%%%%%%%%%%%%%%%
\usepackage{tikz}
\usetikzlibrary{backgrounds}
\usepackage{tcolorbox}
\usepackage{tikzpeople}
\usepackage{xcolor}
\usepackage{amsmath}
%%%%%%%%%%%%%%%%%%%%%%%%%%%
%% 場合分け
\usepackage{cases}
%%%%%%%%%%%%%%%%%%%%%%%%%%%
% \myAnch{<名前>}{<色>}{<テキスト>}
% 指定のテキストを指定の色の四角枠で囲み, 指定の名前をもつTikZの
% ノードとして出力する. 図には remeber picture 属性を付けている
% ので外部から参照可能である.
\newcommand*{\myAnch}[3]{%
  \tikz[remember picture,baseline=(#1.base)]
    \node[draw,rectangle,#2] (#1) {\normalcolor #3};
}
%%%%%%%%%%%%%%%%%%%%%%%%%%%%
%% 音声リンク表示
\newcommand{\myaudio}[1]{\href{#1}{\faVolumeUp}}
%%%%%%%%%%%%%%%%%%%%%%%%%%%
% \myEmph コマンドの定義
%\newcommand{\myEmph}[3]{%
%    \textbf<#1>{\color<#1>{#2}{#3}}%
%}
\usepackage{xparse} % xparseパッケージの読み込み
\NewDocumentCommand{\myEmph}{O{} m m}{%
    \def\argOne{#1}%
    \ifx\argOne\empty
        \textbf{\color{#2}{#3}}% オプション引数が省略された場合
    \else
        \textbf<#1>{\color<#1>{#2}{#3}}% オプション引数が指定された場合
    \fi
}
%%%%%%%%%%%%%%%%%%%%%%%%%%%
%% 文末の上昇イントネーション記号 \myRisingPitch
%% 通常のイントネーション \myDownwardPitch
%% https://note.com/dan_oyama/n/n8be58e8797b2
%%%%%%%%%%%%%%%%%%%%%%%%%%%
\newcommand{\myRisingPitch}{
\begin{tikzpicture}[scale=0.3,baseline=0.3]
\draw[->,>=stealth] (0,0) to[bend right=45] (1,1);
\end{tikzpicture}
}
\newcommand{\myDownwardPitch}{
\begin{tikzpicture}[scale=0.3,baseline=0.3]
\draw[->,>=stealth] (0,1) to[bend left=45] (1,0);
\end{tikzpicture}
}
%%%%%%%%%%%%%%%%%%%%%%%%%%%
\title{English is fun.\,\,{}--- Are you hungry? ---}
\author{}
\institute[]{}
\date[]

%%%%%%%%%%%%%%%%%%%%%%%%%%%%
%% TEXT
%%%%%%%%%%%%%%%%%%%%%%%%%%%%
\begin{document}
\begin{frame}[plain]
  \titlepage
\end{frame}

\section*{授業の流れ}
\begin{frame}[plain]
  \frametitle{授業の流れ}
  \tableofcontents
\end{frame}


\section{be動詞の疑問文}
\subsection{あなたは〜ですか}
\begin{frame}[plain]{あなたは〜ですか?}
 \Large

疑問文のつくり方($\text{主語} = \text{You}
\vspace{20pt}

\myAnch{s-1}{orange}{You} \myAnch{be-1}{olive}{are} happy.
\vspace{40pt}\pause

\myAnch{be-2}{olive}{Are} \myAnch{s-2}{orange}{you} happy\myAnch{question}{orange}{?}
\pause
\begin{tikzpicture}[remember picture, overlay]
 \draw[thick, orange, ->] (s-1.south) to[out=-90, in=90] (s-2.north); \pause
 \draw[thick, olive, ->] (be-1.south) to[out=-90, in=90] (be-2.north);
\pause
\node at (-4,-1) {逆にするだけ};
\pause
\node at (0.5,0.25) {\scalebox{2}{\myRisingPitch}};
\pause
\node at (3.25,-1) {語尾のイントネーションに注意};
\end{tikzpicture}

\end{frame}


\subsection{わたしは〜ですか}
\begin{frame}[plain]{わたしは〜ですか?}
 \Large

疑問文のつくり方($\text{主語} = \text{I
\vspace{20pt}

\myAnch{s-1}{orange}{I} \myAnch{be-1}{olive}{am} late for the movie.
\vspace{50pt}\pause

\myAnch{be-2}{olive}{Am} \myAnch{s-2}{orange}{I} late for the movie\myAnch{question}{orange}{?}
\pause
\begin{tikzpicture}[remember picture, overlay]
 \draw[thick, orange, ->] (s-1.south) to[out=-90, in=90] (s-2.north); \pause
 \draw[thick, olive, ->] (be-1.south) to[out=-90, in=90] (be-2.north);
\pause
\node at (-4.5,-1) {逆にするだけ};
\pause
\node at (0.5,0.25) {\scalebox{2}{\myRisingPitch}};
\pause
\node at (3.25,-1) {語尾のイントネーションに注意};
\end{tikzpicture}

\end{frame}
\end{document}

