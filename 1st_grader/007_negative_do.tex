\documentclass[aspectratio=169]{beamer}
\usepackage[no-math,deluxe,haranoaji]{luatexja-preset}
\renewcommand{\kanjifamilydefault}{\gtdefault}
\renewcommand{\emph}[1]{{\upshape\bfseries #1}}
\usetheme{metropolis}
\metroset{block=fill}
\setbeamertemplate{navigation symbols}{}
\usecolortheme[rgb={0.7,0.2,0.2}]{structure}
%%%%%%%%%%%%%%%%%%%%%%%%%%%
\usepackage{media9}
%%%%%%%%%%%%%%%%%%%%%%%%%%%
%% さまざまなアイコン
%%%%%%%%%%%%%%%%%%%%%%%%%%%
\usepackage{fontawesome}
\usepackage{figchild}
\usepackage{twemojis}
\usepackage{utfsym}
\usepackage{bclogo}
\usepackage{marvosym}
\usepackage{fontmfizz}
%%%%%%%%%%%%%%%%%%%%%%%%%%%
\usepackage{tikz}
\usetikzlibrary{backgrounds}
\usepackage{tcolorbox}
\usepackage{tikzpeople}
\usepackage{xcolor}
\usepackage{amsmath}
%%%%%%%%%%%%%%%%%%%%%%%%%%%
%% 場合分け
\usepackage{cases}
%%%%%%%%%%%%%%%%%%%%%%%%%%%
% \myAnch{<名前>}{<色>}{<テキスト>}
% 指定のテキストを指定の色の四角枠で囲み, 指定の名前をもつTikZの
% ノードとして出力する. 図には remeber picture 属性を付けている
% ので外部から参照可能である.
\newcommand*{\myAnch}[3]{%
  \tikz[remember picture,baseline=(#1.base)]
    \node[draw,rectangle,#2] (#1) {\normalcolor #3};
}
%%%%%%%%%%%%%%%%%%%%%%%%%%%%
%% 音声リンク表示
\newcommand{\myaudio}[1]{\href{#1}{\faVolumeUp}}
%%%%%%%%%%%%%%%%%%%%%%%%%%%
% \myEmph コマンドの定義
%\newcommand{\myEmph}[3]{%
%    \textbf<#1>{\color<#1>{#2}{#3}}%
%}
\usepackage{xparse} % xparseパッケージの読み込み
\NewDocumentCommand{\myEmph}{O{} m m}{%
    \def\argOne{#1}%
    \ifx\argOne\empty
        \textbf{\color{#2}{#3}}% オプション引数が省略された場合
    \else
        \textbf<#1>{\color<#1>{#2}{#3}}% オプション引数が指定された場合
    \fi
}
%%%%%%%%%%%%%%%%%%%%%%%%%%%
\title{English is fun.\,\,{}--- I do not like coffee. ---}
\author{}
\institute[]{}
\date[]

%%%%%%%%%%%%%%%%%%%%%%%%%%%%
%% TEXT
%%%%%%%%%%%%%%%%%%%%%%%%%%%%
\begin{document}
\begin{frame}[plain]
  \titlepage
\end{frame}

\section*{授業の流れ}
\begin{frame}[plain]
  \frametitle{授業の流れ}
  \tableofcontents
\end{frame}

\section{一般動詞の否定}
\subsection{I do not like coffee.}
\begin{frame}<1-11>[plain]\frametitle{do not / don't}

\begin{tabular}{lll}
 \onslide<1->{\scalebox{4}{\twemoji{smiling face with
heart-eyes}\,\,\,{\tiny\mfCoffeeBean}\mfJavaBold{}}}&\onslide<2->{I like coffee.}& \onslide<4->{わたしはコーヒーが好きだ。}\\
\onslide<7->{\scalebox{4}{\twemoji{person gesturing NO}\,\,\,{\tiny\mfCoffeeBean}\mfJavaBold{}}}&\onslide<6->{I \myAnch{long}{orange}{do not} like coffee.}&  \onslide<5->{わたしはコーヒーが好きではない。}\\[20pt]
&\onslide<11->{I \myAnch{short}{orange}{don't} like coffee.}
\end{tabular}

\vspace{15pt}
\onslide<3->{\small \mbox{}\hfill{}coffee: コーヒー}

\onslide<8->{%
\begin{exampleblock}{Topics for Today}
\begin{itemize}
 \item<9->  一般動詞の否定は\textcolor{orange}{$\text{do not} + \text{動詞}$}\pause
 \item<10->  do notは縮めて \textcolor{orange}{$\text{don't} + \text{動詞}$}ということもあります
\end{itemize}
      \end{exampleblock}
}

\onslide<11>{
\begin{tikzpicture}[remember picture, overlay]
 \draw[->, thick, orange] (long) to (short);
\end{tikzpicture}
}
\end{frame}



\begin{frame}<1-12>[plain]\frametitle{be動詞との比較}
\begin{columns}[T]
 \begin{column}{.48\textwidth}
\begin{itemize}
 \item I am \only<2->{\textcolor{orange}{not}} hungry.
 \item You are \only<3->{\textcolor{orange}{not}} hungry.
 \item He is \only<4->{\textcolor{orange}{not}}  hungry.
\end{itemize}
 \end{column}
\begin{column}{.48\textwidth}
\onslide<5->{\begin{tabular}[t]{l}
	      be動詞\\
              {\small $\left(\begin{array}{l}
                      \text{am}\\
                      \text{are}\\
                      \text{is}
                      \end{array}\right)$}
            \end{tabular} $+\,\,\,\,\, \text{not}$
}
\end{column}
\end{columns}

\bigskip

\bigskip

\begin{columns}[T]
 \begin{column}{.48\textwidth}
\begin{itemize}
 \item I \only<6->{\textcolor{orange}{do not}} like coffee.
 \item You \only<7->{\textcolor{orange}{do not}} like coffee.
 \item They \only<8->{\textcolor{orange}{do not}} like coffee.
\end{itemize}
 \end{column}
\begin{column}{.48\textwidth}
\mbox{}

\vspace*{15pt}
\onslide<9->{$\text{do not} + \text{動詞}$} 
\end{column}
\end{columns}

\onslide<10->{%
\begin{exampleblock}{Topics for Today}
\begin{itemize}
 \item<11->  be動詞の否定はnotを後ろにつける
 \item<12-> 一般動詞の否定は do notを前につける
\end{itemize}
      \end{exampleblock}
}
\end{frame}


\begin{frame}[plain]\frametitle{Exercises}
 
つぎの文を否定文にしてください。
\begin{enumerate}
 \item I play soccer after school.\pause{}\hspace{63pt}{\footnotesize after school: 放課後に}\\\pause
       I \textcolor{orange}{do not} play soccer after school.\\\pause
       I \textcolor{orange}{don't} play soccer after school.\pause
 \item You study math at night.\pause\hspace{68pt}{\footnotesize study: 勉強する math: 数学 at night: 夜に}\\\pause
       You \textcolor{orange}{do not} study math at night.\\\pause
       You \textcolor{orange}{don't} study math at night.\pause
 \item We eat meat at home.\pause\hspace{80pt}{\footnotesize eat: 食べる meat: 肉}\\\pause
       We \textcolor{orange}{do not} eat meat at home.\\\pause
       We \textcolor{orange}{don't} eat meat at home.\pause
 \item They like horror movies.\pause\hspace{70pt}{\footnotesize horror movie: ホラー映画}\\\pause
       They \textcolor{orange}{do not} like horror movies.\\\pause
       They \textcolor{orange}{don't} like horror movies.
\end{enumerate}




\end{frame}

\end{document}

