\documentclass[aspectratio=169,xcolor={dvipsnames,table}]{beamer}
\usepackage[no-math,deluxe,haranoaji]{luatexja-preset}
\renewcommand{\kanjifamilydefault}{\gtdefault}
\renewcommand{\emph}[1]{{\upshape\bfseries #1}}
\usetheme{metropolis}
\metroset{block=fill}
\setbeamertemplate{navigation symbols}{}
\setbeamertemplate{blocks}[rounded][shadow=false]
\usecolortheme[rgb={0.7,0.2,0.2}]{structure}
%%%%%%%%%%%%%%%%%%%%%%%%%%%
\usepackage{media9}
%%%%%%%%%%%%%%%%%%%%%%%%%%%
%% さまざまなアイコン
%%%%%%%%%%%%%%%%%%%%%%%%%%%
\usepackage{fontawesome}
%\usepackage{figchild}
\usepackage{twemojis}
\usepackage{utfsym}
\usepackage{bclogo}
\usepackage{marvosym}
\usepackage{fontmfizz}
\usepackage{manfnt}
\usepackage{tipa}
\usepackage{pxrubrica}
%%%%%%%%%%%%%%%%%%%%%%%%%%%
\usepackage{tikz}
\usetikzlibrary{backgrounds}
\usepackage{tcolorbox}
\usepackage{circledsteps}
\usepackage{tikzpeople}
\usepackage{xcolor}
\usepackage{amsmath}
%%%%%%%%%%%%%%%%%%%%%%%%%%%
%% 場合分け
\usepackage{cases}
%%%%%%%%%%%%%%%%%%%%%%%%%%%
% \myAnch{<名前>}{<色>}{<テキスト>}
% 指定のテキストを指定の色の四角枠で囲み, 指定の名前をもつTikZの
% ノードとして出力する. 図には remeber picture 属性を付けている
% ので外部から参照可能である.
\newcommand*{\myAnch}[3]{%
  \tikz[remember picture,baseline=(#1.base)]
    \node[draw,rectangle,#2] (#1) {\normalcolor #3};
}
%%%%%%%%%%%%%%%%%%%%%%%%%%%%
%% 音声リンク表示
\newcommand{\myaudio}[1]{\href{#1}{\faVolumeUp}}
%%%%%%%%%%%%%%%%%%%%%%%%%%%
% \myEmph コマンドの定義
%\newcommand{\myEmph}[3]{%
%    \textbf<#1>{\color<#1>{#2}{#3}}%
%}
\usepackage{xparse} % xparseパッケージの読み込み
\NewDocumentCommand{\myEmph}{O{} m m}{%
    \def\argOne{#1}%
    \ifx\argOne\empty
        \textbf{\color{#2}{#3}}% オプション引数が省略された場合
    \else
        \textbf<#1>{\color<#1>{#2}{#3}}% オプション引数が指定された場合
    \fi
}
%%%%%%%%%%%%%%%%%%%%%%%%%%%
\title{English is fun.}
\subtitle{I drink tea. I do not like coffee.}
\author{}
\institute[]{}
\date[]

%%%%%%%%%%%%%%%%%%%%%%%%%%%%
%% TEXT
%%%%%%%%%%%%%%%%%%%%%%%%%%%%
\begin{document}
\begin{frame}[plain]
  \titlepage
\end{frame}

\section*{授業の流れ}
\begin{frame}[plain]
  \frametitle{授業の流れ}
  \tableofcontents
\end{frame}

\section{一般動詞の否定}
\subsection{基本}
%%%%%%%%%%%%%%%%%%%%%%%%%%%%%%%%%%%%
\begin{frame}<1-11>[plain]\frametitle{do not / don't}
\begin{tabular}{lll}
 \onslide<1->{\scalebox{4}{\twemoji{smiling face with
heart-eyes}\,\,\,{\tiny\mfCoffeeBean}\mfJavaBold{}}}&\onslide<2->{1.\,\,\,I like coffee.}& \onslide<4->{{\scriptsize 私はコーヒーが好きだ。}}\\
\onslide<5->{\scalebox{4}{\twemoji{person gesturing NO}\,\,\,{\tiny\mfCoffeeBean}\mfJavaBold{}}}&\onslide<7->{2.\,\,\,I \myAnch{long}{orange}{\textbf{do not}} like coffee.}&  \onslide<6->{{\scriptsize 私はコーヒーが好きではない。}}\\[20pt]
&\onslide<8->{3.\,\,\,I \myAnch{short}{orange}{\textbf{don't}} like coffee.}
\end{tabular}

\vspace{15pt}
\onslide<3->{\small \mbox{}\hfill{}{\scriptsize coffee \textipa{/k\'O:fi/} コーヒー}}\\
\hfill{\tiny 0115}\,{\scriptsize \myaudio{audio/007_negative_do_00.mp3}}


%
\begin{block}<9->{Topics for Today}
\begin{itemize}\setbeamertemplate{items}[square]
 \item<9->  一般動詞の否定は\textcolor{Maroon}{$\text{\bfseries do not} + \text{原形}$}\hfill\textipa{/d@ n\'At/}
 \item<10->  \textbf{do not}は縮めて \textcolor{Maroon}{\bfseries don't}ということもあります(短縮形)\hfill\textipa{/d\'oUnt/}
\end{itemize}
\end{block}


\onslide<8->{
\begin{tikzpicture}[remember picture, overlay]
 \draw[->, thick, orange] (long) to (short);
\end{tikzpicture}
}
\end{frame}
%%%%%%%%%%%%%%%%%%%%%%%%%%


\begin{frame}<1-12>[plain]\frametitle{be動詞との比較}
\begin{columns}[T]
 \begin{column}{.48\textwidth}
\begin{itemize}
 \item[A--1] I am \only<2->{\textcolor{orange}{\bfseries not}} hungry.
 \item[A--2] You are \only<3->{\textcolor{orange}{\bfseries not}} hungry.
 \item[A--3] He is \only<4->{\textcolor{orange}{\bfseries not}}  hungry.
\end{itemize}
 \end{column}
\begin{column}{.48\textwidth}
\onslide<5->{\begin{tabular}[t]{l}
	      be動詞\\
              {\small $\left(\begin{array}{l}
                      \text{am}\\
                      \text{are}\\
                      \text{is}
                      \end{array}\right)$}
            \end{tabular} $+\,\,\,\,\, \text{not}$
}
\end{column}
\end{columns}

\bigskip

\bigskip

\begin{columns}[T]
 \begin{column}{.48\textwidth}
\begin{itemize}
 \item[B--1] I \only<6->{\textcolor{orange}{\bfseries do not}} like coffee.
 \item[B--2] You \only<7->{\textcolor{orange}{\bfseries do not}} like coffee.
 \item[B--3] They \only<8->{\textcolor{orange}{\bfseries do not}} like coffee.
\end{itemize}
 \end{column}
\begin{column}{.48\textwidth}
\mbox{}

\vspace*{15pt}
\onslide<9->{$\text{do not} + \text{一般動詞}$} 
\end{column}
\end{columns}

\begin{block}<10->{Topics for Today}
\begin{itemize}\setbeamertemplate{items}[square]
 \item<11->  be動詞の否定は\textbf{not}(だけ)を\kenten{後ろ}につける
 \item<12-> 一般動詞の否定は\textbf{do not}\,($=\text{\textbf{don't}}$)を\kenten{前}につける%
\hfill\textipa{/d@ n\'At/}\hspace{5pt}\textipa{/d\'oUnt/}
\end{itemize}
      \end{block}
\end{frame}
%%%%%%%%%%%%%%%%%%%%%%%%%%%%%%%%%%%
\begin{frame}[plain]\frametitle{Exercises}
 
つぎの文を否定文にしてください
\begin{enumerate}
 \item I play soccer after school.\hfill{\scriptsize after school: 放課後に}
         \begin{itemize}\setbeamertemplate{items}[circle]
	 \item<2-> I \textcolor{orange}{\bfseries do not} play soccer after school.
	  \item<3-> I \textcolor{orange}{\bfseries don't} play soccer after school.
	\end{itemize}
 \item You study math at night.\hfill{\scriptsize study: 勉強する math: 数学 at night: 夜に}
       \begin{itemize}\setbeamertemplate{items}[circle]
	\item<4-> You \textcolor{orange}{\bfseries do not} study math at night.\\
	\item<5-> You \textcolor{orange}{\bfseries don't} study math at night.
       \end{itemize}
 \item We eat meat at home.\hfill{\scriptsize eat: 食べる meat: 肉 at home: 家で}
        \begin{itemize}\setbeamertemplate{items}[circle]
 \item<6-> We \textcolor{orange}{\bfseries do not} eat meat at home.\\
 \item<7-> We \textcolor{orange}{\bfseries don't} eat meat at home.
       \end{itemize}
 \item They like horror movies.\hfill{\scriptsize horror movie: ホラー映画}
        \begin{itemize}\setbeamertemplate{items}[circle]
 \item<8-> They \textcolor{orange}{\bfseries do not} like horror movies.\\
 \item<9-> They \textcolor{orange}{\bfseries don't} like horror movies.
        \end{itemize}
\end{enumerate}

% Embed the sound file
\hfill{\tiny 0352}\,{\scriptsize \myaudio{audio/007_negative_do_01.mp3}}

\end{frame}


\subsection{主語が三人称単数のとき}
%%%%%%%%%%%%%%%%%%%%%%%%%%%%%%%%%%%%%%%%%%
\begin{frame}[plain]\frametitle{主語が3人称単数のとき}

\begin{enumerate}
 \item<1-> I live in Paris.\onslide<2->{$\longrightarrow$ I do not\only<3->{($= \text{don't}$)} live in Paris.}
 \item<4-> You live  in Paris. \onslide<5->{$\longrightarrow$ You do not\only<6->{($= \text{don't}$)} live in Paris.}
 \item<7-> They live in Paris.\onslide<8->{$\longrightarrow$ They do not\only<9->{($= \text{don't}$)} live in Paris.}
 \item<11-> He \myAnch{t1}{orange}{lives} in Paris.\onslide<12->{$\longrightarrow$ He \myAnch{t2}{orange}{\textbf{does not}\only<13->{($= \text{\textbf{doesn't}}$)} live} in Paris.\hfill{\scriptsize \textipa{/d@z n\'At/}}\only<14->{{\scriptsize $\rightarrow$\textipa{/d\'\textturnv znt/}}}}
\end{enumerate}

\vspace{20pt}
\onslide<15->{%
\mbox{}\hfill\hfill\hfill\hfill\hfill\myAnch{t3}{orange}{主語が3人称単数のとき}\hfill\mbox{}

\begin{tikzpicture}[remember picture, overlay]
 \draw[thick, orange, ->] (t3.west) to[out=180, in=-60] node[black,sloped,midway,below]{\footnotesize $\text{原形} + \text{s(3単現)}$} (t1.south);
 \draw[thick, orange, ->] (t3.west) to[out=180, in=-90] node[black, pos=0.8, below right,rotate=0]{\footnotesize does not$\text{(}= \text{doesn't)} + \text{原形}$} (t2.south);
\end{tikzpicture}
}

\normalsize
%
\begin{block}<10->{Topics for Today}
\begin{itemize}\setbeamertemplate{items}[square]
 \item<10->  一般動詞の否定は\hspace{50pt}\textcolor{Maroon}{\textbf{do not}($=\text{\textbf{don't})} + \text{原形}$}
 \item<16->  主語が3人称単数のときは\hspace{8pt}\textcolor{Maroon}{\textbf{does not}($=\text{\textbf{doesn't})} + \text{原形}$}\\
\hfill{}\visible<17->{{\scriptsize \dbend\,\,{}*He does not live\Circled[fill color = white]{s} in Paris.}}\\
\hfill{}{\tiny 0110}\,{\scriptsize\myaudio{audio/007_negative_do_02.mp3}}
\end{itemize}
      \end{block}
\end{frame}
%%%%%%%%%%%%%%%%%%%%%%%
\begin{frame}[plain]\frametitle{Exercises}
カッコ内から適当なものを選びましょう%

\begin{enumerate}
 \item I ( \alt<2->{\Circled[outer color=BurntOrange]{am not}}{am not} / do not / not ) a baseball fan.\hfill{\scriptsize fan \textipa{/f\'\ae n/} ファン}
 \item He ( \alt<3->{\Circled[outer color=BurntOrange]{is not}}{is not}  / do not  / does not ) from New Zealand.\\
\hfill{\scriptsize New Zealand \textipa{/n\`(j)u:z\'\i:l@nd/}  ニュージーランド}
 \item They ( are not / \alt<4->{\Circled[outer color=BurntOrange]{do not}}{do not}  / not ) speak French.\hfill{\scriptsize French \textipa{/fr\'entS/} フランス語}
 \item I ( am not / \alt<5->{\Circled[outer color=BurntOrange]{do not}}{do not} / not ) have a dog.
 \item Sue ( is not / do not / \alt<6->{\Circled[outer color=BurntOrange]{does not}}{does not}  ) play tennis on Sundays.\hfill{\scriptsize on Sundays:} {\tiny 日曜日に}
\end{enumerate}

\hfill{\tiny 0211}\,{\scriptsize\myaudio{audio/007_negative_do_03.mp3}}

%\vspace{-100pt}

%\begin{tikzpicture}
%\draw[transparent,white] (0,0) grid[step=0.5] (5,5);
%\draw[orange,thick] (1.2,4.4) rectangle (2.5, 4.9);
%\draw[orange,thick] (1.5,3.8) rectangle (2.6, 4.3);
%\draw[orange,thick] (3.5,3.1) rectangle (4.7, 3.6);
%\draw[orange,thick] (2.8,2.4) rectangle (4.1, 2.9);
%\draw[orange,thick] (5.1,1.8) rectangle (6.7, 2.3);
%\end{tikzpicture}

\end{frame}
%%%%%%%%%%%%%%%%%%%%%
\section{一般動詞の否定(まとめ)}
%%%%%%%%%%%%%%%%%%%%%%%}
\begin{frame}[plain]{この単元で学んだこと}
\small
\begin{block}<1->{一般動詞の否定}
\begin{itemize}\setbeamertemplate{items}[square]
 \item  一般動詞の否定は\hspace{50pt}\textcolor{Maroon}{\textbf{do not}($=\text{\textbf{don't})} + \text{原形}$}%
\hfill{\scriptsize I \textbf{do not} like coffee.}

 \item  主語が3人称単数のときは\hspace{12pt}\textcolor{Maroon}{\textbf{does not}($=\text{\textbf{doesn't})} + \text{原形}$}%
\hfill{\scriptsize He \textbf{does not} live i Paris.}\\
\hfill{}{\scriptsize \dbend\,\,{}*He \textbf{does not} live\Circled[fill color = white]{\textbf{s}} in Paris.}
\end{itemize}
      \end{block}

\begin{block}<2->{be動詞とのちがい}
\begin{itemize}\setbeamertemplate{items}[square]
 \item  be動詞の否定は\textbf{not}(だけ)を\kenten{後ろ}につける
 \item 一般動詞の否定は\textbf{do not}\,($=\text{\textbf{don't}}$) / \textbf{does not}\,($=\text{\textbf{doesn't}}$)を\kenten{前}につける%
\end{itemize}
      \end{block}

\begin{block}<3->{Pronunciation}
 \begin{itemize}\setbeamertemplate{items}[square]
  \item \textbf{do not} \textipa{/d@ n\'At/} \hspace{40pt} \textbf{don't} \textipa{/d\'oUnt/}
  \item \textbf{does not} \textipa{/d@z n\'At/} \hspace{30pt} \textbf{doesn't} \textipa{/d\'2znt/}
 \end{itemize}
\hfill{\tiny audio overview 2000}\,{\scriptsize\myaudio{audio/overview/007_negative_do_audio_overview.mp4}}

\end{block}
\end{frame}
%%%%%%%%%%%%%%%%%%%%%%%%%%%
\end{document}

