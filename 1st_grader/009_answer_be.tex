\documentclass[aspectratio=169]{beamer}
\usepackage[no-math,deluxe,haranoaji]{luatexja-preset}
\renewcommand{\kanjifamilydefault}{\gtdefault}
\renewcommand{\emph}[1]{{\upshape\bfseries #1}}
\usetheme{metropolis}
\metroset{block=fill}
\setbeamertemplate{navigation symbols}{}
\usecolortheme[rgb={0.7,0.2,0.2}]{structure}
%%%%%%%%%%%%%%%%%%%%%%%%%%%
\usepackage{media9}
%%%%%%%%%%%%%%%%%%%%%%%%%%%
%% さまざまなアイコン
%%%%%%%%%%%%%%%%%%%%%%%%%%%
\usepackage{fontawesome}
\usepackage{figchild}
\usepackage{twemojis}
\usepackage{utfsym}
\usepackage{bclogo}
\usepackage{marvosym}
\usepackage{fontmfizz}
%%%%%%%%%%%%%%%%%%%%%%%%%%%
\usepackage{tikz}
\usetikzlibrary{backgrounds}
\usepackage{tcolorbox}
\usepackage{tikzpeople}
\usepackage{xcolor}
\usepackage{amsmath}
%%%%%%%%%%%%%%%%%%%%%%%%%%%
%% 場合分け
\usepackage{cases}
%%%%%%%%%%%%%%%%%%%%%%%%%%%
% \myAnch{<名前>}{<色>}{<テキスト>}
% 指定のテキストを指定の色の四角枠で囲み, 指定の名前をもつTikZの
% ノードとして出力する. 図には remeber picture 属性を付けている
% ので外部から参照可能である.
\newcommand*{\myAnch}[3]{%
  \tikz[remember picture,baseline=(#1.base)]
    \node[draw,rectangle,#2] (#1) {\normalcolor #3};
}
%%%%%%%%%%%%%%%%%%%%%%%%%%%%
%% 音声リンク表示
\newcommand{\myaudio}[1]{\href{#1}{\faVolumeUp}}
%%%%%%%%%%%%%%%%%%%%%%%%%%%
% \myEmph コマンドの定義
%\newcommand{\myEmph}[3]{%
%    \textbf<#1>{\color<#1>{#2}{#3}}%
%}
\usepackage{xparse} % xparseパッケージの読み込み
\NewDocumentCommand{\myEmph}{O{} m m}{%
    \def\argOne{#1}%
    \ifx\argOne\empty
        \textbf{\color{#2}{#3}}% オプション引数が省略された場合
    \else
        \textbf<#1>{\color<#1>{#2}{#3}}% オプション引数が指定された場合
    \fi
}
%%%%%%%%%%%%%%%%%%%%%%%%%%%
%% 文末の上昇イントネーション記号 \myRisingPitch
%% 通常のイントネーション \myDownwardPitch
%% https://note.com/dan_oyama/n/n8be58e8797b2
%%%%%%%%%%%%%%%%%%%%%%%%%%%
\newcommand{\myRisingPitch}{
\begin{tikzpicture}[scale=0.3,baseline=0.3]
\draw[->,>=stealth] (0,0) to[bend right=45] (1,1);
\end{tikzpicture}
}
\newcommand{\myDownwardPitch}{
\begin{tikzpicture}[scale=0.3,baseline=0.3]
\draw[->,>=stealth] (0,1) to[bend left=45] (1,0);
\end{tikzpicture}
}
%%%%%%%%%%%%%%%%%%%%%%%%%%%
\title{English is fun.\,\,{}--- Are you hungry? ---}
\subtitle{Yes, I am. / No, I'm not.}
\author{}
\institute[]{}
\date[]

%%%%%%%%%%%%%%%%%%%%%%%%%%%%
%% TEXT
%%%%%%%%%%%%%%%%%%%%%%%%%%%%
\begin{document}
\begin{frame}[plain]
  \titlepage
\end{frame}

\section*{授業の流れ}
\begin{frame}[plain]
  \frametitle{授業の流れ}
  \tableofcontents
\end{frame}


\section{復習}
\subsection{be動詞の疑問文}
\begin{frame}[plain]{要点}

be動詞を用いた文の疑問文のつくり方\pause
\begin{exampleblock}{Topics for Today}
\begin{itemize}
 \item   主語とbe動詞を逆の順番にする\pause
 \item   文末に`?'をつける\pause
 \item   イントネーションは\myRisingPitch
\end{itemize}
     \end{exampleblock}

\pause
\mbox{}\hfill{}`?'をquestion mark(クエスチョンマーク)といいます
\end{frame}


\section{疑問文への答え方}
\subsection{Are you 〜 ? と聞かれたら}
 \begin{frame}[plain]{Are you 〜 ? と聞かれたら}
 \Large
\pause

Are you from New York?

\vspace{20pt}
\pause

\mbox{}\hspace{100pt}$\left\{\begin{tabular}{l}
         \text{Yes, I am.}\\\pause
         \text{No, I am not.}\\\pause
         \text{(}= \text{No, I'm not.)}
        \end{tabular}\right.$

\pause

\mbox{}\hfill{}{\small NoのときI amを縮めてNo, \textcolor{orange}{I'm} not.ともいいます}

\pause
\myaudio{./audio/009_answer_be_01.mp3}\,\,{}Listen carefully.
\end{frame}

\subsection{Is he 〜 ?  / Is she 〜 ? と聞かれたら}

\begin{frame}[plain]{Is he  〜 ? / Is she 〜 ? と聞かれたら}
 \Large


\begin{columns}[t]
\begin{column}{.475\textwidth}
Is he a baseball player?

\vspace{20pt}

\pause

\mbox{}\hspace{40pt}$\left\{\begin{tabular}{l}
         \text{Yes, he is.}\\\pause
         \text{No, he is not.}\\\pause
         \text{(}= \text{No, he's not.)}\\\pause
         \text{(}= \text{No, he isn't.)}
       \end{tabular}\right.$

\pause

\vspace{10pt}

\mbox{}\hfill{}{\small Noのときhe isを縮めてNo, \textcolor{orange}{he's} not.}

\vspace{-5pt}

\mbox{}\hfill{}{\small \phantom{Noのとき}is notを縮めてNo, he \textcolor{orange}{isn't.}}

\end{column}
\pause
\begin{column}{.475\textwidth}
Is she a good singer?

\vspace{20pt}

\pause

\mbox{}\hspace{40pt}$\left\{\begin{tabular}{l}
         \text{Yes, she is.}\\\pause
         \text{No, she is not.}\\\pause
         \text{(}= \text{No, she's not.)}\\\pause
         \text{(}= \text{No, she isn't.)}
       \end{tabular}\right.$


\pause

\vspace{20pt}
\mbox{}\hfill\myaudio{./audio/009_question_be_02.mp3}\,\,{}

\end{column}
\end{columns}

\end{frame}

\subsection{Is John 〜 ?  / Is Emily 〜 ? と聞かれたら}
\begin{frame}[plain]{Is John  〜 ? / Is Emily 〜 ? と聞かれたら}
 \Large

\pause
\begin{columns}[t]
\begin{column}{.49\textwidth}
Is \myAnch{john}{orange}{John} a teacher?


\pause

\vspace{20pt}

\mbox{}\hfill$\left\{\begin{tabular}{l}
         Yes, \myAnch{he1}{orange}{he} is.\\\pause
         No, \myAnch{he2}{orange}{he} is not.\\\pause
         \text{(}= \text{No, he's not.)}\\\pause
         \text{(}= \text{No, he isn't.)}
        \end{tabular}\right.$

\pause

%\mbox{}\hfill{}{\footnotesize }

\begin{tikzpicture}[remember picture,overlay]
 \draw[->,thick,orange] (john.south) to[out=-30,in=160] (he1.north west);
 \draw[->,thick,orange] (john.south) to[out=-110, in=155] (he2.west);
\end{tikzpicture}

\end{column}
\pause
\begin{column}{.49\textwidth}
Is \myAnch{emily}{orange}{Emily} from Sydney?

\pause

\vspace{20pt}

\mbox{}\hfill$\left\{\begin{tabular}{l}
         Yes, \myAnch{she1}{orange}{she} is.\\\pause
         No, \myAnch{she2}{orange}{she} is not.\\\pause
         \text{(}= \text{No, she's not.)}\\\pause
         \text{(}= \text{No, she isn't.)}
        \end{tabular}\right.$

\begin{tikzpicture}[remember picture,overlay]
 \draw[->,thick,orange] (emily.south) to[out=-30,in=160] (she1.north west);
 \draw[->,thick,orange] (emily.south) to[out=-165, in=165] (she2.west);
\end{tikzpicture}

\pause

\vspace{20pt}

\mbox{}\hfill\myaudio{./audio/009_question_be_03.mp3}\,\,{}
\end{column}
\end{columns}

\pause

{\small 答えるときは、名前そのものではなくheやsheを使います}
\end{frame}

 \subsection{Are they 〜 ? と聞かれたら}
\begin{frame}[plain]{Are they 〜 ? と聞かれたら}
 \Large


\pause

Are they your friends?

\pause

\vspace{10pt}

\mbox{}\hspace{100pt}$\left\{\begin{tabular}{l}ls
         Yes, they are.\\\pause
         No, they are not.\\\pause
         \text{(}= \text{No, they're not.)}\\\pause
         \text{(}= \text{No, they aren't.)}
        \end{tabular}\right.$
\vspace{10pt}

\mbox{}\hfill{}{\small Noのときthey areを縮めてNo, \textcolor{orange}{they're} not.}

\vspace{-8pt}

\mbox{}\hfill{}{\small \phantom{Noのとき}are notを縮めてNo, the \textcolor{orange}{aren't.}}

\vspace{20pt}

\mbox{}\hfill\myaudio{./audio/009_question_be_04.mp3}\,\,{}
\end{frame}



\begin{frame}[plain]{Are A and B 〜 ? と聞かれたら}
 \Large

Are \myAnch{s1}{orange}{George and Peter} your friends?


\pause

\vspace{10pt}

\mbox{}\hspace{100pt}$\left\{\begin{tabular}{l}
         Yes, \myAnch{they4}{orange}{they} are.\\\pause
         No, \myAnch{they5}{orange}{they} are not.\\\pause
          \text{(}= \text{No, they're not.)}\\\pause
         \text{(}= \text{No, they aren't.)}
       \end{tabular}\right.$

\begin{tikzpicture}[remember picture,overlay]
 \draw[->,thick,orange] (s1.south) to[out=-30,in=160] (they4.north west);
 \draw[->,thick,orange] (s1.south) to[out=-90, in=165] (they5.north west);
\end{tikzpicture}

\mbox{}\hfill\myaudio{./audio/009_question_be_05.mp3}\,\,{}

\pause

{\small 答えるときは、名前そのものではなくtheyを使います}

\end{frame}


 \subsection{Is this 〜 ?  / Is that 〜 ? と聞かれたら}
\begin{frame}[plain]{Is this  〜 ? / Is that 〜 ? と聞かれたら}
 \Large

\pause

\begin{columns}[t]
\begin{column}{.49\textwidth}
Is \myAnch{this}{orange}{this} your book?

\pause

\vspace{10pt}

\mbox{}\hfill$\left\{\begin{tabular}{l}
         Yes, \myAnch{it1}{orange}{it} is.\\\pause
         No, \myAnch{it2}{orange}{it} is not.\\\pause
          \text{(}= \text{No, it's not.)}\\\pause
         \text{(}= \text{No, it isn't.)}
       \end{tabular}\right.$

\pause

%\mbox{}\hfill{}{\footnotesize }

\begin{tikzpicture}[remember picture,overlay]
 \draw[->,thick,orange] (this.south) to[out=-30,in=160] (it1.north west);
 \draw[->,thick,orange] (this.south) to[out=-110, in=165] (it2.north west);
\end{tikzpicture}

\end{column}
\pause
\begin{column}{.49\textwidth}
Is \myAnch{that}{orange}{that} your house?

\pause

\vspace{10pt}

\mbox{}\hfill$\left\{\begin{tabular}{l}
         Yes, \myAnch{it3}{orange}{it} is.\\\pause
         No, \myAnch{it4}{orange}{it} is not.\\\pause
          \text{(}= \text{No, it's not.)}\\\pause
         \text{(}= \text{No, it isn't.)}
       \end{tabular}\right.$

\begin{tikzpicture}[remember picture,overlay]
 \draw[->,thick,orange] (that.south) to[out=-30,in=160] (it3.north west);
 \draw[->,thick,orange] (that.south) to[out=-90, in=165] (it4.north west);
\end{tikzpicture}
\pause

\mbox{}\hfill\myaudio{./audio/009_question_be_06.mp3}\,\,{}

\end{column}
\end{columns}

\end{frame}



\begin{frame}<1-9>[plain]\frametitle{Exercises}
例にならって、つぎの質問に対する答えを2通りつくりましょう。



\begin{tabular}{rlcll}
例:& Are you busy?& $\rightarrow$&(1) Yes, I am.&(2)No, I am not.\\
1&Are you from Tokyo?&$\rightarrow$&(1) Yes, I am.&(2) No, I am not.\\
2&Is she a science teacher?&$\rightarrow$& (1) Yes, she is.&(2) No, she is not.\\
3&Is Peter in Japan now?&$\rightarrow$&(1) Yes, he is.&(2) No, he is not.\\
4&Is math easy for you?&$\rightarrow$&(1) Yes, it is.&(2) No, it is not.\\
5&Is that your car?&$\rightarrow$&(1) Yes, it is.&(2) No, it is no.
\end{tabular}

\end{frame}

\end{document}

