\documentclass[aspectratio=169]{beamer}
\usepackage[no-math,deluxe,haranoaji]{luatexja-preset}
\renewcommand{\kanjifamilydefault}{\gtdefault}
\renewcommand{\emph}[1]{{\upshape\bfseries #1}}
\usetheme{metropolis}
\metroset{block=fill}
\setbeamertemplate{navigation symbols}{}
\setbeamertemplate{blocks}[rounded][shadow=false]
\usecolortheme[rgb={0.7,0.2,0.2}]{structure}
%%%%%%%%%%%%%%%%%%%%%%%%%%%
\usepackage{media9}
%%%%%%%%%%%%%%%%%%%%%%%%%%%
%% さまざまなアイコン
%%%%%%%%%%%%%%%%%%%%%%%%%%%
\usepackage{fontawesome}
\usepackage{figchild}
\usepackage{twemojis}
\usepackage{utfsym}
\usepackage{bclogo}
\usepackage{marvosym}
\usepackage{fontmfizz}
\usepackage{circledsteps}
\usepackage{tipa}
\usepackage{manfnt}
%%%%%%%%%%%%%%%%%%%%%%%%%%%
\usepackage{tikz}
\usetikzlibrary{backgrounds}
\usepackage{tcolorbox}
\usepackage{myascolorbox} %ascolorbox.sty(https://github.com/yasunari/ascolorbox)をlualatexで使えるようにzw2箇所を\zwに改変したもの
\usepackage{tikzpeople}
\usepackage[dvipsnames]{xcolor}
\usepackage{amsmath}
%%%%%%%%%%%%%%%%%%%%%%%%%%%
%% 場合分け
\usepackage{cases}
%%%%%%%%%%%%%%%%%%%%%%%%%%%
% \myAnch{<名前>}{<色>}{<テキスト>}
% 指定のテキストを指定の色の四角枠で囲み, 指定の名前をもつTikZの
% ノードとして出力する. 図には remeber picture 属性を付けている
% ので外部から参照可能である.
\newcommand*{\myAnch}[3]{%
  \tikz[remember picture,baseline=(#1.base)]
    \node[draw,rectangle,#2] (#1) {\normalcolor #3};
}
%%%%%%%%%%%%%%%%%%%%%%%%%%%%
%% 音声リンク表示
\newcommand{\myaudio}[1]{\href{#1}{\faVolumeUp}}
%%%%%%%%%%%%%%%%%%%%%%%%%%%
% \myEmph コマンドの定義
%\newcommand{\myEmph}[3]{%
%    \textbf<#1>{\color<#1>{#2}{#3}}%
%}
\usepackage{xparse} % xparseパッケージの読み込み
\NewDocumentCommand{\myEmph}{O{} m m}{%
    \def\argOne{#1}%
    \ifx\argOne\empty
        \textbf{\color{#2}{#3}}% オプション引数が省略された場合
    \else
        \textbf<#1>{\color<#1>{#2}{#3}}% オプション引数が指定された場合
    \fi
}
%%%%%%%%%%%%%%%%%%%%%%%%%%%
%% 文末の上昇イントネーション記号 \myRisingPitch
%% 通常のイントネーション \myDownwardPitch
%% https://note.com/dan_oyama/n/n8be58e8797b2
%%%%%%%%%%%%%%%%%%%%%%%%%%%
\newcommand{\myRisingPitch}{
\begin{tikzpicture}[scale=0.3,baseline=0.3]
\draw[->,>=stealth] (0,0) to[bend right=45] (1,1);
\end{tikzpicture}
}
\newcommand{\myDownwardPitch}{
\begin{tikzpicture}[scale=0.3,baseline=0.3]
\draw[->,>=stealth] (0,1) to[bend left=45] (1,0);
\end{tikzpicture}
}
%%%%%%%%%%%%%%%%%%%%%%%%%%%
\title{English is fun.}
\subtitle{Do you like sushi? --- Yes, I do. / No, I don't.}
\author{}
\institute[]{}
\date[]

%%%%%%%%%%%%%%%%%%%%%%%%%%%%
%% TEXT
%%%%%%%%%%%%%%%%%%%%%%%%%%%%
\begin{document}
\begin{frame}[plain]
  \titlepage
\end{frame}

\section*{授業の流れ}
\begin{frame}[plain]
  \frametitle{授業の流れ}
  \tableofcontents
\end{frame}


\section{一般動詞の疑問文}
\subsection{一般動詞の疑問文}
\begin{frame}[plain]{一般動詞の疑問文のつくり方}

\begin{tabular}{rlcl}
 1& {I have enough time.} &$\rightarrow$ &\onslide<2->{Do I have enough time?} \\
 2& {You play the guitar.}&$\rightarrow$ &\onslide<3->{Do you play the guitar?} \\
 3& {They speak French.}&$\rightarrow$ &\onslide<4->{Do they speak French?}\\
 4& {He sings well.}&$\rightarrow$ &\onslide<5->{Does he sing well?}\hspace{5pt}\onslide<6->{(*Does he sings well?)}
\end{tabular}


\begin{exampleblock}<7->{Topics for Today}
\begin{itemize}
 \item   先頭にDoまたはDoes(DoesをもちいるのはSが3人称単数のとき)\\
	 \Circled[fill color = white]{\,\,Do\,\,}\,$+$\,S\,$+$\,V{\scriptsize ($=$\,原形)} \ldots\,\,\,?\\
	 \Circled[fill color = white]{\,\,Does\,\,}\,$+$\,S\,$+$\,V{\scriptsize ($=$\,原形)} \ldots\,\,\,?
% \item   文末に`?'をつける
\end{itemize}
     \end{exampleblock}

\mbox{}\hfill\myaudio{./audio/011_answer_do_00.mp3}
\end{frame}
%%%%%%%%%%%%%%%%%%
%%%%%%%%%%%%%%%%
\begin{frame}<1-10>[plain]\frametitle{Exercises}

つぎの文を疑問文にしましょう。

 \begin{enumerate}
  \item<1-> You like flowers.\hspace{59.7pt}
        \onslide<2->{$\longrightarrow$\,\,\,\,\, Do you like flowers?\hfill\scalebox{.75}{\bcfleur\bcfleur}}
  \item<1-> They live in Boston.\hspace{47.5pt}%
        \onslide<3->{$\longrightarrow$\,\,\,\,\, Do they live in Boston?}
  \item<1-> She teaches science.\hspace{42pt}%
        \onslide<4->{$\longrightarrow$\,\,\,\,\, Does she teach science?\hfill\scalebox{1.75}{\twemoji{woman scientist}}}
  \item<5-> He has  a car.\hspace{80.5pt}%
        \onslide<6->{$\longrightarrow$\,\,\,\,\, Does he have a car?\hfill\faCar}
  \item<7-> Our teacher walks to school.
        \onslide<8->{$\longrightarrow$\,\,\,\,\, Does our teacher walk to school? }
 \end{enumerate}

\begin{exampleblock}<9->{Topics for Today}
\begin{itemize}
 \item   先頭にDoまたはDoes(DoesをもちいるのはSが3人称単数のとき)\\
	 \Circled[fill color = white]{\,\,Do\,\,}\,$+$\,S\,$+$\,V{\scriptsize ($=$\,原形)} \ldots\,\,\,?\\
	 \Circled[fill color = white]{\,\,Does\,\,}\,$+$\,S\,$+$\,V{\scriptsize ($=$\,原形)} \ldots\,\,\,?
% \item   文末に`?'をつける
\end{itemize}
     \end{exampleblock}

\vspace{-10pt}
% Embed the sound file
\onslide<10>{%
\mbox{}\hfill\myaudio{./audio/010_question_do_06.mp3}
}

\end{frame}
%%%%%%%%%%%%%%%%%%%%%%

%%%%%%%%%%%%%%%%%%%

\section{疑問文への答え方}
\subsection{Do you 〜 ? と聞かれたら}
 \begin{frame}[plain]{Do you 〜 ? と聞かれたら}
 \Large
\pause

Do you like sushi?\hfill\twemoji{sushi}

\vspace{20pt}
\pause

\mbox{}\hspace{100pt}$\left\{\begin{tabular}{l}
         \text{Yes, I do.}\\\pause
         \text{No, I do not.}\\\pause
         \text{(}= \text{No, I don't.)}
        \end{tabular}\right.$

\pause

\mbox{}\hfill{}{\small Noのときdo notを縮めてNo, I \textcolor{orange}{don't}.ともいいます}

\pause
\mbox{}\hfill\myaudio{./audio/011_answer_do_01.mp3}
\end{frame}

\subsection{Does he 〜 ?  / Does she 〜 ? と聞かれたら}

\begin{frame}[plain]{Does he  〜 ? / Does she 〜 ? と聞かれたら}
 \Large


\begin{columns}[t]
\begin{column}{.475\textwidth}
Does he walk to school?

\vspace{20pt}

\pause

\mbox{}\hspace{40pt}$\left\{\begin{tabular}{l}
         \text{Yes, he does.}\\\pause
         \text{No, he does not.}\\\pause
         \text{(}= \text{No, he doesn't.)}\\
       \end{tabular}\right.$

\pause

\vspace{10pt}

{\small Noのとき\\[-5pt]does notを縮めてNo, he \textcolor{orange}{doesn't.}}

\end{column}
\pause
\begin{column}{.475\textwidth}
Does she speak Japanese?

\vspace{20pt}

\pause

\mbox{}\hspace{40pt}$\left\{\begin{tabular}{l}
         \text{Yes, she does.}\\\pause
         \text{No, she does not.}\\\pause
           \text{(}= \text{No, she doesn't.)}
       \end{tabular}\right.$


\pause

\vspace{20pt}
\mbox{}\hfill\myaudio{./audio/011_answer_do_02.mp3}\,\,{}

\end{column}
\end{columns}

\end{frame}

\subsection{Does John 〜 ?  / Does Emily 〜 ? と聞かれたら}
\begin{frame}[plain]{Does John  〜 ? / Does Emily 〜 ? と聞かれたら}
 \Large

\pause
\begin{columns}[t]
\begin{column}{.49\textwidth}
Does \myAnch{john}{orange}{John} have a bike?


\pause

\vspace{20pt}

\mbox{}\hfill$\left\{\begin{tabular}{l}
         Yes, \myAnch{he1}{orange}{he} does.\\\pause
         No, \myAnch{he2}{orange}{he} does not.\\\pause
         \text{(}= \text{No, he doesn't.)}
        \end{tabular}\right.$

\pause

%\mbox{}\hfill{}{\footnotesize }

\begin{tikzpicture}[remember picture,overlay]
 \draw[->,thick,orange] (john.south) to[out=-30,in=160] (he1.north west);
 \draw[->,thick,orange] (john.south) to[out=-110, in=155] (he2.west);
\end{tikzpicture}

\end{column}
\pause
\begin{column}{.49\textwidth}
Does \myAnch{emily}{orange}{Emily} teach math?

\pause

\vspace{20pt}

\mbox{}\hfill$\left\{\begin{tabular}{l}
         Yes, \myAnch{she1}{orange}{she} does.\\\pause
         No, \myAnch{she2}{orange}{she} does not.\\\pause
          \text{(}= \text{No, she doesn't.)}
        \end{tabular}\right.$

\begin{tikzpicture}[remember picture,overlay]
 \draw[->,thick,orange] (emily.south) to[out=-30,in=160] (she1.north west);
 \draw[->,thick,orange] (emily.south) to[out=-165, in=165] (she2.west);
\end{tikzpicture}

\pause

\vspace{20pt}

\mbox{}\hfill\myaudio{./audio/011_answer_do_03.mp3}
\end{column}
\end{columns}

\pause

{\small 答えるときは、名前そのものではなくheやsheを使います}
\end{frame}

 \subsection{Do they 〜 ? と聞かれたら}
\begin{frame}[plain]{Do they 〜 ? と聞かれたら}
 \Large


\pause

Do they sing well??

\pause

\vspace{10pt}

\mbox{}\hspace{100pt}$\left\{\begin{tabular}{l}ls
         Yes, they do.\\\pause
         No, they do not.\\\pause
          \text{(}= \text{No, they don't.)}
        \end{tabular}\right.$
\vspace{10pt}

\mbox{}\hfill{}{\small Noのときdo notを縮めてNo, they \textcolor{orange}{don't.}}

\vspace{20pt}

\mbox{}\hfill\myaudio{./audio/011_answer_do_04.mp3}\,\,{}
\end{frame}



\begin{frame}[plain]{Do A and B 〜 ? と聞かれたら}
 \Large

Do \myAnch{s1}{orange}{George and Peter} go to school together?

\mbox{}\hfill{\footnotesize together: いっしょに}

\pause

\vspace{10pt}

\mbox{}\hspace{100pt}$\left\{\begin{tabular}{l}
         Yes, \myAnch{they4}{orange}{they} do.\\\pause
         No, \myAnch{they5}{orange}{they} do not.\\\pause
         \text{(}= \text{No, they don't.)}
       \end{tabular}\right.$

\begin{tikzpicture}[remember picture,overlay]
 \draw[->,thick,orange] (s1.south) to[out=-30,in=160] (they4.north west);
 \draw[->,thick,orange] (s1.south) to[out=-90, in=165] (they5.north west);
\end{tikzpicture}

\mbox{}\hfill\myaudio{./audio/011_question_do_05.mp3}\,\,{}

\pause

{\small 答えるときは、名前そのものではなくtheyを使います}

\end{frame}


\begin{frame}<1-20>[plain]\frametitle{Exercises}
例にならって、つぎの質問に対する答えを「はい」と「いいえ」の2通りつくりましょう。
\mbox{}\hfill\myaudio{./audio/011_answer_do_06.mp3}


\begin{tabular}{rlcll}
\visible<1->{例}& \visible<1->{Do you have pets?}& \visible<2->{$\rightarrow$}&\visible<3->{(1) Yes, I do.}&\visible<4->{(2) No, I do not.}\\
\visible<1->{1}&\visible<1->{Do they eat Chinese food?}&\visible<5->{$\rightarrow$}&\visible<6->{(1) Yes, they do.}&\visible<7->{(2) No, they do not.}\\
\visible<1->{2}&\visible<1->{Does she teach music?}&\visible<8->{$\rightarrow$}& \visible<9->{(1) Yes, she does.}&\visible<10->{(2) No, she does not.}\\
\visible<1->{3}&\visible<1->{Does Peter live in Japan?}&\visible<11->{$\rightarrow$}&\visible<12->{(1) Yes, he does.}&\visible<13->{(2) No, he does not.}\\
\visible<1->{4}&\visible<1->{Does George like tea?}&\visible<14->{$\rightarrow$}&\visible<15->{(1) Yes, he does.}&\visible<16->{(2) No, he does not.}\\
\visible<1->{5}&\visible<1->{Do John and Paul play the guitar?}&\visible<17->{$\rightarrow$}&\visible<18->{(1) Yes, they do.}&\visible<19->{(2) No, they do not.}
\end{tabular}

\begin{exampleblock}{Topics for Today}
\small
\begin{itemize}
 \item Yes, S $+ \left\{\begin{array}{l}
                  \text{do}\\
                \text{does}\end{array}\right\}$\,\,.
\hspace{20pt}
No, S $+ \left\{\begin{array}{l}
                  \text{do not($=$ don't)}\\
                \text{does not($=$ doesn't)}\end{array}\right\}$\,\,.
\end{itemize}
      \end{exampleblock}
\end{frame}

\begin{frame}[plain,t]\frametitle{Exercises}
%\begin{ascolorbox3}{Susanの自己紹介を読んで、問に答えましょう}[orange][coltitle=orange!50!black]
\begin{tcolorbox}[colframe=ForestGreen,
  colback=ForestGreen!10!white,
  colbacktitle=ForestGreen!40!white,
  coltitle=black, %fonttitle=\bfseries,
  title=次の英文を読んで、問に答えましょう。]
\parindent=15pt

\noindent{}
Hi, I'm Jennifer. I live in London and I'm in junior high school. My parents don't eat meat, but they really like fish. I also enjoy sushi. We often go to Japanese restaurants.
\mbox{}\hfill\myaudio{./audio/011_answer_do_07.mp3}
\end{tcolorbox}
%\end{ascolorbox3}

\pause
\begin{tabular}{rll}
1&\visible<2->{Does Jennifer go to junior high school?}&\visible<3->{Yes, she does.}\\
2&\visible<2->{Do her parents eat meat?}&\visible<4->{No, they don't.}\\
3&\visible<2->{Does she like sushi?}&\visible<5->{Yes, she does.}\\
4&\visible<2->{Do her parents go to Japanese restaurants?\hspace{20pt}\mbox{}}&\visible<6->{Yes, they do.}
\end{tabular}

\pause

\mbox{}\hfill\myaudio{./audio/011_answer_do_08.mp3}\hspace{15pt}\mbox{}

\end{frame}
%%%%%%%%%%%%%%%%%%%%%%%%%%%%%%%
\begin{frame}[plain]{まとめ}
 

\begin{exampleblock}{一般動詞の疑問文のつくり方}\small
\begin{itemize}
 \item   先頭にDoまたはDoes(主語(S)が3人称単数のときは\,\Circled[fill color=white]{\,\,Does\,\,}\,ではじめます)\\
	 \Circled[fill color = white]{\,\,Do\,\,}\,$+$\,S\,$+$\,V{\scriptsize ($=$\,原形)} \ldots\,\,\,?\hfill{}Do you like sushi?\\
	 \Circled[fill color = white]{\,\,Does\,\,}\,$+$\,S\,$+$\,V{\scriptsize ($=$\,原形)} \ldots\,\,\,?\hfill{}Does she play the guitar?\\
\hfill{}*Does she plays the guitar?
% \item   文末に`?'をつける
\end{itemize}
     \end{exampleblock}

\begin{exampleblock}{一般動詞の疑問文への答え方}
\small

\begin{itemize}
 \item 主語(S)が3人称単数のときは\,\Circled[fill color=white]{\,\,does\,\,}\,を用います\\
Yes, S $+ \left\{\begin{array}{l}
		  \text{do}\\
		\text{does}\end{array}\right\}$\,\,.
\hspace{20pt}
No, S $+ \left\{\begin{array}{l}
		  \text{do not($=$ don't)}\\
		\text{does not($=$ doesn't)}\end{array}\right\}$\,\,.
\end{itemize}
      \end{exampleblock}

\begin{exampleblock}{Pronunciation}
 do \textipa{/d\'u:/}\hspace{15pt}does \textipa{/d\'2z/}\hspace{15pt}not \textipa{/n\'At/}\hspace{15pt}doesn't \textipa{/d\'2znt/}
\end{exampleblock}

\end{frame}


\end{document}
