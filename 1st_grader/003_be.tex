\documentclass[aspectratio=169,xcolor={dvipsnames,table}]{beamer}
\usepackage[no-math,deluxe,haranoaji]{luatexja-preset}
\renewcommand{\kanjifamilydefault}{\gtdefault}
\renewcommand{\emph}[1]{{\upshape\bfseries #1}}
\usetheme{metropolis}
\usetheme{metropolis}
\metroset{block=fill}
%%%%%%%%%%%%%%%%%%%%%%%%%%
\setbeamertemplate{navigation symbols}{}
\usecolortheme[rgb={0.7,0.2,0.2}]{structure}
%%%%%%%%%%%%%%%%%%%%%%%%%%%
%% さまざまなアイコン
%%%%%%%%%%%%%%%%%%%%%%%%%%%
\usepackage{fontawesome}
%%%%%%%%%%%%%%%%%%%%%%%%%%%
\usepackage{tikz}
\usepackage{circledsteps}
\usepackage{twemojis}
\usepackage{tikzlings}
\usepackage{scsnowman}
\usepackage{figchild}
%%%%%%%%%%%%%%%%%%%%%%%%%%%
%% 場合分け
\usepackage{cases}
%%%%%%%%%%%%%%%%%%%%%%%%%%%
%% 音声リンク表示
\newcommand{\myaudio}[1]{\href{#1}{\faVolumeUp}}
%%%%%%%%%%%%%%%%%%%%%%%%%%
%% Change alert block colors
%%% 1- Block title (background and text)
\setbeamercolor{block title alerted}{fg=mDarkTeal, bg=mLightBrown!45!yellow!45}
\setbeamercolor{block title example}{fg=magenta!10!black, bg=mLightGreen!70}
%%% 2- Block body (background)
\setbeamercolor{block body alerted}{bg=mLightBrown!25}
\setbeamercolor{block body example}{bg=mLightGreen!15}
%%%%%%%%%%%%%%%%%%%%%%%%%%%
%%%%%%%%%%%%%%%%%%%%%%%%%%%
\title{English is fun.\,\,{}---be動詞---}
\author{}
\institute[]{}
\date[]

%%%%%%%%%%%%%%%%%%%%%%%%%%%%
%% TEXT
%%%%%%%%%%%%%%%%%%%%%%%%%%%%
\begin{document}
%%%%%%%%%%%%%%%%%%%%%%%%%%%%
%\begin{frame}[label=waiting]{}
%%%%%%\phantomsection\label{section}
%\thispagestyle{empty}
%\Large
%\raggedright
%
%予定の時刻になったらはじまります
%
%\vfill
%
%\raggedleft
%
%The lesson will begin at the scheduled time.
%\end{frame}
%%%%%%%%%%%%%%%%%%%%%%%%%%%%%%
\begin{frame}[label=title]
%\phantomsection\label{section-1}
\thispagestyle{empty}
\titlepage
\end{frame}
%%%%%%%%%%%%%%%%%%%%%%%%%%%%
%\begin{frame}[plain]{Quiz}
% \large
%\visible<1->{%
%これからアルファベットを4つ順番に読みあげます。
%聞こえたアルファベットを順番に小文字で書いてください。すると単語になります。その意味を表す図を選んでください
%}
%\mbox{}\hfill\visible<1->{\myaudio{./audio/quiz/quiz_b.mp3}}
%
%\bigskip
%
%\centering
%\begin{tabular}{c@{   }c@{   }c@{   }c}
%\fcBike{.7}{Maroon}{1}&
%\scalebox{6}{\twemoji{doughnut}}&
%\scalebox{6}{\twemoji{sunflower}}&
%\scalebox{6}{\twemoji{book}}\\
%(a)&(b)&(c)&(d)
%\end{tabular}
%
%\bigskip
%
%\Huge
%
%\onslide<2->{b}%
%\onslide<3->{o}%
%\onslide<4->{o}%
%\onslide<5->{k}%
%
%\large
%\mbox{}\hfill\visible<1->{\myaudio{./audio/quiz/answer_b.mp3}}
%
%\end{frame}
%%%%%%%%%%%%%%%%%%%%%%%%%%%
\section*{授業の流れ}
\begin{frame}[plain]
  \frametitle{授業の流れ}
  \tableofcontents
\end{frame}
%%%%%%%%%%%%%%%%%%%%%%%%%%%%%%%
\section{復習--主語と動詞--}

\begin{frame}[plain]{復習}

\Large

主語と動詞
 \begin{exampleblock}{Topics for Today}
\begin{itemize}
 \item   英文の骨格は主語と動詞です
 \item   英語は語順がだいじです
 \item   英文にはかならず主語が必要
\end{itemize}
     \end{exampleblock}

\end{frame}
%%%%%%%%%%%%%%%%%%%%%%%%%%%%%%%%
\begin{frame}[plain]\frametitle{Exercises}
日本語を参考にして、主語と動詞を指摘してください。
\begin{enumerate}
    \item \alt<2->{\Circled[outer color=Maroon]{You}}{You} \alt<3->{\Circled[outer color=NavyBlue]{have}}{have} a nice car. あなたはいい車を持っている。
 \item \alt<4->{\Circled[outer color=Maroon]{I}}{I} \alt<5->{\Circled[outer color=NavyBlue]{play}}{play} the piano. わたしはピアノを弾きます。
    \item \alt<6->{\Circled[outer color=Maroon]{They}}{They} \alt<7->{\Circled[outer color=NavyBlue]{watch}}{watch} TV. 彼らはテレビを見ます。
    \item \alt<8->{\Circled[outer color=Maroon]{We}}{We} \alt<9->{\Circled[outer color=NavyBlue]{study}}{study} English. わたしたちは英語を勉強します。
 \item \alt<10->{\Circled[outer color=Maroon]{I}}{I} \alt<11->{\Circled[outer color=NavyBlue]{write}}{write} a letter. わたしは手紙を書きます。
     \item \alt<12->{\Circled[outer color=Maroon]{Birds}}{Birds} \alt<13->{\Circled[outer color=NavyBlue]{sing}}{sing}. 鳥は歌います。
    \item \alt<14->{\Circled[outer color=Maroon]{Dogs}}{Dogs} \alt<15->{\Circled[outer color=NavyBlue]{swim}}{swim} well. 犬はじょうずに泳ぎます。
     \item \alt<16->{\Circled[outer color=Maroon]{They}}{They} \alt<17->{\Circled[outer color=NavyBlue]{like}}{like} music. 彼らは音楽が好きです。
\end{enumerate}

% Embed the sound file
\onslide<1->{%
\myaudio{./audio/003_sv_03.mp3}
}
\end{frame}

%%%%%%%%%%%%%%%%%%%%%%%%%%%%%%%%
\section{be動詞とは}
\begin{frame}<1-27>[plain,label=what_is_be]\frametitle{be動詞とは}
 % \setbeamercovered{transparent}
  \begin{enumerate}
   \item<1-> I \textbf<14-20>{\color<14-20>{Maroon}{am}} a student. \onslide*<2>{わたしは生徒です。}\onslide*<14-20>{(I $=$ a student)}\hfill\onslide*<21-26>{\footnotesize  a student: (1人の)生徒、学生}
   \item<1-> You \textbf<15-20>{\color<15-20>{Maroon}{are}} my friend. \onslide*<4>{あなたはわたしのともだちです。}\onslide*<15-20>{(You $=$ my friend)}\hfill\onslide*<22-26>{\footnotesize  my: わたしの friend: ともだち}
   \item<1-> He \textbf<16-20>{\color<16-20>{Maroon}{is}} tall. \onslide*<6>{彼は背が高い。}\onslide*<16-20>{(He $=$ tall)}\hfill\onslide*<23-26>{\footnotesize  tall: 背が高い}
   \item<1-> She \textbf<17-20>{\color<17-20>{Maroon}{is}} kind. \onslide*<8>{彼女は親切だ。}\onslide*<17-20>{(She $=$ kind)}\hfill\onslide*<24-26>{\footnotesize  kind: 親切な}
   \item<1-> The sky \textbf<18-20>{\color<18-20>{Maroon}{is}} blue. \onslide*<10>{空は青い。}\onslide*<18-20>{(The sky $=$ blue)}\hfill\onslide*<25-26>{\footnotesize  the sky: 空 blue: 青い}
   \item<1-> They \textbf<19-20>{\color<19-20>{Maroon}{are}} my classmates. \onslide*<12>{彼らはわたしのクラスメートです。}\onslide*<19-20>{(They $=$ my classmates)}\hfill\onslide*<26>{\footnotesize  classmates: (2人以上の)クラスメート}
  \end{enumerate}

\bigskip

\begin{exampleblock}<20->{Topics for Today}
\begin{itemize}
 \item am, are, isはイコール($=$)の意味
 \item まとめてbe動詞といいます
\end{itemize}
\end{exampleblock}

% Embed the sound file
\onslide<27>{%
\myaudio{audio/002_be_01.mp3}
}
\end{frame}

%%%%%%%%%%%%%%%%%%%%%%%%%%%%%%%%%%%%%%%%%%%%%%%%%%%
%\section{復習--be動詞とは--}
%\againframe<19,27>{what_is_be}
%%%%%%%%%%%%%%%%%%%%%%%%%%%%%%%%%%%%%%%%%%%%%%%%%%%%
\begin{frame}[plain]\frametitle{am, are, is --- みんな、なかまです}
 \centering
\begin{tikzpicture}
% 補助グリッドを描画
%\draw[step=1cm, gray!20, very thin] (-6,-2) grid (6,6);
% ノードの定義と配置

5\node[circle, draw=black, fill=pink!30,minimum size=15mm, line width=1pt] (B) at (-5,0) {\LARGE am};\pause
\node[circle, draw=black, fill=blue!30, minimum size=15mm, line width=1pt] (C) at (0,5) {\LARGE are};\pause
\node[circle, draw=black, fill=green!30, minimum size=15mm, line width=1pt] (D) at (5,0) {\LARGE is};\pause
\node[circle, draw=black, fill=yellow!30, minimum size=20mm, line width=1pt] (A) at (0,0) {\LARGE be動詞};\pause
% ノード間の線の描画
\draw[-latex, line width=1.5pt] (A) -- node[above] {} (B);\pause
\draw[-latex, line width=1.5pt] (A) -- node[sloped, above] {} node[sloped, below] {} (C);\pause
\draw[-latex, line width=1.5pt] (A) -- node[above] {} (D);
\end{tikzpicture}
\end{frame}

\begin{frame}[plain]{be動詞、どれ使う?}
 \centering\Large

be動詞$\left\{\begin{tabular}[c]{l}
       am\\are\\is\end{tabular}\right\}$の使い分けを学習しよう
\end{frame}

\section{I am 〜}
\begin{frame}<1-20>[plain,label=i_am]\frametitle{I am 〜.}
 % \setbeamercovered{transparent}
  \begin{enumerate}
   \item<1-> \textbf{\color{Maroon}{I am}} a student. \onslide*<2>{わたしは生徒です。}\hfill\onslide*<15-20>{\footnotesize  student: 生徒、学生}
   \item<1-> \textbf{\color{Maroon}{I am}} tall. \onslide*<4>{わたしは背が高い。}\hfill\onslide*<16-20>{\footnotesize  tall: 背が高い}
   \item<1-> \textbf{\color{Maroon}{I am}} 13 years old. \onslide*<6>{わたしは13歳です。}\hfill\onslide*<17-20>{\footnotesize  〜 years old: 〜歳だ 〜には数字がはいります}
   \item<1-> \textbf{\color{Maroon}{I am}} John. \onslide*<8>{わたしはジョンです。}
   \item<1-> \textbf{\color{Maroon}{I am}} happy. \onslide*<10>{わたしは幸せです。}\hfill\onslide*<18-20>{\footnotesize  happy: 幸せだ}
   \item<1-> \textbf{\color{Maroon}{I am}} from Tokyo. \onslide*<12>{わたしは東京の出身です。}\hfill\onslide*<19-20>{\footnotesize  from 〜: 〜の出身だ}
  \end{enumerate}

\bigskip

\begin{exampleblock}<14->{Topics for Today}
\begin{itemize}
 \item   amはbe動詞
 \item I am 〜.をひとつのパターンとして覚えよう
\end{itemize}
     \end{exampleblock}

% Embed the sound file
\onslide<20>{%
\myaudio{./audio/002_be_02.mp3}
}

\end{frame}
%%%%%%%%%%%%%%%%%%%%%%%%%%%%%%%%%%%%%
%%%%%%%%%%%%%%%%%%%%%%%%%%%%%%%%%%%%%%%%%%%%%%%%%%%%
%\againframe{waiting}
%\againframe{title}
%%%%%%%%%%%%%%%%%%%%%%%%%%%%%%%%%%%%%%%%%%%%%%%%%%%
%\section{quiz}
%%%%%%%%%%%%%%%%%%%%%%%%%%%%
%\begin{frame}[plain]{Quiz}
% \large
%\visible<1->{%
%これからアルファベットを7つ順番に読みあげます。
%聞こえたアルファベットを順番に小文字で書いてください。すると、ある単語になります。その意味を表す図を選んでください
%}
%\mbox{}\hfill\visible<1->{\myaudio{./audio/quiz/quiz_d.mp3}}
%
%\bigskip
%
%\centering
%\begin{tabular}{c@{   }c@{   }c@{   }c}
%\scalebox{7}{\twemoji{club suit}}&
%\scalebox{7}{\twemoji{diamond suit}}&
%\scalebox{7}{\twemoji{heart suit}}&
%\scalebox{7}{\twemoji{spade suit}}\\
%(a)&(b)&(c)&(d)
%\end{tabular}
%
%\bigskip
%
%\Huge
%
%\onslide<2->{d}%
%\onslide<3->{i}%
%\onslide<4->{a}%
%\onslide<5->{m}%
%\onslide<6->{o}%
%\onslide<7->{n}%
%\onslide<8->{d}
%
%\large
%\mbox{}\hfill\visible<1->{\myaudio{./audio/quiz/answer_d.mp3}}
%
%\end{frame}
%%%%%%%%%%%%%%%%%%%%%%%%%%%%%%%%%%%%%%%%%%%%%%%%%%%
%\begin{frame}[plain]{もちろん、これもdiamondです}
%
%\raggedleft
%
%\includegraphics[height=.95\textheight]{./images/diamonds.jpg}
%
%\vspace*{-8pt}
%\tiny
%
%Photo by \href{https://unsplash.com/@edgardo1987?utm_content=creditCopyText&utm_medium=referral&utm_source=unsplash}{Edgar Soto} on \href{https://unsplash.com/photos/two-diamond-studded-silver-rings-gb0BZGae1Nk}{Unsplash}
%  
% \end{frame}
%%%%%%%%%%%%%%%%%%%%%%%%%%%%%%%%%%%%%%
%\section*{復習--I am--}
%\againframe<20>{i_am}
%%%%%%%%%%%%%%%%%%%%%%%%%%%%%%%%%%%%%%
\begin{frame}[plain]{Exercises}

John Smithさんは、
Boston出身で年齢は12歳です。

John Smithさんになったつもりで自己紹介文を作成してみよう。

\begin{enumerate}
 \item わたしはJohn Smithです。
 \item わたしは〜の出身です。\hfill{}from 〜
 \item わたしは〜歳です。\hfill{}〜 years old
\end{enumerate}

\end{frame}
%%%%%%%%%%%%%%%%%%%%%%%%%%%%%%%%%%%%%%

%%%%%%%%%%%%%%%%%%%%%%%%%%%%%%%%%%%%%%%%%%%%%%%%%%
\begin{frame}[plain]{Boston}

\raggedleft

\includegraphics[height=.9\textheight]{./images/boston_1.jpg}

\vspace*{-8pt}
\tiny

``Boston'' by jeffgunn is licensed under CC BY 2.0. \\
To view a copy of this license, visit \url{https://creativecommons.org/licenses/by/2.0/?ref=openverse}.
  \end{frame}
%%%%%%%%%%%%%%%%%%%%%%%%%%%%%%%%%%%%%%%%%%%%%%%%%%

\begin{frame}[plain]{Boston}

\raggedleft

\includegraphics[height=.9\textheight]{./images/boston_2.jpg}

\vspace*{-8pt}
\tiny


``Boston \`{a} l'heure bleue'' by Manu\_H is licensed under CC BY 2.0.\\
To view a copy of this license, visit \url{https://creativecommons.org/licenses/by/2.0/?ref=openverse}.

 \end{frame}

%%%%%%%%%%%%%%%%%%%%%%%%%%%%%%%%%%%%%
%%%%%%%%%%%%%%%%%%%%%%%%%%%%%%%%%%%%%%%%%%%%%%%%%%

\begin{frame}[plain]{Boston}

\raggedleft

\includegraphics[height=.9\textheight]{./images/boston_3.jpg}

\vspace*{-8pt}
\tiny


``Boston Rapid Transit Map'' by michaelvit is licensed under CC BY 2.0.\\
 To view a copy of this license, visit \url{https://creativecommons.org/licenses/by/2.0/?ref=openverse}.
 \end{frame}

%%%%%%%%%%%%%%%%%%%%%%%%%%%%%%%%%%%%%
\begin{frame}[plain]{Answer}
 \Large

\begin{enumerate}
 \item \visible<1->{I am}\onslide*<4->{($=$I'm)} \visible<1->{John Smith.}
 \item \visible<2->{I am}\onslide*<4->{($=$I'm)} \visible<2->{from Boston.}
 \item \visible<3->{I am}\onslide*<4->{($=$I'm)} \visible<3->{twelve years old.}
\end{enumerate}

\normalsize
\onslide<1->{%
\myaudio{audio/002_be_021.mp3}
}%
\hfill{}%
\onslide<5->{%
\myaudio{audio/002_be_022.mp3}
}

\normalsize
\begin{exampleblock}<5->{Tpics for Today}
\setbeamercovered{transparent}
\begin{itemize}
 \item<5>   amはbe動詞
 \item<6> I am 〜.をひとつのパターンとして覚えよう
 \item<7> $\text{I am}=\text{I'm}$\\
\mbox{}\hfill{}短縮形といいます。 \colorbox[gray]{0.9}{~'~}をapostrophy\,(アポストロフィ)といいます\phantom{あ}
\end{itemize}
     \end{exampleblock}
\end{frame}

%%%%%%%%%%%%%%%%%%%%%%%%%%%%%%%%%%%%%%
\section{You are 〜}
\begin{frame}<1-21>[plain]\frametitle{You are 〜.}
 % \setbeamercovered{transparent}
  \begin{enumerate}
   \item<1-> \textbf{\color{Maroon}{You are}} my friend. \onslide*<2>{あなたはわたしのともだちです。}\hfill\onslide*<15-20>{\footnotesize  friend: ともだち、友人}
   \item<1-> \textbf{\color{Maroon}{You are}} very kind. \onslide*<4>{あなたはとても親切だ。}\hfill\onslide*<16-20>{\footnotesize  kind: 親切だ}
   \item<1-> \textbf{\color{Maroon}{You are}} a good student. \onslide*<6>{あなたはいい生徒です。}\hfill\onslide*<17-20>{\footnotesize  student: 生徒}
   \item<1-> \textbf{\color{Maroon}{You are}} good at baseball. \onslide*<8>{あなたは野球がうまい。}\hfill\onslide*<18-20>{\footnotesize  good at 〜: 〜がうまい、得意だ}
   \item<1-> \textbf{\color{Maroon}{You are}} busy. \onslide*<10>{あなたは忙しい。}\hfill\onslide*<19-20>{\footnotesize  happy: 幸せだ}
   \item<1-> \textbf{\color{Maroon}{You are}} from Chiba. \onslide*<12>{あなたは千葉の出身です。}\hfill\onslide*<20>{\footnotesize  from 〜: 〜の出身だ}
  \end{enumerate}

\bigskip

\begin{exampleblock}<14->{Topics for Today}
\begin{itemize}
 \item areはbe動詞
 \item You are 〜.をひとつのパターンとして覚えよう
\end{itemize}
     \end{exampleblock}

% Embed the sound file
\onslide<14->{%
\myaudio{audio/002_be_03.mp3}

}
\end{frame}
%%%%%%%%%%%%%%%%%%%%%%%%%%%%%%%%%%%%
\begin{frame}<1-7>[plain]{Exercises}
あたえられた日本語の意味になるよう(~~~~~~)に適当な語を補いましょう。

\begin{enumerate}
 \item あなたは親切だ。(\alt<1>{~~\phantom{You}~~}{~~\textcolor{Maroon}{You}~~})~~are kind.
 \item あなたは背が高い。You~~(\alt<1-2>{~~\phantom{are}~~}{~~\textcolor{Maroon}{are}~~})~~tall.
 \item あなたは親切だ。(\alt<1-3>{~~\phantom{You're}~~}{~~\textcolor{Maroon}{You're}~~})~~kind.
 \item あなたは背が高い。(\alt<1-4>{~~\phantom{You're}~~}{~~\textcolor{Maroon}{You're}~~})~~tall.
\end{enumerate}

 \begin{exampleblock}<6->{Topics for Today}
\begin{itemize}
 \item areはbe動詞
 \item You are 〜.をひとつのパターンとして覚えよう
 \item $\text{You are}=\text{You're}$\hfill{}短縮形といいます 
\end{itemize}
     \end{exampleblock}

% Embed the sound file
\myaudio{audio/002_be_031.mp3}
\end{frame}

%%%%%%%%%%%%%%%%%%%%%%%%%%%%%%%%%%%
\section{I am, You are以外の場合(1)}
\begin{frame}<1-21>[plain]\frametitle{I am, You are以外の場合(1)}
 % \setbeamercovered{transparent}
  \begin{enumerate}
   \item<1-> This \textbf{\color{Maroon}{is}} my pencil. \onslide*<2>{これはわたしの鉛筆です。}\hfill\onslide*<15-20>{\footnotesize this: これ pencil: 鉛筆}
   \item<1-> He \textbf{\color{Maroon}{is}} my classmate. \onslide*<4>{彼はわたしのクラスメートです。}\hfill\onslide*<16-20>{\footnotesize   classmate: クラスメート、級友}
   \item<1-> She \textbf{\color{Maroon}{is}} a good singer. \onslide*<6>{彼女は歌がうまい。}\hfill\onslide*<17-20>{\footnotesize  singer: 歌い手、歌手}
   \item<1-> Your bike \textbf{\color{Maroon}{is}} new. \onslide*<8>{あなたの自転車は新しい。}\hfill\onslide*<18-20>{\footnotesize  bike: 自転車 new: 新しい}
   \item<1-> George \textbf{\color{Maroon}{is}} busy. \onslide*<10>{ジョージは忙しい。}\hfill\onslide*<19-20>{\footnotesize  busy: 忙しい}
   \item<1-> Jane \textbf{\color{Maroon}{is}} from France. \onslide*<12>{ジェーンはフランスの出身です。}\hfill\onslide*<20>{\footnotesize  from 〜: 〜の出身だ France: フランス}
  \end{enumerate}

\bigskip

\begin{exampleblock}<14->{Topics for Today}
\begin{itemize}
 \item I, You 以外で1つ(This, That, The book \ldots{})、1人(He, She, George, Jane \ldots{})で始まるときはisを使います
\end{itemize}  
     \end{exampleblock}


% Embed the sound file
\onslide<21>{%
\myaudio{audio/002_be_04.mp3}
}

\end{frame}


\section{I am, You are以外の場合(2)}

\begin{frame}<1-21>[plain]\frametitle{I am, You are以外の場合(2)}
 % \setbeamercovered{transparent}
  \begin{enumerate}
   \item<1-> These \textbf{\color{Maroon}{are}} my pencils. \onslide*<2>{これらはわたしの鉛筆です。}\hfill\onslide*<15-20>{\footnotesize  these: これら pencil: 鉛筆}
   \item<1-> They \textbf{\color{Maroon}{are}} my classmates. \onslide*<4>{彼らはわたしのクラスメートです。}\hfill\onslide*<16-20>{\footnotesize  they: 彼ら classmate: クラスメート、級友}
   \item<1-> They \textbf{\color{Maroon}{are}} kind. \onslide*<6>{彼らは親切だ。}\hfill\onslide*<17-20>{\footnotesize  kind: 親切な}
   \item<1-> The flowers \textbf{\color{Maroon}{are}} beautiful. \onslide*<8>{その花は美しい。}\hfill\onslide*<18-20>{\footnotesize  flower: 花 beautifuk: 美しい}
   \item<1-> We \textbf{\color{Maroon}{are}} busy. \onslide*<10>{わたしたちは忙しい。}\hfill\onslide*<19-20>{\footnotesize  we: わたしたち busy: 忙しい}
   \item<1-> Jane and George \textbf{\color{Maroon}{are}} from France. \onslide*<12>{ジェーンとジョージはフランスの出身です。}\hfill\onslide*<20>{\footnotesize  from 〜: 〜の出身だ France: フランス}
  \end{enumerate}

\bigskip

\begin{exampleblock}<14->{Topics for Today}
\begin{itemize}
 \item   I, You 以外で複数(2つ以上)のモノや人で始まるときはareを使います
\end{itemize}
     \end{exampleblock}

% Embed the sound file
\onslide<21>{%
\myaudio{audio/002_be_05.mp3}
}
\end{frame}

\section{まとめ}
\begin{frame}[plain]\frametitle{まとめ}

\begin{block}{be動詞の使い分け}

{\Large
\begin{numcases}{\text{ }}
 \text{\mbox{}\,\,{}Iではじまる}&$\longrightarrow$\,\,\,\,\,\,{}\text{am}\\
 \text{\mbox{}\,\,{}Youではじまる}&$\longrightarrow$\,\,\,\,\,\,{}\text{are}\\
 \text{\mbox{}\,\,{}1つ、1人}&$\longrightarrow$\,\,\,\,\,\,{}\text{is}\\
 \text{\mbox{}\,\,{}2つ、2人以上}&$\longrightarrow$\,\,\,\,\,\,{}\text{are}
\end{numcases}
}
\end{block}
\end{frame}

\begin{frame}[plain]\frametitle{まとめ}
 \centering
\begin{tikzpicture}
% 補助グリッドを描画
%\draw[step=1cm, gray!20, very thin] (-6,-2) grid (6,6);
% ノードの定義と配置
\node[circle, draw=black, fill=yellow!30, minimum size=20mm, line width=1pt] (A) at (0,0) {\LARGE be動詞};\pause
\node[circle, draw=black, fill=pink!30,minimum size=15mm, line width=1pt] (B) at (-6,0) {\LARGE am};\pause
\node[circle, draw=black, fill=blue!30, minimum size=15mm, line width=1pt] (C) at (0,5) {\LARGE are};\pause
\node[circle, draw=black, fill=green!30, minimum size=15mm, line width=1pt] (D) at (6,0) {\LARGE is};\pause

% ノード間の線の描画
\draw[-latex, line width=1.5pt] (A) -- node[above] {Iのとき} (B);\pause
\draw[-latex, line width=1.5pt] (A) -- node[sloped, above] {Youのとき}  node[sloped, below] {複数なら} (C);\pause
\draw[-latex, line width=1.5pt] (A) -- node[above] {1つ、1人なら} node[below] {} (D);
\end{tikzpicture}
\end{frame}
%%%%%%%%%%%%%%%%%%%%%%%%%%%%%%
\begin{frame}[plain]{Exercises}
 
次の英文の(~~~~~~~~)内から動詞の正しい形を選び、○で囲みましょう。

\begin{enumerate}
 \item John~~( am / \alt<2->{\Circled[outer color=orange]{is}}{is} / are )~~a baseball player.
 \item You~~( \alt<3->{\Circled[outer color=orange]{are}}{are} / is / am )~~nice.
 \item This~~( am / \alt<4->{\Circled[outer color=orange]{is}}{is} / are )~~my cup.
 \item They~~( am / is / \alt<5->{\Circled[outer color=orange]{are}}{are})~~my parents.
 \item We~~(am / \alt<6->{\Circled[outer color=orange]{are}}{are} / is )~~Japanese.
\end{enumerate}

\myaudio{audio/002_be_06.mp3}

\end{frame}
\end{document}
