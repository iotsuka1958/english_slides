\documentclass[aspectratio=169,xcolor={dvipsnames,table}]{beamer}
\usepackage[no-math,deluxe,haranoaji]{luatexja-preset}
\renewcommand{\kanjifamilydefault}{\gtdefault}
\renewcommand{\emph}[1]{{\upshape\bfseries #1}}
\usetheme{metropolis}
\metroset{block=fill}
\setbeamertemplate{navigation symbols}{}
\setbeamertemplate{blocks}[rounded][shadow=false]
\usecolortheme[rgb={0.7,0.2,0.2}]{structure}
%%%%%%%%%%%%%%%%%%%%%%%%%%%
\usepackage{media9}
%%%%%%%%%%%%%%%%%%%%%%%%%%%
%% さまざまなアイコン
%%%%%%%%%%%%%%%%%%%%%%%%%%%
\usepackage{fontawesome}
\usepackage{figchild}
\usepackage{twemojis}
\usepackage{utfsym}
\usepackage{bclogo}
\usepackage{marvosym}
%%%%%%%%%%%%%%%%%%%%%%%%%%%
\usepackage{tikz}
\usetikzlibrary{backgrounds}
\usepackage{tcolorbox}
\usepackage{tikzpeople}
\usepackage{pxrubrica}
\usepackage{amsmath}
\usepackage{tipa}
\usepackage{circledsteps}
\usepackage{booktabs}
\usepackage{tabularray}
\UseTblrLibrary{booktabs}
%%%%%%%%%%%%%%%%%%%%%%%%%%%
%% 場合分け
\usepackage{cases}
%%%%%%%%%%%%%%%%%%%%%%%%%%%
% \myAnch{<名前>}{<色>}{<テキスト>}
% 指定のテキストを指定の色の四角枠で囲み, 指定の名前をもつTikZの
% ノードとして出力する. 図には remeber picture 属性を付けている
% ので外部から参照可能である.
\newcommand*{\myAnch}[3]{%
  \tikz[remember picture,baseline=(#1.base)]
    \node[draw,rectangle,#2] (#1) {\normalcolor #3};
}
%%%%%%%%%%%%%%%%%%%%%%%%%%%%
%% 音声リンク表示
\newcommand{\myaudio}[1]{\href{#1}{\faVolumeUp}}
%%%%%%%%%%%%%%%%%%%%%%%%%%%
% \myEmph コマンドの定義
%\newcommand{\myEmph}[3]{%
%    \textbf<#1>{\color<#1>{#2}{#3}}%
%}
\usepackage{xparse} % xparseパッケージの読み込み
\NewDocumentCommand{\myEmph}{O{} m m}{%
    \def\argOne{#1}%
    \ifx\argOne\empty
        \textbf{\color{#2}{#3}}% オプション引数が省略された場合
    \else
        \textbf<#1>{\color<#1>{#2}{#3}}% オプション引数が指定された場合
    \fi
}
%%%%%%%%%%%%%%%%%%%%%%%%%%%
\title{English is fun.}
\subtitle{I am not hungry.}
\author{}
\institute[]{}
\date[]

%%%%%%%%%%%%%%%%%%%%%%%%%%%%
%% TEXT
%%%%%%%%%%%%%%%%%%%%%%%%%%%%
\begin{document}
\begin{frame}[plain]
  \titlepage
\end{frame}

%%%%%%%%%%%%%%%%%%%%%%%%%
\section*{授業の流れ}
%%%%%%%%%%%%%%%%%%%%%%%%
\begin{frame}[plain]
  \frametitle{授業の流れ}
  \tableofcontents
\end{frame}
%%%%%%%%%%%%%%%%%%%%%%%%%%%%
\section{否定を表す単語}
%%%%%%%%%%%%%%%%%%%%%%%%%
%%%%%%%%%%%%%%%%%%%%%%%%%%
\begin{frame}[plain]{否定を表すnot}
 \Large

否定を表すことば: {\LARGE\bfseries not}\hspace{20pt}\textipa{/n\'At/}
\end{frame}

%%%%%%%%%%%%%%%%%%%%%%%%%%
\section{be動詞の否定}
%%%%%%%%%%%%%%%%%%%%%%%
\begin{frame}[plain]\frametitle{$\text{A}\neq\text{B}$}
\large
\begin{tabular}{lll}
\onslide<1->{\textcolor{Maroon}{1.}\,\,\,\,I \textbf{am} hungry.}& \onslide<2->{(I $=$ hungry)}& \onslide<3->{\scriptsize わたしはおなかがすいている。}\hspace{51pt}\visible<11->{肯定文}\\
\onslide<6->{\textcolor{Maroon}{2.}\,\,\,\,I \textbf{am} \textcolor{Maroon}{\bfseries not} hungry.}& \onslide<5->{(I $\neq$ hungry)}& \onslide<4->{\scriptsize わたしはおなかがすいていない。}\hspace{43.2pt}\visible<11->{否定文}\\
\onslide<8->{\textcolor{Maroon}{3.}\,\,\,\,\textbf{I'm} \textcolor{Maroon}{\bfseries not} hungry.}&&\\

%\onslide<8->{I\textcolor{Maroon}{'m not} hungry.}
\end{tabular}

\vspace{40pt}
\onslide<1->{\scriptsize \mbox{}\hfill{}hungry \textipa{/h\'\textturnv Ngri/} 空腹だ}

\begin{block}<7->{Topics for Today}
\begin{itemize}\setbeamertemplate{items}[square]\small
 \item<7-> 否定「~でない」を表す単語は\textbf{not} \textipa{/n\'At/}
 \item<7->  I \textbf{am} X. の否定はI \textbf{am} \textcolor{Maroon}{\bfseries not} X.(\textbf{be}動詞の直後に\textbf{not}を置くだけ) 
 \item<9->  \textbf{I am}を縮めて \textbf {I'm} \textcolor{Maroon}{\bfseries not} X. ということがあります
 \item<10-> \textbf{not}を用いた文を\kenten{否定文}といいます。いっぽう\textbf{not}がない文は\kenten{肯定文}といいます
\end{itemize}
      \end{block}
\vspace{-10pt}
% Embed the sound file
\hfill{\tiny 0130}\,{\scriptsize \myaudio{audio/006_negative_be_01.mp3}}
\end{frame}
%%%%%%%%%%%%%%%%%%%%%%%%%%%%%%%%%%%%
%%%%%%%%%%%%%%%%%%%%%%%%%%%%%%%%%%%%
\begin{frame}[plain]\frametitle{$\text{A}\neq\text{B}$}
\large
\begin{tabular}{lll}
\onslide<1->{\textcolor{Maroon}{1.}\,\,\,\,You \textbf{are} from Canada.}& \onslide<2->{(You $=$ from Canada)}& \onslide<3->{{\scriptsize あなたはカナダの出身です。}}\\
\onslide<6->{\textcolor{Maroon}{2.}\,\,\,\,You \textbf{are} \textcolor{Maroon}{\bfseries not} from Canada.}& \onslide<5->{(You $\neq$ from Canada)}& \onslide<4->{{\scriptsize あなたはカナダの出身ではない。}}\\
\onslide<7->{\textcolor{Maroon}{3.}\,\,\,\,\textbf{You're} \textcolor{Maroon}{\bfseries not} from Canada.}\\
\onslide<8->{\textcolor{Maroon}{4.}\,\,\,\,You \textcolor{Maroon}{\bfseries aren't} from Canada.}
\end{tabular}

\onslide<1->{\scriptsize \mbox{}\hfill{}be from 〜:  〜の出身だ}

\begin{block}<9->{Topics for Today}
\begin{itemize}\setbeamertemplate{items}[square]\small
 \item  \textbf{You are} X. の否定は\textbf{You are} \textcolor{Maroon}{\bfseries not} X.(\textbf{be}動詞の直後に\textbf{not}を置くだけ)  
 \item  \textbf{You are}を縮めて\textbf{You're} \textcolor{Maroon}{\bfseries not} X. ということがあります
 \item  \textbf{are not}を縮めて\textcolor{Maroon}{\bfseries aren't} ということがあります
\end{itemize}
      \end{block}

% Embed the sound file
\hfill{\tiny 0158}\,{\myaudio{audio/006_negative_be_02.mp3}}
\end{frame}
%%%%%%%%%%%%%%%%%%%%%%%%%
%%%%%%%%%%%%%%%%%%%%%%%%%%
\begin{frame}[plain]\frametitle{$\text{A}\neq\text{B}$}
\large
\begin{tabular}{lll}
\onslide<1->{\textcolor{Maroon}{1.}\,\,\,\,She \textbf{is} a student.}& \onslide<2->{(She $=$ a student)}& \onslide<3->{{\scriptsize 彼女は学生です。}}\\
\onslide<6->{\textcolor{Maroon}{2.}\,\,\,\,She \textbf{is} \textcolor{Maroon}{\bfseries not} a student.}& \onslide<5->{(She $\neq$ a student)}& \onslide<4->{{\scriptsize 彼女は学生ではない。}}\\
\onslide<7->{\textcolor{Maroon}{3.}\,\,\,\,\textbf{She's} \textcolor{Maroon}{\bfseries not} a student.}\\
\onslide<8->{\textcolor{Maroon}{4.}\,\,\,\,She \textcolor{Maroon}{\bfseries isn't} a student.}
%\onslide<9->{She\textcolor{Maroon}{'s not} a student.}
\end{tabular}

\hfill\onslide<1->{\scriptsize student \textipa{/st(j)\'u:dnt/} 生徒、学生}

\begin{block}<9->{Topics for Today}
\begin{itemize}\setbeamertemplate{items}[square]\small 
 \item  \textbf{She is} X. の否定は\textbf{She is} \textcolor{Maroon}{\bfseries not} X.(\textbf{be}動詞の直後に\textbf{not}を置くだけ) 
 \item  \textbf{She is}を縮めて\textbf{She's} \textcolor{Maroon}{\bfseries not} ということがあります
 \item  \textbf{is not}を縮めて\textcolor{Maroon}{\bfseries isn't} ということがあります
\end{itemize}
      \end{block}

% Embed the sound file
\hfill{\tiny 0157}\,{\scriptsize \myaudio{audio/006_negative_be_03.mp3}}
\end{frame}
%%%%%%%%%%%%%%%%%%%%%%%%%%%%%%%%
%%%%%%%%%%%%%%%%%%%%%%%%%%%%%%%%%%%%
\begin{frame}[plain]\frametitle{$\text{A}\neq\text{B}$}
\large
\begin{tabular}{lll}
\onslide<1->{\textcolor{Maroon}{1.}\,\,\,\,They \textbf{are} busy.}& \onslide<2->{(They $=$ busy)}& \onslide<3->{{\scriptsize 彼らは忙しい。}}\\
\onslide<6->{\textcolor{Maroon}{2.}\,\,\,\,They \textbf{are} \textcolor{Maroon}{\bfseries not} busy.}& \onslide<5->{(They $\neq$ busy)}& \onslide<4->{{\scriptsize 彼らは忙しくない。}}\\
\onslide<7->{\textcolor{Maroon}{3.}\,\,\,\,\textbf{They're} \textcolor{Maroon}{\bfseries not} busy.}\\
\onslide<8->{\textcolor{Maroon}{4.}\,\,\,\,They \textcolor{Maroon}{\bfseries aren't} busy.}
\end{tabular}

\hfill{\scriptsize busy \textipa{/b\'Izi/} 忙しい}

\begin{block}<9->{Topics for Today}
\begin{itemize}\setbeamertemplate{items}[square]\small
 \item  \textbf{They are} X. の否定は\textbf{They are} \textcolor{Maroon}{\bfseries not} X.(\textbf{be}動詞の直後に\textbf{not}を置くだけ) 
 \item  \textbf{They are}を縮めて\textbf{They're} \textcolor{Maroon}{\bfseries not} X. ということがあります
 \item  \textbf{are not}を縮めて\textcolor{Maroon}{\bfseries aren't} ということがあります
\end{itemize}
      \end{block}

% Embed the sound file
\hfill{\tiny 0152}\,{\scriptsize \myaudio{audio/006_negative_be_031.mp3}}
\end{frame}
%%%%%%%%%%%%%%%%%%%%%%%%%
\begin{frame}[plain]\frametitle{Exercises}
つぎの各文を否定文にしてください(短縮形も可)

 % \setbeamercovered{transparent}
  \begin{enumerate}
   \item His mother is a teacher.\\
         \visible<2->{\,\,\,\,\,$\longrightarrow$\,His mother \textcolor{Maroon}{\bfseries is not} a teacher.}\\
	 \hfill\visible<3->{His mother\textbf{'s} \textcolor{Maroon}{\bfseries not} a teacher. / His mother \textcolor{Maroon}{\bfseries isn't} a teacher.}
   \item The room is clean.\\
         \visible<4->{\,\,\,\,\,$\longrightarrow$\,The room \textcolor{Maroon}{\bfseries is not} clean.}
	 \visible<5->{/ The room\textbf{'s} \textcolor{Maroon}{\bfseries not} clean. / The room \textcolor{Maroon}{\bfseries isn't} clean.}
   \item You are busy.\\
         \visible<6->{\,\,\,\,\,$\longrightarrow$\,You \textcolor{Maroon}{\bfseries are not} busy}.
	 \visible<7->{/ You\textbf{'re} \textcolor{Maroon}{\bfseries not} busy. / You \textcolor{Maroon}{\bfseries aren't} busy.}
   \item They are students.\\
         \visible<8->{\,\,\,\,\,$\longrightarrow$\,They \textcolor{Maroon}{\bfseries are not} students.}
	 \visible<9->{/ They\textbf{'re} \textcolor{Maroon}{\bfseries not} students. / They \textcolor{Maroon}{\bfseries aren't} students.}
   \item I am a doctor.\\
	 \visible<10->{\,\,\,\,\,$\longrightarrow$\,I \textcolor{Maroon}{\bfseries am not} a doctor.}
	 \visible<11->{/ I\textbf{'m} \textcolor{Maroon}{\bfseries not} a doctor.}
	 \visible<12->{/ *I \textcolor{NavyBlue}{amn't} a doctor.}
  \end{enumerate}
\vspace{-10pt}
% Embed the sound file
\hfill{\tiny 0447}\,{\scriptsize \myaudio{audio/006_negative_be_04.mp3}}
\end{frame}
%%%%%%%%%%%%%%%%%%%%%%%%%%
\section{まとめ}
%%%%%%%%%%%%%%%%%%%%%%%%%
\begin{frame}[plain]{まとめ}

\begin{block}<1->{要点}
\begin{itemize}\setbeamertemplate{items}[square]
 \item<1->  be動詞の否定は\,\,\Circled[fill color = white]{\,\,$\text{be動詞} + \text{\textcolor{Maroon}{\bfseries not}}$\,\,}(\textbf{be}動詞の直後に\textbf{not}を置くだけ) 
 \item<2-> 短縮形が用いられることがあります
       \begin{enumerate}\setbeamertemplate{items}[circle]
	\item<3-> \temporal<4,5>{She is not}{\Circled{She is} not}{She \Circled{is not}} busy.
	\item<5-> \textbf{She's} not busy.
	\item<7-> She \textbf{isn't} busy.
       \end{enumerate}
\end{itemize}
\end{block}
\end{frame}
%%%%%%%%%%%%%%%%%%%%%%%%%%%%%%%%%%%%%%%%%%
%\begin{frame}[plain]{短縮形}
% \begin{center}
% \rowcolors{3}{NavyBlue!40}{yellow!40}
%\begin{tabular}{lll}\toprule
%{\small }&\multicolumn{2}{c}{\small 短縮形}\\
%&{\small パターンA}&{\small パターンB}\\\midrule
%\visible<1->{I am not X.}&\visible<2->{{I'm not x.}}&\visible<10->{{*I amn't X.}}\\
%\visible<1->{You are not X.}&\visible<3->{{You're not x.}}&\visible<11->{{You aren't X.}}\\
%\visible<1->{He is not X.}&\visible<4->{{He's not x.}}&\visible<12->{{He isn't X.}}\\
%\visible<1->{She is not X.}&\visible<5->{{She's not x.}}&\visible<13->{{She isn't X.}}\\
%\visible<1->{It is not X.}&\visible<6->{{It's not x.}}&\visible<14->{{It isn't X.}}\\
%\visible<1->{That is not X.}&\visible<7->{{That's not x.}}&\visible<15->{{That isn't X.}}\\
%\visible<1->{We are not X.}&\visible<8->{{We're not x.}}&\visible<16->{{We aren't X.}}\\
%\visible<1->{They are not X.}&\visible<9->{{They're not x.}}&\visible<17->{{They aren't X.}}\\
%\bottomrule
%\end{tabular}%
%\end{center}
%\end{frame}
%%%%%%%%%%%%%%%%%%%%%%%%%%%%%%%%%%%%%%%
\begin{frame}[plain]{短縮形の一覧表}
 
% --- プリアンブルで必要なもの ---
% \usepackage{tabularray}
% \UseTblrLibrary{booktabs}
% \documentclass{beamer} などで色の定義も必要です

\centering
\begin{tblr}{
  colspec = {lll}, % 列の指定:3列とも左揃え
  row{odd} = {bg=NavyBlue!40},
  row{even} = {bg=yellow!40},
  row{1-2} = {font=\small, bg=white}, % 1行目と2行目のフォントを \small に
}
\toprule
% --- 表のヘッダー ---
& \SetCell[c=2]{c} 短縮形 & \\ % 2列を結合(Colspan=2)して中央揃え(center)
& パターンA & パターンB \\
\midrule
% --- 表の本体 ---
\visible<1->{I am not X.}   & \visible<2->{{I'm not x.}}   & \visible<10->{{*I amn't X.}} \\
\visible<1->{You are not X.}  & \visible<3->{{You're not x.}}  & \visible<11->{{You aren't X.}} \\
\visible<1->{He is not X.}    & \visible<4->{{He's not x.}}    & \visible<12->{{He isn't X.}} \\
\visible<1->{She is not X.}   & \visible<5->{{She's not x.}}   & \visible<13->{{She isn't X.}} \\
\visible<1->{It is not X.}    & \visible<6->{{It's not x.}}    & \visible<14->{{It isn't X.}} \\
\visible<1->{That is not X.}  & \visible<7->{{That's not x.}}  & \visible<15->{{That isn't X.}} \\
\visible<1->{We are not X.}   & \visible<8->{{We're not x.}}   & \visible<16->{{We aren't X.}} \\
\visible<1->{They are not X.} & \visible<9->{{They're not x.}} & \visible<17->{{They aren't X.}} \\
\bottomrule
\end{tblr}
\end{frame}
%%%%%%%%%%%%%%%%%%%%%%%%%%%%%%%%%%%
\end{document}

