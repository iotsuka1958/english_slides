\documentclass[aspectratio=169,xcolor={dvipsnames,table}]{beamer}
\usepackage[no-math,deluxe,haranoaji]{luatexja-preset}
\renewcommand{\kanjifamilydefault}{\gtdefault}
\renewcommand{\emph}[1]{{\upshape\bfseries #1}}
\usetheme{metropolis}
\metroset{block=fill}
\setbeamertemplate{navigation symbols}{}
\usecolortheme[rgb={0.7,0.2,0.2}]{structure}
%%%%%%%%%%%%%%%%%%%%%%%%%%%
\usepackage{media9}
%%%%%%%%%%%%%%%%%%%%%%%%%%%
%% さまざまなアイコン
%%%%%%%%%%%%%%%%%%%%%%%%%%%
\usepackage{fontawesome}
\usepackage{figchild}
\usepackage{twemojis}
\usepackage{utfsym}
\usepackage{bclogo}
\usepackage{marvosym}
%%%%%%%%%%%%%%%%%%%%%%%%%%%
\usepackage{tikz}
\usetikzlibrary{backgrounds}
\usepackage{tcolorbox}
\usepackage{tikzpeople}
\usepackage{xcolor}
\usepackage{amsmath}
%%%%%%%%%%%%%%%%%%%%%%%%%%%
%% 場合分け
\usepackage{cases}
%%%%%%%%%%%%%%%%%%%%%%%%%%%
% \myAnch{<名前>}{<色>}{<テキスト>}
% 指定のテキストを指定の色の四角枠で囲み, 指定の名前をもつTikZの
% ノードとして出力する. 図には remeber picture 属性を付けている
% ので外部から参照可能である.
\newcommand*{\myAnch}[3]{%
  \tikz[remember picture,baseline=(#1.base)]
    \node[draw,rectangle,#2] (#1) {\normalcolor #3};
}
%%%%%%%%%%%%%%%%%%%%%%%%%%%%
%% 音声リンク表示
\newcommand{\myaudio}[1]{\href{#1}{\faVolumeUp}}
%%%%%%%%%%%%%%%%%%%%%%%%%%%
% \myEmph コマンドの定義
%\newcommand{\myEmph}[3]{%
%    \textbf<#1>{\color<#1>{#2}{#3}}%
%}
\usepackage{xparse} % xparseパッケージの読み込み
\NewDocumentCommand{\myEmph}{O{} m m}{%
    \def\argOne{#1}%
    \ifx\argOne\empty
        \textbf{\color{#2}{#3}}% オプション引数が省略された場合
    \else
        \textbf<#1>{\color<#1>{#2}{#3}}% オプション引数が指定された場合
    \fi
}
%%%%%%%%%%%%%%%%%%%%%%%%%%%
\title{English is fun.}
\subtitle{I am not hugry.}
\author{}
\institute[]{}
\date[]

%%%%%%%%%%%%%%%%%%%%%%%%%%%%
%% TEXT
%%%%%%%%%%%%%%%%%%%%%%%%%%%%
\begin{document}
\begin{frame}[plain]
  \titlepage
\end{frame}

\section*{授業の流れ}
\begin{frame}[plain]
  \frametitle{授業の流れ}
  \tableofcontents
\end{frame}

\section{be動詞の否定}
\subsection{I am not hungry.}
\begin{frame}<1-8>[plain]\frametitle{$\text{A}\neq\text{B}$}

\begin{tabular}{lll}
\onslide<1->{\textcolor{Maroon}{1.}\,\,\,\,I am hungry.}& \onslide<2->{(I $=$ hungry)}& \onslide<4->{わたしはおなかがすいている。}\\
\onslide<7->{\textcolor{Maroon}{2.}\,\,\,\,I am \textcolor{orange}{not} hungry.}& \onslide<6->{(I $\neq$ hungry)}& \onslide<5->{わたしはおなかがすいていない。}
%\onslide<8->{I\textcolor{orange}{'m not} hungry.}
\end{tabular}

\vspace{50pt}
\onslide<3->{\small \mbox{}\hfill{}hungry: 空腹だ}

\onslide<8->{%
\begin{exampleblock}{Topics for Today}
\begin{itemize}
 \item  I am X. の否定はI am \textcolor{orange}{not} X. 
% \item  縮めて I\textcolor{orange}{'m not} ということもあります
\end{itemize}
      \end{exampleblock}
}
% Embed the sound file
\myaudio{audio/006_negative_be_01.mp3}\,\,{}Listen carefully.(注意して聞いてください)
\end{frame}

\subsection{You are not from Canada.}
\begin{frame}<1-9>[plain]\frametitle{$\text{A}\neq\text{B}$}

\begin{tabular}{lll}
\onslide<1->{\textcolor{Maroon}{1.}\,\,\,\,You are from Canada.}& \onslide<2->{(You $=$ from Canada)}& \onslide<4->{あなたはカナダの出身です。}\\
\onslide<7->{\textcolor{Maroon}{2.}\,\,\,\,You \textcolor{orange}{are not} from Canada.}& \onslide<6->{(You $\neq$ from Canada)}& \onslide<5->{あなたはカナダの出身ではない。}\\
\onslide<8->{\textcolor{Maroon}{3.}\,\,\,\,You \textcolor{orange}{aren't} from Canada.}
%\onslide<9->{You\textcolor{orange}{'re not} from Canada.}\\
\end{tabular}

\vspace{50pt}
\onslide<3->{\small \mbox{}\hfill{}from 〜:  〜の出身だ}

\onslide<9->{%
\begin{exampleblock}{Topics for Today}
\begin{itemize}
 \item  You are X. の否定はYou \textcolor{orange}{are not}  
 \item  縮めて You \textcolor{orange}{aren't} ということもあります
% \item  You\textcolor{orange}{'re not} ということもあります
\end{itemize}
      \end{exampleblock}
}
% Embed the sound file
\myaudio{audio/006_negative_be_02.mp3}\,\,{}Listen carefully.(注意して聞いてください)
\end{frame}

\subsection{She is not a student.}
\begin{frame}<1-9>[plain]\frametitle{$\text{A}\neq\text{B}$}

\begin{tabular}{lll}
\onslide<1->{She is a student.}& \onslide<2->{(She $=$ a student)}& \onslide<4->{彼女は学生です。}\\
\onslide<7->{She \textcolor{orange}{is not} a student.}& \onslide<6->{(She $\neq$ a student)}& \onslide<5->{彼女は学生ではない。}\\
\onslide<8->{She \textcolor{orange}{isn't} a student.}
%\onslide<9->{She\textcolor{orange}{'s not} a student.}
\end{tabular}

\vspace{50pt}
\onslide<3->{\small \mbox{}\hfill{}student: 生徒、学生}

\onslide<9->{%
\begin{exampleblock}{Topics for Today}
\begin{itemize}
 \item  She is X. の否定はShe \textcolor{orange}{is not} X.
 \item  縮めて She \textcolor{orange}{isn't} ということもあります
% \item  She\textcolor{orange}{'s not} ということもあります
\end{itemize}
      \end{exampleblock}
}
% Embed the sound file
\myaudio{audio/006_negative_be_03.mp3}\,\,{}Listen carefully.(注意して聞いてください)
\end{frame}

\begin{frame}<1-18>[plain]\frametitle{Exercises}
つぎの各文を否定文にしてください。


 % \setbeamercovered{transparent}
  \begin{enumerate}
   \item His mother is a teacher.\\
         \pause
         His mother \textcolor{orange}{is not} a teacher. \pause{}/ His mother \textcolor{orange}{isn't} a teacher.\pause
   \item The room is clean.\\
         \pause
         The room \textcolor{orange}{is not} clean. \pause{}/ The room \textcolor{orange}{isn't} clean.\pause
   \item You are busy.\\
         \pause
         You \textcolor{orange}{are not} busy. \pause{}/ You \textcolor{orange}{aren't} busy.\pause
   \item They are students.\\
         \pause
         They \textcolor{orange}{are not} students. \pause{}/ They \textcolor{orange}{aren't} students.
   \item I am a doctor.\\
         \pause
         I \textcolor{orange}{am not} a doctor. \pause{} / *I \textcolor{olive}{amn't} a doctor.\,\,\,{}$\longleftarrow$こうはいいません\pause
  \end{enumerate}



% Embed the sound file
\myaudio{audio/006_negative_be_04.mp3}\,\,{}Listen carefully.(注意して聞いてください)
\end{frame}

\end{document}

