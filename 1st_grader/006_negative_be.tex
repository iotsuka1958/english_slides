\documentclass[aspectratio=169,xcolor={dvipsnames,table}]{beamer}
\usepackage[no-math,deluxe,haranoaji]{luatexja-preset}
\renewcommand{\kanjifamilydefault}{\gtdefault}
\renewcommand{\emph}[1]{{\upshape\bfseries #1}}
\usetheme{metropolis}
\metroset{block=fill}
\setbeamertemplate{navigation symbols}{}
\usecolortheme[rgb={0.7,0.2,0.2}]{structure}
%%%%%%%%%%%%%%%%%%%%%%%%%%%
\usepackage{media9}
%%%%%%%%%%%%%%%%%%%%%%%%%%%
%% さまざまなアイコン
%%%%%%%%%%%%%%%%%%%%%%%%%%%
\usepackage{fontawesome}
\usepackage{figchild}
\usepackage{twemojis}
\usepackage{utfsym}
\usepackage{bclogo}
\usepackage{marvosym}
%%%%%%%%%%%%%%%%%%%%%%%%%%%
\usepackage{tikz}
\usetikzlibrary{backgrounds}
\usepackage{tcolorbox}
\usepackage{tikzpeople}
\usepackage{xcolor}
\usepackage{amsmath}
\usepackage{tipa}
%%%%%%%%%%%%%%%%%%%%%%%%%%%
%% 場合分け
\usepackage{cases}
%%%%%%%%%%%%%%%%%%%%%%%%%%%
% \myAnch{<名前>}{<色>}{<テキスト>}
% 指定のテキストを指定の色の四角枠で囲み, 指定の名前をもつTikZの
% ノードとして出力する. 図には remeber picture 属性を付けている
% ので外部から参照可能である.
\newcommand*{\myAnch}[3]{%
  \tikz[remember picture,baseline=(#1.base)]
    \node[draw,rectangle,#2] (#1) {\normalcolor #3};
}
%%%%%%%%%%%%%%%%%%%%%%%%%%%%
%% 音声リンク表示
\newcommand{\myaudio}[1]{\href{#1}{\faVolumeUp}}
%%%%%%%%%%%%%%%%%%%%%%%%%%%
% \myEmph コマンドの定義
%\newcommand{\myEmph}[3]{%
%    \textbf<#1>{\color<#1>{#2}{#3}}%
%}
\usepackage{xparse} % xparseパッケージの読み込み
\NewDocumentCommand{\myEmph}{O{} m m}{%
    \def\argOne{#1}%
    \ifx\argOne\empty
        \textbf{\color{#2}{#3}}% オプション引数が省略された場合
    \else
        \textbf<#1>{\color<#1>{#2}{#3}}% オプション引数が指定された場合
    \fi
}
%%%%%%%%%%%%%%%%%%%%%%%%%%%
\title{English is fun.}
\subtitle{I am not hugry.}
\author{}
\institute[]{}
\date[]

%%%%%%%%%%%%%%%%%%%%%%%%%%%%
%% TEXT
%%%%%%%%%%%%%%%%%%%%%%%%%%%%
\begin{document}
\begin{frame}[plain]
  \titlepage
\end{frame}

%%%%%%%%%%%%%%%%%%%%%%%%%
\section*{授業の流れ}
%%%%%%%%%%%%%%%%%%%%%%%%
\begin{frame}[plain]
  \frametitle{授業の流れ}
  \tableofcontents
\end{frame}
%%%%%%%%%%%%%%%%%%%%%%%%%%%%
\section{be動詞の否定}
\subsection{I am not hungry.}
%%%%%%%%%%%%%%%%%%%%%%%%%
%%%%%%%%%%%%%%%%%%%%%%%%%%
\begin{frame}[plain]{否定を表すnot}
 \Large

否定を表すことば: {\LARGE\bfseries not}\hspace{20pt}\textipa{/n\'At/}
\end{frame}

%%%%%%%%%%%%%%%%%%%%%%%%%%
\begin{frame}[plain]\frametitle{$\text{A}\neq\text{B}$}

\begin{tabular}{lll}
\onslide<1->{\textcolor{Maroon}{1.}\,\,\,\,I am hungry.}& \onslide<2->{(I $=$ hungry)}& \onslide<3->{\scriptsize わたしはおなかがすいている。}\\
\onslide<6->{\textcolor{Maroon}{2.}\,\,\,\,I am \textcolor{Maroon}{not} hungry.}& \onslide<5->{(I $\neq$ hungry)}& \onslide<4->{\scriptsize わたしはおなかがすいていない。}
%\onslide<8->{I\textcolor{Maroon}{'m not} hungry.}
\end{tabular}

\vspace{50pt}
\onslide<1->{\small \mbox{}\hfill{}hungry \textipa{/h\'\textturnv Ngri/} 空腹だ}

\begin{block}<7->{Topics for Today}
\begin{itemize}\setbeamertemplate{items}[square]\small
 \item 否定「~でない」を表す単語はnot \textipa{/n\'At/}
 \item  I am X. の否定はI am \textcolor{Maroon}{not} X. 
% \item  縮めて I\textcolor{Maroon}{'m not} ということもあります
\end{itemize}
      \end{block}

% Embed the sound file
\hfill{\tiny 0113}\,{\scriptsize \myaudio{audio/006_negative_be_01.mp3}}
\end{frame}
%%%%%%%%%%%%%%%%%%%%%%%%%%%%%%%%%%%%
\subsection{You are not from Canada.}
%%%%%%%%%%%%%%%%%%%%%%%%%%%%%%%%%%%%
\begin{frame}[plain]\frametitle{$\text{A}\neq\text{B}$}

\begin{tabular}{lll}
\onslide<1->{\textcolor{Maroon}{1.}\,\,\,\,You are from Canada.}& \onslide<2->{(You $=$ from Canada)}& \onslide<3->{{\scriptsize あなたはカナダの出身です。}}\\
\onslide<6->{\textcolor{Maroon}{2.}\,\,\,\,You \textcolor{Maroon}{are not} from Canada.}& \onslide<5->{(You $\neq$ from Canada)}& \onslide<4->{{\scriptsize あなたはカナダの出身ではない。}}\\
\onslide<7->{\textcolor{Maroon}{3.}\,\,\,\,You \textcolor{Maroon}{aren't} from Canada.}
%\onslide<9->{You\textcolor{Maroon}{'re not} from Canada.}\\
\end{tabular}

\vspace{50pt}
\onslide<1->{\small \mbox{}\hfill{}from 〜:  〜の出身だ}

\begin{block}<8->{Topics for Today}
\begin{itemize}\setbeamertemplate{items}[square]\small
 \item  You are X. の否定はYou \textcolor{Maroon}{\bfseries are not} X.  
 \item  縮めて \textcolor{Maroon}{\bfseries aren't} ということもあります
% \item  You\textcolor{Maroon}{'re not} ということもあります
\end{itemize}
      \end{block}

% Embed the sound file
\hfill{\tiny 0140}\,{\myaudio{audio/006_negative_be_02.mp3}}
\end{frame}
%%%%%%%%%%%%%%%%%%%%%%%%%
\subsection{She is not a student.}
%%%%%%%%%%%%%%%%%%%%%%%%%%
\begin{frame}[plain]\frametitle{$\text{A}\neq\text{B}$}

\begin{tabular}{lll}
\onslide<1->{She is a student.}& \onslide<2->{(She $=$ a student)}& \onslide<3->{{\scriptsize 彼女は学生です。}}\\
\onslide<6->{She \textcolor{Maroon}{is not} a student.}& \onslide<5->{(She $\neq$ a student)}& \onslide<4->{{\scriptsize 彼女は学生ではない。}}\\
\onslide<7->{She \textcolor{Maroon}{isn't} a student.}
%\onslide<9->{She\textcolor{Maroon}{'s not} a student.}
\end{tabular}

\vspace{50pt}

\hfill\onslide<1->{\scriptsize student \textipa{/st(j)\'u:dnt/} 生徒、学生}

\begin{block}<8->{Topics for Today}
\begin{itemize}\setbeamertemplate{items}[square]\small 
 \item  She is X. の否定はShe \textcolor{Maroon}{\bfseries is not} X.
 \item  縮めて \textcolor{Maroon}{\bfseries isn't} ということもあります
% \item  She\textcolor{Maroon}{'s not} ということもあります
\end{itemize}
      \end{block}

% Embed the sound file
\hfill{\tiny 0139}\,{\scriptsize \myaudio{audio/006_negative_be_03.mp3}}
\end{frame}
%%%%%%%%%%%%%%%%%%%%%%%%%%%%%%%%
\begin{frame}[plain]\frametitle{Exercises}
つぎの各文を否定文にしてください

 % \setbeamercovered{transparent}
  \begin{enumerate}
   \item His mother is a teacher.\\
         \visible<2->{His mother \textcolor{Maroon}{is not} a teacher.} \visible<3->{/ His mother \textcolor{Maroon}{isn't} a teacher.}
   \item The room is clean.\\
         \visible<4->{The room \textcolor{Maroon}{is not} clean.} \visible<5->{/ The room \textcolor{Maroon}{isn't} clean.}
   \item You are busy.\\
         \visible<6->{You \textcolor{Maroon}{are not} busy}. \visible<7->{/ You \textcolor{Maroon}{aren't} busy.}
   \item They are students.\\
         \visible<8->{They \textcolor{Maroon}{are not} students.} \visible<9->{/ They \textcolor{Maroon}{aren't} students.}
   \item I am a doctor.\\
         \visible<10->{I \textcolor{Maroon}{am not} a doctor.}  \visible<11->{/ *I \textcolor{NavyBlue}{amn't} a doctor.\,\,\,{}$\longleftarrow$こうはいいません}
  \end{enumerate}

% Embed the sound file
\hfill{\tiny 0434}\,{\scriptsize \myaudio{audio/006_negative_be_04.mp3}}
\end{frame}
%%%%%%%%%%%%%%%%%%%%%%%%%%
\end{document}

