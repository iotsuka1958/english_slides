\DocumentMetadata{lang=ja-JP}
\documentclass[book,jafontscale=0.9247]{jlreq}
%%%%%%%%%%%%%%%%%%%%%%%%%%%%
%% 欧文TTF/OTFフォントを利用するにはfontspec.styをロードする必要あり
%% 和文TTF/OTFフォントを利用するにはluatexja-fontspec.styをロードする必要あり
%% luatexja-fontspec.styはfontspec.styをないぶてきにロードする
%% lualatex-ja-preset.sty は luatexja-fontspec.styをロードする
%% つまり次の1行でluatexja-fontspec.sty, fontspec.styも自動的にロードされる
\usepackage[no-math,deluxe,expert,haranoaji]{luatexja-preset}
%%%%%%%%%%%%%%%%%%%%%%%%%%%
%%%%%%%%%%%%%%%%%%%%%%%%%%%
%% さまざまなアイコン
%%%%%%%%%%%%%%%%%%%%%%%%%%%
%\usepackage{fontawesome}
\usepackage{fontawesome5}
\usepackage{figchild}
\usepackage{twemojis}
\usepackage{utfsym}
\usepackage{bclogo}
\usepackage{marvosym}
\usepackage{fontmfizz}
\usepackage{pifont}
\usepackage{phaistos}
\usepackage{worldflags}
\usepackage{jigsaw}
\usepackage{tikzlings}
\usepackage{tikzducks}
\usepackage{scsnowman}
\usepackage{epsdice}
\usepackage{halloweenmath}
\usepackage{svrsymbols}
\usepackage{countriesofeurope}
\usepackage{tipa}
%%%%%%%%%%%%%%%%%%%%%%%%%%%
\usepackage{tikz}
\usetikzlibrary{calc,patterns,decorations.pathmorphing,backgrounds}
\usepackage{tcolorbox}
\usepackage{tikzpeople}
\usepackage{circledsteps}
\usepackage{xcolor}
\usepackage{amsmath}
\usepackage{booktabs}
\usepackage{chronology}
\usepackage{signchart}
%%%%%%%%%%%%%%%%%%%%%%%%%%%
%% 場合分け
%%%%%%%%%%%%%%%%%%%%%%%%%%%
\usepackage{cases}
%%%%%%%%%%%%%%%%%%%%%%%%%%
\usepackage{pdfpages}
%%%%%%%%%%%%%%%%%%%%%%%%%%%
%% 音声リンク表示
\newcommand{\myaudio}[1]{\href{#1}{\faVolumeUp}}
%%%%%%%%%%%%%%%%%%%%%%%%%%
%% \myAnch{<名前>}{<色>}{<テキスト>}
%% 指定のテキストを指定の色の四角枠で囲み, 指定の名前をもつTikZの
%% ノードとして出力する. 図には remember picture 属性を付けている
%% ので外部から参照可能である.
\newcommand*{\myAnch}[3]{%
  \tikz[remember picture,baseline=(#1.base)]
    \node[draw,rectangle,line width=1pt,#2] (#1) {\normalcolor #3};
}
%%%%%%%%%%%%%%%%%%%%%%%%%%
%% \myEmph コマンドの定義
%%%%%%%%%%%%%%%%%%%%%%%%%%
%\newcommand{\myEmph}[3]{%
%    \textbf<#1>{\color<#1>{#2}{#3}}%
%}
\usepackage{xparse} % xparseパッケージの読み込み
\NewDocumentCommand{\myEmph}{O{} m m}{%
    \def\argOne{#1}%
    \ifx\argOne\empty
        \textbf{\color{#2}{#3}}% オプション引数が省略された場合
    \else
        \textbf<#1>{\color<#1>{#2}{#3}}% オプション引数が指定された場合
    \fi
}
%%%%%%%%%%%%%%%%%%%%%%%%%%%
%%%%%%%%%%%%%%%%%%%%%%%%%%%
%% 文末の上昇イントネーション記号 \myRisingPitch
%% 通常のイントネーション \myDownwardPitch
%% https://note.com/dan_oyama/n/n8be58e8797b2
%%%%%%%%%%%%%%%%%%%%%%%%%%%
\newcommand{\myRisingPitch}{
\begin{tikzpicture}[scale=0.3,baseline=0.3]
\draw[->,>=stealth] (0,0) to[bend right=45] (1,1);
\end{tikzpicture}
}
\newcommand{\myDownwardPitch}{
\begin{tikzpicture}[scale=0.3,baseline=0.3]
\draw[->,>=stealth] (0,1) to[bend left=45] (1,0);
\end{tikzpicture}
}
%%%%%%%%%%%%%%%%%%%%%%%%%%%%
%\AtBeginSection[%
%]{%
%  \begin{frame}[plain]\frametitle{授業の流れ}
%     \tableofcontents[currentsection]
%   \end{frame}%
%}

%%%%%%%%%%%%%%%%%%%%%%%%%%
%%%%%%%%%%%%%%%%%%%%%%%%%%%%
%% TEXT
%%%%%%%%%%%%%%%%%%%%%%%%%%%%
\begin{document}
\tableofcontents
\clearpage
%%%%%%%
\chapter{母音母}

\section{\textipa{/\textepsilon /} の音}
\begin{description}
    \item[導入]こんにちは!今日は \textipa{/\textepsilon /} の音を学びます。
    \item[音の紹介]今日の音は \textipa{/e/} です。「bed」のような音です。口を少し開けて、舌を前に持ってきます。
    \item[音声聞取]ネイティブスピーカーのbed, red, headを聞きます。
    \item[発音練習]一緒にbed, red, headを練習しましょう。
    \item[ポイント] 日本語の「エ」とは少し異なります。舌の位置を前にすることを意識しましょう。
\end{description}

\section{\textipa{/i:/} の音}
\begin{enumerate}
    \item \textbf{導入}:こんにちは!今日は \textipa{/i:/} の音を学びます。前回の \textipa{/æ/} を復習しましょう。
    \item \textbf{音の紹介}:今日の音は \textipa{/i:/} です。seeのような音です。舌を前に持ってきて、口を少し開けます。
    \item \textbf{音声聞き取り}:ネイティブスピーカーのsee, key, theseを聞きます。
    \item \textbf{発音練習}:一緒にsee, key, theseを練習しましょう。ペアでお互いに練習します。
    \item \textbf{つまづきやすいポイントの強調}:日本語の「イ」とは少し違います。舌をしっかり前に出すことを意識しましょう。
    \item \textbf{まとめと振り返り}:今日の \textipa{/i:/} の音を復習します。次回は \textipa{/I/} の音を学びます。
\end{enumerate}

\section{\textipa{/\textsci /} の音}
\begin{enumerate}
    \item \textbf{導入}:こんにちは!今日は \textipa{/i\textlengthmark /} の音を学びます。
    \item \textbf{音の紹介}:今日の音は \textipa{/\textsci /} です。fishのような音です。舌を前に出して、口を少し開けますが、\textipa{/i:/} より短くします。
    \item \textbf{音声聞取}:ネイティブスピーカーのfish, sit, bigを聞きます。
    \item \textbf{発音練習}:一緒にfish, sit, bigを練習しましょう。
    \item \textbf{ポイント}:\textipa{/i:/} と \textipa{/I/} の違いに注意しましょう。 \textipa{/I/} は短い音です。
\end{enumerate}

\section{\textipa{/u:/} の音}
\begin{enumerate}
    \item \textbf{導入}:こんにちは!今日は \textipa{/u:/} の音を学びます。
    \item \textbf{音の紹介}:今日の音は \textipa{/u:/} です。twoのような音です。唇を丸めて、舌を後ろに引きます。
    \item \textbf{音声聞取}:ネイティブスピーカーのtwo, blue, moonを聞きます。
    \item \textbf{発音練習}:一緒にtwo, blue, moonを練習しましょう。ペアでお互いに練習します。
    \item \textbf{ポイント}:日本語の「ウ」とは違います。唇をしっかり丸めることを意識しましょう。
\end{enumerate}

\section{\textipa{/U/} の音}
\begin{enumerate}
    \item \textbf{導入}:こんにちは!今日は \textipa{/U/} の音を学びます。前回の \textipa{/u:/} を復習しましょう。
    \item \textbf{音の紹介}:今日の音は \textipa{/U/} です。「book」のような音です。唇を少し丸めて、舌を後ろに引きますが、\textipa{/u:/} より短くします。
    \item \textbf{音声聞き取り}:ネイティブスピーカーの「book」「good」「foot」を聞きます。
    \item \textbf{発音練習}:一緒に「book」「good」「foot」を練習しましょう。ペアでお互いに練習します。
    \item \textbf{つまづきやすいポイントの強調}:\textipa{/u:/} と \textipa{/U/} の違いに注意しましょう。 \textipa{/U/} は短い音です。
    \item \textbf{まとめと振り返り}:今日の \textipa{/U/} の音を復習します。次回は \textipa{/eI/} の音を学びます。
\end{enumerate}



\section{\textipa{/æ/} の音}
\begin{enumerate}
    \item \textbf{導入}:こんにちは!今日は新しい音を学びます。前回の音を復習しましょう。
    \item \textbf{音の紹介}:今日の音は \textipa{/æ/} です。catのような音です。口を大きく開けて、舌を低くします。
    \item \textbf{音声聞取}:ネイティブスピーカーのcat, bag, hat, candyを聞きます。
    \item \textbf{発音練習}:一緒にcat, bag, hat, candyを練習しましょう。
    \item \textbf{ポイント}:日本語にはない音なので、口をしっかり開けることを意識しましょう。
 \end{enumerate}

\section{\textipa{/\textscripta /} の音}
\begin{enumerate}
    \item \textbf{導入}:こんにちは!今日は \textipa{/\textscripta /} の音を学びます。前回の \textipa{/oU/} を復習しましょう。
    \item \textbf{音の紹介}:今日の音は \textipa{/\textscripta /} です。hotのような音です。口を大きく開け、舌を低くします。
    \item \textbf{音声聞取}:ネイティブスピーカーのhot, top, dogを聞きます。
    \item \textbf{発音練習}:一緒にhot, top, dogを練習しましょう。ペアでお互いに練習します。
    \item \textbf{ポイント}:日本語の「あ」とは少し異なります。口を大きく開けることを意識しましょう。
\end{enumerate}

\section{\textipa{/\textturnv /} の音}
\begin{enumerate}
    \item \textbf{導入}:こんにちは!今日は \textipa{/\textturnv /} の音を学びます。前回の \textipa{/a:/} を復習しましょう。
    \item \textbf{音の紹介}:今日の音は \textipa{/\textturnv /} です。loveのような音です。口をやや開けて、舌を低くし、中央に位置させます。
    \item \textbf{音声聞き取り}:ネイティブスピーカーのlove, cut, butを聞きます。
    \item \textbf{発音練習}:一緒にlove, cut, butを練習しましょう。ペアでお互いに練習します。
    \item \textbf{つまづきやすいポイントの強調}:日本語の「ア」とは異なり、口を広く開けずに発音します。
    \item \textbf{まとめと振り返り}:今日の \textipa{/\textturnv /} の音を復習します。次回は \textipa{/\textopeno :/} の音を学びます。
\end{enumerate}



\section{\textipa{/\textschwa /} の音}
\begin{enumerate}
    \item \textbf{導入}:こんにちは!今日は \textipa{/\textschwa /} の音を学びます。前回の \textipa{/e/} を復習しましょう。
    \item \textbf{音の紹介}:今日の音は \textipa{/\textschwa /} です。「sofa」のような音です。非常に短く、弱い音です。
    \item \textbf{音声聞き取り}:ネイティブスピーカーの「sofa」「about」「sofa」を聞きます。
    \item \textbf{発音練習}:一緒に「sofa」「about」「sofa」を練習しましょう。ペアでお互いに練習します。
    \item \textbf{つまづきやすいポイントの強調}: \textipa{/\textschwa /} は非常に短く、力を抜いて発音します。日本語にはない音なので特に注意しましょう。
    \item \textbf{まとめと振り返り}:今日の \textipa{/\textschwa /} の音を復習します。これで母音のレッスンが終了です!
\end{enumerate}

\section{\textipa{/\textopeno :/} の音}
\begin{enumerate}
    \item \textbf{導入}:こんにちは!今日は \textipa{/\textopeno :/} の音を学びます。前回の \textipa{/ʌ/} を復習しましょう。
    \item \textbf{音の紹介}:今日の音は \textipa{/\textopeno:/} です。「law」のような音です。口を大きく開け、舌を後ろに引きます。
    \item \textbf{音声聞き取り}:ネイティブスピーカーの「law」「saw」「caught」を聞きます。
    \item \textbf{発音練習}:一緒に「law」「saw」「caught」を練習しましょう。ペアでお互いに練習します。
    \item \textbf{つまづきやすいポイントの強調}:日本語の「オ」とは少し異なり、口を大きく開けることを意識しましょう。
    \item \textbf{まとめと振り返り}:今日の \textipa{/ɔ:/} の音を復習します。次回は \textipa{/ɜ:r/} の音を学びます。
\end{enumerate}
\section{\textipa{/eI/} の音}
\begin{enumerate}
    \item \textbf{導入}:こんにちは!今日は \textipa{/eI/} の音を学びます。前回の \textipa{/U/} を復習しましょう。
    \item \textbf{音の紹介}:今日の音は \textipa{/eI/} です。「cake」のような音です。舌を前に出して、口を少し開けます。 \textipa{/e/} から \textipa{/I/} への移行音です。
    \item \textbf{音声聞き取り}:ネイティブスピーカーの「cake」「day」「say」を聞きます。
    \item \textbf{発音練習}:一緒に「cake」「day」「say」を練習しましょう。ペアでお互いに練習します。
    \item \textbf{つまづきやすいポイントの強調}: \textipa{/eI/} は滑らかな移行音です。日本語の「エ」と「イ」を連続的に発音する感じです。
    \item \textbf{まとめと振り返り}:今日の \textipa{/eI/} の音を復習します。次回は \textipa{/oU/} の音を学びます。
\end{enumerate}

\section{\textipa{/oU/} の音}
\begin{enumerate}
    \item \textbf{導入}:こんにちは!今日は \textipa{/oU/} の音を学びます。前回の \textipa{/eI/} を復習しましょう。
    \item \textbf{音の紹介}:今日の音は \textipa{/oU/} です。「go」のような音です。唇を丸めて、舌を後ろに引きます。 \textipa{/o/} から \textipa{/U/} への移行音です。
    \item \textbf{音声聞き取り}:ネイティブスピーカーの「go」「show」「no」を聞きます。
    \item \textbf{発音練習}:一緒に「go」「show」「no」を練習しましょう。ペアでお互いに練習します。
    \item \textbf{つまづきやすいポイントの強調}: \textipa{/oU/} は滑らかな移行音です。日本語の「オ」と「ウ」を連続的に発音する感じです。
    \item \textbf{まとめと振り返り}:今日の \textipa{/oU/} の音を復習します。次回は \textipa{/a:/} の音を学びます。
\end{enumerate}

\section{\textipa{/aI/} の音}
\begin{enumerate}
     \item \textbf{導入}:こんにちは!今日は \textipa{/aI/} の音を学びます。前回の \textipa{/ɜ:r/} を復習しましょう。
    \item \textbf{音の紹介}:今日の音は \textipa{/aI/} です。「my」のような音です。 \textipa{/a/} から \textipa{/I/} への移行音です。
    \item \textbf{音声聞き取り}:ネイティブスピーカーの「my」「tie」「sigh」を聞きます。
    \item \textbf{発音練習}:一緒に「my」「tie」「sigh」を練習しましょう。ペアでお互いに練習します。
    \item \textbf{つまづきやすいポイントの強調}: \textipa{/aI/} は滑らかな移行音です。日本語の「あい」とは少し異なります。
    \item \textbf{まとめと振り返り}:今日の \textipa{/aI/} の音を復習します。次回は \textipa{/aU/} の音を学びます。
\end{enumerate}

\section{\textipa{/aU/} の音}
\begin{enumerate}
    \item \textbf{導入}:こんにちは!今日は \textipa{/aU/} の音を学びます。前回の \textipa{/aI/} を復習しましょう。
    \item \textbf{音の紹介}:今日の音は \textipa{/aU/} です。「now」のような音です。 \textipa{/a/} から \textipa{/U/} への移行音です。
    \item \textbf{音声聞き取り}:ネイティブスピーカーの「now」「cow」「how」を聞きます。
    \item \textbf{発音練習}:一緒に「now」「cow」「how」を練習しましょう。ペアでお互いに練習します。
    \item \textbf{つまづきやすいポイントの強調}: \textipa{/aU/} は滑らかな移行音です。日本語の「アウ」とは少し異なります。
    \item \textbf{まとめと振り返り}:今日の \textipa{/aU

/} の音を復習します。次回は \textipa{/ɔI/} の音を学びます。
\end{enumerate}

\section{\textipa{/\textopeno I/} の音}
\begin{enumerate}
    \item \textbf{導入}:こんにちは!今日は \textipa{/\textopeno I/} の音を学びます。前回の \textipa{/aU/} を復習しましょう。
    \item \textbf{音の紹介}:今日の音は \textipa{/\textopeno I/} です。「boy」のような音です。 \textipa{/ɔ/} から \textipa{/I/} への移行音です。
    \item \textbf{音声聞き取り}:ネイティブスピーカーの「boy」「toy」「enjoy」を聞きます。
    \item \textbf{発音練習}:一緒に「boy」「toy」「enjoy」を練習しましょう。ペアでお互いに練習します。
    \item \textbf{つまづきやすいポイントの強調}: \textipa{/ɔI/} は滑らかな移行音です。日本語の「オイ」とは少し異なります。
    \item \textbf{まとめと振り返り}:今日の \textipa{/ɔI/} の音を復習します。次回は \textipa{/eə/} の音を学びます。
\end{enumerate}

\section{\textipa{/\textschwa :r/} の音}
\begin{enumerate}
    \item \textbf{導入}:こんにちは!今日は \textipa{/ɜ:r/} の音を学びます。前回の \textipa{/ɔ:/} を復習しましょう。
    \item \textbf{音の紹介}:今日の音は \textipa{/ɜ:r/} です。「bird」のような音です。舌を中央に置き、少し上げます。
    \item \textbf{音声聞き取り}:ネイティブスピーカーの「bird」「herd」「word」を聞きます。
    \item \textbf{発音練習}:一緒に「bird」「herd」「word」を練習しましょう。ペアでお互いに練習します。
    \item \textbf{つまづきやすいポイントの強調}:日本語にはない音なので、舌を中央に置きつつ、唇を少し丸めることを意識しましょう。
    \item \textbf{まとめと振り返り}:今日の \textipa{/ɜ:r/} の音を復習します。次回は \textipa{/aI/} の音を学びます。
\end{enumerate}

\section{\textipa{/e\textschwa r/} の音}
\begin{enumerate}
    \item \textbf{導入}:こんにちは!今日は \textipa{/e\textschwa r/} の音を学びます。前回の \textipa{/\textopeno I/} を復習しましょう。
    \item \textbf{音の紹介}:今日の音は \textipa{/e\textschwa r/} です。「air」のような音です。 \textipa{/e/} から \textipa{/ə/} への移行音です。
    \item \textbf{音声聞き取り}:ネイティブスピーカーの「air」「care」「there」を聞きます。
    \item \textbf{発音練習}:一緒に「air」「care」「there」を練習しましょう。ペアでお互いに練習します。
    \item \textbf{つまづきやすいポイントの強調}: \textipa{/eə/} は滑らかな移行音です。日本語の「エア」とは少し異なります。
    \item \textbf{まとめと振り返り}:今日の \textipa{/eə/} の音を復習します。次回は \textipa{/Iə/} の音を学びます。
\end{enumerate}

\section{\textipa{/I\textschwa r/} の音}
\begin{enumerate}
    \item \textbf{導入}:こんにちは!今日は \textipa{/I\textschwa r/} の音を学びます。前回の \textipa{/eə/} を復習しましょう。
    \item \textbf{音の紹介}:今日の音は \textipa{/I\textschwa r/} です。「ear」のような音です。 \textipa{/I/} から \textipa{/\textschwa /} への移行音です。
    \item \textbf{音声聞き取り}:ネイティブスピーカーの「ear」「near」「fear」を聞きます。
    \item \textbf{発音練習}:一緒に「ear」「near」「fear」を練習しましょう。ペアでお互いに練習します。
    \item \textbf{つまづきやすいポイントの強調}: \textipa{/I\textschwa/} は滑らかな移行音です。日本語の「イア」とは少し異なります。
    \item \textbf{まとめと振り返り}:今日の \textipa{/I\textschwa /} の音を復習します。次回は \textipa{/Uə/} の音を学びます。
\end{enumerate}

\section{\textipa{/U\textschwa r/} の音}
\begin{enumerate}
    \item \textbf{導入}:こんにちは!今日は \textipa{/U\textschwa r/} の音を学びます。前回の \textipa{/Iə/} を復習しましょう。
    \item \textbf{音の紹介}:今日の音は \textipa{/U\textschwa r/} です。「tour」のような音です。 \textipa{/U/} から \textipa{/ə/} への移行音です。
    \item \textbf{音声聞き取り}:ネイティブスピーカーの「tour」「poor」「sure」を聞きます。
    \item \textbf{発音練習}:一緒に「tour」「poor」「sure」を練習しましょう。ペアでお互いに練習します。
    \item \textbf{つまづきやすいポイントの強調}: \textipa{/U\textschwa r/} は滑らかな移行音です。日本語の「ウア」とは少し異なります。
    \item \textbf{まとめと振り返り}:今日の \textipa{/U\textschwa r/} の音を復習します。次回は総まとめのレッスンです。
\end{enumerate}

\section{総まとめ}
\begin{enumerate}
    \item \textbf{総復習}:これまで学んだ母音を一つずつ復習します。ネイティブスピーカーの音声を再度聞き、各音を復習します。
    \item \textbf{練習とテスト}:ペアでお互いに練習し、教師が個別に発音をチェックします。必要に応じてフィードバックを行います。
    \item \textbf{質問と回答}:学生からの質問を受け付け、わからない点を解消します。
    \item \textbf{まとめと展望}:母音の学習を総括し、今後の発音練習の方向性を示します。次回からは子音のレッスンに移ります。
\end{enumerate}



\end{document}
