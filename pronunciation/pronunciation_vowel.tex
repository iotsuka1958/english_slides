\documentclass[aspectratio=169,xcolor={dvipsnames,table}]{beamer}
\usepackage[no-math,deluxe,haranoaji]{luatexja-preset}
\renewcommand{\kanjifamilydefault}{\gtdefault}
\renewcommand{\emph}[1]{{\upshape\bfseries #1}}
\usetheme{metropolis}
\usetheme{metropolis}
\metroset{block=fill}
%%%%%%%%%%%%%%%%%%%%%%%%%%
\setbeamertemplate{navigation symbols}{}
\usecolortheme[rgb={0.7,0.2,0.2}]{structure}
%%%%%%%%%%%%%%%%%%%%%%%%%%
%% Change alert block colors
%%% 1- Block title (background and text)
\setbeamercolor{block title alerted}{fg=mDarkTeal, bg=mLightBrown!45!yellow!45}
\setbeamercolor{block title example}{fg=magenta!10!black, bg=mLightGreen!70}
%%% 2- Block body (background)
\setbeamercolor{block body alerted}{bg=mLightBrown!25}
\setbeamercolor{block body example}{bg=mLightGreen!15}
%%%%%%%%%%%%%%%%%%%%%%%%%%%
%%%%%%%%%%%%%%%%%%%%%%%%%%%
%% さまざまなアイコン
%%%%%%%%%%%%%%%%%%%%%%%%%%%
%\usepackage{fontawesome}
\usepackage{fontawesome5}
\usepackage{figchild}
\usepackage{twemojis}
\usepackage{utfsym}
\usepackage{bclogo}
\usepackage{marvosym}
\usepackage{fontmfizz}
\usepackage{pifont}
\usepackage{phaistos}
\usepackage{worldflags}
\usepackage{jigsaw}
\usepackage{tikzlings}
\usepackage{tikzducks}
\usepackage{scsnowman}
\usepackage{epsdice}
\usepackage{halloweenmath}
\usepackage{svrsymbols}
\usepackage{countriesofeurope}
\usepackage{tipa}
\usepackage{manfnt}
%%%%%%%%%%%%%%%%%%%%%%%%%%%
\usepackage{tikz}
\usetikzlibrary{calc,patterns,decorations.pathmorphing,backgrounds}
\usepackage{tcolorbox}
\usepackage{tikzpeople}
\usepackage{circledsteps}
\usepackage{xcolor}
\usepackage{amsmath}
\usepackage{booktabs}
\usepackage{chronology}
\usepackage{signchart}
%%%%%%%%%%%%%%%%%%%%%%%%%%%
%% 場合分け
%%%%%%%%%%%%%%%%%%%%%%%%%%%
\usepackage{cases}
%%%%%%%%%%%%%%%%%%%%%%%%%%
\usepackage{pdfpages}
%%%%%%%%%%%%%%%%%%%%%%%%%%%
%% 音声リンク表示
\newcommand{\myaudio}[1]{\href{#1}{\faVolumeUp}}
%%%%%%%%%%%%%%%%%%%%%%%%%%
%% \myAnch{<名前>}{<色>}{<テキスト>}
%% 指定のテキストを指定の色の四角枠で囲み, 指定の名前をもつTikZの
%% ノードとして出力する. 図には remember picture 属性を付けている
%% ので外部から参照可能である.
\newcommand*{\myAnch}[3]{%
  \tikz[remember picture,baseline=(#1.base)]
    \node[draw,rectangle,line width=1pt,#2] (#1) {\normalcolor #3};
}
%%%%%%%%%%%%%%%%%%%%%%%%%%
%% \myEmph コマンドの定義
%%%%%%%%%%%%%%%%%%%%%%%%%%
%\newcommand{\myEmph}[3]{%
%    \textbf<#1>{\color<#1>{#2}{#3}}%
%}
\usepackage{xparse} % xparseパッケージの読み込み
\NewDocumentCommand{\myEmph}{O{} m m}{%
    \def\argOne{#1}%
    \ifx\argOne\empty
        \textbf{\color{#2}{#3}}% オプション引数が省略された場合
    \else
        \textbf<#1>{\color<#1>{#2}{#3}}% オプション引数が指定された場合
    \fi
}
%%%%%%%%%%%%%%%%%%%%%%%%%%%
%%%%%%%%%%%%%%%%%%%%%%%%%%%
%% 文末の上昇イントネーション記号 \myRisingPitch
%% 通常のイントネーション \myDownwardPitch
%% https://note.com/dan_oyama/n/n8be58e8797b2
%%%%%%%%%%%%%%%%%%%%%%%%%%%
\newcommand{\myRisingPitch}{
\begin{tikzpicture}[scale=0.3,baseline=0.3]
\draw[->,>=stealth] (0,0) to[bend right=45] (1,1);
\end{tikzpicture}
}
\newcommand{\myDownwardPitch}{
\begin{tikzpicture}[scale=0.3,baseline=0.3]
\draw[->,>=stealth] (0,1) to[bend left=45] (1,0);
\end{tikzpicture}
}
%%%%%%%%%%%%%%%%%%%%%%%%%%%%
%\AtBeginSection[%
%]{%
%  \begin{frame}[plain]\frametitle{授業の流れ}
%     \tableofcontents[currentsection]
%   \end{frame}%
%}

%%%%%%%%%%%%%%%%%%%%%%%%%%%
\title{English is fun.}
\subtitle{Pronunciation---vowel---}
\author{}
\institute[]{}
\date[]

%%%%%%%%%%%%%%%%%%%%%%%%%%%%
%% TEXT
%%%%%%%%%%%%%%%%%%%%%%%%%%%%
\begin{document}
%%%%%%%%%%%%%%%%%%%%%%%%%%%
% 背景色を黒に変更
\setbeamercolor{background canvas}{bg=black}
\begin{frame}
\centering
  \textcolor{white}{\Huge\bfseries Today's Pronunciation}\pause

 \vspace{30pt}

  \textcolor{white}{\Huge\bfseries \textipa{/\ae /}}
\end{frame}
\setbeamercolor{background canvas}{bg=}
%%%%%%%%%%%%%%%%%%%%%%%%%%
\begin{frame}[plain]{\textipa{/\ae /}}

\Huge
 \textipa{/\ae /}

\vspace*{20pt}

\normalsize
ポイント

\begin{itemize}
 \item 「エ」の口をして「ア」
 \item 唇を左右に広げる
\end{itemize}
\end{frame}
%%%%%%%%%%%%%%%%%%%%%%%%%%
\begin{frame}[plain]{実際の単語で確認しよう}
\LARGE

\begin{enumerate}
 \item c\textcolor{NavyBlue}{\bfseries a}t%
\hfill\makebox[80pt][l]{\textipa{/k\textcolor{BurntOrange}{\'\ae}t/}}\hspace{150pt}\mbox{}
 \item c\textcolor{NavyBlue}{\bfseries a}p
\hfill\makebox[80pt][l]{\textipa{/k\textcolor{BurntOrange}{\'\ae}p/}}\hspace{150pt}\mbox{}
 \item h\textcolor{NavyBlue}{\bfseries a}t
\hfill\makebox[80pt][l]{\textipa{/h\textcolor{BurntOrange}{\'\ae}t/}}\hspace{150pt}\mbox{}
 \item b\textcolor{NavyBlue}{\bfseries a}g
\hfill\makebox[80pt][l]{\textipa{/b\textcolor{BurntOrange}{\'\ae}g/}}\hspace{150pt}\mbox{}
 \item m\textcolor{NavyBlue}{\bfseries a}n
\hfill\makebox[80pt][l]{\textipa{/m\textcolor{BurntOrange}{\'\ae}n/}}\hspace{150pt}\mbox{}
 \item m\textcolor{NavyBlue}{\bfseries a}p
\hfill\makebox[80pt][l]{\textipa{/m\textcolor{BurntOrange}{\'\ae}p/}}\hspace{150pt}\mbox{}
 \item h\textcolor{NavyBlue}{\bfseries a}nd
\hfill\makebox[80pt][l]{\textipa{/h\textcolor{BurntOrange}{\'\ae}nd/}}\hspace{150pt}\mbox{}
\end{enumerate}
\end{frame}
%%%%%%%%%%%%%%%%%%%%%%%%%%%%%%
\begin{frame}[plain]{Exercises}
\LARGE

\begin{enumerate}
 \item The c\textcolor{NavyBlue}{\bfseries a}t sits on the m\textcolor{NavyBlue}{\bfseries a}t.
 \item D\textcolor{NavyBlue}{\bfseries a}d eats a j\textcolor{NavyBlue}{\bfseries a}m s\textcolor{NavyBlue}{\bfseries a}ndwich.
 \item The m\textcolor{NavyBlue}{\bfseries a}n likes his bl\textcolor{NavyBlue}{\bfseries a}ck c\textcolor{NavyBlue}{\bfseries a}p.
\end{enumerate}
\end{frame}
%%%%%%%%%%%%%%%%%%%%%%%%%%%
%%%%%%%%%%%%%%%%%%%%%%%%%%%
% 背景色を黒に変更
\setbeamercolor{background canvas}{bg=black}
\begin{frame}
\centering
  \textcolor{white}{\Huge\bfseries Today's Pronunciation}\pause

 \vspace{30pt}

  \textcolor{white}{\Huge\bfseries \textipa{/\textscripta /}}
\end{frame}
\setbeamercolor{background canvas}{bg=}
%%%%%%%%%%%%%%%%%%%%%%%%%%
\begin{frame}[plain]{\textipa{/\textscripta /}}

\Huge
 \textipa{/\textscripta /}

\vspace*{20pt}

\normalsize
ポイント

\begin{itemize}
 \item 口を縦に大きく開ける
 \item あくびをするイメージ
\end{itemize}
\end{frame}

%%%%%%%%%%%%%%%%%%%%%%%%%%
\begin{frame}[plain]{実際の単語で確認しよう}
\LARGE

\begin{enumerate}
 \item h\textcolor{NavyBlue}{\bfseries o}t%
\hfill\makebox[80pt][l]{\textipa{/h\textcolor{BurntOrange}{\'\textscripta}t/}}\hspace{150pt}\mbox{}
 \item d\textcolor{NavyBlue}{\bfseries o}g%
\hfill\makebox[80pt][l]{\textipa{/d\textcolor{BurntOrange}{\'\textscripta}g/}}\hspace{150pt}\mbox{} 
\item t\textcolor{NavyBlue}{\bfseries o}p%
\hfill\makebox[80pt][l]{\textipa{/t\textcolor{BurntOrange}{\'\textscripta}p/}}\hspace{150pt}\mbox{}
 \item n\textcolor{NavyBlue}{\bfseries o}t%
\hfill\makebox[80pt][l]{\textipa{/n\textcolor{BurntOrange}{\'\textscripta}t/}}\hspace{150pt}\mbox{} \item b\textcolor{NavyBlue}{\bfseries o}x%
\hfill\makebox[80pt][l]{\textipa{/b\textcolor{BurntOrange}{\'\textscripta}ks/}}\hspace{150pt}\mbox{}
 \item b\textcolor{NavyBlue}{\bfseries o}dy%
\hfill\makebox[80pt][l]{\textipa{/b\textcolor{BurntOrange}{\'\textscripta}di/}}\hspace{150pt}\mbox{}
 \item w\textcolor{NavyBlue}{\bfseries a}sh%
\hfill\makebox[80pt][l]{\textipa{/w\textcolor{BurntOrange}{\'\textscripta}\textesh /}}\hspace{150pt}\mbox{}
 \item w\textcolor{NavyBlue}{\bfseries a}nt%
\hfill\makebox[80pt][l]{\textipa{/w\textcolor{BurntOrange}{\'\textscripta}nt/}}\hspace{150pt}\mbox{}

\end{enumerate}
\end{frame}
%%%%%%%%%%%%%%%%%%%%%%%%%%%%%%
\begin{frame}[plain]{Exercises}
\LARGE

\begin{enumerate}
 \item T\textcolor{NavyBlue}{\bfseries o}m likes h\textcolor{NavyBlue}{\bfseries o}t d\textcolor{NavyBlue}{\bfseries o}gs.
 \item M\textcolor{NavyBlue}{\bfseries o}m has a j\textcolor{NavyBlue}{\bfseries o}b  at the sh\textcolor{NavyBlue}{\bfseries o}p.
 \item B\textcolor{NavyBlue}{\bfseries o}b makes h\textcolor{NavyBlue}{\bfseries o}t soup in a p\textcolor{NavyBlue}{\bfseries o}t.
\end{enumerate}
\end{frame}
%%%%%%%%%%%%%%%%%%%%%%%%%%%%%
%%%%%%%%%%%%%%%%%%%%%%%%%%%
% 背景色を黒に変更
\setbeamercolor{background canvas}{bg=black}
\begin{frame}
\centering
  \textcolor{white}{\Huge\bfseries Today's Pronunciation}\pause

 \vspace{30pt}

  \textcolor{white}{\Huge\bfseries \textipa{/\textturnv /}}
\end{frame}
\setbeamercolor{background canvas}{bg=}
%%%%%%%%%%%%%%%%%%%%%%%%%%
\begin{frame}[plain]{\textipa{/\textturnv /}}

\Huge
 \textipa{/\textturnv /}

\vspace*{20pt}

\normalsize
ポイント

\begin{itemize}
 \item 口をあまり開かず、喉の奥から声を出す
 \item 短くはっきり
\end{itemize}
\end{frame}

%%%%%%%%%%%%%%%%%%%%%%%%%%
\end{document}
