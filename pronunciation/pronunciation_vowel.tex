\documentclass[aspectratio=169,xcolor={dvipsnames,table}]{beamer}
\usepackage[no-math,deluxe,haranoaji]{luatexja-preset}
\renewcommand{\kanjifamilydefault}{\gtdefault}
\renewcommand{\emph}[1]{{\upshape\bfseries #1}}
\usetheme{metropolis}
\metroset{block=fill}
%%%%%%%%%%%%%%%%%%%%%%%%%%
\setbeamertemplate{navigation symbols}{}
\usecolortheme[rgb={0.7,0.2,0.2}]{structure}
%%%%%%%%%%%%%%%%%%%%%%%%%%
%% Change alert block colors
%%% 1- Block title (background and text)
\setbeamercolor{block title alerted}{fg=mDarkTeal, bg=mLightBrown!45!yellow!45}
\setbeamercolor{block title example}{fg=magenta!10!black, bg=mLightGreen!70}
%%% 2- Block body (background)
\setbeamercolor{block body alerted}{bg=mLightBrown!25}
\setbeamercolor{block body example}{bg=mLightGreen!15}
%%%%%%%%%%%%%%%%%%%%%%%%%%%
%%%%%%%%%%%%%%%%%%%%%%%%%%%
%% さまざまなアイコン
%%%%%%%%%%%%%%%%%%%%%%%%%%%
%\usepackage{fontawesome}
\usepackage{fontawesome5}
\usepackage{figchild}
\usepackage{twemojis}
\usepackage{utfsym}
\usepackage{bclogo}
\usepackage{marvosym}
\usepackage{fontmfizz}
\usepackage{pifont}
\usepackage{phaistos}
\usepackage{worldflags}
\usepackage{jigsaw}
\usepackage{tikzlings}
\usepackage{tikzducks}
\usepackage{scsnowman}
\usepackage{epsdice}
\usepackage{halloweenmath}
\usepackage{svrsymbols}
\usepackage{countriesofeurope}
\usepackage{tipa}
\usepackage{manfnt}
%%%%%%%%%%%%%%%%%%%%%%%%%%%
\usepackage{tikz}
\usetikzlibrary{calc,patterns,decorations.pathmorphing,backgrounds}
\usepackage{tcolorbox}
\usepackage{tikzpeople}
\usepackage{circledsteps}
\usepackage{xcolor}
\usepackage{amsmath}
\usepackage{booktabs}
\usepackage{chronology}
\usepackage{signchart}
%%%%%%%%%%%%%%%%%%%%%%%%%%%
%% 場合分け
%%%%%%%%%%%%%%%%%%%%%%%%%%%
\usepackage{cases}
%%%%%%%%%%%%%%%%%%%%%%%%%%
\usepackage{pdfpages}
%%%%%%%%%%%%%%%%%%%%%%%%%%%
%% 音声リンク表示
\newcommand{\myaudio}[1]{\href{#1}{\faVolumeUp}}
%%%%%%%%%%%%%%%%%%%%%%%%%%
%% \myAnch{<名前>}{<色>}{<テキスト>}
%% 指定のテキストを指定の色の四角枠で囲み, 指定の名前をもつTikZの
%% ノードとして出力する. 図には remember picture 属性を付けている
%% ので外部から参照可能である.
\newcommand*{\myAnch}[3]{%
  \tikz[remember picture,baseline=(#1.base)]
    \node[draw,rectangle,line width=1pt,#2] (#1) {\normalcolor #3};
}
%%%%%%%%%%%%%%%%%%%%%%%%%%
%% \myEmph コマンドの定義
%%%%%%%%%%%%%%%%%%%%%%%%%%
%\newcommand{\myEmph}[3]{%
%    \textbf<#1>{\color<#1>{#2}{#3}}%
%}
\usepackage{xparse} % xparseパッケージの読み込み
\NewDocumentCommand{\myEmph}{O{} m m}{%
    \def\argOne{#1}%
    \ifx\argOne\empty
        \textbf{\color{#2}{#3}}% オプション引数が省略された場合
    \else
        \textbf<#1>{\color<#1>{#2}{#3}}% オプション引数が指定された場合
    \fi
}
%%%%%%%%%%%%%%%%%%%%%%%%%%%
%%%%%%%%%%%%%%%%%%%%%%%%%%%
%% 文末の上昇イントネーション記号 \myRisingPitch
%% 通常のイントネーション \myDownwardPitch
%% https://note.com/dan_oyama/n/n8be58e8797b2
%%%%%%%%%%%%%%%%%%%%%%%%%%%
\newcommand{\myRisingPitch}{
\begin{tikzpicture}[scale=0.3,baseline=0.3]
\draw[->,>=stealth] (0,0) to[bend right=45] (1,1);
\end{tikzpicture}
}
\newcommand{\myDownwardPitch}{
\begin{tikzpicture}[scale=0.3,baseline=0.3]
\draw[->,>=stealth] (0,1) to[bend left=45] (1,0);
\end{tikzpicture}
}
%%%%%%%%%%%%%%%%%%%%%%%%%%%%
%\AtBeginSection[%
%]{%
%  \begin{frame}[plain]\frametitle{授業の流れ}
%     \tableofcontents[currentsection]
%   \end{frame}%
%}

\usepackage{vowel}
\usepackage{lua-ul}
\usepackage{pxrubrica}
\usepackage{tikzducks}
\usetikzlibrary{decorations.pathmorphing}
\usetikzlibrary{ducks}
\usepackage{scsnowman}
\usepackage{tikzlings}
%%%%%%%%%%%%%%%%%%%%%%%%%%%
\makeatletter
\newcommand*{\themonth}{\two@digits\month}
\newcommand*{\theday}{\two@digits\day}
\makeatother
\newcommand{\mytoday}{{\the\year}--{\themonth}--{\theday}}
%%%%%%%%%%%%%%%%%%%%%%%%%%%
\title{English is fun.}
\subtitle{Pronunciation---vowel---}
\author{}
\institute[]{}
\date[]

%%%%%%%%%%%%%%%%%%%%%%%%%%%%
%% TEXT
%%%%%%%%%%%%%%%%%%%%%%%%%%%%
\begin{document}
%%%%%%%%%%%%%%%%%%%%%%%%%%%
%%%%%%%%%%%%%%%%%%%%%%%%%%%%%%%%%%%%%%%%%%%%%%%%%%%%%
% 背景色をグレイに変更
%\setbeamercolor{background canvas}{bg=gray}
\setbeamercolor{background canvas}{bg=black}
\begin{frame}
%\centering
\raggedleft
  \textcolor{white}{\Huge\bfseries English is fun.}

\vfill

\raggedleft
% \textcolor{white}{\LARGE\bfseries 2024--11--26}
% \textcolor{white}{\LARGE\bfseries \today}
 \textcolor{white}{\LARGE\bfseries \mytoday}

\vfill
\vfill
\vfill

\raggedleft
\textcolor{white}{\large The lesson will begin at the scheduled time.}

%\textcolor{white}{\large 可能なら、鏡を用意してください}
\end{frame}
\setbeamercolor{background canvas}{bg=}
%%%%%%%%%%%%%%%%%%%%%%%%%%
%%%%%%%%%%%%%%%%%%%
%%% youtube
%%%%%%%%%%%%%%%%%%%%%%%%%%%%%%%%%%%%%%%%%%%%%%%%%%%%%
% 背景色をグレイに変更
%\setbeamercolor{background canvas}{bg=gray}
\setbeamercolor{background canvas}{bg=black}
\begin{frame}
%\centering
\raggedleft
  \textcolor{white}{\Huge\bfseries \textcolor{yellow}{E}nglish is fun.}

\vfill

\vfill

\raggedleft
 \textcolor{white}{\LARGE\bfseries \textcolor{yellow}{H}ello, everybody!}

 \textcolor{white}{\LARGE\bfseries \textcolor{yellow}{H}ow are you today?}

\raggedleft
 \textcolor{white}{\LARGE\bfseries \textcolor{yellow}{A}re you ready to start?}

 \textcolor{white}{\LARGE\bfseries \textcolor{yellow}{L}et's begin today's lesson.}

\vfill

\raggedleft
% \textcolor{white}{\LARGE\bfseries 2024--11--26}
% \textcolor{white}{\LARGE\bfseries \today}
 \textcolor{white}{\Large \bfseries \mytoday}

%\pause\textcolor{white}{\Large \bfseries 20250509}

%\pause\textcolor{white}{\Large \bfseries A prime number!}

\hyperlink{today}{\beamergotobutton{Today's Pronunciation}}%%todayにジャンプ
\end{frame}
\setbeamercolor{background canvas}{bg=}
%%%%%%%%%%%%%%%%%%%%%%%%%
%\begin{frame}[plain]{授業の進め方}\large
%
%\pause
%\begin{enumerate}
% \item<2-> スライドで進めていきます
% \item<3-> だれかを指名することはありません
%\end{enumerate}
%
%\bigskip
%
%\hfill{}\visible<4->{安心して授業に参加してください!
%\begin{tikzpicture}
%\duck[laughing,bowtie,
%strawhat=brown!50!white,
%ribbon=black,
%think={\scriptsize Enjoy!},
%bubblecolour=white!50!pink]
%\end{tikzpicture}}
%\end{frame}
%%%%%%%%%%%%%%%%%%%%%%%%%
\begin{frame}[plain]{だいじなこと}\large
\pause

基礎基本をたいせつにしよう!

\pause

\hfill{}やさしいことからはじめて着実に
\begin{tikzpicture}
\duck[signpost=\scalebox{0.25}{
\parbox{2.5cm}{\color{black}
Slow but steady wins the race.}},
signcolour=brown!70!gray,
signback=white!80!brown,
graduate=gray!20!black,
tassel=red!70!black
]
\end{tikzpicture} 
\end{frame}
%%%%%%%%%%%%%%%%%%%%%%%%%
\begin{frame}[plain]{準備するもの}
\pause
 \begin{itemize}\setbeamertemplate{items}[square]
  \item<2-> ノート
  \item<3-> 筆記具(黒以外に2色あるとなおいい)
 \end{itemize}


\hfill%
{\begin{tikzpicture}
\duck[squareglasses=blue!50!black,
speech={\scriptsize \twemoji{spiral notepad}\,{$+$}\,\usymW{2710}{.37cm}},
laughing
]
\end{tikzpicture}}
\end{frame}
%%%%%%%%%%%%%%%%%%%%%%%%%%
\begin{frame}[plain]{予習復習}\large
 \begin{itemize}\setbeamertemplate{items}[square]
  \item<1-> 予習\visible<2->{$\longrightarrow$\,必要}\visible<3->{ありません} 
  \item<4-> 復習\visible<5->{$\longrightarrow$\,効果的!} 
 \end{itemize}

\hfill%
\visible<6->{\begin{tikzpicture}
\duck[tshirt,
jacket=gray,
bowtie,
crazyhair,
speech={\tiny できる範囲で},
laughing,
signpost=\scalebox{0.4}{
\parbox{2cm}{
気楽に\\
気が\\向いたら}},
]
\end{tikzpicture}}
\end{frame}
%%%%%%%%%%%%%%%%%%%%%%%%%
\begin{frame}[plain]{音声をだいじにしよう}\large

英語の音声に親しみましょう

\begin{enumerate}
 \item<2-> Please listen carefully.\hfill{\scriptsize \myaudio{./audio/listen.mp3}}
 \item<3-> Please repeat after me.\hfill{\scriptsize \myaudio{./audio/repeat.mp3}}
\end{enumerate}
 
\hfill%
\begin{tikzpicture}
\duck[signpost=\scalebox{0.4}{
\parbox{2cm}{\color{black}
\centering よく使う\\ 指示です}},
signcolour=brown!70!gray,
signback=white!80!brown,
sombrero=orange!70!yellow,
sombreroa=green!70!blue,
sombrerob=red,
sombreroc=blue,
glasses,
icecream=brown,
flavoura=green!50!brown,
flavourb=white,
flavourc=red,
speech={\tiny \parbox{2cm}{\centering Lsten carefully.\\Repeat after me.}}]\end{tikzpicture}
\end{frame}
%%%%%%%%%%%%%%%%%%%%%%%%%
%\begin{frame}[plain,t]{授業について(まとめ)}
% \begin{itemize}\setbeamertemplate{items}[square]
%  \item 授業の進め方
%	\begin{itemize}\setbeamertemplate{items}[circle]
%	 \item スライド
%	 \item 指名はありません
%	\end{itemize}
%  \item 基礎基本をだいじに
%  \item 筆記具・ノート
%	\begin{itemize}\setbeamertemplate{items}[circle]
%	 \item 筆記具は黒以外に2色あるとなおいい
%	\end{itemize}
%  \item 復習
%	\begin{itemize}\setbeamertemplate{items}[circle]
%	 \item 効果的
%	 \item でも、余裕があればくらいの気持ちで
%	\end{itemize}
%  \item 英語の音声に親しみましょう
%	\begin{itemize}\setbeamertemplate{items}[circle]
%	 \item Please listen carefully. 
%	 \item Please repeat after me.
%	\end{itemize}
% \end{itemize}
%
%\vspace{-2.5cm}
%
%\hfill\begin{tikzpicture}
%\bear[
%scale=.75,
%signpost={\scriptsize よろしく},
%signcolour= brown!50!black,
%signback=green!40!black
%]
%\end{tikzpicture}
%
%\end{frame}
%%%%%%%%%%%%%%%%%%%%%%%%%
% 背景色を黒に変更
\setbeamercolor{background canvas}{bg=black}
\begin{frame}
\centering
  \textcolor{white}{\Huge\bfseries Today's Pronunciation}\pause

 \vspace{30pt}

  \textcolor{white}{\Huge\bfseries \textipa{/\ae /}}
\end{frame}
\setbeamercolor{background canvas}{bg=}
%%%%%%%%%%%%%%%%%%%%%%%%%%
\begin{frame}[plain]{\textipa{/\ae /}}

\Huge
 \textipa{/\ae /}%

\normalsize
ポイント%
\hfill\begin{tikzpicture}
\duck[%tshirt,
cap=NavyBlue!80!black,
glasses,
megaphone,
%jacket=gray,
buttons,
bowtie,
speech={\tiny 長めに!},
laughing,
signpost={\textipa{/\ae /}},
bubblecolour=white!30!yellow]
]
\duck[xshift=90pt, scale=.3, yshift=150pt]
\duck[xshift=60pt, scale=.3, yshift=100pt]
\duck[body=gray!50!white, head=gray!50!white,
xshift=80pt, scale=.3, yshift=50pt]
\end{tikzpicture}

\begin{itemize}\setbeamertemplate{items}[circle]
 \item 「エ」の口をして「ア」
 \item 唇を左右に広げる
\end{itemize}
\end{frame}
%%%%%%%%%%%%%%%%%%%%%%%%%%
\begin{frame}[plain]{実際の単語で確認しよう}
\LARGE
\hfill{\tiny 0432}\,{\scriptsize \myaudio{./audio/vowel_ae_01.mp3}}

\begin{enumerate}
 \item c\textcolor{NavyBlue}{\bfseries a}t%
\hfill\makebox[80pt][l]{\textipa{/k\textcolor{BurntOrange}{\'\ae}t/}}\hspace{150pt}\mbox{}
 \item c\textcolor{NavyBlue}{\bfseries a}p
\hfill\makebox[80pt][l]{\textipa{/k\textcolor{BurntOrange}{\'\ae}p/}}\hspace{150pt}\mbox{}
 \item h\textcolor{NavyBlue}{\bfseries a}t
\hfill\makebox[80pt][l]{\textipa{/h\textcolor{BurntOrange}{\'\ae}t/}}\hspace{150pt}\mbox{}
 \item b\textcolor{NavyBlue}{\bfseries a}g
\hfill\makebox[80pt][l]{\textipa{/b\textcolor{BurntOrange}{\'\ae}g/}}\hspace{150pt}\mbox{}
 \item m\textcolor{NavyBlue}{\bfseries a}n
\hfill\makebox[80pt][l]{\textipa{/m\textcolor{BurntOrange}{\'\ae}n/}}\hspace{150pt}\mbox{}
 \item m\textcolor{NavyBlue}{\bfseries a}p
\hfill\makebox[80pt][l]{\textipa{/m\textcolor{BurntOrange}{\'\ae}p/}}\hspace{150pt}\mbox{}
 \item h\textcolor{NavyBlue}{\bfseries a}nd
\hfill\makebox[80pt][l]{\textipa{/h\textcolor{BurntOrange}{\'\ae}nd/}}\hspace{150pt}\mbox{}
\end{enumerate}
\end{frame}
%%%%%%%%%%%%%%%%%%%%%%%%%%%%%%
\begin{frame}[plain]{Exercises \textipa{/\ae /}}
\LARGE

\begin{enumerate}
 \item The c\textcolor{Maroon}{\bfseries a}t sits on the m\textcolor{Maroon}{\bfseries a}t.
 \item D\textcolor{Maroon}{\bfseries a}d eats a j\textcolor{Maroon}{\bfseries a}m s\textcolor{Maroon}{\bfseries a}ndwich.
 \item The m\textcolor{Maroon}{\bfseries a}n likes his bl\textcolor{Maroon}{\bfseries a}ck c\textcolor{Maroon}{\bfseries a}p.
\end{enumerate}

\hfill{\tiny 0140}\,{\scriptsize \myaudio{./audio/vowel_ae_02.mp3}}

\end{frame}
%%%%%%%%%%%%%%%%%%%%%%%%%%%
\begin{frame}[plain]{Quiz 1}
発音記号\textipa{/\ae /}の音が含まれていたらT、含まれていなければFと答えてください。余裕があれば、なんという単語か書き取ってみましょう

\LARGE
\begin{enumerate}
 \item \mbox{}\onslide<7->{bag\hfill{}\textipa{/b\'\ae g/}}\hspace{40pt}\visible<2->{\textcolor{NavyBlue}{\bfseries T}}\hspace{200pt}\mbox{}
 \item \mbox{}\visible<8->{man\hfill{}\textipa{/m\'\ae n/}}\hspace{40pt}\visible<3->{\textcolor{NavyBlue}{\bfseries T}}\hspace{200pt}\mbox{}
 \item \mbox{}\visible<9->{cut\hfill\textipa{/k\'2t/}}\hspace{40pt}\visible<4->{\textcolor{Maroon}{\bfseries F}}\hspace{200pt}\mbox{}
 \item \mbox{}\visible<10->{boy\hfill\textipa{/b\'OI/}}\hspace{40pt}\visible<5->{\textcolor{Maroon}{\bfseries F}}\hspace{200pt}\mbox{}
 \item \mbox{}\visible<11->{apple\hfill\textipa{/\'\ae pl/}}\hspace{40pt}\visible<6->{\textcolor{NavyBlue}{\bfseries T}}\hspace{200pt}\mbox{}
\end{enumerate}

\hfill{\tiny 0145}\,{\scriptsize \myaudio{./audio/vowel_ae_03.mp3}}

\end{frame}
%%%%%%%%%%%%%%%%%%%%%%%%%%%
\begin{frame}[plain]{Quiz 2}

 発音記号\textipa{/\ae /}の音が含まれていたらT、含まれていなければFと答えてください。余裕があれば、なんという単語か書き取ってみましょう

\LARGE
\begin{enumerate}
 \item \mbox{}\visible<7->{cap\hfill\textipa{/k\'\ae p/}}\hspace{40pt}\visible<2->{\textcolor{NavyBlue}{\bfseries T}}\hspace{160pt}\mbox{}
 \item \mbox{}\visible<8->{cup\hfill\textipa{/k\'2p/}}\hspace{40pt}\visible<3->{\textcolor{Maroon}{\bfseries F}}\hspace{160pt}\mbox{}
 \item \mbox{}\visible<9->{cat\hfill\textipa{/k\'\ae t/}}\hspace{40pt}\visible<4->{\textcolor{NavyBlue}{\bfseries T}}\hspace{160pt}\mbox{}
 \item \mbox{}\visible<10->{desk\hfill\textipa{/d\'esk/}}\hspace{40pt}\visible<5->{\textcolor{Maroon}{\bfseries F}}\hspace{160pt}\mbox{}
 \item \mbox{}\visible<11->{animal\hfill\textipa{/\'\ae n@\@m(@)l/}}\hspace{40pt}\visible<6->{\textcolor{NavyBlue}{\bfseries T}}\hspace{160pt}\mbox{}
\end{enumerate}

\hfill{\tiny 0145}\,{\scriptsize \myaudio{./audio/vowel_ae_04.mp3}}

\end{frame}
%%%%%%%%%%%%%%%%%%%%%%%%%%%
\begin{frame}[plain]{Quiz 3}

 発音記号\textipa{/\ae /}の音が含まれていたらT、含まれていなければFと答えてください。余裕があれば、なんという単語か書き取ってみましょう

\LARGE
\begin{enumerate}
 \item \mbox{}\visible<7->{map\hfill\textipa{/m\'\ae p/}}\hspace{40pt}\visible<2->{\textcolor{NavyBlue}{\bfseries T}}\hspace{160pt}\mbox{}
 \item \mbox{}\visible<8->{hand\hfill\textipa{/h\'\ae nd/}}\hspace{40pt}\visible<3->{\textcolor{NavyBlue}{\bfseries T}}\hspace{160pt}\mbox{}
 \item \mbox{}\visible<9->{come\hfill\textipa{/k\'2m/}}\hspace{40pt}\visible<4->{\textcolor{Maroon}{\bfseries F}}\hspace{160pt}\mbox{}
 \item \mbox{}\visible<10->{mother\hfill\textipa{/m\'2D\textrhookschwa /}}\hspace{40pt}\visible<5->{\textcolor{Maroon}{\bfseries F}}\hspace{160pt}\mbox{}
 \item \mbox{}\visible<11->{happy\hfill\textipa{/h\'\ae pi/}}\hspace{40pt}\visible<6->{\textcolor{NavyBlue}{\bfseries T}}\hspace{160pt}\mbox{}
\end{enumerate}

\hfill{\tiny 0146}\,{\scriptsize \myaudio{./audio/vowel_ae_05.mp3}}

\end{frame}
%%%%%%%%%%%%%%%%%%%%%%%%%%%
% 背景色を黒に変更
\setbeamercolor{background canvas}{bg=black}
\begin{frame}
\centering
  \textcolor{white}{\Huge\bfseries Today's Pronunciation}\pause

 \vspace{30pt}

  \textcolor{white}{\Huge\bfseries \textipa{/\textscripta /}}
\end{frame}
\setbeamercolor{background canvas}{bg=}
%%%%%%%%%%%%%%%%%%%%%%%%%%
\begin{frame}[plain]{\textipa{/\textscripta /}}

\Huge
 \textipa{/\textscripta /}

\vspace*{20pt}

\normalsize
ポイント

\begin{itemize}
 \item 口を縦に大きく開ける
 \item あくびをするイメージ
\end{itemize}

\end{frame}
%%%%%%%%%%%%%%%%%%%%%%%%%%
\begin{frame}[plain]{実際の単語で確認しよう}
\LARGE

\hfill{\tiny 0506}\,{\scriptsize \myaudio{./audio/vowel_textscripta_01.mp3}}

\vspace{-10pt}

\begin{enumerate}
 \item h\textcolor{NavyBlue}{\bfseries o}t%
\hfill\makebox[80pt][l]{\textipa{/h\textcolor{BurntOrange}{\'\textscripta}t/}}\hspace{150pt}\mbox{}
 \item d\textcolor{NavyBlue}{\bfseries o}g%
\hfill\makebox[80pt][l]{\textipa{/d\textcolor{BurntOrange}{\'\textscripta}g/}}\hspace{150pt}\mbox{} 
\item t\textcolor{NavyBlue}{\bfseries o}p%
\hfill\makebox[80pt][l]{\textipa{/t\textcolor{BurntOrange}{\'\textscripta}p/}}\hspace{150pt}\mbox{}
 \item n\textcolor{NavyBlue}{\bfseries o}t%
\hfill\makebox[80pt][l]{\textipa{/n\textcolor{BurntOrange}{\'\textscripta}t/}}\hspace{150pt}\mbox{} \item b\textcolor{NavyBlue}{\bfseries o}x%
\hfill\makebox[80pt][l]{\textipa{/b\textcolor{BurntOrange}{\'\textscripta}ks/}}\hspace{150pt}\mbox{}
 \item b\textcolor{NavyBlue}{\bfseries o}dy%
\hfill\makebox[80pt][l]{\textipa{/b\textcolor{BurntOrange}{\'\textscripta}di/}}\hspace{150pt}\mbox{}
 \item w\textcolor{NavyBlue}{\bfseries a}sh%
\hfill\makebox[80pt][l]{\textipa{/w\textcolor{BurntOrange}{\'\textscripta}\textesh /}}\hspace{150pt}\mbox{}
 \item w\textcolor{NavyBlue}{\bfseries a}nt%
\hfill\makebox[80pt][l]{\textipa{/w\textcolor{BurntOrange}{\'\textscripta}nt/}}\hspace{150pt}\mbox{}

\end{enumerate}

\end{frame}
%%%%%%%%%%%%%%%%%%%%%%%%%%%%%%
\begin{frame}[plain]{Exercises \textipa{/A/}}

\LARGE

\begin{enumerate}
 \item T\textcolor{Maroon}{\bfseries o}m likes h\textcolor{Maroon}{\bfseries o}t d\textcolor{Maroon}{\bfseries o}gs.
 \item M\textcolor{Maroon}{\bfseries o}m has a j\textcolor{Maroon}{\bfseries o}b  at the sh\textcolor{Maroon}{\bfseries o}p.
 \item B\textcolor{Maroon}{\bfseries o}b makes h\textcolor{Maroon}{\bfseries o}t soup in a p\textcolor{Maroon}{\bfseries o}t.
\end{enumerate}

\hfill{\tiny 0141}\,{\scriptsize \myaudio{./audio/vowel_textscripta_02.mp3}}

\end{frame}
%%%%%%%%%%%%%%%%%%%%%%%%%%%%%
%%%%%%%%%%%%%%%%%%%%%%%%%%%
\begin{frame}[plain]{Quiz 1}

 発音記号\textipa{/\textscripta /}の音が含まれていたらT、含まれていなければFと答えてください。余裕があれば、なんという単語か書き取ってみましょう

\LARGE
\begin{enumerate}
 \item \mbox{}\onslide<7->{hot\hfill\textipa{/h\'At/}}\hspace{40pt}\visible<2->{\textcolor{NavyBlue}{\bfseries T}}\hspace{160pt}\mbox{}
 \item \mbox{}\visible<8->{not\hfill\textipa{/n\'At/}}\hspace{40pt}\visible<3->{\textcolor{NavyBlue}{\bfseries T}}\hspace{160pt}\mbox{}
 \item \mbox{}\visible<9->{hat\hfill\textipa{/h\'\ae t/}}\hspace{40pt}\visible<4->{\textcolor{Maroon}{\bfseries F}}\hspace{160pt}\mbox{}
 \item \mbox{}\visible<10->{God\hfill\textipa{/g\'Ad/}}\hspace{40pt}\visible<5->{\textcolor{NavyBlue}{\bfseries T}}\hspace{160pt}\mbox{}
 \item \mbox{}\visible<11->{map\hfill\textipa{/m\'\ae p/}}\hspace{40pt}\visible<6->{\textcolor{Maroon}{\bfseries F}}\hspace{160pt}\mbox{}
\end{enumerate}

\hfill{\tiny 0144}\,{\scriptsize \myaudio{./audio/vowel_textscripta_03.mp3}}

\end{frame}
%%%%%%%%%%%%%%%%%%%%%%%%%%%
\begin{frame}[plain]{Quiz 2}
 発音記号\textipa{/\textscripta /}の音が含まれていたらT、含まれていなければFと答えてください。余裕があれば、なんという単語か書き取ってみましょう

\LARGE
\begin{enumerate}
 \item \mbox{}\visible<7->{black\hfill\textipa{/bl\'\ae k/}}\hspace{40pt}\visible<2->{\textcolor{Maroon}{\bfseries F}}\hspace{160pt}\mbox{}
 \item \mbox{}\onslide<8->{box\hfill\textipa{/b\'Aks/}}\hspace{40pt}\visible<3->{\textcolor{NavyBlue}{\bfseries T}}\hspace{160pt}\mbox{}
 \item \mbox{}\visible<9->{top\hfill\textipa{/t\'Ap/}}\hspace{40pt}\visible<4->{\textcolor{NavyBlue}{\bfseries T}}\hspace{160pt}\mbox{}
 \item \mbox{}\visible<10->{mother\hfill\textipa{/m\'2\dh \textrhookschwa /}}\hspace{40pt}\visible<5->{\textcolor{Maroon}{\bfseries F}}\hspace{160pt}\mbox{}
 \item \mbox{}\visible<11->{pot\hfill\textipa{/p\'At/}}\hspace{40pt}\visible<6->{\textcolor{NavyBlue}{\bfseries T}}\hspace{160pt}\mbox{}
\end{enumerate}

\hfill{\tiny 0145}\,{\scriptsize \myaudio{./audio/vowel_textscripta_04.mp3}}

\end{frame}
%%%%%%%%%%%%%%%%%%%%%%%%%%
\begin{frame}[plain]{Quiz 3}
 発音記号\textipa{/\textscripta /}の音が含まれていたらT、含まれていなければFと答えてください。余裕があれば、なんという単語か書き取ってみましょう

\LARGE
\begin{enumerate}
 \item \mbox{}\visible<7->{job\hfill\textipa{/dZ\'Ab/}}\hspace{40pt}\visible<2->{\textcolor{NavyBlue}{\bfseries T}}\hspace{200pt}\mbox{}
 \item \mbox{}\visible<8->{map\hfill\textipa{/m\'\ae p/}}\hspace{40pt}\visible<3->{\textcolor{Maroon}{\bfseries F}}\hspace{200pt}\mbox{}
 \item \mbox{}\visible<9->{shop\hfill\textipa{/S\'Ap/}}\hspace{40pt}\visible<4->{\textcolor{NavyBlue}{\bfseries T}}\hspace{200pt}\mbox{}
 \item \mbox{}\visible<10->{hand\hfill\textipa{/h\'\ae nd/}}\hspace{40pt}\visible<5->{\textcolor{Maroon}{\bfseries F}}\hspace{200pt}\mbox{}
 \item \mbox{}\visible<11->{wash\hfill\textipa{/w\'AS/}}\hspace{40pt}\visible<6->{\textcolor{NavyBlue}{\bfseries T}}\hspace{200pt}\mbox{}
\end{enumerate}

\hfill{\tiny 0146}\,{\scriptsize \myaudio{./audio/vowel_textscripta_05.mp3}}

\end{frame}
%%%%%%%%%%%%%%%%%%%%%%%%%%%
%%%%%%%%%%%%%%%%%%%%%%%%%%%
% 背景色を黒に変更
\setbeamercolor{background canvas}{bg=black}
\begin{frame}

\centering
  \textcolor{white}{\Huge\bfseries Today's Pronunciation}\pause

 \vspace{30pt}

  \textcolor{white}{\Huge\bfseries \textipa{/\textturnv /}}

\end{frame}
\setbeamercolor{background canvas}{bg=}
%%%%%%%%%%%%%%%%%%%%%%%%%%
\begin{frame}[plain]{\textipa{/\textturnv /}}

\Huge
 \textipa{/\textturnv /}

\vspace*{20pt}

\normalsize
ポイント

\begin{itemize}\setbeamertemplate{items}[circle]
 \item 口をやや半開きで「ア」
\end{itemize}

\end{frame}
%%%%%%%%%%%%%%%%%%%%%%%%%%
\begin{frame}[plain]{実際の単語で確認しよう}

\LARGE
\hfill{\tiny 0509}\,{\scriptsize \myaudio{./audio/vowel_textturnv_01.mp3}}

\begin{enumerate}
 \item f\textcolor{NavyBlue}{\bfseries u}n%
\hfill\makebox[80pt][l]{\textipa{/f\textcolor{BurntOrange}{\'\textturnv}n/}}\hspace{150pt}\mbox{}
 \item c\textcolor{NavyBlue}{\bfseries u}t%
\hfill\makebox[80pt][l]{\textipa{/k\textcolor{BurntOrange}{\'\textturnv}t/}}\hspace{150pt}\mbox{}
 \item b\textcolor{NavyBlue}{\bfseries u}s%
\hfill\makebox[80pt][l]{\textipa{/b\textcolor{BurntOrange}{\'\textturnv}s/}}\hspace{150pt}\mbox{}
 \item l\textcolor{NavyBlue}{\bfseries u}cky%
\hfill\makebox[80pt][l]{\textipa{/l\textcolor{BurntOrange}{\'\textturnv}ki /}}\hspace{150pt}\mbox{}
 \item \textcolor{NavyBlue}{\bfseries u}ncle%
\hfill\makebox[80pt][l]{\textipa{/\textcolor{BurntOrange}{\'\textturnv}Nkl/}}\hspace{150pt}\mbox{}
 \item l\textcolor{NavyBlue}{\bfseries o}ve%
\hfill\makebox[80pt][l]{\textipa{/l\textcolor{BurntOrange}{\'\textturnv}v/}}\hspace{150pt}\mbox{} 
\item c\textcolor{NavyBlue}{\bfseries o}me%
\hfill\makebox[80pt][l]{\textipa{/k\textcolor{BurntOrange}{\'\textturnv}m/}}\hspace{150pt}\mbox{}
 \item m\textcolor{NavyBlue}{\bfseries o}ney%
\hfill\makebox[80pt][l]{\textipa{/m\textcolor{BurntOrange}{\'\textturnv}ni/}}\hspace{150pt}\mbox{}
\end{enumerate}
\end{frame}
%%%%%%%%%%%%%%%%%%%%%%%%%%%%%%
%%%%%%%%%%%%%%%%%%%%%%%%%%%%%%
\begin{frame}[plain]{Exercises \textipa{/2/}}

\LARGE

\begin{enumerate}
 \item My m\textcolor{Maroon}{\bfseries o}ther l\textcolor{Maroon}{\bfseries o}ves the s\textcolor{Maroon}{\bfseries u}mmer s\textcolor{Maroon}{\bfseries u}n.
 \item My br\textcolor{Maroon}{\bfseries o}ther j\textcolor{Maroon}{\bfseries u}mps and r\textcolor{Maroon}{\bfseries u}ns in the s\textcolor{Maroon}{\bfseries u}n.
 \item The c\textcolor{Maroon}{\bfseries u}p is \textcolor{Maroon}{\bfseries u}nder the \textcolor{Maroon}{\bfseries u}mbrella.
\end{enumerate}

\hfill\tiny{0143}\,{\scriptsize \myaudio{./audio/vowel_textturnv_02.mp3}}

\end{frame}
%%%%%%%%%%%%%%%%%%%%%%%%%%%
%%%%%%%%%%%%%%%%%%%%%%%%%%%
\begin{frame}[plain]{Quiz 1}

 発音記号\textipa{/\textturnv /}の音が含まれていたらT、含まれていなければFと答えてください。
余裕があれば、なんという単語か書き取ってみましょう

\LARGE
\begin{enumerate}
 \item \mbox{}\onslide<7->{bag\hfill\textipa{/b\'\ae g/}}\hspace{40pt}\visible<2->{\textcolor{Maroon}{\bfseries F}}\hspace{150pt}\mbox{}
 \item \mbox{}\visible<8->{cut\hfill\textipa{/k\'2t/}}\hspace{40pt}\visible<3->{\textcolor{NavyBlue}{\bfseries T}}\hspace{150pt}\mbox{}
 \item \mbox{}\visible<9->{mother\hfill\textipa{/m\'2\dh\textrhookschwa /}}\hspace{40pt}\visible<4->{\textcolor{NavyBlue}{\bfseries T}}\hspace{150pt}\mbox{}
 \item \mbox{}\visible<10->{dog\hfill\textipa{/d\'Ag/}}\hspace{40pt}\visible<5->{\textcolor{Maroon}{\bfseries F}}\hspace{150pt}\mbox{}
 \item \mbox{}\visible<11->{umbrella\hfill\textipa{/2mbr\'el@/}}\hspace{40pt}\visible<6->{\textcolor{NavyBlue}{\bfseries T}}\hspace{150pt}\mbox{}
\end{enumerate}

\hfill\tiny{0146}\,{\scriptsize \myaudio{./audio/vowel_textturnv_03.mp3}}

\end{frame}
%%%%%%%%%%%%%%%%%%%%%%%%%%%
\begin{frame}[plain]{Quiz 2}

 発音記号\textipa{/\textturnv /}の音が含まれていたらT、含まれていなければFと答えてください。
余裕があれば、なんという単語か書き取ってみましょう

\LARGE
\begin{enumerate}
 \item \mbox{}\visible<7->{brother\hfill\textipa{/br\'2\dh\textrhookschwa /}}\hspace{40pt}\visible<2->{\textcolor{NavyBlue}{\bfseries T}}\hspace{150pt}\mbox{}
 \item \mbox{}\visible<8->{father\hfill\textipa{/f\'A:\dh\textrhookschwa /}}\hspace{40pt}\visible<3->{\textcolor{Maroon}{\bfseries F}}\hspace{150pt}\mbox{}
 \item \mbox{}\visible<9->{under\hfill\textipa{/\'2nd\textrhookschwa /}}\hspace{40pt}\visible<4->{\textcolor{NavyBlue}{\bfseries T}}\hspace{150pt}\mbox{}
 \item \mbox{}\visible<10->{summer\hfill\textipa{/s\'2m\textrhookschwa /}}\hspace{40pt}\visible<5->{\textcolor{NavyBlue}{\bfseries T}}\hspace{150pt}\mbox{}
 \item \mbox{}\visible<11->{family\hfill\textipa{/f\'\ae m(@)li/}}\hspace{40pt}\visible<6->{\textcolor{Maroon}{\bfseries F}}\hspace{150pt}\mbox{}

\end{enumerate}

\hfill\tiny{0147}\,{\scriptsize \myaudio{./audio/vowel_textturnv_04.mp3}}

\end{frame}
%%%%%%%%%%%%%%%%%%%%%%%%%%%
\begin{frame}[plain]{Quiz 3}

 発音記号\textipa{/\textturnv /}の音が含まれていたらT、含まれていなければFと答えてください。
余裕があれば、なんという単語か書き取ってみましょう

\LARGE
\begin{enumerate}
 \item \mbox{}\visible<7->{money\hfill\textipa{/m\'2ni/}}\hspace{40pt}\visible<2->{\textcolor{NavyBlue}{\bfseries T}}\hspace{150pt}\mbox{}
 \item \mbox{}\visible<8->{hot\hfill\textipa{/h\'At/}}\hspace{40pt}\visible<3->{\textcolor{Maroon}{\bfseries F}}\hspace{150pt}\mbox{}
 \item \mbox{}\visible<9->{come\hfill\textipa{/k\'2m/}}\hspace{40pt}\visible<4->{\textcolor{NavyBlue}{\bfseries T}}\hspace{150pt}\mbox{}
 \item \mbox{}\visible<10->{cake\hfill\textipa{/k\'eIk/}}\hspace{40pt}\visible<5->{\textcolor{Maroon}{\bfseries F}}\hspace{150pt}\mbox{}
 \item \mbox{}\visible<11->{lucky\hfill\textipa{/l\'2ki/}}\hspace{40pt}\visible<6->{\textcolor{NavyBlue}{\bfseries T}}\hspace{150pt}\mbox{}
\end{enumerate}
\hfill\tiny{0145}\,{\scriptsize \myaudio{./audio/vowel_textturnv_05.mp3}}
\end{frame}
%%%%%%%%%%%%%%%
%%%%%%%%%%%%%%%%%%%%%%%%%%%
%%%%%%%%%%%%%%%%%%%%%%%%%
% 背景色を黒に変更
\setbeamercolor{background canvas}{bg=black}
\begin{frame}
\centering
  \textcolor{white}{\Huge\bfseries Today's Pronunciation}\pause

 \vspace{30pt}

  \textcolor{white}{\Huge\bfseries \textipa{/@/}}


\end{frame}
\setbeamercolor{background canvas}{bg=}
%%%%%%%%%%%%%%%%%%%%%%%%%%
\begin{frame}[plain]{\textipa{/@/}}

\large
 {\Huge \textipa{/@/}}%
\hspace{30pt}\underLine{a}bout \textipa{/@b\'aUt/}
\hspace{30pt}\underLine{a}gain \textipa{/@g\'en/}
\hspace{30pt}chocol\underLine{a}te \textipa{/tS\'Akl@t/}

\normalsize

\bigskip

ポイント%
\hfill\begin{tikzpicture}
\duck[%tshirt,
cap=NavyBlue!80!black,
glasses,
megaphone,
%jacket=gray,
buttons,
bowtie,
speech={\tiny あいまいな音},
laughing,
signpost={\textipa{/@/}},
bubblecolour=white!30!yellow]
]
\duck[xshift=90pt, scale=.3, yshift=150pt]
\duck[xshift=60pt, scale=.3, yshift=100pt]
\duck[body=gray!50!white, head=gray!50!white,
xshift=80pt, scale=.3, yshift=50pt]
\end{tikzpicture}

\begin{itemize}\setbeamertemplate{items}[circle]
 \item 口の開きがほとんどない「ア」
 \item アクセントがないところ
 \end{itemize}

\hfill{\tiny 0215}\,{\scriptsize \myaudio{./audio/vowel_textschwa_01.mp3}}


\end{frame}
%%%%%%%%%%%%%%%%%%%%%%%%%%
\begin{frame}[plain]{実際の単語で確認しよう}

\LARGE
\hfill{\tiny 0512}\,{\scriptsize \myaudio{./audio/vowel_textschwa_02.mp3}}

\vspace{-\baselineskip}

\begin{enumerate}
 \item \textcolor{NavyBlue}{\bfseries a}bout
\hfill\makebox[80pt][l]{\textipa{/\textcolor{BurntOrange}{@}b\'oUt/}}\hspace{150pt}\mbox{}
 \item \textcolor{NavyBlue}{\bfseries a}gain
\hfill\makebox[80pt][l]{\textipa{/\textcolor{BurntOrange}{@}g\'en/}}\hspace{150pt}\mbox{}
 \item b\textcolor{NavyBlue}{\bfseries a}nan\textcolor{NavyBlue}{\bfseries a}%
\hfill\makebox[80pt][l]{\textipa{/b\textcolor{BurntOrange}{@}n\'\ae n\textcolor{BurntOrange}{@}/}}\hspace{150pt}\mbox{}
 \item tel\textcolor{NavyBlue}{\bfseries e}phone
\hfill\makebox[80pt][l]{\textipa{/t\'el\textcolor{BurntOrange}{@}f\`oUn/}}\hspace{150pt}\mbox{}
 \item Apr\textcolor{NavyBlue}{\bfseries i}l
\hfill\makebox[80pt][l]{\textipa{/\'eIpr\textcolor{BurntOrange}{@}l/}}\hspace{150pt}\mbox{}

 \item lem\textcolor{NavyBlue}{\bfseries o}n
\hfill\makebox[80pt][l]{\textipa{/l\'em\textcolor{BurntOrange}{@}n/}}\hspace{150pt}\mbox{}
 \item p\textcolor{NavyBlue}{\bfseries o}lice
\hfill\makebox[80pt][l]{\textipa{/p\textcolor{BurntOrange}{@}l\'\i:s/}}\hspace{150pt}\mbox{}
 \item alb\textcolor{NavyBlue}{\bfseries u}m
\hfill\makebox[80pt][l]{\textipa{/\'\ae lb\textcolor{BurntOrange}{@}m/}}\hspace{150pt}\mbox{}

\end{enumerate}
\end{frame}
%%%%%%%%%%%%%%%%%%%%%%%%%%%%%%
\begin{frame}[plain]{Quiz 1}

発音記号\textipa{/\textschwa /}の音が含まれていたらT、含まれていなければFと答えてください。余裕があれば、なんという単語か書き取ってみましょう

\LARGE
\begin{enumerate}
 \item \mbox{}\onslide<7->{about\hfill{}\textipa{/@b\'oUt/}}\hspace{40pt}\visible<2->{\textcolor{NavyBlue}{\bfseries T}}\hspace{150pt}\mbox{}
 \item \mbox{}\visible<8->{vegetable\hfill{}\textipa{/v\'edZt@bl/}}\hspace{40pt}\visible<3->{\textcolor{NavyBlue}{\bfseries T}}\hspace{150pt}\mbox{}
 \item \mbox{}\visible<9->{boy\hfill\textipa{/b\'OI/}}\hspace{40pt}\visible<4->{\textcolor{Maroon}{\bfseries F}}\hspace{150pt}\mbox{}
 \item \mbox{}\visible<10->{banana\hfill\textipa{/b@n\'\ae n@/}}\hspace{40pt}\visible<5->{\textcolor{NavyBlue}{\bfseries T}}\hspace{150pt}\mbox{}
 \item \mbox{}\visible<11->{apple\hfill\textipa{/\'\ae pl/}}\hspace{40pt}\visible<6->{\textcolor{Maroon}{\bfseries F}}\hspace{150pt}\mbox{}
\end{enumerate}

\hfill{\tiny 0146}\,{\scriptsize \myaudio{./audio/vowel_textschwa_03.mp3}}

\end{frame}
%%%%%%%%%%%%%%%%%%%%%%%%%%%
\begin{frame}[plain]{Quiz 2}

 発音記号\textipa{/\textschwa /}の音が含まれていたらT、含まれていなければFと答えてください。余裕があれば、なんという単語か書き取ってみましょう

\LARGE
\begin{enumerate}
 \item \mbox{}\visible<7->{job\hfill\textipa{/dZ\'Ab/}}\hspace{40pt}\visible<2->{\textcolor{Maroon}{\bfseries F}}\hspace{200pt}\mbox{}
 \item \mbox{}\visible<8->{again\hfill\textipa{/@g\'en/}}\hspace{40pt}\visible<3->{\textcolor{NavyBlue}{\bfseries T}}\hspace{200pt}\mbox{}
 \item \mbox{}\visible<9->{shop\hfill\textipa{/S\'Ap/}}\hspace{40pt}\visible<4->{\textcolor{Maroon}{\bfseries F}}\hspace{200pt}\mbox{}
 \item \mbox{}\visible<10->{hand\hfill\textipa{/h\'\ae nd/}}\hspace{40pt}\visible<5->{\textcolor{Maroon}{\bfseries F}}\hspace{200pt}\mbox{}
 \item \mbox{}\visible<11->{lemon\hfill\textipa{/l\'em@n/}}\hspace{40pt}\visible<6->{\textcolor{NavyBlue}{\bfseries T}}\hspace{200pt}\mbox{}
\end{enumerate}

\hfill{\tiny 0146}\,{\scriptsize \myaudio{./audio/vowel_textschwa_04.mp3}}

\end{frame}
%%%%%%%%%%%%%%%%%%%%%%%%%%%
%%%%%%%%%%%%%%%%%%%%%%%%%%%
\begin{frame}[plain]{Quiz 3}

発音記号\textipa{/@/}の音が含まれていたらT、含まれていなければFと答えてください。余裕があれば、なんという単語か書き取ってみましょう

\LARGE
\begin{enumerate}
 \item \mbox{}\visible<7->{America\hfill\textipa{/@m\'erIk@/}}\hspace{40pt}\visible<2->{\textcolor{NavyBlue}{\bfseries T}}\hspace{160pt}\mbox{}
 \item \mbox{}\visible<8->{hot\hfill\textipa{/h\'At/}}\hspace{40pt}\visible<3->{\textcolor{Maroon}{\bfseries F}}\hspace{160pt}\mbox{}
 \item \mbox{}\visible<9->{chocolate\hfill\textipa{/tS\'Akl@t/}}\hspace{40pt}\visible<4->{\textcolor{NavyBlue}{\bfseries T}}\hspace{160pt}\mbox{}
 \item \mbox{}\visible<10->{desk\hfill\textipa{/d\'esk/}}\hspace{40pt}\visible<5->{\textcolor{Maroon}{\bfseries F}}\hspace{160pt}\mbox{}
 \item \mbox{}\visible<11->{breakfast\hfill\textipa{/br\'ekf@st/}}\hspace{40pt}\visible<6->{\textcolor{NavyBlue}{\bfseries T}}\hspace{160pt}\mbox{}
\end{enumerate}

\hfill{\tiny 0155}\,{\scriptsize \myaudio{./audio/vowel_textschwa_05.mp3}}

\end{frame}
%%%%%%%%%%%%%%%%%%%%%%%%%%%
\begin{frame}[plain]{「ア」の類の母音}
\Large


\begin{description}
 \item[\textipa{/\ae /}] c\underLine{a}t\hspace{20pt}c\underLine{a}p\hspace{20pt}h\underLine{a}t%
\hfill{}{\small 口を縦横全開にしてエとアの中間}
 \item[\textipa{/\textscripta /}] h\underLine{o}t\hspace{20pt}n\underLine{o}t\hspace{20pt}sh\underLine{o}p%
\hfill{}{\small 口を縦に全開にしてア}
 \item[\textipa{/\textturnv /}] c\underLine{u}p\hspace{20pt}c\underLine{u}t\hspace{20pt}n\underLine{u}t%
\hfill{}{\small 口を半開きでア}
 \item[\textipa{/\textschwa /}] \underLine{a}bout\hspace{20pt}\underLine{a}gain\hspace{20pt}b\underLine{a}nan\underLine{a}%
\hfill{}{\small 口の開きがほとんどないア}
 \end{description}

\hfill{\tiny 0346}\,{\scriptsize \myaudio{./audio/vowel_like_a_01.mp3}}
\end{frame}
%%%%%%%%%%%%%%%%%%%%%%%%
\begin{frame}[plain]{Quiz}\large

cap(帽子)またはcup(茶わん)を発音していきます。どちらを発音したか聞き取って線でつなぎましょう

\bigskip

 \begin{columns}[t]
   \begin{column}{.45\textwidth}
    \begin{tabular}{rlr}
     1& \visible<2->{cup}&\myAnch{q1}{white}{\textbullet} \\
     2& \visible<3->{cap}&\myAnch{q2}{white}{\textbullet} \\
     3& \visible<4->{cap}&\myAnch{q3}{white}{\textbullet} \\
     4& \visible<5->{cup}&\myAnch{q4}{white}{\textbullet} \\
     5& \visible<6->{cap}&\myAnch{q5}{white}{\textbullet} \\
     6& \visible<7->{cup}&\myAnch{q6}{white}{\textbullet} 
    \end{tabular}
   \end{column}
%%%%%%%%%%%
   \begin{column}{.45\textwidth}
    \begin{tabular}{lll}
     \myAnch{a1}{white}{\textbullet}& cap& \textipa{/k\'\ae p/}\\
     &\\
     \myAnch{a2}{white}{\textbullet}& cup& \textipa{/k\'\textturnv p/}  \\
    \end{tabular}
   \end{column}
 \end{columns}

\begin{tikzpicture}[remember picture, overlay]
\tikzset{hoge/.style = {line width=4pt, ->, opacity=.6}}
 \visible<2->{\draw[hoge, Maroon] (q1.east) to[out=0, in=180] (a2.west);}
 \visible<3->{\draw[hoge, NavyBlue] (q2.east) to[out=0, in=180] (a1.west);}
 \visible<4->{\draw[hoge, NavyBlue] (q3.east) to[out=0, in=180] (a1.west);}
 \visible<5->{\draw[hoge, Maroon] (q4.east) to[out=0, in=180] (a2.west);}
 \visible<6->{\draw[hoge, NavyBlue] (q5.east) to[out=0, in=180] (a1.west);}
 \visible<7->{\draw[hoge, Maroon] (q6.east) to[out=0, in=180] (a2.west);}
\end{tikzpicture}

\hfill{\tiny 0147}\,{\scriptsize \myaudio{./audio/vowel_cap_cup_01.mp3}}

\end{frame}
%%%%%%%%%%%%%%%%%%%%%%%%%%
%%%%%%%%%%%%%%%%%%%%%%%%%%
\begin{frame}[plain]{Quiz}\large

hat(帽子)またはhot(暑い)を発音していきます。どちらを発音したか聞き取って線でつなぎましょう

\bigskip

 \begin{columns}[t]
   \begin{column}{.45\textwidth}
    \begin{tabular}{rlr}
     1& \visible<2->{hat}&\myAnch{q1}{white}{\textbullet} \\
     2& \visible<3->{hot}&\myAnch{q2}{white}{\textbullet} \\
     3& \visible<4->{hat}&\myAnch{q3}{white}{\textbullet} \\
     4& \visible<5->{hot}&\myAnch{q4}{white}{\textbullet} \\
     5& \visible<6->{hot}&\myAnch{q5}{white}{\textbullet} \\
     6& \visible<7->{hat}&\myAnch{q6}{white}{\textbullet} 
    \end{tabular}
   \end{column}
%%%%%%%%%%%
   \begin{column}{.45\textwidth}
    \begin{tabular}{lll}
     \myAnch{a1}{white}{\textbullet}& hat &\textipa{/h\'\ae t/}\\
     &\\
     \myAnch{a2}{white}{\textbullet}& hot &\textipa{/h\'\textscripta t/}\\
    \end{tabular}
   \end{column}
 \end{columns}

\begin{tikzpicture}[remember picture, overlay]
\tikzset{hoge/.style = {line width=4pt, ->, opacity=.6}}
 \visible<2->{\draw[hoge, Maroon] (q1.east) to[out=0, in=180] (a1.west);}
 \visible<3->{\draw[hoge, NavyBlue] (q2.east) to[out=0, in=180] (a2.west);}
 \visible<4->{\draw[hoge, Maroon] (q3.east) to[out=0, in=180] (a1.west);}
 \visible<5->{\draw[hoge, NavyBlue] (q4.east) to[out=0, in=180] (a2.west);}
 \visible<6->{\draw[hoge, NavyBlue] (q5.east) to[out=0, in=180] (a2.west);}
 \visible<7->{\draw[hoge, Maroon] (q6.east) to[out=0, in=180] (a1.west);}
\end{tikzpicture}

\hfill{\tiny 0153}\,{\scriptsize \myaudio{./audio/vowel_hat_hot_01.mp3}}

\end{frame}
%%%%%%%%%%%%%%%%%%%%%%%%%%%
%%%%%%%%%%%%%%%%%%%%%%%%%%
\begin{frame}[plain]{Quiz}\large

cat(ネコ)またはcut(切る)を発音していきます。どちらを発音したか聞き取って線でつなぎましょう

\bigskip

 \begin{columns}[t]
   \begin{column}{.45\textwidth}
    \begin{tabular}{rlr}
     1& \visible<2->{cut}&\myAnch{q1}{white}{\textbullet} \\
     2& \visible<3->{cat}&\myAnch{q2}{white}{\textbullet} \\
     3& \visible<4->{cut}&\myAnch{q3}{white}{\textbullet} \\
     4& \visible<5->{cat}&\myAnch{q4}{white}{\textbullet} \\
     5& \visible<6->{cat}&\myAnch{q5}{white}{\textbullet} \\
     6& \visible<7->{cut}&\myAnch{q6}{white}{\textbullet} 
    \end{tabular}
   \end{column}
%%%%%%%%%%%
   \begin{column}{.45\textwidth}
    \begin{tabular}{ll@{  }l}
     \myAnch{a1}{white}{\textbullet}& cut &\textipa{/k\'\textturnv t/}\\
     &\\
     \myAnch{a2}{white}{\textbullet}& cat &\textipa{/k\'\ae t/}\\
    \end{tabular}
   \end{column}
 \end{columns}

\begin{tikzpicture}[remember picture, overlay]
\tikzset{hoge/.style = {line width=4pt, ->, opacity=.6}}
 \visible<2->{\draw[hoge, Maroon] (q1.east) to[out=0, in=180] (a1.west);}
 \visible<3->{\draw[hoge, NavyBlue] (q2.east) to[out=0, in=180] (a2.west);}
 \visible<4->{\draw[hoge, Maroon] (q3.east) to[out=0, in=180] (a1.west);}
 \visible<5->{\draw[hoge, NavyBlue] (q4.east) to[out=0, in=180] (a2.west);}
 \visible<6->{\draw[hoge, NavyBlue] (q5.east) to[out=0, in=180] (a2.west);}
 \visible<7->{\draw[hoge, Maroon] (q6.east) to[out=0, in=180] (a1.west);}
\end{tikzpicture}

\hfill{\tiny 0153}\,{\scriptsize \myaudio{./audio/vowel_cat_cut_01.mp3}}


\end{frame}
%%%%%%%%%%%%%%%%%%%%%%%%
\begin{frame}[plain]{Quiz}\large

not(否定語)またはnut(ナッツ、木の実)を発音していきます。どちらを発音したか聞き取って線でつなぎましょう

\bigskip

 \begin{columns}[t]
   \begin{column}{.45\textwidth}
    \begin{tabular}{rlr}
     1& \visible<2->{nut}&\myAnch{q1}{white}{\textbullet} \\
     2& \visible<3->{not}&\myAnch{q2}{white}{\textbullet} \\
     3& \visible<4->{not}&\myAnch{q3}{white}{\textbullet} \\
     4& \visible<5->{nut}&\myAnch{q4}{white}{\textbullet} \\
     5& \visible<6->{not}&\myAnch{q5}{white}{\textbullet} \\
     6& \visible<7->{nut}&\myAnch{q6}{white}{\textbullet} 
    \end{tabular}
   \end{column}
%%%%%%%%%%%
   \begin{column}{.45\textwidth}
    \begin{tabular}{lll}
     \myAnch{a1}{white}{\textbullet}& not& \textipa{/n\'\textscripta t/}\\
     &\\
     \myAnch{a2}{white}{\textbullet}& nut& \textipa{/n\'\textturnv t/}  \\
    \end{tabular}
   \end{column}
 \end{columns}

\begin{tikzpicture}[remember picture, overlay]
\tikzset{hoge/.style = {line width=4pt, ->, opacity=.6}}
 \visible<2->{\draw[hoge, Maroon] (q1.east) to[out=0, in=180] (a2.west);}
 \visible<3->{\draw[hoge, NavyBlue] (q2.east) to[out=0, in=180] (a1.west);}
 \visible<4->{\draw[hoge, NavyBlue] (q3.east) to[out=0, in=180] (a1.west);}
 \visible<5->{\draw[hoge, Maroon] (q4.east) to[out=0, in=180] (a2.west);}
 \visible<6->{\draw[hoge, NavyBlue] (q5.east) to[out=0, in=180] (a1.west);}
 \visible<7->{\draw[hoge, Maroon] (q6.east) to[out=0, in=180] (a2.west);}
\end{tikzpicture}

\hfill{\tiny 0154}\,{\scriptsize \myaudio{./audio/vowel_not_nut_01.mp3}}

\end{frame}
%%%%%%%%%%%%%%%%%%%%%%%%%%
%%%%%%%%%%%%%%%%%%%%%%%%%%%
\begin{frame}[plain]{「イ」の類の母音}
\Large

\begin{description}
 \item[\textipa{/i:/}]<2->
\hspace{50pt}\visible<3->{{\underLine{ea}t\hspace{20pt}tr\underLine{ee}\hspace{20pt}p\underLine{eo}ple}}
 \item[\textipa{/I/}]<2->
\hspace{50pt}\visible<4->{{w\underLine{i}nter\hspace{20pt}b\underLine{i}g\hspace{20pt}p\underLine{i}cture}}
 \end{description}

\vfill

\hfill%
\visible<2->{\begin{tikzpicture}
\duck[%tshirt,
cap=NavyBlue!80!black,
glasses,
megaphone,
%jacket=gray,
buttons,
bowtie,
speech={\tiny 2つだけ},
laughing,
signpost={{\scriptsize イの類}},
bubblecolour=white!30!yellow]
]
\duck[xshift=90pt, scale=.3, yshift=150pt]
\duck[xshift=60pt, scale=.3, yshift=100pt]
\duck[body=gray!50!white, head=gray!50!white,
xshift=80pt, scale=.3, yshift=50pt]
\end{tikzpicture}}

\hfill{\tiny 0150}\,{\scriptsize \myaudio{./audio/vowel_like_i_01.mp3}}

\end{frame}
%%%%%%%%%%%%%%%%%%%%%%%%%%%%%%%%%%%%%%%%%%%%%%%%%%%
%%%%%%%%%%%%%%%%%%%%%%%%%%%
% 背景色を黒に変更
\setbeamercolor{background canvas}{bg=black}
\begin{frame}
\centering
  \textcolor{white}{\Huge\bfseries Today's Pronunciation}\pause

 \vspace{30pt}

  \textcolor{white}{\Huge\bfseries \textipa{/i:/}}

\end{frame}
\setbeamercolor{background canvas}{bg=}
%%%%%%%%%%%%%%%%%%%%%%%%%%
\begin{frame}[plain]{\textipa{/i:/}}

\LARGE
{\normalsize 長母音} \textipa{/i:/}%
\hspace{20pt}\underLine{ea}t%
\hspace{10pt}\textipa{/\'\i:t/}%
\hspace{73.75pt}cf.\hspace{15pt}%
{\normalsize 短母音}\textipa{/I/}%
\hspace{20pt}
\underLine{i}t%
\hspace{10pt}\textipa{/\'It/}%

\hspace{80pt}s\underLine{ea}t%
\hspace{10pt}\textipa{/s\'\i:t/}%
\hspace{167.5pt}s\underLine{i}t%
\hspace{10pt}\textipa{/s\'It/}

\vspace*{20pt}

\normalsize
長母音\textipa{/i:/の}ポイント

\begin{itemize}\setbeamertemplate{items}[circle]
 \item 唇を左右に大きく広げ「イー」
 \item 日本語の「イー」で通じます
\end{itemize}

\vspace{-50pt}

\hfill{\begin{tikzpicture}
\duck[squareglasses=blue!50!black,
speech={\tiny 「イー」でOK},
laughing
]
\end{tikzpicture}}

\hfill{\tiny 0232}\,{\scriptsize \myaudio{./audio/vowel_long_i_00.mp3}}

\end{frame}
%%%%%%%%%%%%%%%%%%%%%%%%%%
\begin{frame}[plain]{長母音 \textipa{/i:/} 実際の単語で確認しよう}

\LARGE
\hfill{\tiny 0508}\,{\scriptsize \myaudio{./audio/vowel_long_i_01.mp3}}

\vspace{-15pt}

\begin{enumerate}
 \item h\textcolor{NavyBlue}{\bfseries e}%
\hfill\makebox[80pt][l]{\textipa{/h\textcolor{BurntOrange}{\'\i:}/}}\hspace{150pt}\mbox{}

\item t\textcolor{NavyBlue}{\bfseries ea}%
\hfill\makebox[80pt][l]{\textipa{/t\textcolor{BurntOrange}{\'\i:}/}}\hspace{150pt}\mbox{}
 \item sh\textcolor{NavyBlue}{\bfseries e}%
\hfill\makebox[80pt][l]{\textipa{/S\textcolor{BurntOrange}{\'\i:}/}}\hspace{150pt}\mbox{}
 \item s\textcolor{NavyBlue}{\bfseries ea}%
\hfill\makebox[80pt][l]{\textipa{/s\textcolor{BurntOrange}{\'\i:}/}}\hspace{150pt}\mbox{}
 \item t\textcolor{NavyBlue}{\bfseries ea}m%
\hfill\makebox[80pt][l]{\textipa{/t\textcolor{BurntOrange}{\'\i:}m/}}\hspace{150pt}\mbox{}
 \item tr\textcolor{NavyBlue}{\bfseries ee}%
\hfill\makebox[80pt][l]{\textipa{/tr\textcolor{BurntOrange}{\'\i:}/}}\hspace{150pt}\mbox{}
 \item gr\textcolor{NavyBlue}{\bfseries ee}n%
\hfill\makebox[80pt][l]{\textipa{/gr\textcolor{BurntOrange}{\'\i:}n/}}\hspace{150pt}\mbox{}
 \item p\textcolor{NavyBlue}{\bfseries eo}ple%
\hfill\makebox[80pt][l]{\textipa{/p\textcolor{BurntOrange}{\'\i:}pl/}}\hspace{150pt}\mbox{}

\end{enumerate}
\end{frame}
%%%%%%%%%%%%%%%%%%%%%%%%%%%%%%
%%%%%%%%%%%%%%%%%%%%%%%%%%%%%%
\begin{frame}[plain]{Exercises \textipa{/i:/}}
\LARGE

\begin{enumerate}
 \item The t\textcolor{Maroon}{\bfseries ea}cher s\textcolor{Maroon}{\bfseries ee}s thr\textcolor{Maroon}{\bfseries ee} gr\textcolor{Maroon}{\bfseries ee}n tr\textcolor{Maroon}{\bfseries ee}s.
 \item We m\textcolor{Maroon}{\bfseries ee}t at the b\textcolor{Maroon}{\bfseries ea}ch every w\textcolor{Maroon}{\bfseries ee}k.
 \item Sh\textcolor{Maroon}{\bfseries e} \textcolor{Maroon}{\bfseries ea}ts sw\textcolor{Maroon}{\bfseries ee}t p\textcolor{Maroon}{\bfseries ea}ches.
\end{enumerate}
\hfill{\tiny 0142}\,{\scriptsize \myaudio{./audio/vowel_long_i_02.mp3}}

\end{frame}
%%%%%%%%%%%%%%%%%%%%%%%%%%%%%
%%%%%%%%%%%%%%%%%%%%%%%%%%%
\begin{frame}[plain]{Quiz 1 \textipa{/i:/}}

発音記号\textipa{/i:/}の音が含まれていたらT、含まれていなければFと答えてください。余裕があれば、なんという単語か書き取ってみましょう

\LARGE
\begin{enumerate}
 \item \mbox{}\onslide<7->{speak\hfill\textipa{/sp\'\i{}:k/}}\hspace{40pt}\visible<2->{\textcolor{NavyBlue}{\bfseries T}}\hspace{150pt}\mbox{}
 \item \mbox{}\visible<8->{week\hfill\textipa{/w\'\i{}:k/}}\hspace{40pt}\visible<3->{\textcolor{NavyBlue}{\bfseries T}}\hspace{150pt}\mbox{}
 \item \mbox{}\visible<9->{English\hfill\textipa{/\'I\ng glIS/}}\hspace{40pt}\visible<4->{\textcolor{Maroon}{\bfseries F}}\hspace{150pt}\mbox{}
 \item \mbox{}\visible<10->{it\hfill\textipa{/\'It/}}\hspace{40pt}\visible<5->{\textcolor{Maroon}{\bfseries F}}\hspace{150pt}\mbox{}
 \item \mbox{}\visible<11->{clean\hfill\textipa{/kl\'\i{}:n/}}\hspace{40pt}\visible<6->{\textcolor{NavyBlue}{\bfseries T}}\hspace{150pt}\mbox{}
\end{enumerate}
\hfill{\tiny 0145}\,{\scriptsize \myaudio{./audio/vowel_long_i_03.mp3}}

\end{frame}
%%%%%%%%%%%%%%%%%%%%%%%%%%%
\begin{frame}[plain]{Quiz 2}

 発音記号\textipa{/i:/}の音が含まれていたらT、含まれていなければFと答えてください。余裕があれば、なんという単語か書き取ってみましょう

\LARGE
\begin{enumerate}
 \item \mbox{}\onslide<7->{people\hfill\textipa{/p\'\i{}:pl/}}\hspace{40pt}\visible<2->{\textcolor{NavyBlue}{\bfseries T}}\hspace{200pt}\mbox{}
 \item \mbox{}\visible<8->{ski\hfill\textipa{/sk\'\i{}:/}}\hspace{40pt}\visible<3->{\textcolor{NavyBlue}{\bfseries T}}\hspace{200pt}\mbox{}
 \item \mbox{}\visible<9->{river\hfill\textipa{/r\'Iv\textrhookschwa /}}\hspace{40pt}\visible<4->{\textcolor{Maroon}{\bfseries F}}\hspace{200pt}\mbox{}
 \item \mbox{}\visible<10->{big\hfill\textipa{/b\'Ig/}}\hspace{40pt}\visible<5->{\textcolor{Maroon}{\bfseries F}}\hspace{200pt}\mbox{}
 \item \mbox{}\visible<11->{read\hfill\textipa{/r\'\i{}:d/}}\hspace{40pt}\visible<6->{\textcolor{NavyBlue}{\bfseries T}}\hspace{200pt}\mbox{}
\end{enumerate}
\hfill{\tiny 0146}\,{\scriptsize \myaudio{./audio/vowel_long_i_04.mp3}}

\end{frame}
%%%%%%%%%%%%%%%%%%%%%%%%%%%
\begin{frame}[plain]{Quiz 3}

 発音記号\textipa{/i:/}の音が含まれていたらT、含まれていなければFと答えてください

\LARGE
\begin{enumerate}
 \item \mbox{}\visible<7->{fish\hfill\textipa{/f\'IS/}}\hspace{40pt}\visible<2->{\textcolor{Maroon}{\bfseries F}}\hspace{150pt}\mbox{}
 \item \mbox{}\onslide<8->{peach\hfill\textipa{/p\'\i{}:tS/}}\hspace{40pt}\visible<3->{\textcolor{NavyBlue}{\bfseries T}}\hspace{150pt}\mbox{}
 \item \mbox{}\visible<9->{green\hfill\textipa{/gr\'\i{}:n/}}\hspace{40pt}\visible<4->{\textcolor{NavyBlue}{\bfseries T}}\hspace{150pt}\mbox{}
 \item \mbox{}\visible<10->{dinner\hfill\textipa{/d\'In\textrhookschwa /}}\hspace{40pt}\visible<5->{\textcolor{Maroon}{\bfseries F}}\hspace{150pt}\mbox{}
 \item \mbox{}\visible<11->{teacher\hfill\textipa{/t\'\i{}:tS\textrhookschwa /}}\hspace{40pt}\visible<6->{\textcolor{NavyBlue}{\bfseries T}}\hspace{150pt}\mbox{}
\end{enumerate}

\hfill{\tiny 0146}\,{\scriptsize \myaudio{./audio/vowel_long_i_05.mp3}}

\end{frame}
%%%%%%%%%%%%%%%%%%%%%%%%%%%

%%%%%%%%%%%%%%%%%%%%%%%%%%%
% 背景色を黒に変更
\setbeamercolor{background canvas}{bg=black}
\begin{frame}
\hypertarget{today}{}

\centering
  \textcolor{white}{\Huge\bfseries Today's Pronunciation}\pause

 \vspace{30pt}

  \textcolor{white}{\Huge\bfseries \textipa{/I/}}
\end{frame}
\setbeamercolor{background canvas}{bg=}
%%%%%%%%%%%%%%%%%%%%%%%%%%
\begin{frame}[plain]{短母音 \textipa{/I/}}

\Huge
 \textipa{/I/}

\vspace*{20pt}

\normalsize
ポイント

\begin{itemize}\setbeamertemplate{items}[circle]
 \item\<「エ」と「イ」の中間\,\,\,$\longrightarrow$「エ」の口で「イ」
 \item 短めに
\end{itemize}

\vspace{-50pt}

\hfill{\begin{tikzpicture}
\duck[squareglasses=blue!50!black,
speech={\tiny エの口でイ},
laughing
]
\end{tikzpicture}}

\hfill\hyperlink{ex}{\beamergotobutton{Today's Exercises}}

\end{frame}
%%%%%%%%%%%%%%%%%%%%%%%%%%
%%%%%%%%%%%%%%%%%%%%%%%%%%
\begin{frame}[plain]{短母音\textipa{/I/}\,\,\,実際の単語で確認しよう}

\LARGE
\hfill{\tiny 0508}\,{\scriptsize \myaudio{./audio/vowel_short_i_01.mp3}}

\vspace{-10pt}

\begin{enumerate}
 \item b\textcolor{NavyBlue}{\bfseries i}g%
\hfill\makebox[80pt][l]{\textipa{/b\textcolor{BurntOrange}{\'I}g/}}\hspace{150pt}\mbox{}
 \item f\textcolor{NavyBlue}{\bfseries i}sh%
\hfill\makebox[80pt][l]{\textipa{/f\textcolor{BurntOrange}{\'I}S/}}\hspace{150pt}\mbox{}
 \item r\textcolor{NavyBlue}{\bfseries i}ver%
\hfill\makebox[80pt][l]{\textipa{/r\textcolor{BurntOrange}{\'I}v\textrhookschwa /}}\hspace{150pt}\mbox{}
 \item dr\textcolor{NavyBlue}{\bfseries i}nk%
\hfill\makebox[80pt][l]{\textipa{/dr\textcolor{BurntOrange}{\'I}Nk/}}\hspace{150pt}\mbox{}
 \item mus\textcolor{NavyBlue}{\bfseries i}c%
\hfill\makebox[80pt][l]{\textipa{/mj\'u:z\textcolor{BurntOrange}{I}k/}}\hspace{150pt}\mbox{} 
\item k\textcolor{NavyBlue}{\bfseries i}tchen%
\hfill\makebox[80pt][l]{\textipa{/k\textcolor{BurntOrange}{\'I}tS@n/}}\hspace{150pt}\mbox{}
 \item p\textcolor{NavyBlue}{\bfseries i}cture%
\hfill\makebox[80pt][l]{\textipa{/p\textcolor{BurntOrange}{\'I}ktS\textrhookschwa /}}\hspace{150pt}\mbox{}
 \item w\textcolor{NavyBlue}{\bfseries i}nter%
\hfill\makebox[80pt][l]{\textipa{/w\textcolor{BurntOrange}{\'I}nt\textrhookschwa /}}\hspace{150pt}\mbox{}
\end{enumerate}
\end{frame}
%%%%%%%%%%%%%%%%%%%%%%%%%%%%%%
%%%%%%%%%%%%%%%%%%%%%%%%%%%%%%
\begin{frame}[plain]{Exercises \textipa{/I/}}

\LARGE

\begin{enumerate}
 \item The b\textcolor{Maroon}{\bfseries i}g sh\textcolor{Maroon}{\bfseries i}p \textcolor{Maroon}{\bfseries i}s \textcolor{Maroon}{\bfseries i}n the r\textcolor{Maroon}{\bfseries i}ver.
 \item My s\textcolor{Maroon}{\bfseries i}ster s\textcolor{Maroon}{\bfseries i}ts and s\textcolor{Maroon}{\bfseries i}ngs \textcolor{Maroon}{\bfseries i}n the b\textcolor{Maroon}{\bfseries i}g k\textcolor{Maroon}{\bfseries i}tchen.
 \item The k\textcolor{Maroon}{\bfseries i}ds v\textcolor{Maroon}{\bfseries i}s\textcolor{Maroon}{\bfseries i}t the c\textcolor{Maroon}{\bfseries i}ty \textcolor{Maroon}{\bfseries i}n spr\textcolor{Maroon}{\bfseries i}ng and w\textcolor{Maroon}{\bfseries i}nter.

\end{enumerate}
\hfill{\tiny 0146}\,{\scriptsize \myaudio{./audio/vowel_short_i_02.mp3}}

\end{frame}
%%%%%%%%%%%%%%%%%%%%%%%%%%%%%
%%%%%%%%%%%%%%%%%%%%%%%%%%%
\begin{frame}[plain]{Quiz 1}

発音記号\textipa{/I/}の音が含まれていたらT、含まれていなければFと答えてください。余裕があれば、なんという単語か書き取ってみましょう

\LARGE
\begin{enumerate}
 \item \mbox{}\onslide<7->{river\hfill\textipa{/r\'Iv\textrhookschwa /}}\hspace{40pt}\visible<2->{\textcolor{NavyBlue}{\bfseries T}}\hspace{150pt}\mbox{}
 \item \mbox{}\visible<8->{spring\hfill\textipa{/spr\'I\ng /}}\hspace{40pt}\visible<3->{\textcolor{NavyBlue}{\bfseries T}}\hspace{150pt}\mbox{}
 \item \mbox{}\visible<9->{beach\hfill\textipa{/b\'\i{}:tS/}}\hspace{40pt}\visible<4->{\textcolor{Maroon}{\bfseries F}}\hspace{150pt}\mbox{}
 \item \mbox{}\visible<10->{eat\hfill\textipa{/\'\i{}:t/}}\hspace{40pt}\visible<5->{\textcolor{Maroon}{\bfseries F}}\hspace{150pt}\mbox{}
 \item \mbox{}\visible<11->{picture\hfill\textipa{/p\'IktS\textrhookschwa /}}\hspace{40pt}\visible<6->{\textcolor{NavyBlue}{\bfseries T}}\hspace{150pt}\mbox{}
\end{enumerate}
\hfill{\tiny 0145}\,{\scriptsize \myaudio{./audio/vowel_short_i_03.mp3}}

\end{frame}
%%%%%%%%%%%%%%%%%%%%%%%%%%%
\begin{frame}[plain]{Quiz 2}
 発音記号\textipa{/I/}の音が含まれていたらT、含まれていなければFと答えてください。余裕があれば、なんという単語か書き取ってみましょう

\LARGE
\begin{enumerate}
 \item \mbox{}\visible<7->{blue\hfill\textipa{/bl\'u:/}}\hspace{40pt}\visible<2->{\textcolor{Maroon}{\bfseries F}}\hspace{150pt}\mbox{}
 \item \mbox{}\onslide<8->{sister\hfill\textipa{/s\'Ist\textrhookschwa /}}\hspace{40pt}\visible<3->{\textcolor{NavyBlue}{\bfseries T}}\hspace{150pt}\mbox{}
 \item \mbox{}\visible<9->{winter\hfill\textipa{/w\'Int\textrhookschwa /}}\hspace{40pt}\visible<4->{\textcolor{NavyBlue}{\bfseries T}}\hspace{150pt}\mbox{}
 \item \mbox{}\visible<10->{green\hfill\textipa{/gr\'\i{}:n/}}\hspace{40pt}\visible<5->{\textcolor{Maroon}{\bfseries F}}\hspace{150pt}\mbox{}
 \item \mbox{}\visible<11->{picture\hfill\textipa{/p\'IktS\textrhookschwa /}}\hspace{40pt}\visible<6->{\textcolor{NavyBlue}{\bfseries T}}\hspace{150pt}\mbox{}
\end{enumerate}
\hfill{\tiny 0146}\,{\scriptsize \myaudio{./audio/vowel_short_i_04.mp3}}

\end{frame}
%%%%%%%%%%%%%%%%%%%%%%%%%%%
\begin{frame}[plain]{Quiz 3}

 発音記号\textipa{/I/}の音が含まれていたらT、含まれていなければFと答えてください。余裕があれば、なんという単語か書き取ってみましょう

\LARGE
\begin{enumerate}
 \item \mbox{}\onslide<7->{ship\hfill\textipa{/S\'Ip/}}\hspace{40pt}\visible<2->{\textcolor{NavyBlue}{\bfseries T}}\hspace{150pt}\mbox{}
 \item \mbox{}\visible<8->{visit\hfill\textipa{/v\'IzIt/}}\hspace{40pt}\visible<3->{\textcolor{NavyBlue}{\bfseries T}}\hspace{150pt}\mbox{}
 \item \mbox{}\visible<9->{tree\hfill\textipa{/tr\'\i{}:/}}\hspace{40pt}\visible<4->{\textcolor{Maroon}{\bfseries F}}\hspace{150pt}\mbox{}
 \item \mbox{}\visible<10->{read\hfill\textipa{/r\'\i{}:d/}}\hspace{40pt}\visible<5->{\textcolor{Maroon}{\bfseries F}}\hspace{150pt}\mbox{}
 \item \mbox{}\visible<11->{kitchen\hfill\textipa{/k\'ItS@n/}}\hspace{40pt}\visible<6->{\textcolor{NavyBlue}{\bfseries T}}\hspace{150pt}\mbox{}
\end{enumerate}
\hfill{\tiny 0145}\,{\scriptsize \myaudio{./audio/vowel_short_i_05.mp3}}

\end{frame}
%%%%%%%%%%%%%%%%%%%%%%%%%%%
\begin{frame}[plain]{Quiz 4}\large
\hypertarget{ex}{}

it(それ)またはeat(たべる)を発音していきます。どちらを発音したか聞き取って線でつなぎましょう

\bigskip

 \begin{columns}[t]
   \begin{column}{.45\textwidth}
    \begin{tabular}{rlr}
     1& \visible<2->{it}&\myAnch{q1}{white}{\textbullet} \\
     2& \visible<3->{eat}&\myAnch{q2}{white}{\textbullet} \\
     3& \visible<4->{it}&\myAnch{q3}{white}{\textbullet} \\
     4& \visible<5->{eat}&\myAnch{q4}{white}{\textbullet} \\
     5& \visible<6->{eat}&\myAnch{q5}{white}{\textbullet} \\
     6& \visible<7->{it}&\myAnch{q6}{white}{\textbullet} 
    \end{tabular}
   \end{column}
%%%%%%%%%%%
   \begin{column}{.45\textwidth}
    \begin{tabular}{lll}
     \myAnch{a1}{white}{\textbullet}& it &\textipa{/\'It/}\\
     &\\
     \myAnch{a2}{white}{\textbullet}& eat &\textipa{/\'\i:t/}\\
    \end{tabular}
   \end{column}
 \end{columns}

\begin{tikzpicture}[remember picture, overlay]
\tikzset{hoge/.style = {line width=4pt, ->, opacity=.6}}
 \visible<2->{\draw[hoge, Maroon] (q1.east) to[out=0, in=180] (a1.west);}
 \visible<3->{\draw[hoge, NavyBlue] (q2.east) to[out=0, in=180] (a2.west);}
 \visible<4->{\draw[hoge, Maroon] (q3.east) to[out=0, in=180] (a1.west);}
 \visible<5->{\draw[hoge, NavyBlue] (q4.east) to[out=0, in=180] (a2.west);}
 \visible<6->{\draw[hoge, NavyBlue] (q5.east) to[out=0, in=180] (a2.west);}
 \visible<7->{\draw[hoge, Maroon] (q6.east) to[out=0, in=180] (a1.west);}
\end{tikzpicture}

\hfill{\tiny 0135}\,{\scriptsize \myaudio{./audio/vowel_it_eat_01.mp3}}

\end{frame}
%%%%%%%%%%%%%%%%%%%%%%%%%%%
%%%%%%%%%%%%%%%%%%%%%%%%%%
%%%%%%%%%%%%%%%%%%%%%%%%%%
\begin{frame}[plain]{Quiz 5}\large

ship(船)またはsheep(ヒツジ)を発音していきます。どちらを発音したか聞き取って線でつなぎましょう

\bigskip

 \begin{columns}[t]
   \begin{column}{.45\textwidth}
    \begin{tabular}{rlr}
     1& \visible<2->{sheep}&\myAnch{q1}{white}{\textbullet} \\
     2& \visible<3->{ship}&\myAnch{q2}{white}{\textbullet} \\
     3& \visible<4->{sheep}&\myAnch{q3}{white}{\textbullet} \\
     4& \visible<5->{ship}&\myAnch{q4}{white}{\textbullet} \\
     5& \visible<6->{ship}&\myAnch{q5}{white}{\textbullet} \\
     6& \visible<7->{sheep}&\myAnch{q6}{white}{\textbullet} 
    \end{tabular}
   \end{column}
%%%%%%%%%%%
   \begin{column}{.45\textwidth}
    \begin{tabular}{ll@{  }l}
     \myAnch{a1}{white}{\textbullet}& sheep &\textipa{/S\'i:p/}\\
     &\\
     \myAnch{a2}{white}{\textbullet}& ship &\textipa{/S\'Ip/}\\
    \end{tabular}
   \end{column}
 \end{columns}

\begin{tikzpicture}[remember picture, overlay]
\tikzset{hoge/.style = {line width=4pt, ->, opacity=.6}}
 \visible<2->{\draw[hoge, Maroon] (q1.east) to[out=0, in=180] (a1.west);}
 \visible<3->{\draw[hoge, NavyBlue] (q2.east) to[out=0, in=180] (a2.west);}
 \visible<4->{\draw[hoge, Maroon] (q3.east) to[out=0, in=180] (a1.west);}
 \visible<5->{\draw[hoge, NavyBlue] (q4.east) to[out=0, in=180] (a2.west);}
 \visible<6->{\draw[hoge, NavyBlue] (q5.east) to[out=0, in=180] (a2.west);}
 \visible<7->{\draw[hoge, Maroon] (q6.east) to[out=0, in=180] (a1.west);}
\end{tikzpicture}

\hfill{\tiny 0140}\,{\scriptsize \myaudio{./audio/vowel_ship_sheep_01.mp3}}
\end{frame}
%%%%%%%%%%%%%%%%%%%%%%%%
%%%%%%%%%%%%%%%%%%%%%%%%
\begin{frame}[plain]{Quiz 6}\large

seat(座席)またはsit(座る)を発音していきます。どちらを発音したか聞き取って線でつなぎましょう

\bigskip

 \begin{columns}[t]
   \begin{column}{.45\textwidth}
    \begin{tabular}{rlr}
     1& \visible<2->{sit}&\myAnch{q1}{white}{\textbullet} \\
     2& \visible<3->{seat}&\myAnch{q2}{white}{\textbullet} \\
     3& \visible<4->{seat}&\myAnch{q3}{white}{\textbullet} \\
     4& \visible<5->{sit}&\myAnch{q4}{white}{\textbullet} \\
     5& \visible<6->{seat}&\myAnch{q5}{white}{\textbullet} \\
     6& \visible<7->{sit}&\myAnch{q6}{white}{\textbullet} 
    \end{tabular}
   \end{column}
%%%%%%%%%%%
   \begin{column}{.45\textwidth}
    \begin{tabular}{lll}
     \myAnch{a1}{white}{\textbullet}& seat& \textipa{/s\'\i:t/}\\
     &\\
     \myAnch{a2}{white}{\textbullet}& sit& \textipa{/s\'It/}  \\
    \end{tabular}
   \end{column}
 \end{columns}

\begin{tikzpicture}[remember picture, overlay]
\tikzset{hoge/.style = {line width=4pt, ->, opacity=.6}}
 \visible<2->{\draw[hoge, Maroon] (q1.east) to[out=0, in=180] (a2.west);}
 \visible<3->{\draw[hoge, NavyBlue] (q2.east) to[out=0, in=180] (a1.west);}
 \visible<4->{\draw[hoge, NavyBlue] (q3.east) to[out=0, in=180] (a1.west);}
 \visible<5->{\draw[hoge, Maroon] (q4.east) to[out=0, in=180] (a2.west);}
 \visible<6->{\draw[hoge, NavyBlue] (q5.east) to[out=0, in=180] (a1.west);}
 \visible<7->{\draw[hoge, Maroon] (q6.east) to[out=0, in=180] (a2.west);}
\end{tikzpicture}

\hfill{\tiny 0139}\,{\scriptsize \myaudio{./audio/vowel_seat_sit_01.mp3}}

\end{frame}
%%%%%%%%%%%%%%%%%%%%%%%%%%
%%%%%%%%%%%%%%%%%%%%%%%%%%%
%%%%%%%%%%%%%%%%%%%%%%%%%%%
\begin{frame}[plain]{「ウ」の類の母音}
\Large

\begin{description}
 \item[長母音 \textipa{/u:/}]<3->
\hspace{50pt}\visible<3->{{r\underLine{oo}m\hspace{20pt}f\underLine{oo}d\hspace{20pt}j\underLine{ui}ce}\hspace{20pt}bl\underLine{ue}}
 \item[短母音\textipa{/U/}]<4->
\hspace{58pt}\visible<4->{{b\underLine{oo}k\hspace{20pt}g\underLine{oo}d\hspace{20pt}c\underLine{oo}k}\hspace{20pt}p\underLine{u}t}
 \end{description}

\vfill

\hfill%
\visible<2->{\begin{tikzpicture}
\duck[%tshirt,
cap=NavyBlue!80!black,
glasses,
megaphone,
%jacket=gray,
buttons,
bowtie,
speech={\tiny 2つだけ},
laughing,
signpost={{\scriptsize ウの類}},
bubblecolour=white!30!yellow]
]
\duck[xshift=90pt, scale=.3, yshift=150pt]
\duck[xshift=60pt, scale=.3, yshift=100pt]
\duck[body=gray!50!white, head=gray!50!white,
xshift=80pt, scale=.3, yshift=50pt]
\end{tikzpicture}}

\hfill{\tiny 0211}\,{\scriptsize \myaudio{./audio/vowel_like_u_01.mp3}}

\end{frame}
%%%%%%%%%%%%%%%%%%%%%%%%%%%%%%%%%%%%%%%%%%%%%%%%%%%%%%%%%%%%%%%%%%%%%%%%%%%%%%
% 背景色を黒に変更
\setbeamercolor{background canvas}{bg=black}
\begin{frame}
\centering
  \textcolor{white}{\Huge\bfseries Today's Pronunciation}\pause

 \vspace{30pt}

  \textcolor{white}{\Huge\bfseries \textipa{/u:/}}
\end{frame}
\setbeamercolor{background canvas}{bg=}
%%%%%%%%%%%%%%%%%%%%%%%%%%
\begin{frame}[plain]{長母音 \textipa{/u:/}}

{\Large 長母音} {\Huge\textipa{/u:/}}

\vspace*{20pt}

\normalsize
ポイント

\begin{itemize}
 \item 唇をしっかり丸めて
 \item 長めに
\end{itemize}
\end{frame}
%%%%%%%%%%%%%%%%%%%%%%%%%%
%%%%%%%%%%%%%%%%%%%%%%%%%%
\begin{frame}[plain]{実際の単語で確認しよう}
\LARGE
\hfill{\tiny 0508}\,{\scriptsize \myaudio{./audio/vowel_long_u_01.mp3}}

\begin{enumerate}
 \item c\textcolor{NavyBlue}{\bfseries oo}l%
\hfill\makebox[80pt][l]{\textipa{/k\textcolor{BurntOrange}{\'u:}l/}}\hspace{150pt}\mbox{}
 \item f\textcolor{NavyBlue}{\bfseries oo}d%
\hfill\makebox[80pt][l]{\textipa{/f\textcolor{BurntOrange}{\'u:}d/}}\hspace{150pt}\mbox{}
\item r\textcolor{NavyBlue}{\bfseries oo}m%
\hfill\makebox[80pt][l]{\textipa{/r\textcolor{BurntOrange}{\'u:}m/}}\hspace{150pt}\mbox{}
 \item m\textcolor{NavyBlue}{\bfseries oo}n%
\hfill\makebox[80pt][l]{\textipa{/m\textcolor{BurntOrange}{\'u:}n/}}\hspace{150pt}\mbox{}
 \item tw\textcolor{NavyBlue}{\bfseries o}%
\hfill\makebox[80pt][l]{\textipa{/t\textcolor{BurntOrange}{\'u:}/}}\hspace{150pt}\mbox{}
 \item bl\textcolor{NavyBlue}{\bfseries u}e%
\hfill\makebox[80pt][l]{\textipa{/bl\textcolor{BurntOrange}{\'u:}/}}\hspace{150pt}\mbox{} 
 \item s\textcolor{NavyBlue}{\bfseries ou}p%
\hfill\makebox[80pt][l]{\textipa{/s\textcolor{BurntOrange}{\'u:}p/}}\hspace{150pt}\mbox{}
 \item gr\textcolor{NavyBlue}{\bfseries ou}p%
\hfill\makebox[80pt][l]{\textipa{/gr\textcolor{BurntOrange}{\'u:}p/}}\hspace{150pt}\mbox{}
\end{enumerate}
\end{frame}
%%%%%%%%%%%%%%%%%%%%%%%%%%%%%%
\begin{frame}[plain]{Exercises \textipa{/u:/}}
\LARGE
\begin{enumerate}
 \item The m\textcolor{Maroon}{\bfseries oo}n is bl\textcolor{Maroon}{\bfseries u}e and c\textcolor{Maroon}{\bfseries oo}l.
 \item She eats n\textcolor{Maroon}{\bfseries oo}dles and drinks j\textcolor{Maroon}{\bfseries ui}ce.
 \item The r\textcolor{Maroon}{\bfseries oo}m is full of bl\textcolor{Maroon}{\bfseries u}e sh\textcolor{Maroon}{\bfseries oe}s.

\end{enumerate}
\hfill{\tiny 0143}\,{\scriptsize \myaudio{./audio/vowel_long_u_02.mp3}}

\end{frame}
%%%%%%%%%%%%%%%%%%%%%%%%%%%
\begin{frame}[plain]{Quiz 1}

 発音記号\textipa{/u:/}の音が含まれていたらT、含まれていなければFと答えてください。余裕があれば、なんという単語か書き取ってみましょう

\LARGE
\begin{enumerate}
 \item \mbox{}\onslide<7->{food\hfill\textipa{/f\'u:d/}}\hspace{40pt}\visible<2->{\textcolor{NavyBlue}{\bfseries T}}\hspace{150pt}\mbox{}
 \item \mbox{}\visible<8->{two\hfill\textipa{/t\'u:/}}\hspace{40pt}\visible<3->{\textcolor{NavyBlue}{\bfseries T}}\hspace{150pt}\mbox{}
 \item \mbox{}\visible<9->{cook\hfill\textipa{/k\'Uk/}}\hspace{40pt}\visible<4->{\textcolor{Maroon}{\bfseries F}}\hspace{150pt}\mbox{}
 \item \mbox{}\visible<10->{book\hfill\textipa{/b\'Uk/}}\hspace{40pt}\visible<5->{\textcolor{Maroon}{\bfseries F}}\hspace{150pt}\mbox{}
 \item \mbox{}\visible<11->{soup\hfill\textipa{/s\'u:p/}}\hspace{40pt}\visible<6->{\textcolor{NavyBlue}{\bfseries T}}\hspace{150pt}\mbox{}
\end{enumerate}
\hfill{\tiny 0145}\,{\scriptsize \myaudio{./audio/vowel_long_u_03.mp3}}

\end{frame}
%%%%%%%%%%%%%%%%%%%%%%%%%%%
%%%%%%%%%%%%%%%%%%%%%%%%%%%
\begin{frame}[plain]{Quiz 2}

 発音記号\textipa{/u:/}の音が含まれていたらT、含まれていなければFと答えてください。余裕があれば、なんという単語か書き取ってみましょう

\LARGE
\begin{enumerate}
 \item \mbox{}\visible<7->{push\hfill\textipa{/p\'US/}}\hspace{40pt}\visible<2->{\textcolor{Maroon}{\bfseries F}}\hspace{150pt}\mbox{}
 \item \mbox{}\onslide<8->{moon\hfill\textipa{/m\'u:n/}}\hspace{40pt}\visible<3->{\textcolor{NavyBlue}{\bfseries T}}\hspace{150pt}\mbox{}
 \item \mbox{}\visible<9->{group\hfill\textipa{/gr\'u:p/}}\hspace{40pt}\visible<4->{\textcolor{NavyBlue}{\bfseries T}}\hspace{150pt}\mbox{}
 \item \mbox{}\visible<10->{sugar\hfill\textipa{/S\'Ug\textrhookschwa /}}\hspace{40pt}\visible<5->{\textcolor{Maroon}{\bfseries F}}\hspace{150pt}\mbox{}
 \item \mbox{}\visible<11->{afternoon\hfill\textipa{/\ae ft\textrhookschwa n\'u:n/}}\hspace{40pt}\visible<6->{\textcolor{NavyBlue}{\bfseries T}}\hspace{150pt}\mbox{}
\end{enumerate}
\hfill{\tiny 0147}\,{\scriptsize \myaudio{./audio/vowel_long_u_04.mp3}}

\end{frame}
%%%%%%%%%%%%%%%%%%%%%%%%%%%
\begin{frame}[plain]{Quiz 3}

 発音記号\textipa{/u:/}の音が含まれていたらT、含まれていなければFと答えてください。余裕があれば、なんという単語か書き取ってみましょう

\LARGE
\begin{enumerate}
 \item \mbox{}\onslide<7->{blue\hfill\textipa{/bl\'u:/}}\hspace{40pt}\visible<2->{\textcolor{NavyBlue}{\bfseries T}}\hspace{150pt}\mbox{}
 \item \mbox{}\visible<8->{school\hfill\textipa{/sk\'u:l/}}\hspace{40pt}\visible<3->{\textcolor{NavyBlue}{\bfseries T}}\hspace{150pt}\mbox{}
 \item \mbox{}\visible<9->{good\hfill\textipa{/g\'Ud/}}\hspace{40pt}\visible<4->{\textcolor{Maroon}{\bfseries F}}\hspace{150pt}\mbox{}
 \item \mbox{}\visible<10->{look\hfill\textipa{/l\'Uk/}}\hspace{40pt}\visible<5->{\textcolor{Maroon}{\bfseries F}}\hspace{150pt}\mbox{}
 \item \mbox{}\visible<11->{shoes\hfill\textipa{/S\'u:z/}}\hspace{40pt}\visible<6->{\textcolor{NavyBlue}{\bfseries T}}\hspace{150pt}\mbox{}
\end{enumerate}
\hfill{\tiny 0145}\,{\scriptsize \myaudio{./audio/vowel_long_u_05.mp3}}

\end{frame}
%%%%%%%%%%%%%%%%%%%%%%%%%%%
%%%%%%%%%%%%%%%%%%%%%%%%%%%
% 背景色を黒に変更
\setbeamercolor{background canvas}{bg=black}
\begin{frame}
\centering
  \textcolor{white}{\Huge\bfseries Today's Pronunciation}\pause

 \vspace{30pt}

  \textcolor{white}{\Huge\bfseries \textipa{/U/}}
\end{frame}
\setbeamercolor{background canvas}{bg=}
%%%%%%%%%%%%%%%%%%%%%%%%%%
\begin{frame}[plain]{短母音 \textipa{/U/}}

 {\Large 短母音} {\Huge \textipa{/U/}}

\vspace*{20pt}

\normalsize
ポイント

\begin{itemize}
 \item \<「オ」の口で「ウ」
 \item 力を抜いて短く(ぞんざいに)
\end{itemize}
\end{frame}
%%%%%%%%%%%%%%%%%%%%%%%%%%
\begin{frame}[plain]{実際の単語で確認しよう}
\LARGE
\hfill{\tiny 0506}\,{\scriptsize \myaudio{./audio/vowel_short_u_01.mp3}}

\begin{enumerate}
 \item b\textcolor{NavyBlue}{\bfseries oo}k%
\hfill\makebox[80pt][l]{\textipa{/b\textcolor{BurntOrange}{\'U}k/}}\hspace{150pt}\mbox{}
 \item c\textcolor{NavyBlue}{\bfseries oo}k%
\hfill\makebox[80pt][l]{\textipa{/k\textcolor{BurntOrange}{\'U}k/}}\hspace{150pt}\mbox{}
\item f\textcolor{NavyBlue}{\bfseries oo}t%
\hfill\makebox[80pt][l]{\textipa{/f\textcolor{BurntOrange}{\'U}t/}}\hspace{150pt}\mbox{}
 \item l\textcolor{NavyBlue}{\bfseries oo}k%
\hfill\makebox[80pt][l]{\textipa{/l\textcolor{BurntOrange}{\'U}k/}}\hspace{150pt}\mbox{}
 \item g\textcolor{NavyBlue}{\bfseries oo}d%
\hfill\makebox[80pt][l]{\textipa{/g\textcolor{BurntOrange}{\'U}d/}}\hspace{150pt}\mbox{}
 \item w\textcolor{NavyBlue}{\bfseries o}man%
\hfill\makebox[80pt][l]{\textipa{/w\textcolor{BurntOrange}{\'U}m@n/}}\hspace{150pt}\mbox{} 
 \item p\textcolor{NavyBlue}{\bfseries u}sh%
\hfill\makebox[80pt][l]{\textipa{/p\textcolor{BurntOrange}{\'U}S/}}\hspace{150pt}\mbox{}
 \item s\textcolor{NavyBlue}{\bfseries u}gar%
\hfill\makebox[80pt][l]{\textipa{/S\textcolor{BurntOrange}{\'U}g\textrhookschwa /}}\hspace{150pt}\mbox{}
\end{enumerate}
\end{frame}
%%%%%%%%%%%%%%%%%%%%%%%%%%%%%%
\begin{frame}[plain]{Exercises \textipa{/U/}}
\LARGE
\begin{enumerate}
 \item The w\textcolor{Maroon}{\bfseries o}man l\textcolor{Maroon}{\bfseries oo}ks at the b\textcolor{Maroon}{\bfseries oo}k.
 \item The c\textcolor{Maroon}{\bfseries oo}k p\textcolor{Maroon}{\bfseries u}ts some g\textcolor{Maroon}{\bfseries oo}d s\textcolor{Maroon}{\bfseries u}gar in the c\textcolor{Maroon}{\bfseries oo}kies.
\end{enumerate}
\hfill{\tiny 0118}\,{\scriptsize \myaudio{./audio/vowel_short_u_02.mp3}}

\end{frame}
%%%%%%%%%%%%%%%%%%%%%%%%%%%
\begin{frame}[plain]{Quiz 1}

 発音記号\textipa{/U/}の音が含まれていたらT、含まれていなければFと答えてください。余裕があれば、なんという単語か書き取ってみましょう

\LARGE
\begin{enumerate}
 \item \mbox{}\onslide<7->{foot\hfill\textipa{/f\'Ut/}}\hspace{40pt}\visible<2->{\textcolor{NavyBlue}{\bfseries T}}\hspace{150pt}\mbox{}
 \item \mbox{}\visible<8->{good\hfill\textipa{/g\'Ud/}}\hspace{40pt}\visible<3->{\textcolor{NavyBlue}{\bfseries T}}\hspace{150pt}\mbox{}
 \item \mbox{}\visible<9->{soup\hfill\textipa{/s\'u:p/}}\hspace{40pt}\visible<4->{\textcolor{Maroon}{\bfseries F}}\hspace{150pt}\mbox{}
 \item \mbox{}\visible<10->{two\hfill\textipa{/t\'u:/}}\hspace{40pt}\visible<5->{\textcolor{Maroon}{\bfseries F}}\hspace{150pt}\mbox{}
 \item \mbox{}\visible<11->{sugar\hfill\textipa{/S\'Ug\textrhookschwa /}}\hspace{40pt}\visible<6->{\textcolor{NavyBlue}{\bfseries T}}\hspace{150pt}\mbox{}
\end{enumerate}
\hfill{\tiny 0108}\,{\scriptsize \myaudio{./audio/vowel_short_u_03.mp3}}

\end{frame}
%%%%%%%%%%%%%%%%%%%%%%%%%%%
\begin{frame}[plain]{Quiz 2}

 発音記号\textipa{/U/}の音が含まれていたらT、含まれていなければFと答えてください。余裕があれば、なんという単語か書き取ってみましょう

\LARGE
\begin{enumerate}
 \item \mbox{}\onslide<7->{woman\hfill\textipa{/w\'Um@n /}}\hspace{40pt}\visible<2->{\textcolor{NavyBlue}{\bfseries T}}\hspace{150pt}\mbox{}
 \item \mbox{}\visible<8->{book\hfill\textipa{/b\'Uk/}}\hspace{40pt}\visible<3->{\textcolor{NavyBlue}{\bfseries T}}\hspace{150pt}\mbox{}
 \item \mbox{}\visible<9->{shoes\hfill\textipa{/S\'u:z/}}\hspace{40pt}\visible<4->{\textcolor{Maroon}{\bfseries F}}\hspace{150pt}\mbox{}
 \item \mbox{}\visible<10->{look\hfill\textipa{/l\'Uk/}}\hspace{40pt}\visible<5->{\textcolor{NavyBlue}{\bfseries T}}\hspace{150pt}\mbox{}
 \item \mbox{}\visible<11->{cookie\hfill\textipa{/k\'Uki/}}\hspace{40pt}\visible<6->{\textcolor{NavyBlue}{\bfseries T}}\hspace{150pt}\mbox{}
\end{enumerate}
\hfill{\tiny 0108}\,{\scriptsize \myaudio{./audio/vowel_short_u_04.mp3}}

\end{frame}
%%%%%%%%%%%%%%%%%%%%%%%%%%%
\begin{frame}[plain]{Quiz 3}

 発音記号\textipa{/U/}の音が含まれていたらT、含まれていなければFと答えてください。余裕があれば、なんという単語か書き取ってみましょう

\LARGE
\begin{enumerate}
 \item \mbox{}\visible<7->{group\hfill\textipa{/gr\'u:p/}}\hspace{40pt}\visible<2->{\textcolor{Maroon}{\bfseries F}}\hspace{150pt}\mbox{}
 \item \mbox{}\onslide<8->{put\hfill\textipa{/p\'Ut/}}\hspace{40pt}\visible<3->{\textcolor{NavyBlue}{\bfseries T}}\hspace{150pt}\mbox{}
 \item \mbox{}\visible<9->{push\hfill\textipa{/p\'US/}}\hspace{40pt}\visible<4->{\textcolor{NavyBlue}{\bfseries T}}\hspace{150pt}\mbox{}
 \item \mbox{}\visible<10->{cool\hfill\textipa{/k\'u:l/}}\hspace{40pt}\visible<5->{\textcolor{Maroon}{\bfseries F}}\hspace{150pt}\mbox{}
 \item \mbox{}\visible<11->{wood\hfill\textipa{/w\'Ud/}}\hspace{40pt}\visible<6->{\textcolor{NavyBlue}{\bfseries T}}\hspace{150pt}\mbox{}
\end{enumerate}
\hfill{\tiny 0107}\,{\scriptsize \myaudio{./audio/vowel_short_u_05.mp3}}

\end{frame}
%%%%%%%%%%%%%%%%%%%%%%%%
\begin{frame}[plain]{Quiz}\large

pool(プール)またはpull(ひっぱる)を発音していきます。どちらを発音したか聞き取って線でつなぎましょう

\bigskip

 \begin{columns}[t]
   \begin{column}{.45\textwidth}
    \begin{tabular}{rlr}
     1& \visible<2->{pull}&\myAnch{q1}{white}{\textbullet} \\
     2& \visible<3->{pool}&\myAnch{q2}{white}{\textbullet} \\
     3& \visible<4->{pool}&\myAnch{q3}{white}{\textbullet} \\
     4& \visible<5->{pull}&\myAnch{q4}{white}{\textbullet} \\
     5& \visible<6->{pool}&\myAnch{q5}{white}{\textbullet} \\
     6& \visible<7->{pull}&\myAnch{q6}{white}{\textbullet} 
    \end{tabular}
   \end{column}
%%%%%%%%%%%
   \begin{column}{.45\textwidth}
    \begin{tabular}{lll}
     \myAnch{a1}{white}{\textbullet}& pool& \textipa{/p\'u:l/}\\
     &\\
     \myAnch{a2}{white}{\textbullet}& pull& \textipa{/p\'Ul/}  \\
    \end{tabular}
   \end{column}
 \end{columns}

\begin{tikzpicture}[remember picture, overlay]
\tikzset{hoge/.style = {line width=4pt, ->, opacity=.6}}
 \visible<2->{\draw[hoge, Maroon] (q1.east) to[out=0, in=180] (a2.west);}
 \visible<3->{\draw[hoge, NavyBlue] (q2.east) to[out=0, in=180] (a1.west);}
 \visible<4->{\draw[hoge, NavyBlue] (q3.east) to[out=0, in=180] (a1.west);}
 \visible<5->{\draw[hoge, Maroon] (q4.east) to[out=0, in=180] (a2.west);}
 \visible<6->{\draw[hoge, NavyBlue] (q5.east) to[out=0, in=180] (a1.west);}
 \visible<7->{\draw[hoge, Maroon] (q6.east) to[out=0, in=180] (a2.west);}
\end{tikzpicture}

\hfill{\tiny 0140}\,{\scriptsize \myaudio{./audio/vowel_pool_pull_01.mp3}}

\hfill{\scriptsize He \textbf{pull}ed the rope. 彼はロープを引っ張った}
\end{frame}
%%%%%%%%%%%%%%%%%%%%%%%%%%
%%%%%%%%%%%%%%%%%%%%%%%%%%
\begin{frame}[plain]{Quiz}\large

full(いっぱいの)またはfool(愚か者)を発音していきます。どちらを発音したか聞き取って線でつなぎましょう

\bigskip

 \begin{columns}[t]
   \begin{column}{.45\textwidth}
    \begin{tabular}{rlr}
     1& \visible<2->{fool}&\myAnch{q1}{white}{\textbullet} \\
     2& \visible<3->{full}&\myAnch{q2}{white}{\textbullet} \\
     3& \visible<4->{fool}&\myAnch{q3}{white}{\textbullet} \\
     4& \visible<5->{full}&\myAnch{q4}{white}{\textbullet} \\
     5& \visible<6->{full}&\myAnch{q5}{white}{\textbullet} \\
     6& \visible<7->{fool}&\myAnch{q6}{white}{\textbullet} 
    \end{tabular}
   \end{column}
%%%%%%%%%%%
   \begin{column}{.45\textwidth}
    \begin{tabular}{ll@{  }l}
     \myAnch{a1}{white}{\textbullet}& fool &\textipa{/f\'u:l/}\\
     &\\
     \myAnch{a2}{white}{\textbullet}& full &\textipa{/f\'Ul/}\\
    \end{tabular}
   \end{column}
 \end{columns}

\begin{tikzpicture}[remember picture, overlay]
\tikzset{hoge/.style = {line width=4pt, ->, opacity=.6}}
 \visible<2->{\draw[hoge, Maroon] (q1.east) to[out=0, in=180] (a1.west);}
 \visible<3->{\draw[hoge, NavyBlue] (q2.east) to[out=0, in=180] (a2.west);}
 \visible<4->{\draw[hoge, Maroon] (q3.east) to[out=0, in=180] (a1.west);}
 \visible<5->{\draw[hoge, NavyBlue] (q4.east) to[out=0, in=180] (a2.west);}
 \visible<6->{\draw[hoge, NavyBlue] (q5.east) to[out=0, in=180] (a2.west);}
 \visible<7->{\draw[hoge, Maroon] (q6.east) to[out=0, in=180] (a1.west);}
\end{tikzpicture}

\hfill{\tiny 0141}\,{\scriptsize \myaudio{./audio/vowel_full_fool_01.mp3}}

\hfill{\scriptsize The case was \textbf{full} of clothes. そのケースは衣類でいっぱいだった}
\end{frame}
%%%%%%%%%%%%%%%%%%%%%%%%





%%%%%%%%%%%%%%%%%%%%%%%%%%%
\begin{frame}[plain]{図解}
\centering
\scalebox{1.4}{%
\begin{tikzpicture}
%\draw[color=gray] (-2,-2) grid (2,2);
 \node (table)
    {\begin{vowel}
    \putcvowel[l]{i:}{1}
    \putcvowel[l]{e}{2}
    \putcvowel[l]{\textepsilon}{3}
    \putcvowel[l]{\textscripta}{5}
    \putcvowel[r]{\textturnv}{12}
    \putcvowel[r]{\textopeno}{6}
    %\putcvowel[r]{o}{7}
    \putcvowel[r]{u:}{8}
    \putcvowel{\textschwa}{11}
    \putcvowel{\textsci}{13}
    \putcvowel{\textupsilon}{14}
    \putcvowel{\ae}{16}
    \end{vowel}
};
\draw[<->, line width=3pt, color=Maroon!50,opacity=.75] (-2,-1.5) -- node[below,font=\scriptsize]{舌が高くなる位置} (2,-1.5);
\draw[<->, line width=3pt, color=NavyBlue!50,opacity=.75] (-2.5,-1.5) -- node[above,font=\scriptsize,sloped]{開口度}(-2.5,1.7);
\node at (-2,-2) {\scriptsize 前};
\node at (2,-2) {\scriptsize 後};
\node at (-3,1.6) {\scriptsize 小};
\node at (-3,-1.25) {\scriptsize 大};
\end{tikzpicture}
}

\end{frame}
\end{document}
