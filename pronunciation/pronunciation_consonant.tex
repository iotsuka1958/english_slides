\documentclass[aspectratio=169,xcolor={dvipsnames,table}]{beamer}
\usepackage[no-math,deluxe,haranoaji]{luatexja-preset}
\renewcommand{\kanjifamilydefault}{\gtdefault}
\renewcommand{\emph}[1]{{\upshape\bfseries #1}}
\usetheme{metropolis}
\metroset{block=fill}
%%%%%%%%%%%%%%%%%%%%%%%%%%
\setbeamertemplate{navigation symbols}{}
\usecolortheme[rgb={0.7,0.2,0.2}]{structure}
%%%%%%%%%%%%%%%%%%%%%%%%%%
%% Change alert block colors
%%% 1- Block title (background and text)
\setbeamercolor{block title alerted}{fg=mDarkTeal, bg=mLightBrown!45!yellow!45}
\setbeamercolor{block title example}{fg=magenta!10!black, bg=mLightGreen!70}
%%% 2- Block body (background)
\setbeamercolor{block body alerted}{bg=mLightBrown!25}
\setbeamercolor{block body example}{bg=mLightGreen!15}
%%%%%%%%%%%%%%%%%%%%%%%%%%%
\usepackage[absolute,overlay]{textpos}
%\usepackage[grid=true,gridcolor=Maroon,subgridcolor=gray,gridunit=pt,texcoord]{eso-pic} %場所決めのためのgrid表示
%%%%%%%%%%%%%%%%%%%%%%%%%%%
%% さまざまなアイコン
%%%%%%%%%%%%%%%%%%%%%%%%%%%
%\usepackage{fontawesome}
\usepackage{fontawesome5}
\usepackage{figchild}
\usepackage{twemojis}
\usepackage{utfsym}
\usepackage{bclogo}
\usepackage{marvosym}
\usepackage{fontmfizz}
\usepackage{pifont}
\usepackage{phaistos}
\usepackage{worldflags}
\usepackage{jigsaw}
\usepackage{tikzlings}
\usepackage{tikzducks}
\usepackage{scsnowman}
\usepackage{epsdice}
\usepackage{halloweenmath}
\usepackage{svrsymbols}
\usepackage{countriesofeurope}
\usepackage{tipa}
\usepackage{manfnt}
%%%%%%%%%%%%%%%%%%%%%%%%%%%
\usepackage{tikz}
\usetikzlibrary{calc,patterns,decorations.pathmorphing,backgrounds}
\usepackage{tcolorbox}
\usepackage{tikzpeople}
\usepackage{circledsteps}
\usepackage{xcolor}
\usepackage{amsmath}
\usepackage{booktabs}
\usepackage{chronology}
\usepackage{signchart}
%%%%%%%%%%%%%%%%%%%%%%%%%%%
%% 場合分け
%%%%%%%%%%%%%%%%%%%%%%%%%%%
\usepackage{cases}
%%%%%%%%%%%%%%%%%%%%%%%%%%
\usepackage{pdfpages}
%%%%%%%%%%%%%%%%%%%%%%%%%%%
%% 音声リンク表示
\newcommand{\myaudio}[1]{\href{#1}{\faVolumeUp}}
%%%%%%%%%%%%%%%%%%%%%%%%%%
%% \myAnch{<名前>}{<色>}{<テキスト>}
%% 指定のテキストを指定の色の四角枠で囲み, 指定の名前をもつTikZの
%% ノードとして出力する. 図には remember picture 属性を付けている
%% ので外部から参照可能である.
\newcommand*{\myAnch}[3]{%
  \tikz[remember picture,baseline=(#1.base)]
    \node[draw,rectangle,line width=1pt,#2] (#1) {\normalcolor #3};
}
%%%%%%%%%%%%%%%%%%%%%%%%%%
%% \myEmph コマンドの定義
%%%%%%%%%%%%%%%%%%%%%%%%%%
%\newcommand{\myEmph}[3]{%
%    \textbf<#1>{\color<#1>{#2}{#3}}%
%}
\usepackage{xparse} % xparseパッケージの読み込み
\NewDocumentCommand{\myEmph}{O{} m m}{%
    \def\argOne{#1}%
    \ifx\argOne\empty
        \textbf{\color{#2}{#3}}% オプション引数が省略された場合
    \else
        \textbf<#1>{\color<#1>{#2}{#3}}% オプション引数が指定された場合
    \fi
}
%%%%%%%%%%%%%%%%%%%%%%%%%%%
%%%%%%%%%%%%%%%%%%%%%%%%%%%
%% 文末の上昇イントネーション記号 \myRisingPitch
%% 通常のイントネーション \myDownwardPitch
%% https://note.com/dan_oyama/n/n8be58e8797b2
%%%%%%%%%%%%%%%%%%%%%%%%%%%
\newcommand{\myRisingPitch}{
\begin{tikzpicture}[scale=0.3,baseline=0.3]
\draw[->,>=stealth] (0,0) to[bend right=45] (1,1);
\end{tikzpicture}
}
\newcommand{\myDownwardPitch}{
\begin{tikzpicture}[scale=0.3,baseline=0.3]
\draw[->,>=stealth] (0,1) to[bend left=45] (1,0);
\end{tikzpicture}
}
%%%%%%%%%%%%%%%%%%%%%%%%%%%%
%\AtBeginSection[%
%]{%
%  \begin{frame}[plain]\frametitle{授業の流れ}
%     \tableofcontents[currentsection]
%   \end{frame}%
%}

\usepackage{lua-ul}
\usepackage{pxrubrica}
\usepackage{soup}
\usepackage{circledsteps}
\usepackage{tikzducks}
\usetikzlibrary{decorations.pathmorphing}
\usetikzlibrary{ducks}
\usepackage{scsnowman}
\usetikzlibrary{tikzmark}
\usetikzlibrary{backgrounds}
%%%%%%%%%%%%%%%%%%%%%%%%%%
\usepackage{multimedia}
%%%%%%%%%%%%%%%%%%%%%%%%%%%
\makeatletter
\newcommand*{\themonth}{\two@digits\month}
\newcommand*{\theday}{\two@digits\day}
\makeatother
\newcommand{\mytoday}{{\the\year}--{\themonth}--{\theday}}
%%%%%%%%%%%%%%%%%%%%%%%%%%
\title{English is fun.}
\subtitle{Pronunciation---consonant---}
\author{}
\institute[]{}
\date[]

%%%%%%%%%%%%%%%%%%%%%%%%%%%%
%% TEXT
%%%%%%%%%%%%%%%%%%%%%%%%%%%%
\begin{document}
%%%%%%%%%%%%%%%%%%%%%%
%%%%%%%%%%%%%%%%%%%%%%%%%%%%%%%%%%%%%%%%%%%%%%%%%%%%%
% 背景色をグレイに変更
%\setbeamercolor{background canvas}{bg=gray}
\setbeamercolor{background canvas}{bg=black}
\begin{frame}
%\centering
\raggedleft
  \textcolor{white}{\Huge\bfseries English is fun.}

\vfill

\raggedleft
% \textcolor{white}{\LARGE\bfseries 2024--11--26}
% \textcolor{white}{\LARGE\bfseries \today}
 \textcolor{white}{\LARGE\bfseries \mytoday}

\vfill
\vfill
\vfill

\raggedleft
\textcolor{white}{\large The lesson will begin at the scheduled time.}

%\textcolor{white}{\large 可能なら、鏡を用意してください}
\end{frame}
\setbeamercolor{background canvas}{bg=}
%%%%%%%%%%%%%%%%%%%%%%%%%%
%%%%%%%%%%%%%%%%%%%
%%% youtube
%%%%%%%%%%%%%%%%%%%%%%%%%%%%%%%%%%%%%%%%%%%%%%%%%%%%%
% 背景色をグレイに変更
%\setbeamercolor{background canvas}{bg=gray}
\setbeamercolor{background canvas}{bg=black}

\begin{frame}
%\centering
\raggedleft
  \textcolor{white}{\Huge\bfseries \textcolor{yellow}{E}nglish is fun.}

\vfill

\vfill

\raggedleft
 \textcolor{white}{\LARGE\bfseries \textcolor{yellow}{H}ello, everybody!}

 \textcolor{white}{\LARGE\bfseries \textcolor{yellow}{H}ow are you today?}

\raggedleft
 \textcolor{white}{\LARGE\bfseries \textcolor{yellow}{A}re you ready to start?}

 \textcolor{white}{\LARGE\bfseries \textcolor{yellow}{L}et's begin today's lesson.}

\vfill

\raggedleft
% \textcolor{white}{\LARGE\bfseries 2024--11--26}
% \textcolor{white}{\LARGE\bfseries \today}
 \textcolor{white}{\Large \bfseries \mytoday}

\hyperlink{today}{\beamergotobutton{Today's Pronunciation}}

%\pause
%\textcolor{white}{$20250919=13\times{}1,557,763$}
\end{frame}
\setbeamercolor{background canvas}{bg=}
%%%%%%%%%%%%%%%%%%%%%%%%%%
%%%%%%%%%%%%%%%%%%%%%%%%%%%%%%%%%%%%%%%%%%%
\begin{frame}[plain]

\hspace*{-22pt}
\includegraphics[width=1.1\textwidth]{../../misc/heat_map/heatmap-crop.pdf}
\end{frame}

%%%%%%%%%%%%%%%%%%%%%
% 背景色をグレイに変更
%\setbeamercolor{background canvas}{bg=gray}
%\begin{frame}
%\centering
%\raggedleft
%  \textcolor{white}{\Huge\bfseries Hello, everybody!}
%
%\vfill
%
%\raggedleft
% \textcolor{white}{\Huge\bfseries How are you today?}
%\vfill
%
%\raggedleft
% \textcolor{white}{\Huge\bfseries Are you ready to start?}
%\vfill
%\vfill
%\vfill
%\raggedleft
% \textcolor{white}{\Large\bfseries 2024--11--26}
% \textcolor{white}{\Large\bfseries \mytoday}
%
%\hypertarget{top_page}{}
%\hyperlink{today}{\beamergotobutton{Today's Pronunciation}}
%\end{frame}
%\setbeamercolor{background canvas}{bg=}
%%%%%%%%%%%%%%%%%
%%%%%%%%%%%%%%%%%%%%%%%%%%%%%
% 背景色を黒に変更
%\setbeamercolor{background canvas}{bg=black}
%\begin{frame}
%\centering
%  \textcolor{white}{\Huge\bfseries The Fourth of July}\pause
%
% \vspace{30pt}
%
%  \textcolor{white}{\Huge\bfseries Independence Day}
%
%\end{frame}
%%%%%%%%%%%%%%%%%%%%%%%%%%%%%%%%%%%%%
\setbeamercolor{background canvas}{bg=}
%%%%%%%%%%%%%%%%%%%%%%%%%%%%%%%
{
  \usebackgroundtemplate{\includegraphics[width=\paperwidth,height=\paperheight]{../1st_grader/images/betsy_ross_flag.png}}
  \begin{frame}
    %\frametitle{Slide 3 with background image}
    %Content of slide 3.
  \end{frame}
}
%%%%%%%%%%%%%%%%%%%%%%%%%%%%%%%%%%%

%%%%%%%%%%%%%%%%%%%%%%%%%%%%%%%%%%%%%%%%
\begin{frame}[plain]{A Prime Number}
\Large
\pause

 a prime number

\pause
\begin{description}
 \item[prime] 「もっとも重要な」という意味の形容詞
\end{description}

\pause

\bigskip
\hfill{}the prime minister
%\vfill
%\hyperlink{today}{\beamergotobutton{Today's Pronunciation}}

\end{frame}

%%%%%%%%%%%%%%%%%%%%%%%%%%%%%%%%%%%%%%%%
\begin{frame}[plain]{素数(a prime number)}
\Large

\pause

1\hspace{10pt}\pause
\Circled{2}\hspace{10pt}\pause
\Circled{3}\hspace{10pt}\pause
4\hspace{10pt}\pause
\Circled{5}\hspace{10pt}\pause
6\hspace{10pt}\pause
\Circled{7}\hspace{10pt}\pause
8\hspace{10pt}\pause
9\hspace{10pt}\pause
10\hspace{10pt}\pause
\Circled{11}\hspace{10pt}\pause
12\hspace{10pt}\pause
\Circled{13}\hspace{10pt}\pause
14\hspace{10pt}\pause
15\hspace{10pt}\pause
16\hspace{10pt}\pause
\Circled{17} \ldots\hspace{10pt}\pause

\bigskip

%ところで\\
%きょう\hspace{10pt}
%20250701\,\,\,\,\pause
%$\longrightarrow$\,\,\,なんと素数!\\\pause


%\hyperlink{today}{\beamergotobutton{Today's Pronunciation}}
\end{frame}
\normalsize
%%%%%%%%%%%%%%%%%%%%%%%%%%
\begin{frame}[plain]{20250701}
 \Large
\centering
\pause
\vfill
きょう$20250701$は\pause
素数\pause

\vfill

\hfill{20250509以来ひさびさ}
%$20{,}250{,}225=3^{2}\times{}5^{2}\times{}90{,}001$\pause

%$3^{2}\times{}5^{2}=\pause{}225$

%ざんねん\pause

%リラックスしていきましょう
%\vfill
%\raggedleft
%\hyperlink{today}{\beamergotobutton{Today's Pronunciation}}
\end{frame}
%%%%%%%%%%%%%%%%%%%%%%%%%%%%%

\setbeamercolor{background canvas}{bg=gray}
{
\IfFileExists{./images/fireworks.jpg}{%
  \usebackgroundtemplate{\includegraphics[width=.625\paperwidth]{./images/fireworks.jpg}}%
}{\relax}
  \begin{frame}[b]
    \frametitle{きょう20250701は素数!}
\tiny
\raggedright
  \textcolor{white}{ ``Fireworks'' by bayasaa is licensed under CC BY 2.0. }\\
   \textcolor{white}{To view a copy of this license,}\\
   \textcolor{white}{visit \url{https://creativecommons.org/licenses/by/2.0/?ref=openverse}.}
\hfill\hyperlink{today}{\beamergotobutton{Today's Pronunciation}}
  \end{frame}
}
\setbeamercolor{background canvas}{bg=}
%%%%%%%%%%%%%%%%%%%%%%%%%%%
%%
%% /p/ /b/
%%
%%%%%%%%%%%%%%%%%%%%%%%%%%
% 背景色を黒に変更
\setbeamercolor{background canvas}{bg=black}
%%%%%%%%%%%%%%%%%%%%%%%%%
\begin{frame}
\centering
  \textcolor{white}{\Huge\bfseries Today's Pronunciation}

 \vspace{30pt}

  \textcolor{white}{\Huge\bfseries \textipa{/p/}, \textipa{/b/}}
\end{frame}
\setbeamercolor{background canvas}{bg=}
%%%%%%%%%%%%%%%%%%%%%%%%%%
\begin{frame}[plain,label=slide_p_b]{破裂音\textipa{/p/}と\textipa{/b/}}

\large

\begin{enumerate}
 \item 無声音 \textipa{/p/}\hspace{20pt}\underLine{p}ig,\hspace{1\zw}cu\underLine{p},\hspace{1\zw}\underLine{p}a\underLine{p}er
 \item  有声音 \textipa{/b/}\hspace{20pt}\underLine{b}ig,\hspace{1\zw}jo\underLine{b},\hspace{1\zw}num\underLine{b}er
\end{enumerate}

\vspace*{20pt}

\normalsize
ポイント

\begin{itemize}\setbeamertemplate{items}[circle]
 \item しっかりと口を閉じて、空気が口の外へ漏れないように
 \item 一気に唇を離して空気をいきおいよく口の外へ出す
 \item \textipa{/p/と\textipa{/b/}}のちがいは声の有無だけ(口の動きはまったく同じ)
\end{itemize}


\bigskip

\visible<1->{\textipa{/p/} はパ行の子音、}\visible<1->{\textipa{/b/} はバ行の子音}


\hfill{\tiny 0205}\,{\scriptsize \myaudio{./audio/consonant_p_b_01.mp3}}

\end{frame}
%%%%%%%%%%%%%%%%%%%%%%%%%%
\begin{frame}[plain]{実際の単語で確認しよう}
\Large
 {\small \textipa{/p/}の音を含む英単語}\hfill{\tiny 0508}\,{\scriptsize \myaudio{./audio/consonant_p_b_02.mp3}}

\begin{enumerate}
 \item\textcolor{NavyBlue}{\bfseries p}ig%
\hfill\makebox[80pt][l]{\textipa{/\textcolor{BurntOrange}{p}\'Ig/}}\hspace{150pt}\mbox{}
 \item \textcolor{NavyBlue}{\bfseries p}in
\hfill\makebox[80pt][l]{\textipa{/\textcolor{BurntOrange}{p}\'In/}}\hspace{150pt}\mbox{}
 \item \textcolor{NavyBlue}{\bfseries p}en
\hfill\makebox[80pt][l]{\textipa{/\textcolor{BurntOrange}{p}\'en/}}\hspace{150pt}\mbox{}
 \item cu\textcolor{NavyBlue}{\bfseries p}
\hfill\makebox[80pt][l]{\textipa{/k\'\textturnv\textcolor{BurntOrange}{p}/}}\hspace{150pt}\mbox{}
 \item to\textcolor{NavyBlue}{\bfseries p}
\hfill\makebox[80pt][l]{\textipa{/t\'A\textcolor{BurntOrange}{p}/}}\hspace{150pt}\mbox{}
 \item shi\textcolor{NavyBlue}{\bfseries p}
\hfill\makebox[80pt][l]{\textipa{/S\'I\textcolor{BurntOrange}{p}/}}\hspace{150pt}\mbox{}
 \item \textcolor{NavyBlue}{\bfseries p}a\textcolor{NavyBlue}{\bfseries p}er
\hfill\makebox[80pt][l]{\textipa{/\textcolor{BurntOrange}{p}\'eI\textcolor{BurntOrange}{p}\textrhookschwa /}}\hspace{150pt}\mbox{}

\item s\textcolor{NavyBlue}{\bfseries p}eak
\hfill\makebox[80pt][l]{\textipa{/s\textcolor{BurntOrange}{p}\'\i:k/}}\hspace{150pt}\mbox{}
\end{enumerate}
\end{frame}
%%%%%%%%%%%%%%%%%%%%%%%%%%
\begin{frame}[plain]{実際の単語で確認しよう}
\Large
{\small \textipa{/b/}の音を含む英単語}\hfill{\tiny 0506}\,{\scriptsize \myaudio{./audio/consonant_p_b_03.mp3}}

\begin{enumerate}
 \item\textcolor{NavyBlue}{\bfseries b}ig%
\hfill\makebox[80pt][l]{\textipa{/\textcolor{BurntOrange}{b}\'Ig/}}\hspace{150pt}\mbox{}
 \item \textcolor{NavyBlue}{\bfseries b}us
\hfill\makebox[80pt][l]{\textipa{/\textcolor{BurntOrange}{b}\'\textturnv s/}}\hspace{150pt}\mbox{}
 \item \textcolor{NavyBlue}{\bfseries b}ed
\hfill\makebox[80pt][l]{\textipa{/\textcolor{BurntOrange}{b}\'ed/}}\hspace{150pt}\mbox{}

\item \textcolor{NavyBlue}{\bfseries b}ag
\hfill\makebox[80pt][l]{\textipa{/\textcolor{BurntOrange}{b}\'\ae g/}}\hspace{150pt}\mbox{}
 \item \textcolor{NavyBlue}{\bfseries b}ack
\hfill\makebox[80pt][l]{\textipa{/\textcolor{BurntOrange}{b}\'\ae k/}}\hspace{150pt}\mbox{}

 \item \textcolor{NavyBlue}{\bfseries b}ird
\hfill\makebox[80pt][l]{\textipa{/\textcolor{BurntOrange}{b}\'\textrhookschwa :d/}}\hspace{150pt}\mbox{}
 \item jo\textcolor{NavyBlue}{\bfseries b}
\hfill\makebox[80pt][l]{\textipa{/dZ\'A\textcolor{BurntOrange}{b}/}}\hspace{150pt}\mbox{}
 \item num\textcolor{NavyBlue}{\bfseries b}er
\hfill\makebox[80pt][l]{\textipa{/n\'\textturnv m\textcolor{BurntOrange}{b}\textrhookschwa /}}\hspace{150pt}\mbox{}
\end{enumerate}
\end{frame}
%%%%%%%%%%%%%%%%%%%%%%%%%%
%%%%%%%%%%%%%%%%%%%%%%%%%%
\begin{frame}[plain]{Quiz \textipa{/p/}}

\large
発音記号\textipa{/p/}で示した音声が含まれていたらT、含まれていなければFと答えてください。余裕がある人はなんという単語か書き取ってみましょう\\
\hfill{}{\scriptsize T: true(正しい)\hspace{5pt} F: false(まちがい)}
 \begin{enumerate}
  \item \mbox{}\visible<2->{T}\hspace{1\zw}\visible<7->{pet}
  \item \mbox{}\visible<3->{F}\hspace{1\zw}\visible<8->{bed}
  \item \mbox{}\visible<4->{T}\hspace{1\zw}\visible<9->{party}
  \item \mbox{}\visible<5->{T}\hspace{1\zw}\visible<10->{popcorn}
  \item \mbox{}\visible<6->{F}\hspace{1\zw}\visible<11->{dog}
 \end{enumerate}

\hfill{\tiny 0146}\,{\scriptsize \myaudio{./audio/consonant_p_b_04.mp3}}
\end{frame}
%%%%%%%%%%%%%%%%%%%%%%%%%%
\begin{frame}[plain]{Quiz \textipa{/b/}}
\large
発音記号\textipa{/b/}で示した音声が含まれていたらT、含まれていなければFと答えてください。余裕がある人はなんという単語か書き取ってみましょう\\
\hfill{}{\scriptsize T: true(正しい)\hspace{5pt}F: false(まちがい)}
 \begin{enumerate}
  \item \mbox{}\visible<2->{T}\hspace{1\zw}\visible<7->{bus}
  \item \mbox{}\visible<3->{T}\hspace{1\zw}\visible<8->{back}
  \item \mbox{}\visible<4->{F}\hspace{1\zw}\visible<9->{top}
  \item \mbox{}\visible<5->{F}\hspace{1\zw}\visible<10->{ship}
  \item \mbox{}\visible<6->{T}\hspace{1\zw}\visible<11->{number}
 \end{enumerate}

\hfill{\tiny 0144}\,{\scriptsize \myaudio{./audio/consonant_p_b_05.mp3}}
\end{frame}
%%%%%%%%%%%%%%%%%%%%%%%%%%%%%%%%%
\setbeamercolor{background canvas}{bg=}
%%%%%%%%%%%%%%%%%%%%%%%%%%
%%%%%%%%%%%%%%%%%%%%%%%%%%%
%%
%% /t/ /d/
%%
%%%%%%%%%%%%%%%%%%%%%%%%%%
% 背景色を黒に変更
\setbeamercolor{background canvas}{bg=black}
\begin{frame}

\centering
  \textcolor{white}{\Huge\bfseries Today's Pronunciation}\pause

 \vspace{30pt}

  \textcolor{white}{\Huge\bfseries \textipa{/t/}, \textipa{/d/}}

\vfill

\end{frame}
%%%%%%%%%%%%%%%%%%%%%%%%%%%%%%%%%%%%
\setbeamercolor{background canvas}{bg=}
%%%%%%%%%%%%%%%%%%%%%%%%%%%%%%%%%%%
\begin{frame}[plain, label=slide_t_d]{破裂音\textipa{/t/}と\textipa{/d/}}

\large

\begin{enumerate}
 \item  無声音 \textipa{/t/}\hspace{20pt}\underLine{t}able,\hspace{1\zw}mee\underLine{t},\hspace{1\zw}\underLine{t}oma\underLine{t}o
 \item  有声音 \textipa{/d/}\hspace{18pt}\underLine{d}ance,\hspace{1\zw}ba\underLine{d},\hspace{1\zw}\underLine{d}iamon\underLine{d}

\end{enumerate}

\vspace*{20pt}

\normalsize
ポイント

\begin{itemize}\setbeamertemplate{items}[circle]
 \item 舌の先を上の歯茎にしっかりとつける
 \item そのあとに舌の先を離して、いきおいよく息を口から外へ出す
 \end{itemize}

\textipa{/t/}はタの子音、
\textipa{/d/}はダの子音

\hfill{\tiny 0207}\,{\scriptsize \myaudio{./audio/consonant_t_d_01.mp3}}


\end{frame}
%%%%%%%%%%%%%%%%%%%%%%%%%%%%%%%%%%%%%%%%%%%%%%%%%%%
\begin{frame}[plain]{日本語の発音}
 
\visible<2->{ア行とン以外は、「子音$+$母音」の組み合わせ}

\visible<3->{たとえばカ行は}

\visible<4->{\begin{tblr}{lll}
\tikzmark{hoge}\textipa{/k/} $+$ ア &$\rightarrow$ &カ \\
\textipa{/k/} $+$ イ &$\rightarrow$ &キ \\
\textipa{/k/} $+$ ウ &$\rightarrow$ &ク \\
\textipa{/k/} $+$ エ &$\rightarrow$ &ケ \\
\textipa{/k/}\tikzmark{uke} $+$ オ &$\rightarrow$ &コ \\
\end{tblr}}

\hfill\tikzmark{seme}\visible<5->{ぜんぶ同じ子音がつかわれてる}

\hfill\visible<6->{とても論理的。うつくしい}

\begin{tikzpicture}[remember picture,overlay]
 \visible<5->{\draw[<-,opacity=.4,line width=2pt] ([xshift=-2pt,yshift=-8pt]pic cs:uke) to[out=-30, in=190] ([xshift=-2pt, yshift=2pt] pic cs:seme);}
    \visible<5->{\fill[fill=yellow!30,rounded corners]
      ([shift={(-0.3em,2ex)}]pic cs:hoge) 
      rectangle 
      ([shift={(0.2em,-1ex)}]pic cs:uke);}
 % 文字を再描画(上に重ねる)
  \visible<5->{\node[anchor=base west] at ([shift={(-3.9pt,0pt)}]pic cs:hoge) {\textipa{/k/}};}
  \visible<5->{\node[anchor=base east] at ([shift={(3.9pt,60pt)}]pic cs:uke) {\textipa{/k/}};}
  \visible<5->{\node[anchor=base east] at ([shift={(3.9pt,40pt)}]pic cs:uke) {\textipa{/k/}};}
  \visible<5->{\node[anchor=base east] at ([shift={(3.9pt,20pt)}]pic cs:uke) {\textipa{/k/}};}
  \visible<5->{\node[anchor=base east] at ([shift={(3.9pt,0pt)}]pic cs:uke) {\textipa{/k/}};}
\end{tikzpicture}
\end{frame}
%%%%%%%%%%%%%%%%%%%%%%%%%%%%%%%%%%%%%%%%%%%%%%%%%%%%
\begin{frame}[plain]{日本語の発音}
\visible<2->{もうひとつ}

\visible<3->{じゃあマ行は}

\visible<4->{\begin{tblr}{lll}
\tikzmark{hoge2}\textipa{/m/} $+$ ア &$\rightarrow$ &マ \\
\textipa{/m/} $+$ イ &$\rightarrow$ &ミ \\
\textipa{/m/} $+$ ウ &$\rightarrow$ &ム \\
\textipa{/m/} $+$ エ &$\rightarrow$ &メ \\
\textipa{/m/}\tikzmark{uke2} $+$ オ &$\rightarrow$ &モ \\
\end{tblr}}

\hfill\tikzmark{seme2}\visible<5->{ぜんぶ同じ子音がつかわれてる}

\hfill\visible<6->{とても論理的。うつくしい}

\begin{tikzpicture}[remember picture,overlay]
 \visible<5->{\draw[<-,opacity=.4,line width=2pt] ([xshift=-2pt,yshift=-8pt]pic cs:uke2) to[out=-30, in=190] ([xshift=-2pt, yshift=2pt] pic cs:seme2);}
    \visible<5->{\fill[fill=yellow!30,rounded corners]
      ([shift={(-0.3em,2ex)}]pic cs:hoge2) 
      rectangle 
      ([shift={(0.2em,-1ex)}]pic cs:uke2);}
 % 文字を再描画(上に重ねる)
  \visible<5->{\node[anchor=base west] at ([shift={(-3.9pt,0pt)}]pic cs:hoge2) {\textipa{/m/}};}
  \visible<5->{\node[anchor=base east] at ([shift={(3.9pt,60pt)}]pic cs:uke2) {\textipa{/m/}};}
  \visible<5->{\node[anchor=base east] at ([shift={(3.9pt,40pt)}]pic cs:uke2) {\textipa{/m/}};}
  \visible<5->{\node[anchor=base east] at ([shift={(3.9pt,20pt)}]pic cs:uke2) {\textipa{/m/}};}
  \visible<5->{\node[anchor=base east] at ([shift={(3.9pt,0pt)}]pic cs:uke2) {\textipa{/m/}};}
\end{tikzpicture}
\end{frame}
%%%%%%%%%%%%%%%%%%%%%%%%%%%%%%%%%%%%%%%%%%%%%%%%%%%
\begin{frame}[plain]{日本語の発音}
\visible<2->{もうひとつ}

\visible<3->{じゃあタ行は}

\visible<4->{\begin{tblr}{rllll}
\tikzmark{hoge3}\textipa{/t/} $+$ ア &$\rightarrow$ &タ \\
\textipa{/tS/} $+$ イ &$\rightarrow$ &チ &\visible<6->{$\neq$}&\visible<6->{\textipa{/ti/}}\\
\textipa{/ts/} $+$ ウ &$\rightarrow$ &ツ &\visible<6->{$\neq$}&\visible<6->{\textipa{/tu/}}\\
\textipa{/t/} $+$ エ &$\rightarrow$ &テ \\
\textipa{/t/}\tikzmark{uke3} $+$ オ &$\rightarrow$ &ト \\
\end{tblr}}

\hfill\tikzmark{seme3}\visible<5->{ぜんぶ同じ子音というわけじゃない}

\begin{tikzpicture}[remember picture,overlay]
 \visible<5->{\draw[<-,opacity=.4,line width=2pt] ([xshift=-2pt,yshift=-8pt]pic cs:uke3) to[out=-30, in=190] ([xshift=-2pt, yshift=2pt] pic cs:seme3);}
    \visible<5->{\fill[fill=yellow!30,rounded corners]
      ([shift={(-0.3em,2ex)}]pic cs:hoge3) 
      rectangle 
      ([shift={(0.2em,-1ex)}]pic cs:uke3);}
 % 文字を再描画(上に重ねる)
  \visible<5->{\node[anchor=base east] at ([shift={(3.9pt,80pt)}]pic cs:uke3) {\textipa{/t/}};}
  \visible<5->{\node[anchor=base east] at ([shift={(4pt,60pt)}]pic cs:uke3) {\textipa{/tS/}};}
  \visible<5->{\node[anchor=base east] at ([shift={(4pt,40pt)}]pic cs:uke3) {\textipa{/ts/}};}
  \visible<5->{\node[anchor=base east] at ([shift={(3.9pt,20pt)}]pic cs:uke3) {\textipa{/t/}};}
  \visible<5->{\node[anchor=base east] at ([shift={(3.9pt,0pt)}]pic cs:uke3) {\textipa{/t/}};}
\end{tikzpicture}
\end{frame}
%%%%%%%%%%%%%%%%%%%%%%%%%%%%%%%%%%%%%%%%%%%%%%%%%%%
\begin{frame}[plain]{日本語なまり}
 日本語話者は\textipa{/t/}のあとに「イ」や「ウ」の類の母音が続くとなまりやすい

\begin{enumerate}
 \item team \textipa{/t\'\i:m/}\hfill{\scriptsize 野球\kenten{チ}ーム}
 \item tip \textipa{/t\'Ip/}\hfill{\scriptsize \kenten{チ}ップという習慣}
 \item ticket \textipa{/t\'IkIt/}\hfill{\scriptsize 映画の\kenten{チ}ケット}
 \item two \textipa{/t\'u:/}\hfill{\scriptsize \kenten{ツ}ーアウト満塁}
 \item tool \textipa{/t\'u:l/}\hfill{\scriptsize \kenten{ツ}ールボックス}
 \item tour \textipa{/t\'U\textrhookschwa /}\hfill{\scriptsize 観光\kenten{ツ}アー}
\end{enumerate}
\end{frame}
%%%%%%%%%%%%%%%%%%%%%%%%%%%%%%%%%%%%%%%%%%%%%%%%%%%%
\begin{frame}[plain]{実際の単語で確認しよう}

\Large
{\small \textipa{/t/}の音を含む英単語}\hfill{\tiny 0504}\,{\scriptsize \myaudio{./audio/consonant_t_d_02.mp3}}


\begin{enumerate}
 \item ca\textcolor{NavyBlue}{\bfseries t}\hfill\makebox[80pt][l]{\textipa{/k\'\ae \textcolor{BurntOrange}{t}/}}\hspace{150pt}\mbox{}
 \item cu\textcolor{NavyBlue}{\bfseries t}\hfill\makebox[80pt][l]{\textipa{/k\'\textturnv\textcolor{BurntOrange}{t}/}}\hspace{150pt}\mbox{} 
\item mee\textcolor{NavyBlue}{\bfseries t}\hfill\makebox[80pt][l]{\textipa{/m\'\i:\textcolor{BurntOrange}{t}/}}\hspace{150pt}\mbox{} 
 \item \textcolor{NavyBlue}{\bfseries t}op\hfill\makebox[80pt][l]{\textipa{/\textcolor{BurntOrange}{t}\'Ap/}}\hspace{150pt}\mbox{}
 \item\textcolor{NavyBlue}{\bfseries t}ime\hfill\makebox[80pt][l]{\textipa{/\textcolor{BurntOrange}{t}\'aIm/}}\hspace{150pt}\mbox{}
 \item \textcolor{NavyBlue}{\bfseries t}able\hfill\makebox[80pt][l]{\textipa{/\textcolor{BurntOrange}{t}\'eIbl/}}\hspace{150pt}\mbox{}
\item \textcolor{NavyBlue}{\bfseries t}oy\hfill\makebox[80pt][l]{\textipa{/\textcolor{BurntOrange}{t}\'OI/}}\hspace{150pt}\mbox{}
% \item bir\textcolor{NavyBlue}{\bfseries d}
%\hfill\makebox[80pt][l]{\textipa{/b\'\textrhookschwa :\textcolor{BurntOrange}{d}/}}\hspace{150pt}\mbox{}
\item \textcolor{NavyBlue}{\bfseries t}eam\hfill\makebox[80pt][l]{\textipa{/\textcolor{BurntOrange}{t}\'\i:m/}}\hspace{150pt}\mbox{}
%\item ci\textcolor{NavyBlue}{\bfseries t}y\hfill\makebox[80pt][l]{\textipa{/s\'I\textcolor{BurntOrange}{t}i/}}\hspace{150pt}\mbox{}
\item \textcolor{NavyBlue}{\bfseries t}ool\hfill\makebox[80pt][l]{\textipa{/\textcolor{BurntOrange}{t}\'u:l/}}\hspace{150pt}\mbox{} 
\end{enumerate}
\end{frame}
%%%%%%%%%%%%%%%%%%%%%%%%%%
%%%%%%%%%%%%%%%%%%%%%%%%%%
\begin{frame}[plain]{実際の単語で確認しよう}

\Large
{\small \textipa{/d/}の音を含む英単語}\hfill{\tiny 0508}\,{\scriptsize \myaudio{./audio/consonant_t_d_03.mp3}}

\begin{enumerate}
 \item \textcolor{NavyBlue}{\bfseries d}o
\hfill\makebox[80pt][l]{\textipa{/\textcolor{BurntOrange}{d}\'u:/}}\hspace{150pt}\mbox{}
 \item \textcolor{NavyBlue}{\bfseries d}ay
\hfill\makebox[80pt][l]{\textipa{/\textcolor{BurntOrange}{d}\'eI/}}\hspace{150pt}\mbox{}
 \item\textcolor{NavyBlue}{\bfseries d}og%
\hfill\makebox[80pt][l]{\textipa{/\textcolor{BurntOrange}{d}\'Ag/}}\hspace{150pt}\mbox{}
\item \textcolor{NavyBlue}{\bfseries d}ance
\hfill\makebox[80pt][l]{\textipa{/\textcolor{BurntOrange}{d}\'\ae ns/}}\hspace{150pt}\mbox{}
 \item be\textcolor{NavyBlue}{\bfseries d}
\hfill\makebox[80pt][l]{\textipa{/b\'e\textcolor{BurntOrange}{d}/}}\hspace{150pt}\mbox{}
% \item bir\textcolor{NavyBlue}{\bfseries d}
%\hfill\makebox[80pt][l]{\textipa{/b\'\textrhookschwa :\textcolor{BurntOrange}{d}/}}\hspace{150pt}\mbox{}


 \item ba\textcolor{NavyBlue}{\bfseries d}
\hfill\makebox[80pt][l]{\textipa{/b\'\ae\textcolor{BurntOrange}{d}/}}\hspace{150pt}\mbox{} 

\item goo\textcolor{NavyBlue}{\bfseries d}
\hfill\makebox[80pt][l]{\textipa{/g\'U\textcolor{BurntOrange}{d}/}}\hspace{150pt}\mbox{} 
\item \textcolor{NavyBlue}{\bfseries d}iamon\textcolor{NavyBlue}{\bfseries d}
\hfill\makebox[80pt][l]{\textipa{/\textcolor{BurntOrange}{d}\'ai@m@n\textcolor{BurntOrange}{d}/}}\hspace{150pt}\mbox{}
 
\end{enumerate}
\end{frame}
%%%%%%%%%%%%%%%%%%%%%%
\begin{frame}[plain]{Quiz \textipa{/t/}}

\large
発音記号\textipa{/t/}で示した音声が含まれていたらT、含まれていなければFと答えてください\hfill{}{\scriptsize T: true(正しい)\hspace{5pt} F: false(まちがい)}
 \begin{enumerate}
  \item \mbox{}\visible<2->{T}\hspace{1\zw}\visible<7->{time}
  \item \mbox{}\visible<3->{F}\hspace{1\zw}\visible<8->{dance}
  \item \mbox{}\visible<4->{T}\hspace{1\zw}\visible<9->{cut}
  \item \mbox{}\visible<5->{T}\hspace{1\zw}\visible<10->{team}
  \item \mbox{}\visible<6->{F}\hspace{1\zw}\visible<11->{good}
 \end{enumerate}

\hfill{\tiny 0135}\,{\scriptsize \myaudio{./audio/consonant_t_d_04.mp3}}

\end{frame}
%%%%%%%%%%%%%%%%%%%%%%%%%%%
\begin{frame}[plain]{Quiz \textipa{/d/}}
\large
発音記号\textipa{/d/}で示した音声が含まれていたらT、含まれていなければFと答えてください\hfill{}{\scriptsize T: true(正しい)\hspace{5pt} F: false(まちがい)}
 \begin{enumerate}
  \item \mbox{}\visible<2->{F}\hspace{1\zw}\visible<7->{toy}
  \item \mbox{}\visible<3->{T}\hspace{1\zw}\visible<8->{day}
  \item \mbox{}\visible<4->{T}\hspace{1\zw}\visible<9->{bad}
  \item \mbox{}\visible<5->{F}\hspace{1\zw}\visible<10->{meet}
  \item \mbox{}\visible<6->{T}\hspace{1\zw}\visible<11->{good}
 \end{enumerate}

\hfill{\tiny 0144}\,{\scriptsize \myaudio{./audio/consonant_t_d_05.mp3}}
\end{frame}
%%%%%%%%%%%%%%%%%%%%%%%%%%%
%%
%% /k/ /g/
%%
%%%%%%%%%%%%%%%%%%%%%%%%%%
% 背景色を黒に変更
\setbeamercolor{background canvas}{bg=black}
\begin{frame}
\centering
  \textcolor{white}{\Huge\bfseries Today's Pronunciation}

 \vspace{30pt}

  \textcolor{white}{\Huge\bfseries \textipa{/k/}, \textipa{/g/}}

\end{frame}
\setbeamercolor{background canvas}{bg=}
%%%%%%%%%%%%%%%%%%%%%%%%%%
\begin{frame}[plain,label=slide_k_g]{破裂音\textipa{/k/}と\textipa{/g/}}

\large

\begin{enumerate}
 \item  無声音 \textipa{/k/}\hspace{20pt}\underLine{k}ind,\hspace{1\zw}\underLine{c}ool,\hspace{1\zw}spea\underLine{k}
 \item  有声音 \textipa{/g/}\hspace{20pt}\underLine{g}ood,\hspace{1\zw}ba\underLine{g},\hspace{1\zw}\underLine{g}irl
\end{enumerate}

\vspace*{20pt}

\normalsize
ポイント

\begin{itemize}\setbeamertemplate{items}[circle]
 \item 舌の奥(先じゃないですよ)を口の奥にあてて
 \item そのあとに舌を離して、いきおいよく息を口から外へ出す
\end{itemize}

\textipa{/k/} はカの子音

\hfill{\tiny 0206}\,{\scriptsize \myaudio{./audio/consonant_k_g_01.mp3}}
\end{frame}
%%%%%%%%%%%%%%%%%%%%%%%%%%%%%%%%%%%%%%%%%%%%%%%%%%%%
\begin{frame}[plain]{実際の単語で確認しよう}
\Large
{\small \textipa{/k/}の音を含む英単語}\hfill{\tiny 0509}\,{\scriptsize \myaudio{./audio/consonant_k_g_02.mp3}}


\begin{enumerate}
\item \textcolor{NavyBlue}{\bfseries c}ut
\hfill\makebox[80pt][l]{\textipa{/\textcolor{BurntOrange}{k}\'\textturnv t/}}\hspace{150pt}\mbox{}
 \item \textcolor{NavyBlue}{\bfseries c}at
\hfill\makebox[80pt][l]{\textipa{/\textcolor{BurntOrange}{k}\'\ae t/}}\hspace{150pt}\mbox{}
 \item \textcolor{NavyBlue}{\bfseries c}ome
\hfill\makebox[80pt][l]{\textipa{/\textcolor{BurntOrange}{k}\'\textturnv m/}}\hspace{150pt}\mbox{}
\item \textcolor{NavyBlue}{\bfseries c}ool
\hfill\makebox[80pt][l]{\textipa{/\textcolor{BurntOrange}{k}\'u:l/}}\hspace{150pt}\mbox{} 
 \item \textcolor{NavyBlue}{\bfseries k}ey
\hfill\makebox[80pt][l]{\textipa{/\textcolor{BurntOrange}{k}\'I:/}}\hspace{150pt}\mbox{} 
 \item \textcolor{NavyBlue}{\bfseries k}ind
\hfill\makebox[80pt][l]{\textipa{/\textcolor{BurntOrange}{k}\'aInd/}}\hspace{150pt}\mbox{}
\item s\textcolor{NavyBlue}{\bfseries k}y
\hfill\makebox[80pt][l]{\textipa{/s\textcolor{BurntOrange}{k}\'aI/}}\hspace{150pt}\mbox{}
 \item spea\textcolor{NavyBlue}{\bfseries k}%
\hfill\makebox[80pt][l]{\textipa{/sp\'\i:\textcolor{BurntOrange}{k}/}}\hspace{150pt}\mbox{}

% \item bir\textcolor{NavyBlue}{\bfseries d}
%\hfill\makebox[80pt][l]{\textipa{/b\'\textrhookschwa :\textcolor{BurntOrange}{d}/}}\hspace{150pt}\mbox{}

\end{enumerate}
\end{frame}
%%%%%%%%%%%%%%%%%%%%%%%%%%
%%%%%%%%%%%%%%%%%%%%%%%%%%
\begin{frame}[plain]{実際の単語で確認しよう}
\Large
{\small \textipa{/g/}の音を含む英単語}\hfill{\tiny 0509}\,{\scriptsize \myaudio{./audio/consonant_k_g_03.mp3}}


\begin{enumerate}
 \item \textcolor{NavyBlue}{\bfseries g}o
\hfill\makebox[80pt][l]{\textipa{/\textcolor{BurntOrange}{g}\'oU/}}\hspace{150pt}\mbox{}
\item \textcolor{NavyBlue}{\bfseries g}as
\hfill\makebox[80pt][l]{\textipa{/\textcolor{BurntOrange}{g}\'\ae s/}}\hspace{150pt}\mbox{}
 \item \textcolor{NavyBlue}{\bfseries G}od
\hfill\makebox[80pt][l]{\textipa{/\textcolor{BurntOrange}{g}\'\textscripta d/}}\hspace{150pt}\mbox{}
 \item\textcolor{NavyBlue}{\bfseries g}ood%
\hfill\makebox[80pt][l]{\textipa{/\textcolor{BurntOrange}{g}\'Ud/}}\hspace{150pt}\mbox{}
 \item \textcolor{NavyBlue}{\bfseries g}irl
\hfill\makebox[80pt][l]{\textipa{/\textcolor{BurntOrange}{g}\'\textrhookschwa :l/}}\hspace{150pt}\mbox{} 

 \item ba\textcolor{NavyBlue}{\bfseries g}
\hfill\makebox[80pt][l]{\textipa{/b\'\ae \textcolor{BurntOrange}{g}/}}\hspace{150pt}\mbox{}
% \item bir\textcolor{NavyBlue}{\bfseries d}
%\hfill\makebox[80pt][l]{\textipa{/b\'\textrhookschwa :\textcolor{BurntOrange}{d}/}}\hspace{150pt}\mbox{}

\item bi\textcolor{NavyBlue}{\bfseries g}
\hfill\makebox[80pt][l]{\textipa{/b\'I\textcolor{BurntOrange}{g}/}}\hspace{150pt}\mbox{} 
\item a\textcolor{NavyBlue}{\bfseries g}ain
\hfill\makebox[80pt][l]{\textipa{/@\textcolor{BurntOrange}{g}\'en/}}\hspace{150pt}\mbox{}
\end{enumerate}

\end{frame}
%%%%%%%%%%%%%%%%%%%%%%%%%%%
\begin{frame}[plain]{Quiz \textipa{/k/}}
\large
発音記号\textipa{/k/}で示した音声が含まれていたらT、含まれていなければFと答えてください。余裕がある人はなんという単語か書き取ってみましょう

\hfill{}{\scriptsize T: true(正しい)\hspace{5pt} F: false(まちがい)}
 \begin{enumerate}
  \item \mbox{}\visible<2->{T}\hspace{1\zw}\visible<7->{come}
  \item \mbox{}\visible<3->{F}\hspace{1\zw}\visible<8->{bag}
  \item \mbox{}\visible<4->{T}\hspace{1\zw}\visible<9->{speak}
  \item \mbox{}\visible<5->{F}\hspace{1\zw}\visible<10->{girl}
  \item \mbox{}\visible<6->{T}\hspace{1\zw}\visible<11->{kind}
 \end{enumerate}

\hfill{\tiny 0146}\,{\scriptsize \myaudio{./audio/consonant_k_g_04.mp3}}
\end{frame}
%%%%%%%%%%%%%%%%%%%%%%%%%%%
\begin{frame}[plain]{Quiz \textipa{/g/}}
\large
発音記号\textipa{/g/}で示した音声が含まれていたらT、含まれていなければFと答えてください。余裕がある人はなんという単語か書き取ってみましょう。

\hfill{}{\scriptsize T: true(正しい)\hspace{5pt} F: false(まちがい)}
 \begin{enumerate}
  \item \mbox{}\visible<2->{T}\hspace{1\zw}\visible<7->{big}
  \item \mbox{}\visible<3->{T}\hspace{1\zw}\visible<8->{go}
  \item \mbox{}\visible<4->{F}\hspace{1\zw}\visible<9->{kick}
  \item \mbox{}\visible<5->{T}\hspace{1\zw}\visible<10->{girl}
  \item \mbox{}\visible<6->{T}\hspace{1\zw}\visible<11->{again}
 \end{enumerate}

\hfill{\tiny 0145}\,{\scriptsize \myaudio{./audio/consonant_k_g_05.mp3}}
\end{frame}
%%%%%%%%%%%%%%%%%%%%%%%%%%%
%%
%% /f/ /v/
%%
%%%%%%%%%%%%%%%%%%%%%%%%%%
% 背景色を黒に変更
\setbeamercolor{background canvas}{bg=black}
\begin{frame}

\centering
  \textcolor{white}{\Huge\bfseries Today's Pronunciation}\pause

 \vspace{30pt}

  \textcolor{white}{\Huge\bfseries \textipa{/f/}, \textipa{/v/}}

\end{frame}
\setbeamercolor{background canvas}{bg=}
%%%%%%%%%%%%%%%%%%%%%%%%%%
\begin{frame}[plain]{破裂音と摩擦音}
\large

 \begin{description}
  \item[破裂音] 息をせき止め、たまった息を爆発するかのように放出する音\\
\textipa{/p/, /b/}\hspace{10pt} \textipa{/t/, /d/} \hspace{10pt}\textipa{/k/, /g/} 
  \item[摩擦音] 息の通り道を狭くし、その道を流れる空気の摩擦によって出す音
 \end{description}
\end{frame}
%%%%%%%%%%%%%%%%%%%%%%%%%%
 \begin{frame}[plain,label=slide_f_v]{摩擦音\textipa{/f/と\textipa{/v/}}}

\large

\begin{enumerate}
 \item  無声音 \textipa{/f/}\hspace{22pt}\underLine{f}ace,\hspace{1\zw}\underLine{f}amily,\hspace{1\zw}li\underLine{f}e
 \item  有声音 \textipa{/v/}\hspace{20pt}\underLine{v}egetable,\hspace{1\zw}\underLine{v}oice,\hspace{1\zw}lo\underLine{v}e
\end{enumerate}


\vspace*{15pt}

\normalsize
ポイント

\begin{itemize}[<+->]\setbeamertemplate{items}[circle]
 \item 普通に口を閉じましょう
 \item 上の前歯が下唇の内側に\kenten{軽く}触れていることを確認
 \item 上唇をもちあげて上の前歯を出す(歯の付け根が見えるくらい)
 \item 上の前歯と下唇の隙間から、空気を出します
 \item \textipa{/f:::::::/} $\longrightarrow$ \textipa{/f::::v::::/}
\end{itemize}

\vspace{-10pt}

\hfill{\tiny 0208}\,{\scriptsize \myaudio{./audio/consonant_f_v_01.mp3}}

%\hfill\hyperlink{ex}{\beamergotobutton{Today's Exercises}}
\end{frame}
%%%%%%%%%%%%%%%%%%%%%%%%%%%%%%%%%%%%%%%%%%%%%%%%%%%%
\begin{frame}[plain]{実際の単語で確認しよう \textipa{/f/}}
\Large
{\small \textipa{/f/}の音を含む英単語}\hfill{\tiny 0514}\,{\scriptsize \myaudio{./audio/consonant_f_v_02.mp3}}

\begin{enumerate}
\item \textcolor{NavyBlue}{\bfseries f}ace
\hfill\makebox[80pt][l]{\textipa{/\textcolor{BurntOrange}{f}\'eIs/}}\hspace{150pt}\mbox{}
 \item \textcolor{NavyBlue}{\bfseries f}ish
       \hfill\makebox[80pt][l]{\textipa{/\textcolor{BurntOrange}{f}\'IS/}}\hspace{150pt}\mbox{}
\item \textcolor{NavyBlue}{\bfseries f}ine
\hfill\makebox[80pt][l]{\textipa{/\textcolor{BurntOrange}{f}\'aIn/}}\hspace{150pt}\mbox{} 
 \item \textcolor{NavyBlue}{\bfseries f}ather
\hfill\makebox[80pt][l]{\textipa{/\textcolor{BurntOrange}{f}\'A:D\textrhookschwa /}}\hspace{150pt}\mbox{} 
 \item \textcolor{NavyBlue}{\bfseries f}riend
\hfill\makebox[80pt][l]{\textipa{/\textcolor{BurntOrange}{f}r\'end/}}\hspace{150pt}\mbox{}
\item \textcolor{NavyBlue}{\bfseries f}amily
\hfill\makebox[80pt][l]{\textipa{/\textcolor{BurntOrange}{f}\'\ae m@li/}}\hspace{150pt}\mbox{}
 \item \textcolor{NavyBlue}{\bfseries f}lower
\hfill\makebox[80pt][l]{\textipa{/\textcolor{BurntOrange}{f}l\'aU\textrhookschwa /}}\hspace{150pt}\mbox{}
 \item li\textcolor{NavyBlue}{\bfseries f}e%
\hfill\makebox[80pt][l]{\textipa{/l\'aI\textcolor{BurntOrange}{f}/}}\hspace{150pt}\mbox{}

% \item bir\textcolor{NavyBlue}{\bfseries d}
%\hfill\makebox[80pt][l]{\textipa{/b\'\textrhookschwa :\textcolor{BurntOrange}{d}/}}\hspace{150pt}\mbox{}
\end{enumerate}
\end{frame}
%%%%%%%%%%%%%%%%%%%%%%%%%%
%%%%%%%%%%%%%%%%%%%%%%%%%%
\begin{frame}[plain]{実際の単語で確認しよう \textipa{/v/}}
\Large
{\small \textipa{/v/}の音を含む英単語}\hfill{\tiny 0511}\,{\scriptsize \myaudio{./audio/consonant_f_v_03.mp3}}

\begin{enumerate}
 \item \textcolor{NavyBlue}{\bfseries v}ase
\hfill\makebox[80pt][l]{\textipa{/\textcolor{BurntOrange}{v}\'eIs/}}\hspace{150pt}\mbox{}
\item \textcolor{NavyBlue}{\bfseries v}ery%
\hfill\makebox[80pt][l]{\textipa{/\textcolor{BurntOrange}{v}\'eri/}}\hspace{150pt}\mbox{}
 \item \textcolor{NavyBlue}{\bfseries v}isit
\hfill\makebox[80pt][l]{\textipa{/\textcolor{BurntOrange}{v}\'IzIt/}}\hspace{150pt}\mbox{} 
 \item\textcolor{NavyBlue}{\bfseries v}oice%
\hfill\makebox[80pt][l]{\textipa{/\textcolor{BurntOrange}{v}\'OIs/}}\hspace{150pt}\mbox{}

 \item \textcolor{NavyBlue}{\bfseries v}egetable
\hfill\makebox[80pt][l]{\textipa{/\textcolor{BurntOrange}{v}\'edZt@bl/}}\hspace{150pt}\mbox{}
 \item mo\textcolor{NavyBlue}{\bfseries v}e
\hfill\makebox[80pt][l]{\textipa{/m\'u:\textcolor{BurntOrange}{v}/}}\hspace{150pt}\mbox{}
% \item bir\textcolor{NavyBlue}{\bfseries d}
%\hfill\makebox[80pt][l]{\textipa{/b\'\textrhookschwa :\textcolor{BurntOrange}{d}/}}\hspace{150pt}\mbox{}

\item lo\textcolor{NavyBlue}{\bfseries v}e
\hfill\makebox[80pt][l]{\textipa{/l\'\textturnv \textcolor{BurntOrange}{v}/}}\hspace{150pt}\mbox{} 
\item fi\textcolor{NavyBlue}{\bfseries v}e
\hfill\makebox[80pt][l]{\textipa{/f\'aI\textcolor{BurntOrange}{v}/}}\hspace{150pt}\mbox{}
\end{enumerate}
\end{frame}
%%%%%%%%%%%%%%%%%%%%%%%%%%
\begin{frame}[plain]{Quiz 1 \textipa{/f/}}

\large
発音記号\textipa{/f/}で示した音声が含まれていたらT、含まれていなければFと答えてください。余裕がある人はなんという単語か書き取ってみましょう

\hfill{}{\scriptsize T: true(正しい)\hspace{5pt} F: false(まちがい)}
 \begin{enumerate}
  \item \mbox{}\visible<2->{F}\hspace{1\zw}\visible<7->{very}
  \item \mbox{}\visible<3->{T}\hspace{1\zw}\visible<8->{face}
  \item \mbox{}\visible<4->{T}\hspace{1\zw}\visible<9->{flower}
  \item \mbox{}\visible<5->{F}\hspace{1\zw}\visible<10->{voice}
  \item \mbox{}\visible<6->{T}\hspace{1\zw}\visible<11->{fifteen}
 \end{enumerate}

\hfill{\tiny 0148}\,{\scriptsize \myaudio{./audio/consonant_f_v_04.mp3}}
\end{frame}
%%%%%%%%%%%%%%%%%%%%%%%%%%
\begin{frame}[plain]{Quiz 2 \textipa{/v/}}
\large
発音記号\textipa{/v/}で示した音声が含まれていたらT、含まれていなければFと答えてください。余裕がある人はなんという単語か書き取ってみましょう。

\hfill{}{\scriptsize T: true(正しい)\hspace{5pt} F: false(まちがい)}
 \begin{enumerate}
  \item \mbox{}\visible<2->{F}\hspace{1\zw}\visible<7->{life}
  \item \mbox{}\visible<3->{F}\hspace{1\zw}\visible<8->{family}
  \item \mbox{}\visible<4->{T}\hspace{1\zw}\visible<9->{love}
  \item \mbox{}\visible<5->{T}\hspace{1\zw}\visible<10->{visit}
  \item \mbox{}\visible<6->{T}\hspace{1\zw}\visible<11->{five}
 \end{enumerate}

\hfill{\tiny 0147}\,{\scriptsize \myaudio{./audio/consonant_f_v_05.mp3}}

\end{frame}
%%%%%%%%%%%%%%%%%%%%%%%%
\begin{frame}[plain]{Quiz}\large

free(自由な)またはthree(3)を発音していきます。どちらを発音したか判断しましょう

\bigskip

 \begin{columns}[t]
   \begin{column}{.45\textwidth}
    \begin{tabular}{rlr}
     1& \visible<2->{three}&\myAnch{q1}{white}{\textbullet} \\
     2& \visible<3->{free}&\myAnch{q2}{white}{\textbullet} \\
     3& \visible<4->{free}&\myAnch{q3}{white}{\textbullet} \\
     4& \visible<5->{three}&\myAnch{q4}{white}{\textbullet} \\
     5& \visible<6->{free}&\myAnch{q5}{white}{\textbullet} \\
     6& \visible<7->{three}&\myAnch{q6}{white}{\textbullet} 
    \end{tabular}
   \end{column}
%%%%%%%%%%%
   \begin{column}{.45\textwidth}
    \begin{tabular}{lll}
     \myAnch{a1}{white}{\textbullet}& free& \textipa{/fr\'\i :/}\\
     &\\
     \myAnch{a2}{white}{\textbullet}& three& \textipa{/Tr\'\i :/}  \\
    \end{tabular}
   \end{column}
 \end{columns}

\begin{tikzpicture}[remember picture, overlay]
\tikzset{hoge/.style = {line width=4pt, ->, opacity=.6}}
 \visible<2->{\draw[hoge, Maroon] (q1.east) to[out=0, in=180] (a2.west);}
 \visible<3->{\draw[hoge, NavyBlue] (q2.east) to[out=0, in=180] (a1.west);}
 \visible<4->{\draw[hoge, NavyBlue] (q3.east) to[out=0, in=180] (a1.west);}
 \visible<5->{\draw[hoge, Maroon] (q4.east) to[out=0, in=180] (a2.west);}
 \visible<6->{\draw[hoge, NavyBlue] (q5.east) to[out=0, in=180] (a1.west);}
 \visible<7->{\draw[hoge, Maroon] (q6.east) to[out=0, in=180] (a2.west);}
\end{tikzpicture}

\hfill{\tiny 0136}\,{\scriptsize \myaudio{./audio/consonant_free_three_00.mp3}}

\end{frame}
%%%%%%%%%%%%%%%%%%%%%%%%%%
%%%%%%%%%%%%%%%%%%%%%%%%%%%
%%
%% /s/ /z/
%%
%%%%%%%%%%%%%%%%%%%%%%%%%%
% 背景色を黒に変更
\setbeamercolor{background canvas}{bg=black}
\begin{frame}

\centering
  \textcolor{white}{\Huge\bfseries Today's Pronunciation}

 \vspace{30pt}

  \textcolor{white}{\Huge\bfseries \textipa{/s/}, \textipa{/z/}}


\end{frame}
\setbeamercolor{background canvas}{bg=}
%%%%%%%%%%%%%%%%%%%%%%%%%%
 \begin{frame}[plain,label=slide_s_z]{摩擦音\textipa{/s/と\textipa{/z/}}}

\large

\begin{enumerate}
 \item  無声音 \textipa{/s/}\hspace{20pt}\underLine{s}leep,\hspace{1\zw}\underLine{s}ky,\hspace{1\zw}ri\underLine{c}e
 \item  有声音 \textipa{/z/}\hspace{20pt}\underLine{z}ero,\hspace{1\zw}vi\underLine{s}it,\hspace{1\zw}ea\underLine{s}y
\end{enumerate}

\vspace*{20pt}

\normalsize
ポイント

\begin{itemize}\setbeamertemplate{items}[circle]
 \item \visible<1->{日本語の「サスセソ」の子音$\longrightarrow$\,\textipa{/s/}}%
\hfill\visible<1->{{\scriptsize 「シ」の子音じゃないですよ!}}
 \item \visible<1->{\textipa{/s:::::/}といいながら、途中から声を出す$\longrightarrow$\,\textipa{/s::::z::::/}$\longrightarrow$\,\textipa{/z/}のできあがり}\\
\hfill{}\visible<1->{{\scriptsize \textipa{/:/}は長音記号。例 meet \textipa{/m\'\i:t/}}}
% \item 舌の先を上の歯茎に近づけて
% \item その隙間から息を出す
 
\end{itemize}
\hfill{\tiny 0207}\,{\scriptsize \myaudio{./audio/consonant_s_z_01.mp3}}

\end{frame}
%%%%%%%%%%%%%%%%%%%%%%%%%%%%%%%%%%%%%%%%%%%%%%%%%%%%
\begin{frame}[plain]{実際の単語で確認しよう}
\Large
{\small \textipa{/s/}の音を含む英単語}\hfill{\tiny 0510}\,{\scriptsize \myaudio{./audio/consonant_s_z_02.mp3}}

\begin{enumerate}
 \item \textcolor{NavyBlue}{\bfseries s}it
\hfill\makebox[80pt][l]{\textipa{/\textcolor{BurntOrange}{s}\'It /}}\hspace{150pt}\mbox{}
\item \textcolor{NavyBlue}{\bfseries s}ee
\hfill\makebox[80pt][l]{\textipa{/\textcolor{BurntOrange}{s}\'\i:/}}\hspace{150pt}\mbox{}
 \item \textcolor{NavyBlue}{\bfseries s}ea
       \hfill\makebox[80pt][l]{\textipa{/\textcolor{BurntOrange}{s}\'\i:/}}\hspace{150pt}\mbox{}
\item \textcolor{NavyBlue}{\bfseries s}leep
\hfill\makebox[80pt][l]{\textipa{/\textcolor{BurntOrange}{s}l\'\i:p/}}\hspace{150pt}\mbox{} 
 \item le\textcolor{NavyBlue}{\bfseries ss}on
\hfill\makebox[80pt][l]{\textipa{/l\'e\textcolor{BurntOrange}{s}n/}}\hspace{150pt}\mbox{} 
\item bu\textcolor{NavyBlue}{\bfseries s}
\hfill\makebox[80pt][l]{\textipa{/b\'\textturnv\textcolor{BurntOrange}{s}/}}\hspace{150pt}\mbox{}
 \item ri\textcolor{NavyBlue}{\bfseries c}e%
\hfill\makebox[80pt][l]{\textipa{/r\'aI\textcolor{BurntOrange}{s}/}}\hspace{150pt}\mbox{}
 \item fa\textcolor{NavyBlue}{\bfseries c}e
\hfill\makebox[80pt][l]{\textipa{/f\'eI\textcolor{BurntOrange}{s}/}}\hspace{150pt}\mbox{}
% \item bir\textcolor{NavyBlue}{\bfseries d}
%\hfill\makebox[80pt][l]{\textipa{/b\'\textrhookschwa :\textcolor{BurntOrange}{d}/}}\hspace{150pt}\mbox{}
\end{enumerate}
\end{frame}
%%%%%%%%%%%%%%%%%%%%%%%%%%
\begin{frame}[plain]{実際の単語で確認しよう}
\Large
{\small \textipa{/z/}の音を含む英単語}\hfill{\tiny 0510}\,{\scriptsize \myaudio{./audio/consonant_s_z_03.mp3}}

\begin{enumerate}[<+->]
 \item \textcolor{NavyBlue}{\bfseries z}ero
\hfill\makebox[80pt][l]{\textipa{/\textcolor{BurntOrange}{z}\'IroU/}}\hspace{150pt}\mbox{}
\item \textcolor{NavyBlue}{\bfseries z}oo%
\hfill\makebox[80pt][l]{\textipa{/\textcolor{BurntOrange}{z}\'u:/}}\hspace{150pt}\mbox{}
\item \textcolor{NavyBlue}{\bfseries z}ipper
\hfill\makebox[80pt][l]{\textipa{/\textcolor{BurntOrange}{z}\'Ip\textrhookschwa /}}\hspace{150pt}\mbox{}
  \item vi\textcolor{NavyBlue}{\bfseries s}it
\hfill\makebox[80pt][l]{\textipa{/v\'I\textcolor{BurntOrange}{z}It/}}\hspace{150pt}\mbox{} 
 \item ea\textcolor{NavyBlue}{\bfseries s}y%
\hfill\makebox[80pt][l]{\textipa{/\'\i:\textcolor{BurntOrange}{z}i/}}\hspace{150pt}\mbox{}

 \item hi\textcolor{NavyBlue}{\bfseries s}
\hfill\makebox[80pt][l]{\textipa{/h\'\i\textcolor{BurntOrange}{z}/}}\hspace{150pt}\mbox{}
 \item car\textcolor{NavyBlue}{\bfseries s}
\hfill\makebox[80pt][l]{\textipa{/k\'\textscripta\textrhookschwa\textcolor{BurntOrange}{z}/}}\hspace{150pt}\mbox{}
% \item bir\textcolor{NavyBlue}{\bfseries d}
%\hfill\makebox[80pt][l]{\textipa{/b\'\textrhookschwa :\textcolor{BurntOrange}{d}/}}\hspace{150pt}\mbox{}
\item play\textcolor{NavyBlue}{\bfseries s}
\hfill\makebox[80pt][l]{\textipa{/pl\'eI\textcolor{BurntOrange}{z}/}}\hspace{150pt}\mbox{} 
\end{enumerate}

\normalsize
\begin{textblock*}{0.4\linewidth}(270pt,90pt)
\visible<3->{\begin{tikzpicture}
\cat[
scale=1.5,
speech={\small zipper},
signpost=\scalebox{0.6}{
\parbox{2.8cm}{\color{black}
\centering 「\kenten{ジ}ッパー」\\じゃない!\\[5pt]「\kenten{ズィ}ッパー」!}},
signcolour= brown!70!gray,
signback=white!80!brown
]
\end{tikzpicture}}
\end{textblock*}
\end{frame}
%%%%%%%%%%%%%%%%%%%%%%%%%%
\begin{frame}[plain]{Quiz \textipa{/s/}}
発音記号\textipa{/s/}で示した音声が含まれていたらT、含まれていなければFと答えてください。余裕がある人はなんという単語か書き取ってみましょう

\hfill{}{\scriptsize T: true(正しい)\hspace{5pt} F: false(まちがい)}

\Large

 \begin{enumerate}
  \item \mbox{}\visible<2->{F}\hspace{1\zw}\visible<7->{zoo}
  \item \mbox{}\visible<3->{T}\hspace{1\zw}\visible<8->{sit}
  \item \mbox{}\visible<4->{T}\hspace{1\zw}\visible<9->{sleep}
  \item \mbox{}\visible<5->{T}\hspace{1\zw}\visible<10->{rice}
  \item \mbox{}\visible<6->{F}\hspace{1\zw}\visible<11->{zero}
 \end{enumerate}

\hfill{\tiny 0146}\,{\scriptsize \myaudio{./audio/consonant_s_z_04.mp3}}
\end{frame}
%%%%%%%%%%%%%%%%%%%%%%%%%%
\begin{frame}[plain]{Quiz \textipa{/z/}}
発音記号\textipa{/z/}で示した音声が含まれていたらT、含まれていなければFと答えてください。余裕がある人はなんという単語か書き取ってみましょう

\hfill{}{\scriptsize T: true(正しい)\hspace{5pt} F: false(まちがい)}

\Large

 \begin{enumerate}
  \item \mbox{}\visible<2->{F}\hspace{1\zw}\visible<7->{sky}
  \item \mbox{}\visible<3->{T}\hspace{1\zw}\visible<8->{easy}
  \item \mbox{}\visible<4->{T}\hspace{1\zw}\visible<9->{visit}
  \item \mbox{}\visible<5->{T}\hspace{1\zw}\visible<10->{cars}
  \item \mbox{}\visible<6->{T}\hspace{1\zw}\visible<11->{plays}
 \end{enumerate}

\hfill{\tiny 0146}\,{\scriptsize \myaudio{./audio/consonant_s_z_05.mp3}}
\end{frame}
%%%%%%%%%%%%%%%%%%%%%%%%%%%
%%
%% /T/ /D/
%%
%%%%%%%%%%%%%%%%%%%%%%%%%%
% 背景色を黒に変更
\setbeamercolor{background canvas}{bg=black}
\begin{frame}
\hypertarget{today}{}

\centering
  \textcolor{white}{\Huge\bfseries Today's Pronunciation}

 \vspace{30pt}

  \textcolor{white}{\Huge\bfseries \textipa{/T/}, \textipa{/D/}}

\end{frame}
\setbeamercolor{background canvas}{bg=}
%%%%%%%%%%%%%%%%%%%%%%%%%%
 \begin{frame}[plain,label=slide_Th]{摩擦音\textipa{/T/と\textipa{/D/}}}

\large

\begin{enumerate}
 \item  無声音 \textipa{/T/}\hspace{20pt}\underLine{Th}ank you.\hspace{1\zw}\underLine{th}irty\hspace{1\zw}mou\underLine{th}
 \item  有声音 \textipa{/D/}\hspace{20pt}\underLine{th}at\hspace{1\zw}\underLine{th}ese\hspace{1\zw}wi\underLine{th}
\end{enumerate}

\vspace*{20pt}

\normalsize
ポイント

\begin{itemize}\setbeamertemplate{items}[circle]
 \item 舌の先を上の前歯に\kenten{軽く}あてる(上下の前歯のすきまから舌をちょっと出す)
 \item その隙間から\kenten{強く}息を出す
 
\end{itemize}
\hfill{\tiny 0205}\,{\scriptsize \myaudio{./audio/consonant_Th_01.mp3}}

\hfill\hyperlink{ex}{\beamergotobutton{Today's Exercises}}

\end{frame}
%%%%%%%%%%%%%%%%%%%%%%%%%%%%%%%%%%%%%%%%%%%%%%%%%%%%
\begin{frame}[plain]{実際の単語で確認しよう}
\Large
\hypertarget{ex}{}

{\small \textipa{/T/}の音を含む英単語}\hfill{\tiny 0310}\,{\scriptsize \myaudio{./audio/consonant_Th_02.mp3}}

\begin{enumerate}
 \item \textcolor{NavyBlue}{\bfseries th}ree
\hfill\makebox[80pt][l]{\textipa{/\textcolor{BurntOrange}{T}r\'\i:/}}\hspace{150pt}\mbox{}
\item \textcolor{NavyBlue}{\bfseries th}ird%
      \hfill\makebox[80pt][l]{\textipa{/\textcolor{BurntOrange}{T}\'\textrhookschwa :d/}}\hspace{150pt}\mbox{}
\item \textcolor{NavyBlue}{\bfseries th}irteen
\hfill\makebox[80pt][l]{\textipa{/\textcolor{BurntOrange}{T}\textrhookschwa :t\'\i:n/}}\hspace{150pt}\mbox{} 
 \item \textcolor{NavyBlue}{\bfseries th}ousand
       \hfill\makebox[80pt][l]{\textipa{/\textcolor{BurntOrange}{T}\'aUz@nd/}}\hspace{150pt}\mbox{}
\item \textcolor{NavyBlue}{\bfseries th}ing
\hfill\makebox[80pt][l]{\textipa{/\textcolor{BurntOrange}{T}\'IN/}}\hspace{150pt}\mbox{}
 \item \textcolor{NavyBlue}{\bfseries th}ink
\hfill\makebox[80pt][l]{\textipa{/\textcolor{BurntOrange}{T}\'INk/}}\hspace{150pt}\mbox{} 
 \item mou\textcolor{NavyBlue}{\bfseries th}
\hfill\makebox[80pt][l]{\textipa{/m\'aU\textcolor{BurntOrange}{T}/}}\hspace{150pt}\mbox{}
 \item ba\textcolor{NavyBlue}{\bfseries th}%
\hfill\makebox[80pt][l]{\textipa{/b\'\ae\textcolor{BurntOrange}{T}/}}\hspace{150pt}\mbox{}
% \item bir\textcolor{NavyBlue}{\bfseries d}
%\hfill\makebox[80pt][l]{\textipa{/b\'\textrhookschwa :\textcolor{BurntOrange}{d}/}}\hspace{150pt}\mbox{}
\end{enumerate}
\end{frame}
%%%%%%%%%%%%%%%%%%%%%%%%%%
%%%%%%%%%%%%%%%%%%%%%%%%%%
\begin{frame}[plain]{実際の単語で確認しよう}
\Large
{\small \textipa{/D/}の音を含む英単語}\hfill{\tiny 0503}\,{\scriptsize \myaudio{./audio/consonant_Th_03.mp3}}

\begin{enumerate}
 \item \textcolor{NavyBlue}{\bfseries th}is
\hfill\makebox[80pt][l]{\textipa{/\textcolor{BurntOrange}{D}\'Is/}}\hspace{150pt}\mbox{}
 \item \textcolor{NavyBlue}{\bfseries th}at
\hfill\makebox[80pt][l]{\textipa{/\textcolor{BurntOrange}{D}\'\ae t/}}\hspace{150pt}\mbox{}

\item \textcolor{NavyBlue}{\bfseries th}ese%
\hfill\makebox[80pt][l]{\textipa{/\textcolor{BurntOrange}{D}\'\i:z/}}\hspace{150pt}\mbox{}
\item \textcolor{NavyBlue}{\bfseries th}ose
\hfill\makebox[80pt][l]{\textipa{/\textcolor{BurntOrange}{D}\'oUz/}}\hspace{150pt}\mbox{}
  \item \textcolor{NavyBlue}{\bfseries th}ey
\hfill\makebox[80pt][l]{\textipa{/\textcolor{BurntOrange}{D}\'eI/}}\hspace{150pt}\mbox{} 
 \item \textcolor{NavyBlue}{\bfseries th}em%
\hfill\makebox[80pt][l]{\textipa{/\textcolor{BurntOrange}{D}\'em/}}\hspace{150pt}\mbox{}

 \item \textcolor{NavyBlue}{\bfseries th}ere
\hfill\makebox[80pt][l]{\textipa{/\textcolor{BurntOrange}{D}\'e\textrhookschwa /}}\hspace{150pt}\mbox{}
 \item wi\textcolor{NavyBlue}{\bfseries th}
\hfill\makebox[80pt][l]{\textipa{/w\'I\textcolor{BurntOrange}{D}/}}\hspace{150pt}\mbox{}

\end{enumerate}
\end{frame}
%%%%%%%%%%%%%%%%%%%%%%%%%%
\begin{frame}[plain]{Quiz \textipa{/T/}}
\large
発音記号\textipa{/T/}で示した音声が含まれていたらT、含まれていなければFと答えてください。余裕がある人はなんという単語か書き取ってみましょう

\hfill{}{\scriptsize T: true(正しい)\hspace{5pt} F: false(まちがい)}
 \begin{enumerate}
  \item \mbox{}\visible<2->{T}\hspace{1\zw}\visible<7->{three}
  \item \mbox{}\visible<3->{F}\hspace{1\zw}\visible<8->{these}
  \item \mbox{}\visible<4->{T}\hspace{1\zw}\visible<9->{thirty}
  \item \mbox{}\visible<5->{F}\hspace{1\zw}\visible<10->{sleep}
  \item \mbox{}\visible<6->{T}\hspace{1\zw}\visible<11->{think}
 \end{enumerate}

\hfill{\tiny 0145}\,{\scriptsize \myaudio{./audio/consonant_Th_04.mp3}}

\end{frame}
%%%%%%%%%%%%%%%%%%%%%%%%%%
\begin{frame}[plain]{Quiz\textipa{/D/}}
\large
発音記号\textipa{/D/}で示した音声が含まれていたらT、含まれていなければFと答えてください。余裕がある人はなんという単語か書き取ってみましょう

\hfill{}{\scriptsize T: true(正しい)\hspace{5pt} F: false(まちがい)}
 \begin{enumerate}
  \item \mbox{}\visible<2->{T}\hspace{1\zw}\visible<7->{there}
  \item \mbox{}\visible<3->{T}\hspace{1\zw}\visible<8->{those}
  \item \mbox{}\visible<4->{T}\hspace{1\zw}\visible<9->{that}
  \item \mbox{}\visible<5->{F}\hspace{1\zw}\visible<10->{mouth}
  \item \mbox{}\visible<6->{F}\hspace{1\zw}\visible<11->{zero}
 \end{enumerate}

\hfill{\tiny 0146}\,{\scriptsize \myaudio{./audio/consonant_Th_05.mp3}}
\end{frame}
%%%%%%%%%%%%%%%%%%%%%%%%%
%%%%%%%%%%%%%%%%%%%%%%%%%%%
%%
%% /S/ /Z/
%%
%%%%%%%%%%%%%%%%%%%%%%%%%%
% 背景色を黒に変更
\setbeamercolor{background canvas}{bg=black}
\begin{frame}
\centering
  \textcolor{white}{\Huge\bfseries Today's Pronunciation}\pause

 \vspace{30pt}

  \textcolor{white}{\Huge\bfseries \textipa{/S/}, \textipa{/Z/}}
\end{frame}
\setbeamercolor{background canvas}{bg=}
%%%%%%%%%%%%%%%%%%%%%%%%%%
 \begin{frame}[plain,label=slide_textesh_textyogh]{摩擦音\textipa{/S/と\textipa{/Z/}}}

\large

\begin{enumerate}
 \item  無声音 \textipa{/S/}\hspace{20pt}\underLine{sh}op,\hspace{1\zw}sta\underLine{t}ion,\hspace{1\zw}pu\underLine{sh}
 \item  有声音 \textipa{/Z/}\hspace{20pt}A\underLine{s}ia,\hspace{1\zw}televi\underLine{s}ion,\hspace{1\zw}u\underLine{s}ually
\end{enumerate}

\vspace*{20pt}

\normalsize
ポイント

\begin{itemize}\setbeamertemplate{items}[circle]
 \item ちょっとお静かに --- という気持ちで「シー!」と長くいいましょう
 \item ほんのちょっぴり唇を丸めましょう$\rightarrow$\,これで\,\textipa{/S/}\,のできあがり 
 \item 声を出せば$\rightarrow$これで\,\textipa{/Z/}\,のできあがり
\end{itemize}
\hfill{\tiny 0152}\,{\scriptsize \myaudio{./audio/consonant_textesh_textyogh_01.mp3}}

\end{frame}
%%%%%%%%%%%%%%%%%%%%%%%%%%%%%%%%%%%%%%%%%%%%%%%%%%%%
%%%%%%%%%%%%%%%%%%%%%%%%%%%%%%%%%%%%%%%%%%%%%%%%%%%%
\begin{frame}[plain]{実際の単語で確認しよう \textipa{/S/}}
\Large
{\small \textipa{/S/}の音を含む英単語}\hfill{\tiny 0437}\,{\scriptsize \myaudio{./audio/consonant_textesh_textyogh_02.mp3}}

\begin{enumerate}

 \item \textcolor{NavyBlue}{\bfseries sh}e
\hfill\makebox[80pt][l]{\textipa{/\textcolor{BurntOrange}{S}\'\i:/}}\hspace{150pt}\mbox{}
\item \textcolor{NavyBlue}{\bfseries sh}ip%
      \hfill\makebox[80pt][l]{\textipa{/\textcolor{BurntOrange}{S}\'Ip/}}\hspace{150pt}\mbox{}
 \item \textcolor{NavyBlue}{\bfseries sh}eep
\hfill\makebox[80pt][l]{\textipa{/\textcolor{BurntOrange}{S}\'\i:p/}}\hspace{150pt}\mbox{}
 \item \textcolor{NavyBlue}{\bfseries sh}oes%
\hfill\makebox[80pt][l]{\textipa{/\textcolor{BurntOrange}{S}\'u:z/}}\hspace{150pt}\mbox{}
\item pu\textcolor{NavyBlue}{\bfseries sh}
\hfill\makebox[80pt][l]{\textipa{/p\'U\textcolor{BurntOrange}{S}/}}\hspace{150pt}\mbox{}
  \item wa\textcolor{NavyBlue}{\bfseries sh}
\hfill\makebox[80pt][l]{\textipa{/w\'A\textcolor{BurntOrange}{S}/}}\hspace{150pt}\mbox{}
\item sta\textcolor{NavyBlue}{\bfseries t}ion
\hfill\makebox[80pt][l]{\textipa{/st\'eI\textcolor{BurntOrange}{S}@n/}}\hspace{150pt}\mbox{} 
 \item o\textcolor{NavyBlue}{\bfseries c}ean
       \hfill\makebox[80pt][l]{\textipa{/\'oU\textcolor{BurntOrange}{S}@n/}}\hspace{150pt}\mbox{}

% \item bir\textcolor{NavyBlue}{\bfseries d}
%\hfill\makebox[80pt][l]{\textipa{/b\'\textrhookschwa :\textcolor{BurntOrange}{d}/}}\hspace{150pt}\mbox{}
\end{enumerate}

\hfill\visible<2->{{\scriptsize a sheep}}\visible<3->{{\scriptsize / two sheep}}

\end{frame}
%%%%%%%%%%%%%%%%%%%%%%%%%%
%%%%%%%%%%%%%%%%%%%%%%%%%%%%%%%%%%%%%%%%%%%%%%%%%%%%
\begin{frame}[plain]{実際の単語で確認しよう \textipa{/Z/}}
\Large
{\small \textipa{/Z/}の音を含む英単語}\hfill{\tiny 0411}\,{\scriptsize \myaudio{./audio/consonant_textesh_textyogh_03.mp3}}

\begin{enumerate}

 \item A\textcolor{NavyBlue}{\bfseries s}ia
\hfill\makebox[80pt][l]{\textipa{/\'eI\textcolor{BurntOrange}{Z}@/}}\hspace{150pt}\mbox{}
\item televi\textcolor{NavyBlue}{\bfseries s}ion%
      \hfill\makebox[80pt][l]{\textipa{/t\'el@v\`I\textcolor{BurntOrange}{Z}@n/}}\hspace{150pt}\mbox{}
 \item u\textcolor{NavyBlue}{\bfseries s}ually
\hfill\makebox[80pt][l]{\textipa{/j\'u:\textcolor{BurntOrange}{Z}u@li/}}\hspace{150pt}\mbox{}
 \item ca\textcolor{NavyBlue}{\bfseries s}ual%
\hfill\makebox[80pt][l]{\textipa{/k\'\ae\textcolor{BurntOrange}{Z}u@l/}}\hspace{150pt}\mbox{}
\item mea\textcolor{NavyBlue}{\bfseries s}ure
\hfill\makebox[80pt][l]{\textipa{/m\'e\textcolor{BurntOrange}{Z}\textrhookschwa /}}\hspace{150pt}\mbox{}
\item plea\textcolor{NavyBlue}{\bfseries s}ure
\hfill\makebox[80pt][l]{\textipa{/pl\'e\textcolor{BurntOrange}{Z}\textrhookschwa /}}\hspace{150pt}\mbox{}
\item trea\textcolor{NavyBlue}{\bfseries s}ure
\hfill\makebox[80pt][l]{\textipa{/tr\'e\textcolor{BurntOrange}{Z}\textrhookschwa /}}\hspace{150pt}\mbox{}
% \item bir\textcolor{NavyBlue}{\bfseries d}
%\hfill\makebox[80pt][l]{\textipa{/b\'\textrhookschwa :\textcolor{BurntOrange}{d}/}}\hspace{150pt}\mbox{}
\end{enumerate}
\end{frame}
%%%%%%%%%%%%%%%%%%%%%%%%%%
%%%%%%%%%%%%%%%%%%%%%%%%%%
\begin{frame}[plain]{Quiz \textipa{/S/}}
\large
発音記号\textipa{/S/}で示した音声が含まれていたらT、含まれていなければFと答えてください。余裕がある人はなんという単語か書き取ってみましょう

\hfill{}{\scriptsize T: true(正しい)\hspace{5pt} F: false(まちがい)}
 \begin{enumerate}
  \item \mbox{}\visible<2->{T}\hspace{1\zw}\visible<7->{she}
  \item \mbox{}\visible<3->{F}\hspace{1\zw}\visible<8->{sea / see }
  \item \mbox{}\visible<4->{T}\hspace{1\zw}\visible<9->{ship}
  \item \mbox{}\visible<5->{F}\hspace{1\zw}\visible<10->{sick}
  \item \mbox{}\visible<6->{T}\hspace{1\zw}\visible<11->{machine}
 \end{enumerate}

\hfill{\tiny 0136}\,{\scriptsize \myaudio{./audio/consonant_textesh_textyogh_04.mp3}}

\end{frame}
%%%%%%%%%%%%%%%%%%%%%%%%
%%%%%%%%%%%%%%%%%%%%%%%%%%
\begin{frame}[plain]{Quiz \textipa{/Z/}}
発音記号\textipa{/Z/}で示した音声が含まれていたらT、含まれていなければFと答えてください。余裕がある人はなんという単語か書き取ってみましょう

\hfill{}{\scriptsize T: true(正しい)\hspace{5pt} F: false(まちがい)}

\Large
 \begin{enumerate}
  \item \mbox{}\visible<2->{T}\hspace{1\zw}\visible<7->{television}
  \item \mbox{}\visible<3->{T}\hspace{1\zw}\visible<8->{Asia}
  \item \mbox{}\visible<4->{F}\hspace{1\zw}\visible<9->{shop}
  \item \mbox{}\visible<5->{F}\hspace{1\zw}\visible<10->{easy}
  \item \mbox{}\visible<6->{T}\hspace{1\zw}\visible<11->{usually}
 \end{enumerate}

\hfill{\tiny 0136}\,{\scriptsize \myaudio{./audio/consonant_textesh_textyogh_04.mp3}}
\end{frame}
%%%%%%%%%%%%%%%%%%%%%%%%
%%%%%%%%%%%%%%%%%%%%%%%%%%%
%%
%% /tS/ /dZ/
%%
%%%%%%%%%%%%%%%%%%%%%%%%%%
% 背景色を黒に変更
\setbeamercolor{background canvas}{bg=black}
\begin{frame}
\centering
  \textcolor{white}{\Huge\bfseries Today's Pronunciation}\pause

 \vspace{30pt}

  \textcolor{white}{\Huge\bfseries \textipa{/tS/}, \textipa{/dZ/}}


\end{frame}
\setbeamercolor{background canvas}{bg=}
%%%%%%%%%%%%%%%%%%%%%%%%%%
ls \begin{frame}[plain,label=slide_ttextesh_dtextyogh]{破擦音\textipa{/tS/と\textipa{/dZ/}}}

\large

\begin{enumerate}
 \item  無声音 \textipa{/tS/}\hspace{20pt}\underLine{ch}ange,\hspace{1\zw}tou\underLine{ch},\hspace{1\zw}ca\underLine{tch}
 \item  有声音 \textipa{/dZ/}\hspace{20pt}\underLine{j}ump,\hspace{1\zw}en\underLine{j}oy,\hspace{1\zw}bri\underLine{dg}e
\end{enumerate}

\vspace*{20pt}

\normalsize
ポイント

\begin{itemize}\setbeamertemplate{items}[circle]
 \item 唇を丸めて「チ」$\rightarrow$\,これで\,\textipa{/tS/}\,のできあがり 
 \item 声を出せば$\rightarrow$これで\,\textipa{/dZ/}\,のできあがり
 \item 舌先を歯茎につけてから破裂させます(\textipa{/Z/}は舌がどこにもつきません)
 \item \textipa{/tS dZ/}を何度も繰り返してみましょう(時間のあるとき100回くらい)
\end{itemize}
\hfill{\tiny 0151}\,{\scriptsize \myaudio{./audio/consonant_ttextesh_dtextyogh_01.mp3}}
\end{frame}
%%%%%%%%%%%%%%%%%%%%%%%%%%%%%%%%%%%%%%%%%%%%%%%%%%%%
%%%%%%%%%%%%%%%%%%%%%%%%%%%%%%%%%%%%%%%%%%%%%%%%%%%%
\begin{frame}[plain]{実際の単語で確認しよう \textipa{/tS/}}
\Large
{\small \textipa{/tS/}の音を含む英単語}\hfill{\tiny 0439}\,{\scriptsize \myaudio{./audio/consonant_ttextesh_dtextyogh_02.mp3}}

\begin{enumerate}

 \item \textcolor{NavyBlue}{\bfseries ch}ange
\hfill\makebox[80pt][l]{\textipa{/\textcolor{BurntOrange}{tS}\'eIndZ/}}\hspace{150pt}\mbox{}
\item \textcolor{NavyBlue}{\bfseries ch}eck%
      \hfill\makebox[80pt][l]{\textipa{/\textcolor{BurntOrange}{tS}\'ek/}}\hspace{150pt}\mbox{}
 \item \textcolor{NavyBlue}{\bfseries ch}eap
\hfill\makebox[80pt][l]{\textipa{/\textcolor{BurntOrange}{tS}\'\i:p/}}\hspace{150pt}\mbox{}
 \item \textcolor{NavyBlue}{\bfseries ch}eese%
\hfill\makebox[80pt][l]{\textipa{/\textcolor{BurntOrange}{tS}\'\i:z/}}\hspace{150pt}\mbox{}
\item \textcolor{NavyBlue}{\bfseries ch}ild
\hfill\makebox[80pt][l]{\textipa{/\textcolor{BurntOrange}{tS}\'aIld/}}\hspace{150pt}\mbox{}
  \item fu\textcolor{NavyBlue}{\bfseries t}ure
\hfill\makebox[80pt][l]{\textipa{/fj\'u:\textcolor{BurntOrange}{tS}\textrhookschwa /}}\hspace{150pt}\mbox{}
\item lun\textcolor{NavyBlue}{\bfseries ch}
\hfill\makebox[80pt][l]{\textipa{/l\'2n\textcolor{BurntOrange}{tS}/}}\hspace{150pt}\mbox{} 
 \item tou\textcolor{NavyBlue}{\bfseries ch}
       \hfill\makebox[80pt][l]{\textipa{/t\'2\textcolor{BurntOrange}{tS}/}}\hspace{150pt}\mbox{}

% \item bir\textcolor{NavyBlue}{\bfseries d}
%\hfill\makebox[80pt][l]{\textipa{/b\'\textrhookschwa :\textcolor{BurntOrange}{d}/}}\hspace{150pt}\mbox{}
\end{enumerate}
\end{frame}
%%%%%%%%%%%%%%%%%%%%%%%%%%
%%%%%%%%%%%%%%%%%%%%%%%%%%%%%%%%%%%%%%%%%%%%%%%%%%%%
\begin{frame}[plain]{実際の単語で確認しよう \textipa{/dZ/}}
\Large
{\small \textipa{/dZ/}の音を含む英単語}\hfill{\tiny 0437}\,{\scriptsize \myaudio{./audio/consonant_ttextesh_dtextyogh_03.mp3}}

\begin{enumerate}
 \item \textcolor{NavyBlue}{\bfseries j}uice
\hfill\makebox[80pt][l]{\textipa{/\textcolor{BurntOrange}{dZ}\'u:s/}}\hspace{150pt}\mbox{}
 \item \textcolor{NavyBlue}{\bfseries j}ump
\hfill\makebox[80pt][l]{\textipa{/\textcolor{BurntOrange}{dZ}\'2mp/}}\hspace{150pt}\mbox{}
  \item \textcolor{NavyBlue}{\bfseries j}oke
\hfill\makebox[80pt][l]{\textipa{/\textcolor{BurntOrange}{dZ}\'oUk/}}\hspace{150pt}\mbox{}
\item en\textcolor{NavyBlue}{\bfseries j}oy
\hfill\makebox[80pt][l]{\textipa{/In\textcolor{BurntOrange}{dZ}\'OI/}}\hspace{150pt}\mbox{}
\item \textcolor{NavyBlue}{\bfseries g}iant%
      \hfill\makebox[80pt][l]{\textipa{/\textcolor{BurntOrange}{dZ}\'aI@nt/}}\hspace{150pt}\mbox{}
\item bri\textcolor{NavyBlue}{\bfseries dg}e
\hfill\makebox[80pt][l]{\textipa{/br\'I\textcolor{BurntOrange}{dZ}/}}\hspace{150pt}\mbox{} 
 \item ba\textcolor{NavyBlue}{\bfseries dg}e
       \hfill\makebox[80pt][l]{\textipa{/b\'\ae \textcolor{BurntOrange}{dZ}/}}\hspace{150pt}\mbox{}
 \item \textcolor{NavyBlue}{\bfseries j}u\textcolor{NavyBlue}{\bfseries dg}e%
\hfill\makebox[80pt][l]{\textipa{/\textcolor{BurntOrange}{dZ}\'2\textcolor{BurntOrange}{dZ}/}}\hspace{150pt}\mbox{}
% \item bir\textcolor{NavyBlue}{\bfseries d}
%\hfill\makebox[80pt][l]{\textipa{/b\'\textrhookschwa :\textcolor{BurntOrange}{d}/}}\hspace{150pt}\mbox{}
\end{enumerate}
\end{frame}
%%%%%%%%%%%%%%%%%%%%%%%%%%
%%%%%%%%%%%%%%%%%%%%%%%%%%
\begin{frame}[plain]{Quiz \textipa{/tS/}}
発音記号\textipa{/tS/}で示した音声が含まれていたらT、含まれていなければFと答えてください。余裕がある人はなんという単語か書き取ってみましょう

\hfill{}{\scriptsize T: true(正しい)\hspace{5pt} F: false(まちがい)}

\Large
 \begin{enumerate}
  \item \mbox{}\visible<2->{T}\hspace{1\zw}\visible<7->{change}
  \item \mbox{}\visible<3->{F}\hspace{1\zw}\visible<8->{party}
  \item \mbox{}\visible<4->{T}\hspace{1\zw}\visible<9->{lunch}
  \item \mbox{}\visible<5->{F}\hspace{1\zw}\visible<10->{tea}
  \item \mbox{}\visible<6->{T}\hspace{1\zw}\visible<11->{cheese}
 \end{enumerate}

\hfill{\tiny 0136}\,{\scriptsize \myaudio{./audio/consonant_ttextesh_dtextyogh_04.mp3}}

\end{frame}
%%%%%%%%%%%%%%%%%%%%%%%%
%%%%%%%%%%%%%%%%%%%%%%%%%%
\begin{frame}[plain]{Quiz \textipa{/dZ/}}
\large
発音記号\textipa{/dZ/}で示した音声が含まれていたらT、含まれていなければFと答えてください。余裕がある人はなんという単語か書き取ってみましょう

\hfill{}{\scriptsize T: true(正しい)\hspace{5pt} F: false(まちがい)}
 \begin{enumerate}
  \item \mbox{}\visible<2->{T}\hspace{1\zw}\visible<7->{judge}
  \item \mbox{}\visible<3->{F}\hspace{1\zw}\visible<8->{television}
  \item \mbox{}\visible<4->{T}\hspace{1\zw}\visible<9->{bridge}
  \item \mbox{}\visible<5->{F}\hspace{1\zw}\visible<10->{visit}
  \item \mbox{}\visible<6->{T}\hspace{1\zw}\visible<11->{juice}
 \end{enumerate}
\hfill{\tiny 0136}\,{\scriptsize \myaudio{./audio/consonant_ttextesh_dtextyogh_05.mp3}}
\end{frame}
%%%%%%%%%%%%%%%%%%%%%%%%
%%%%%%%%%%%%%%%%%%%%%%%%%%%
%%
%% /ts/ /dz/
%%
%%%%%%%%%%%%%%%%%%%%%%%%%%
% 背景色を黒に変更
\setbeamercolor{background canvas}{bg=black}
\begin{frame}
\centering
  \textcolor{white}{\Huge\bfseries Today's Pronunciation}\pause

 \vspace{30pt}

  \textcolor{white}{\Huge\bfseries \textipa{/ts/}, \textipa{/dz/}}


\end{frame}
\setbeamercolor{background canvas}{bg=}
%%%%%%%%%%%%%%%%%%%%%%%%%%
 \begin{frame}[plain,label=slide_ts_dz]{破擦音\textipa{/ts/}\,\,\,\textipa{/dz/}}

\large

\begin{enumerate}
 \item  無声音 \textipa{/ts/}\hspace{20pt}ca\underLine{ts}\hspace{1\zw}donu\underLine{ts}\hspace{1\zw}ea\underLine{ts}\hspace{1\zw}si\underLine{ts}\hspace{1\zw}i\underLine{t's}($=$ it is)
 \item  有声音 \textipa{/dz/}\hspace{20pt}car\underLine{ds}\hspace{1\zw}bir\underLine{ds}\hspace{1\zw}soun\underLine{ds}\hspace{1\zw}frien\underLine{ds}\hspace{1\zw}rea\underLine{ds}
\end{enumerate}

\vspace*{20pt}

\normalsize
ポイント

\begin{itemize}\setbeamertemplate{items}[circle]
 \item 「ツ」の子音部が\textipa{/ts/}
 \item 歯茎にくっついていた舌の先っぽだけが下がったことを確認しましょう
 \item 声を出せば\textipa{/dz/}
 \item \textipa{/ts dz/}を100回繰り返しましょう
\end{itemize}
\hfill{\tiny 0308}\,{\scriptsize \myaudio{./audio/consonant_ts_dz_01.mp3}}
\end{frame}
%%%%%%%%%%%%%%%%%%%%%%%%%%%%%%%%%%%%%%%%%%%%%%%%%%%%
\begin{frame}[plain]{実際の単語で確認しよう \textipa{/ts/}}
\Large
{\small \textipa{/ts/}の音を含む英単語}\hfill{\tiny 0309}\,{\scriptsize \myaudio{./audio/consonant_ts_dz_02.mp3}}

\begin{enumerate}

 \item ca\textcolor{NavyBlue}{\bfseries ts}
\hfill\makebox[80pt][l]{\textipa{/k\'\ae \textcolor{BurntOrange}{ts}/}}\hspace{150pt}\mbox{}
 \item boa\textcolor{NavyBlue}{\bfseries ts}
\hfill\makebox[80pt][l]{\textipa{/b\'oU\textcolor{BurntOrange}{ts}/}}\hspace{150pt}\mbox{}
 \item cen\textcolor{NavyBlue}{\bfseries ts}
\hfill\makebox[80pt][l]{\textipa{/s\'en\textcolor{BurntOrange}{ts}/}}\hspace{150pt}\mbox{}
 \item donu\textcolor{NavyBlue}{\bfseries ts}
\hfill\makebox[80pt][l]{\textipa{/d\'oUn\textturnv\textcolor{BurntOrange}{ts}/}}\hspace{150pt}\mbox{}
 \item peanu\textcolor{NavyBlue}{\bfseries ts}
\hfill\makebox[80pt][l]{\textipa{/p\'\i:n\textturnv\textcolor{BurntOrange}{ts}/}}\hspace{150pt}\mbox{}
 \item par\textcolor{NavyBlue}{\bfseries ts}
\hfill\makebox[80pt][l]{\textipa{/p\'A\textrhookschwa\textcolor{BurntOrange}{ts}/}}\hspace{150pt}\mbox{}
 \item i\textcolor{NavyBlue}{\bfseries t's}
\hfill\makebox[80pt][l]{\textipa{/\'I\textcolor{BurntOrange}{ts}/}}\hspace{150pt}\mbox{}
 \item si\textcolor{NavyBlue}{\bfseries ts}
\hfill\makebox[80pt][l]{\textipa{/s\'I\textcolor{BurntOrange}{ts}/}}\hspace{150pt}\mbox{}
\end{enumerate}

\hfill{}{\scriptsize century, percent}

\end{frame}
%%%%%%%%%%%%%%%%%%%%%%%%%%
%%%%%%%%%%%%%%%%%%%%%%%%%%%%%%%%%%%%%%%%%%%%%%%%%%%%
\begin{frame}[plain]{実際の単語で確認しよう \textipa{/dz/}}
\Large
{\small \textipa{/dz/}の音を含む英単語}\hfill{\tiny 0311}\,{\scriptsize \myaudio{./audio/consonant_ts_dz_03.mp3}}

\begin{enumerate}
 \item car\textcolor{NavyBlue}{\bfseries ds}
\hfill\makebox[80pt][l]{\textipa{/k\'A\textrhookschwa\textcolor{BurntOrange}{dz}/}}\hspace{150pt}\mbox{}
 \item bir\textcolor{NavyBlue}{\bfseries ds}
\hfill\makebox[80pt][l]{\textipa{/b\'\textrhookschwa:\textcolor{BurntOrange}{dz}/}}\hspace{150pt}\mbox{}
 \item roa\textcolor{NavyBlue}{\bfseries ds}
\hfill\makebox[80pt][l]{\textipa{/r\'oU\textcolor{BurntOrange}{dz}/}}\hspace{150pt}\mbox{}
 \item ki\textcolor{NavyBlue}{\bfseries ds}
\hfill\makebox[80pt][l]{\textipa{/k\'\i\textcolor{BurntOrange}{dz}/}}\hspace{150pt}\mbox{}
 \item frien\textcolor{NavyBlue}{\bfseries ds}
\hfill\makebox[80pt][l]{\textipa{/fr\'en\textcolor{BurntOrange}{dz}/}}\hspace{150pt}\mbox{}
 \item rea\textcolor{NavyBlue}{\bfseries ds}
\hfill\makebox[80pt][l]{\textipa{/r\'\i:\textcolor{BurntOrange}{dz}/}}\hspace{150pt}\mbox{}
 \item clou\textcolor{NavyBlue}{\bfseries ds}
\hfill\makebox[80pt][l]{\textipa{/kl\'aU\textcolor{BurntOrange}{dz}/}}\hspace{150pt}\mbox{}
 \item stan\textcolor{NavyBlue}{\bfseries ds}
\hfill\makebox[80pt][l]{\textipa{/st\'\ae n\textcolor{BurntOrange}{dz}/}}\hspace{150pt}\mbox{}
\end{enumerate}
\end{frame}
%%%%%%%%%%%%%%%%%%%%%%%%%%
%%%%%%%%%%%%%%%%%%%%%%%%%%
\begin{frame}[plain]{\textipa{/z/}と\textipa{/dz/}を区別しよう1}
\Large

\textipa{/s/} $\rightarrow$ \textipa{/z/}\hspace{95pt}\pause{}mu\underline{s}ic\hspace{10pt}play\underline{s}

\pause

\textipa{/ts/} $\rightarrow$ \textipa{/dz/}\hspace{80pt}\pause{}bir\underline{ds}\hspace{10pt}ki\underline{ds}

\vfill

\pause

 \begin{enumerate}
  \item cars \textipa{/k\'A\textrhookschwa \textcolor{Maroon}{z}/}\hfill{\scriptsize car \textipa{/k\'A\textrhookschwa /}の複数形}
  \item cards \textipa{/k\'A\textrhookschwa \textcolor{BurntOrange}{dz}/}\hfill{\scriptsize card  \textipa{/k\'A\textrhookschwa d/}の複数形} 

 \end{enumerate}

\hfill{\tiny 0138}\,{\scriptsize \myaudio{./audio/consonant_ts_dz_04.mp3}}
\end{frame}
%%%%%%%%%%%%%%%%%%%%%%%%
\begin{frame}[plain]{\textipa{/z/}と\textipa{/dz/}を区別しよう}

まずは肩慣らし\pause{} いや口慣らし\pause

\Large

\begin{enumerate}
 \item \textipa{/s:::z:::/}\pause
 \item \textipa{/ts dz/}
\end{enumerate}

\end{frame}
%%%%%%%%%%%%%%%%%%%%%%%%
\begin{frame}[plain]{Quiz 1}\large

carsまたはcardsを発音していきます。どちらを発音したか聞き取って線でつなぎましょう

\bigskip

 \begin{columns}[t]
   \begin{column}{.45\textwidth}
    \begin{tabular}{rlr}
     1& \visible<2->{cars}&\myAnch{q1}{white}{\textbullet} \\
     2& \visible<3->{cards}&\myAnch{q2}{white}{\textbullet} \\
     3& \visible<4->{cards}&\myAnch{q3}{white}{\textbullet} \\
     4& \visible<5->{cars}&\myAnch{q4}{white}{\textbullet} \\
     5& \visible<6->{cards}&\myAnch{q5}{white}{\textbullet} 
    \end{tabular}
   \end{column}
%%%%%%%%%%%
   \begin{column}{.45\textwidth}
    \begin{tabular}{ll}
     \myAnch{a1}{white}{\textbullet}& \raisebox{-.5\height}{\scalebox{6}{🂨\,🂫}}\\
     &\\
     \myAnch{a2}{white}{\textbullet}& \raisebox{-.5\height}{\scalebox{4}{\usym{1F697}\,\usym{1F699}}} \\
    \end{tabular}
   \end{column}
 \end{columns}

\begin{tikzpicture}[remember picture, overlay]
 \visible<2->{\draw[ultra thick, orange, ->] (q1.east) to[out=0, in=180] (a2.west);}
 \visible<3->{\draw[ultra thick, olive, ->] (q2.east) to[out=0, in=180] (a1.west);}
 \visible<4->{\draw[ultra thick, olive, ->] (q3.east) to[out=0, in=180] (a1.west);}
 \visible<5->{\draw[ultra thick, orange, ->] (q4.east) to[out=0, in=180] (a2.west);}
 \visible<6->{\draw[ultra thick, olive, ->] (q5.east) to[out=0, in=180] (a1.west);}
\end{tikzpicture}

\hfill{\tiny 0141}\,{\scriptsize \myaudio{./audio/consonant_ts_dz_05.mp3}}

\end{frame}
%%%%%%%%%%%%%%%%%%%%%%%%%%
\begin{frame}[plain]{\textipa{/z/}と\textipa{/dz/}を区別しよう2}
\Large

\textipa{/s:::z:::/}

\textipa{/ts dz/}

\hfill\begin{tikzpicture}
 \duck[tophat, bowtie,%
     %crazyhair=brown!60!black, glasses, eyebrow,
    think={\small \textipa{/z/?}}, speech={\small \textipa{/dz/?}}, laughing,
    water];
\end{tikzpicture}

 \begin{enumerate}
  \item rose \textipa{/r\'oU\textcolor{Maroon}{z}/}\hfill{\scriptsize rose: ばら}
  \item roads \textipa{/r\'oU\textcolor{BurntOrange}{dz}/}\hfill{\scriptsize road  \textipa{/r\'oUd/}の複数形} 

 \end{enumerate}

\hfill{\tiny 0137}\,{\scriptsize \myaudio{./audio/consonant_ts_dz_06.mp3}}

\end{frame}
%%%%%%%%%%%%%%%%%%%%%%%%
%%%%%%%%%%%%%%%%%%%%%%%%
\begin{frame}[plain]{Quiz 2}\large

rose(ばら)またはroads(道路の複数形)を発音していきます。どちらを発音したか聞き取って線でつなぎましょう

\bigskip

 \begin{columns}[t]
   \begin{column}[T]{.4\textwidth}
    \begin{tabular}{rlr}
     1& \visible<2->{\Large rose}&\myAnch{q1}{white}{\textbullet} \\
     2& \visible<3->{\Large roads}&\myAnch{q2}{white}{\textbullet} \\
     3& \visible<4->{\Large rose}&\myAnch{q3}{white}{\textbullet} \\
     4& \visible<5->{\Large roads}&\myAnch{q4}{white}{\textbullet} \\
     5& \visible<6->{\Large roads}&\myAnch{q5}{white}{\textbullet} 
    \end{tabular}
   \end{column}
%%%%%%%%%%%
   \begin{column}[T]{.55\textwidth}
    \begin{tabular}{ll}
     \myAnch{a1}{white}{\textbullet}& 道路 \\\\
     \myAnch{a2}{white}{\textbullet}& \IfFileExists{./images/rose.jpg}{\raisebox{-.5\height}{\scalebox{.15}{\includegraphics{./images/rose.jpg}}}}{ばら}\\
     &\\
    \end{tabular}
   \end{column}
 \end{columns}

\raggedleft
{\tiny ``Flowers \& Roses'' by SoulRiser is licensed under CC BY-SA 2.0. To view a copy of this license,}\\[-5pt]
{\tiny visit \url{https://creativecommons.org/licenses/by-sa/2.0/?ref=openverse}.}


\begin{tikzpicture}[remember picture, overlay]
 \visible<2->{\draw[ultra thick, orange, ->] (q1.east) to[out=0, in=180] (a2.west);}
 \visible<3->{\draw[ultra thick, olive, ->] (q2.east) to[out=0, in=180] (a1.west);}
 \visible<4->{\draw[ultra thick, orange, ->] (q3.east) to[out=0, in=180] (a2.west);}
 \visible<5->{\draw[ultra thick, olive, ->] (q4.east) to[out=0, in=180] (a1.west);}
 \visible<6->{\draw[ultra thick, olive, ->] (q5.east) to[out=0, in=180] (a1.west);}
\end{tikzpicture}

\vspace{-15pt}

\hfill{\tiny 0137}{\scriptsize \myaudio{./audio/consonant_ts_dz_07.mp3}}

\end{frame}
%%%%%%%%%%%%%%%%%%%%%%%%%%%%%%%%%%%%%%%%%%%%%%%%%%%%
%%%%%%%%%%%%%%%%%%%%%%%%
\begin{frame}[plain]{Quiz 3}\large

rose(ばらの単数形)、roads(道路の複数形)、cars(自動車の複数形)またはcards(カードの複数形)を順不同で発音していきます。どれを発音したのか聞き取って線でつなぎましょう

\bigskip

 \begin{columns}[t]
   \begin{column}[T]{.4\textwidth}
    \begin{tabular}{rlr}
     1& \visible<2->{\Large rose}&\myAnch{q1}{white}{\textbullet} \\
     2& \visible<3->{\Large cards}&\myAnch{q2}{white}{\textbullet} \\
     3& \visible<4->{\Large cars}&\myAnch{q3}{white}{\textbullet} \\
     4& \visible<5->{\Large roads}&\myAnch{q4}{white}{\textbullet} \\
    \end{tabular}
   \end{column}
%%%%%%%%%%%
   \begin{column}[T]{.55\textwidth}
    \begin{tabular}{ll}
     \myAnch{a1}{white}{\textbullet}& 道路の複数形 \\
     \myAnch{a2}{white}{\textbullet}& カードの複数形\\
     \myAnch{a3}{white}{\textbullet}& 自動車の複数形 \\
     \myAnch{a4}{white}{\textbullet}& ばらの単数形\\
     &\\
    \end{tabular}
   \end{column}
 \end{columns}

\begin{tikzpicture}[remember picture, overlay]
 \visible<2->{\draw[ultra thick, orange, ->] (q1.east) to[out=0, in=180] (a4.west);}
 \visible<3->{\draw[ultra thick, olive, ->] (q2.east) to[out=0, in=180] (a2.west);}
 \visible<4->{\draw[ultra thick, NavyBlue, ->] (q3.east) to[out=0, in=180] (a3.west);}
 \visible<5->{\draw[ultra thick, Maroon, ->] (q4.east) to[out=0, in=180] (a1.west);}
\end{tikzpicture}

\vspace{-15pt}

\hfill{\tiny 0117}{\scriptsize \myaudio{./audio/consonant_ts_dz_08.mp3}}

%\hyperlink{tips}{\beamerreturnbutton{Back to Tips}}
\end{frame}
%%%%%%%%%%%%%%%%%%%%%%%%%%%%%%%%%%%%%%%%%%%%%%%%%%%%
%%%%%%%%%%%%%%%%%%%%%%%%%%%
%%
%% /m/ /n/ /N/
%%
%%%%%%%%%%%%%%%%%%%%%%%%%%
% 背景色を黒に変更
\setbeamercolor{background canvas}{bg=black}
\begin{frame}
\centering
  \textcolor{white}{\Huge\bfseries Today's Pronunciation}\pause

 \vspace{30pt}

  \textcolor{white}{\Huge\bfseries \textipa{/m/}, \textipa{/n/}, \textipa{/N/}}


\vfill
%\hfill\hyperlink{tips}{\beamergotobutton{Excercises}}

\end{frame}
\setbeamercolor{background canvas}{bg=}
%%%%%%%%%%%%%%%%%%%%%%%%%%
%%%%%%%%%%%%%%%%%%%%%%%%%%
 \begin{frame}[plain]{鼻音\,\,\textipa{/m/}\,\,\,\,\textipa{/n/}\,\,\,\,\textipa{/N/}}

\large

\vspace*{-20pt}

\textipa{/m/}\hfill{}せんべいの`ん'が\textipa{/m/} $\longleftarrow$\,マ行の子音部
\begin{tikzpicture}[baseline=20pt]
 \duck[tophat, bowtie,%
     %crazyhair=brown!60!black, glasses, eyebrow,
     signpost=\scalebox{0.4}{
\parbox{2cm}{\color{black}
\centering \textipa{/m/}}},
signcolour=brown!70!gray,
signback=white!80!brown,
speech=\scalebox{0.4}{
\parbox{2cm}{\color{black}
\centering マ行の\\ 子音部}},
    think={\scriptsize かんたん}!,
     laughing,
     water];
\end{tikzpicture}

\vspace*{-20pt}
\begin{enumerate}
 \item \underLine{m}ake
 \item \underLine{m}eet
 \item \underLine{m}an
 \item \underLine{m}other
 \item su\underLine{mm}er
 \item co\underLine{m}e
 \item so\underLine{m}e
\end{enumerate}

\normalsize


{\tiny 0432}\,{\scriptsize \myaudio{./audio/consonant_m_n_ng_01.mp3}}
\end{frame}
%%%%%%%%%%%%%%%%%%%%%%%%%%%%%%%%%%%%
 \begin{frame}[plain]{鼻音\,\,\textipa{/m/}\,\,\,\,\textipa{/n/}\,\,\,\,\textipa{/N/}}

\large

\vspace*{-20pt}

\textipa{/n/}\hfill{}あんドーナツの`ん'が\textipa{/n/} $\longleftarrow$\,ナヌネノの子音部
\begin{tikzpicture}[baseline=10pt]
 \duck[tophat, bowtie,%
     %crazyhair=brown!60!black, glasses, eyebrow,
     signpost=\scalebox{0.5}{
\parbox{2cm}{\color{black}
\centering \textipa{/n/}}},
signcolour=brown!70!gray,
signback=white!80!brown,
speech=\scalebox{0.4}{
\parbox{2.2cm}{\color{black}
\centering ナヌネノの\\ 子音部}},
    think={\scriptsize 舌}!,
     laughing,
     water];
\end{tikzpicture}

\vspace{-20pt}

\begin{enumerate}
 \item \underLine{n}ight
 \item \underLine{n}ot
 \item \underLine{n}ice
 \item \underLine{n}i\underLine{n}e
 \item ru\underLine{n}
 \item kitche\underLine{n}
 \item \underLine{n}eed
\end{enumerate}


\vspace*{20pt}

\normalsize

{\tiny 0432}\,{\scriptsize \myaudio{./audio/consonant_m_n_ng_02.mp3}}
%\hfill%
%\hypertarget{tips}{}
\end{frame}
%%%%%%%%%%%%%%%%%%%%%%%%%%%%%%%%%%%%
 \begin{frame}[plain]{鼻音\,\,\textipa{/m/}\,\,\,\,\textipa{/n/}\,\,\,\,\textipa{/N/}}

\large

\textipa{/N/}\pause\hfill{}トンカツの`ン'が\textipa{/N/}
\begin{tikzpicture}[baseline=10pt]
 \duck[tophat, bowtie,%
     %crazyhair=brown!60!black, glasses, eyebrow,
     signpost=\scalebox{0.5}{
\parbox{2cm}{\color{black}
\centering \textipa{/N/}}},
signcolour=brown!70!gray,
signback=white!80!brown,
speech=\scalebox{0.4}{
\parbox{2.2cm}{\color{black}
\centering トンカツ}},
     laughing,
     water];
\end{tikzpicture}

\vspace{-20pt}

\pause
\begin{enumerate}
 \item ki\underLine{ng}
 \item si\underline{ng}
 \item wro\underLine{ng}
 \item everythi\underLine{ng}
 \item runni\underLine{ng}
 \item pi\underLine{n}k
 \item i\underLine{n}k
\end{enumerate}


\vspace*{20pt}

\normalsize

{\tiny 0433}\,{\scriptsize \myaudio{./audio/consonant_m_n_ng_03.mp3}}
\end{frame}
%%%%%%%%%%%%%%%%%
%%%%%%%%%%%%%%%%%%%%%%%%
\begin{frame}[plain]{Quiz 1}\large

some(いくつかの)とsun(太陽)を発音していきます。どちらを発音したか聞き取って線でつなぎましょう\hfill{\scriptsize some books(何冊かの本)}

\bigskip

 \begin{columns}[t]
   \begin{column}{.45\textwidth}
    \begin{tabular}{rlr}
     1& \visible<2->{sun}&\myAnch{q1}{white}{\textbullet} \\
     2& \visible<3->{some}&\myAnch{q2}{white}{\textbullet} \\
     3& \visible<4->{some}&\myAnch{q3}{white}{\textbullet} \\
     4& \visible<5->{sun}&\myAnch{q4}{white}{\textbullet} \\
     5& \visible<6->{some}&\myAnch{q5}{white}{\textbullet} 
    \end{tabular}
   \end{column}
%%%%%%%%%%%
   \begin{column}{.45\textwidth}
    \begin{tabular}{ll}
     \myAnch{a1}{white}{\textbullet}& some \textipa{/s\'\textturnv m/}\\
     &\\
     \myAnch{a2}{white}{\textbullet}& sun \textipa{/s\'\textturnv n/}\\
    \end{tabular}
   \end{column}
 \end{columns}

\begin{tikzpicture}[remember picture, overlay]
\tikzset{hoge/.style = {line width=4pt, ->, opacity=.6}}
 \visible<2->{\draw[hoge,orange] (q1.east) to[out=0, in=180] (a2.west);}
 \visible<3->{\draw[hoge,olive] (q2.east) to[out=0, in=180] (a1.west);}
 \visible<4->{\draw[hoge,olive] (q3.east) to[out=0, in=180] (a1.west);}
 \visible<5->{\draw[hoge,orange] (q4.east) to[out=0, in=180] (a2.west);}
 \visible<6->{\draw[hoge,olive] (q5.east) to[out=0, in=180] (a1.west);}
\end{tikzpicture}

\hfill{}\visible<7->{{\scriptsize ところでsun(太陽)とson(息子)は同じ発音です}}

\hfill{\tiny 0128}{\scriptsize \myaudio{./audio/consonant_m_n_ng_04.mp3}}

\end{frame}
%%%%%%%%%%%%%%%%%%%%%%%%%%
\begin{frame}[plain]{Quiz 2}\large

win(勝つ)とwing(つばさ)を発音していきます。どちらを発音したか聞き取って線でつなぎましょう

\bigskip

 \begin{columns}[t]
   \begin{column}{.45\textwidth}
    \begin{tabular}{rlr}
     1& \visible<2->{win}&\myAnch{q1}{white}{\textbullet} \\
     2& \visible<3->{wing}&\myAnch{q2}{white}{\textbullet} \\
     3& \visible<4->{win}&\myAnch{q3}{white}{\textbullet} \\
     4& \visible<5->{win}&\myAnch{q4}{white}{\textbullet} \\
     5& \visible<6->{wing}&\myAnch{q5}{white}{\textbullet} 
    \end{tabular}
   \end{column}
%%%%%%%%%%%
   \begin{column}{.45\textwidth}
    \begin{tabular}{ll}
     \myAnch{a1}{white}{\textbullet}& win \textipa{/w\'In/}\\
     &\\
     \myAnch{a2}{white}{\textbullet}& wing \textipa{/w\'IN/}\\
    \end{tabular}
   \end{column}
 \end{columns}

\begin{tikzpicture}[remember picture, overlay]
\tikzset{hoge/.style = {line width=4pt, ->, opacity=.6}}
 \visible<2->{\draw[hoge,Maroon] (q1.east) to[out=0, in=180] (a1.west);}
 \visible<3->{\draw[hoge,NavyBlue] (q2.east) to[out=0, in=180] (a2.west);}
 \visible<4->{\draw[hoge,Maroon] (q3.east) to[out=0, in=180] (a1.west);}
 \visible<5->{\draw[hoge,Maroon] (q4.east) to[out=0, in=180] (a1.west);}
 \visible<6->{\draw[hoge,NavyBlue] (q5.east) to[out=0, in=180] (a2.west);}
\end{tikzpicture}

\hfill{\tiny 0130}\,{\scriptsize \myaudio{./audio/consonant_m_n_ng_05.mp3}}

\end{frame}
%%%%%%%%%%%%%%%%%%%%%%%%
\begin{frame}[plain]{Quiz 3}\large

鼻音\textipa{/m/}, \textipa{/n/}, \textipa{/N/}の含まれる単語を発音していきます。どの鼻音が含まれていたか聞き取り、線でつなぎましょう

\bigskip

 \begin{columns}[t]
   \begin{column}{.45\textwidth}
    \begin{tabular}{rlr}
     1& \visible<2->{ten}&\myAnch{q1}{white}{\textbullet} \\
     2& \visible<3->{long}&\myAnch{q2}{white}{\textbullet} \\
     3& \visible<4->{time}&\myAnch{q3}{white}{\textbullet} \\
     4& \visible<5->{sing}&\myAnch{q4}{white}{\textbullet} \\
     5& \visible<6->{pen}&\myAnch{q5}{white}{\textbullet} \\
     6& \visible<7->{come}&\myAnch{q6}{white}{\textbullet} 
    \end{tabular}
   \end{column}
%%%%%%%%%%%
   \begin{column}{.45\textwidth}
    \begin{tabular}{ll}
     \myAnch{a1}{white}{\textbullet}& \textipa{/m/}\\
     &\\
     \myAnch{a2}{white}{\textbullet}& \textipa{/n/} \\
     &\\
     \myAnch{a3}{white}{\textbullet}& \textipa{/N/} 
    \end{tabular}
   \end{column}
 \end{columns}

\begin{tikzpicture}[remember picture, overlay]
\tikzset{hoge/.style = {line width=4pt, ->, opacity=.6}}
 \visible<2->{\draw[hoge,NavyBlue] (q1.east) to[out=0, in=180] (a2.west);}
 \visible<3->{\draw[hoge, Maroon] (q2.east) to[out=0, in=180] (a3.west);}
 \visible<4->{\draw[hoge, olive] (q3.east) to[out=0, in=180] (a1.west);}
 \visible<5->{\draw[hoge, Maroon] (q4.east) to[out=0, in=180] (a3.west);}
 \visible<6->{\draw[hoge, NavyBlue] (q5.east) to[out=0, in=180] (a2.west);}
 \visible<7->{\draw[hoge, olive] (q6.east) to[out=0, in=180] (a1.west);}
\end{tikzpicture}

\hfill{\tiny 0145}\,{\scriptsize \myaudio{./audio/consonant_m_n_ng_06.mp3}}

\end{frame}
%%%%%%%%%%%%%%%%%%%%%%%%%%
%%%%%%%%%%%%%%%%%%%%%%%%%%%
%%
%% /l/ /r/
%%
%%%%%%%%%%%%%%%%%%%%%%%%%%
% 背景色を黒に変更
\setbeamercolor{background canvas}{bg=black}
\begin{frame}
\centering
  \textcolor{white}{\Huge\bfseries Today's Pronunciation}\pause

 \vspace{30pt}

  \textcolor{white}{\Huge\bfseries \textipa{/l/}, \textipa{/r/}}
\end{frame}
\setbeamercolor{background canvas}{bg=}
%%%%%%%%%%%%%%%%%%%%%%%%%%
\begin{frame}[plain,label=slide_l_r]{\textipa{/l/}\,\,\,\textipa{/r/}}

\large

\begin{enumerate}
 \item  \textipa{/l/}\hspace{20pt}\underLine{l}ook\hspace{1\zw}\underLine{l}ight\hspace{1\zw}\underLine{l}ong\hspace{1\zw}\underLine{l}ake\hspace{1\zw}\underLine{l}ate
 \item  \textipa{/r/}\hspace{20pt}\underLine{r}ight\hspace{1\zw}\underLine{r}oad\hspace{1\zw}\underLine{r}ain\hspace{1\zw}\underLine{r}oom\hspace{1\zw}\underLine{r}ead
\end{enumerate}

\vspace*{20pt}

\normalsize
ポイント

\begin{description}
 \item[\textipa{/l/}] 舌先を上の前歯の後ろにベタっとつけます
 \item[\textipa{/r/}] 舌を口の中で少し浮かせる(舌はどこにも触れない)
\end{description}
\hfill\myaudio{./audio/consonant_l_r_01.mp3}
\end{frame}
%%%%%%%%%%%%%%%%%%%%%%%%%%
\begin{frame}[plain]{実際の単語で確認しよう \textipa{/l/}}
\textipa{/l/}\hspace{20pt}舌先を上の前歯の後ろにベタっとつけます%
\hfill{\scriptsize \myaudio{./audio/consonant_l_r_02.mp3}}
\LARGE
\begin{enumerate}
 \item \textcolor{NavyBlue}{\bfseries l}ike%
\hfill\makebox[80pt][l]{\textipa{/\textcolor{BurntOrange}{l}\'aIk/}}\hspace{150pt}\mbox{}
 \item \textcolor{NavyBlue}{\bfseries l}ight
\hfill\makebox[80pt][l]{\textipa{/\textcolor{BurntOrange}{l}\'aIt/}}\hspace{150pt}\mbox{}
 \item \textcolor{NavyBlue}{\bfseries l}isten
\hfill\makebox[80pt][l]{\textipa{/\textcolor{BurntOrange}{l}\'Isn/}}\hspace{150pt}\mbox{}
 \item \textcolor{NavyBlue}{\bfseries l}emon
\hfill\makebox[80pt][l]{\textipa{/\textcolor{BurntOrange}{l}\'em@n/}}\hspace{150pt}\mbox{}
 \item \textcolor{NavyBlue}{\bfseries l}ake
\hfill\makebox[80pt][l]{\textipa{/\textcolor{BurntOrange}{l}\'eIk/}}\hspace{150pt}\mbox{}
 \item \textcolor{NavyBlue}{\bfseries l}ook
\hfill\makebox[80pt][l]{\textipa{/\textcolor{BurntOrange}{l}\'Uk/}}\hspace{150pt}\mbox{}
 \item \textcolor{NavyBlue}{\bfseries l}unch
\hfill\makebox[80pt][l]{\textipa{/\textcolor{BurntOrange}{l}\'\textturnv ntS/}}\hspace{150pt}\mbox{}
\end{enumerate}
\end{frame}
%%%%%%%%%%%%%%%%%%%%%%%%%%%%%%%%%%%%%%%%%%%%%%%%%%%%
\begin{frame}[plain]{実際の単語で確認しよう \textipa{/r/}}
\textipa{/r/}\hspace{20pt}舌を口の中で少し浮かせる(舌はどこにもに触れない)%
\hfill{\scriptsize \myaudio{./audio/consonant_l_r_03.mp3}}
\LARGE
\begin{enumerate}
 \item \textcolor{NavyBlue}{\bfseries r}ead%
\hfill\makebox[80pt][l]{\textipa{/\textcolor{BurntOrange}{r}\'\i:d/}}\hspace{150pt}\mbox{}
 \item \textcolor{NavyBlue}{\bfseries r}ight
\hfill\makebox[80pt][l]{\textipa{/\textcolor{BurntOrange}{r}\'aIt/}}\hspace{150pt}\mbox{}
 \item \textcolor{NavyBlue}{\bfseries r}ain
\hfill\makebox[80pt][l]{\textipa{/\textcolor{BurntOrange}{r}\'eIn/}}\hspace{150pt}\mbox{}
 \item \textcolor{NavyBlue}{\bfseries r}ed
\hfill\makebox[80pt][l]{\textipa{/\textcolor{BurntOrange}{r}\'ed/}}\hspace{150pt}\mbox{}
 \item \textcolor{NavyBlue}{\bfseries r}iver
\hfill\makebox[80pt][l]{\textipa{/\textcolor{BurntOrange}{r}\'\i v\textrhookschwa /}}\hspace{150pt}\mbox{}
\item w\textcolor{NavyBlue}{\bfseries r}ite
\hfill\makebox[80pt][l]{\textipa{/\textcolor{BurntOrange}{r}\'aIt/}}\hspace{150pt}\mbox{}
\item w\textcolor{NavyBlue}{\bfseries r}ong
\hfill\makebox[80pt][l]{\textipa{/\textcolor{BurntOrange}{r}\'O:N/}}\hspace{150pt}\mbox{}
\end{enumerate}
\end{frame}
%%%%%%%%%%%%%%%%%%%%%%%%%%%%%%%%
\begin{frame}[plain]{\textipa{/l/}と\textipa{/r/}を区別しよう1}

\begin{description}
 \item[\textipa{/l/}] 舌先を上の前歯の後ろにベタっとつけます
 \item[\textipa{/r/}] 舌を口の中で少し浮かせる(舌はどこにも触れない)\\
	    「ダ」と何度も$\rightarrow$舌をそり返す$\rightarrow$舌を少し離す%
	    $\rightarrow$唇を丸めて突き出す
\end{description}

\Large 
 \begin{enumerate}
  \item light\hfill\textipa{/l\'aIt/}\hspace{250pt}\mbox{}
  \item right\hfill\textipa{/r\'aIt/}\hspace{250pt}\mbox{}

 \end{enumerate}

\hfill{\scriptsize \myaudio{./audio/consonant_l_r_04.mp3}}

\end{frame}
%%%%%%%%%%%%%%%%%%%%%%%%
\begin{frame}[plain]{Quiz 1}\large

light(光)またはright(右)を発音していきます。どちらを発音したか聞き取って線でつなぎましょう

\bigskip

 \begin{columns}[t]
   \begin{column}[T]{.45\textwidth}
    \begin{tabular}{rlr}
     1& \visible<2->{\Large right}&\myAnch{q1}{white}{\textbullet} \\
     2& \visible<3->{\Large light}&\myAnch{q2}{white}{\textbullet} \\
     3& \visible<4->{\Large right}&\myAnch{q3}{white}{\textbullet} \\
     4& \visible<5->{\Large light}&\myAnch{q4}{white}{\textbullet} \\
     5& \visible<6->{\Large light}&\myAnch{q5}{white}{\textbullet} 
    \end{tabular}
   \end{column}
%%%%%%%%%%%
   \begin{column}[T]{.5\textwidth}
    \begin{tabular}{ll}
     \myAnch{a1}{white}{\textbullet}& light \\\\
     \myAnch{a2}{white}{\textbullet}& right\\
     &\\
    \end{tabular}
   \end{column}
 \end{columns}

\begin{tikzpicture}[remember picture, overlay]
\tikzset{hoge/.style = {line width=4pt, ->, opacity=.6}}
 \visible<2->{\draw[hoge,orange] (q1.east) to[out=0, in=180] (a2.west);}
 \visible<3->{\draw[hoge,olive] (q2.east) to[out=0, in=180] (a1.west);}
 \visible<4->{\draw[hoge,orange] (q3.east) to[out=0, in=180] (a2.west);}
 \visible<5->{\draw[hoge,olive] (q4.east) to[out=0, in=180] (a1.west);}
 \visible<6->{\draw[hoge, olive] (q5.east) to[out=0, in=180] (a1.west);}
\end{tikzpicture}

\vspace{-15pt}

\hfill{\scriptsize \myaudio{./audio/consonant_l_r_05.mp3}}

\end{frame}
%%%%%%%%%%%%%%%%%%%%%%%%
%%%%%%%%%%%%%%%%%%%%%%%%%%%%%%%%
\begin{frame}[plain]{\textipa{/l/}と\textipa{/r/}を区別しよう2}
\begin{description}
 \item[\textipa{/l/}] 舌先を上の前歯の後ろにベタっとつけます
 \item[\textipa{/r/}] 舌を口の中で少し浮かせる(舌はどこにも触れない)\\
	    「ダ」と何度も$\rightarrow$舌をそり返す$\rightarrow$舌を少し離す%
	    $\rightarrow$唇を丸めて突き出す\end{description}

\Large 

 \begin{enumerate}
  \item long\hfill\textipa{/l\'O:N/}\hspace{250pt}\mbox{}
  \item wrong\hfill\textipa{/r\'O:N/}\hspace{250pt}\mbox{}

 \end{enumerate}
\hfill{\scriptsize \myaudio{./audio/consonant_l_r_06.mp3}}
\end{frame}
%%%%%%%%%%%%%%%%%%%%%%%%
\begin{frame}[plain]{Quiz 2}\large

longまたはwrongを発音していきます。どちらを発音したか聞き取って線でつなぎましょう

\bigskip

 \begin{columns}[t]
   \begin{column}{.45\textwidth}
    \begin{tabular}{rlr}
     1& \visible<2->{long}&\myAnch{q1}{white}{\textbullet} \\
     2& \visible<3->{wrong}&\myAnch{q2}{white}{\textbullet} \\
     3& \visible<4->{wrong}&\myAnch{q3}{white}{\textbullet} \\
     4& \visible<5->{long}&\myAnch{q4}{white}{\textbullet} \\
     5& \visible<6->{wrong}&\myAnch{q5}{white}{\textbullet} 
    \end{tabular}
   \end{column}
%%%%%%%%%%%
   \begin{column}{.45\textwidth}
    \begin{tabular}{ll}
     \myAnch{a1}{white}{\textbullet}& wrong\\
     &\\
     \myAnch{a2}{white}{\textbullet}& long \\
    \end{tabular}
   \end{column}
 \end{columns}

\begin{tikzpicture}[remember picture, overlay]
\tikzset{hoge/.style = {line width=4pt, ->, opacity=.6}}
 \visible<2->{\draw[hoge, Maroon] (q1.east) to[out=0, in=180] (a2.west);}
 \visible<3->{\draw[hoge, NavyBlue] (q2.east) to[out=0, in=180] (a1.west);}
 \visible<4->{\draw[hoge, NavyBlue] (q3.east) to[out=0, in=180] (a1.west);}
 \visible<5->{\draw[hoge, Maroon] (q4.east) to[out=0, in=180] (a2.west);}
 \visible<6->{\draw[hoge, NavyBlue] (q5.east) to[out=0, in=180] (a1.west);}
\end{tikzpicture}

\hfill\myaudio{./audio/consonant_l_r_07.mp3}

\end{frame}
%%%%%%%%%%%%%%%%%%%%%%%%%%%
%%%%%%%%%%%%%%%%%%%%%%%%%%%%%%%%
\begin{frame}[plain]{\textipa{/l/}と\textipa{/r/}を区別しよう3}

\begin{description}
 \item[\textipa{/l/}] 舌先を上の前歯の後ろにベタっとつけます
 \item[\textipa{/r/}] 舌を口の中で少し浮かせる(舌はどこにも触れない)\\
	    「ダ」と何度も$\rightarrow$舌をそり返す$\rightarrow$舌を少し離す%
	    $\rightarrow$唇を丸めて突き出す\end{description}

\Large 

 \begin{enumerate}
  \item lead\hfill\textipa{/l\'\i:d/}\hspace{250pt}\mbox{}
  \item read\hfill\textipa{/r\'\i:d/}\hspace{250pt}\mbox{}

 \end{enumerate}
\hfill{\scriptsize \myaudio{./audio/consonant_l_r_08.mp3}}
\end{frame}
%%%%%%%%%%%%%%%%%%%%%%%%
\begin{frame}[plain]{Quiz 3}\large

leadまたはreadを発音していきます。どちらを発音したか聞き取って線でつなぎましょう

\bigskip

 \begin{columns}[t]
   \begin{column}{.45\textwidth}
    \begin{tabular}{rlr}
     1& \visible<2->{read}&\myAnch{q1}{white}{\textbullet} \\
     2& \visible<3->{lead}&\myAnch{q2}{white}{\textbullet} \\
     3& \visible<4->{lead}&\myAnch{q3}{white}{\textbullet} \\
     4& \visible<5->{read}&\myAnch{q4}{white}{\textbullet} \\
     5& \visible<6->{lead}&\myAnch{q5}{white}{\textbullet} 
    \end{tabular}
   \end{column}
%%%%%%%%%%%
   \begin{column}{.45\textwidth}
    \begin{tabular}{ll}
     \myAnch{a1}{white}{\textbullet}& lead\\
     &\\
     \myAnch{a2}{white}{\textbullet}& read \\
    \end{tabular}
   \end{column}
 \end{columns}

\begin{tikzpicture}[remember picture, overlay]
\tikzset{hoge/.style = {line width=4pt, ->, opacity=.6}}
 \visible<2->{\draw[hoge, Maroon] (q1.east) to[out=0, in=180] (a2.west);}
 \visible<3->{\draw[hoge, NavyBlue] (q2.east) to[out=0, in=180] (a1.west);}
 \visible<4->{\draw[hoge, NavyBlue] (q3.east) to[out=0, in=180] (a1.west);}
 \visible<5->{\draw[hoge, Maroon] (q4.east) to[out=0, in=180] (a2.west);}
 \visible<6->{\draw[hoge, NavyBlue] (q5.east) to[out=0, in=180] (a1.west);}
\end{tikzpicture}

\hfill\myaudio{./audio/consonant_l_r_09.mp3}

\end{frame}
%%%%%%%%%%%%%%%%%%%%%%%%%%%
%%
%% /j/
%%
%%%%%%%%%%%%%%%%%%%%%%%%%%
% 背景色を黒に変更
\setbeamercolor{background canvas}{bg=black}
\begin{frame}
\centering
  \textcolor{white}{\Huge\bfseries Today's Pronunciation}\pause

 \vspace{30pt}

  \textcolor{white}{\Huge\bfseries \textipa{/j/}}

\end{frame}
\setbeamercolor{background canvas}{bg=}
%%%%%%%%%%%%%%%%%%%%%%%%%%
\begin{frame}[plain,label=slide_j]{\textipa{/j/}}

\large

\begin{enumerate}
 \item \underLine{y}oung%
\hfill\makebox[80pt][l]{\textipa{/\textcolor{NavyBlue}{j}\'\textturnv N/}}\hspace{150pt}\mbox{}
 \item \underLine{y}acht%
\hfill\makebox[80pt][l]{\textipa{/\textcolor{NavyBlue}{j}\'At/}}\hspace{150pt}\mbox{}
 \item \underLine{y}ou%
\hfill\makebox[80pt][l]{\textipa{/\textcolor{NavyBlue}{j}\'u:/}}\hspace{150pt}\mbox{}
 \item \underLine{y}ogurt%
\hfill\makebox[80pt][l]{\textipa{/\textcolor{NavyBlue}{j}\'oUg\textrhookschwa t/}}\hspace{150pt}\mbox{}
 \item \underLine{y}es%
\hfill\makebox[80pt][l]{\textipa{/\textcolor{BurntOrange}{j}\'es/}}\hspace{150pt}\mbox{}
 \item \underLine{y}esterday%
\hfill\makebox[80pt][l]{\textipa{/\textcolor{BurntOrange}{j}\'est\textrhookschwa d\`eI/}}\hspace{150pt}\mbox{}
 \item \underLine{y}ellow%
\hfill\makebox[80pt][l]{\textipa{/\textcolor{BurntOrange}{j}\'eloU/}}\hspace{150pt}\mbox{}
 \item \underLine{y}ear%
\hfill\makebox[80pt][l]{\textipa{/\textcolor{Maroon}{j}\'I\textrhookschwa /}}\hspace{150pt}\mbox{}
\end{enumerate}

%\vspace*{5pt}

\small
ポイント

\begin{enumerate}
 \item 「ヤユヨ」の子音部
 \item \textipa{/j/} $+$ 「ア」「ウ」「オ」に類する母音
 \item \textipa{/j/} $+$ 「イ」「エ」に類する母音%
\hfill{\scriptsize \myaudio{./audio/consonant_j_01.mp3}}
\end{enumerate}
\end{frame}
%%%%%%%%%%%%%%%%%%%%%%%%%%%%%%%%
\begin{frame}[plain]{Question}
 
つぎの日本語の誤りを修正してください

\bigskip

\Huge
\tikz[baseline=(hiyashi.base)]{\node[draw, fill=yellow!50] (hiyashi) {謹賀新耳};}

\end{frame}
%%%%%%%%%%%%%%%%%%%%%%%%%%%%%%%%
\begin{frame}[plain,label=ear_year_1]{ear(耳)とyear(年)を区別しよう}
\Large

\begin{tikzpicture}[baseline=0pt]
 \duck[tophat, bowtie,%
     %crazyhair=brown!60!black, glasses, eyebrow,
     signpost=\scalebox{0.4}{
\parbox{2cm}{\color{black}
\centering \textipa{/j/}}},
signcolour=brown!70!gray,
signback=white!80!brown,
speech=\scalebox{0.4}{
\parbox{2cm}{\color{black}
\centering ヤユヨの\\ 子音部}},
%    think={\scriptsize かんたん}!,
     laughing,
     water];
\end{tikzpicture}%
\hspace{20pt}{\small ヒッヒッヒッ}$\longrightarrow$ {\small 声を出して} \textipa{/j/}


 \begin{enumerate}
  \item ear\hfill\textipa{/\'I\textrhookschwa /}\hspace{250pt}\mbox{}
  \item year\hfill\textipa{/j\'I\textrhookschwa /}\hspace{250pt}\mbox{}
 \end{enumerate}

\hfill{\scriptsize \myaudio{./audio/consonant_j_02.mp3}}

\end{frame}
%%%%%%%%%%%%%%%%%%%%%%%%
%%%%%%%%%%%%%%%%%%%%%%%%
\begin{frame}[plain,label=quiz_ear_year]{Quiz 1}\large

earまたはyearを発音していきます。どちらを発音したか聞き取って線でつなぎましょう

\bigskip

 \begin{columns}[t]
   \begin{column}{.45\textwidth}
    \begin{tabular}{rlr}
     1& \visible<2->{ear}&\myAnch{q1}{white}{\textbullet} \\
     2& \visible<3->{year}&\myAnch{q2}{white}{\textbullet} \\
     3& \visible<4->{ear}&\myAnch{q3}{white}{\textbullet} \\
     4& \visible<5->{year}&\myAnch{q4}{white}{\textbullet} \\
     5& \visible<6->{year}&\myAnch{q5}{white}{\textbullet} \\
     6& \visible<7->{ear}&\myAnch{q6}{white}{\textbullet} 
    \end{tabular}
   \end{column}
%%%%%%%%%%%
   \begin{column}{.45\textwidth}
    \begin{tabular}{lll}
     \myAnch{a1}{white}{\textbullet}& ear &\textipa{/\'I\textrhookschwa /}\\
     &\\
     \myAnch{a2}{white}{\textbullet}& year &\textipa{/j\'I\textrhookschwa /}\\
    \end{tabular}
   \end{column}
 \end{columns}

\begin{tikzpicture}[remember picture, overlay]
\tikzset{hoge/.style = {line width=4pt, ->, opacity=.6}}
 \visible<2->{\draw[hoge, Maroon] (q1.east) to[out=0, in=180] (a1.west);}
 \visible<3->{\draw[hoge, NavyBlue] (q2.east) to[out=0, in=180] (a2.west);}
 \visible<4->{\draw[hoge, Maroon] (q3.east) to[out=0, in=180] (a1.west);}
 \visible<5->{\draw[hoge, NavyBlue] (q4.east) to[out=0, in=180] (a2.west);}
 \visible<6->{\draw[hoge, NavyBlue] (q5.east) to[out=0, in=180] (a2.west);}
 \visible<7->{\draw[hoge, Maroon] (q6.east) to[out=0, in=180] (a1.west);}
\end{tikzpicture}

\hfill{\scriptsize \myaudio{./audio/consonant_j_03.mp3}}

\end{frame}
%%%%%%%%%%%%%%%%%%%%%%%%%%%
%%%%%%%%%%%%%%%%%%%%%%%%%%%
%%
%% /h/
%%
%%%%%%%%%%%%%%%%%%%%%%%%%%
% 背景色を黒に変更
\setbeamercolor{background canvas}{bg=black}
\begin{frame}
\centering
  \textcolor{white}{\Huge\bfseries Today's Pronunciation}\pause

 \vspace{30pt}

  \textcolor{white}{\Huge\bfseries \textipa{/h/}}
\vfill


\end{frame}
\setbeamercolor{background canvas}{bg=}
%%%%%%%%%%%%%%%%%%%%%%%%%%
\begin{frame}[plain]{ハヒフヘホ}

\begin{description}
 \item<1->[Question 1] 日本語のハ行、つまり\kenten{ハヒフヘホ}をローマ字で書くとどうなりますか
 \item<2->[Answer 1] ha\hspace{10pt} hi\hspace{10pt} hu(またはfu)\hspace{10pt} he\hspace{10pt} ho
\end{description}

\bigskip

\begin{description}
 \item<3->[Question 2] 日本語のハ行、つまり\kenten{ハヒフヘホ}の子音はどれもおなじでしょうか
 \item<4->[Answer 2] ちがいます\\
\visible<5->{\textipa{/ha/}\hspace{10pt} \textipa{/\c{c}i/}\hspace{10pt} \textipa{/\textphi u/}\hspace{10pt}\textipa{/he/}\hspace{10pt} \textipa{/ho/}}
 \end{description} 

\bigskip

\hfill\visible<6->{\kenten{ハヘホ}の子音は\textipa{/h/}ですが、\kenten{ヒ}と\kenten{フ}の子音は\textipa{/h/}ではありません}

\end{frame}
%%%%%%%%%%%%%%%%%%%%%%%%%%
\begin{frame}[plain]{\textipa{/h/}}

\large

\begin{enumerate}
 \item<2-> \underLine{h}ow
 \item<3-> \underLine{h}ard
 \item<4-> \underLine{h}appy
 \item<5-> \underLine{h}ead
 \item<6-> \underLine{h}elp
 \item<7-> \underLine{h}orse
 \item<8-> \underLine{h}e\hspace{17pt}\textipa{/h\'\i:/}\hspace{28pt}\visible<9->{\textipa{/i/}{\scriptsize を伸ばして発音して、口や舌の形はそのままで声を抜きます。小刻みに繰り返す}}%$\rightarrow$\textipa{/i:h:i:h:/}
 \item<10-> \underLine{wh}o\hspace{7pt}\textipa{/h\'u:/}\hspace{20pt} \visible<11->{\textipa{/u/}{\scriptsize を伸ばして発音して、口や舌の形はそのままで声を抜きます。小刻みに繰り返す}}%\textipa{/u:h:u:h:/}
\end{enumerate}

\vspace*{10pt}

\normalsize
ポイント

\begin{enumerate}
 \item<1-> ハヘホの子音部が\textipa{/h/}
 \item<8-> ヒ \textipa{/\c{c}i/} $\neq$ \textipa{/hi/}
 \item<10-> フ \textipa{/\textphi u/} $\neq$ \textipa{/hu/}%
\hfill{\scriptsize \myaudio{./audio/consonant_h_01.mp3}}

\end{enumerate}
\end{frame}
%%%%%%%%%%%%%%%%%%%%%%%%%%
%%%%%%%%%%%%%%%%%%%%%%%%%%%%%%%%%%%%%%%%%%%%%%%%%%%%
\begin{frame}[plain]{\textipa{/h/}実際の単語で確認しよう}
\Large
{\small \textipa{/h/}の音を含む英単語}\hfill{\scriptsize \myaudio{./audio/consonant_h_02.mp3}}


\begin{enumerate}
\item \textcolor{NavyBlue}{\bfseries h}at
\hfill{\scriptsize \textipa{/h/}$+$アに類する母音}\hspace{20pt}\makebox[80pt][l]{\textipa{/\textcolor{BurntOrange}{h}\'\ae t/}}\hspace{150pt}\mbox{}
 \item \textcolor{NavyBlue}{\bfseries h}ot
\hfill\makebox[80pt][l]{\textipa{/\textcolor{BurntOrange}{h}\'At/}}\hspace{150pt}\mbox{}
 \item \textcolor{NavyBlue}{\bfseries h}undred
\hfill\makebox[80pt][l]{\textipa{/\textcolor{BurntOrange}{h}\'\textturnv ndr@d/}}\hspace{150pt}\mbox{}
\item \textcolor{NavyBlue}{\bfseries h}ead
\hfill{\scriptsize \textipa{/h/}$+$エに類する母音}\hspace{20pt}\makebox[80pt][l]{\textipa{/\textcolor{BurntOrange}{h}\'ed/}}\hspace{150pt}\mbox{} 
 \item \textcolor{NavyBlue}{\bfseries h}ealth
\hfill\makebox[80pt][l]{\textipa{/\textcolor{BurntOrange}{h}\'elT/}}\hspace{150pt}\mbox{} 
 \item \textcolor{NavyBlue}{\bfseries h}air
\hfill\makebox[80pt][l]{\textipa{/\textcolor{BurntOrange}{h}\'e\textrhookschwa /}}\hspace{150pt}\mbox{}
\item \textcolor{NavyBlue}{\bfseries h}orse
\hfill{\scriptsize \textipa{/h/}$+$オに類する母音}\hspace{20pt}\makebox[80pt][l]{\textipa{/\textcolor{BurntOrange}{h}\'O\textrhookschwa s/}}\hspace{150pt}\mbox{}
 \item \textcolor{NavyBlue}{\bfseries h}ome%
\hfill\makebox[80pt][l]{\textipa{/\textcolor{BurntOrange}{h}\'oUm/}}\hspace{150pt}\mbox{}
 \item \textcolor{NavyBlue}{\bfseries h}ope%
\hfill\makebox[80pt][l]{\textipa{/\textcolor{BurntOrange}{h}\'oUp/}}\hspace{150pt}\mbox{}

% \item bir\textcolor{NavyBlue}{\bfseries d}
%\hfill\makebox[80pt][l]{\textipa{/b\'\textrhookschwa :\textcolor{BurntOrange}{d}/}}\hspace{150pt}\mbox{}
\end{enumerate}
\end{frame}
%%%%%%%%%%%%%%%%%%%%%%%%%%%%%%%%%%%%%%%%%%%%%%%%%%%%
\begin{frame}[plain]{\textipa{/h/}実際の単語で確認しよう}
{\small \textipa{/h/}の音を含む英単語}\hfill{\scriptsize \myaudio{./audio/consonant_h_03.mp3}}

{\small \textipa{/h/}にイ、ウに類する母音が続くとき、ヒ、フにならないように注意しましょう}

\Large

{\small 練習}\hspace{20pt}
\textipa{/i:h:i:h:/}\hspace{20pt}
\textipa{/u:h:u:h:/}
\begin{enumerate}
\item \textcolor{NavyBlue}{\bfseries h}e
\hfill{\scriptsize \textipa{/h/}$+$イに類する母音}\hspace{20pt}\makebox[80pt][l]{\textipa{/\textcolor{BurntOrange}{h}\'\i:/}}\hspace{140pt}\mbox{}
 \item \textcolor{NavyBlue}{\bfseries h}it
\hfill\makebox[80pt][l]{\textipa{/\textcolor{BurntOrange}{h}\'It/}}\hspace{140pt}\mbox{}
 \item \textcolor{NavyBlue}{\bfseries h}eat
\hfill\makebox[80pt][l]{\textipa{/\textcolor{BurntOrange}{h}\'\i:t/}}\hspace{140pt}\mbox{}
\item \textcolor{NavyBlue}{\bfseries w}ho
\hfill{\scriptsize \textipa{/h/}$+$ウに類する母音}\hspace{20pt}\makebox[80pt][l]{\textipa{/\textcolor{BurntOrange}{h}\'u:/}}\hspace{140pt}\mbox{} 
 \item \textcolor{NavyBlue}{\bfseries w}hose
\hfill\makebox[80pt][l]{\textipa{/\textcolor{BurntOrange}{h}\'u:z/}}\hspace{140pt}\mbox{} 
 \item \textcolor{NavyBlue}{\bfseries h}ook
\hfill\makebox[80pt][l]{\textipa{/\textcolor{BurntOrange}{h}\'Uk/}}\hspace{140pt}\mbox{}
\end{enumerate}
\end{frame}
%%%%%%%%%%%%%%%%%%%%%%%%%%
%%%%%%%%%%%%%%%%%%%%%%%%%%%
%%
%% /w/
%%
%%%%%%%%%%%%%%%%%%%%%%%%%%
% 背景色を黒に変更
\setbeamercolor{background canvas}{bg=black}
\begin{frame}
\centering
  \textcolor{white}{\Huge\bfseries Today's Pronunciation}\pause

 \vspace{30pt}

  \textcolor{white}{\Huge\bfseries \textipa{/w/}}

\end{frame}
\setbeamercolor{background canvas}{bg=}
%%%%%%%%%%%%%%%%%%%%%%%%%%
%%%%%%%%%%%%%%%%%%%%%%%%%%
\begin{frame}[plain]{ワ行}

\begin{description}
 \item[歴史的仮名遣い] \begin{tabular}[t]{ccccc}
		わ&ゐ&う&ゑ&を\\
		ワ&ヰ&ウ&ヱ&ヲ
		   \end{tabular}\pause
 \item[現代仮名遣い ] \begin{tabular}[t]{ccccc}
		わ& & & &を\\
		ワ&&&&ヲ
		   \end{tabular}
\end{description}


\begin{enumerate}
 \item 現代の日本語でwの音はワだけ
 \item 厳密にいうと日本語のワの子音$=$\textipa{[\textsublhalfring{w}]}
$\longrightarrow$つまり英語の\textipa{/w/}じゃない!

\end{enumerate}
\end{frame}
%%%%%%%%%%%%%%%%%%%%%%%%%%
%%%%%%%%%%%%%%%%%%%%%%%%%%
\begin{frame}[plain]{\textipa{/w/}}

\large

\begin{enumerate}
 \item \underLine{w}eek
 \item \underLine{w}inter
 \item \underLine{w}atch
 \item \underLine{w}ait
 \item \underLine{o}ne
 \item s\underLine{w}im
 \item s\underLine{w}eet
 \item q\underLine{u}estion
 \item \underLine{w}oman
\end{enumerate}

\vspace*{10pt}

\small
ポイント

\begin{enumerate}
 \item 唇を激しく丸めて、少し突き出します
 \item 口笛を吹くときの要領
\hfill{\scriptsize \myaudio{./audio/consonant_w_01.mp3}}

\end{enumerate}
\end{frame}
%%%%%%%%%%%%%%%%%%%%%%%%%%
%%%%%%%%%%%%%%%%%%%%%%%%%%%
%%
%% /l/
%% dark L
%%%%%%%%%%%%%%%%%%%%%%%%%%
% 背景色を黒に変更
\setbeamercolor{background canvas}{bg=black}
\begin{frame}
\centering
  \textcolor{white}{\Huge\bfseries Today's Pronunciation}\pause

 \vspace{30pt}

  \textcolor{white}{{\small 単語の最後の}{\Huge\bfseries \textipa{/l/}}}

%\hyperlink{tip}{\beamerreturnbutton{Back to \textipa{/p/} \textipa{/b/}}}
\end{frame}
\setbeamercolor{background canvas}{bg=}
%%%%%%%%%%%%%%%%%%%%%%%%%%%%%%%%%%%%%%%%%%%%%%%%%%%%
\begin{frame}[plain]{2つのL}

 \begin{description}[単語の最後の/l/]
 \item[明るいL] \underLine{l}ook\hspace{1\zw}\underLine{l}ight\hspace{1\zw}\underLine{l}ong\hspace{1\zw}\underLine{l}ake\hspace{1\zw}\underLine{l}ate
 \item[暗いL] beautifu\underLine{l}\hspace{1\zw}gir\underLine{l}\hspace{1\zw}peop\underLine{l}e\hspace{1\zw}tab\underLine{l}e\hspace{1\zw}app\underLine{l}e
\end{description}

\hfill{\scriptsize \myaudio{./audio/consonant_dark_l_00.mp3}}

\end{frame}
%%%%%%%%%%%%%%%%%%%%%%%%%%
\begin{frame}[plain]{Dark L}

\large

\begin{enumerate}
 \item peop\underLine{l}e
 \item tab\underLine{l}e
 \item app\underLine{l}e
 \item unc\underLine{l}e
 \item beautifu\underLine{l}
 \item hospita\underLine{l}
 \item litt\underLine{l}e
 \item gir\underLine{l}
 \item sma\underLine{ll}
\end{enumerate}

\vspace*{10pt}

\small
ポイント

\begin{enumerate}
 \item 唇を丸めて終わる
 \item オとかウみたいな感じ
\hfill{\scriptsize \myaudio{./audio/consonant_dark_l_01.mp3}}

\end{enumerate}
\end{frame}
%%%%%%%%%%%%%%%%%%%%%%%%%%
\begin{frame}[plain]{Dark L}

\large

\begin{enumerate}
 \item a\underLine{ll}
 \item ba\underLine{ll}
 \item ca\underLine{ll}
 \item we\underLine{ll}
 \item be\underLine{ll}
 \item coo\underLine{l}
 \item co\underLine{l}d
 \item simp\underLine{l}e
 \item anima\underLine{l}
\end{enumerate}

\vspace*{10pt}

\small
ポイント

\begin{enumerate}
 \item 唇を丸めて終わる
 \item オとかウみたいな感じ
\hfill{\scriptsize \myaudio{./audio/consonant_dark_l_02.mp3}}

\end{enumerate}
\end{frame}
%%%%%%%%%%%%%%%%%%%%%%%%%%
\begin{frame}[plain]{実験}
 \large

ティッシュペーパーを用意してください\pause

\bigskip

破裂音\textipa{/t/と\textipa{/p/}}の実験です\pause

\bigskip

\begin{enumerate}
 \item 「タイム」と``time''\hspace{30pt}\textipa{/t\'aIm/}\pause
 \item 「パイ」と``pie''\hspace{49pt}\textipa{/p\'aI/}
\end{enumerate}

\hfill{\scriptsize \myaudio{./audio/consonant_time_pie_01.mp3}}

\end{frame}
%%%%%%%%%%%%%%%%%%%%%%%%
%%%%%%%%%%%%%%%%%%%%%%%%%%
\begin{frame}[plain]{Quiz}\large

singまたはthingを発音していきます。どちらを発音したか聞き取って線でつなぎましょう

\bigskip

 \begin{columns}[t]
   \begin{column}{.45\textwidth}
    \begin{tabular}{rlr}
     1& \visible<2->{sing}&\myAnch{q1}{white}{\textbullet} \\
     2& \visible<3->{thing}&\myAnch{q2}{white}{\textbullet} \\
     3& \visible<4->{sing}&\myAnch{q3}{white}{\textbullet} \\
     4& \visible<5->{thing}&\myAnch{q4}{white}{\textbullet} \\
     5& \visible<6->{thing}&\myAnch{q5}{white}{\textbullet} \\
     6& \visible<7->{sing}&\myAnch{q6}{white}{\textbullet} 
    \end{tabular}
   \end{column}
%%%%%%%%%%%
   \begin{column}{.45\textwidth}
    \begin{tabular}{lll}
     \myAnch{a1}{white}{\textbullet}& sing &\textipa{/s\'IN/}\\
     &\\
     \myAnch{a2}{white}{\textbullet}& thing &\textipa{/T\'IN/}\\
    \end{tabular}
   \end{column}
 \end{columns}

\begin{tikzpicture}[remember picture, overlay]
\tikzset{hoge/.style = {line width=4pt, ->, opacity=.6}}
 \visible<2->{\draw[hoge, Maroon] (q1.east) to[out=0, in=180] (a1.west);}
 \visible<3->{\draw[hoge, NavyBlue] (q2.east) to[out=0, in=180] (a2.west);}
 \visible<4->{\draw[hoge, Maroon] (q3.east) to[out=0, in=180] (a1.west);}
 \visible<5->{\draw[hoge, NavyBlue] (q4.east) to[out=0, in=180] (a2.west);}
 \visible<6->{\draw[hoge, NavyBlue] (q5.east) to[out=0, in=180] (a2.west);}
 \visible<7->{\draw[hoge, Maroon] (q6.east) to[out=0, in=180] (a1.west);}
\end{tikzpicture}

\hfill{\scriptsize \myaudio{./audio/consonant_sing_thing_01.mp3}}

\end{frame}
%%%%%%%%%%%%%%%%%%%%%%%%%%%
%%%%%%%%%%%%%%%%%%%%%%%%
\begin{frame}[plain]{Quiz}\large

freeまたはthreeを発音していきます。どちらを発音したか聞き取って線でつなぎましょう

\bigskip

 \begin{columns}[t]
   \begin{column}{.45\textwidth}
    \begin{tabular}{rlr}
     1& \visible<2->{three}&\myAnch{q1}{white}{\textbullet} \\
     2& \visible<3->{free}&\myAnch{q2}{white}{\textbullet} \\
     3& \visible<4->{free}&\myAnch{q3}{white}{\textbullet} \\
     4& \visible<5->{three}&\myAnch{q4}{white}{\textbullet} \\
     5& \visible<6->{free}&\myAnch{q5}{white}{\textbullet} \\
     6& \visible<7->{three}&\myAnch{q6}{white}{\textbullet} 
    \end{tabular}
   \end{column}
%%%%%%%%%%%
   \begin{column}{.45\textwidth}
    \begin{tabular}{lll}
     \myAnch{a1}{white}{\textbullet}& free& \textipa{/fr\'\i:/}\\
     &\\
     \myAnch{a2}{white}{\textbullet}& three& \textipa{/Tr\'\i:/}  \\
    \end{tabular}
   \end{column}
 \end{columns}

\begin{tikzpicture}[remember picture, overlay]
\tikzset{hoge/.style = {line width=4pt, ->, opacity=.6}}
 \visible<2->{\draw[hoge, Maroon] (q1.east) to[out=0, in=180] (a2.west);}
 \visible<3->{\draw[hoge, NavyBlue] (q2.east) to[out=0, in=180] (a1.west);}
 \visible<4->{\draw[hoge, NavyBlue] (q3.east) to[out=0, in=180] (a1.west);}
 \visible<5->{\draw[hoge, Maroon] (q4.east) to[out=0, in=180] (a2.west);}
 \visible<6->{\draw[hoge, NavyBlue] (q5.east) to[out=0, in=180] (a1.west);}
 \visible<7->{\draw[hoge, Maroon] (q6.east) to[out=0, in=180] (a2.west);}
\end{tikzpicture}

\hfill\myaudio{./audio/consonant_free_three_01.mp3}

\end{frame}
%%%%%%%%%%%%%%%%%%%%%%%%%%
\begin{frame}[plain]{Quiz}\large
\hypertarget{quiz}{}

sheまたはsea (see / Cと同じ)を発音していきます。どちらを発音したか聞き取って線でつなぎましょう

\bigskip

 \begin{columns}[t]
   \begin{column}{.45\textwidth}
    \begin{tabular}{rlr}
     1& \visible<2->{sea}&\myAnch{q1}{white}{\textbullet} \\
     2& \visible<3->{she}&\myAnch{q2}{white}{\textbullet} \\
     3& \visible<4->{sea}&\myAnch{q3}{white}{\textbullet} \\
     4& \visible<5->{she}&\myAnch{q4}{white}{\textbullet} \\
     5& \visible<6->{she}&\myAnch{q5}{white}{\textbullet} \\
     6& \visible<7->{sea}&\myAnch{q6}{white}{\textbullet} 
    \end{tabular}
   \end{column}
%%%%%%%%%%%
   \begin{column}{.45\textwidth}
    \begin{tabular}{ll@{  }l}
     \myAnch{a1}{white}{\textbullet}& sea / see / C &\textipa{/s\'\i:/}\\
     &\\
     \myAnch{a2}{white}{\textbullet}& she &\textipa{/S\'\i:/}\\
    \end{tabular}
   \end{column}
 \end{columns}

\begin{tikzpicture}[remember picture, overlay]
\tikzset{hoge/.style = {line width=4pt, ->, opacity=.6}}
 \visible<2->{\draw[hoge, Maroon] (q1.east) to[out=0, in=180] (a1.west);}
 \visible<3->{\draw[hoge, NavyBlue] (q2.east) to[out=0, in=180] (a2.west);}
 \visible<4->{\draw[hoge, Maroon] (q3.east) to[out=0, in=180] (a1.west);}
 \visible<5->{\draw[hoge, NavyBlue] (q4.east) to[out=0, in=180] (a2.west);}
 \visible<6->{\draw[hoge, NavyBlue] (q5.east) to[out=0, in=180] (a2.west);}
 \visible<7->{\draw[hoge, Maroon] (q6.east) to[out=0, in=180] (a1.west);}
\end{tikzpicture}

\hfill{\scriptsize \myaudio{./audio/consonant_sea_she_01.mp3}}


\end{frame}
%%%%%%%%%%%%%%%%%%%%%%%%%%%
%%%%%%%%%%%%%%%%%%%%%%%%%%%
\begin{frame}[plain]{英語の子音}
\pause
いったいいくつの子音を学習したでしょうか?\pause

\bigskip

破裂音:\pause
 \textipa{/p/}\hspace{7pt}
 \textipa{/b/}\hspace{25pt}\pause
 \textipa{/t/}\hspace{7pt}
 \textipa{/d/}\hspace{25pt}\pause
 \textipa{/k/}\hspace{7pt}
 \textipa{/g/}\pause
 
摩擦音:\pause
\textipa{/f/}\hspace{7pt}
 \textipa{/v/}\hspace{25pt}\pause
 \textipa{/s/}\hspace{7pt}
 \textipa{/z/}\hspace{25pt}\pause
 \textipa{/T/}\hspace{7pt}
 \textipa{/D/}\hspace{25pt}\pause
 \textipa{/S/}\hspace{7pt}
 \textipa{/Z/}\hspace{30pt}\pause
\textipa{/h/}\pause
 
破擦音:\pause
\textipa{/tS/}\hspace{7pt}
 \textipa{/dZ/}\hspace{25pt}\pause
 \textipa{/ts/}\hspace{7pt}
 \textipa{/dz/}\pause

 鼻音 :\pause
\textipa{/m/}\hspace{10pt}
 \textipa{/n/}\hspace{10pt}
 \textipa{/N/}\hspace{10pt}\pause
 
 その他:\pause
\textipa{/l/}\hspace{7pt}
 \textipa{/r/}\hspace{25pt}\pause
 \textipa{/j/}\hspace{25pt}
  \textipa{/w/}

\end{frame}
%%%%%%%%%%%%%%%%%
\againframe[plain]{slide_p_b}
%%%%%%%%%%%%%%%%%
\againframe[plain]{slide_t_d}
%%%%%%%%%%%%%%%%%
\againframe[plain]{slide_k_g}
%%%%%%%%%%%%%%%%%
\againframe[plain]{slide_f_v}
%%%%%%%%%%%%%%%%%%
\againframe[plain]{slide_s_z}
%%%%%%%%%%%%%%%%
\againframe[plain]{slide_Th}
%%%%%%%%%%%%%%%%%%%%%%%%%%%%%%%%%%%%%%%%%%%%%%%%%%%%%%%%%%%%
\againframe[plain]{slide_textesh_textyogh}
%%%%%%%%%%%%%%%%%
\againframe[plain]{slide_ts_dz}
%%%%%%%%%%%%%%%%%%%%
\againframe[plain]{slide_ttextesh_dtextyogh}
\againframe[plain]{slide_l_r}
%%%%%%%%%%%%%%%%
%\hfill\hyperlink{tips}{\beamergotobutton{Excercises}}
%\hypertarget{tips}{}
\againframe[plain]{slide_j}
\againframe[plain]{ear_year_1}
\againframe<1-7>[plain]{quiz_ear_year}
\hyperlink{commencement}{\beamergotobutton{!}}
%%%%%%%%%%%%%%%
\begin{frame}[plain]{ところで}
 \Large
\centering
\pause
\vfill
今年、最後の素数の日は\pause{}20241229です\pause

来年、最初の素数の日は\pause{}20250101です\pause

\vfill


\raggedleft
\scsnowman[scale=5,hat=Maroon,snow,arms,buttons,note,muffler=NavyBlue]


\end{frame}
%%%%%%%%%%%%%%%%%%%%%%%%%%%%%%%%%%%%%%%%%%
%%%%%%%%%%%%%%%%%%%
%%% youtube
%%%%%%%%%%%%%%%%%%%%%%%%%%%%%%%%%%%%%%%%%%%%%%%%%%%%%
% 背景色をグレイに変更
%\setbeamercolor{background canvas}{bg=gray}
\setbeamercolor{background canvas}{bg=black}

\begin{frame}
\hypertarget{commencement}{}
\raggedleft
 \textcolor{white}{\Huge\bfseries \textcolor{yellow}{T}hank you for a great year!}

 \textcolor{white}{\Huge\bfseries \textcolor{red}{K}eep learning.}

 \textcolor{white}{\Huge\bfseries \textcolor{Green}{K}eep smiling.}

 \textcolor{white}{\Huge\bfseries \textcolor{yellow}{G}ood luck!}

\vfill

%\textcolor{white}{\scsnowman[scale=6,hat=red,muffler=Green,snow,arms,buttons=yellow,note]}

\begin{tikzpicture}
\duck[laughing,bowtie,
strawhat=brown!50!white,
ribbon=gray,
think={\scriptsize See you.},
bubblecolour=white!95!yellow]
\end{tikzpicture}
\end{frame}
%%%%%%%%%%%%%%%%%%%%%%%%%%
\setbeamercolor{background canvas}{bg=}
\begin{frame}[plain]{ところで}
 \Large
\centering
\pause
\vfill
今年度、最後の素数の日は\pause{}20250317です\pause

来年度、最初の素数の日は\pause{}20250401です\pause

\vfill


\raggedleft
\scsnowman[scale=5,hat=Maroon,snow,arms,buttons,note,muffler=NavyBlue]


\end{frame}
%%%%%%%%%%%%%%%%%%%%%%%%%%%%%%%%%%%%%%%%%%
\begin{frame}[plain]{hoge}
%\movie[width=.8\textwidth,height=.6\textheight,poster]{oyoyo}{../video/1st_20240919_otsuka.mp4} 
\href{run:../video/001_hogehoge.mp4}{piyopiyo}
\end{frame}
%%%%%%%%%%%%%%%%%%%%%%%%%%%%%%%%%%%%%%%%%%%%%%%%%%%%%
% 背景色をグレイに変更
%\setbeamercolor{background canvas}{bg=gray}
\setbeamercolor{background canvas}{bg=black}

\begin{frame}
\hypertarget{commencement}{}
\raggedleft
 \textcolor{white}{\Huge\bfseries \textcolor{yellow}{T}hank you for joining us this year.}

 \textcolor{white}{\Huge\bfseries \textcolor{red}{E}njoy your winter holiday!}

 \textcolor{white}{\Huge\bfseries \textcolor{Green}{S}ee you next year.}

\vfill

\textcolor{white}{\scsnowman[scale=6,hat=red,muffler=Green,snow,arms,buttons=yellow,note]}
\end{frame}
\setbeamercolor{background canvas}{bg=}
%%%%%%%%%%%%%%%%%%%%%%%%%%
\begin{frame}[plain]{soup}
\centering
	\begin{minipage}{2.5in}
 	\begin{alphabetsoup}*[6][6][\sffamily]
	  \hideinsoup{1}{2}{downright}{o,n,i,o,n}
	  \hideinsoup{1}{1}{down}{t,o,m,a,t,o}
	  \hideinsoup{6}{6}{up}{p,o,t,a,t,o}
	  \hideinsoup{5}{2}{down}{l,e,m,o,n}

	\end{alphabetsoup}
\end{minipage}

\end{frame}
%%%%%%%%%%%%%%%%%%%%%%%%%%%%%%%
\begin{frame}[plain]{tikz practice}
 \begin{tikzpicture}
\duck[laughing];
\draw[gray,thin,dotted] (-3,-3) grid (3,3);
\draw (0,0) pic[duck/laughing] {duck};
  \end{tikzpicture}
\end{frame}
\end{document}
