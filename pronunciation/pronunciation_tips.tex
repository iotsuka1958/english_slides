\documentclass[aspectratio=169,xcolor={dvipsnames,table}]{beamer}
\usepackage[no-math,deluxe,haranoaji]{luatexja-preset}
\renewcommand{\kanjifamilydefault}{\gtdefault}
\renewcommand{\emph}[1]{{\upshape\bfseries #1}}
\usetheme{metropolis}
\metroset{block=fill}
%%%%%%%%%%%%%%%%%%%%%%%%%%
\setbeamertemplate{navigation symbols}{}
\usecolortheme[rgb={0.7,0.2,0.2}]{structure}
%%%%%%%%%%%%%%%%%%%%%%%%%%
%% Change alert block colors
%%% 1- Block title (background and text)
\setbeamercolor{block title alerted}{fg=mDarkTeal, bg=mLightBrown!45!yellow!45}
\setbeamercolor{block title example}{fg=magenta!10!black, bg=mLightGreen!70}
%%% 2- Block body (background)
\setbeamercolor{block body alerted}{bg=mLightBrown!25}
\setbeamercolor{block body example}{bg=mLightGreen!15}
%%%%%%%%%%%%%%%%%%%%%%%%%%%
\usepackage[absolute,overlay]{textpos}
%\usepackage[grid=true,gridcolor=Maroon,subgridcolor=gray,gridunit=pt,texcoord]{eso-pic} %場所決めのためのgrid表示
%%%%%%%%%%%%%%%%%%%%%%%%%%%
%% さまざまなアイコン
%%%%%%%%%%%%%%%%%%%%%%%%%%%
%\usepackage{fontawesome}
\usepackage{fontawesome5}
\usepackage{figchild}
\usepackage{twemojis}
\usepackage{utfsym}
\usepackage{bclogo}
\usepackage{marvosym}
\usepackage{fontmfizz}
\usepackage{pifont}
\usepackage{phaistos}
\usepackage{worldflags}
\usepackage{jigsaw}
\usepackage{tikzlings}
\usepackage{tikzducks}
\usepackage{scsnowman}
\usepackage{epsdice}
\usepackage{halloweenmath}
\usepackage{svrsymbols}
\usepackage{countriesofeurope}
\usepackage{tipa}
%%%%%%%%%%%%%%%%%%%%%%%%%%%
\usepackage{tikz}
\usetikzlibrary{calc,patterns,decorations.pathmorphing,backgrounds}
\usepackage{tcolorbox}
\usepackage{tikzpeople}
\usepackage{circledsteps}
\usepackage{xcolor}
\usepackage{amsmath}
\usepackage{booktabs}
\usepackage{chronology}
\usepackage{signchart}
%%%%%%%%%%%%%%%%%%%%%%%%%%%
%% 場合分け
%%%%%%%%%%%%%%%%%%%%%%%%%%%
\usepackage{cases}
%%%%%%%%%%%%%%%%%%%%%%%%%%
\usepackage{pdfpages}
%%%%%%%%%%%%%%%%%%%%%%%%%%%
%% 音声リンク表示
\newcommand{\myaudio}[1]{\href{#1}{\faVolumeUp}}
%%%%%%%%%%%%%%%%%%%%%%%%%%
%% \myAnch{<名前>}{<色>}{<テキスト>}
%% 指定のテキストを指定の色の四角枠で囲み, 指定の名前をもつTikZの
%% ノードとして出力する. 図には remember picture 属性を付けている
%% ので外部から参照可能である.
\newcommand*{\myAnch}[3]{%
  \tikz[remember picture,baseline=(#1.base)]
    \node[draw,rectangle,line width=1pt,#2] (#1) {\normalcolor #3};
}
%%%%%%%%%%%%%%%%%%%%%%%%%%
%% \myEmph コマンドの定義
%%%%%%%%%%%%%%%%%%%%%%%%%%
%\newcommand{\myEmph}[3]{%
%    \textbf<#1>{\color<#1>{#2}{#3}}%
%}
\usepackage{xparse} % xparseパッケージの読み込み
\NewDocumentCommand{\myEmph}{O{} m m}{%
    \def\argOne{#1}%
    \ifx\argOne\empty
        \textbf{\color{#2}{#3}}% オプション引数が省略された場合
    \else
        \textbf<#1>{\color<#1>{#2}{#3}}% オプション引数が指定された場合
    \fi
}
%%%%%%%%%%%%%%%%%%%%%%%%%%%
%%%%%%%%%%%%%%%%%%%%%%%%%%%
%% 文末の上昇イントネーション記号 \myRisingPitch
%% 通常のイントネーション \myDownwardPitch
%% https://note.com/dan_oyama/n/n8be58e8797b2
%%%%%%%%%%%%%%%%%%%%%%%%%%%
\newcommand{\myRisingPitch}{
\begin{tikzpicture}[scale=0.3,baseline=0.3]
\draw[->,>=stealth] (0,0) to[bend right=45] (1,1);
\end{tikzpicture}
}
\newcommand{\myDownwardPitch}{
\begin{tikzpicture}[scale=0.3,baseline=0.3]
\draw[->,>=stealth] (0,1) to[bend left=45] (1,0);
\end{tikzpicture}
}
%%%%%%%%%%%%%%%%%%%%%%%%%%%%
%\AtBeginSection[%
%]{%
%  \begin{frame}[plain]\frametitle{授業の流れ}
%     \tableofcontents[currentsection]
%   \end{frame}%
%}

\usepackage{lua-ul}
\usepackage{pxrubrica}
\usepackage{soup}
\usepackage{circledsteps}
\usepackage{tikzducks}
\usetikzlibrary{decorations.pathmorphing}
\usetikzlibrary{ducks}
\usepackage{scsnowman}
\usetikzlibrary{tikzmark}
\usetikzlibrary{backgrounds}
%%%%%%%%%%%%%%%%%%%%%%%%%%
\usepackage{multimedia}
%%%%%%%%%%%%%%%%%%%%%%%%%%%
\makeatletter
\newcommand*{\themonth}{\two@digits\month}
\newcommand*{\theday}{\two@digits\day}
\makeatother
\newcommand{\mytoday}{{\the\year}--{\themonth}--{\theday}}
%%%%%%%%%%%%%%%%%%%%%%%%%%
\title{English is fun.}
\subtitle{Pronunciation---consonant---}
\author{}
\institute[]{}
\date[]

%%%%%%%%%%%%%%%%%%%%%%%%%%%%
%% TEXT
%%%%%%%%%%%%%%%%%%%%%%%%%%%%
\begin{document}
%%%%%%%%%%%%%%%%%%%%%%
%%%%%%%%%%%%%%%%%%%%%%%%%%%%%%%%%%%%%%%%%%%%%%%%%%%%%
% 背景色をグレイに変更
%\setbeamercolor{background canvas}{bg=gray}
\setbeamercolor{background canvas}{bg=black}
\begin{frame}
%\centering
\raggedleft
  \textcolor{white}{\Huge\bfseries English is fun.}

\vfill

\raggedleft
% \textcolor{white}{\LARGE\bfseries 2024--11--26}
% \textcolor{white}{\LARGE\bfseries \today}
 \textcolor{white}{\LARGE\bfseries \mytoday}

\vfill
\vfill
\vfill

\raggedleft
\textcolor{white}{\large The lesson will begin at the scheduled time.}

%\textcolor{white}{\large 可能なら、鏡を用意してください}
\end{frame}
\setbeamercolor{background canvas}{bg=}
%%%%%%%%%%%%%%%%%%%%%%%%%%
%%%%%%%%%%%%%%%%%%%%%%%%%%%%%%%%%%%%%%%%%%%%%%%%%%%%%
% 背景色をグレイに変更
%\setbeamercolor{background canvas}{bg=gray}
\setbeamercolor{background canvas}{bg=black}

\begin{frame}
%\centering
\raggedleft
  \textcolor{white}{\Huge\bfseries \textcolor{yellow}{E}nglish is fun.}

\vfill

\vfill

\raggedleft
 \textcolor{white}{\LARGE\bfseries \textcolor{yellow}{H}ello, everybody!}

 \textcolor{white}{\LARGE\bfseries \textcolor{yellow}{H}ow are you today?}

\raggedleft
 \textcolor{white}{\LARGE\bfseries \textcolor{yellow}{A}re you ready to start?}

 \textcolor{white}{\LARGE\bfseries \textcolor{yellow}{L}et's begin today's lesson.}

\vfill

\raggedleft
% \textcolor{white}{\LARGE\bfseries 2024--11--26}
% \textcolor{white}{\LARGE\bfseries \today}
 \textcolor{white}{\Large \bfseries \mytoday}

\hyperlink{today}{\beamergotobutton{Today's Pronunciation}}

%\pause
%\textcolor{white}{$20250919=13\times{}1,557,763$}
\end{frame}
\setbeamercolor{background canvas}{bg=}
%%%%%%%%%%%%%%%%%%%%%%%%%%
%%%%%%%%%%%%%%%%%%%%%%%%%%%
%%
%% /S/ /Z/
%%
%%%%%%%%%%%%%%%%%%%%%%%%%%
% 背景色を黒に変更
\setbeamercolor{background canvas}{bg=black}
\begin{frame}
\hypertarget{today}{}
\centering
  \textcolor{white}{\Huge\bfseries Today's Pronunciation}e

 \vspace{30pt}

  \textcolor{white}{\Large 音の連続}
\end{frame}
\setbeamercolor{background canvas}{bg=}
%%%%%%%%%%%%%%%%%%%%%%%%%%
\begin{frame}[plain]{音の連続}
 \Large
\begin{enumerate}
 \item an apple\hspace{20pt}✕アン アップル\hspace{20pt}ア{\bfseries\Huge ナ}ポウ
 \item an egg\hspace{33.75pt}✕アン エッグ\hspace{34pt}ア{\bfseries\Huge ネ}ーグ
\end{enumerate}
\end{frame}
\end{document}
