%%%%%%%%%%%%%%%%%%%%%%%%%%%%%%%%%%%%%%%%%%%%%%%%%%%%%%%%%%%%%%%%%
% LaTeX Beamer Presentation Source Code
% Generated by Gemini based on Google Slides data
%
% Recommended Compiler: LuaLaTeX
% (to correctly handle Japanese characters)
%%%%%%%%%%%%%%%%%%%%%%%%%%%%%%%%%%%%%%%%%%%%%%%%%%%%%%%%%%%%%%%%%

% ===== PACKAGE SETUP =====
\documentclass[aspectratio=169,xcolor={dvipsnames,table}]{beamer}
\usepackage[no-math,deluxe,haranoaji]{luatexja-preset}
\renewcommand{\kanjifamilydefault}{\gtdefault}
\renewcommand{\emph}[1]{{\upshape\bfseries #1}}
\usetheme{metropolis}
\metroset{block=fill}
\setbeamertemplate{navigation symbols}{}
\setbeamertemplate{blocks}[rounded][shadow=false]
\usecolortheme[rgb={0.7,0.2,0.2}]{structure}

%%%%%%%%%%%%%%%%%%%%%%%%%%
%%%%%%%%%%%%%%%%%%%%%%%%%%%
%% さまざまなアイコン
%%%%%%%%%%%%%%%%%%%%%%%%%%%
%\usepackage{fontawesome}
\usepackage{fontawesome5}
\usepackage{figchild}
\usepackage{twemojis}
\usepackage{utfsym}
\usepackage{bclogo}
\usepackage{marvosym}
\usepackage{fontmfizz}
\usepackage{pifont}
\usepackage{phaistos}
\usepackage{worldflags}
\usepackage{jigsaw}
\usepackage{tikzlings}
\usepackage{tikzducks}
\usepackage{scsnowman}
\usepackage{epsdice}
\usepackage{halloweenmath}
\usepackage{svrsymbols}
\usepackage{countriesofeurope}
\usepackage{tipa}
\usepackage{manfnt}
%%%%%%%%%%%%%%%%%%%%%%%%%%%
\usepackage{tikz}
\usetikzlibrary{calc,patterns,decorations.pathmorphing,backgrounds}
\usepackage{tcolorbox}
\usepackage{tikzpeople}
\usepackage{circledsteps}
\usepackage{xcolor}
\usepackage{amsmath}
\usepackage{booktabs}
\usepackage{chronology}
\usepackage{signchart}
%%%%%%%%%%%%%%%%%%%%%%%%%%%
%% 場合分け
%%%%%%%%%%%%%%%%%%%%%%%%%%%
\usepackage{cases}
%%%%%%%%%%%%%%%%%%%%%%%%%%
\usepackage{pdfpages}
%%%%%%%%%%%%%%%%%%%%%%%%%%%
%% 音声リンク表示
\newcommand{\myaudio}[1]{\href{#1}{\faVolumeUp}}
%%%%%%%%%%%%%%%%%%%%%%%%%%
%% \myAnch{<名前>}{<色>}{<テキスト>}
%% 指定のテキストを指定の色の四角枠で囲み, 指定の名前をもつTikZの
%% ノードとして出力する. 図には remember picture 属性を付けている
%% ので外部から参照可能である.
\newcommand*{\myAnch}[3]{%
  \tikz[remember picture,baseline=(#1.base)]
    \node[draw,rectangle,line width=1pt,#2] (#1) {\normalcolor #3};
}
%%%%%%%%%%%%%%%%%%%%%%%%%%
%% \myEmph コマンドの定義
%%%%%%%%%%%%%%%%%%%%%%%%%%
%\newcommand{\myEmph}[3]{%
%    \textbf<#1>{\color<#1>{#2}{#3}}%
%}
\usepackage{xparse} % xparseパッケージの読み込み
\NewDocumentCommand{\myEmph}{O{} m m}{%
    \def\argOne{#1}%
    \ifx\argOne\empty
        \textbf{\color{#2}{#3}}% オプション引数が省略された場合
    \else
        \textbf<#1>{\color<#1>{#2}{#3}}% オプション引数が指定された場合
    \fi
}
%%%%%%%%%%%%%%%%%%%%%%%%%%%
%%%%%%%%%%%%%%%%%%%%%%%%%%%
%% 文末の上昇イントネーション記号 \myRisingPitch
%% 通常のイントネーション \myDownwardPitch
%% https://note.com/dan_oyama/n/n8be58e8797b2
%%%%%%%%%%%%%%%%%%%%%%%%%%%
\newcommand{\myRisingPitch}{
\begin{tikzpicture}[scale=0.3,baseline=0.3]
\draw[->,>=stealth] (0,0) to[bend right=45] (1,1);
\end{tikzpicture}
}
\newcommand{\myDownwardPitch}{
\begin{tikzpicture}[scale=0.3,baseline=0.3]
\draw[->,>=stealth] (0,1) to[bend left=45] (1,0);
\end{tikzpicture}
}
%%%%%%%%%%%%%%%%%%%%%%%%%%%%
%\AtBeginSection[%
%]{%
%  \begin{frame}[plain]\frametitle{授業の流れ}
%     \tableofcontents[currentsection]
%   \end{frame}%
%}

\usepackage{highlightx}
\usepackage{lua-ul}
%%%%%%%%%%%%%%%%%%%%%%%%%%%

% ===== THEME AND COLOR DEFINITIONS =====

% Define colors to match the Google Slides template
\definecolor{primaryblue}{RGB}{66, 133, 244}
\definecolor{textprimary}{RGB}{51, 51, 51}
\definecolor{neutralgray}{RGB}{158, 158, 158}

% Set Beamer colors
\setbeamercolor{palette primary}{bg=primaryblue,fg=white}
\setbeamercolor{palette secondary}{bg=primaryblue!75!white,fg=white}
\setbeamercolor{palette tertiary}{bg=primaryblue!50!white,fg=white}
\setbeamercolor{structure}{fg=primaryblue}
\setbeamercolor{block title}{bg=primaryblue,fg=white}
\setbeamercolor{block title alerted}{bg=red!80!black,fg=white}
\setbeamercolor{block body}{bg=gray!10!white}
\setbeamercolor{normal text}{fg=textprimary}

% ===== PRESENTATION METADATA =====


% Helper command for styled text
\newcommand{\gstyle}[1]{\textbf{\textcolor{primaryblue}{#1}}}
%%%%%%%%%%%%%%%%%%%%%%%%%%%
\title{English is fun.}
\subtitle{Where does Jane live?}
\author{}
\institute[]{}
\date[]

%%%%%%%%%%%%%%%%%%%%%
\begin{document}

% ===== TITLE PAGE =====
\begin{frame}
    \titlepage
\end{frame}

% ===== TABLE OF CONTENTS =====
\begin{frame}{本日の学習内容(アジェンダ)}
    \tableofcontents
\end{frame}

% ===== SECTION 1: VOWELS =====
\section{1. 母音 (Vowels)}

\begin{frame}
    \frametitle{「ア」に近い音のバリエーション}
    \begin{columns}[T,totalwidth=\textwidth]
        \begin{column}{.3\textwidth}
            \begin{block}{\textipa{/\ae /}}
                \gstyle{アとエの中間音}。エの口で「ア」と発音するイメージ。\\ \vspace{0.3cm}
                \textbf{例}: cat, apple, map
            \end{block}
        \end{column}
        \begin{column}{.3\textwidth}
            \begin{block}{\textipa{/\textscripta :/}}
                口を\textbf{大きく縦に}開け、喉の奥から「アー」と長く発音。\\ \vspace{0.3cm}
                \textbf{例}: hot, father, car
            \end{block}
        \end{column}
        \begin{column}{.3\textwidth}
            \begin{block}{\textipa{/2/}}
                口を\textbf{軽く開け}、短く「アッ」と発音。曖昧母音のアクセント版。\\ \vspace{0.3cm}
                \textbf{例}: cut, sun, money
            \end{block}
        \end{column}
    \end{columns}
\end{frame}

\begin{frame}
    \frametitle{「イ」「ウ」に近い音 (長母音 vs 短母音)}
     \begin{columns}[T,totalwidth=\textwidth]
        \begin{column}{.45\textwidth}
            \begin{block}{\textipa{/i:/} (長母音)}
                口を\textbf{横に強く引いて}「イー」と長く発音。\\
                \textbf{例}: see, eat, key
            \end{block}
            \begin{block}{\textipa{/u:/ }(長母音)}
                唇を\textbf{強く丸めて}前に突き出し「ウー」。\\
                \textbf{例}: food, blue, shoe
            \end{block}
        \end{column}
        \begin{column}{.45\textwidth}
            \begin{block}{\textipa{/\textsci /} (短母音)}
                「イ」と「エ」の中間。力を抜いて短く「イッ」。\\
                \textbf{例}: sit, this, ship
            \end{block}
            \begin{block}{\textipa{/\textupsilon /} (短母音)}
                 「ウ」と「オ」の中間。力を抜いて短く「ウッ」。\\
                \textbf{例}: book, put, good
            \end{block}
        \end{column}
    \end{columns}
\end{frame}

\begin{frame}
    \frametitle{曖昧母音「シュワ」とRの音}
    \begin{columns}[T,totalwidth=\textwidth]
        \begin{column}{.45\textwidth}
            \begin{block}{\textipa{/@/} (シュワ)}
                最も重要な母音の一つ。\textbf{完全に脱力}し、弱く短く「ァ」と発音。\\ \vspace{0.3cm}
                \textbf{例}: about, sofa, teacher
            \end{block}
        \end{column}
        \begin{column}{.45\textwidth}
            \begin{block}{\textipa{/\textrhookschwa :/}}
                舌を巻きながら\textbf{長く伸ばす}「アー」。Rの母音。\\ \vspace{0.3cm}
                \textbf{例}: bird, girl, first
            \end{block}
        \end{column}
    \end{columns}
\end{frame}


% ===== SECTION 2: DIPHTHONGS =====
\section{2. 二重母音 (Diphthongs)}

\begin{frame}
    \frametitle{主要な二重母音}
    \begin{columns}[T,totalwidth=\textwidth]
        \begin{column}{.45\textwidth}
            \begin{block}{\textipa{/eI/} (エイ)}
                \textbf{エ→イ}への変化を滑らかに。\\ \textbf{例}: day, train, make
            \end{block}
            \begin{block}{\textipa{/OI/} (オイ)}
                \textbf{オ→イ}へ。\\ \textbf{例}: boy, toy, voice
            \end{block}
            \begin{block}{\textipa{/aU/} (アウ)}
                \textbf{ア→ウ}へ。口を大きく開けてからすぼめる。\\ \textbf{例}: now, out, house
            \end{block}
        \end{column}
        \begin{column}{.45\textwidth}
            \begin{block}{\textipa{/aI/} (アイ)}
                \textbf{ア→イ}へ。口を大きく開けた状態から始める。\\ \textbf{例}: my, time, high
            \end{block}
            \begin{block}{\textipa{/oU/} (オウ)}
                \textbf{オ→ウ}へ。口をすぼめる動きを意識。\\ \textbf{例}: go, boat, home
            \end{block}
        \end{column}
    \end{columns}
\end{frame}

% ===== SECTION 3: CONSONANTS =====
\section{3. 子音 (Consonants)}

\begin{frame}
    \frametitle{破裂音のペア (無声音 vs 有声音)}
    \framesubtitle{息を止めてから一気に破裂させる音。喉が震えるかどうかが違い。}
    \begin{columns}[T,totalwidth=\textwidth]
        \begin{column}{.45\textwidth}
            \begin{alertblock}{無声音 (息のみ)}
                \begin{itemize}
                    \item \textipa{/p/} : \textbf{p}en, co\textbf{p}y
                    \item \textipa{/t/} : \textbf{t}ea, ge\textbf{t}
                    \item \textipa{/k/} : \textbf{c}at, ba\textbf{ck}
                \end{itemize}
            \end{alertblock}
        \end{column}
        \begin{column}{.45\textwidth}
            \begin{block}{有声音 (声帯を震わす)}
                \begin{itemize}
                    \item \textipa{/b/} : \textbf{b}ook, jo\textbf{b}
                    \item \textipa{/d/} : \textbf{d}ay, goo\textbf{d}
                    \item \textipa{/g/} : \textbf{g}o, bi\textbf{g}
                \end{itemize}
            \end{block}
        \end{column}
    \end{columns}
\end{frame}

\begin{frame}
    \frametitle{摩擦音のペア (無声音 vs 有声音)}
    \framesubtitle{隙間から息を漏らして摩擦させる音。}
    \begin{columns}[T,totalwidth=\textwidth]
        \begin{column}{.45\textwidth}
            \begin{alertblock}{無声音 \textipa{/f/}, \textipa{/T/}, \textipa{/s/}}
                \begin{itemize}
                    \item[\textipa{/f/}] \textbf{f}an (上の歯で下唇を噛む)
                    \item[\textipa{/T/}] \textbf{th}ink (舌を歯で挟む)
                    \item[\textipa{/s/}] \textbf{s}ee (歯を閉じて「スー」)
                \end{itemize}
            \end{alertblock}
        \end{column}
        \begin{column}{.45\textwidth}
            \begin{block}{有声音 \textipa{/v/}, \textipa{/D/}, \textipa{/z/}}
                \begin{itemize}
                    \item[\textipa{/v/}] \textbf{v}an (上の歯で下唇を噛む)
                    \item[\textipa{/D/}] \textbf{th}is (舌を歯で挟む)
                    \item[\textipa{/z/}] \textbf{z}oo (歯を閉じて「ズー」)
                \end{itemize}
            \end{block}
        \end{column}
    \end{columns}
\end{frame}

\begin{frame}
    \frametitle{その他の摩擦音・破擦音ペア}
     \begin{columns}[T,totalwidth=\textwidth]
        \begin{column}{.45\textwidth}
            \begin{alertblock}{無声音 \textipa{/S/}, \textipa{/tS/}}
                \begin{itemize}
                    \item[\textipa{/S/}] \textbf{sh}e, fi\textbf{sh} (口を丸め「シー」)
                    \item[\textipa{/tS/}] \textbf{ch}air, wa\textbf{tch} (「チュ」に近い)
                \end{itemize}
            \end{alertblock}
        \end{column}
        \begin{column}{.45\textwidth}
            \begin{block}{有声音 \textipa{/Z/}, \textipa{/dZ/}}
                \begin{itemize}
                    \item[\textipa{/Z/}] televi\textbf{si}on (「ジ」に近い)
                    \item[\textipa{/dZ/}] \textbf{j}u\textbf{dge}, a\textbf{ge} (「ヂュ」に近い)
                \end{itemize}
            \end{block}
        \end{column}
    \end{columns}
\end{frame}

\begin{frame}
    \frametitle{最重要: LとRの比較}
    \framesubtitle{舌の使い方が全く異なります。}
     \begin{columns}[T,totalwidth=\textwidth]
        \begin{column}{.45\textwidth}
            \begin{block}{\textipa{/l/} (Light \& Dark L)}
                \begin{itemize}
                    \item 舌先を\textbf{上の歯茎にしっかりつける}
                    \item 例: \textbf{l}ight, fee\textbf{l}, \textbf{l}amp
                \end{itemize}
            \end{block}
        \end{column}
        \begin{column}{.45\textwidth}
            \begin{block}{\textipa{/r/}}
                \begin{itemize}
                    \item 舌を\textbf{口のどこにもつけずに}奥に引く
                    \item 例: \textbf{r}ight, ca\textbf{r}, \textbf{r}oad
                \end{itemize}
            \end{block}
        \end{column}
    \end{columns}
\end{frame}

\begin{frame}
    \frametitle{鼻音と半母音}
    \begin{columns}[T,totalwidth=\textwidth]
        \begin{column}{.3\textwidth}
            \begin{block}{\textipa{/m/}}
                唇を\textbf{閉じて}「ン」。\\ \textbf{例}: man
            \end{block}
            \begin{block}{\textipa{/h/}}
                喉の奥から\textbf{息だけ}を出す。\\ \textbf{例}: hot
            \end{block}
        \end{column}
        \begin{column}{.3\textwidth}
            \begin{block}{/n/}
                舌先を\textbf{歯茎につけて}「ン」。\\ \textbf{例}: no
            \end{block}
            \begin{block}{\textipa{/j/}}
                「ヤユヨ」の音に近い。\\ \textbf{例}: yes
            \end{block}
        \end{column}
        \begin{column}{.3\textwidth}
            \begin{block}{\textipa{/N/}}
                舌の\textbf{奥を上げて}「ング」。\\ \textbf{例}: sing
            \end{block}
            \begin{block}{\textipa{/w/}}
                口を\textbf{すぼめて}「ウ」。\\ \textbf{例}: we
            \end{block}
        \end{column}
    \end{columns}
\end{frame}


% ===== SECTION 4: SUMMARY =====
\section{4. まとめ}

\begin{frame}{発音記号学習のポイントと今後のステップ}
    \begin{itemize}\setbeamertemplate{items}[circle]
        \item 発音記号は\gstyle{丸暗記ではなく、音とセットで理解}する
        \item \textbf{有声音と無声音}のペアを意識し、喉の振動で区別する
        \item \gstyle{LとR、THなど日本語にない音}を重点的に反復練習する
        \item 新しい単語を覚える際は、\textbf{必ず発音記号を確認}する習慣をつける
    \end{itemize}
\end{frame}


\end{document}
