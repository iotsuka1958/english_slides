\documentclass[aspectratio=169,xcolor={dvipsnames,table}]{beamer}
\usepackage[no-math,deluxe,haranoaji]{luatexja-preset}
\renewcommand{\kanjifamilydefault}{\gtdefault}
\renewcommand{\emph}[1]{{\upshape\bfseries #1}}
\usetheme{gotham}
   \gothamset{
      numbering= framenumber,
      % tocframe template= gotham simple,
      parttocframe default=off,
      sectiontocframe default=off,
      subsectiontocframe default=off,
   }
%%%%%%%%%%%%%%%%%%%%%%%%%%%
%%%%%%%%%%%%%%%%%%%%%%%%%%%
%% さまざまなアイコン
%%%%%%%%%%%%%%%%%%%%%%%%%%%
%\usepackage{fontawesome}
\usepackage{fontawesome5}
\usepackage{figchild}
\usepackage{twemojis}
\usepackage{utfsym}
\usepackage{bclogo}
\usepackage{marvosym}
\usepackage{fontmfizz}
\usepackage{pifont}
\usepackage{phaistos}
\usepackage{worldflags}
\usepackage{jigsaw}
\usepackage{tikzlings}
\usepackage{tikzducks}
\usepackage{scsnowman}
\usepackage{epsdice}
\usepackage{halloweenmath}
\usepackage{svrsymbols}
\usepackage{countriesofeurope}
\usepackage{tipa}
%%%%%%%%%%%%%%%%%%%%%%%%%%%
\usepackage{tikz}
\usetikzlibrary{calc,patterns,decorations.pathmorphing,backgrounds}
\usepackage{tcolorbox}
\usepackage{tikzpeople}
\usepackage{circledsteps}
\usepackage{xcolor}
\usepackage{amsmath}
\usepackage{booktabs}
\usepackage{chronology}
\usepackage{signchart}
%%%%%%%%%%%%%%%%%%%%%%%%%%%
%% 場合分け
%%%%%%%%%%%%%%%%%%%%%%%%%%%
\usepackage{cases}
%%%%%%%%%%%%%%%%%%%%%%%%%%
\usepackage{pdfpages}
%%%%%%%%%%%%%%%%%%%%%%%%%%%
%% 音声リンク表示
\newcommand{\myaudio}[1]{\href{#1}{\faVolumeUp}}
%%%%%%%%%%%%%%%%%%%%%%%%%%
%% \myAnch{<名前>}{<色>}{<テキスト>}
%% 指定のテキストを指定の色の四角枠で囲み, 指定の名前をもつTikZの
%% ノードとして出力する. 図には remember picture 属性を付けている
%% ので外部から参照可能である.
\newcommand*{\myAnch}[3]{%
  \tikz[remember picture,baseline=(#1.base)]
    \node[draw,rectangle,line width=1pt,#2] (#1) {\normalcolor #3};
}
%%%%%%%%%%%%%%%%%%%%%%%%%%
%% \myEmph コマンドの定義
%%%%%%%%%%%%%%%%%%%%%%%%%%
%\newcommand{\myEmph}[3]{%
%    \textbf<#1>{\color<#1>{#2}{#3}}%
%}
\usepackage{xparse} % xparseパッケージの読み込み
\NewDocumentCommand{\myEmph}{O{} m m}{%
    \def\argOne{#1}%
    \ifx\argOne\empty
        \textbf{\color{#2}{#3}}% オプション引数が省略された場合
    \else
        \textbf<#1>{\color<#1>{#2}{#3}}% オプション引数が指定された場合
    \fi
}
%%%%%%%%%%%%%%%%%%%%%%%%%%%
%%%%%%%%%%%%%%%%%%%%%%%%%%%
%% 文末の上昇イントネーション記号 \myRisingPitch
%% 通常のイントネーション \myDownwardPitch
%% https://note.com/dan_oyama/n/n8be58e8797b2
%%%%%%%%%%%%%%%%%%%%%%%%%%%
\newcommand{\myRisingPitch}{
\begin{tikzpicture}[scale=0.3,baseline=0.3]
\draw[->,>=stealth] (0,0) to[bend right=45] (1,1);
\end{tikzpicture}
}
\newcommand{\myDownwardPitch}{
\begin{tikzpicture}[scale=0.3,baseline=0.3]
\draw[->,>=stealth] (0,1) to[bend left=45] (1,0);
\end{tikzpicture}
}
%%%%%%%%%%%%%%%%%%%%%%%%%%%%
%\AtBeginSection[%
%]{%
%  \begin{frame}[plain]\frametitle{授業の流れ}
%     \tableofcontents[currentsection]
%   \end{frame}%
%}

\usepackage{lua-ul}
%%%%%%%%%%%%%%%%%%%%%%%%%%%
\makeatletter
\newcommand*{\themonth}{\two@digits\month}
\newcommand*{\theday}{\two@digits\day}
\makeatother
\newcommand{\mytoday}{{\the\year}--{\themonth}--{\theday}}
%%%%%%%%%%%%%%%%%%%%%%%%%%
%%%%%%%%%%%%%%%%%%%%%%%%%%
\title{Today's Pronunciation}
\subtitle{Pronunciation of Japanese Character `ん'}
\date[]{\mytoday}
\author{iotsuka}
\institute{Eduop ちば}

%%%%%%%%%%%%%%%%%%%%%%%%%%%%
%% TEXT
%%%%%%%%%%%%%%%%%%%%%%%%%%%%
\begin{document}
%%%%%%%%%%%%%%%%%%%%%%%%%%%%%%
\gothamset{background=dark}
\maketitle

   \begin{frame}[toc]{Table of contents}%
      \tableofcontents%[hideallsubsections]
   \end{frame}

%%%%%%%%%%%%%%%%%%%%%%%%%%%%%%
\gothamset{background=light}
%%%%%%%%%%%%%%%%%%%%%%%%%%%%
%%%%%%%%%%%%%%%%%%%%%%%%%%%%%%
\section{`ん'をローマ字で書くと}
\begin{frame}[plain]{`ん'をローマ字で書くと}

\Huge n

\end{frame}
%%%%%%%%%%%%%%%%%%%%%%%%%%%%%%%
\gothamset{background=light}
\section{`ん'を含む単語を発音してみよう}
\subsection{発音できますか}
\begin{frame}[plain]{発音できますか}
 \begin{itemize}
  \item あ\Circled{ん}ドーナツ
  \item せ\Circled{ん}べい
  \item と\Circled{ん}かつ
  \item こ\Circled{ん}にゃく
  \item パ\Circled{ン}
 \end{itemize}
\end{frame}
%%%%%%%%%%%%%%%%%%%%%%%%%%%%%%%
\subsection{どのように発音していますか}
\begin{frame}[plain]{この`ん'は}\Huge
\begin{columns}
 \begin{column}{.4\textwidth}
 あ\Circled{ん}ドーナツ

\hspace{18pt}\visible<2->{\textipa{/n/}}
 \end{column}
\begin{column}{.55\textwidth}
\IfFileExists{../images/doughnuts.jpg}{%
\includegraphics[width=.99\textwidth]{../images/doughnuts.jpg}
\raggedleft
{\tiny ``fánk / doughnut / gogoşi'' by debreczeniemoke is licensed under CC BY 2.0.}}{\relax}
\end{column}
\end{columns}
\end{frame}
%%%%%%%%%%%%%%%%%%%%%%%%%%%%%%
\begin{frame}[plain]{この`ん'は}\Huge
\begin{columns}
 \begin{column}{.3\textwidth}
 せ\Circled{ん}べい

\hspace{16pt}\visible<2->{\textipa{/m/}}
 \end{column}
\begin{column}{.6\textwidth}
\IfFileExists{../images/senbei.jpg}{%
\includegraphics[width=.99\textwidth]{../images/senbei.jpg}
\raggedleft
{\tiny ``Japanese Senbeis’’ by DryPot is licensed under CC BY 2.5.}}{\relax}
\end{column}
\end{columns}
\end{frame}
%%%%%%%%%%%%%%%%%%%%%%%%%%%%%%
\begin{frame}[plain]{この`ん'は}\Huge

\begin{columns}
 \begin{column}{.3\textwidth}
 と\Circled{ん}かつ

\hspace{18pt}\visible<2->{\textipa{/\ng /}}
 \end{column}
\begin{column}{.6\textwidth}
\IfFileExists{../images/tonkatsu.jpg}{%
\includegraphics[width=.99\textwidth]{../images/tonkatsu.jpg}
\raggedleft
{\tiny ``Kurobuta Tonkatsu 黒豚とんかつ定食'' by jetalone is licensed under CC BY 2.0.}}{\relax}
\end{column}
\end{columns}
\end{frame}
%%%%%%%%%%%%%%%%%%%%%%%%%%%%%%
\begin{frame}[plain]{この`ん'は}\Huge\begin{columns}
 \begin{column}{.4\textwidth}
 こ\Circled{ん}にゃく

\hspace{18pt}\textipa{/\textltailn /}
 \end{column}
\begin{column}{.575\textwidth}
\IfFileExists{../images/konnyaku.jpg}{%
\includegraphics[width=\textwidth]{../images/konnyaku.jpg}
\raggedleft
{\tiny ``Konnyaku'' by preetamrai is licensed under CC BY 2.0.}}{\relax}
\end{column}
\end{columns}
\end{frame}
%%%%%%%%%%%%%%%%%%%%%%%%%%%%%%
\begin{frame}[plain]{この`ん'は}\Huge

\begin{columns}
 \begin{column}{.3\textwidth}
 パ\Circled{ン}

\hspace{18pt}\textipa{/\textscn/}
 \end{column}
\begin{column}{.6\textwidth}
\IfFileExists{../images/bread.jpg}{%
\includegraphics[width=.99\textwidth]{../images/bread.jpg}
\raggedleft
{\tiny ``Bread'' by karenandbrademerson is licensed under CC BY 2.0.}}{\relax}
\end{column}
\end{columns}
\end{frame}
%%%%%%%%%%%%%%%%%%%%%%%%%%%%%%
\begin{frame}[plain]{Exercises}
下線部の発音が`あ\Circled{ん}ドーナツ'の\Circled{ん}と同じなのはどれですか

\begin{enumerate}
 \item あ\underLine{ん}まん
 \item て\underLine{ん}どん(天丼)
 \item て\underLine{ん}ぷら(天ぷら)

\end{enumerate}
\end{frame}

%%%%%%%%%%%%%%%%%%%%%%%%%%%%%%
\begin{frame}[plain]{Exercises}
下線部の発音が`せ\Circled{ん}べい'の\Circled{ん}と同じなのはどれですか

\begin{enumerate}
 \item ちゃ\underLine{ん}ぽん
 \item みたらしだ\underLine{ん}ご
 \item あ\underLine{ん}にんどうふ(杏仁豆腐)

\end{enumerate}
\end{frame}
%%%%%%%%%%%%%%%%%%%%%%%%%%
%%%%%%%%%%%%%%%%%%%%%%%%%%%%%%
\begin{frame}[plain]{Exercises}
下線部の発音が`と\Circled{ん}かつ'の\Circled{ん}と同じなのはどれですか

\begin{enumerate}
 \item さ\underLine{ん}ま
 \item あ\underLine{ん}ころもち(あんころ餅)
 \item め\underLine{ん}ちかつ(メンチカツ)

\end{enumerate}
\end{frame}
%%%%%%%%%%%%%%%%%%%%%%%%%%%%%
\end{document}
