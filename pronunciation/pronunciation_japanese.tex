\documentclass[aspectratio=169,xcolor={dvipsnames,table}]{beamer}
\usepackage[no-math,deluxe,haranoaji]{luatexja-preset}
\renewcommand{\kanjifamilydefault}{\gtdefault}
\renewcommand{\emph}[1]{{\upshape\bfseries #1}}
\usetheme{gotham}
   \gothamset{
      numbering= framenumber,
      % tocframe template= gotham simple,
      parttocframe default=off,
      sectiontocframe default=off,
      subsectiontocframe default=off,
   }
%%%%%%%%%%%%%%%%%%%%%%%%%%%
%%%%%%%%%%%%%%%%%%%%%%%%%%%
%% さまざまなアイコン
%%%%%%%%%%%%%%%%%%%%%%%%%%%
%\usepackage{fontawesome}
\usepackage{fontawesome5}
\usepackage{figchild}
\usepackage{twemojis}
\usepackage{utfsym}
\usepackage{bclogo}
\usepackage{marvosym}
\usepackage{fontmfizz}
\usepackage{pifont}
\usepackage{phaistos}
\usepackage{worldflags}
\usepackage{jigsaw}
\usepackage{tikzlings}
\usepackage{tikzducks}
\usepackage{scsnowman}
\usepackage{epsdice}
\usepackage{halloweenmath}
\usepackage{svrsymbols}
\usepackage{countriesofeurope}
\usepackage{tipa}
%%%%%%%%%%%%%%%%%%%%%%%%%%%
\usepackage{tikz}
\usetikzlibrary{calc,patterns,decorations.pathmorphing,backgrounds}
\usepackage{tcolorbox}
\usepackage{tikzpeople}
\usepackage{circledsteps}
\usepackage{xcolor}
\usepackage{amsmath}
\usepackage{booktabs}
\usepackage{chronology}
\usepackage{signchart}
%%%%%%%%%%%%%%%%%%%%%%%%%%%
%% 場合分け
%%%%%%%%%%%%%%%%%%%%%%%%%%%
\usepackage{cases}
%%%%%%%%%%%%%%%%%%%%%%%%%%
\usepackage{pdfpages}
%%%%%%%%%%%%%%%%%%%%%%%%%%%
%% 音声リンク表示
\newcommand{\myaudio}[1]{\href{#1}{\faVolumeUp}}
%%%%%%%%%%%%%%%%%%%%%%%%%%
%% \myAnch{<名前>}{<色>}{<テキスト>}
%% 指定のテキストを指定の色の四角枠で囲み, 指定の名前をもつTikZの
%% ノードとして出力する. 図には remember picture 属性を付けている
%% ので外部から参照可能である.
\newcommand*{\myAnch}[3]{%
  \tikz[remember picture,baseline=(#1.base)]
    \node[draw,rectangle,line width=1pt,#2] (#1) {\normalcolor #3};
}
%%%%%%%%%%%%%%%%%%%%%%%%%%
%% \myEmph コマンドの定義
%%%%%%%%%%%%%%%%%%%%%%%%%%
%\newcommand{\myEmph}[3]{%
%    \textbf<#1>{\color<#1>{#2}{#3}}%
%}
\usepackage{xparse} % xparseパッケージの読み込み
\NewDocumentCommand{\myEmph}{O{} m m}{%
    \def\argOne{#1}%
    \ifx\argOne\empty
        \textbf{\color{#2}{#3}}% オプション引数が省略された場合
    \else
        \textbf<#1>{\color<#1>{#2}{#3}}% オプション引数が指定された場合
    \fi
}
%%%%%%%%%%%%%%%%%%%%%%%%%%%
%%%%%%%%%%%%%%%%%%%%%%%%%%%
%% 文末の上昇イントネーション記号 \myRisingPitch
%% 通常のイントネーション \myDownwardPitch
%% https://note.com/dan_oyama/n/n8be58e8797b2
%%%%%%%%%%%%%%%%%%%%%%%%%%%
\newcommand{\myRisingPitch}{
\begin{tikzpicture}[scale=0.3,baseline=0.3]
\draw[->,>=stealth] (0,0) to[bend right=45] (1,1);
\end{tikzpicture}
}
\newcommand{\myDownwardPitch}{
\begin{tikzpicture}[scale=0.3,baseline=0.3]
\draw[->,>=stealth] (0,1) to[bend left=45] (1,0);
\end{tikzpicture}
}
%%%%%%%%%%%%%%%%%%%%%%%%%%%%
%\AtBeginSection[%
%]{%
%  \begin{frame}[plain]\frametitle{授業の流れ}
%     \tableofcontents[currentsection]
%   \end{frame}%
%}

\usepackage{lua-ul}
%%%%%%%%%%%%%%%%%%%%%%%%%%%
\makeatletter
\newcommand*{\themonth}{\two@digits\month}
\newcommand*{\theday}{\two@digits\day}
\makeatother
\newcommand{\mytoday}{{\the\year}--{\themonth}--{\theday}}
%%%%%%%%%%%%%%%%%%%%%%%%%%
%%%%%%%%%%%%%%%%%%%%%%%%%%
\title{Today's Pronunciation}
\subtitle{Pronunciation of Japanese Character `ん'}
\date[]{\mytoday}
\author{iotsuka}
\institute{Eduop ちば}

%%%%%%%%%%%%%%%%%%%%%%%%%%%%
%% TEXT
%%%%%%%%%%%%%%%%%%%%%%%%%%%%
\begin{document}
%%%%%%%%%%%%%%%%%%%%%%%%%%%%%%
\gothamset{background=dark}
\maketitle

   \begin{frame}[toc]{Table of contents}%
      \tableofcontents%[hideallsubsections]
   \end{frame}

%%%%%%%%%%%%%%%%%%%%%%%%%%%%%%
\gothamset{background=light}
%%%%%%%%%%%%%%%%%%%%%%%%%%%%
%%%%%%%%%%%%%%%%%%%%%%%%%%%%%%
\section{`ん'をローマ字で書くと}
\begin{frame}[plain]{`ん'をローマ字で書くと}
\centering
\only<2>{\scalebox{.5}{{n}}}
\only<3>{\scalebox{1}{{n}}}
\only<4>{\scalebox{2}{{n}}}
\only<5>{\scalebox{3}{{n}}}
\only<6>{\scalebox{4}{{n}}}
\only<7>{\scalebox{5}{{n}}}
\only<8>{\scalebox{6}{{n}}}
\only<9>{\scalebox{7}{{n}}}
\only<10>{\scalebox{8}{{n}}}
\only<11>{\scalebox{9}{{n}}}
\only<12>{\scalebox{10}{{n}}}
\only<13>{\scalebox{11}{{n}}}
\only<14>{\scalebox{12}{{n}}}
\only<15>{\scalebox{13}{{n}}}
\only<16>{\scalebox{14}{{n}}}
\only<17>{\scalebox{15}{{n}}}
\only<18>{\scalebox{16}{{n}}}
\only<19>{\scalebox{17}{{n}}}
\only<20->{\scalebox{18}{{n}}}

\only<21>{ということは\ldots}
\end{frame}
%%%%%%%%%%%%%%%%%%%%%%%%%%%%%%%
\gothamset{background=light}
\section{`ん'を含む単語を発音してみよう}
\subsection{発音できますか}
\begin{frame}[plain]{発音できますか}
つぎのことばを発音してみましょう

 \setbeamercovered{transparent}
 \begin{itemize}
  \item<2> あ\Circled{ん}ドーナツ
  \item<3> せ\Circled{ん}べい
  \item<4> と\Circled{ん}かつ
  \item<5> こ\Circled{ん}にゃく
  \item<6> せ\Circled{ん}えん
  \item<7> パ\Circled{ン}
 \end{itemize}
\end{frame}
%%%%%%%%%%%%%%%%%%%%%%%%%%%%%%%
\subsection{どのように発音していますか}
\begin{frame}[plain]{この`ん'は}
\begin{columns}\Huge
 \begin{column}{.4\textwidth}
 あ\Circled{ん}ドーナツ

\hspace{18pt}\visible<2->{\textipa{/n/}}
 \end{column}
\begin{column}{.55\textwidth}
\IfFileExists{../images/doughnuts.jpg}{%
\includegraphics[width=.95\textwidth]{../images/doughnuts.jpg}
}{\relax}
\end{column}
\end{columns}
\raggedleft
{\tiny ``fánk/doughnut/gogoşi'' by debreczeniemoke is licensed under CC BY 2.0.}\\[-5pt]
{\tiny To view a copy of this license,}\\[-5pt]
{\tiny visit \url{https://creativecommons.org/licenses/by/2.0/?ref=openverse}.}
\end{frame}
%%%%%%%%%%%%%%%%%%%%%%%%%%%%%%%%%
\begin{frame}[plain]{\textipa{/n/}って発音する\Circled{ん}}

\large
 
\begin{enumerate}[<+->]
 \item あ\Circled{ん}ドーナツ
 \item メ\Circled{ン}チカツ
 \item イ\Circled{ン}ド
 \item 反対(は\Circled{ん}たい)
 \item 関東(か\Circled{ん}とう)
 \item ほ\Circled{ん}とう
% \item \underLine{n}ot
% \item 天丼(て\Circled{ん}どん)
\end{enumerate}

\pause

これらはすべて\textipa{/n/}\pause{}(舌先が上の歯茎にぴったりくっついてる)

\pause

でも、いつも\Circled{ん}を\textipa{/n/}と発音してるわけじゃない

\pause

\begin{quote}
 
天丼の最初の「ん」は\textipa{/n/}だけど、最後の「ん」は\textipa{/n/}じゃない

\pause

関東の「ん」は\textipa{/n/}だけど、関西の「ん」は\textipa{/n/じゃない}
\end{quote}
\end{frame}
%%%%%%%%%%%%%%%%%%%%%%%%%%%%%%
\begin{frame}[plain]{この`ん'は}
\begin{columns}\Huge
 \begin{column}{.3\textwidth}
 せ\Circled{ん}べい

\hspace{16pt}\visible<3->{\textipa{/m/}}

\normalsize

\vspace{50pt}
\visible<2->{唇が\\ぴたっと閉じています}
 \end{column}
\begin{column}{.6\textwidth}
\IfFileExists{../images/senbei.jpg}{%
\includegraphics[width=.95\textwidth]{../images/senbei.jpg}
}{\relax}
\end{column}
\end{columns}
\raggedleft
{\tiny ``Japanese Senbeis’’ by DryPot is licensed under CC BY 2.5.}\\[-5pt]
{\tiny To view a copy of this license, visit \url{https://creativecommons.org/licenses/by/2.5/?ref=openverse}.}
\end{frame}
%%%%%%%%%%%%%%%%%%%%%%%%%%%%%%
\begin{frame}[plain]{\textipa{/m/}って発音する\Circled{ん}}

\large

\begin{enumerate}[<+->]
 \item せ\Circled{ん}べい
 \item 天ぷら(て\Circled{ん}ぷら)
 \item 散歩(さ\Circled{ん}ぽ)
 \item ちゃ\Circled{ん}ぽん
 \item 秋刀魚(さ\Circled{ん}ま)
 \item あ\Circled{ん}まん
\end{enumerate}

\pause

「天ぷら」の\Circled{ん}は\textipa{/m/}だけど、\\
\hspace{80pt}「天丼」の最初の\Circled{ん}は\textipa{/n/}\\\pause
\hspace{80pt}\phantom{「天丼」の}最後の\Circled{ん}は\textipa{/m/}でもないし\textipa{/n/}でもない

\pause

「ちゃんぽん」の最初の\Circled{ん}は\textipa{/m/}だけど、\\
\phantom{「ちゃんぽん」の}最後の\Circled{ん}は\textipa{/m/}でもないし\textipa{/n/}でもない

\end{frame}
%%%%%%%%%%%%%%%%%%%%%%%%%%%%%
\begin{frame}[plain]{この`ん'は}

\begin{columns}\Huge
 \begin{column}{.3\textwidth}
 と\Circled{ん}かつ

\hspace{18pt}\visible<6->{\textipa{/\ng /}}

\vspace{20pt}

\normalsize

\visible<2->{{\small 舌の先?}}
\hfill\visible<3->{{\small \textipa{/n/}じゃない}}

\visible<4->{{\small 口閉じてる?}}
\hfill\visible<5->{{\small \textipa{/m/}じゃない}}
 \end{column}
\begin{column}{.6\textwidth}
\IfFileExists{../images/tonkatsu.jpg}{%
\includegraphics[width=.95\textwidth]{../images/tonkatsu.jpg}
}{\relax}
\end{column}
\end{columns}
\raggedleft
{\tiny ``Kurobuta Tonkatsu 黒豚とんかつ定食'' by jetalone is licensed under CC BY 2.0.}\\[-5pt]
{\tiny To view a copy of this license, visit \url{https://creativecommons.org/licenses/by/2.0/?ref=openverse}.}
\end{frame}
%%%%%%%%%%%%%%%%%%%%%%%%%%%%%%
%%%%%%%%%%%%%%%%%%%%%%%%%%%%%%
\begin{frame}[plain]{\textipa{/N/}って発音する\Circled{ん}}

\large

\begin{enumerate}[<+->]
 \item と\Circled{ん}かつ(とんーーーーーーーーーーーーかつ)
 \item だ\Circled{ん}ご
 \item あ\Circled{ん}こ
 \item 看護(か\Circled{ん}ご)
 \item マ\Circled{ン}ガ
 \item 演歌(え\Circled{ん}か)
 \item 延期(え\Circled{ん}き)
 \item 元気(げ\Circled{ん}き)
\end{enumerate}


舌の背が後ろの歯茎にくっついてます

\end{frame}
%%%%%%%%%%%%%%%%%%%%
\begin{frame}[plain]{いったん、ここまでまとめます}
\large
 
\begin{enumerate}
 \item<2-> 天つゆ(て\Circled{ん}つゆ)\hspace{20pt}\visible<5->{\textipa{/n/}}\hspace{43pt}\visible<8->{\underLine{n}ight / \underLine{n}ot / \underLine{n}otebook}
 \item<3-> 天ぷら(て\Circled{ん}ぷら)\hspace{20pt}\visible<6->{\textipa{/m/}}\hspace{40pt}\visible<9->{\underLine{m}an / \underLine{m}ilk / \underLine{m}eat }
 \item<4-> 天かす(て\Circled{ん}かす)\hspace{20pt}\visible<7->{\textipa{/N/}}\hspace{44pt}\visible<10->{ki\underLine{ng} / si\underLine{ng} / lo\underLine{ng}}
\end{enumerate}


\hfill{\tiny 0230}\,{\scriptsize \myaudio{./audio/japanese_nmng_01.mp3}}
\end{frame}
%%%%%%%%%%%%%%%%%%%%
\begin{frame}[plain]{この`ん'は}
\begin{columns}\Huge
 \begin{column}{.4\textwidth}
 こ\Circled{ん}にゃく

\hspace{18pt}\textipa{/\textltailn /}
 \end{column}
\begin{column}{.575\textwidth}
\IfFileExists{../images/konnyaku.jpg}{%
\includegraphics[width=.95\textwidth]{../images/konnyaku.jpg}
}{\relax}
\end{column}
\end{columns}
\raggedleft
{\tiny ``Konnyaku'' by preetamrai is licensed under CC BY 2.0.}\\[-5pt]
{\tiny To view a copy of this license, visit \url{https://creativecommons.org/licenses/by/2.0/?ref=openverse}.}
\end{frame}
%%%%%%%%%%%%%%%%%%%%%%%%%%%%%
\begin{frame}[plain]{この`ん'は}
 \begin{columns}
 \begin{column}{.35\textwidth}
{\Huge せ\Circled{ん}えん}

\hspace{40pt}{\small 鼻母音}
 \end{column}
\begin{column}{.575\textwidth}
\IfFileExists{../images/konnyaku.jpg}{%
\includegraphics[width=.75\textwidth]{../images/japanese_currency.jpg}}{\relax}
\end{column}
\end{columns}
\raggedleft
{\tiny ``More Funny Money'' by cogdogblog is marked with CC0 1.0.}\\[-5pt]
{\tiny To view the terms, visit \url{https://creativecommons.org/publicdomain/zero/1.0/?ref=openverse}.}
\end{frame}
%%%%%%%%%%%%%%%%%%%%%%%%%%%%%%
\begin{frame}[plain]{この`ん'は}

\begin{columns}\Huge
 \begin{column}{.3\textwidth}
 パ\Circled{ン}

\hspace{18pt}\textipa{/\textscn/}
 \end{column}
\begin{column}{.6\textwidth}
\IfFileExists{../images/bread.jpg}{%
\includegraphics[width=.95\textwidth]{../images/bread_2.jpg}
}{\relax}
\end{column}
\end{columns}
\raggedleft
{\tiny ``rosemary bread.'' by kathryn in stereo is licensed under CC BY-SA 2.0. }\\[-5pt]
{\tiny To view a copy of this license, visit \url{https://creativecommons.org/licenses/by-sa/2.0/?ref=openverse}.}
\end{frame}
%%%%%%%%%%%%%%%%%%%%%%%%%%%%%%
\begin{frame}[plain]{Quiz 1}
下線部の発音が`あ\Circled{ん}ドーナツ'の\Circled{ん}と同じなのはどれですか%
\hfill\visible<2->{\textipa{/n/}}

\begin{enumerate}
 \item あ\underLine{ん}まん\hfill\visible<3->{\textipa{/m/}}
 \item て\underLine{ん}どん(天丼)\hfill\visible<4->{\textipa{/n/}}
 \item て\underLine{ん}ぷら(天ぷら)\hfill\visible<5->{\textipa{/m/}}

\end{enumerate}
\end{frame}

%%%%%%%%%%%%%%%%%%%%%%%%%%%%%%
\begin{frame}[plain]{Quiz 2}
下線部の発音が`せ\Circled{ん}べい'の\Circled{ん}と同じなのはどれですか%
\hfill\visible<2->{\textipa{/m/}}

\begin{enumerate}
 \item みたらしだ\underLine{ん}ご\hfill\visible<3->{\textipa{/N/}}
 \item あ\underLine{ん}にんどうふ(杏仁豆腐)\hfill\visible<4->{\textipa{/\textltailn /}}
 \item ちゃ\underLine{ん}ぽん\hfill\visible<5->{\textipa{/m/}}
\end{enumerate}
\end{frame}
%%%%%%%%%%%%%%%%%%%%%%%%%%
%%%%%%%%%%%%%%%%%%%%%%%%%%%%%%
\begin{frame}[plain]{Quiz 3}
下線部の発音が`と\Circled{ん}かつ'の\Circled{ん}と同じなのはどれですか\hfill\visible<2->{\textipa{/N/}}

\begin{enumerate}
 \item さ\underLine{ん}ま\hfill\visible<3->{\textipa{/m/}}
 \item あ\underLine{ん}ころもち(あんころ餅)\hfill\visible<4->{\textipa{/N/}}
 \item め\underLine{ん}ちかつ(メンチカツ)\hfill\visible<5->{\textipa{/n/}}

\end{enumerate}
\end{frame}
%%%%%%%%%%%%%%%%%%%%%%%%%%%%%
\end{document}
