\documentclass[aspectratio=169]{beamer}
\usepackage[no-math,deluxe,haranoaji]{luatexja-preset}
\renewcommand{\kanjifamilydefault}{\gtdefault}
\renewcommand{\emph}[1]{{\upshape\bfseries #1}}
\usetheme{Madrid}
\setbeamertemplate{navigation symbols}{}
\usecolortheme[rgb={0.7,0.2,0.2}]{structure}
%%%%%%%%%%%%%%%%%%%%%%%%%%%
%% さまざまなアイコン
%%%%%%%%%%%%%%%%%%%%%%%%%%%
\usepackage{fontawesome}
%%%%%%%%%%%%%%%%%%%%%%%%%%%
%% 音声リンク表示
\newcommand{\myaudio}[1]{\href{#1}{\faVolumeUp}}

%%%%%%%%%%%%%%%%%%%%%%%%%%%
\title{be動詞の学習をしましょう}
\author{}
\institute[]{}
\date[]

%%%%%%%%%%%%%%%%%%%%%%%%%%%%
%% TEXT
%%%%%%%%%%%%%%%%%%%%%%%%%%%%
\begin{document}
\begin{frame}[plain]
  \titlepage
\end{frame}

\section*{目次}
\begin{frame}[plain]
  \frametitle{授業の流れ}
  \tableofcontents
\end{frame}

\section{「AはBだ」という表現}
\begin{frame}<1-27>[plain]\frametitle{「AはBだ」という表現}
 % \setbeamercovered{transparent}
  \begin{enumerate}
   \item<1-> I \textbf{\textcolor{red}{am}} a student. \onslide*<2>{わたしは生徒です。}\onslide*<14-20>{(I $=$ a student)}\onslide*<21-26>{\footnotesize  a:1つの、1人の student: 生徒、学生}
   \item<3-> You \textbf{\textcolor{red}{are}} my friend. \onslide*<4>{あなたはわたしのともだちです。}\onslide*<15-20>{(You $=$ my friend)}\onslide*<22-26>{\footnotesize  my: わたしの friend: ともだち}
   \item<5-> He \textbf{\textcolor{red}{is}} tall. \onslide*<6>{彼は背が高い。}\onslide*<16-20>{(He $=$ tall)}\onslide*<23-26>{\footnotesize  tall: 背が高い}
   \item<7-> She \textbf{\textcolor{red}{is}} kind. \onslide*<8>{彼女は親切だ。}\onslide*<17-20>{(She $=$ kind)}\onslide*<24-26>{\footnotesize  kind: 親切な}
   \item<9-> The sky \textbf{\textcolor{red}{is}} blue. \onslide*<10>{空は青い。}\onslide*<18-20>{(The sky $=$ blue)}\onslide*<25-26>{\footnotesize  the sky: 空 blue: 青い}
   \item<11-> They \textbf{\textcolor{red}{are}} my classmates. \onslide*<12>{彼らはわたしのクラスメートです。}\onslide*<19-20>{(They $=$ my classmates)}\onslide*<26>{\footnotesize  classmates: (2人以上の)クラスメート}
  \end{enumerate}

\bigskip

\onslide<20->{\begin{exampleblock}{ポイント}
  am, is, areはイコール($=$)の意味です
     \end{exampleblock}
}

% Embed the sound file
\onslide<27>{%
\myaudio{./hoge.mp3}\,\,{}Listen carefully.(注意して聞いてください)

}
\end{frame}

\section{I am 〜}
\begin{frame}<1-20>[plain]\frametitle{I am 〜.}
 % \setbeamercovered{transparent}
  \begin{enumerate}
   \item<1-> \textbf{\textcolor{red}{I am}} a student. \onslide*<2>{わたしは生徒です。}\onslide*<15-20>{\footnotesize (student: 生徒、学生)}
   \item<3-> \textbf{\textcolor{red}{I am}} tall. \onslide*<4>{わたしは背が高い。}\onslide*<16-20>{\footnotesize (tall: 背が高い)}
   \item<5-> \textbf{\textcolor{red}{I am}} 13 years old. \onslide*<6>{わたしは13歳です。}\onslide*<17-20>{\footnotesize (〜 years old: 〜歳だ) 〜には数字がはいります}
   \item<7-> \textbf{\textcolor{red}{I am}} John. \onslide*<8>{わたしはジョンです。}
   \item<9-> \textbf{\textcolor{red}{I am}} happy. \onslide*<10>{わたしは幸せです。}\onslide*<18-20>{\footnotesize (happy: 幸せだ)}
   \item<11-> \textbf{\textcolor{red}{I am}} from Tokyo. \onslide*<12>{わたしは東京の出身です。}\onslide*<19>{\footnotesize (from 〜: 〜の出身だ)}
  \end{enumerate}

\bigskip

\onslide<14->{\begin{exampleblock}{ポイント}
  amはbe動詞。I am 〜.をひとつのパターンとして覚えよう
     \end{exampleblock}
}

% Embed the sound file
\onslide<20>{%
\myaudio{./hoge.mp3}\,\,{}Listen carefully.(注意して聞いてください)

}


\end{frame}

\section{You are 〜}
\begin{frame}<1-21>[plain]\frametitle{You are 〜.}
 % \setbeamercovered{transparent}
  \begin{enumerate}
   \item<1-> \textbf{\textcolor{red}{You are}} my friend. \onslide*<2>{あなたはわたしのともだちです。}\onslide*<15-20>{\footnotesize (friend: ともだち、友人)}
   \item<3-> \textbf{\textcolor{red}{You are}} very kind. \onslide*<4>{あなたはとても親切だ。}\onslide*<16-20>{\footnotesize (kind: 親切だ)}
   \item<5-> \textbf{\textcolor{red}{You are}} a good student. \onslide*<6>{あなたはいい生徒です。}\onslide*<17-20>{\footnotesize (student: 生徒)}
   \item<7-> \textbf{\textcolor{red}{You are}} good at baseball. \onslide*<8>{あなたは野球がうまい。}\onslide*<18-20>{\footnotesize (good at 〜: 〜がうまい、得意だ)}
   \item<9-> \textbf{\textcolor{red}{You are}} busy. \onslide*<10>{あなたは忙しい。}\onslide*<19-20>{\footnotesize (happy: 幸せだ)}
   \item<11-> \textbf{\textcolor{red}{You are}} from Chiba. \onslide*<12>{あなたは千葉の出身です。}\onslide*<20->{\footnotesize (from 〜: 〜の出身だ)}
  \end{enumerate}

\bigskip

\onslide<14->{\begin{exampleblock}{ポイント}
  areはbe動詞。You are 〜.をひとつのパターンとして覚えよう
     \end{exampleblock}
}

% Embed the sound file
\onslide<21>{%
\myaudio{./hoge.mp3}\,\,{}Listen carefully.(注意して聞いてください)

}

\end{frame}

\section{I am, You are以外の場合(1)}
\begin{frame}<1-21>[plain]\frametitle{I am, You are以外の場合(1)}
 % \setbeamercovered{transparent}
  \begin{enumerate}
   \item<1-> This \textbf{\textcolor{red}{is}} my pencil. \onslide*<2>{これはわたしの鉛筆です。}\onslide*<15-20>{\footnotesize (this: これ pencil: 鉛筆)}
   \item<3-> He \textbf{\textcolor{red}{is}} my classmate. \onslide*<4>{彼はわたしのクラスメートです。}\onslide*<16-20>{\footnotesize (classmate: クラスメート、級友)}
   \item<5-> She \textbf{\textcolor{red}{is}} a good singer. \onslide*<6>{彼女は歌がうまい。}\onslide*<17-20>{\footnotesize (singer: 歌い手、歌手)}
   \item<7-> Your bike \textbf{\textcolor{red}{is}} new. \onslide*<8>{あなたの自転車は新しい。}\onslide*<18-20>{\footnotesize (bike: 自転車 new: 新しい)}
   \item<9-> George \textbf{\textcolor{red}{is}} busy. \onslide*<10>{ジョージは忙しい。}\onslide*<19-20>{\footnotesize (busy: 忙しい)}
   \item<11-> Jane \textbf{\textcolor{red}{is}} from France. \onslide*<12>{ジェーンはフランスの出身です。}\onslide*<20>{\footnotesize (from 〜: 〜の出身だ France: フランス)}
  \end{enumerate}

\bigskip

\onslide<14->{\begin{exampleblock}{ポイント}
  I, You 以外で1つ(This, That, The book \ldots{})、1人(He, She, George, Jane \ldots{})で始まるときはisを使います
     \end{exampleblock}
}

% Embed the sound file
\onslide<21>{%
\myaudio{./hoge.mp3}\,\,{}Listen carefully.(注意して聞いてください)
}

\end{frame}


\section{I am, You are以外の場合(2)}

\begin{frame}<1-21>[plain]\frametitle{I am, You are以外の場合(2)}
 % \setbeamercovered{transparent}
  \begin{enumerate}
   \item<1-> These \textbf{\textcolor{red}{are}} my pencils. \onslide*<2>{これらはわたしの鉛筆です。}\onslide*<15-20>{\footnotesize (these: これら pencil: 鉛筆)}
   \item<3-> They \textbf{\textcolor{red}{are}} my classmates. \onslide*<4>{彼らはわたしのクラスメートです。}\onslide*<16-20>{\footnotesize (they: 彼ら classmate: クラスメート、級友)}
   \item<5-> They \textbf{\textcolor{red}{are}} kind. \onslide*<6>{彼らは親切だ。}\onslide*<17-20>{\footnotesize (kind: 親切な)}
   \item<7-> The flowers \textbf{\textcolor{red}{are}} beautiful. \onslide*<8>{その花は美しい。}\onslide*<18-20>{\footnotesize (flower: 花 beautifuk: 美しい)}
   \item<9-> We \textbf{\textcolor{red}{are}} busy. \onslide*<10>{わたしたちは忙しい。}\onslide*<19-20>{\footnotesize (we: わたしたち busy: 忙しい)}
   \item<11-> Jane and George \textbf{\textcolor{red}{are}} from France. \onslide*<12>{ジェーンとジョージはフランスの出身です。}\onslide*<20>{\footnotesize (from 〜: 〜の出身だ France: フランス)}
  \end{enumerate}

\bigskip

\onslide<14->{\begin{exampleblock}{ポイント}
  I, You 以外で複数(2つ以上)のモノや人で始まるときはareを使います
     \end{exampleblock}
}
% Embed the sound file
\onslide<21>{%
\myaudio{hoge.mp3}\,\,{}Listen carefully.(注意して聞いてください)
}
\end{frame}

\myaudio{hoge.mp3}


\end{document}
