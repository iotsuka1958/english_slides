\documentclass[aspectratio=169,xcolor={dvipsnames,table}]{beamer}
\usepackage[no-math,deluxe,expert,haranoaji]{luatexja-preset}
\usepackage{luatexja-otf}
\renewcommand{\kanjifamilydefault}{\gtdefault}
\renewcommand{\emph}[1]{{\upshape\bfseries #1}}
\usetheme{metropolis}
\metroset{block=fill}
\setbeamertemplate{navigation symbols}{}
\setbeamertemplate{blocks}[rounded][shadow=false]
\usecolortheme[rgb={0.7,0.2,0.2}]{structure}
%%%%%%%%%%%%%%%%%%%%%%%%%%%
\usepackage{media9}
\usepackage[absolute,overlay]{textpos}
%\usepackage[grid=true,gridcolor=Maroon,subgridcolor=gray,gridunit=pt,texcoord]{eso-pic} %場所決めのためのgrid表示
%%%%%%%%%%%%%%%%%%%%%%%%%%%
%% さまざまなアイコン
%%%%%%%%%%%%%%%%%%%%%%%%%%%
\usepackage{fontawesome}
%\usepackage{figchild}
\usepackage{twemojis}
\usepackage{utfsym}
\usepackage{bclogo}
\usepackage{marvosym}
\usepackage{fontmfizz}
\usepackage{pifont}
\usepackage{phaistos}
\usepackage{worldflags}
\usepackage{jigsaw}
\usepackage{tipa}
\usepackage{manfnt}
\usepackage{pxrubrica}
%%%%%%%%%%%%%%%%%%%%%%%%%%%
\usepackage{tikz}
\usetikzlibrary{backgrounds}
\usepackage{tcolorbox}
\usepackage{tikzpeople}
\usepackage{tikzducks}
\usepackage{tikzlings}
\usepackage{circledsteps}
\usepackage{xcolor}
\usepackage{amsmath}
\usepackage{booktabs}
\usepackage{chronology}
\usepackage{signchart}
%%%%%%%%%%%%%%%%%%%%%%%%%%%
%% 場合分け
\usepackage{cases}
%%%%%%%%%%%%%%%%%%%%%%%%%%%
% \myAnch{<名前>}{<色>}{<テキスト>}
% 指定のテキストを指定の色の四角枠で囲み, 指定の名前をもつTikZの
% ノードとして出力する. 図には remeber picture 属性を付けている
% ので外部から参照可能である.
\newcommand*{\myAnch}[3]{%
  \tikz[remember picture,baseline=(#1.base)]
    \node[draw,rectangle,#2] (#1) {\normalcolor #3};
}
%%%%%%%%%%%%%%%%%%%%%%%%%%%%
%% 音声リンク表示
\newcommand{\myaudio}[1]{\href{#1}{\faVolumeUp}}
%%%%%%%%%%%%%%%%%%%%%%%%%%%
% \myEmph コマンドの定義
%\newcommand{\myEmph}[3]{%
%    \textbf<#1>{\color<#1>{#2}{#3}}%
%}
\usepackage{xparse} % xparseパッケージの読み込み
\NewDocumentCommand{\myEmph}{O{} m m}{%
    \def\argOne{#1}%
    \ifx\argOne\empty
        \textbf{\color{#2}{#3}}% オプション引数が省略された場合
    \else
        \textbf<#1>{\color<#1>{#2}{#3}}% オプション引数が指定された場合
    \fi
}
%%%%%%%%%%%%%%%%%%%%%%%%%%%
%% 文末の上昇イントネーション記号 \myRisingPitch
%% 通常のイントネーション \myDownwardPitch
%% https://note.com/dan_oyama/n/n8be58e8797b2
%%%%%%%%%%%%%%%%%%%%%%%%%%%
\newcommand{\myRisingPitch}{
\begin{tikzpicture}[scale=0.3,baseline=0.3]
\draw[->,>=stealth] (0,0) to[bend right=45] (1,1);
\end{tikzpicture}
}
\newcommand{\myDownwardPitch}{
\begin{tikzpicture}[scale=0.3,baseline=0.3]
\draw[->,>=stealth] (0,1) to[bend left=45] (1,0);
\end{tikzpicture}
}
%%%%%%%%%%%%%%%%%%%%%%%%%%%
\title{English is fun.}
\subtitle{I have lost my bag.}
  \author{}
\institute[]{}
\date[]

%%%%%%%%%%%%%%%%%%%%%%%%%%%%
%% TEXT
%%%%%%%%%%%%%%%%%%%%%%%%%%%%
\begin{document}
\begin{frame}[plain]
  \titlepage
\end{frame}

%%%%%%%%%%%%%%%%%%%%%%%%%%%
\section*{授業の流れ}
\begin{frame}[plain]
  \frametitle{授業の流れ}
  \tableofcontents
\end{frame}
%%%%%%%%%%%%%%%%%%%%%%%%%%%%
\section{現在完了とは(復習)}
%%%%%%%%%%%%%%%%%%%%%%%%%%%%
\begin{frame}[plain]{現在完了とは(復習)}
 \begin{enumerate}
 \item Jane \textcolor{NavyBlue}{\bfseries has stayed} in London for six years.
 \item I \textcolor{NavyBlue}{\bfseries have watched} the movie three times.
 \item Bob \textcolor{NavyBlue}{\bfseries has lost} his bag.
\end{enumerate}

\vspace{30pt}

 \begin{block}<2->{現在完了の基本}
\small
\begin{itemize}\setbeamertemplate{items}[square]
 \item<3->  現在完了形\,\Circled[fill color=white]{\,\textbf{have} $+$ 過去分詞\,}\,は「過去と現在にまたがる表現」です\\
\hfill\visible<4->{{\scriptsize 主語が三人称単数のときは$\textcolor{NavyBlue}{\text{\bfseries has} + \text{過去分詞}}$}}
 \item \visible<5->{過去分詞は\kenten{受け身}とならんで\kenten{完了}の意味をあらわします}%
\end{itemize}
      \end{block}
\hfill{\tiny 0144}\,{\scriptsize \myaudio{./audio/014_have_pp_kekka_01.mp3}}

\begin{textblock*}{0.4\linewidth}(365pt,50pt)
\visible<1->{\begin{tikzpicture}
\duck[signpost=\scalebox{0.3}{
\parbox{2.5cm}{\color{black}\centering
{\Large 現在完了$=$\\過去$+$現在}}},
signcolour=brown!70!gray,
signback=white!80!brown,
graduate=gray!20!black,
tassel=red!70!black,
laughing,
%speech={\tiny 反復練習!}
]
\end{tikzpicture}}
\end{textblock*}
\end{frame}
%%%%%%%%%%%%%%%%%%%%%%%%
\section{結果・完了を表す現在完了}
%%%%%%%%%%%%%%%%%%%%%%%%
\begin{frame}[plain]{現在完了--結果--}
 
\visible<1->{\begin{enumerate}%\setcounter{enumi}{1}
 \item $\left\{\begin{tabular}{rl}
(A)& I \textcolor{Maroon}{\bfseries lost} my bag.\hspace{12\zw}{\scriptsize lost: lose\,(失くす)の過去形} \textipa{/l\'Ost/}\\
(B)& I \textcolor{NavyBlue}{\bfseries have lost} my bag.\hspace{9.5\zw}{\scriptsize lost: loseの過去分詞}
\end{tabular}
\right.$
\end{enumerate}}


\visible<2->{\signchart[width=10,height=.5]{,{\textcolor{Maroon}{\textbf{lost}}},,,,,,今}{,,,}}

\vspace{20pt}

\visible<5->{\signchart[width=10,height=.5]{,,,,{\textcolor{NavyBlue}{\textbf{have lost}}},,,今}{,,,}}

\begin{tikzpicture}[overlay]
 %\draw[gray!50] (0,0) grid (12,5);
 %\fill[ForestGreen!70,opacity=.5] (11.57,3.75) circle [radius=.25];
\visible<3->{ \fill[Maroon!70,opacity=.9] (4.25,3.6) circle [radius=.25];}
\visible<5->{\fill[NavyBlue!70,opacity=.9] (4.25,1.2) circle [radius=.25];}
\visible<5->{\fill[NavyBlue!70,opacity=.9] (10.9,1.2) circle [radius=.25];}
\visible<5->{\draw[NavyBlue!70,line width=6pt,opacity=.7] (4.2,1.21) -- (10.9,1.21);}

\visible<3->{\node[] at (4.25,2.9) {\scriptsize \begin{tabular}{c}
				   $\uparrow$\\
				   なくした\end{tabular}};}

\visible<4->{\node[] at (11.6,2.9) {\scriptsize $\left(\begin{tabular}{@{}l@{}}
				   今は\\
				 みつかったかも\end{tabular}\right)$};}

\visible<6->{\node[] at (4.25,.41) {\scriptsize \begin{tabular}{c}
				   $\uparrow$\\
				   なくした\end{tabular}};}

\visible<7->{\node[] at (10.9,.4) {\scriptsize \begin{tabular}{c}
				   $\uparrow$\\
				   今も\\
				 ない\end{tabular}};}
\end{tikzpicture}

\visible<8->{{\scriptsize 過去に起こした動作の現在における\kenten{結果}を表しています}}%
\hfill{\tiny 0111}\,{\scriptsize \myaudio{./audio/014_have_pp_kekka_02.mp3}}

\end{frame}
%%%%%%%%%%%%%%%%%%%%%%%%%%%%%%%%%%%%%%%%%%%%%%%%%
\begin{frame}[plain]{現在完了--完了--}
 
\visible<1->{\begin{enumerate}\setcounter{enumi}{1}
 \item $\left\{\begin{tabular}{rl}
(A)& She \textcolor{Maroon}{\bfseries went} to America.\hspace{90pt}{\scriptsize went: go\,(行く)の過去形 \textipa{/w\'ent/}}\\
(B)& She \textcolor{NavyBlue}{\bfseries has gone} to America.\hspace{70pt}{\scriptsize gone: goの過去分詞 \textipa{/g\'O:n/}}
\end{tabular}
\right.$
\end{enumerate}}


\visible<2->{\signchart[width=10,height=.5]{,{\textcolor{Maroon}{\textbf{went}}},,,,,,今}{,,,}}

\vspace{20pt}

\visible<5->{\signchart[width=10,height=.5]{,,,,{\textcolor{NavyBlue}{\textbf{has gone}}},,,今}{,,,}}

\begin{tikzpicture}[overlay]
 %\draw[gray!50] (0,0) grid (12,5);
 %\fill[ForestGreen!70,opacity=.5] (11.57,3.75) circle [radius=.25];
\visible<3->{ \fill[Maroon!70,opacity=.9] (4.25,3.62) circle [radius=.25];}
\visible<5->{\fill[NavyBlue!70,opacity=.9] (4.25,1.2) circle [radius=.25];}
\visible<5->{\fill[NavyBlue!70,opacity=.9] (10.9,1.2) circle [radius=.25];}
\visible<5->{\draw[NavyBlue!70,line width=6pt,opacity=.7] (4.2,1.21) -- (10.9,1.21);}

\visible<3->{\node[] at (4.25,2.9) {\scriptsize \begin{tabular}{c}
				   $\uparrow$\\
				   行った\end{tabular}};}

\visible<4->{\node[] at (11.6,2.9) {\scriptsize $\left(\begin{tabular}{@{}l@{}}
				   今どこにいるかは\\
				 わからない\end{tabular}\right)$};}

\visible<6->{\node[] at (4.25,.41) {\scriptsize \begin{tabular}{c}
				   $\uparrow$\\
				   行った\end{tabular}};}

\visible<7->{\node[] at (10.9,0.22) {\scriptsize \begin{tabular}{c}
				   $\uparrow$\\
				   今もアメリカにいる\\
				 (ここにはいない)\end{tabular}};}
\end{tikzpicture}

\visible<8->{{\scriptsize \kenten{完了}(~してしまった)を表しています}}%
\hfill{\tiny 0112}\,{\scriptsize \myaudio{./audio/014_have_pp_kekka_03.mp3}}

\end{frame}
%%%%%%%%%%%%%%%%%%%%%%%%%%%%%%%%%
\begin{frame}[plain]{Exercises}
 次の各組の2文の意味の違いについて考えましょう%


\begin{enumerate}
 \item $\left\{\begin{tabular}{rl}
(A)& George \textcolor{Maroon}{\bfseries bought} a new car.\hspace{43pt}{\scriptsize buy \textipa{/b\'aI/} --- bought \textipa{/b\'O:t/} --- bought \textipa{/b\'O:t/}}\\
(B)& George \textcolor{NavyBlue}{\bfseries has bought} a new car.
\end{tabular}
\right.$

 \item $\left\{\begin{tabular}{rl}
(A)& He \textcolor{Maroon}{\bfseries went} to Australia.\hspace{78pt}{\scriptsize go \textipa{/g\'oU/} --- went \textipa{/w\'ent/} --- gone \textipa{/g\'O:n/}}\\
(B)& He \textcolor{NavyBlue}{\bfseries has gone} to Australia.
\end{tabular}
\right.$

\item $\left\{\begin{tabular}{rl}
(A)&Bob \textcolor{Maroon}{\bfseries lost} his job.\hspace{40pt}{\scriptsize lose\textipa{/l\'u:z/} --- lost \textipa{/l\'O:st/} --- lost \textipa{/l\'O:st/}\hspace{15pt}job \textipa{/j\'Ab/} 仕事}\\
(B)&Bob \textcolor{NavyBlue}{\bfseries has lost} his job.
\end{tabular}
\right.$
\end{enumerate}

\hfill{\tiny 0246}\,{\scriptsize \myaudio{./audio/014_have_pp_kekka_04.mp3}}
\end{frame}
%%%%%%%%%%%%%%%%%%%%%%%%%%%%
\begin{frame}[plain]{Exercises}
 あたえられた日本語の意味になるよう、次の英文の(~~~~~~~~)内から正しいものを選び、○で囲みましょう%
\mbox{}\hfill{\scriptsize \myaudio{./audio/014_have_pp_kekka_05.mp3}}


\begin{enumerate}
 \item {\small わたしは辞書をなくしてしまった。}{\footnotesize (今もまだ見つかっていない)}\\
I  ( lose / \alt<2->{\Circled[outer color=orange]{have lost}}{have lost} / lost ) my dictionary.\hfill{\scriptsize dictionary \textipa{/d\'IkS@n\`eri/} 辞書}
 \item {\small 父はきのうバッグをなくした。\textdbend}\\
My father ( \alt<3->{\Circled[outer color=orange]{lost}}{lost} / has lost / loses ) his bag \myEmph[3-]{Maroon}{yesterday}.
 \item {\small 彼はフランスに行ってしまった。}{\footnotesize (今もフランスにいる)}\\
He ( have gone / \alt<4->{\Circled[outer color=orange]{has gone}}{has gone} / went  ) to France.
 \item {\small 彼女は2年前にイタリアに行きました。\textdbend}\\
She ( \alt<5->{\Circled[outer color=orange]{went}}{went} / has gone ) to Italy \myEmph[5-]{Maroon}{two years ago}.
\end{enumerate}

\begin{block}<6->{Topic for Today}\small
 \begin{itemize}\setbeamertemplate{items}[square]\small
 \item 現在完了形は「あきらかに過去を示す表現」といっしょには使えません\,\,\,\,\raisebox{-7pt}{\textdbend}
 \end{itemize}
\end{block}

\normalsize
\begin{textblock*}{0.4\linewidth}(310pt,112pt)
\visible<1->{\begin{tikzpicture}
\bear[
scale=1,
speech={\tiny  過去を示す表現},
signpost=\scalebox{.5}{
\parbox{2.5cm}{\color{black}
\centering 現在完了では使えない!}},
signcolour= brown!70!gray,
signback=white!80!brown
]
\end{tikzpicture}}
\end{textblock*}
\end{frame}
%%%%%%%%%%%%%%%%%%%%%%%%%%%%%%%%%%%
\section{現在完了形といっしょに用いる副詞}
%%%%%%%%%%%%%%%%%%%%%%%%%%%%%%%%%%
\begin{frame}[plain,t]{現在完了形と組み合わせるjust, already, yet}
\begin{enumerate}
 \item<1-> I \myEmph[1-]{NavyBlue}{have finished} the task.%
\hfill{}{\scriptsize task \textipa{/t\'\ae sk/} 仕事}
 \item<2-> I \myEmph[2-]{NavyBlue}{have} \myEmph[2-]{Maroon}{just} \myEmph[2-]{NavyBlue}{finished} the task.%
\hfill{}{\scriptsize just \textipa{/dZ\'\textturnv st/} たった今、ちょうど}
 \item<3-> I \myEmph[3-]{NavyBlue}{have} \myEmph[3-]{Maroon}{already} \myEmph[3-]{NavyBlue}{finished} the task.%
\hfill{}{\scriptsize already \textipa{/O:lr\'edi/} すでに}
 \item<4-> I \myEmph[4-]{NavyBlue}{have} \myEmph[4-]{Maroon}{not} \myEmph[4-]{NavyBlue}{finished} the task \myEmph[4-]{Maroon}{yet}.%
\hfill{}{\scriptsize yet \textipa{/j\'et/} (否定文で)まだ(~ない)}
 \item<5-> \myEmph[5-]{NavyBlue}{Have} you \myEmph[5-]{NavyBlue}{finished} the task \myEmph[4-]{Maroon}{yet}?%
\hfill{}{\scriptsize yet \textipa{/j\'et/} (疑問文で)もう}
\end{enumerate}

\bigskip

 \begin{block}<6->{Topics for Today}
\small
\begin{itemize}\setbeamertemplate{items}[square]\small
 \item \textbf{just}(たった今、ちょうど)、\textbf{already}(すでに)、\textbf{yet}(否定文で「まだ(~ない)」、疑問文で「もう」)を現在完了形と組み合わせることがあります
 \item \textbf{just}や\textbf{already}は\textbf{have}のあとにきます%
\hspace{70pt}\textbf{have} $\left\{
       \begin{tabular}{l}
	\textbf{just}\\
	\textbf{already}
       \end{tabular}
\right\}$ 過去分詞
 \item 否定文や疑問文の\textbf{yet}は文末にきます
\end{itemize}
      \end{block}

\vspace{-10pt}

\hfill{\scriptsize 0229}\,{\scriptsize \myaudio{./audio/014_have_pp_kekka_05b.mp3}}

\end{frame}
%%%%%%%%%%%%%%%%%%%%%%%%%%%%%%%%%%%
\begin{frame}[plain]{Exercises}
 あたえられた動詞をもちいて英語で表現してください。動詞は必要に応じて適当な形にしましょう

\begin{enumerate}
 \item 彼女は朝食をもう食べましたか( eat )\hfill{\scriptsize breakfast \textipa{/br\'ekf@st/} 朝食}\\
\visible<2->{Has she eaten breakfast yet?}
 \item 彼はまだ昼食を食べていません( eat )\hfill{\scriptsize lunch \textipa{/l\'\textturnv ntS/} 昼食}\\
\visible<3->{He has not eaten lunch yet.}
 \item わたしはたったいま宿題を終えたところだ( finish )\hfill{\scriptsize homework \textipa{/h\'oUmw\`\textrhookschwa :k/} 宿題}\\
\visible<4->{I have just finished my homework.}
 \item その会議はもう始まっています( start )\hfill{\scriptsize meeting \textipa{/m\'\i:tIN/} 会議}\\
\visible<5>{The meeting has already started.}
\end{enumerate}

\hfill{\scriptsize 0203}\,{\scriptsize \myaudio{./audio/014_have_pp_kekka_06.mp3}}

\end{frame}
%%%%%%%%%%%%%%%%%%%%%%%%%%%%%%%%%%%%
\section{まとめ}
\begin{frame}[plain,t]{まとめ1}

 \begin{block}<1->{「結果・完了」を表す現在完了}
\small
\begin{itemize}\setbeamertemplate{items}[square]\small
 \item[]<1-> 基本: 現在完了\,\Circled[fill color=white]{have $+$ 過去分詞\,}\,は「過去と現在にまたがる表現」
 \item<2-> 「結果・完了」(~してしまった、~したところだ)を表す用法があります\\
\hfill{}I {\bfseries have lost} my bag. / I {\bfseries have eaten} breakfast.
\end{itemize}
\end{block}
\vspace{-4pt}
\begin{block}<3->{現在完了と組み合わせる副詞}
\small
\begin{itemize}\setbeamertemplate{items}[square]\small
\setlength{\itemsep}{-1pt}
 \item<4-> \textbf{already}(すでに)
 \hfill{}She {\bfseries has} \textcolor{Maroon}{\bfseries already} {\bfseries cleaned} her room.
 \item<5-> \textbf{just}(たった今、ちょうど)
\hfill{}The movie {\bfseries has} \textcolor{Maroon}{\bfseries just} {\bfseries started}.
 \item<6-> \textbf{yet}\hspace{5pt}疑問文で「もう」%
\hfill{}{\bfseries Have} you {\bfseries finished} your homework \textcolor{Maroon}{\bfseries yet}?\\
\mbox{}\hspace{19pt}否定文で「まだ(~ない)」\hfill{}I {\bfseries have} \textcolor{Maroon}{\bfseries not} {\bfseries finished} my homework \textcolor{Maroon}{\bfseries yet}.
\item<7-> \textbf{already}, \textbf{just}は\textbf{have}のあと%
\hfill\textbf{have} $\left\{
       \begin{tabular}{l}
	\textbf{already}\\
	\textbf{just}
       \end{tabular}
\right\}$ 過去分詞\\
\textbf{yet}は文末
\end{itemize}
      \end{block}

\vspace{-15pt}

\hfill{\scriptsize 0254}\,{\scriptsize \myaudio{./audio/014_have_pp_kekka_07.mp3}}

\end{frame}
%%%%%%%%%%%%%%%%%%%%%%%%%%%%%%%
\begin{frame}[plain]{まとめ2}
  \begin{block}<1->{現在完了形といっしょに使えない表現}
\small

\begin{itemize}\setbeamertemplate{items}[square]
 \item {現在完了形は「あきらかに過去を表す表現」といっしょには使えません\textdbend}
\end{itemize}

       \begin{enumerate}
	\item \begin{enumerate}
	       \item \visible<2->{I have lost my key.}
	       \item \visible<3->{*I have lost my key \textbf{yesterday}.\hfill{}* {\scriptsize まちがった英文ですという記号}}
	       \item \visible<4->{I lost my key yesterday.}
	      \end{enumerate}
	\item \begin{enumerate}
	       \item \visible<5->{She has gone to Italy.}
	       \item \visible<6->{*She has gone to Italy \textbf{two years ago}.}\\
	       \item \visible<7->{She went to Italy two years ago.}
	      \end{enumerate}
       \end{enumerate}
\end{block}

\hfill{\scriptsize 0145}\,{\scriptsize \myaudio{./audio/014_have_pp_kekka_07b.mp3}}

\end{frame}
%%%%%%%%%%%%%%%%%%%%%%%%%%%%%%%%
\section{もうひとつだけ}
%%%%%%%%%%%%%%%%%%%%%%%%%%%%%%%
\begin{frame}[plain,t]{~へ行ったことがある}

\begin{enumerate}
 \item<1-> She \textbf{went to} Hawaii.
 \item<2-> She \textbf{has gone to} Hawaii.
 \item<3-> She \textbf{has been to} Hawaii.
 \item<4-> \textbf{Have} you \myEmph{Maroon}{ever} \textbf{been to} Hawaii?
 \item<5-> I \textbf{have} \myEmph{Maroon}{never} \textbf{been to} Hawaii.
\end{enumerate}

\vfill

\begin{block}<5->{Topic for Today}\small
%\textdbend
\begin{itemize}\setbeamertemplate{items}[square]\small
 \item \textbf{have been to ~}は「~に行ったことがある」という意味で、\kenten{経験}をあらわす決まりきった表現です
\end{itemize}
\end{block}

\vspace{-8pt}

\hfill{\scriptsize 0225}\,{\scriptsize \myaudio{./audio/014_have_pp_kekka_08.mp3}}

\end{frame}
%%%%%%%%%%%%%%%%%%%%%%%%%%%%%
%%%%%%%%%%%%%%%%%%%%%%%%%%%%%%%%%%%%%%%%%%
{
  \usebackgroundtemplate{\includegraphics[height=\paperheight]{./images/hiyashi_chuka.jpg}}
  \begin{frame}[t]
    \frametitle{We have started serving chilled Chinese noodles.}
\tiny
\raggedright
  \textcolor{white}{ ``冷やし中華'' by Roulex45 is licensed under CC BY 2.0. }\\
   \textcolor{white}{To view a copy of this license,}\\
\textcolor{white}{visit \url{https://creativecommons.org/licenses/by/2.0/?ref=openverse}}.
  \end{frame}
}
%%%%%%%%%%%%%%%%%%%%%%%%%%%%%%%%%%%%%%%%%%%
\end{document}
