\documentclass[aspectratio=169,xcolor={dvipsnames,table}]{beamer}
\usepackage[no-math,deluxe,expert,haranoaji]{luatexja-preset}
\usepackage{luatexja-otf}
\renewcommand{\kanjifamilydefault}{\gtdefault}
\renewcommand{\emph}[1]{{\upshape\bfseries #1}}
\usetheme{metropolis}
\metroset{block=fill}
\setbeamertemplate{navigation symbols}{}
\usecolortheme[rgb={0.7,0.2,0.2}]{structure}
%%%%%%%%%%%%%%%%%%%%%%%%%%%
\usepackage{media9}
%%%%%%%%%%%%%%%%%%%%%%%%%%%
%% さまざまなアイコン
%%%%%%%%%%%%%%%%%%%%%%%%%%%
\usepackage{fontawesome}
\usepackage{figchild}
\usepackage{twemojis}
\usepackage{utfsym}
\usepackage{bclogo}
\usepackage{marvosym}
\usepackage{fontmfizz}
\usepackage{pifont}
\usepackage{phaistos}
\usepackage{worldflags}
\usepackage{jigsaw}
%%%%%%%%%%%%%%%%%%%%%%%%%%%
\usepackage{tikz}
\usetikzlibrary{backgrounds}
\usepackage{tcolorbox}
\usepackage{tikzpeople}
\usepackage{circledsteps}
\usepackage{xcolor}
\usepackage{amsmath}
\usepackage{booktabs}
\usepackage{chronology}
\usepackage{signchart}
%%%%%%%%%%%%%%%%%%%%%%%%%%%
%% 場合分け
\usepackage{cases}
%%%%%%%%%%%%%%%%%%%%%%%%%%%
% \myAnch{<名前>}{<色>}{<テキスト>}
% 指定のテキストを指定の色の四角枠で囲み, 指定の名前をもつTikZの
% ノードとして出力する. 図には remeber picture 属性を付けている
% ので外部から参照可能である.
\newcommand*{\myAnch}[3]{%
  \tikz[remember picture,baseline=(#1.base)]
    \node[draw,rectangle,#2] (#1) {\normalcolor #3};
}
%%%%%%%%%%%%%%%%%%%%%%%%%%%%
%% 音声リンク表示
\newcommand{\myaudio}[1]{\href{#1}{\faVolumeUp}}
%%%%%%%%%%%%%%%%%%%%%%%%%%%
% \myEmph コマンドの定義
%\newcommand{\myEmph}[3]{%
%    \textbf<#1>{\color<#1>{#2}{#3}}%
%}
\usepackage{xparse} % xparseパッケージの読み込み
\NewDocumentCommand{\myEmph}{O{} m m}{%
    \def\argOne{#1}%
    \ifx\argOne\empty
        \textbf{\color{#2}{#3}}% オプション引数が省略された場合
    \else
        \textbf<#1>{\color<#1>{#2}{#3}}% オプション引数が指定された場合
    \fi
}
%%%%%%%%%%%%%%%%%%%%%%%%%%%
%% 文末の上昇イントネーション記号 \myRisingPitch
%% 通常のイントネーション \myDownwardPitch
%% https://note.com/dan_oyama/n/n8be58e8797b2
%%%%%%%%%%%%%%%%%%%%%%%%%%%
\newcommand{\myRisingPitch}{
\begin{tikzpicture}[scale=0.3,baseline=0.3]
\draw[->,>=stealth] (0,0) to[bend right=45] (1,1);
\end{tikzpicture}
}
\newcommand{\myDownwardPitch}{
\begin{tikzpicture}[scale=0.3,baseline=0.3]
\draw[->,>=stealth] (0,1) to[bend left=45] (1,0);
\end{tikzpicture}
}
%%%%%%%%%%%%%%%%%%%%%%%%%%%
\title{English is fun.\,\,{}--- I have lost my bag. ---}
  \author{}
\institute[]{}
\date[]

%%%%%%%%%%%%%%%%%%%%%%%%%%%%
%% TEXT
%%%%%%%%%%%%%%%%%%%%%%%%%%%%
\begin{document}
\begin{frame}[plain]
  \titlepage
\end{frame}

\section*{授業の流れ}
\begin{frame}[plain]
  \frametitle{授業の流れ}
  \tableofcontents
\end{frame}

\section{現在完了--結果--}
\subsection{現在完了とは(復習)}

\begin{frame}<1-3>[plain]{現在完了とは(復習)}
 \begin{enumerate}
 \item Jane has stayed in London for six years.\visible<2->{\small (継続)}
 \item I have watched the movie three times.\visible<3->{\small (経験)}
 \item Bob has lost his bag.
\end{enumerate}



 \begin{exampleblock}{Topic for Today}
\small
\begin{itemize}
 \item  $\text{現在完了(}=\textcolor{NavyBlue}{\text{have} + \text{過去分詞\,)}}$%
は「過去と現在にまたがる表現」です
\item 主語が三人称単数のときは {\textcolor{NavyBlue}{\bfseries has $+$ 過去分詞}}
\end{itemize}
      \end{exampleblock}
\end{frame}

\subsection{現在完了--結果--}
\begin{frame}[plain]{現在完了--結果--}
 
\visible<1->{\begin{enumerate}%\setcounter{enumi}{1}
 \item $\left\{\begin{tabular}{rl}
(A)& I \textcolor{Maroon}{\bfseries lost} my bag.\hspace{7\zw}{\scriptsize lost: lose\,(失くす)の過去形}\\
(B)& I \textcolor{NavyBlue}{\bfseries have lost} my bag.\hspace{5\zw}{\scriptsize lost: lose\,(失くす)の過去分詞}
\end{tabular}
\right.$
\end{enumerate}}


\visible<2->{\signchart[width=10,height=.5]{,{\textcolor{Maroon}{lost}},,,,,,今}{,,,}}

\vspace{20pt}

\visible<5->{\signchart[width=10,height=.5]{,,,,{\textcolor{NavyBlue}{have lost}},,,}{,,,}}

\begin{tikzpicture}[overlay]
 %\draw[gray!50] (0,0) grid (12,5);
 %\fill[ForestGreen!70,opacity=.5] (11.57,3.75) circle [radius=.25];
\visible<3->{ \fill[Maroon!70,opacity=.9] (4.25,3.64) circle [radius=.25];}
\visible<5->{\fill[NavyBlue!70,opacity=.9] (4.25,1.2) circle [radius=.25];
\draw[NavyBlue!70,line width=6pt,opacity=.7] (4.2,1.21) -- (10.9,1.21);}

\visible<3->{\node[] at (4.25,2.9) {\scriptsize \begin{tabular}{c}
				   $\uparrow$\\
				   なくした\end{tabular}};}

\visible<4->{\node[] at (11.6,2.9) {\scriptsize $\left(\begin{tabular}{@{}l@{}}
				   今は\\
				 みつかったかも\end{tabular}\right)$};}

\visible<6->{\node[] at (4.25,.41) {\scriptsize \begin{tabular}{c}
				   $\uparrow$\\
				   なくした\end{tabular}};}

\visible<7->{\node[] at (10.9,.4) {\scriptsize \begin{tabular}{c}
				   $\uparrow$\\
				   今も\\
				 ない\end{tabular}};}
\end{tikzpicture}

\visible<8->{過去に起こした動作の現在における「結果」を表しています}
\end{frame}

\begin{frame}[plain]{現在完了--結果--}
 
\visible<1->{\begin{enumerate}\setcounter{enumi}{1}
 \item $\left\{\begin{tabular}{rl}
(A)& She \textcolor{Maroon}{\bfseries went} to America.\hspace{7\zw}{\scriptsize went: go\,(行く)の過去形}\\
(B)& She \textcolor{NavyBlue}{\bfseries has gone} to America.\hspace{5\zw}{\scriptsize gone: go\,(行く)の過去分詞}
\end{tabular}
\right.$
\end{enumerate}}


\visible<2->{\signchart[width=10,height=.5]{,{\textcolor{Maroon}{went}},,,,,,今}{,,,}}

\vspace{20pt}

\visible<5->{\signchart[width=10,height=.5]{,,,,{\textcolor{NavyBlue}{has gone}},,,}{,,,}}

\begin{tikzpicture}[overlay]
 %\draw[gray!50] (0,0) grid (12,5);
 %\fill[ForestGreen!70,opacity=.5] (11.57,3.75) circle [radius=.25];
\visible<3->{ \fill[Maroon!70,opacity=.9] (4.25,3.64) circle [radius=.25];}
\visible<5->{\fill[NavyBlue!70,opacity=.9] (4.25,1.2) circle [radius=.25];
\draw[NavyBlue!70,line width=6pt,opacity=.7] (4.2,1.21) -- (10.9,1.21);}

\visible<3->{\node[] at (4.25,2.9) {\scriptsize \begin{tabular}{c}
				   $\uparrow$\\
				   行った\end{tabular}};}

\visible<4->{\node[] at (11.6,2.9) {\scriptsize $\left(\begin{tabular}{@{}l@{}}
				   今どこにいるかは\\
				 わからない\end{tabular}\right)$};}

\visible<6->{\node[] at (4.25,.41) {\scriptsize \begin{tabular}{c}
				   $\uparrow$\\
				   行った\end{tabular}};}

\visible<7->{\node[] at (10.9,.4) {\scriptsize \begin{tabular}{c}
				   $\uparrow$\\
				   今もアメリカにいる\\
				 (ここにはいない)\end{tabular}};}
\end{tikzpicture}

\visible<8->{過去に起こした動作の現在における「結果」を表しています}
\end{frame}

\begin{frame}[plain]{Exercises}
 次の各組の2文の意味の違いについて考えましょう。

\begin{enumerate}
 \item $\left\{\begin{tabular}{rl}
(A)& George \textcolor{Maroon}{\bfseries bought} a new car.\hspace{6\zw}{\small bought: buy\,(買う)の過去形}\\
(B)& George \textcolor{NavyBlue}{\bfseries has bought} a new car.
\end{tabular}
\right.$

 \item $\left\{\begin{tabular}{rl}
(A)& He \textcolor{Maroon}{\bfseries went} to Australia.\hspace{9\zw}{\small went: go\,(行く)の過去形}\\
(B)& He \textcolor{NavyBlue}{\bfseries has gone} to Australia.\hspace{7\zw}{\small gone: go\,(行く)の過去分詞}
\end{tabular}
\right.$

\item $\left\{\begin{tabular}{rl}
(A)&Bob \textcolor{Maroon}{\bfseries lost} his job.\hspace{4\zw}{\small job: 仕事}\\
(B)&Bob \textcolor{NavyBlue}{\bfseries has lost} his job.\hspace{2\zw}{\small lose\,(失くす)の過去形、過去分詞はともにlost}
\end{tabular}
\right.$
\end{enumerate}
\end{frame}

\begin{frame}[plain]{Exercises}
 あたえられた日本語の意味になるよう、次の英文の(~~~~~~~~)内から正しいものを選び、○で囲みましょう。

\begin{enumerate}
 \item わたしは辞書をなくしてしまった。{\footnotesize (今もまだ見つかっていない)}\\
I  ( lose / \alt<2->{\Circled[outer color=orange]{have lost}}{have lost} / lost ) my dictionary\hfill{\small dictionary: 辞書}.
 \item 父はきのうバッグをなくしてしまった。\\
My father( \alt<3->{\Circled[outer color=orange]{lost}}{lost} / has lost / loses ) his bag yesterday.
 \item 彼はフランスに行ってしまった。{\footnotesize (今もフランスにいる)}\\
He ( have gone /\alt<4->{\Circled[outer color=orange]{has gone}}{has gone} / went  ) to France.
 \item 彼女は2年前にイタリアに行きました。\\
She ( \alt<5->{\Circled[outer color=orange]{went}}{went} / has gone ) to Italy two years ago.
\end{enumerate}
\end{frame}

\subsection{まとめ}
\begin{frame}<1-10>[plain]{まとめ}

 \begin{exampleblock}<1->{Topics for Today \textcolor{black}{\mdseries --「結果」を表す現在完了--}}
\small


\begin{itemize}
 \item[]<2-> 基本: $\text{現在完了(}=\textcolor{NavyBlue}{\text{have} + \text{過去分詞\,})}$は「過去と現在にまたがる表現」

 \item \visible<3->{「結果」を表す用法があります}

 \item \visible<4->{現在完了は、あきらかに過去を表す表現といっしょに使えません}
       \begin{enumerate}
	\item \begin{enumerate}
	       \item \visible<5->{I have lost my key.}
	       \item \visible<6->{*I have lost my key yesterday.}
	       \item \visible<7->{I lost my key yesterday.}
	      \end{enumerate}
	\item \begin{enumerate}
	       \item \visible<8->{She has gone to Italy.}
	       \item \visible<9->{*She has gone to Italy two years ago.}\\
	       \item \visible<10->{She went to Italy two years ago.}
	      \end{enumerate}
       \end{enumerate}

\end{itemize}
      \end{exampleblock}
\end{frame}





\end{document}
