\documentclass[aspectratio=169,xcolor={dvipsnames,table}]{beamer}
\usepackage[no-math,deluxe,haranoaji]{luatexja-preset}
\renewcommand{\kanjifamilydefault}{\gtdefault}
\renewcommand{\emph}[1]{{\upshape\bfseries #1}}
\usetheme{metropolis}
\metroset{block=fill}
\setbeamertemplate{navigation symbols}{}
\usecolortheme[rgb={0.7,0.2,0.2}]{structure}
%%%%%%%%%%%%%%%%%%%%%%%%%%%
\usepackage{media9}
%%%%%%%%%%%%%%%%%%%%%%%%%%%
%% さまざまなアイコン
%%%%%%%%%%%%%%%%%%%%%%%%%%%
\usepackage{fontawesome}
\usepackage{figchild}
\usepackage{twemojis}
\usepackage{utfsym}
\usepackage{bclogo}
\usepackage{marvosym}
\usepackage{fontmfizz}
\usepackage{pifont}
\usepackage{phaistos}
\usepackage{worldflags}
%%%%%%%%%%%%%%%%%%%%%%%%%%%
\usepackage{tikz}
\usetikzlibrary{backgrounds}
\usepackage{tcolorbox}
\usepackage{tikzpeople}
\usepackage{circledsteps}
\usepackage{xcolor}
\usepackage{amsmath}
\usepackage{booktabs}
\usepackage{chronology}
\usepackage{signchart}
%%%%%%%%%%%%%%%%%%%%%%%%%%%
%% 場合分け
\usepackage{cases}
%%%%%%%%%%%%%%%%%%%%%%%%%%%
% \myAnch{<名前>}{<色>}{<テキスト>}
% 指定のテキストを指定の色の四角枠で囲み, 指定の名前をもつTikZの
% ノードとして出力する. 図には remeber picture 属性を付けている
% ので外部から参照可能である.
\newcommand*{\myAnch}[3]{%
  \tikz[remember picture,baseline=(#1.base)]
    \node[draw,rectangle,#2] (#1) {\normalcolor #3};
}
%%%%%%%%%%%%%%%%%%%%%%%%%%%%
%% 音声リンク表示
\newcommand{\myaudio}[1]{\href{#1}{\faVolumeUp}}
%%%%%%%%%%%%%%%%%%%%%%%%%%%
% \myEmph コマンドの定義
%\newcommand{\myEmph}[3]{%
%    \textbf<#1>{\color<#1>{#2}{#3}}%
%}
\usepackage{xparse} % xparseパッケージの読み込み
\NewDocumentCommand{\myEmph}{O{} m m}{%
    \def\argOne{#1}%
    \ifx\argOne\empty
        \textbf{\color{#2}{#3}}% オプション引数が省略された場合
    \else
        \textbf<#1>{\color<#1>{#2}{#3}}% オプション引数が指定された場合
    \fi
}
%%%%%%%%%%%%%%%%%%%%%%%%%%%
%% 文末の上昇イントネーション記号 \myRisingPitch
%% 通常のイントネーション \myDownwardPitch
%% https://note.com/dan_oyama/n/n8be58e8797b2
%%%%%%%%%%%%%%%%%%%%%%%%%%%
\newcommand{\myRisingPitch}{
\begin{tikzpicture}[scale=0.3,baseline=0.3]
\draw[->,>=stealth] (0,0) to[bend right=45] (1,1);
\end{tikzpicture}
}
\newcommand{\myDownwardPitch}{
\begin{tikzpicture}[scale=0.3,baseline=0.3]
\draw[->,>=stealth] (0,1) to[bend left=45] (1,0);
\end{tikzpicture}
}
%%%%%%%%%%%%%%%%%%%%%%%%%%%
\title{English is fun.\,\,{}--- I have lived in London for ten years. ---}
  \author{}
\institute[]{}
\date[]

%%%%%%%%%%%%%%%%%%%%%%%%%%%%
%% TEXT
%%%%%%%%%%%%%%%%%%%%%%%%%%%%
\begin{document}
\begin{frame}[plain]
  \titlepage
\end{frame}

\section*{授業の流れ}
\begin{frame}[plain]
  \frametitle{授業の流れ}
  \tableofcontents
\end{frame}

\section{現在完了}
\subsection{現在完了とは}

\begin{frame}[plain]{現在完了とは}
 

\begin{enumerate}
 \item \visible<1->{I \textcolor{ForestGreen}{\bfseries live} in London now. {\small わたしはいまロンドンに住んでいます。(現在形)}}
\visible<2->{\signchart[width=10]{,,,,,,,,,,{\textcolor{ForestGreen}{live}}}{,,,}}
 \item \visible<3->{I \textcolor{Maroon}{\bfseries lived} in London five years ago.{\small わたしは5年前ロンドンに住んでいた。(過去形)}}
\visible<4->{\signchart[width=10,height=.5]{,,,,,{\textcolor{Maroon}{lived}},,,,,}{,,,}}

 \item \visible<5->{I \textcolor{NavyBlue}{\bfseries have lived} in London for eight years.}

\visible<6->{\signchart[width=10,height=.5]{,,8年前,,,,{\textcolor{NavyBlue}{have lived}},,,,今}{,,,}}
\end{enumerate}

\vspace*{-20pt}

\begin{tikzpicture}[overlay]
 %\draw[gray!50] (0,0) grid (12,5);
 \visible<2->{\fill[ForestGreen!70,opacity=.5] (11.57,3.75) circle [radius=.25];}
\visible<4-> {\fill[Maroon!70,opacity=.5] (7.4,2.05) circle [radius=.25];}
\visible<6->{ \draw[NavyBlue!70,line width=6pt,opacity=.7] (4.88,0.3) -- (11.55,0.3);}
\end{tikzpicture}

\begin{exampleblock}<7->{Topic for Today}
\small
\begin{itemize}
 \item  $\text{現在完了(}=\textcolor{NavyBlue}{\text{have} + \text{過去分詞\,)}}$%
は「過去と現在にまたがる表現」です
\end{itemize}
      \end{exampleblock}
\end{frame}


\begin{frame}[plain]{Exercises}
 次の各組の2文の意味の違いについて考えましょう。

\begin{enumerate}
 \item $\left\{\begin{tabular}{rl}
(A)& Jane \textcolor{Maroon}{\bfseries stayed} in London two years ago.\hspace{9\zw}{\small stay: 滞在する}\\
(B)& Jane \textcolor{NavyBlue}{\bfseries has stayed} in London for six years.
\end{tabular}
\right.$

 \item $\left\{\begin{tabular}{rl}
(A)& I \textcolor{Maroon}{\bfseries watched} the movie two years ago.\\
(B)& I \textcolor{NavyBlue}{\bfseries have watched} the movie three times.
\end{tabular}
\right.$

 \item $\left\{\begin{tabular}{rl}
(A)&Bob \textcolor{Maroon}{\bfseries lost} his bag.\hspace{4\zw}{\small lose(失くす)の過去形、過去分詞はともにlost}\\
(B)&Bob \textcolor{NavyBlue}{\bfseries has lost} his bag.
\end{tabular}
\right.$

\end{enumerate}
\end{frame}

\begin{frame}[plain]{Exercises}
 
\visible<1->{\begin{enumerate}
 \item $\left\{\begin{tabular}{rl}
(A)& Jane \textcolor{Maroon}{\bfseries stayed} in London two years ago.\hspace{7\zw}{\small stay: 滞在する}\\
(B)& Jane \textcolor{NavyBlue}{\bfseries has stayed} in London for six years.
\end{tabular}
\right.$
\end{enumerate}}

\visible<2->{\signchart[width=10,height=.5]{,,,,,{\textcolor{Maroon}{stayed}},,今}{,,,}}
\visible<3->{\signchart[width=10,height=.5]{,6年前,,,{\textcolor{NavyBlue}{has stayed}},,,今}{,,,}}

\begin{tikzpicture}[overlay]
 %\draw[gray!50] (0,0) grid (12,5);
 %\fill[ForestGreen!70,opacity=.5] (11.57,3.75) circle [radius=.25];
\visible<2->{ \fill[Maroon!70,opacity=.5] (8.675,2.66) circle [radius=.25];}
\visible<3->{ \draw[NavyBlue!70,line width=6pt,opacity=.7] (4.2,1.21) -- (10.9,1.21);}
\end{tikzpicture}

\visible<4->{過去のある時から現在までの「継続」を表しています}
\end{frame}


\begin{frame}[plain]{Exercises}

\visible<1->{\begin{enumerate}\setcounter{enumi}{1}
 \item $\left\{\begin{tabular}{rl}
(A)& I \textcolor{Maroon}{\bfseries watched} the movie two years ago.\hspace{9\zw}{\small movie: 映画}\\
(B)& I \textcolor{NavyBlue}{\bfseries have watched} the movie three times.
\end{tabular}
\right.$
\end{enumerate}}

\visible<2->{\signchart[width=10,height=.5]{,,,,,{\textcolor{Maroon}{watched}},,今}{,,},}
\visible<3->{\signchart[width=10,height=.5]{1回目,,2回目,,{\textcolor{NavyBlue}{have watched}},,3回目,}{,,,}}

\begin{tikzpicture}[overlay]
%\draw[gray!50] (0,0) grid (12,5);
 %\fill[ForestGreen!70,opacity=.5] (11.57,3.75) circle [radius=.25];
\visible<2->{ \fill[Maroon!90,opacity=.75] (8.675,3.8) circle [radius=.25];}
\visible<3->{\fill[NavyBlue!90,opacity=.75] (3.1,1.21) circle [radius=.25];
\fill[NavyBlue!90,opacity=.75] (5.35,1.21) circle [radius=.25];
\fill[NavyBlue!90,opacity=.75] (9.78,1.21) circle [radius=.25];}
%\node[fill=NavyBlue!90,opacity=.75] at (10.88,1.21) { };
\visible<4->{\node[] at (10.89,1.21) {\LARGE \textcolor{NavyBlue}{x }};
\node[] at (10.9,.4) {\scriptsize \begin{tabular}{c}
				   これまでに\\
				 3回見た\\
				  ことがある\end{tabular}};}
% \draw[NavyBlue!70,line width=6pt,opacity=.7] (4.2,1.21) -- (10.9,1.21);
\end{tikzpicture}

\visible<5->{過去から現在までの「経験」を表しています}
\end{frame}

\begin{frame}[plain]{Exercises}
 
\visible<1->{\begin{enumerate}\setcounter{enumi}{2}
 \item $\left\{\begin{tabular}{rl}
(A)& Bob \textcolor{Maroon}{\bfseries lost} his bag.\hspace{7\zw}{\scriptsize lost: lose\,(失くす)の過去形}\\
(B)& Bob \textcolor{NavyBlue}{\bfseries has lost} his bag.\hspace{5\zw}{\scriptsize lost: lose\,(失くす)の過去分詞}
\end{tabular}
\right.$
\end{enumerate}}

\visible<2->{\signchart[width=10,height=.5]{,{\textcolor{Maroon}{lost}},,,,,,今}{,,,}}

\vspace{20pt}

\visible<5->{\signchart[width=10,height=.5]{,,,,{\textcolor{NavyBlue}{has lost}},,,}{,,,}}

\begin{tikzpicture}[overlay]
 %\draw[gray!50] (0,0) grid (12,5);
 %\fill[ForestGreen!70,opacity=.5] (11.57,3.75) circle [radius=.25];
\visible<3->{ \fill[Maroon!70,opacity=.9] (4.25,3.64) circle [radius=.25];}
\visible<5->{\fill[NavyBlue!70,opacity=.9] (4.25,1.2) circle [radius=.25];
\draw[NavyBlue!70,line width=6pt,opacity=.7] (4.2,1.21) -- (10.9,1.21);}

\visible<3->{\node[] at (4.25,2.9) {\scriptsize \begin{tabular}{c}
				   $\uparrow$\\
				   なくした\end{tabular}};}

\visible<4->{\node[] at (11.6,2.9) {\scriptsize $\left(\begin{tabular}{@{}l@{}}
				   今は\\
				 みつかったかも\end{tabular}\right)$};}

\visible<6->{\node[] at (4.25,.41) {\scriptsize \begin{tabular}{c}
				   $\uparrow$\\
				   なくした\end{tabular}};}

\visible<7->{\node[] at (10.9,.4) {\scriptsize \begin{tabular}{c}
				   $\uparrow$\\
				   今も\\
				 ない\end{tabular}};}
\end{tikzpicture}

\visible<8->{過去に起こした動作の現在での「結果」を表しています}
\end{frame}

\begin{frame}[plain]{まとめ}

\begin{enumerate}
 \item Jane has stayed in London for six years.
 \item I have watched the movie three times.
 \item Bob has lost his bag.
\end{enumerate}



 \begin{exampleblock}{Topic for Today}
\small
\begin{itemize}
 \item  $\text{現在完了(}=\textcolor{NavyBlue}{\text{have} + \text{過去分詞\,)}}$%
は「過去と現在にまたがる表現」です
\end{itemize}
      \end{exampleblock}
\end{frame}

\end{document}
