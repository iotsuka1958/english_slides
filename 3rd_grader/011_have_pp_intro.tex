\documentclass[aspectratio=169,xcolor={dvipsnames,table}]{beamer}
\usepackage[no-math,deluxe,haranoaji]{luatexja-preset}
\renewcommand{\kanjifamilydefault}{\gtdefault}
\renewcommand{\emph}[1]{{\upshape\bfseries #1}}
\usetheme{metropolis}
\metroset{block=fill}
\setbeamertemplate{navigation symbols}{}
\setbeamertemplate{blocks}[rounded][shadow=false]
\usecolortheme[rgb={0.7,0.2,0.2}]{structure}
%%%%%%%%%%%%%%%%%%%%%%%%%%%
\usepackage{media9}
\usepackage[absolute,overlay]{textpos}
%\usepackage[grid=true,gridcolor=Maroon,subgridcolor=gray,gridunit=pt,texcoord]{eso-pic} %場所決めのためのgrid表示
%%%%%%%%%%%%%%%%%%%%%%%%%%%
%% さまざまなアイコン
%%%%%%%%%%%%%%%%%%%%%%%%%%%
\usepackage{fontawesome}
%\usepackage{figchild}
\usepackage{twemojis}
\usepackage{utfsym}
\usepackage{bclogo}
\usepackage{marvosym}
\usepackage{fontmfizz}
\usepackage{pifont}
\usepackage{phaistos}
\usepackage{worldflags}
\usepackage{tipa}
%%%%%%%%%%%%%%%%%%%%%%%%%%%
\usepackage{tikz}
\usetikzlibrary{backgrounds,positioning}
\usepackage{tcolorbox}
\usepackage{tikzpeople}
\usepackage{tikzducks}
\usepackage{tikzlings}
\usepackage{circledsteps}
\usepackage{xcolor}
\usepackage{amsmath}
\usepackage{booktabs}
\usepackage{chronology}
\usepackage{signchart}
\usepackage{pxrubrica}
%%%%%%%%%%%%%%%%%%%%%%%%%%%
%% 場合分け
\usepackage{cases}
%%%%%%%%%%%%%%%%%%%%%%%%%%%
% \myAnch{<名前>}{<色>}{<テキスト>}
% 指定のテキストを指定の色の四角枠で囲み, 指定の名前をもつTikZの
% ノードとして出力する. 図には remeber picture 属性を付けている
% ので外部から参照可能である.
\newcommand*{\myAnch}[3]{%
  \tikz[remember picture,baseline=(#1.base)]
    \node[draw,rectangle,#2] (#1) {\normalcolor #3};
}
%%%%%%%%%%%%%%%%%%%%%%%%%%%%
%% 音声リンク表示
\newcommand{\myaudio}[1]{\href{#1}{\faVolumeUp}}
%%%%%%%%%%%%%%%%%%%%%%%%%%%
% \myEmph コマンドの定義
%\newcommand{\myEmph}[3]{%
%    \textbf<#1>{\color<#1>{#2}{#3}}%
%}
\usepackage{xparse} % xparseパッケージの読み込み
\NewDocumentCommand{\myEmph}{O{} m m}{%
    \def\argOne{#1}%
    \ifx\argOne\empty
        \textbf{\color{#2}{#3}}% オプション引数が省略された場合
    \else
        \textbf<#1>{\color<#1>{#2}{#3}}% オプション引数が指定された場合
    \fi
}
%%%%%%%%%%%%%%%%%%%%%%%%%%%
%% 文末の上昇イントネーション記号 \myRisingPitch
%% 通常のイントネーション \myDownwardPitch
%% https://note.com/dan_oyama/n/n8be58e8797b2
%%%%%%%%%%%%%%%%%%%%%%%%%%%
\newcommand{\myRisingPitch}{
\begin{tikzpicture}[scale=0.3,baseline=0.3]
\draw[->,>=stealth] (0,0) to[bend right=45] (1,1);
\end{tikzpicture}
}
\newcommand{\myDownwardPitch}{
\begin{tikzpicture}[scale=0.3,baseline=0.3]
\draw[->,>=stealth] (0,1) to[bend left=45] (1,0);
\end{tikzpicture}
}
%%%%%%%%%%%%%%%%%%%%%%%%%%%
\title{English is fun.}
\subtitle{冷やし中華はじめました}
  \author{}
\institute[]{}
\date[]

%%%%%%%%%%%%%%%%%%%%%%%%%%%%
%% TEXT
%%%%%%%%%%%%%%%%%%%%%%%%%%%%
\begin{document}
%%%%%%%%%%%%%%%%%%%%%%%%%%%%
\begin{frame}[plain]
  \titlepage
\end{frame}
%%%%%%%%%%%%%%%%%%%%%%%%%%%%
%%%%%%%%%%%%%%%%%%%%%%%%%%%%%%%%%%%%%%%%%%
{
  \usebackgroundtemplate{\includegraphics[width=.7\paperwidth]{./images/hiyashi_chuka.jpg}}
  \begin{frame}[b,plain]
    \frametitle{}
\tiny
\raggedright
  \textcolor{white}{ ``冷やし中華'' by Roulex45 is licensed under CC BY 2.0. }\\
   \textcolor{white}{To view a copy of this license,}\\
\textcolor{white}{visit \url{https://creativecommons.org/licenses/by/2.0/?ref=openverse}}.
\begin{textblock*}{0.4\linewidth}(335pt,180pt)
\visible<1->{\begin{tikzpicture}
\duck[signpost=\scalebox{0.3}{
\parbox{2.5cm}{\color{black}\centering
{\Large 酢醤油か\\[5pt]ごまだれか}}},
signcolour=brown!70!gray,
signback=white!80!brown,
graduate=gray!20!black,
tassel=red!70!black,
laughing,
speech={\tiny それが問題}
]
\end{tikzpicture}}
\end{textblock*}
  \end{frame}
}
%%%%%%%%%%%%%%%%%%%%%%%%%%%%%%%%%%%%%%%%%%%
\section*{授業の流れ}
\begin{frame}[plain]
  \frametitle{授業の流れ}
  \tableofcontents
\end{frame}
%%%%%%%%%%%%%%%%%%%%%%%%%%%
\section{現在完了形}
\subsection{現在完了形とは}
%%%%%%%%%%%%%%%%%%%%%%%%%%%%%%%%%%%%
\begin{frame}[plain]{現在完了形とは}
 
\begin{enumerate}
 \item \visible<1->{I \textcolor{ForestGreen}{\bfseries live} in London now. {\small わたしはいまロンドンに住んでいます。(現在形)}}
\visible<2->{\signchart[width=10]{,,,,,,,,,,{\textcolor{ForestGreen}{\textbf{live}}}}{,,,}}
 \item \visible<3->{I \textcolor{Maroon}{\bfseries lived} in London five years ago.{\small わたしは5年前ロンドンに住んでいた。(過去形)}}
\visible<4->{\signchart[width=10,height=.5]{,,,,,{\textcolor{Maroon}{\textbf{lived}}},,,,,}{,,,}}

 \item \visible<5->{I \textcolor{NavyBlue}{\bfseries have lived} in London for eight years.{\small (現在完了形)}}

\visible<6->{\signchart[width=10,height=.5]{,,8年前,,,,{\textcolor{NavyBlue}{\textbf{have lived}}},,,,今}{,,,}}
\end{enumerate}

\vspace*{-20pt}

\begin{tikzpicture}[overlay]
 %\draw[gray!50] (0,0) grid (12,5);
 \visible<2->{\fill[ForestGreen!70,opacity=.5] (11.57,3.75) circle [radius=.25];}
\visible<4-> {\fill[Maroon!70,opacity=.5] (7.4,2.05) circle [radius=.25];}
\visible<6->{ \draw[NavyBlue!70,line width=6pt,opacity=.7] (4.88,0.3) -- (11.55,0.3);}
\end{tikzpicture}

\begin{block}<7->{Topics for Today}
\small
\begin{itemize}\setbeamertemplate{items}[square]
 \item  現在完了形\,\Circled[fill color=white]{\,\textbf{have} $+$ 過去分詞\,}\,は「過去と現在にまたがる表現」です\\
% \item  \visible<8->{$\text{現在完了形(}=\textcolor{NavyBlue}{\text{\bfseries have} + \text{過去分詞}}$\,\,)は「過去と現在にまたがる表現」です}\\
\hfill\visible<9->{{\scriptsize 主語が三人称単数のときは$\textcolor{NavyBlue}{\text{\bfseries has} + \text{過去分詞}}$}  She \textcolor{NavyBlue}{\bfseries has lived} in London for eight years.}
 \item \visible<10->{過去分詞は\kenten{受け身}とならんで\kenten{完了}の意味をあらわすことがあります}%
\hfill{\tiny 0142}\,{\scriptsize \myaudio{./audio/011_have_pp_intro_01.mp3}}
\end{itemize}
      \end{block}

\begin{textblock*}{0.4\linewidth}(385pt,110pt)
\visible<8->{\begin{tikzpicture}
\duck[signpost=\scalebox{0.3}{
\parbox{2.5cm}{\color{black}\centering
{\Large 現在完了$=$\\過去$+$現在}}},
signcolour=brown!70!gray,
signback=white!80!brown,
graduate=gray!20!black,
tassel=red!70!black,
laughing,
%speech={\tiny 反復練習!}
]
\end{tikzpicture}}
\end{textblock*}
\end{frame}
%%%%%%%%%%%%%%%%%%%%%%%%%%%%%%%%%%%%
\begin{frame}[plain,t]{Exercises}
 次の各組の2文の意味の違いについて考えましょう%
\hfill{\tiny 0236}\,{\scriptsize \myaudio{./audio/011_have_pp_intro_02.mp3}}

\begin{enumerate}
 \item $\left\{\begin{tabular}{rl}
(A)& Jane \textcolor{Maroon}{\bfseries stayed} in London two days ago.\hspace{9\zw}{\scriptsize stay: 滞在する}\\
(B)& Jane \textcolor{NavyBlue}{\bfseries has stayed} in London for six days.
\end{tabular}
\right.$

 \item $\left\{\begin{tabular}{rl}
(A)& I \textcolor{Maroon}{\bfseries watched} the movie two years ago.\\
(B)& I \textcolor{NavyBlue}{\bfseries have watched} the movie three times.
\end{tabular}
\right.$

 \item $\left\{\begin{tabular}{rl}
(A)&Bob \textcolor{Maroon}{\bfseries lost} his bag.\hspace{8\zw}{\scriptsize lose(失くす)の過去形、過去分詞はともにlost}\\
(B)&Bob \textcolor{NavyBlue}{\bfseries has lost} his bag.\hspace{6.1\zw}{\scriptsize lose \textipa{/l\'u:z/} --- lost \textipa{/l\'O:st/} --- lost \textipa{/l\'O:st/}}
\end{tabular}
\right.$
\end{enumerate}

\begin{textblock*}{0.4\linewidth}(310pt,157pt)
\visible<1->{\begin{tikzpicture}
\pig[
speech={\bfseries\tiny have $+$ p.p.},
signpost=\scalebox{0.5}{
\parbox{2.2cm}{\color{black}
\centering 現在との\\つながりに\\注意!}},
signcolour= brown!70!gray,
signback=white!80!brown
]
\end{tikzpicture}}
\end{textblock*}

\end{frame}
%%%%%%%%%%%%%%%%%%%%%%%%%%%%%%%%%%%%%%%%%%%%
\begin{frame}[plain]{Exercises}
 
\visible<1->{\begin{enumerate}
 \item $\left\{\begin{tabular}{rl}
(A)& Jane \textcolor{Maroon}{\bfseries stayed} in London two days ago.%
	\hspace{100pt}{\scriptsize ago \textipa{/@g\'oU/} 前に}\\
(B)& Jane \textcolor{NavyBlue}{\bfseries has stayed} in London for six days.%
	\hspace{89pt}{\scriptsize for \textipa{/f\textrhookschwa /} ~の間}\\
\end{tabular}
\right.$
\end{enumerate}}

\visible<2->{\signchart[width=10,height=.5]{,,,,,{\textcolor{Maroon}{\textbf{stayed}}},,今}{,,,}}
\visible<3->{\signchart[width=10,height=.5]{,6日前,,,{\textcolor{NavyBlue}{\textbf{has stayed}}},,,今}{,,,}}

\begin{tikzpicture}[overlay]
 %\draw[gray!50] (0,0) grid (12,5);
 %\fill[ForestGreen!70,opacity=.5] (11.57,3.75) circle [radius=.25];
\visible<2->{ \fill[Maroon!70,opacity=.5] (8.675,2.66) circle [radius=.25];}
\visible<3->{ \draw[NavyBlue!70,line width=6pt,opacity=.7] (4.2,1.21) -- (10.9,1.21);}
\end{tikzpicture}

\visible<4->{過去のある時から現在までの「継続」を表しています}%
\hfill{\tiny 0111}\<{\scriptsize \myaudio{./audio/011_have_pp_intro_03.mp3}}
\end{frame}
%%%%%%%%%%%%%%%%%%%%%%%%%%%%%%%%%%%%%

\begin{frame}[plain]{Exercises}

\visible<1->{\begin{enumerate}\setcounter{enumi}{1}
 \item $\left\{\begin{tabular}{rl}
		(A)& I \textcolor{Maroon}{\bfseries watched} the movie two years ago.\hspace{90pt}{\scriptsize movie \textipa{/m\'u:vi/} 映画}\\
(B)& I \textcolor{NavyBlue}{\bfseries have watched} the movie three times.%
	\hspace{75pt}{\scriptsize three times 3回}\\

\end{tabular}
\right.$
\end{enumerate}}

\visible<2->{\signchart[width=10,height=.5]{,,,,,{\textcolor{Maroon}{\textbf{watched}}},,今}{,,},}
\visible<3->{\signchart[width=10,height=.5]{1回目,,2回目,,{\textcolor{NavyBlue}{\textbf{have watched}}},,3回目,}{,,,}}

\begin{tikzpicture}[overlay]
%\draw[gray!50] (0,0) grid (12,5);
 %\fill[ForestGreen!70,opacity=.5] (11.57,3.75) circle [radius=.25];
\visible<2->{ \fill[Maroon!90,opacity=.75] (8.675,3.8) circle [radius=.25];}
\visible<3->{\fill[NavyBlue!90,opacity=.75] (3.1,1.21) circle [radius=.25];
\fill[NavyBlue!90,opacity=.75] (5.35,1.21) circle [radius=.25];
\fill[NavyBlue!90,opacity=.75] (9.78,1.21) circle [radius=.25];}
%\node[fill=NavyBlue!90,opacity=.75] at (10.88,1.21) { };
\visible<4->{\node[] at (10.89,1.21) {\LARGE \textcolor{NavyBlue}{x }};
\node[] at (10.9,.4) {\scriptsize \begin{tabular}{c}
				   これまでに\\
				 3回見た\\
				  ことがある\end{tabular}};}
% \draw[NavyBlue!70,line width=6pt,opacity=.7] (4.2,1.21) -- (10.9,1.21);
\end{tikzpicture}

\visible<5->{過去から現在までの「経験」を表しています}%
\hfill{\tiny 0117}\,{\scriptsize \myaudio{./audio/011_have_pp_intro_04.mp3}}
\end{frame}
%%%%%%%%%%%%%%%%%%%%%%%%%%%%%%%%%%%%
\begin{frame}[plain]{Exercises}
 
\visible<1->{\begin{enumerate}\setcounter{enumi}{2}
 \item $\left\{\begin{tabular}{rl}
(A)& Bob \textcolor{Maroon}{\bfseries lost} his bag.\hspace{7\zw}{\scriptsize lost \textipa{/l\'O:st/} $\longleftarrow$ lose \textipa{/l\'u:z/}\,(失くす)の過去形}\\
(B)& Bob \textcolor{NavyBlue}{\bfseries has lost} his bag.\hspace{5.1\zw}{\scriptsize lost \textipa{/l\'O:st/} $\longleftarrow$ lose \textipa{/l\'u:z/}\,(失くす)の過去分詞}
\end{tabular}
\right.$
\end{enumerate}}

\visible<2->{\signchart[width=10,height=.5]{,{\textcolor{Maroon}{\textbf{lost}}},,,,,,今}{,,,}}

\vspace{20pt}

\visible<5->{\signchart[width=10,height=.5]{,,,,{\textcolor{NavyBlue}{\textbf{has lost}}},,,}{,,,}}

\begin{tikzpicture}[overlay]
 %\draw[gray!50] (0,0) grid (12,5);
 %\fill[ForestGreen!70,opacity=.5] (11.57,3.75) circle [radius=.25];
\visible<3->{ \fill[Maroon!70,opacity=.9] (4.25,3.64) circle [radius=.25];}
\visible<5->{\fill[NavyBlue!70,opacity=.9] (4.25,1.2) circle [radius=.25];
\draw[NavyBlue!70,line width=6pt,opacity=.7] (4.2,1.21) -- (10.9,1.21);}

\visible<3->{\node[] at (4.25,2.9) {\scriptsize \begin{tabular}{c}
				   $\uparrow$\\
				   なくした\end{tabular}};}

\visible<4->{\node[] at (11.6,2.9) {\scriptsize $\left(\begin{tabular}{@{}l@{}}
				   今は\\
				 みつかったかも\end{tabular}\right)$};}

\visible<6->{\node[] at (4.25,.41) {\scriptsize \begin{tabular}{c}
				   $\uparrow$\\
				   なくした\end{tabular}};}

\visible<7->{\node[] at (10.9,.4) {\scriptsize \begin{tabular}{c}
				   $\uparrow$\\
				   今も\\
				 ない\end{tabular}};}
\end{tikzpicture}

\visible<8->{過去に起こした動作の現在での「結果」を表しています}\mbox{}\hfill{\tiny 0113}\,{\scriptsize \myaudio{./audio/011_have_pp_intro_05.mp3}}
\end{frame}
%%%%%%%%%%%%%%%%%%%%%%%%%%%%%%%%%%%%%%%%
\section{まとめ}
\begin{frame}[plain]{まとめ}


 \begin{block}{現在完了形の基本}
\small
\begin{itemize}\setbeamertemplate{items}[square]
 \item  現在完了形\,\Circled[fill color=white]{\,\textbf{have} $+$ 過去分詞\,}\,は「過去と現在にまたがる表現」です
% \item 過去分詞は\kenten{受け身}とならんで\kenten{完了}の意味をあらわします
\end{itemize}
      \end{block}

\begin{enumerate}
 \item Jane \textcolor{NavyBlue}{\bfseries has stayed} in London for six days.
 \item I \textcolor{NavyBlue}{\bfseries have watched} the movie three times.
 \item Bob \textcolor{NavyBlue}{\bfseries has lost} his bag.
\end{enumerate}

\mbox{}\hfill{\tiny 0111}\,{\scriptsize \myaudio{./audio/011_have_pp_intro_06.mp3}}
\end{frame}
%%%%%%%%%%%%%%%%%%%%%%%%
\begin{frame}[plain]{過去分詞の2つの使い方}
\large 

\begin{enumerate}
 \item English {\bfseries is spoken} there.\hfill{}受け身\\
\hfill{}{\scriptsize speak \textipa{/sp\'\i :k/} --- spoke \textipa{/sp\'oUk/} --- spoken \textipa{/sp\'oUk(@)n/}}
 \item All the leaves {\bfseries have fallen}.\hfill{}完了\\
\hfill{}{\scriptsize fall \textipa{/f\'O:l/} --- fell \textipa{/f\'el/} --- fallen \textipa{/f\'O:l@n/}}
\end{enumerate}
\vspace{20pt}
be動詞と組み合わせると\kenten{受け身}、haveと組み合わせると\kenten{現在完了}になります

\vspace{40pt}

\mbox{}\hfill{\tiny 0111}\,{\scriptsize \myaudio{./audio/011_have_pp_intro_07.mp3}}

\end{frame}
%%%%%%%%%%%%%%%%%%%%%%%%%
\section*{ところで}
\begin{frame}[plain,t]{ところで}
 \Large

\vspace*{20pt}

町の中華料理屋の店先に\\
\tikz[baseline=(hiyashi.base)]{\node[draw, fill=yellow!50] (hiyashi) {冷やし中華はじめました};}\,\,というはりがみが掲示されています

\vspace{50pt}

\normalsize
\begin{enumerate}
 \item<2-> いつから冷やし中華の提供がされたのでしょうか
 \item<3-> きょうも冷やし中華は提供されているでしょうか
\end{enumerate}

\begin{textblock*}{0.4\linewidth}(310pt,150pt)
\visible<4->{\begin{tikzpicture}
\pig[
speech={\scriptsize 現在完了},
signpost=\scalebox{0.5}{
\parbox{2.2cm}{\color{black}
\centering 冷やし中華\\はじめました!}},
signcolour= brown!70!gray,
signback=white!80!brown
]
\end{tikzpicture}}
\end{textblock*}


\end{frame}
%%%%%%%%%%%%%%%%%%%%%%%%%%
%%%%%%%%%%%%%%%%%%%%%%%%%%%%%%%%%%%%%%%%%%
{
  \usebackgroundtemplate{\includegraphics[width=.7\paperwidth]{./images/hiyashi_chuka.jpg}}
  \begin{frame}[b,plain]
    \frametitle{}
\tiny
\raggedright
  \textcolor{white}{ ``冷やし中華'' by Roulex45 is licensed under CC BY 2.0. }\\
   \textcolor{white}{To view a copy of this license,}\\
\textcolor{white}{visit \url{https://creativecommons.org/licenses/by/2.0/?ref=openverse}}.
\begin{textblock*}{0.4\linewidth}(335pt,180pt)
\visible<1->{\begin{tikzpicture}
\duck[signpost=\scalebox{0.3}{
\parbox{2.5cm}{\color{black}\centering
{\Large 酢醤油か\\[5pt]ごまだれか}}},
signcolour=brown!70!gray,
signback=white!80!brown,
graduate=gray!20!black,
tassel=red!70!black,
laughing,
speech={\tiny それが問題}
]
\end{tikzpicture}}
\end{textblock*}
  \end{frame}
}
%%%%%%%%%%%%%%%%%%%%%%%%%%%%%%%%%%%%%%%%%%%
%%%%%%%%%%%%%%%%%%%
\begin{frame}[plain]

 {\tiny audio\_overview 1653}\,{\scriptsize \myaudio{./audio/overview/011_have_pp_intro_audio_overview.m4a}}
\end{frame}
%%%%%%%%%%%%%%%%%%
\end{document}
