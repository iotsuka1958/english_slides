\documentclass[aspectratio=169,xcolor={dvipsnames,table}]{beamer}
\usepackage[no-math,deluxe,haranoaji]{luatexja-preset}
\renewcommand{\kanjifamilydefault}{\gtdefault}
\renewcommand{\emph}[1]{{\upshape\bfseries #1}}
\usetheme{metropolis}
\metroset{block=fill}
\setbeamertemplate{navigation symbols}{}
\setbeamertemplate{blocks}[rounded][shadow=false]
\usecolortheme[rgb={0.7,0.2,0.2}]{structure}
%%%%%%%%%%%%%%%%%%%%%%%%%%
%% Change alert block colors
%%% 1- Block title (background and text)
\setbeamercolor{block title alerted}{fg=mDarkTeal, bg=mLightBrown!45!yellow!45}
\setbeamercolor{block title example}{fg=magenta!10!black, bg=mLightGreen!70}
%%% 2- Block body (background)
\setbeamercolor{block body alerted}{bg=mLightBrown!25}
\setbeamercolor{block body example}{bg=mLightGreen!15}
%%%%%%%%%%%%%%%%%%%%%%%%%%%
%%%%%%%%%%%%%%%%%%%%%%%%%%%
%% さまざまなアイコン
%%%%%%%%%%%%%%%%%%%%%%%%%%%
%\usepackage{fontawesome}
\usepackage{fontawesome5}
\usepackage{figchild}
\usepackage{twemojis}
\usepackage{utfsym}
\usepackage{bclogo}
\usepackage{marvosym}
\usepackage{fontmfizz}
\usepackage{pifont}
\usepackage{phaistos}
\usepackage{worldflags}
\usepackage{jigsaw}
\usepackage{tikzlings}
\usepackage{tikzducks}
\usepackage{scsnowman}
\usepackage{epsdice}
\usepackage{halloweenmath}
\usepackage{svrsymbols}
\usepackage{countriesofeurope}
\usepackage{tipa}
\usepackage{manfnt}
%%%%%%%%%%%%%%%%%%%%%%%%%%%
\usepackage{tikz}
\usetikzlibrary{calc,patterns,decorations.pathmorphing,backgrounds}
\usepackage{tcolorbox}
\usepackage{tikzpeople}
\usepackage{circledsteps}
\usepackage{xcolor}
\usepackage{amsmath}
\usepackage{booktabs}
\usepackage{chronology}
\usepackage{signchart}
%%%%%%%%%%%%%%%%%%%%%%%%%%%
%% 場合分け
%%%%%%%%%%%%%%%%%%%%%%%%%%%
\usepackage{cases}
%%%%%%%%%%%%%%%%%%%%%%%%%%
\usepackage{pdfpages}
%%%%%%%%%%%%%%%%%%%%%%%%%%%
%% 音声リンク表示
\newcommand{\myaudio}[1]{\href{#1}{\faVolumeUp}}
%%%%%%%%%%%%%%%%%%%%%%%%%%
%% \myAnch{<名前>}{<色>}{<テキスト>}
%% 指定のテキストを指定の色の四角枠で囲み, 指定の名前をもつTikZの
%% ノードとして出力する. 図には remember picture 属性を付けている
%% ので外部から参照可能である.
\newcommand*{\myAnch}[3]{%
  \tikz[remember picture,baseline=(#1.base)]
    \node[draw,rectangle,line width=1pt,#2] (#1) {\normalcolor #3};
}
%%%%%%%%%%%%%%%%%%%%%%%%%%
%% \myEmph コマンドの定義
%%%%%%%%%%%%%%%%%%%%%%%%%%
%\newcommand{\myEmph}[3]{%
%    \textbf<#1>{\color<#1>{#2}{#3}}%
%}
\usepackage{xparse} % xparseパッケージの読み込み
\NewDocumentCommand{\myEmph}{O{} m m}{%
    \def\argOne{#1}%
    \ifx\argOne\empty
        \textbf{\color{#2}{#3}}% オプション引数が省略された場合
    \else
        \textbf<#1>{\color<#1>{#2}{#3}}% オプション引数が指定された場合
    \fi
}
%%%%%%%%%%%%%%%%%%%%%%%%%%%
%%%%%%%%%%%%%%%%%%%%%%%%%%%
%% 文末の上昇イントネーション記号 \myRisingPitch
%% 通常のイントネーション \myDownwardPitch
%% https://note.com/dan_oyama/n/n8be58e8797b2
%%%%%%%%%%%%%%%%%%%%%%%%%%%
\newcommand{\myRisingPitch}{
\begin{tikzpicture}[scale=0.3,baseline=0.3]
\draw[->,>=stealth] (0,0) to[bend right=45] (1,1);
\end{tikzpicture}
}
\newcommand{\myDownwardPitch}{
\begin{tikzpicture}[scale=0.3,baseline=0.3]
\draw[->,>=stealth] (0,1) to[bend left=45] (1,0);
\end{tikzpicture}
}
%%%%%%%%%%%%%%%%%%%%%%%%%%%%
%\AtBeginSection[%
%]{%
%  \begin{frame}[plain]\frametitle{授業の流れ}
%     \tableofcontents[currentsection]
%   \end{frame}%
%}

%%%%%%%%%%%%%%%%%%%%%%%%%%%
\title{English is fun.}
\subtitle{}
\author{}
\institute[]{}
\date[]

%%%%%%%%%%%%%%%%%%%%%%%%%%%%
%% TEXT
%%%%%%%%%%%%%%%%%%%%%%%%%%%%
\begin{document}

\begin{frame}[plain]
  \titlepage
\end{frame}

\section*{授業の流れ}
\begin{frame}[plain]
  \frametitle{授業の流れ}
  \tableofcontents
\end{frame}

\section{}
\subsection{}
%%%%%%%%%%%%%%%%%%%%%%%%%%%%%%%%%%%%%%%%%%%%%
 \begin{frame}[plain]{形容詞 名詞}
 \begin{enumerate}
  \item a big dog / big dogs
  \item a cute cat / cute cats
  \item a small box / small boxes
  \item an expensive car / expensive cars\hfill{\scriptsize expensive: 高価な}
  \item an interesting book / interesting books\hfill{\scriptsize interesting: おもしろい}
  \item an important question / important questions\hfill{\scriptsize important: 重要な}
 \end{enumerate}

\visible<2->{%
\begin{exampleblock}{Topic for Today}
名詞を修飾することばを「形容詞」といいます
\begin{itemize}\small
 \item 形容詞は名詞の前に置きます
 \item \fbox{\,$\text{(冠詞)}+\text{形容詞}+\text{名詞}$\,\,}\,\,\,は意味のかたまりになります
 \end{itemize}
     \end{exampleblock}
}
 \end{frame}
%%%%%%%%%%%%%%%%%%%%%%
\begin{frame}[plain]{冠詞 ---ing 名詞}
 \begin{enumerate}
	 \item I looked at \fbox{the cat}.\hfill{\scriptsize {\fbox{$\text{冠詞}+\text{名詞}$}}}
	 \item I looked at \fbox{the cute cat}.\hfill{\scriptsize {\fbox{$\text{冠詞}+\text{形容詞}+\text{名詞}$}}}
	 \item I looked at \fbox{the sleeping cat}.\hfill{\scriptsize {\fbox{$\text{冠詞}+\text{\,\,---ing\,\,}+\text{名詞}$}}}
 \end{enumerate}

\visible<2->{%
\begin{exampleblock}{Topic for Today}
\Circled[fill color = white]{\,\,---ing\,\,}\,\,が名詞を修飾することがあります%
\hfill{\scriptsize つまり「形容詞」の役目をしています}

\begin{itemize}\small
 \item この\,\,\Circled[fill color = white]{\,\,---ing\,\,}\,\,は「~している」という意味
 \item この\,\,\Circled[fill color = white]{\,\,---ing\,\,}\,\,を「現在分詞」といいます
 \end{itemize}
     \end{exampleblock}
}
\end{frame}
%%%%%%%%%%%%%%%%%%%%%%
\begin{frame}[plain]{Exercises}
日本語の意味になるよう(~~~~~~)の語を並べ替えましょう。先頭に来る語は大文字で始めてください
 \begin{enumerate}
  \item あなたは歌っている女性を知っていますか。\\
	( you / singing / the / woman / do / know ) ?\\
	\visible<2->{Do you know the singing woman?}
  \item 沸騰しているお湯は熱い。
	( hot / is / water / boiling ) .\hfill{\scriptsize boil: 沸騰する}\\
	\visible<3->{Boiling water is hot.}
  \item あの眠っているネコはかわいい。
	That ( is / cat / sleeping ) cute.\\
	\visible<4->{That sleeping cat is cute.}
  \item 昇っている太陽が美しかった。
	( sun / was / rising  / the ) beautiful.\hfill{\scriptsize rise: 昇る}\\
	\visible<5->{The rising sun was beautiful.}
  \item 飛んでいる鳥を見てください。
	( at / birds / the / look / flying ) .\\
	\visible<6->{Look at the flying birds.}
 \end{enumerate}
\end{frame}
%%%%%%%%%%%%%%%%%%%%%%%%%%%%
\begin{frame}[plain]{名詞 ---ing \ldots}
 \begin{enumerate} 
  \item<1-> The girl is Jane.\hfill{}{\scriptsize The girl($=\text{S}$)\,\,\,\,is($=\text{V}$)}
  \item<3-> The girl in the yard is Jane.%
	\hfill{}\visible<4->{\scriptsize The girl $\longleftarrow$\,\fbox{\,in the yard\,}}
  \item<5-> The girl playing in the yard is Jane.%
	\hfill{}\visible<6->{\scriptsize The girl $\longleftarrow$\,\fbox{\,playing in the yard\,}}
 \end{enumerate}

\visible<7->{%
\begin{exampleblock}{Topic for Today}
\Circled[fill color = white]{\,\,---ing\,\,}\,\,がほかの語句を伴うときは、
後ろから前の名詞を修飾します%

\begin{itemize}\small
 \item \Circled[fill color = white]{\,\,---ing\,\,}\,\, $\longrightarrow \text{\,\,名詞}$\hfill{\scriptsize the sleeping cat}
 \item $\text{名詞\,\,}\longleftarrow$\,\,\Circled[fill color = white]{\,\,---ing\,\,+\text{\,\,ほかの語句\,\,}}\hfill{\scriptsize the cat sleeping in the sun}
 \end{itemize}
     \end{exampleblock}
}
\end{frame}
%%%%%%%%%%%%%%%%%%%%%%%%%%%%
\begin{frame}[plain]{Exercises}
日本語の意味になるよう(~~~~~~)の語を並べ替えましょう
 \begin{enumerate}
  \item 英語を話している少年たちを見なさい。\\
	Look ( English / the boys / speaking / at ) .\\
	\visible<2->{Look at the boys speaking English.}
  \item 海で泳いでいる男性はわたしの父です。\\
	The ( swimming / my / is / man / in / the sea ) father.\\
	\visible<3->{The man swimming in the sea is my father.}
  \item その部屋で読書している少年はスペンサーです。\\
	The ( the room / a book / boy / is / in / reading ) Spencer.\\
	\visible<4->{The boy reading a book in the room is Spencer.}
  \item 犬と走っている少女を知っていますか。\\
	Do ( the girl / a dog / know / with / running  / you ) ?\\
	\visible<5->{Do you know the girl running with a dog?}
 \end{enumerate}
\end{frame}
\end{document}
