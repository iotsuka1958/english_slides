\documentclass[aspectratio=169,xcolor={dvipsnames,table}]{beamer}
\usepackage[no-math,deluxe,haranoaji]{luatexja-preset}
\renewcommand{\kanjifamilydefault}{\gtdefault}
\renewcommand{\emph}[1]{{\upshape\bfseries #1}}
\usetheme{metropolis}
\metroset{block=fill}
\setbeamertemplate{navigation symbols}{}
\setbeamertemplate{blocks}[rounded][shadow=false]
\usecolortheme[rgb={0.7,0.2,0.2}]{structure}
%%%%%%%%%%%%%%%%%%%%%%%%%%
%% Change alert block colors
%%% 1- Block title (background and text)
\setbeamercolor{block title alerted}{fg=mDarkTeal, bg=mLightBrown!45!yellow!45}
\setbeamercolor{block title example}{fg=magenta!10!black, bg=mLightGreen!70}
%%% 2- Block body (background)
\setbeamercolor{block body alerted}{bg=mLightBrown!25}
\setbeamercolor{block body example}{bg=mLightGreen!15}
%%%%%%%%%%%%%%%%%%%%%%%%%%%
\usepackage[absolute,overlay]{textpos}
%\usepackage[grid=true,gridcolor=Maroon,subgridcolor=gray,gridunit=pt,texcoord]{eso-pic} %場所決めのためのgrid表示
%%%%%%%%%%%%%%%%%%%%%%%%%%%
%% さまざまなアイコン
%%%%%%%%%%%%%%%%%%%%%%%%%%%
%\usepackage{fontawesome}
\usepackage{fontawesome5}
\usepackage{figchild}
\usepackage{twemojis}
\usepackage{utfsym}
\usepackage{bclogo}
\usepackage{marvosym}
\usepackage{fontmfizz}
\usepackage{pifont}
\usepackage{phaistos}
\usepackage{worldflags}
\usepackage{jigsaw}
\usepackage{tikzlings}
\usepackage{tikzducks}
\usepackage{scsnowman}
\usepackage{epsdice}
\usepackage{halloweenmath}
\usepackage{svrsymbols}
\usepackage{countriesofeurope}
\usepackage{tipa}
\usepackage{manfnt}
%%%%%%%%%%%%%%%%%%%%%%%%%%%
\usepackage{tikz}
\usetikzlibrary{calc,patterns,decorations.pathmorphing,backgrounds}
\usepackage{tcolorbox}
\usepackage{tikzpeople}
\usepackage{circledsteps}
\usepackage{xcolor}
\usepackage{amsmath}
\usepackage{booktabs}
\usepackage{chronology}
\usepackage{signchart}
%%%%%%%%%%%%%%%%%%%%%%%%%%%
%% 場合分け
%%%%%%%%%%%%%%%%%%%%%%%%%%%
\usepackage{cases}
%%%%%%%%%%%%%%%%%%%%%%%%%%
\usepackage{pdfpages}
%%%%%%%%%%%%%%%%%%%%%%%%%%%
%% 音声リンク表示
\newcommand{\myaudio}[1]{\href{#1}{\faVolumeUp}}
%%%%%%%%%%%%%%%%%%%%%%%%%%
%% \myAnch{<名前>}{<色>}{<テキスト>}
%% 指定のテキストを指定の色の四角枠で囲み, 指定の名前をもつTikZの
%% ノードとして出力する. 図には remember picture 属性を付けている
%% ので外部から参照可能である.
\newcommand*{\myAnch}[3]{%
  \tikz[remember picture,baseline=(#1.base)]
    \node[draw,rectangle,line width=1pt,#2] (#1) {\normalcolor #3};
}
%%%%%%%%%%%%%%%%%%%%%%%%%%
%% \myEmph コマンドの定義
%%%%%%%%%%%%%%%%%%%%%%%%%%
%\newcommand{\myEmph}[3]{%
%    \textbf<#1>{\color<#1>{#2}{#3}}%
%}
\usepackage{xparse} % xparseパッケージの読み込み
\NewDocumentCommand{\myEmph}{O{} m m}{%
    \def\argOne{#1}%
    \ifx\argOne\empty
        \textbf{\color{#2}{#3}}% オプション引数が省略された場合
    \else
        \textbf<#1>{\color<#1>{#2}{#3}}% オプション引数が指定された場合
    \fi
}
%%%%%%%%%%%%%%%%%%%%%%%%%%%
%%%%%%%%%%%%%%%%%%%%%%%%%%%
%% 文末の上昇イントネーション記号 \myRisingPitch
%% 通常のイントネーション \myDownwardPitch
%% https://note.com/dan_oyama/n/n8be58e8797b2
%%%%%%%%%%%%%%%%%%%%%%%%%%%
\newcommand{\myRisingPitch}{
\begin{tikzpicture}[scale=0.3,baseline=0.3]
\draw[->,>=stealth] (0,0) to[bend right=45] (1,1);
\end{tikzpicture}
}
\newcommand{\myDownwardPitch}{
\begin{tikzpicture}[scale=0.3,baseline=0.3]
\draw[->,>=stealth] (0,1) to[bend left=45] (1,0);
\end{tikzpicture}
}
%%%%%%%%%%%%%%%%%%%%%%%%%%%%
%\AtBeginSection[%
%]{%
%  \begin{frame}[plain]\frametitle{授業の流れ}
%     \tableofcontents[currentsection]
%   \end{frame}%
%}

\usepackage{pxrubrica}
%%%%%%%%%%%%%%%%%%%%%%%%%%%
\title{English is fun.}
\subtitle{The girl playing in the yard is Jane.}
\author{}
\institute[]{}
\date[]

%%%%%%%%%%%%%%%%%%%%%%%%%%%%
%% TEXT
%%%%%%%%%%%%%%%%%%%%%%%%%%%%
\begin{document}

\begin{frame}[plain]
  \titlepage
\end{frame}

\section*{授業の流れ}
\begin{frame}[plain]
  \frametitle{授業の流れ}
  \tableofcontents
\end{frame}

%%%%%%%%%%%%%%%%%%%%%%%%%%%%%%%%%%%%%%%%%%%%%
\section{形容詞}
%%%%%%%%%%%%%%%%%%%%%%%%%%%%%%%%%%%%%%%%%%%%%
 \begin{frame}[plain]{形容詞 $+$ 名詞}
 \begin{enumerate}
  \item a big dog / big dogs
  \item a cute cat / cute cats\hfill{\scriptsize cute \textipa{/kj\'u:t/} かわいい}
  \item a small box / small boxes\hfill{\scriptsize small \textipa{/sm\'O:l/} 小さい}
  \item an expensive car / expensive cars\hfill{\scriptsize expensive \textipa{/Iksp\'ensIv/} 高価な}
  \item an interesting book / interesting books\hfill{\scriptsize interesting \textipa{/\'Intr@stIN/} おもしろい}
  \item an important question / important questions\hfill{\scriptsize important \textipa{/Imp\'O\textrhookschwa tnt/} 重要な}
 \end{enumerate}

\begin{block}<2->{Topic for Today}\small
名詞を修飾することばを「形容詞」といいます
\begin{itemize}\setbeamertemplate{items}[square]\small
 \item 形容詞は名詞の前に置きます
 \item \fcolorbox{black}{white}{\,$\text{(冠詞)}+\text{形容詞}+\text{名詞}$\,\,}\,\,\,は意味のかたまりになります
 \end{itemize}
     \end{block}

\mbox{}\hfill{\tiny 0400}\,{\scriptsize \myaudio{./audio/021_N_ing_01.mp3}}
 \end{frame}
%%%%%%%%%%%%%%%%%%%%%%
\section{現在分詞 $+$ 名詞}
%%%%%%%%%%%%%%%%%%%%%
\begin{frame}[plain,t]{冠詞 $+$ ---ing $+$ 名詞}
 \begin{enumerate}
	 \item<1-> I looked at \fbox{the cat}.\hspace{35pt}{\scriptsize look at ~を見る}\hfill{\scriptsize {\fbox{$\text{冠詞}+\text{名詞}$}}}
	 \item<2-> I looked at \fbox{the cute cat}.\hspace{10pt}{\scriptsize cute \textipa{/kj\'u:t/} かわいい}\hfill{\scriptsize {\fbox{$\text{冠詞}+\text{形容詞}+\text{名詞}$}}}
	 \item<3-> I looked at \fbox{the {\bfseries sleeping} cat}.\hfill{\scriptsize {\fbox{$\text{冠詞}+\textbf{\,\,---ing\,\,}+\text{名詞}$}}}
 \end{enumerate}

\vspace{20pt}

\begin{block}<4->{Topics for Today}\small
\Circled[fill color = white]{\,\,---ing\,\,}\,\,が名詞を修飾することがあります%
\hfill{\scriptsize つまり「形容詞」の役目をしています}

\begin{itemize}\setbeamertemplate{items}[square]\small
 \item<5-> この\,\,\Circled[fill color = white]{\,\,\textbf{---in}g\,\,}\,\,は「~している」という\kenten{進行}の意味
 \item<6-> この\,\,\Circled[fill color = white]{\,\,\textbf{---ing}\,\,}\,\,を「現在分詞」といいます\\
{\scriptsize \textdbend\textdbend 「現在分詞」は\kenten{現在}の意味ではなく\kenten{進行}の意味}\\
\hfill{\scriptsize 「過去分詞」は\kenten{過去}の意味ではなく「されている」という\kenten{受け身}の意味でしたね}
 \end{itemize}
     \end{block}
\vspace*{-5pt}

\mbox{}\hfill\tiny 0135{\,}{\scriptsize \myaudio{./audio/021_N_ing_02.mp3}}
\end{frame}
%%%%%%%%%%%%%%%%%%%%%%
\begin{frame}[plain]{Exercises}

{\small 日本語の意味になるよう(~~~~~~)の語を並べ替えましょう。先頭に来る語は大文字で始めてください}\hfill{\tiny 0226}\,{\scriptsize \myaudio{./audio/021_N_ing_03.mp3}}
 \begin{enumerate}
  \item {\small あなたは歌っているあの女性を知っていますか。}\\
	( you / singing / the / woman / do / know ) ?\\
	\visible<2->{Do you know the singing woman?}\hfill{\scriptsize singing \textipa{/s\'INiN/}}
  \item {\small 沸騰しているお湯は熱い。}
	( hot / is / water / boiling ) .\hfill{\scriptsize boil \textipa{/b\'OIl/} 沸騰する}\\
	\visible<3->{Boiling water is hot.}
  \item {\small 眠っているあのネコはかわいい。}
	That ( is / cat / sleeping ) cute.\\
	\visible<4->{That sleeping cat is cute.}
  \item {\small 昇っていく太陽が美しかった。}
	( sun / was / rising  / the ) beautiful.\\
	\visible<5->{The rising sun was beautiful.}\hfill{\scriptsize rise \textipa{/r\'aIz/} 昇る}
  \item {\small 飛んでいるあの鳥を見てください。}
	( at / birds / the / look / flying ) .\\
	\visible<6->{Look at the flying birds.}
 \end{enumerate}
\end{frame}
%%%%%%%%%%%%%%%%%%%%%%%%%%%%
\section{名詞 $+$ 現在分詞}
%%%%%%%%%%%%%%%%%%%%%%%%%%%
\begin{frame}[plain,t]{名詞 $+$ ---ing \ldots}
 \begin{enumerate} 
  \item<1-> The girl is Jane.\hfill{}{\scriptsize The girl($=\text{S}$)\,\,\,\,is($=\text{V}$)}
  \item<3-> The girl in the yard is Jane.%
	\hfill{}\visible<4->{\scriptsize The girl $\longleftarrow$\,\fbox{\,in the yard\,}}
  \item<5-> The girl {\bfseries playing} in the yard is Jane.%
	\hfill{}\visible<6->{\scriptsize The girl $\longleftarrow$\,\fbox{\,play{\bfseries ing} in the yard\,}}
 \end{enumerate}

\vspace{60pt}

\begin{block}<7->{Topic for Today}
\Circled[fill color = white]{\,\,\textbf{---ing}\,\,}\,\,がほかの語句を伴うときは、
後ろから前の名詞を修飾します%

\begin{itemize}\setbeamertemplate{items}[square]\small
 \item \Circled[fill color = white]{\,\,\textbf{---ing}\,\,}\,\, $\longrightarrow \text{\,\,名詞}$\hfill{\scriptsize the \textbf{sleeping} cat}
 \item $\text{名詞\,\,}\longleftarrow$\,\,\Circled[fill color = white]{\,\,\textbf{---ing}\,\,+\text{\,\,ほかの語句\,\,}}\hfill{\scriptsize the cat \textbf{sleeping} in the sun}
 \end{itemize}
     \end{block}
\vspace{-13pt}

\hfill{\tiny 0139}\,{\scriptsize \myaudio{./audio/021_N_ing_04.mp3}}

\begin{textblock*}{0.4\linewidth}(330pt,87pt)
\visible<8->{\begin{tikzpicture}
\duck[
signpost=\scalebox{0.3}{
\parbox{2.5cm}{\color{black}\centering
{\Large 名詞の後にきます!}}},
signcolour=brown!70!gray,
signback=white!80!brown,
graduate=gray!20!black,
tassel=red!70!black,
laughing,
speech={\tiny ---ing $+ \alpha$}
]
\end{tikzpicture}}
\end{textblock*}
\end{frame}
%%%%%%%%%%%%%%%%%%%%%%%%%%%%
\begin{frame}[plain]{Exercises}

{\small 日本語の意味になるよう(~~~~~~)の語を並べ替えましょう}%
\hfill{\tiny 0212}\,{\scriptsize \myaudio{./audio/021_N_ing_05.mp3}}
 \begin{enumerate}
  \item  {\small 英語を話している少年たちを見なさい。}\\
	Look ( English / the boys / speaking / at ) .\\
	\visible<2->{Look at the boys speaking English.}
  \item  {\small 海で泳いでいる男性はわたしの父です。}\\
	The ( swimming / my / is / man / in / the sea ) father.\\
	\visible<3->{The man swimming in the sea is my father.}
  \item  {\small その部屋で読書している少年はスペンサーです。}\\
	The ( the room / a book / boy / is / in / reading ) Spencer.\\
	\visible<4->{The boy reading a book in the room is Spencer.}
  \item  {\small 犬と走っている少女を知っていますか。}\\
	Do ( the girl / a dog / know / with / running  / you ) ?\\
	\visible<5->{Do you know the girl running with a dog?}
 \end{enumerate}

\begin{textblock*}{0.4\linewidth}(320pt,150pt)
\visible<1->{\begin{tikzpicture}
\bear[
scale=1,
speech={\tiny  ---ing $+ \alpha$},
signpost=\scalebox{.5}{
\parbox{2.5cm}{\color{black}
\centering 後から前の\\名詞を修飾\\します!}},
signcolour= brown!70!gray,
signback=white!80!brown
]
\end{tikzpicture}}
\end{textblock*}
\end{frame}
%%%%%%%%%%%%%%%%%%%%%%%%%%%%%
\section{まとめ}
%%%%%%%%%%%%%%%%%%%%%%%%%%%%%
\begin{frame}[plain]{まとめ1}

 \begin{block}{形容詞とは}\small
名詞を修飾することばを「形容詞」といいます
\begin{itemize}\setbeamertemplate{items}[square]\small
\setlength{\itemsep}{5pt}
 \item 形容詞は名詞の前に置きます
 \item \fcolorbox{black}{white}{\,$\text{(冠詞)}+\text{形容詞}+\text{名詞}$\,\,}\,\,\,は意味のかたまりになります
 \end{itemize}
     \end{block}

\begin{block}{名詞を修飾する---ing}\small
\Circled[fill color = white]{\,\,---ing\,\,}\,\,が名詞を修飾することがあります%
\hfill{\scriptsize つまり「形容詞」の役目をしています}

\begin{itemize}\setbeamertemplate{items}[square]\small
\setlength{\itemsep}{5pt}
 \item この\,\,\Circled[fill color = white]{\,\,---ing\,\,}\,\,は「~している」という\kenten{進行}の意味
 \item この\,\,\Circled[fill color = white]{\,\,---ing\,\,}\,\,を「現在分詞」といいます\\
{\scriptsize \textdbend\textdbend 「現在分詞」は\kenten{現在}の意味ではなく\kenten{進行}の意味}\\
\hfill{\scriptsize 「過去分詞」は\kenten{過去}の意味ではなく「されている」という\kenten{受け身}の意味でしたね}
 \end{itemize}
     \end{block}

\end{frame}
%%%%%%%%%%%%%%%%%%%%%%%%%%%%%%%%%
\begin{frame}[plain]{まとめ2}

\begin{block}{名詞を修飾する---ingの位置}\small


\begin{itemize}\setbeamertemplate{items}[square]\small
\setlength{\itemsep}{5pt}
 \item \Circled[fill color = white]{\,\,---ing\,\,}\,\,が単独のときは、前から後ろの名詞を修飾します

\hspace{50pt}\Circled[fill color = white]{\,\,---ing\,\,}\,\, $\longrightarrow \text{\,\,名詞}$\hfill{\scriptsize the \textbf{sleepin}g cat}\\
\hfill{\scriptsize the \textbf{singing} woman}\\
\hfill{\scriptsize \textbf{boiling} water}\\
\hfill{\scriptsize the \textbf{rising} sun}\\
\hfill{\scriptsize the \textbf{flying} birds}
 \item \Circled[fill color = white]{\,\,---ing\,\,}\,\,がほかの語句を伴うときは、
後ろから前の名詞を修飾します

\hspace{50pt}$\text{名詞\,\,}\longleftarrow$\,\,\Circled[fill color = white]{\,\,---ing\,\,+\text{\,\,ほかの語句\,\,}}\hfill{\scriptsize the cat \textbf{sleeping} in the sun}\\
\hfill{\scriptsize the boys \textbf{speaking} English}\\
\hfill{\scriptsize the man \textbf{swimming} in the sea}\\
\hfill{\scriptsize the boy \textbf{reading} a book}\\
\hfill{\scriptsize the girl \textbf{running} with a dog}\\
 \end{itemize}
     \end{block}
\end{frame}
%%%%%%%%%%%%%%%%%%%
\end{document}
