\documentclass[aspectratio=169,xcolor={dvipsnames,table}]{beamer}
\usepackage[no-math,deluxe,haranoaji]{luatexja-preset}
\renewcommand{\kanjifamilydefault}{\gtdefault}
\renewcommand{\emph}[1]{{\upshape\bfseries #1}}
\usetheme{metropolis}
\metroset{block=fill}
\setbeamertemplate{navigation symbols}{}
\setbeamertemplate{blocks}[rounded][shadow=false]
\usecolortheme[rgb={0.7,0.2,0.2}]{structure}
%%%%%%%%%%%%%%%%%%%%%%%%%%
%% Change alert block colors
%%% 1- Block title (background and text)
\setbeamercolor{block title alerted}{fg=mDarkTeal, bg=mLightBrown!45!yellow!45}
\setbeamercolor{block title example}{fg=magenta!10!black, bg=mLightGreen!60}
%%% 2- Block body (background)
\setbeamercolor{block body alerted}{bg=mLightBrown!25}
\setbeamercolor{block body example}{bg=mLightGreen!15}
%%%%%%%%%%%%%%%%%%%%%%%%%%%
%%%%%%%%%%%%%%%%%%%%%%%%%%%
%% さまざまなアイコン
%%%%%%%%%%%%%%%%%%%%%%%%%%%
%\usepackage{fontawesome}
\usepackage{fontawesome5}
\usepackage{figchild}
\usepackage{twemojis}
\usepackage{utfsym}
\usepackage{bclogo}
\usepackage{marvosym}
\usepackage{fontmfizz}
\usepackage{pifont}
\usepackage{phaistos}
\usepackage{worldflags}
\usepackage{jigsaw}
\usepackage{tikzlings}
\usepackage{tikzducks}
\usepackage{scsnowman}
\usepackage{epsdice}
\usepackage{halloweenmath}
\usepackage{svrsymbols}
\usepackage{countriesofeurope}
\usepackage{tipa}
\usepackage{manfnt}
%%%%%%%%%%%%%%%%%%%%%%%%%%%
\usepackage{tikz}
\usetikzlibrary{calc,patterns,decorations.pathmorphing,backgrounds}
\usepackage{tcolorbox}
\usepackage{tikzpeople}
\usepackage{circledsteps}
\usepackage{xcolor}
\usepackage{amsmath}
\usepackage{booktabs}
\usepackage{chronology}
\usepackage{signchart}
%%%%%%%%%%%%%%%%%%%%%%%%%%%
%% 場合分け
%%%%%%%%%%%%%%%%%%%%%%%%%%%
\usepackage{cases}
%%%%%%%%%%%%%%%%%%%%%%%%%%
\usepackage{pdfpages}
%%%%%%%%%%%%%%%%%%%%%%%%%%%
%% 音声リンク表示
\newcommand{\myaudio}[1]{\href{#1}{\faVolumeUp}}
%%%%%%%%%%%%%%%%%%%%%%%%%%
%% \myAnch{<名前>}{<色>}{<テキスト>}
%% 指定のテキストを指定の色の四角枠で囲み, 指定の名前をもつTikZの
%% ノードとして出力する. 図には remember picture 属性を付けている
%% ので外部から参照可能である.
\newcommand*{\myAnch}[3]{%
  \tikz[remember picture,baseline=(#1.base)]
    \node[draw,rectangle,line width=1pt,#2] (#1) {\normalcolor #3};
}
%%%%%%%%%%%%%%%%%%%%%%%%%%
%% \myEmph コマンドの定義
%%%%%%%%%%%%%%%%%%%%%%%%%%
%\newcommand{\myEmph}[3]{%
%    \textbf<#1>{\color<#1>{#2}{#3}}%
%}
\usepackage{xparse} % xparseパッケージの読み込み
\NewDocumentCommand{\myEmph}{O{} m m}{%
    \def\argOne{#1}%
    \ifx\argOne\empty
        \textbf{\color{#2}{#3}}% オプション引数が省略された場合
    \else
        \textbf<#1>{\color<#1>{#2}{#3}}% オプション引数が指定された場合
    \fi
}
%%%%%%%%%%%%%%%%%%%%%%%%%%%
%%%%%%%%%%%%%%%%%%%%%%%%%%%
%% 文末の上昇イントネーション記号 \myRisingPitch
%% 通常のイントネーション \myDownwardPitch
%% https://note.com/dan_oyama/n/n8be58e8797b2
%%%%%%%%%%%%%%%%%%%%%%%%%%%
\newcommand{\myRisingPitch}{
\begin{tikzpicture}[scale=0.3,baseline=0.3]
\draw[->,>=stealth] (0,0) to[bend right=45] (1,1);
\end{tikzpicture}
}
\newcommand{\myDownwardPitch}{
\begin{tikzpicture}[scale=0.3,baseline=0.3]
\draw[->,>=stealth] (0,1) to[bend left=45] (1,0);
\end{tikzpicture}
}
%%%%%%%%%%%%%%%%%%%%%%%%%%%%
%\AtBeginSection[%
%]{%
%  \begin{frame}[plain]\frametitle{授業の流れ}
%     \tableofcontents[currentsection]
%   \end{frame}%
%}

\usepackage{pxrubrica}
%%%%%%%%%%%%%%%%%%%%%%%%%%%
\title{English is fun.}
\subtitle{The book which I read was interesting.}
\author{}
\institute[]{}
\date[]

%%%%%%%%%%%%%%%%%%%%%%%%%%%%
%% TEXT
%%%%%%%%%%%%%%%%%%%%%%%%%%%%
\begin{document}

\begin{frame}[plain]
  \titlepage
\end{frame}

\section*{授業の流れ}
\begin{frame}[plain]
  \frametitle{授業の流れ}
  \tableofcontents
\end{frame}

\section{関係代名詞}
\subsection{which}
%%%%%%%%%%%%%%%%%%%%%%%
\begin{frame}[plain]{もの $\leftarrow$ \fbox{which S  + V ...}}
\begin{enumerate}
 \item<1->  \begin{enumerate}
	 \item<1-> We went to the restaurant which was famous for seafood.\\
	       \hfill\visible<2->{\scriptsize the restaurant $\longleftarrow$ \,\,\fbox{which was famous for seafood}}
	 \item<1-> She works for the factory which makes computers.\\
	       \hfill\visible<3->{\scriptsize the factory $\longleftarrow$ \fbox{\,\,which makes computers\,\,}}
	\end{enumerate}
 \item<4->  \begin{enumerate}
	 \item<4-> I ate the cake which Bob made.%
	       \hfill{}\visible<5->{{\scriptsize the cake $\longleftarrow$\,\fbox{\,which Bob made\,}}}
	 \item<4-> The book which I read yesterday was interesting.\\%
	\hfill{}\visible<6->{\scriptsize the book $\longleftarrow$\,\fbox{\,which I read yesterday\,}}
	\end{enumerate}
\end{enumerate}

\visible<8->{%
\begin{exampleblock}{Topics for Today}
「もの」について、後ろから詳しく説明するときwhichをつかいます

\begin{enumerate}\small
 \item $\text{もの\,\,}\longleftarrow$\,\,\Circled[fill color = white]{\,\,which\,\,\,$+$\,\,\,\,V\,\text{\,\,\ldots\,\,\,}}
 \item $\text{もの\,\,}\longleftarrow$\,\,\Circled[fill color = white]{\,\,which\,\,\,\,S\,\,$+$\,\,V\,\,\text{\,\,\ldots\,\,\,}}
 \end{enumerate}
     \end{exampleblock}
}
\end{frame}
%%%%%%%%%%%%%%%%%%%%%%%%%%%%
\begin{frame}[plain]{もの $\leftarrow$ \fbox{which S + V ...}}

 \begin{enumerate}
  \item<1-> I ate \myAnch{a1}{Maroon}{\bfseries the cake}.
  \item<2-> Bob made \myAnch{a2}{NavyBlue}{the cake}.\\[10pt]
	\hspace{15pt}\visible<3->{\myAnch{a3}{white}{\Circled[fill color = white]{\,\,which\,\,}}Bob made}
  \item<4-> I ate \myAnch{a4}{Maroon}{\bfseries the cake} \myAnch{a5}{black}{\Circled[fill color = white]{\,\,which\,\,} Bob made}.
 \end{enumerate}

\begin{tikzpicture}[remember picture, overlay]
\visible<3->{\draw[->, thick, NavyBlue] (a2.south) to[out=-120, in = 60] (a3.north);}
\coordinate (A5) at ($(a5) - (0,20pt)$); % 10pt below a1
\coordinate (A4) at ($(a4) - (0,20pt)$); % 10pt below a1
%\visible<5->{\draw[->, thick, orange] (a5.south) to (a4.north);}
 % Draw the arrow with right angles
 \visible<5->{\draw[->,black,thick] (a5) -- (A5) -- (A4) -- (a4);}
\end{tikzpicture}

\visible<8->{%
\begin{exampleblock}{Topics for Today}
 関係代名詞のwhichは後ろから前の「もの」について詳しく説明します

\begin{enumerate}\small
 \item 2つの文で、同じ「もの」を指している語句を見つけます
 \item 2つめを\,\,\Circled[fill color = white]{\,\,which\,\,}\,\,にして、先頭に移動させます
 \item 1つめの直後につなげます
 \end{enumerate}
     \end{exampleblock}
}
\end{frame}
%%%%%%%%%%%%%%%%%%%%%%%%%%%%%%%%%%
%%%%%%%%%%%%%%%%%%%%%%%
\begin{frame}[plain]{もの $\leftarrow$ \fbox{which S + V ...}}

 \begin{enumerate}
  \item<1-> \myAnch{a1}{Maroon}{\bfseries The book} was interesting.
  \item<2-> I read \myAnch{a2}{NavyBlue}{the book} yesterday.\\[10pt]
	\visible<3->{\myAnch{a3}{white}{\Circled[fill color = white]{\,\,which\,\,}}I read yesterday}
  \item<4-> \myAnch{a4}{Maroon}{\bfseries The book} \myAnch{a5}{black}{\Circled[fill color = white]{\,\,which\,\,} I read yesterday} was interesting.
 \end{enumerate}

\begin{tikzpicture}[remember picture, overlay]
\visible<3->{\draw[->, thick, NavyBlue] (a2.south) to (a3.north);}
\coordinate (A5) at ($(a5) - (0,20pt)$); % 10pt below a1
\coordinate (A4) at ($(a4) - (0,20pt)$); % 10pt below a1
%\visible<5->{\draw[->, thick, orange] (a5.south) to (a4.north);}
 % Draw the arrow with right angles
 \visible<5->{\draw[->,black,thick] (a5) -- (A5) -- (A4) -- (a4);}
\end{tikzpicture}

\visible<8->{%
\begin{exampleblock}{Topics for Today}
 関係代名詞のwhichは後ろから前の「もの」について詳しく説明します

\begin{enumerate}\small
 \item 2つの文で、同じ「もの」を指している語句を見つけます
 \item 2つめを\,\,\Circled[fill color = white]{\,\,which\,\,}\,\,にして、先頭に移動させます
 \item 1つめの直後につなげます
 \end{enumerate}
     \end{exampleblock}
}
\end{frame}
%%%%%%%%%%%%%%%%%%%%%%%%%%%%%%%%%%
\begin{frame}[plain]{Exercises}
\fbox{  }\,\,で囲まれた語を関係代名詞whichにかえて各2文を1文にまとめましょう
\begin{enumerate}
 \item \begin{enumerate}
	\item She bought {\bfseries the house}.
	\item She liked \fbox{the house}.
	\item \visible<2->{$\text{1.1}+\text{1.2}\rightarrow$\,\,\,She bought the house which she liked.}
       \end{enumerate} \item \begin{enumerate}
	\item {\bfseries The cake} was delicious.
	\item Helen made \fbox{the cake}.
	\item \visible<3->{$\text{2.1}+\text{2.2}\rightarrow$\,\,\,The cake which Helen made was delicious.}
       \end{enumerate}

 \item \begin{enumerate}
	\item {\bfseries The book} was interesting.
	\item David read \fbox{the book} last night.
	\item \visible<4->{$\text{3.1}+\text{3.2}\rightarrow$\,\,\,The book which David read last night was interesting.}
       \end{enumerate}
\end{enumerate} 

\hfill\fbox{  }\,\,で囲まれた語をwhichにかえて太字の語の後ろにつなげます
\end{frame}
%%%%%%%%%%%%%%%%%%%%%%%%%%%%%%%%%%%
\begin{frame}[plain]{Exercises}
日本語の意味になるよう(~~~~~~)の語句を並べ替えましょう
 \begin{enumerate}
  \item これはその有名な画家が書いた手紙です。%
	\hfill{\scriptsize painter: 画家}\\
	This is ( the famous painter / a letter / wrote / which ) .\\
	\visible<2->{This is a letter which the famous painter wrote.}
  \item 私の母がイタリアで撮影した写真を見てください。\\
	Please ( took / look / in / at / which / my mother / the pictures ) Italy.\\
	\visible<3->{Please look at the pictures which my mother took in Italy.}
  \item 彼らが歌った歌はとても美しかった。\\
	The song ( very beautiful / they / was / which / sang ) .\\
	\visible<4->{The song which they sang was very beautiful.}
  \item われわれが見た映画は長かった。\\
	The movie ( we / was / saw / which / long ) .\\
	\visible<5->{The movie which we saw was long. }
 \end{enumerate}
\end{frame}
%%%%%%%%%%%%%%%%%%%%%%%%%%%%
%%%%%%%%%%%%%%%%%%%%%%%
\begin{frame}[plain]{人 $\leftarrow$ \fbox{whom S + V ...}}
\begin{enumerate}
 \item<1->  \begin{enumerate}
	 \item<1-> I know a girl who plays tennis well.\\
	       \hfill\visible<2->{\scriptsize a girl $\longleftarrow$ \,\,\fbox{who plays tennis well}}
	 \item<1-> The girl who wrote this poem is Jennifer.\\
	       \hfill\visible<3->{\scriptsize The girl $\longleftarrow$ \fbox{\,\,who wrote this poem\,\,}}
	\end{enumerate}
 \item<4->  \begin{enumerate}
	 \item<4-> I ate the cake which Bob made.%
	       \hfill{}\visible<5->{{\scriptsize the cake $\longleftarrow$\,\fbox{\,which Bob made\,}}}
	 \item<4-> The book which I read yesterday was interesting.\\%
	\hfill{}\visible<6->{\scriptsize the book $\longleftarrow$\,\fbox{\,which I read yesterday\,}}
	\end{enumerate}
\end{enumerate}

\visible<8->{%
\begin{exampleblock}{Topics for Today}
「もの」について、後ろから詳しく説明するときwhichをつかいます

\begin{enumerate}\small
 \item $\text{もの\,\,}\longleftarrow$\,\,\Circled[fill color = white]{\,\,which\,\,\,$+$\,\,\,\,V\,\text{\,\,\ldots\,\,\,}}
 \item $\text{もの\,\,}\longleftarrow$\,\,\Circled[fill color = white]{\,\,which\,\,\,\,S\,\,$+$\,\,V\,\,\text{\,\,\ldots\,\,\,}}
 \end{enumerate}
     \end{exampleblock}
}
\end{frame}
%%%%%%%%%%%%%%%%%%%%%%%%%%%%

\end{document}
