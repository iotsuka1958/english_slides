\documentclass[aspectratio=169,xcolor={dvipsnames,table}]{beamer}
\usepackage[no-math,deluxe,haranoaji]{luatexja-preset}
\renewcommand{\kanjifamilydefault}{\gtdefault}
\renewcommand{\emph}[1]{{\upshape\bfseries #1}}
\usetheme{metropolis}
\metroset{block=fill}
\setbeamertemplate{navigation symbols}{}
\setbeamertemplate{blocks}[rounded][shadow=false]
\usecolortheme[rgb={0.7,0.2,0.2}]{structure}
%%%%%%%%%%%%%%%%%%%%%%%%%%
%% Change alert block colors
%%% 1- Block title (background and text)
\setbeamercolor{block title alerted}{fg=mDarkTeal, bg=mLightBrown!45!yellow!45}
\setbeamercolor{block title example}{fg=magenta!10!black, bg=mLightGreen!60}
%%% 2- Block body (background)
\setbeamercolor{block body alerted}{bg=mLightBrown!25}
\setbeamercolor{block body example}{bg=mLightGreen!15}
%%%%%%%%%%%%%%%%%%%%%%%%%%%
%%%%%%%%%%%%%%%%%%%%%%%%%%%
%% さまざまなアイコン
%%%%%%%%%%%%%%%%%%%%%%%%%%%
%\usepackage{fontawesome}
\usepackage{fontawesome5}
\usepackage{figchild}
\usepackage{twemojis}
\usepackage{utfsym}
\usepackage{bclogo}
\usepackage{marvosym}
\usepackage{fontmfizz}
\usepackage{pifont}
\usepackage{phaistos}
\usepackage{worldflags}
\usepackage{jigsaw}
\usepackage{tikzlings}
\usepackage{tikzducks}
\usepackage{scsnowman}
\usepackage{epsdice}
\usepackage{halloweenmath}
\usepackage{svrsymbols}
\usepackage{countriesofeurope}
\usepackage{tipa}
\usepackage{manfnt}
%%%%%%%%%%%%%%%%%%%%%%%%%%%
\usepackage{tikz}
\usetikzlibrary{calc,patterns,decorations.pathmorphing,backgrounds}
\usepackage{tcolorbox}
\usepackage{tikzpeople}
\usepackage{circledsteps}
\usepackage{xcolor}
\usepackage{amsmath}
\usepackage{booktabs}
\usepackage{chronology}
\usepackage{signchart}
%%%%%%%%%%%%%%%%%%%%%%%%%%%
%% 場合分け
%%%%%%%%%%%%%%%%%%%%%%%%%%%
\usepackage{cases}
%%%%%%%%%%%%%%%%%%%%%%%%%%
\usepackage{pdfpages}
%%%%%%%%%%%%%%%%%%%%%%%%%%%
%% 音声リンク表示
\newcommand{\myaudio}[1]{\href{#1}{\faVolumeUp}}
%%%%%%%%%%%%%%%%%%%%%%%%%%
%% \myAnch{<名前>}{<色>}{<テキスト>}
%% 指定のテキストを指定の色の四角枠で囲み, 指定の名前をもつTikZの
%% ノードとして出力する. 図には remember picture 属性を付けている
%% ので外部から参照可能である.
\newcommand*{\myAnch}[3]{%
  \tikz[remember picture,baseline=(#1.base)]
    \node[draw,rectangle,line width=1pt,#2] (#1) {\normalcolor #3};
}
%%%%%%%%%%%%%%%%%%%%%%%%%%
%% \myEmph コマンドの定義
%%%%%%%%%%%%%%%%%%%%%%%%%%
%\newcommand{\myEmph}[3]{%
%    \textbf<#1>{\color<#1>{#2}{#3}}%
%}
\usepackage{xparse} % xparseパッケージの読み込み
\NewDocumentCommand{\myEmph}{O{} m m}{%
    \def\argOne{#1}%
    \ifx\argOne\empty
        \textbf{\color{#2}{#3}}% オプション引数が省略された場合
    \else
        \textbf<#1>{\color<#1>{#2}{#3}}% オプション引数が指定された場合
    \fi
}
%%%%%%%%%%%%%%%%%%%%%%%%%%%
%%%%%%%%%%%%%%%%%%%%%%%%%%%
%% 文末の上昇イントネーション記号 \myRisingPitch
%% 通常のイントネーション \myDownwardPitch
%% https://note.com/dan_oyama/n/n8be58e8797b2
%%%%%%%%%%%%%%%%%%%%%%%%%%%
\newcommand{\myRisingPitch}{
\begin{tikzpicture}[scale=0.3,baseline=0.3]
\draw[->,>=stealth] (0,0) to[bend right=45] (1,1);
\end{tikzpicture}
}
\newcommand{\myDownwardPitch}{
\begin{tikzpicture}[scale=0.3,baseline=0.3]
\draw[->,>=stealth] (0,1) to[bend left=45] (1,0);
\end{tikzpicture}
}
%%%%%%%%%%%%%%%%%%%%%%%%%%%%
%\AtBeginSection[%
%]{%
%  \begin{frame}[plain]\frametitle{授業の流れ}
%     \tableofcontents[currentsection]
%   \end{frame}%
%}

\usepackage{pxrubrica}
%%%%%%%%%%%%%%%%%%%%%%%%%%%
\title{English is fun.}
\subtitle{It is important for you to go there.}
\author{}
\institute[]{}
\date[]

%%%%%%%%%%%%%%%%%%%%%%%%%%%%
%% TEXT
%%%%%%%%%%%%%%%%%%%%%%%%%%%%
\begin{document}

\begin{frame}[plain]
  \titlepage
\end{frame}

\section*{授業の流れ}
\begin{frame}[plain]
  \frametitle{授業の流れ}
  \tableofcontents
\end{frame}
%%%%%%%%%%%%%%%%%%%%%%%%%
\begin{frame}[plain]{It is \ldots\,\, to--- .}
\large
 \begin{enumerate}
  \item<1->  Science is important.\hfill{\scriptsize important \textipa{/Imp\'O\textrhookschwa t@nt/} 重要な }
  \item<2->  \alt<3->{{\setlength{\fboxrule}{1pt}\fcolorbox{black}{yellow!20}{To study science}}}{To study science}\,\only<3->{($=$S)} is\only<3->{($=$V)} important.
  \item<4-> \myAnch{a1}{Maroon}{It}\,is important\,\,\,\,\myAnch{a2}{Maroon}{to study science.}
 \end{enumerate}

\begin{tikzpicture}[remember picture, overlay]
%\visible<3->{\draw[->, thick, NavyBlue] (a1.south) to (a2.north);}
\coordinate (A1) at ($(a1) - (0,20pt)$); % 10pt below a1
\coordinate (A2) at ($(a2) - (0,20pt)$); % 10pt below a2
%\visible<5->{\draw[->, thick, orange] (a5.south) to (a4.north);}
 % Draw the arrow with right angles
 \visible<5->{\draw[->,Maroon,thick] (a1) -- (A1) -- (A2) -- (a2);}
\end{tikzpicture}

\begin{block}<6->{hoge}\small
\begin{itemize}\setbeamertemplate{items}[square]
 \item 主語が長くなるのを避けるために\\
\mbox{}\hspace{30pt}{\setlength{\fboxrule}{1pt}\fcolorbox{black}{yellow!20}{\bfseries To--- \hspace{30pt}\mbox{}}}\,\,\,is\,\,\ldots\,. を\,\,\,
{\bfseries It} is \ldots\,\,\,{\setlength{\fboxrule}{1pt}\fcolorbox{Maroon}{white}{\bfseries to--- \hspace{30pt}\mbox{}}}\,.\,
とすることがあります
 \item このitは形式上の主語です\hfill{\scriptsize このitを\kenten{形式主語}と呼びます}
 \item 実質的な主語は\,{\setlength{\fboxrule}{1pt}\fcolorbox{Maroon}{white}{\bfseries to---\hspace{30pt}\mbox{}}} のところです
\end{itemize}

\end{block}
\end{frame}

\begin{frame}[plain]{It is \ldots\,\, for ~ to--- .}
 \large

\begin{enumerate}
 \item {\bfseries It} is important {\bfseries to} study math.\hfill{\scriptsize important \textipa{/Imp\'O\textrhookschwa t@nt/} 重要な}
 \item {\bfseries It} is important {\bfseries for} kids {\bfseries to} study math.\hfill{\scriptsize kid \textipa{/k\'Id/} こども}.
\end{enumerate}

\begin{block}{to不定詞の意味上の主語}\small
\begin{itemize}\setbeamertemplate{items}[square]
 \item to不定詞が表す動作をだれがするのかを示すときはfor ~をto不定詞の前に置きます
 \item It is \ldots\,\, for ~ to--\,\,で「~が---するのは\ldots\,\,だ」の意味になります
\end{itemize}
 
\end{block}
\end{frame}
%%%%%%%%%%%%%%%%%%%%%%%%%
\begin{frame}[plain]{Exercises}
 あたえられた日本語の意味になるよう(~~~~~~~~)内の語句を並べかえましょう。ただし先頭の単語は大文字ではじめてください

\begin{enumerate}
 \item こどもがその川で泳ぐのは危険です\hfill{\scriptsize dangerous \textipa{/d\'eIndZ(@)r@s/} 危険な}\\
It ( kids / swim / is / in / for / to / dangerous ) the river.\\
It is dangerous for children to swim in the river.
 \item 彼がそこに行くことは不可能だ\hfill{\scriptsize impossible \textipa{/imp\'As@bl/} 不可能な}\\
It ( impossible / to / for / go / him / is ) there.\\
It is impossible for him to go there.
 \item 彼女がその問題を解くのはむずかしい\hfill{\scriptsize difficult \textipa{/d\'IfIk(@)lt/} 難しい}\\
( difficult / her / it / to / is / solve / for ) the problem.\\
It is difficult for her to solve the problem.

\end{enumerate}
\end{frame}
\end{document}
