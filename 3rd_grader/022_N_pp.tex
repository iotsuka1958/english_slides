\documentclass[aspectratio=169,xcolor={dvipsnames,table}]{beamer}
\usepackage[no-math,deluxe,haranoaji]{luatexja-preset}
\renewcommand{\kanjifamilydefault}{\gtdefault}
\renewcommand{\emph}[1]{{\upshape\bfseries #1}}
\usetheme{metropolis}
\metroset{block=fill}
\setbeamertemplate{navigation symbols}{}
\setbeamertemplate{blocks}[rounded][shadow=false]
\usecolortheme[rgb={0.7,0.2,0.2}]{structure}
%%%%%%%%%%%%%%%%%%%%%%%%%%
%% Change alert block colors
%%% 1- Block title (background and text)
\setbeamercolor{block title alerted}{fg=mDarkTeal, bg=mLightBrown!45!yellow!45}
\setbeamercolor{block title example}{fg=magenta!10!black, bg=mLightGreen!70}
%%% 2- Block body (background)
\setbeamercolor{block body alerted}{bg=mLightBrown!25}
\setbeamercolor{block body example}{bg=mLightGreen!15}
%%%%%%%%%%%%%%%%%%%%%%%%%%%
%%%%%%%%%%%%%%%%%%%%%%%%%%%
%% さまざまなアイコン
%%%%%%%%%%%%%%%%%%%%%%%%%%%
%\usepackage{fontawesome}
\usepackage{fontawesome5}
\usepackage{figchild}
\usepackage{twemojis}
\usepackage{utfsym}
\usepackage{bclogo}
\usepackage{marvosym}
\usepackage{fontmfizz}
\usepackage{pifont}
\usepackage{phaistos}
\usepackage{worldflags}
\usepackage{jigsaw}
\usepackage{tikzlings}
\usepackage{tikzducks}
\usepackage{scsnowman}
\usepackage{epsdice}
\usepackage{halloweenmath}
\usepackage{svrsymbols}
\usepackage{countriesofeurope}
\usepackage{tipa}
\usepackage{manfnt}
%%%%%%%%%%%%%%%%%%%%%%%%%%%
\usepackage{tikz}
\usetikzlibrary{calc,patterns,decorations.pathmorphing,backgrounds}
\usepackage{tcolorbox}
\usepackage{tikzpeople}
\usepackage{circledsteps}
\usepackage{xcolor}
\usepackage{amsmath}
\usepackage{booktabs}
\usepackage{chronology}
\usepackage{signchart}
%%%%%%%%%%%%%%%%%%%%%%%%%%%
%% 場合分け
%%%%%%%%%%%%%%%%%%%%%%%%%%%
\usepackage{cases}
%%%%%%%%%%%%%%%%%%%%%%%%%%
\usepackage{pdfpages}
%%%%%%%%%%%%%%%%%%%%%%%%%%%
%% 音声リンク表示
\newcommand{\myaudio}[1]{\href{#1}{\faVolumeUp}}
%%%%%%%%%%%%%%%%%%%%%%%%%%
%% \myAnch{<名前>}{<色>}{<テキスト>}
%% 指定のテキストを指定の色の四角枠で囲み, 指定の名前をもつTikZの
%% ノードとして出力する. 図には remember picture 属性を付けている
%% ので外部から参照可能である.
\newcommand*{\myAnch}[3]{%
  \tikz[remember picture,baseline=(#1.base)]
    \node[draw,rectangle,line width=1pt,#2] (#1) {\normalcolor #3};
}
%%%%%%%%%%%%%%%%%%%%%%%%%%
%% \myEmph コマンドの定義
%%%%%%%%%%%%%%%%%%%%%%%%%%
%\newcommand{\myEmph}[3]{%
%    \textbf<#1>{\color<#1>{#2}{#3}}%
%}
\usepackage{xparse} % xparseパッケージの読み込み
\NewDocumentCommand{\myEmph}{O{} m m}{%
    \def\argOne{#1}%
    \ifx\argOne\empty
        \textbf{\color{#2}{#3}}% オプション引数が省略された場合
    \else
        \textbf<#1>{\color<#1>{#2}{#3}}% オプション引数が指定された場合
    \fi
}
%%%%%%%%%%%%%%%%%%%%%%%%%%%
%%%%%%%%%%%%%%%%%%%%%%%%%%%
%% 文末の上昇イントネーション記号 \myRisingPitch
%% 通常のイントネーション \myDownwardPitch
%% https://note.com/dan_oyama/n/n8be58e8797b2
%%%%%%%%%%%%%%%%%%%%%%%%%%%
\newcommand{\myRisingPitch}{
\begin{tikzpicture}[scale=0.3,baseline=0.3]
\draw[->,>=stealth] (0,0) to[bend right=45] (1,1);
\end{tikzpicture}
}
\newcommand{\myDownwardPitch}{
\begin{tikzpicture}[scale=0.3,baseline=0.3]
\draw[->,>=stealth] (0,1) to[bend left=45] (1,0);
\end{tikzpicture}
}
%%%%%%%%%%%%%%%%%%%%%%%%%%%%
%\AtBeginSection[%
%]{%
%  \begin{frame}[plain]\frametitle{授業の流れ}
%     \tableofcontents[currentsection]
%   \end{frame}%
%}

\usepackage{pxrubrica}
%%%%%%%%%%%%%%%%%%%%%%%%%%%
\title{English is fun.}
\subtitle{I have a book written in English.}
\author{}
\institute[]{}
\date[]

%%%%%%%%%%%%%%%%%%%%%%%%%%%%
%% TEXT
%%%%%%%%%%%%%%%%%%%%%%%%%%%%
\begin{document}

\begin{frame}[plain]
  \titlepage
\end{frame}

\section*{授業の流れ}
\begin{frame}[plain]
  \frametitle{授業の流れ}
  \tableofcontents
\end{frame} 
%%%%%%%%%%%%%%%%%%%%%%%%%%%%
\section{名詞を修飾する過去分詞}
\subsection{過去分詞 $+$ 名詞}
%%%%%%%%%%%%%%%%
 \begin{frame}[plain]{p.p. $+$ 名詞}
  \begin{enumerate}
   \item \begin{enumerate}
	  \item<1-> break --- broke --- broken%
\hfill{}{\scriptsize 原形 --- 過去形 --- 過去分詞}
	  \item<2-> He broke the window.
	  \item<3-> The window was broken by him..\hfill\visible<4->{\scriptsize $\text{受け身}=\text{be動詞}+\text{過去分詞}$}
	 \end{enumerate}
   \item \begin{enumerate}
	  \item<5-> Look at \fbox{the window}.
	  \item<6-> Look at \fbox{the old window}.%
\hfill\visible<7->{\scriptsize \fbox{$\text{冠詞}+\text{形容詞}+\text{名詞}$}}
	  \item<8-> Look at \fbox{the \myEmph[8-]{Maroon}{broken} window.}%
\hfill\visible<8->{\scriptsize \fbox{$\text{冠詞}+\text{過去分詞}+\text{名詞}$}}
	 \end{enumerate}
  \end{enumerate}

\visible<10->{%
\begin{exampleblock}{Topic for Today}\small
\Circled[fill color = white]{\,\,過去分詞\,\,}\,\,が名詞を修飾することがあります%
\hfill{\scriptsize つまり「形容詞」の役目をしています}

\begin{itemize}\setbeamertemplate{items}[square]\small
 \item \,\,\Circled[fill color = white]{\,\,過去分詞\,\,}\,\,は「~された」という\kenten{受け身}の意味
 \item \,\,\Circled[fill color = white]{\,\,過去分詞\,\,}\,\,を\,\,\Circled[fill color = white]{\,\,p.p.\,\,}\,\,と表記することがあります
 \end{itemize}
     \end{exampleblock}
}
\mbox{}\hfill{\scriptsize \myaudio{./audio/022_N_pp_01.mp3}}
\end{frame}
%%%%%%%%%%%%%%%%%%%%%%%%%%%
\subsection{名詞 $+$ 過去分詞}
\begin{frame}[plain]{名詞 $+$ p.p. \ldots}
 \begin{enumerate}
  \item \begin{enumerate}
	 \item<1-> The man is my father.\hfill\visible<2->{\scriptsize  \fbox{$\text{冠詞}+\text{名詞}$}}
	 \item<3-> The singing man is my father.\hfill\visible<4->{\scriptsize \fbox{$\text{冠詞}+\text{---ing}+\text{名詞}$}}
	 \item<5-> The man singing over there is my father.\hfill\visible<6->{\scriptsize $\text{名詞}\longleftarrow$\fbox{$\text{\,---ing\,}+\text{ほかの語句}$}}
	\end{enumerate}
  \item \begin{enumerate}
	 \item<6-> Look at the window.\hfill\visible<7->{\scriptsize  \fbox{$\text{冠詞}+\text{名詞}$}}
	 \item<8-> Look at the broken window.\hfill\visible<9->{\scriptsize \fbox{$\text{冠詞}+\text{p.p.}+\text{名詞}$}}
	 \item<10-> Look at the window broken in the storm last night.\hfill\visible<11->{\scriptsize $\text{名詞}\longleftarrow$\fbox{$\text{\,p.p.\,}+\text{ほかの語句}$}}
	\end{enumerate}
 \end{enumerate}

\visible<12->{%
\begin{exampleblock}{Topic for Today}\small
\Circled[fill color = white]{\,\,p.p.\,\,}\,\,がほかの語句を伴うときは、
後ろから前の名詞を修飾します%

\begin{itemize}\setbeamertemplate{items}[square]\small
 \item \Circled[fill color = white]{\,\,p.p.\,\,}\,\, $\longrightarrow \text{\
                                                                               \,\,名詞}$\hfill{\scriptsize the broken window}
 \item $\text{名詞\,\,}\longleftarrow$\,\,\Circled[fill color = white]{\,\,p.p.\,\,+\text{\,\,ほかの語句\,\,}}\hfill{\scriptsize the window broken in the storm last night}
 \end{itemize}
     \end{exampleblock}
}
\mbox{}\hfill{\scriptsize \myaudio{./audio/022_N_pp_02.mp3}}
\end{frame}
%%%%%%%%%%%%%%%%%%%%%%%%%%%%%%%%
\begin{frame}[plain]{Exercises}
日本語の意味になるよう(~~~~~~)内の語句を並べ替えましょう%
\mbox{}\hfill{\scriptsize \myaudio{./audio/022_N_pp_03.mp3}}
 \begin{enumerate}
  \item デイブはドイツ製の新車を購入した。\\
	Dave ( new / a / in / made / bought / car ) Germany.\\
	\visible<2->{Dave bought a new car made in Germany.}
  \item 彼らはヘレンが調理した食事を楽しんだ。\hfill{\scriptsize meal: 食事}\\
	They ( by / cooked / enjoyed / the meal ) Helen.\\
	\visible<3->{They enjoyed the meal cooked by Helen.}
  \item これはパティが撮影した有名な写真です。\\
	This ( by / picture / famous / is / taken / a ) Pattie.\\
	\visible<4->{This is a famous picture taken by Pattie.}
  \item 彼女は英語で書かれたおもしろい本を読んだ。\\
	She ( an / English / written / interesting / read / in / book ).\\
	\visible<4->{She read an interesting book written in English.}%
\hfill{}{\scriptsize readの過去形read \textipa{/r\'ed/}}
 \end{enumerate}
\end{frame}
%%%%%%%%%%%%%%%%%%%%%%%%%%%%%
\section{まとめ}
\begin{frame}[plain]{まとめ}
 
\begin{block}{Topic for Today}\setbeamertemplate{items}[square]\small

\begin{itemize}\setbeamertemplate{items}[square]\small
 \item \Circled[fill color = white]{\,\,過去分詞\,\,}\,\,は「~される」という意味\hfill{}{\scriptsize 過去分詞のことを\Circled[fill color = white]{\,\,p.p.\,\,}\,\,と表記することがあります}
 \item \Circled[fill color = white]{\,\,過去分詞\,\,}\,\,が名詞を修飾することがあります%
\hfill{\scriptsize つまり「形容詞」の役目をしています}
 \end{itemize}
     \end{block}

\begin{block}{Topic for Today}\small
\Circled[fill color = white]{\,\,過去分詞\,\,}\,\,がほかの語句を伴うときは、
後ろから前の名詞を修飾します%

\begin{itemize}\setbeamertemplate{items}[square]\small
 \item \Circled[fill color = white]{\,\,p.p.\,\,}\,\, $\longrightarrow \text{\,\,名詞}$\hfill{\scriptsize the broken window}
 \item $\text{名詞\,\,}\longleftarrow$\,\,\Circled[fill color = white]{\,\,p.p.\,\,+\text{\,\,ほかの語句\,\,}}\hfill{\scriptsize a car made in Germany}
 \end{itemize}
     \end{block}
\end{frame}
%%%%%%%%%%%%%%%%%%%
\end{document}
