\documentclass[aspectratio=169,xcolor={dvipsnames,table}]{beamer}
\usepackage[no-math,deluxe,haranoaji]{luatexja-preset}
\renewcommand{\kanjifamilydefault}{\gtdefault}
\renewcommand{\emph}[1]{{\upshape\bfseries #1}}
\usetheme{metropolis}
\metroset{block=fill}
\setbeamertemplate{navigation symbols}{}
\setbeamertemplate{blocks}[rounded][shadow=false]
\usecolortheme[rgb={0.7,0.2,0.2}]{structure}
%%%%%%%%%%%%%%%%%%%%%%%%%%
%% Change alert block colors
%%% 1- Block title (background and text)
\setbeamercolor{block title alerted}{fg=mDarkTeal, bg=mLightBrown!45!yellow!45}
\setbeamercolor{block title example}{fg=magenta!10!black, bg=mLightGreen!70}
%%% 2- Block body (background)
\setbeamercolor{block body alerted}{bg=mLightBrown!25}
\setbeamercolor{block body example}{bg=mLightGreen!15}
%%%%%%%%%%%%%%%%%%%%%%%%%%%
\usepackage[absolute,overlay]{textpos}
%\usepackage[grid=true,gridcolor=Maroon,subgridcolor=gray,gridunit=pt,texcoord]{eso-pic} %場所決めのためのgrid表示
%%%%%%%%%%%%%%%%%%%%%%%%%%%
%% さまざまなアイコン
%%%%%%%%%%%%%%%%%%%%%%%%%%%
%\usepackage{fontawesome}
\usepackage{fontawesome5}
\usepackage{figchild}
\usepackage{twemojis}
\usepackage{utfsym}
\usepackage{bclogo}
\usepackage{marvosym}
\usepackage{fontmfizz}
\usepackage{pifont}
\usepackage{phaistos}
\usepackage{worldflags}
\usepackage{jigsaw}
\usepackage{tikzlings}
\usepackage{tikzducks}
\usepackage{scsnowman}
\usepackage{epsdice}
\usepackage{halloweenmath}
\usepackage{svrsymbols}
\usepackage{countriesofeurope}
\usepackage{tipa}
\usepackage{manfnt}
%%%%%%%%%%%%%%%%%%%%%%%%%%%
\usepackage{tikz}
\usetikzlibrary{calc,patterns,decorations.pathmorphing,backgrounds}
\usepackage{tcolorbox}
\usepackage{tikzpeople}
\usepackage{circledsteps}
\usepackage{xcolor}
\usepackage{amsmath}
\usepackage{booktabs}
\usepackage{chronology}
\usepackage{signchart}
%%%%%%%%%%%%%%%%%%%%%%%%%%%
%% 場合分け
%%%%%%%%%%%%%%%%%%%%%%%%%%%
\usepackage{cases}
%%%%%%%%%%%%%%%%%%%%%%%%%%
\usepackage{pdfpages}
%%%%%%%%%%%%%%%%%%%%%%%%%%%
%% 音声リンク表示
\newcommand{\myaudio}[1]{\href{#1}{\faVolumeUp}}
%%%%%%%%%%%%%%%%%%%%%%%%%%
%% \myAnch{<名前>}{<色>}{<テキスト>}
%% 指定のテキストを指定の色の四角枠で囲み, 指定の名前をもつTikZの
%% ノードとして出力する. 図には remember picture 属性を付けている
%% ので外部から参照可能である.
\newcommand*{\myAnch}[3]{%
  \tikz[remember picture,baseline=(#1.base)]
    \node[draw,rectangle,line width=1pt,#2] (#1) {\normalcolor #3};
}
%%%%%%%%%%%%%%%%%%%%%%%%%%
%% \myEmph コマンドの定義
%%%%%%%%%%%%%%%%%%%%%%%%%%
%\newcommand{\myEmph}[3]{%
%    \textbf<#1>{\color<#1>{#2}{#3}}%
%}
\usepackage{xparse} % xparseパッケージの読み込み
\NewDocumentCommand{\myEmph}{O{} m m}{%
    \def\argOne{#1}%
    \ifx\argOne\empty
        \textbf{\color{#2}{#3}}% オプション引数が省略された場合
    \else
        \textbf<#1>{\color<#1>{#2}{#3}}% オプション引数が指定された場合
    \fi
}
%%%%%%%%%%%%%%%%%%%%%%%%%%%
%%%%%%%%%%%%%%%%%%%%%%%%%%%
%% 文末の上昇イントネーション記号 \myRisingPitch
%% 通常のイントネーション \myDownwardPitch
%% https://note.com/dan_oyama/n/n8be58e8797b2
%%%%%%%%%%%%%%%%%%%%%%%%%%%
\newcommand{\myRisingPitch}{
\begin{tikzpicture}[scale=0.3,baseline=0.3]
\draw[->,>=stealth] (0,0) to[bend right=45] (1,1);
\end{tikzpicture}
}
\newcommand{\myDownwardPitch}{
\begin{tikzpicture}[scale=0.3,baseline=0.3]
\draw[->,>=stealth] (0,1) to[bend left=45] (1,0);
\end{tikzpicture}
}
%%%%%%%%%%%%%%%%%%%%%%%%%%%%
%\AtBeginSection[%
%]{%
%  \begin{frame}[plain]\frametitle{授業の流れ}
%     \tableofcontents[currentsection]
%   \end{frame}%
%}

\usepackage{pxrubrica}
%%%%%%%%%%%%%%%%%%%%%%%%%%%
\title{English is fun.}
\subtitle{I have a book written in English.}
\author{}
\institute[]{}
\date[]

%%%%%%%%%%%%%%%%%%%%%%%%%%%%
%% TEXT
%%%%%%%%%%%%%%%%%%%%%%%%%%%%
\begin{document}

\begin{frame}[plain]
  \titlepage
\end{frame}

\section*{授業の流れ}
\begin{frame}[plain]
  \frametitle{授業の流れ}
  \tableofcontents
\end{frame} 
%%%%%%%%%%%%%%%%%%%%%%%%%%%%
\section*{名詞を修飾する過去分詞}
\section{過去分詞 $+$ 名詞}
%%%%%%%%%%%%%%%%
 \begin{frame}[plain,t]{p.p. $+$ 名詞}
  \begin{enumerate}
   \item \begin{enumerate}
	  \item<1-> break --- broke --- broken%
\hfill{}{\scriptsize 原形 --- 過去形 --- 過去分詞}
	  \item<2-> He broke the window.
	  \item<3-> The window was broken by him.\hfill\visible<4->{\scriptsize \text{受け身} $=$ \Circled{\,\,be動詞 $+$ 過去分詞\,\,}}
	 \end{enumerate}
   \item \begin{enumerate}
	  \item<5-> Look at \fbox{the window}.%
\hfill\visible<6->{\scriptsize \fbox{$\text{冠詞}+\text{名詞}$}}
	  \item<7-> Look at \fbox{the old window}.%
\hfill\visible<8->{\scriptsize \fbox{$\text{冠詞}+\text{形容詞}+\text{名詞}$}}
	  \item<9-> Look at \fbox{the \myEmph[8-]{Maroon}{broken} window.}%
\hfill\visible<10->{\scriptsize \fbox{$\text{冠詞}+\text{過去分詞}+\text{名詞}$}}
	 \end{enumerate}
  \end{enumerate}

\vspace{8pt}

\begin{block}<11->{Topic for Today}\small
\Circled[fill color = white]{\,\,過去分詞\,\,}\,\,が名詞を修飾することがあります%
\hfill{\scriptsize つまり「形容詞」の役目をしています}

\begin{itemize}\setbeamertemplate{items}[square]\small
 \item \,\,\Circled[fill color = white]{\,\,過去分詞\,\,}\,\,は「~された」という\kenten{受け身}の意味\\
\hfill{\scriptsize \Circled[fill color = white]{\,\,現在分詞\,\,}\,\,は\kenten{進行}(~している)の意味でしたね}
 \item \,\,\Circled[fill color = white]{\,\,過去分詞\,\,}\,\,を\,\,\Circled[fill color = white]{\,\,p.p.\,\,}\,\,と表記することがあります
 \end{itemize}
     \end{block}

\vspace*{-12pt}

\hfill{\tiny 0253}\,{\scriptsize \myaudio{./audio/022_N_pp_01.mp3}}
\end{frame}
%%%%%%%%%%%%%%%%%%%%%%%%%%%
\begin{frame}[plain,t]{Exercises}
 
{\small 日本語の意味になるよう(~~~~~~)の語を並べ替えましょう}%
\hfill{\tiny 02362}\,{\scriptsize \myaudio{./audio/022_N_pp_01b.mp3}}
 
\begin{enumerate}
 \item  ( a / has / toy / little brother / my / broken ) .\hfill{\small 弟は壊れたおもちゃを持っている。}\\
My little brother has a broken toy.
 \item  ( money / found / stolen / the police / the )\hfill{\small 警察は盗まれたお金を見つけた。}\\
The police found the stolen money.
 \item  ( yesterday / used / he / a / car / bought ) .\hfill{\small 彼はきのう中古車を買った。}\\
He bought a used car yesterday.
 \item  ( boiled / I / a / ate / breakfast / egg ) this morning.\hfill{\small 私は朝食にゆで卵を食べた。}\\
I ate a boiled egg this morning.
 \item  ( like / potatoes / baked / I  ) very much.\hfill{\small 私は焼いたポテトが大好きだ。}\\
I like baked potatoes very much.
\end{enumerate}
\end{frame}
%%%%%%%%%%%%%%%%%%%%%%%%%%%
\section{名詞 $+$ 過去分詞}
%%%%%%%%%%%%%%%%%%%%%%%%%%
\begin{frame}[plain,t]{名詞 $+$ p.p. \ldots}
 \begin{enumerate}
  \item \begin{enumerate}
	 \item<1-> The man is my father.\hfill\visible<2->{\scriptsize  \fbox{$\text{冠詞}+\text{名詞}$}}
	 \item<3-> The singing man is my father.\hfill\visible<4->{\scriptsize \fbox{$\text{冠詞}+\text{---ing}+\text{名詞}$}}
	 \item<5-> The man singing over there is my father.\hfill\visible<6->{\scriptsize $\text{名詞}\longleftarrow$\fbox{$\text{\,---ing\,}+\text{ほかの語句}$}}
	\end{enumerate}
  \item \begin{enumerate}
	 \item<6-> Look at the window.\hfill\visible<7->{\scriptsize  \fbox{$\text{冠詞}+\text{名詞}$}}
	 \item<8-> Look at the broken window.\hfill\visible<9->{\scriptsize \fbox{$\text{冠詞}+\text{p.p.}+\text{名詞}$}}
	 \item<10-> Look at the window broken in the storm last night.\hfill\visible<11->{\scriptsize $\text{名詞}\longleftarrow$\fbox{$\text{\,p.p.\,}+\text{ほかの語句}$}}
	\end{enumerate}
 \end{enumerate}

\vspace{8pt}

\begin{block}<12->{Topic for Today}\small
\Circled[fill color = white]{\,\,p.p.\,\,}\,\,がほかの語句を伴うときは、
後ろから前の名詞を修飾します%

\begin{itemize}\setbeamertemplate{items}[square]\small
 \item \Circled[fill color = white]{\,\,p.p.\,\,}\,\, $\longrightarrow \text{\,\,名詞}$\hfill{\scriptsize the \textbf{broken} window}
 \item $\text{名詞\,\,}\longleftarrow$\,\,\Circled[fill color = white]{\,\,p.p.\,\,+\text{\,\,ほかの語句\,\,}}\hfill{\scriptsize the window \textbf{broken} in the storm last night}
 \end{itemize}
     \end{block}

\vspace{-12pt}

\hfill{\tiny 0243}\,{\scriptsize \myaudio{./audio/022_N_pp_02.mp3}}
\end{frame}
%%%%%%%%%%%%%%%%%%%%%%%%%%%%%%%%
\begin{frame}[plain]{Exercises}

{\small 日本語の意味になるよう(~~~~~~)内の語句を並べ替えましょう}%
\hfill{\tiny 0214}\,{\scriptsize \myaudio{./audio/022_N_pp_03.mp3}}
 \begin{enumerate}
  \item {\small デイブはドイツ製の新車を購入した。}%
\hfill{\scriptsize bought \textipa{/b\'O:t/} buyの過去形}\\
	Dave ( new / a / in / made / bought / car ) Germany.\\
	\visible<2->{Dave bought a new car made in Germany.}
  \item {\small 彼らはヘレンが調理した食事を楽しんだ。}\hfill{\scriptsize meal \textipa{/m\'\i :l/} 食事}\\
	They ( by / cooked / enjoyed / the meal ) Helen.\\
	\visible<3->{They enjoyed the meal cooked by Helen.}
  \item {\small これはパティが撮影した有名な写真です。}\\
	This ( by / picture / famous / is / taken / a ) Pattie.\\
	\visible<4->{This is a famous picture taken by Pattie.}
  \item {\small 彼女は英語で書かれたおもしろい本を読んだ}。\\
	She ( an / English / written / interesting / read / in / book ).\\
	\visible<5->{She read an interesting book written in English.}%
\hfill{}{\scriptsize readの過去形read \textipa{/r\'ed/}}
 \end{enumerate}

\begin{textblock*}{0.4\linewidth}(350pt,120pt)
\visible<1->{
\includegraphics[width=.65\textwidth]{./images/couple.jpg}
}
\end{textblock*}


\end{frame}
%%%%%%%%%%%%%%%%%%%%%%%%%%%%%
\section{まとめ}
\begin{frame}[plain,t]{まとめ1}
 
\begin{block}{名詞を修飾する過去分詞}\setbeamertemplate{items}[square]\small

\begin{itemize}\setbeamertemplate{items}[square]\small
 \item \Circled[fill color = white]{\,\,過去分詞\,\,}\,\,は「~される」という受け身の意味\\
\hfill{\scriptsize \Circled[fill color = white]{\,\,現在分詞\,\,}\,\,は\kenten{進行}(~している)の意味でしたね}
 \item \Circled[fill color = white]{\,\,過去分詞\,\,}\,\,が名詞を修飾することがあります\\
\hfill{\scriptsize つまり「形容詞」の役目をしています}
 \item 過去分詞のことを\Circled[fill color = white]{\,\,p.p.\,\,}\,\,と表記することがあります
 \end{itemize}
     \end{block}

\begin{textblock*}{0.4\linewidth}(320pt,150pt)
\visible<1->{\begin{tikzpicture}
\bear[
scale=1,
speech={\tiny  ---ingとp.p.},
signpost=\scalebox{.5}{
\parbox{2.5cm}{\color{black}
\centering ---ingは進行p.p.は受け身}},
signcolour= brown!70!gray,
signback=white!80!brown
]
\end{tikzpicture}}
\end{textblock*}
\end{frame}
%%%%%%%%%%%%%%%%%%%
\begin{frame}[plain]{まとめ2}

\begin{block}{名詞を修飾する過去分詞の位置}\small


\begin{itemize}\setbeamertemplate{items}[square]\small
\setlength{\itemsep}{5pt}
 \item \Circled[fill color = white]{\,\,\,\,p.p.\,\,\,\,}\,\,が単独のときは、前から後ろの名詞を修飾します

\hspace{50pt}\Circled[fill color = white]{\,\,\,\,p.p.\,\,\,\,}\,\, $\longrightarrow \text{\,\,名詞}$\hfill{\scriptsize the \textbf{broken} window}\\
\hfill{\scriptsize a \textbf{broken} toy}\\
\hfill{\scriptsize the \textbf{stolen} money}\\
\hfill{\scriptsize a \textbf{used} car}\\
\hfill{\scriptsize \textbf{boiled} egg}\\
\hfill{\scriptsize \textbf{baked} potatoes}\\

 \item \Circled[fill color = white]{\,\,p.p.\,\,}\,\,がほかの語句を伴うときは、
後ろから前の名詞を修飾します

\hspace{50pt}$\text{名詞\,\,}\longleftarrow$\,\,\Circled[fill color = white]{\,\,\,\,p.p.\,\,+\text{\,\,\,\,ほかの語句\,\,\,\,}}\hfill{\scriptsize the window \textbf{broken} in the storm last night}\\
\hfill{\scriptsize a new car \textbf{made} in Germany}\\
\hfill{\scriptsize the meal \textbf{cooked} byHelen}\\
\hfill{\scriptsize a famouse picture \textbf{taken} by Pattie}\\
\hfill{\scriptsize an interesting book \textbf{written} in English}\\
 \end{itemize}
     \end{block}
\end{frame}
%%%%%%%%%%%%%%%%%%%



\end{document}
%%%%%%%%%%%%%%%%%%%
the {\tikz[remember picture, baseline=(a.base)]\node (a){broken};}  {\tikz[remember picture, baseline=(b.base)]\node (b){window};} 
\begin{tikzpicture}[remember picture, overlay]
 \draw[->] (a.north) to[out=30, in=160] (b.north);
\end{tikzpicture}

