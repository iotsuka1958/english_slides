\documentclass[aspectratio=169,xcolor={dvipsnames,table}]{beamer}
\usepackage[no-math,deluxe,haranoaji]{luatexja-preset}
\renewcommand{\kanjifamilydefault}{\gtdefault}
\renewcommand{\emph}[1]{{\upshape\bfseries #1}}
\usetheme{metropolis}
\metroset{block=fill}
\setbeamertemplate{navigation symbols}{}
\setbeamertemplate{blocks}[rounded][shadow=false]
\usecolortheme[rgb={0.7,0.2,0.2}]{structure}
%%%%%%%%%%%%%%%%%%%%%%%%%%
%% Change alert block colors
%%% 1- Block title (background and text)
\setbeamercolor{block title alerted}{fg=mDarkTeal, bg=mLightBrown!45!yellow!45}
\setbeamercolor{block title example}{fg=magenta!10!black, bg=mLightGreen!60}
%%% 2- Block body (background)
\setbeamercolor{block body alerted}{bg=mLightBrown!25}
\setbeamercolor{block body example}{bg=mLightGreen!15}
%%%%%%%%%%%%%%%%%%%%%%%%%%%
\usepackage[absolute,overlay]{textpos}
\usepackage[grid=true,gridcolor=Maroon,subgridcolor=gray,gridunit=pt,texcoord]{eso-pic} %場所決めのためのgrid表示
%%%%%%%%%%%%%%%%%%%%%%%%%%%
%% さまざまなアイコン
%%%%%%%%%%%%%%%%%%%%%%%%%%%
%\usepackage{fontawesome}
\usepackage{fontawesome5}
\usepackage{figchild}
\usepackage{twemojis}
\usepackage{utfsym}
\usepackage{bclogo}
\usepackage{marvosym}
\usepackage{fontmfizz}
\usepackage{pifont}
\usepackage{phaistos}
\usepackage{worldflags}
\usepackage{jigsaw}
\usepackage{tikzlings}
\usepackage{tikzducks}
\usepackage{scsnowman}
\usepackage{epsdice}
\usepackage{halloweenmath}
\usepackage{svrsymbols}
\usepackage{countriesofeurope}
\usepackage{tipa}
\usepackage{manfnt}
%%%%%%%%%%%%%%%%%%%%%%%%%%%
\usepackage{tikz}
\usetikzlibrary{calc,patterns,decorations.pathmorphing,backgrounds}
\usepackage{tcolorbox}
\usepackage{tikzpeople}
\usepackage{circledsteps}
\usepackage{xcolor}
\usepackage{amsmath}
\usepackage{booktabs}
\usepackage{chronology}
\usepackage{signchart}
%%%%%%%%%%%%%%%%%%%%%%%%%%%
%% 場合分け
%%%%%%%%%%%%%%%%%%%%%%%%%%%
\usepackage{cases}
%%%%%%%%%%%%%%%%%%%%%%%%%%
\usepackage{pdfpages}
%%%%%%%%%%%%%%%%%%%%%%%%%%%
%% 音声リンク表示
\newcommand{\myaudio}[1]{\href{#1}{\faVolumeUp}}
%%%%%%%%%%%%%%%%%%%%%%%%%%
%% \myAnch{<名前>}{<色>}{<テキスト>}
%% 指定のテキストを指定の色の四角枠で囲み, 指定の名前をもつTikZの
%% ノードとして出力する. 図には remember picture 属性を付けている
%% ので外部から参照可能である.
\newcommand*{\myAnch}[3]{%
  \tikz[remember picture,baseline=(#1.base)]
    \node[draw,rectangle,line width=1pt,#2] (#1) {\normalcolor #3};
}
%%%%%%%%%%%%%%%%%%%%%%%%%%
%% \myEmph コマンドの定義
%%%%%%%%%%%%%%%%%%%%%%%%%%
%\newcommand{\myEmph}[3]{%
%    \textbf<#1>{\color<#1>{#2}{#3}}%
%}
\usepackage{xparse} % xparseパッケージの読み込み
\NewDocumentCommand{\myEmph}{O{} m m}{%
    \def\argOne{#1}%
    \ifx\argOne\empty
        \textbf{\color{#2}{#3}}% オプション引数が省略された場合
    \else
        \textbf<#1>{\color<#1>{#2}{#3}}% オプション引数が指定された場合
    \fi
}
%%%%%%%%%%%%%%%%%%%%%%%%%%%
%%%%%%%%%%%%%%%%%%%%%%%%%%%
%% 文末の上昇イントネーション記号 \myRisingPitch
%% 通常のイントネーション \myDownwardPitch
%% https://note.com/dan_oyama/n/n8be58e8797b2
%%%%%%%%%%%%%%%%%%%%%%%%%%%
\newcommand{\myRisingPitch}{
\begin{tikzpicture}[scale=0.3,baseline=0.3]
\draw[->,>=stealth] (0,0) to[bend right=45] (1,1);
\end{tikzpicture}
}
\newcommand{\myDownwardPitch}{
\begin{tikzpicture}[scale=0.3,baseline=0.3]
\draw[->,>=stealth] (0,1) to[bend left=45] (1,0);
\end{tikzpicture}
}
%%%%%%%%%%%%%%%%%%%%%%%%%%%%
%\AtBeginSection[%
%]{%
%  \begin{frame}[plain]\frametitle{授業の流れ}
%     \tableofcontents[currentsection]
%   \end{frame}%
%}

\usepackage{pxrubrica}
%%%%%%%%%%%%%%%%%%%%%%%%%%%
\title{English is fun.}
\subtitle{I know a girl who speaks Spanish.}
\author{}
\institute[]{}
\date[]

%%%%%%%%%%%%%%%%%%%%%%%%%%%%
%% TEXT
%%%%%%%%%%%%%%%%%%%%%%%%%%%%
\begin{document}

\begin{frame}[plain]
  \titlepage
\end{frame}

\section*{授業の流れ}
\begin{frame}[plain]
  \frametitle{授業の流れ}
  \tableofcontents
\end{frame}

\section{関係代名詞who \textipa{/hu:/}}
%%%%%%%%%%%%%%%%%%%%%%%%%%%%
\begin{frame}[plain,t]{人 $\leftarrow$ \fbox{who +V ...}}
 \begin{enumerate}
  \item<1-> I know the girl.\hfill{\scriptsize the girl}
  \item<2-> I know the girl in the room.%
        \hfill{}\visible<3->{\scriptsize the girl $\longleftarrow$\,\fbox{\,in the room\,}}
  \item<4-> I know the girl playing in the room.%
        \hfill{}\visible<5->{\scriptsize the girl $\longleftarrow$\,\fbox{\,playing in the room\,}}
  \item<6-> I know the girl \textbf{who} speaks two languages.%
        \hfill{}\visible<7->{\scriptsize the girl $\longleftarrow$\,\fbox{\,who speaks two languages\,}}
 \end{enumerate}

\vspace{50pt}

\begin{block}<8->{Topics for Today}
「人」について、後ろから詳しく説明するとき\textbf{who}をつかいます

\begin{itemize}\setbeamertemplate{items}[square]\small
 \item この\textbf{who}を「関係代名詞」といいます
% \item whoには動詞が続きます
 \item $\text{人\,\,}\longleftarrow$\,\,\Circled[fill color = white]{\,\,\textbf{who}\,\,+\text{\,\,V\,\,}\text{\,\,\ldots\,\,\,}}\hfill{\scriptsize the girl \textbf{who} speaks two languages}
 \end{itemize}
     \end{block}

\vspace{-10pt}

\hfill{\tiny 0202}\,{\scriptsize \myaudio{./audio/031_N_wh_V_01.mp3}}
\end{frame}
%%%%%%%%%%%%%%%%%%%%%%%%%%%%
\begin{frame}[plain,t]{図解}
 \begin{enumerate}
  \item<1-> I know \myAnch{a1}{Maroon}{\bfseries the girl}.
  \item<2-> \myAnch{a2}{NavyBlue}{\bfseries The girl} speaks two languages.\\[10pt]
	\hspace{6pt}\visible<3->{\myAnch{a3}{white}{\Circled[fill color = NavyBlue!40]{\,\,\textbf{who}\,\,}}}\visible<3->{speaks two languages}
  \item<4-> I know \myAnch{a4}{Maroon}{\bfseries the girl} \myAnch{a5}{black}{\Circled[fill color = NavyBlue!40]{\,\,\textbf{who}\,\,} speaks two languages}.
 \end{enumerate}

\begin{tikzpicture}[remember picture, overlay]
\visible<3->{\draw[->, thick, NavyBlue] (a2.south) to (a3.north);}
\coordinate (A5) at ($(a5) - (0,20pt)$); % 10pt below a1
\coordinate (A4) at ($(a4) - (0,20pt)$); % 10pt below a1
%\visible<5->{\draw[->, thick, orange] (a5.south) to (a4.north);}
 % Draw the arrow with right angles
 \visible<5->{\draw[->,black,thick] (a5) -- (A5) -- (A4) -- (a4);}
\end{tikzpicture}

\begin{block}<6->{Topics for Today}\small
関係代名詞の\textbf{who}は後ろから前の「人」について詳しく説明します(~する人)%
\mbox{}\hfill{\tiny 0139}\,{\scriptsize \myaudio{./audio/031_N_wh_V_02.mp3}}

\begin{enumerate}\small
 \item 2つの文で、同じ「人」を指している語句を見つけます
 \item 2つめを\,\,\Circled[fill color = white]{\,\,\textbf{who}\,\,}\,\,にかえて1つめの後ろにつなげて、2文を1文にまとめます
 \end{enumerate}

\hfill{\small 修飾される側の名詞{(\scriptsize ここでは\textbf{the girl})}を「先行詞」といいます}
     \end{block}

\end{frame}
%%%%%%%%%%%%%%%%%%%%%%%%%%%%%%%%%%
\begin{frame}[plain,t]{Exercises}

{\small \fbox{  }\,\,で囲まれた語を関係代名詞whoにかえて各2文を1文にまとめましょう}%
\hfill{\tiny 0148}\,{\scriptsize \myaudio{./audio/031_N_wh_V_03.mp3}}
\begin{enumerate}
 \item \begin{enumerate}
	\item I know {\bfseries the man}.
	\item \fbox{The man} came from Hawaii.
	\item \visible<2->{$\text{1.1}+\text{1.2}\rightarrow$\,\,\,I know the man who came from Hawaii.}
       \end{enumerate}
 \item \begin{enumerate}
	\item I met {\bfseries the teacher}.
	\item \fbox{The teacher} lived near my house.
	\item \visible<3->{$\text{2.1}+\text{2.2}\rightarrow$\,\,\,I met the teacher who lived near my house.}
       \end{enumerate}
 \item \begin{enumerate}
	\item {\bfseries The woman} is my mother.
	\item \fbox{The woman} can speak English well.
	\item \visible<4->{$\text{3.1}+\text{3.2}\rightarrow$\,\,\,The woman who can speak English well is my mother.}
       \end{enumerate}
\end{enumerate} 

\vspace{30pt}

\hfill{\small \fbox{  }\,\,で囲まれた語をwhoにかえて太字の語の後ろにつなげます}
\end{frame}
%%%%%%%%%%%%%%%%%%%%%%%%%%%%%%%%%%%
\begin{frame}[plain]{Exercises}

{\small 日本語の意味になるよう(~~~~~~)の語句を並べ替えましょう}
\hfill{\tiny 0215}\,{\scriptsize \myaudio{./audio/031_N_wh_V_04.mp3}}
 \begin{enumerate}
  \item わたしはテニスがとてもじょうずな少女を知っています。\\
	I ( a girl / tennis / who / know / plays ) very well.\\
	\visible<2->{I know a girl who plays tennis very well.}
  \item 彼には京都に住んでいるともだちがいます。\\
	He ( lives / has / who / in / lives / a friend ) Kyoto.\\
	\visible<3->{He has a friend who lives in Kyoto.}
  \item 公園で走っている男性はわたしの父親です。\\
	The man ( running / in / who / the park / is ) is my father.\\
	\visible<4->{The man who is running in the park is my father.}
  \item この詩を書いた少女はジェニファーだ。\\
	The ( who / this poem / wrote / girl ) is Jennifer.\\
	\visible<5->{The girl who wrote this poem is Jennifer.}
 \end{enumerate}
\end{frame}
%%%%%%%%%%%%%%%%%%%%%%%%%%%%
\section{関係代名詞which \textipa{/wItS/}}
%%%%%%%%%%%%%%%%%%%%%%%%%%%%
\begin{frame}[plain,t]{もの $\leftarrow$ \fbox{which +V ...}}
 \begin{enumerate}
  \item<1-> We went to the restaurant.\hfill{\scriptsize the restaurant}
  \item<2-> We went to the restaurant near the beach.%
        \hfill{}\visible<3->{\scriptsize the restaurant $\longleftarrow$\,\fbox{\,near the beach\,}}
  \item<4-> We went to the restaurant overlooking the beach.\hfill{\scriptsize overlook: 見下ろす}\\%
        \hfill{}\visible<5->{\scriptsize the restaurant $\longleftarrow$\,\fbox{\,overlooking the beach\,}}
  \item<6-> We went to the restaurant \textbf{which} was famous for seafood.\\%
        \hfill{}\visible<7->{\scriptsize the restaurant $\longleftarrow$\,\fbox{\,which was famous for seafood\,}}
 \end{enumerate}

\vspace{20pt}

\begin{block}<8->{Topics for Today}\small
「もの」について、後ろから詳しく説明するとき\textbf{which}をつかいます(~するもの)

\begin{itemize}\setbeamertemplate{items}[square]\small
 \item この\textbf{which}を「関係代名詞」といいます
 %\item whoには動詞が続きます
 \item もの\,\,$\longleftarrow$\,\,\Circled[fill color = white]{\,\,\textbf{which}\,\,$+$\,\,\textbf{V}\,\,\,\,\textbf{\ldots}\,\,\,}\hfill{\scriptsize the restaurant \textbf{which} is famous for seafood}
 \end{itemize}
     \end{block}
\vspace{-10pt}

\hfill{\tiny 0211}\,{\scriptsize \myaudio{./audio/031_N_wh_V_05.mp3}}

\end{frame}
%%%%%%%%%%%%%%%%%%%%%%%%%%%%
\begin{frame}[plain]{図解}
 \begin{enumerate}
  \item<1-> She works for \myAnch{a1}{Maroon}{\bfseries the factory}.
  \item<2-> \myAnch{a2}{NavyBlue}{\bfseries The factory} makes computers.\\[10pt]
	\hspace{7pt}\visible<3->{\myAnch{a3}{white}{\Circled[fill color = NavyBlue!40]{\,\,\textbf{which}\,\,}}}\visible<3->{makes computers}
  \item<4-> She works for \myAnch{a4}{Maroon}{\bfseries the factory} \myAnch{a5}{black}{\Circled[fill color = NavyBlue!40]{\,\,\textbf{which}\,\,} makes computers}.
 \end{enumerate}

\begin{tikzpicture}[remember picture, overlay]
\visible<3->{\draw[->, thick, NavyBlue] (a2.south) to (a3.north);}
\coordinate (A5) at ($(a5) - (0,20pt)$); % 10pt below a1
\coordinate (A4) at ($(a4) - (0,20pt)$); % 10pt below a1
%\visible<5->{\draw[->, thick, orange] (a5.south) to (a4.north);}
 % Draw the arrow with right angles
 \visible<5->{\draw[->,black,thick] (a5) -- (A5) -- (A4) -- (a4);}
\end{tikzpicture}

\begin{block}<6->{Topics for Today}
 関係代名詞の\textbf{which}は後ろから前の「もの」について詳しく説明します%
\hfill{\tiny 0143}\,{\scriptsize \myaudio{./audio/031_N_wh_V_06.mp3}}

\begin{enumerate}\small
 \item 2つの文で、同じ「もの」を指している語句を見つけます
 \item 2つめを\,\,\Circled[fill color = white]{\,\,\textbf{which}\,\,}\,\,にして、1つめの後ろにつなげて2文を1文にまとめます
 \end{enumerate}

\hfill{\small 修飾される側の名詞{(\scriptsize ここでは\textbf{the factory})}を「先行詞」といいます}
     \end{block}

\end{frame}
%%%%%%%%%%%%%%%%%%%%%%%%%%%%%%%%%%
\begin{frame}[plain,t]{Exercises}

{\small \fbox{  }\,\,で囲まれた語を関係代名詞whichにかえて各2文を1文にまとめましょう}%
\hfill{\scriptsize \myaudio{./audio/031_N_wh_V_07.mp3}}

\begin{enumerate}
 \item \begin{enumerate}
	\item This is {\bfseries the road}.
	\item \fbox{The road} leads to Rome.\hfill{\scriptsize lead to ~: (道が)~に通じている}
	\item \visible<2->{$\text{1.1}+\text{1.2}\rightarrow$\,\,\,This is the road which leads to Rome.\hfill{\scriptsize これはローマに通じる道だ}}
       \end{enumerate}
 \item \begin{enumerate}
	\item She bought {\bfseries the house}.\hfill{\scriptsize buy -- bought -- bought}
	\item \fbox{The house} had a big garden.
	\item \visible<3->{$\text{2.1}+\text{2.2}\rightarrow$\,\,\,She bought the house which had a big garden.\\
\hfill{\scriptsize 彼女は大きな庭のある家を買った}}
       \end{enumerate}
 \item \begin{enumerate}
	\item {\bfseries The book} is interesting.
	\item \fbox{The book} was written 20 years ago.\hfill{\scriptsize write -- wrote -- written}
	\item \visible<4->{$\text{3.1}+\text{3.2}\rightarrow$\,\,\,The book which was written 20 years ago is interesting.\\
\hfill{\scriptsize 20年前に書かれたその本はおもしろい}}
       \end{enumerate}
\end{enumerate} 

\hfill{\small \fbox{  }\,\,で囲まれた語をwhichにかえて太字の語の後ろにつなげます}%
\end{frame}
%%%%%%%%%%%%%%%%%%%%%%%%%%%%%%%%%%%
\begin{frame}[plain]{Exercises}

{\small 日本語の意味になるよう(~~~~~~)の語句を並べ替えましょう}%
\hfill{\tiny 0219}\,{\scriptsize \myaudio{./audio/031_N_wh_V_08.mp3}}
 \begin{enumerate}
  \item 彼女は賞を獲得した本を読んだ。%
	\hfill{\scriptsize win \textipa{/w\'In/} (賞などを)獲得する}\\
	She read ( the book / won / which ) the prize.\hfill{\scriptsize won \textipa{/w\'\textturnv n/} winの過去形}\\
	\visible<2->{She read the book which won the prize.}%
  \item 赤い屋根の家が見えますか。%
	\hfill{\scriptsize roof \textipa{/r\'u:f/} 屋根}\\
	Can you ( a house / a red roof / has / see / which ) ?\\
	\visible<3->{Can you see a house which has a red roof?}
  \item その庭に立っている木はとても高い。\\
	The tree ( the garden / stands / which / in / is ) very tall.\\
	\visible<4->{The tree which stands in the garden is very tall.}
  \item カナダで話されていることばは英語とフランス語です。%
	\hfill{\scriptsize language \textipa{/l\'\ae NgwIdZ/} 言語}\\
	The languages ( Canada /  are /  spoken /  which /  in ) are English and French\\
	\visible<5->{The languages which are spoken in Canada are English and French.} \end{enumerate}
\end{frame}
%%%%%%%%%%%%%%%%%%%%%%%%%%%%
\section{関係代名詞that \textipa{/D@t/}}
%%%%%%%%%%%%%%%%%%%%%%%%%%%
\begin{frame}[plain,t]{関係代名詞のthat}
 \begin{enumerate}
  \item<1-> わたしはテニスがうまい少女を知っています。%
\hfill{\tiny 0218}\,{\scriptsize \myaudio{./audio/031_N_wh_V_09.mp3}}
	\begin{enumerate}
	 \item<1-> I know a girl (~~~~~~) plays tennis well.
	 \item<2-> I know a girl \myEmph[2-]{Maroon}{who} plays tennis well.
	 \item<5-> I know a girl \myEmph[4-]{Maroon}{that} plays tennis well.
	\end{enumerate}
  \item<1-> 20年前に書かれたその本はおもしろい。
       \begin{enumerate}
	\item<1-> The book (~~~~~~~) was written 20 years ago is interesting.
	\item<3->  The book \myEmph[3-]{Maroon}{which} was written 20 years ago is interesting.
	\item<6->  The book \myEmph[3-]{Maroon}{that} was written 20 years ago is interesting.
       \end{enumerate}
 \end{enumerate}

\begin{block}<4->{Topics for Today}
\small
 

\begin{itemize}\setbeamertemplate{items}[square]
 \item<7-> 関係代名詞の\textbf{that}は先行詞が「人」でも「もの」でも使えます
\end{itemize}

\begin{enumerate}\small
 \item<4->  人\,\,$\longleftarrow$\,\,\Circled[fill color = white]{\,\,\textbf{who}\,\,$+$\,\,\textbf{V}\,\,\,\,\textbf{\ldots}\,\,\,}
 \item<4->  もの\,\,$\longleftarrow$\,\,\Circled[fill color = white]{\,\,\textbf{which}\,\,$+$ \,\,\textbf{V}\,\,\,\,\textbf{\ldots}\,\,\,}
 \item<7->  人・もの\,\,$\longleftarrow$\,\,\Circled[fill color = white]{\,\,\textbf{that}\,\,$+$ \,\,\textbf{V}\,\,\,\,\textbf{\ldots}\,\,\,}
 \end{enumerate}
     \end{block}

\end{frame}
%%%%%%%%%%%%%%%%%%%%%%%%%%%%
\begin{frame}[plain,label=ichiran]{人・もの $\leftarrow$ \fbox{that +V ...}}
 \begin{enumerate}
  \item I know the girl \alt<2->{{\bfseries that}}{{\bfseries who}} speaks two languages.%
\hfill{\tiny 0609}\,{\scriptsize \myaudio{./audio/031_N_wh_V_10.mp3}}
%  \item I know the man who came from Hawaii.
  \item I met the teacher \alt<2->{{\bfseries that}}{{\bfseries who}} lived near my house.
  \item He has a friend \alt<2->{{\bfseries that}}{{\bfseries who}} lives in Kyoto.
  \item The woman \alt<2->{{\bfseries that}}{{\bfseries who}} can speak English well is my mother.
%  \item I know a girl who plays tennis very well.
  \item The man \alt<2->{{\bfseries that}}{{\bfseries who}} is running in the park is my father.
  \item The girl \alt<2->{{\bfseries that}}{{\bfseries who}} wrote this poem is Jennifer.
  \item We went to the restaurant \alt<2->{{\bfseries that}}{{\bfseries which}} was famous for seafood.
  \item She works for the factory \alt<2->{{\bfseries that}}{{\bfseries which}} makes computers.
%  \item This is the road which leads to Rome.
  \item She bought the house \alt<2->{{\bfseries that}}{{\bfseries which}} had a big garden
%  \item The book which was written 20 years ago is interesting.
  \item She read the book \alt<2->{{\bfseries that}}{{\bfseries which}} won the prize.\hfill{\scriptsize won \textipa{/w\'2n/} win(賞などを獲得する)の過去形}
%  \item Can you see a house which has a red roof?
  \item The tree \alt<2->{{\bfseries that}}{{\bfseries which}} stands in the garden is very tall.
  \item The languages \alt<2->{{\bfseries that}}{{\bfseries which}} are spoken in Canada are English and French.
 \end{enumerate}
\end{frame}
%%%%%%%%%%%%%%%%%%%%%%%%%%%%%%%%%%%%
\section{格}
\subsection{主格}
%%%%%%%%%%%%%%%%%%%%%%%%%%%%%%%%%%%
\begin{frame}[plain,t]{主格の関係代名詞}

\begin{enumerate}
 \item<1-> I know a girl. The girl plays tennis well.
       \begin{enumerate}
	\item<2-> 第2文の主語と動詞はなんでしょう
	\item<3-> 第2文の主語はThe girlで動詞がplaysであることを確認しましょう
	\item<4-> 第2文のThe girlを関係代名詞whoにかえて1つの文にまとめてください
       \end{enumerate}
 \item<5-> I know a girl who plays tennis well.
       \begin{enumerate}
	\item<6-> 動詞playsの主語はなんでしょう
	\item<7-> The girlをwhoにして1つの文にまとめたのが、この文です
	\item<8-> このwhoは、直後に続く動詞playsの主語ということになります
       \end{enumerate}
\end{enumerate}


\vspace{25pt}


\begin{block}<9->{Topic for Today}
\small
\small
 \begin{itemize}\setbeamertemplate{items}[square]
  \item  直後に「動詞」が続く\textbf{who}, \textbf{which}, \textbf{that}を「\kenten{主格}の関係代名詞」といいます\\
\hfill{\scriptsize 後続する動詞に対して\kenten{主}語の働きをしているからです}\raisebox{0pt}{\dbend}\\
\mbox{}
\end{itemize}
     \end{block}

\end{frame}
%%%%%%%%%%%%%%%%%%%%%%%%%%%%%%%%%%%
\section{まとめ}
%%%%%%%%%%%%%%%%%%%%%%%%%%%%%%%%%%%
\begin{frame}[plain]{まとめ1}
 \begin{block}<1->{関係代名詞who \textipa{/hu:/}}

\begin{itemize}\setbeamertemplate{items}[square]\small
 \item 「人」について、後ろから詳しく説明するとき\textbf{who}をつかいます\\
\hfill{\scriptsize the girl \textbf{who} speaks two languages}
 \item 人\,\,$\longleftarrow$\,\,\Circled[fill color = white]{\,\,\textbf{who}\,\,$+$\,\,\textbf{V}\,\,\,\,\textbf{\ldots}\,\,\,}
 \item この\textbf{who}を「関係代名詞」といいます
% \item whoには動詞が続きます
 \end{itemize}
     \end{block}


\end{frame}
%%%%%%%%%%%%%%%%%%%%%%%%%%%%
\begin{frame}[plain]{まとめ2}
\begin{block}<1->{関係代名詞which \textipa{/wItS/}}
 \begin{itemize}\setbeamertemplate{items}[square]
 \item 関係代名詞の\textbf{which}は後ろから前の「もの」について詳しく説明します%
 \item もの\,\,$\longleftarrow$\,\,\Circled[fill color = white]{\,\,\textbf{which}\,\,$+$\,\,\textbf{V}\,\,\,\,\textbf{\ldots}\,\,\,}\hfill{\scriptsize the book \textbf{which} was written 20 years ago}
 \end{itemize}

%\begin{enumerate}\small
% \item 2つの文で、同じ「もの」を指している語句を見つけます
% \item 2つめを\,\,\Circled[fill color = white]{\,\,which\,\,}\,\,にして、1つめの後ろにつなげてまとめます
% \end{enumerate}

\hfill{\small 修飾される側の名詞{(\scriptsize ここではthe book)}を「先行詞」といいます}
     \end{block}

\end{frame}
%%%%%%%%%%%%%%%%%%%%%%%%%%%%
\begin{frame}[plain]{まとめ3}

\begin{block}<1->{関係代名詞that \textipa{/D@t/}}
\small
 \begin{itemize}\setbeamertemplate{items}[square]
  \item 関係代名詞の\textbf{that}は先行詞が「人」でも「もの」でも使えます
\end{itemize}

\begin{enumerate}\small
 \item  人\,\,$\longleftarrow$\,\,\Circled[fill color = white]{\,\,\textbf{who}\,\,$+$\,\,\textbf{V}\,\,\,\,\textbf{\ldots}\,\,\,}
 \item  もの\,\,$\longleftarrow$\,\,\Circled[fill color = white]{\,\,\textbf{which}\,\,$+$\,\,\textbf{V}\,\,\,\,\ldots\,\,\,}
 \item  人・もの\,\,$\longleftarrow$\,\,\Circled[fill color = white]{\,\,\textbf{that}\,\,$+$\,\,\textbf{V}\,\,\,\,\ldots\,\,\,}
 \end{enumerate}
     \end{block}

\begin{block}<1->{関係代名詞の格}
\small
 \begin{itemize}\setbeamertemplate{items}[square]
  \item  直後に「動詞」が続く\textbf{who}, \textbf{which}, \textbf{that}を「\kenten{主格}の関係代名詞」といいます\\
\hfill{\scriptsize 後続する動詞に対して\kenten{主}語の働きをしているからです}\raisebox{0pt}{\dbend}\\
\mbox{}
\end{itemize}
     \end{block}

\end{frame}
%%%%%%%%%%%%%%%%%%%%%%%%%%%%%%%%%%%
\againframe{ichiran}
\end{document}

