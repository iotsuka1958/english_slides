\documentclass[aspectratio=169,xcolor={dvipsnames,table}]{beamer}
\usepackage[no-math,deluxe,haranoaji]{luatexja-preset}
\renewcommand{\kanjifamilydefault}{\gtdefault}
\renewcommand{\emph}[1]{{\upshape\bfseries #1}}
\usetheme{metropolis}
\metroset{block=fill}
\setbeamertemplate{navigation symbols}{}
\setbeamertemplate{blocks}[rounded][shadow=false]
\usecolortheme[rgb={0.7,0.2,0.2}]{structure}
%%%%%%%%%%%%%%%%%%%%%%%%%%
%% Change alert block colors
%%% 1- Block title (background and text)
\setbeamercolor{block title alerted}{fg=mDarkTeal, bg=mLightBrown!45!yellow!45}
\setbeamercolor{block title example}{fg=magenta!10!black, bg=mLightGreen!60}
%%% 2- Block body (background)
\setbeamercolor{block body alerted}{bg=mLightBrown!25}
\setbeamercolor{block body example}{bg=mLightGreen!15}
%%%%%%%%%%%%%%%%%%%%%%%%%%%
%%%%%%%%%%%%%%%%%%%%%%%%%%%
%% さまざまなアイコン
%%%%%%%%%%%%%%%%%%%%%%%%%%%
%\usepackage{fontawesome}
\usepackage{fontawesome5}
\usepackage{figchild}
\usepackage{twemojis}
\usepackage{utfsym}
\usepackage{bclogo}
\usepackage{marvosym}
\usepackage{fontmfizz}
\usepackage{pifont}
\usepackage{phaistos}
\usepackage{worldflags}
\usepackage{jigsaw}
\usepackage{tikzlings}
\usepackage{tikzducks}
\usepackage{scsnowman}
\usepackage{epsdice}
\usepackage{halloweenmath}
\usepackage{svrsymbols}
\usepackage{countriesofeurope}
\usepackage{tipa}
\usepackage{manfnt}
%%%%%%%%%%%%%%%%%%%%%%%%%%%
\usepackage{tikz}
\usetikzlibrary{calc,patterns,decorations.pathmorphing,backgrounds}
\usepackage{tcolorbox}
\usepackage{tikzpeople}
\usepackage{circledsteps}
\usepackage{xcolor}
\usepackage{amsmath}
\usepackage{booktabs}
\usepackage{chronology}
\usepackage{signchart}
%%%%%%%%%%%%%%%%%%%%%%%%%%%
%% 場合分け
%%%%%%%%%%%%%%%%%%%%%%%%%%%
\usepackage{cases}
%%%%%%%%%%%%%%%%%%%%%%%%%%
\usepackage{pdfpages}
%%%%%%%%%%%%%%%%%%%%%%%%%%%
%% 音声リンク表示
\newcommand{\myaudio}[1]{\href{#1}{\faVolumeUp}}
%%%%%%%%%%%%%%%%%%%%%%%%%%
%% \myAnch{<名前>}{<色>}{<テキスト>}
%% 指定のテキストを指定の色の四角枠で囲み, 指定の名前をもつTikZの
%% ノードとして出力する. 図には remember picture 属性を付けている
%% ので外部から参照可能である.
\newcommand*{\myAnch}[3]{%
  \tikz[remember picture,baseline=(#1.base)]
    \node[draw,rectangle,line width=1pt,#2] (#1) {\normalcolor #3};
}
%%%%%%%%%%%%%%%%%%%%%%%%%%
%% \myEmph コマンドの定義
%%%%%%%%%%%%%%%%%%%%%%%%%%
%\newcommand{\myEmph}[3]{%
%    \textbf<#1>{\color<#1>{#2}{#3}}%
%}
\usepackage{xparse} % xparseパッケージの読み込み
\NewDocumentCommand{\myEmph}{O{} m m}{%
    \def\argOne{#1}%
    \ifx\argOne\empty
        \textbf{\color{#2}{#3}}% オプション引数が省略された場合
    \else
        \textbf<#1>{\color<#1>{#2}{#3}}% オプション引数が指定された場合
    \fi
}
%%%%%%%%%%%%%%%%%%%%%%%%%%%
%%%%%%%%%%%%%%%%%%%%%%%%%%%
%% 文末の上昇イントネーション記号 \myRisingPitch
%% 通常のイントネーション \myDownwardPitch
%% https://note.com/dan_oyama/n/n8be58e8797b2
%%%%%%%%%%%%%%%%%%%%%%%%%%%
\newcommand{\myRisingPitch}{
\begin{tikzpicture}[scale=0.3,baseline=0.3]
\draw[->,>=stealth] (0,0) to[bend right=45] (1,1);
\end{tikzpicture}
}
\newcommand{\myDownwardPitch}{
\begin{tikzpicture}[scale=0.3,baseline=0.3]
\draw[->,>=stealth] (0,1) to[bend left=45] (1,0);
\end{tikzpicture}
}
%%%%%%%%%%%%%%%%%%%%%%%%%%%%
%\AtBeginSection[%
%]{%
%  \begin{frame}[plain]\frametitle{授業の流れ}
%     \tableofcontents[currentsection]
%   \end{frame}%
%}

\usepackage{pxrubrica}
%%%%%%%%%%%%%%%%%%%%%%%%%%%
\title{English is fun.}
\subtitle{I wish I had a lot of money.}
\author{}
\institute[]{}
\date[]

%%%%%%%%%%%%%%%%%%%%%%%%%%%%
%% TEXT
%%%%%%%%%%%%%%%%%%%%%%%%%%%%
\begin{document}

\begin{frame}[plain]
  \titlepage
\end{frame}

\section*{授業の流れ}
\begin{frame}[plain]
  \frametitle{授業の流れ}
  \tableofcontents
\end{frame}

\section{仮定法}
\subsection{仮定法とは}
\begin{frame}[plain]{仮定法とは}
 \large

\begin{enumerate}
 \item<1-> I don't have a lot of money.
 \item<2-> I wish I \myEmph[2-]{Maroon}{had} a lot of money.\hfill\visible<3->{{\scriptsize お金がたくさんあったらいいのになあ}}
\end{enumerate}

\vfill

\begin{block}<4->{Topics for Today}
\small

\begin{itemize}
 \item \kenten{現在}の\kenten{事実}の\kenten{反対}のことをあらわすときは、
現在のことでも\kenten{過去形}を使います
 \item これを「仮定法過去」といいます
\end{itemize}
\end{block}

\mbox{}\hfill\visible{\myaudio{./audio/041_mood_intro_01.mp3}} 
\end{frame}
%%%%%%%%%%%%%%%%%%%
\begin{frame}[plain]{もうひとつ}
 \begin{enumerate}
  \item<1-> If I \myEmph[1-]{NavyBlue}{am} hungry, I  eat a sandwich.%
\hfill\visible<2->{{\scriptsize おなかがすいたときは、いつもサンドイッチを食べます}}
  \item<3-> If I \myEmph[3-]{Maroon}{were} hungry, I would eat the sandwich.%
\hfill\visible<4->{{\scriptsize 空腹ならば、そのサンドイッチを食べるのですが}}\\
\hfill\visible<4->{{\scriptsize (いまはおなかがすいていない)}}
 \end{enumerate}

\vfill

\begin{block}<5->{be動詞の仮定法過去}
\small

\begin{itemize}
 \item be動詞の仮定法過去は、主語にかかわらずwereをもちいます
\end{itemize}
\end{block}
\mbox{}\hfill\visible{\myaudio{./audio/041_mood_intro_02.mp3}} 

\end{frame}
%%%%%%%%%%%%%%%%%%%%%%%%%%%%%
\begin{frame}[plain]{Exercises}
 つぎの英文がどういう意味か、考えましょう\mbox{}\hfill\visible{\myaudio{./audio/041_mood_intro_03.mp3}} 



\begin{enumerate}
 \item \begin{enumerate}
	\item If I have money, I will buy the book. 
	\item If I \myEmph[2-]{Maroon}{had} money, I would buy the book.
       \end{enumerate}
 \item \begin{enumerate}
	\item If you have time, you can watch television.
	\item If you \myEmph[2-]{Maroon}{had} time, you could watch television.
       \end{enumerate}
 \item \begin{enumerate}
	\item I hope it will be fine tomorrow.
	\item I wish it \myEmph[2-]{Maroon}{were} fine today.
       \end{enumerate}
 \item \begin{enumerate}
	\item I am sorry I don't have a car.
	\item I wish I \myEmph[2-]{Maroon}{had} a car.
       \end{enumerate}
\end{enumerate}
\end{frame}
%%%%%%%%%%%%%%%%%%%%%%%%%
\begin{frame}[plain]{まとめ}
 \begin{block}{仮定法過去}
\small

\begin{itemize}\setbeamertemplate{items}[square]
 \item \kenten{現在}の\kenten{事実}の\kenten{反対}のことをあらわすときは、
現在のことでも\kenten{過去形}を使います\\
\hfill{}$\longrightarrow$ これを\,\Circled[fill color = white]{\,仮定法過去\,}\,といいます\\
\hfill{}I wish I \textcolor{Maroon}{{\bfseries had}} a lot of money.{\scriptsize たくさんお金があればいいのに}\\
\hfill{}If I \textcolor{Maroon}{{\bfseries had}} a lot of money, I would buy a big house.
\hfill{\scriptsize たくさんお金があったら大きな家を買うのに}\\
\hfill{\scriptsize (現実には、たくさんお金を持っていない)}\\
\mbox{}
 \item be動詞の\,\Circled[fill color = white]{\,仮定法過去\,}\,は、主語にかかわらずwereです\\
\hfill{}I wish I \textcolor{Maroon}{{\bfseries were}} rich.{\scriptsize お金持ちだったらいいのに}\\
\hfill{}If I \textcolor{Maroon}{{\bfseries were}} rich, I would buy a big house.{\scriptsize お金持ちだったら大きな家を買うのに}\\
\hfill{\scriptsize (現実には、私は金持ちではない)}
\end{itemize}
\end{block}
\end{frame}
\end{document}
