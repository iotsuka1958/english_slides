\documentclass[aspectratio=169,xcolor={dvipsnames,table}]{beamer}
\usepackage[no-math,deluxe,haranoaji]{luatexja-preset}
\renewcommand{\kanjifamilydefault}{\gtdefault}
\renewcommand{\emph}[1]{{\upshape\bfseries #1}}
\usetheme{metropolis}
\metroset{block=fill}
\setbeamertemplate{navigation symbols}{}
\setbeamertemplate{blocks}[rounded][shadow=false]
\usecolortheme[rgb={0.7,0.2,0.2}]{structure}
%%%%%%%%%%%%%%%%%%%%%%%%%%
%% Change alert block colors
%%% 1- Block title (background and text)
\setbeamercolor{block title alerted}{fg=mDarkTeal, bg=mLightBrown!45!yellow!45}
\setbeamercolor{block title example}{fg=magenta!10!black, bg=mLightGreen!60}
%%% 2- Block body (background)
\setbeamercolor{block body alerted}{bg=mLightBrown!25}
\setbeamercolor{block body example}{bg=mLightGreen!15}
%%%%%%%%%%%%%%%%%%%%%%%%%%%
\usepackage[absolute,overlay]{textpos}
\usepackage[grid=true,gridcolor=Maroon,subgridcolor=gray,gridunit=pt,texcoord]{eso-pic} %場所決めのためのgrid表示
%%%%%%%%%%%%%%%%%%%%%%%%%%%
%% さまざまなアイコン
%%%%%%%%%%%%%%%%%%%%%%%%%%%
%\usepackage{fontawesome}
\usepackage{fontawesome5}
\usepackage{figchild}
\usepackage{twemojis}
\usepackage{utfsym}
\usepackage{bclogo}
\usepackage{marvosym}
\usepackage{fontmfizz}
\usepackage{pifont}
\usepackage{phaistos}
\usepackage{worldflags}
\usepackage{jigsaw}
\usepackage{tikzlings}
\usepackage{tikzducks}
\usepackage{scsnowman}
\usepackage{epsdice}
\usepackage{halloweenmath}
\usepackage{svrsymbols}
\usepackage{countriesofeurope}
\usepackage{tipa}
\usepackage{manfnt}
%%%%%%%%%%%%%%%%%%%%%%%%%%%
\usepackage{tikz}
\usetikzlibrary{calc,patterns,decorations.pathmorphing,backgrounds}
\usepackage{tcolorbox}
\usepackage{tikzpeople}
\usepackage{circledsteps}
\usepackage{xcolor}
\usepackage{amsmath}
\usepackage{booktabs}
\usepackage{chronology}
\usepackage{signchart}
%%%%%%%%%%%%%%%%%%%%%%%%%%%
%% 場合分け
%%%%%%%%%%%%%%%%%%%%%%%%%%%
\usepackage{cases}
%%%%%%%%%%%%%%%%%%%%%%%%%%
\usepackage{pdfpages}
%%%%%%%%%%%%%%%%%%%%%%%%%%%
%% 音声リンク表示
\newcommand{\myaudio}[1]{\href{#1}{\faVolumeUp}}
%%%%%%%%%%%%%%%%%%%%%%%%%%
%% \myAnch{<名前>}{<色>}{<テキスト>}
%% 指定のテキストを指定の色の四角枠で囲み, 指定の名前をもつTikZの
%% ノードとして出力する. 図には remember picture 属性を付けている
%% ので外部から参照可能である.
\newcommand*{\myAnch}[3]{%
  \tikz[remember picture,baseline=(#1.base)]
    \node[draw,rectangle,line width=1pt,#2] (#1) {\normalcolor #3};
}
%%%%%%%%%%%%%%%%%%%%%%%%%%
%% \myEmph コマンドの定義
%%%%%%%%%%%%%%%%%%%%%%%%%%
%\newcommand{\myEmph}[3]{%
%    \textbf<#1>{\color<#1>{#2}{#3}}%
%}
\usepackage{xparse} % xparseパッケージの読み込み
\NewDocumentCommand{\myEmph}{O{} m m}{%
    \def\argOne{#1}%
    \ifx\argOne\empty
        \textbf{\color{#2}{#3}}% オプション引数が省略された場合
    \else
        \textbf<#1>{\color<#1>{#2}{#3}}% オプション引数が指定された場合
    \fi
}
%%%%%%%%%%%%%%%%%%%%%%%%%%%
%%%%%%%%%%%%%%%%%%%%%%%%%%%
%% 文末の上昇イントネーション記号 \myRisingPitch
%% 通常のイントネーション \myDownwardPitch
%% https://note.com/dan_oyama/n/n8be58e8797b2
%%%%%%%%%%%%%%%%%%%%%%%%%%%
\newcommand{\myRisingPitch}{
\begin{tikzpicture}[scale=0.3,baseline=0.3]
\draw[->,>=stealth] (0,0) to[bend right=45] (1,1);
\end{tikzpicture}
}
\newcommand{\myDownwardPitch}{
\begin{tikzpicture}[scale=0.3,baseline=0.3]
\draw[->,>=stealth] (0,1) to[bend left=45] (1,0);
\end{tikzpicture}
}
%%%%%%%%%%%%%%%%%%%%%%%%%%%%
%\AtBeginSection[%
%]{%
%  \begin{frame}[plain]\frametitle{授業の流れ}
%     \tableofcontents[currentsection]
%   \end{frame}%
%}

\usepackage{pxrubrica}
%%%%%%%%%%%%%%%%%%%%%%%%%%%
\title{English is fun.}
\subtitle{I wish I were rich.}
\author{}
\institute[]{}
\date[]

%%%%%%%%%%%%%%%%%%%%%%%%%%%%
%% TEXT
%%%%%%%%%%%%%%%%%%%%%%%%%%%%
\begin{document}

\begin{frame}[plain]
  \titlepage
\end{frame}

\section*{授業の流れ}
\begin{frame}[plain]
  \frametitle{授業の流れ}
  \tableofcontents
\end{frame}

%%%%%%%%%%%%%%%%%%%%%%%%%
\section{I wish S $+$ V($=$ 仮定法過去)}
\subsection{~だったらいいのに}
%%%%%%%%%%%%%%%%%%%%%%%%%
\begin{frame}[plain,t]{I wish}
\large
 \begin{enumerate}
  \item<1-> \textbf{I wish} I \textcolor{Maroon}{\bfseries had} a lot of money.\hfill{}{\scriptsize wish \textipa{/w\'IS/}~であればいいと思う}
  \item<2-> \textbf{I wish} I \textcolor{Maroon}{\bfseries were} rich.
\end{enumerate}

\vspace{80pt}

\begin{block}<3->{~だったらいいのに}
\small
\begin{itemize}\setbeamertemplate{items}[square]
 \item {\bfseries I wish S $+$ V}($=$ {\scriptsize 仮定法過去}) \ldots .
 \item be動詞の仮定法過去は主語にかかわらず\textbf{were}です
\end{itemize}
\end{block}

\hfill{\tiny 0112}\,{{\scriptsize \myaudio{./audio/042_mood_i_wish_01.mp3}}} 
\end{frame}
%%%%%%%%%%%%%%%%%%%%%
\begin{frame}[plain]{Exericises}

 {\small あたえられた日本語の意味になるよう(~~~~~~~~)内の語句を並べ替えましょう}\mbox{}\hfill{\tiny 0204}\,{{\scriptsize \myaudio{./audio/042_mood_i_wish_02.mp3}}} 


\begin{enumerate}
 \item 歌がうまかったらいいのに{\scriptsize (実際は歌がうまくないので残念)}\\
( I / good / I / singer / were / wish / a )\hfill{}I wish I were a good singer.
 \item きょう天気がよければなあ{\scriptsize (実際はいい天気ではなくて残念)}\\
( it / I / fine / were / wish ) today.\hfill{}I wish it were fine today.
 \item 答えがわかっていればいいのに{\scriptsize (実際は答えを知らなくて残念)}\\
( I / knew / answer / I / the / wish )\hfill{}I wish I knew the answer.
 \item 英語をうまく話せたらいいのに{\scriptsize (実際はうまく英語を話せないので残念)}\\
( wish / well / English / could / I / I / could )\\
\hfill{}I wish I could speak English well.
\end{enumerate}
\end{frame}
%%%%%%%%%%%%%%%%%%%%
\begin{frame}[plain]{Exercises}

 {\small あたえられた日本語の意味になるよう適切なものを選びましょう}
\hfill{\tiny 0140}\,{{\scriptsize \myaudio{./audio/042_mood_i_wish_03.mp3}}} 

\begin{enumerate}
 \item ジェニファーがわたしの友人であったらなあ{\scriptsize (実際には知り合いでなくて残念)}\\
I wish Jennifer ( is / are / \alt<2->{\Circled{\bfseries were}}{were} ) my friend.
 \item 携帯電話を持っていたらなあ{\scriptsize (実際には持っていなくて残念)}\\
I wish I ( were / have / \alt<3->{\Circled{\bfseries had}}{had} )  a cellphone.%
\hfill{\scriptsize cellphone \textipa{/s\'elf\`oUn/} 携帯電話}
 \item {\scriptsize (実際には日本を離れられないが)}君とスペインに行けたらいいのに\\
I wish I ( can / \alt<4>{\Circled{\bfseries could}}{could} ) go to Spain with you.
\end{enumerate}
\end{frame}
%%%%%%%%%%%%%%%%%%%%%%%%%%%%%%%%
\section{まとめ}
%%%%%%%%%%%%%%%%%%%%%%%%%%%%%%%
\begin{frame}[plain]{まとめ}
 
\begin{block}{~だったらいいのに}
\small
\begin{itemize}\setbeamertemplate{items}[square]
 \item {\bfseries I wish S $+$ V}($=$ {\scriptsize 仮定法過去}) \ldots .\\
\hspace*{40pt}{\bfseries I wish I} \textcolor{Maroon}{\bfseries had} a computer.
 \item be動詞の仮定法過去は主語にかかわらずwereです\\
\hspace*{40pt}{\bfseries I wish I} \textcolor{Maroon}{\bfseries were} a pianist
\end{itemize}
\end{block}



\end{frame}
\end{document}

