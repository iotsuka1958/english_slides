\documentclass[aspectratio=169,xcolor={dvipsnames,table}]{beamer}
\usepackage[no-math,deluxe,expert,haranoaji]{luatexja-preset}
\usepackage{luatexja-otf}
\renewcommand{\kanjifamilydefault}{\gtdefault}
\renewcommand{\emph}[1]{{\upshape\bfseries #1}}
\usetheme{metropolis}
\metroset{block=fill}
\setbeamertemplate{navigation symbols}{}
\setbeamertemplate{blocks}[rounded][shadow=false]
\usecolortheme[rgb={0.7,0.2,0.2}]{structure}
%%%%%%%%%%%%%%%%%%%%%%%%%%%
\usepackage{media9}
%%%%%%%%%%%%%%%%%%%%%%%%%%%
%% さまざまなアイコン
%%%%%%%%%%%%%%%%%%%%%%%%%%%
\usepackage{fontawesome}
\usepackage{figchild}
\usepackage{twemojis}
\usepackage{utfsym}
\usepackage{bclogo}
\usepackage{marvosym}
\usepackage{fontmfizz}
\usepackage{pifont}
\usepackage{phaistos}
\usepackage{worldflags}
%%%%%%%%%%%%%%%%%%%%%%%%%%%
\usepackage{tikz}
\usetikzlibrary{backgrounds}
\usepackage{tcolorbox}
\usepackage{tikzpeople}
\usepackage{circledsteps}
\usepackage{xcolor}
\usepackage{amsmath}
\usepackage{booktabs}
\usepackage{chronology}
\usepackage{signchart}
\usepackage{tipa}
%%%%%%%%%%%%%%%%%%%%%%%%%%%
%% 場合分け
\usepackage{cases}
%%%%%%%%%%%%%%%%%%%%%%%%%%%
% \myAnch{<名前>}{<色>}{<テキスト>}
% 指定のテキストを指定の色の四角枠で囲み, 指定の名前をもつTikZの
% ノードとして出力する. 図には remeber picture 属性を付けている
% ので外部から参照可能である.
\newcommand*{\myAnch}[3]{%
  \tikz[remember picture,baseline=(#1.base)]
    \node[draw,rectangle,#2] (#1) {\normalcolor #3};
}
%%%%%%%%%%%%%%%%%%%%%%%%%%%%
%% 音声リンク表示
\newcommand{\myaudio}[1]{\href{#1}{\faVolumeUp}}
%%%%%%%%%%%%%%%%%%%%%%%%%%%
% \myEmph コマンドの定義
%\newcommand{\myEmph}[3]{%
%    \textbf<#1>{\color<#1>{#2}{#3}}%
%}
\usepackage{xparse} % xparseパッケージの読み込み
\NewDocumentCommand{\myEmph}{O{} m m}{%
    \def\argOne{#1}%
    \ifx\argOne\empty
        \textbf{\color{#2}{#3}}% オプション引数が省略された場合
    \else
        \textbf<#1>{\color<#1>{#2}{#3}}% オプション引数が指定された場合
    \fi
}
%%%%%%%%%%%%%%%%%%%%%%%%%%%
%% 文末の上昇イントネーション記号 \myRisingPitch
%% 通常のイントネーション \myDownwardPitch
%% https://note.com/dan_oyama/n/n8be58e8797b2
%%%%%%%%%%%%%%%%%%%%%%%%%%%
\newcommand{\myRisingPitch}{
\begin{tikzpicture}[scale=0.3,baseline=0.3]
\draw[->,>=stealth] (0,0) to[bend right=45] (1,1);
\end{tikzpicture}
}
\newcommand{\myDownwardPitch}{
\begin{tikzpicture}[scale=0.3,baseline=0.3]
\draw[->,>=stealth] (0,1) to[bend left=45] (1,0);
\end{tikzpicture}
}
%%%%%%%%%%%%%%%%%%%%%%%%%%%
\title{English is fun.}
\subtitle{I have known him for eight years.}
  \author{}
\institute[]{}
\date[]

%%%%%%%%%%%%%%%%%%%%%%%%%%%%
%% TEXT
%%%%%%%%%%%%%%%%%%%%%%%%%%%%
\begin{document}
\begin{frame}[plain]
  \titlepage
\end{frame}

\section*{授業の流れ}
\begin{frame}[plain]
  \frametitle{授業の流れ}
  \tableofcontents
\end{frame}

\section{現在完了--継続--}
\subsection{現在完了とは(復習)}
%%%%%%%%%%%%%%%%%%%%%%%%%%%%
\begin{frame}[plain]{現在完了とは(復習)}
 \begin{enumerate}
 \item Jane \textcolor{NavyBlue}{\bfseries has stayed} in London for six years.
 \item I \textcolor{NavyBlue}{\bfseries have watched} the movie three times.
 \item Bob \textcolor{NavyBlue}{\bfseries has lost} his bag.
\end{enumerate}



 \begin{exampleblock}{Topic for Today}
\small
\begin{itemize}
 \item  $\text{現在完了(}=\textcolor{NavyBlue}{\text{\bfseries have} + \text{\bfseries 過去分詞\,)}}$%
は「過去と現在にまたがる表現」です
 \item 主語が三人称単数のときは \textcolor{NavyBlue}{\bfseries has $+$ 過去分詞}
\end{itemize}
      \end{exampleblock}
\mbox{}\hfill\myaudio{./audio/012_have_pp_keizoku_01.mp3}
\end{frame}
%%%%%%%%%%%%%%%%%%%%%%%%
\subsection{現在完了--継続--}
%%%%%%%%%%%%%%%%%%%%%%
\begin{frame}[plain]{現在完了--継続--}
 
\visible<1->{\begin{enumerate}
 \item $\left\{\begin{tabular}{rl}
(A)& Jane \textcolor{Maroon}{\bfseries stayed} in London two years ago.\hspace{7\zw}{\small stay \textipa{/st\'eI/} 滞在する}\\
(B)& Jane \textcolor{NavyBlue}{\bfseries has stayed} in London for six years.
\end{tabular}
\right.$
\end{enumerate}}

\visible<2->{\signchart[width=10,height=.5]{,,,,,{\textcolor{Maroon}{stayed}},,今}{,,,}}
\visible<3->{\signchart[width=10,height=.5]{,6年前,,,{\textcolor{NavyBlue}{has stayed}},,,今}{,,,}}

\begin{tikzpicture}[overlay]
 %\draw[gray!50] (0,0) grid (12,5);
 %\fill[ForestGreen!70,opacity=.5] (11.57,3.75) circle [radius=.25];
\visible<2->{ \fill[Maroon!70,opacity=.5] (8.675,2.66) circle [radius=.25];}
\visible<3->{ \draw[NavyBlue!70,line width=6pt,opacity=.7] (4.2,1.21) -- (10.9,1.21);}
\end{tikzpicture}

\visible<4->{主語が三人称単数のときは has $+$ 過去分詞}

\visible<5->{for six yearsは「期間」を表す表現です}\mbox{}\hfill\myaudio{./audio/012_have_pp_keizoku_02.mp3}
\end{frame}
%%%%%%%%%%%%%%%%%%%%%%
\begin{frame}[plain]{継続}
\visible<1->{\begin{enumerate}\setcounter{enumi}{1}
 \item $\left\{\begin{tabular}{rl}
(A)& I \textcolor{Maroon}{\bfseries am} sick today.\hspace{9\zw}{\small sick \textipa{/s\'Ik/} 病気だ、具合が悪い}\\
%(B)& I \textcolor{NavyBlue}{\bfseries have been} sick since yeaterday.\\
		(B)& I \textcolor{NavyBlue}{\bfseries have been} sick for six days.\hspace{3\zw}{\small been \textipa{/bIn/} beの過去分詞}
\end{tabular}
\right.$
\end{enumerate}}


\visible<2->{\signchart[width=10,height=.5]{,,,,,,,{\textcolor{Maroon}{am}}}{,,,}}
%\visible<3->{\signchart[width=10,height=.5]{,,,,,,{\textcolor{NavyBlue}{have been}},}{,,,}}

\visible<3->{\signchart[width=10,height=.5]{,6日前,,,{\textcolor{NavyBlue}{have been}},,,今}{,,,}}

\begin{tikzpicture}[overlay]
% \draw[gray!50] (0,0) grid (12,5);
 \visible<2-> {\fill[Maroon!70,opacity=.5] (10.9,2.7) circle [radius=.25];}
%\visible<3->{ \draw[NavyBlue!70,line width=7pt,opacity=.7] (9.8,2.65) -- (10.9,2.65);}
\visible<3->{ \draw[NavyBlue!70,line width=7pt,opacity=.7] (4.2,1.21) -- (10.9,1.21);}
\end{tikzpicture}

\vspace{-20pt}

\visible<4->{for a weekは「期間」を表します}\mbox{}\hfill\myaudio{./audio/012_have_pp_keizoku_03.mp3}

%\visible<5->{since 〜は「起点」を表す表現です。「〜から」}


\end{frame}
%%%%%%%%%%%%%%%%%%%%%%%%%%%%%%%
\begin{frame}[plain]{継続}
\visible<1->{\begin{enumerate}\setcounter{enumi}{2}
 \item $\left\{\begin{tabular}{rl}
(A)& I \textcolor{Maroon}{\bfseries know} him.\\
(B)& I \textcolor{NavyBlue}{\bfseries have known} him since 1995.\\
%(B)& I \textcolor{NavyBlue}{\bfseries have known} sick for six days.\hspace{3\zw}{\small known \textipa{/n\'oUn/} knowの過去分詞}
\end{tabular}
\right.$
\end{enumerate}}


\visible<2->{\signchart[width=10,height=.5]{,,,,,,,{\textcolor{Maroon}{know}}}{,,,}}
%\visible<3->{\signchart[width=10,height=.5]{,,,,,,{\textcolor{NavyBlue}{have been}},}{,,,}}

\visible<3->{\signchart[width=10,height=.5]{,1995,,,{\textcolor{NavyBlue}{have known}},,,2025}{,,,}}

\begin{tikzpicture}[overlay]
 %\draw[gray!50] (0,0) grid (12,5);
 \visible<2-> {\fill[Maroon!70,opacity=.5] (10.9,2.5) circle [radius=.25];}
%\visible<3->{ \draw[NavyBlue!70,line width=7pt,opacity=.7] (9.8,2.65) -- (10.9,2.65);}
\visible<3->{ \draw[NavyBlue!70,line width=7pt,opacity=.7] (4.2,1.21) -- (10.9,1.21);}
\end{tikzpicture}

\vspace{-20pt}

%\visible<4->{for a weekは「期間」を表します。}
\visible<5->{since 〜は「起点」を表す表現です。「〜から」}\mbox{}\hfill{\scriptsize \myaudio{./audio/012_have_pp_keizoku_04.mp3}}
\end{frame}
%%%%%%%%%%%%%%%%%%%%%%%%%%%%%%%%%%%%%%%
\begin{frame}<1-11>[plain]{Exercises}
あたえられた日本語の意味になるよう、(~~~~~~~~)内に適当な語を補いましょう\mbox{}\hfill\myaudio{./audio/012_have_pp_keizoku_05.mp3}

\begin{enumerate}
 \item {\small われわれはゲームをこの3時間楽しんだ。}\hfill{}{\scriptsize enjoy: 楽しむ}\\
We \alt<1>{(~~\phantom{have}~~)}{(~~have~~)} enjoyed the game for three hours. \item {\small ビリーはピアノを2年間練習しています。}\hfill{}{\scriptsize practice: 練習する}\\
Billy \alt<1-2>{(~~\phantom{has}~~)}{(~~has~~)} practiced the piano \alt<1-3>{(~~\phantom{for}~~)}{(~~for~~)} two years.
 \item {\small わたしは今朝からずっと具合が悪い。}\hfill{}{\scriptsize be sick: 病気だ}\\
I have \alt<1-4>{(~~\phantom{been}~~)}{(~~been~~)} sick \alt<1-5>{(~~\phantom{since}~~)}{(~~since~~)} this morning.\hfill{\small since $+$ 名詞}
 \item {\small 彼が子供のころから、わたしは彼を知っています。}\\
I have \alt<1-6>{(~~\phantom{known}~~)}{(~~known~~)} him \alt<1-7>{(~~\phantom{since}~~)}{(~~since~~)} he was a child.\hfill{\small since \Circled{\.\.S $+$ V\,\,}}
\end{enumerate}

\begin{exampleblock}<9->{Topics for Today}\small
\begin{itemize}\setbeamertemplate{items}[square]
 \item \visible<10->{for 〜は期間を表します。for two days, for three months, for a long time \ldots}
 \item \visible<11->{since 〜は起点を表します。since yesterday, since 2000, since  I was a child \ldots}

\end{itemize}
      \end{exampleblock}
\end{frame}
%%%%%%%%%%%%%%%%%%%%%%%%%%%%%%%%%%%%%%
\begin{frame}[plain]{Exercises}

あたえられた日本語の意味になるよう、(~~~~~~~~)内の語句を並べ替えましょう。なお、先頭の語は大文字ではじめてください\mbox{}\hfill{\scriptsize \myaudio{./audio/012_have_pp_keizoku_06.mp3}}

 \begin{enumerate}
  \item {\small 彼は先週の金曜日から病気だ。}
( has / sick /  he / been ) since last  Friday.\\
\visible<2->{He has been sick since last Friday.}
  \item 
{\small 彼らはこの町に10年間住んでいます。}\\
( lived / they / this town / have / in ) for ten years.\\
\visible<3->{They have lived in this town for ten years.}
  \item 
{\small 昨日からずっと雨が降っています。}
( been / it / rainy / has ) since yesterday.\\
\visible<4->{It has been rainy since yesterday.}
  \item 
{\small 私は彼女を2000年から知っている}。
( her / known / I / have ) since 2000.\\
\visible<5->{I have known her since 2000.}
 \end{enumerate}
\end{frame}
%%%%%%%%%%%%%%%%%%%%%%%%%%%%%
\section{否定文}
\begin{frame}[plain]{否定文}
 \Large

\visible<1->{I \,\myAnch{aux1}{NavyBlue}{have known} her for a long time.}


\visible<2->{I \,\myAnch{aux2}{NavyBlue}{have} \textcolor{Maroon}{\bfseries not} \,\myAnch{aux3}{NavyBlue}{known} her for a long time.}

\visible<3->{%
\begin{exampleblock}{Topics for Today}\small
\begin{itemize}\setbeamertemplate{items}[square]
 \item  \visible<3->{現在完了の否定は、$\text{have} + \text{not} + \text{過去分詞}$\\
\mbox{}\hfill%
\visible<4->{{\scriptsize 主語が三人称単数のときは$\text{has} + \text{not} + \text{過去分詞}$}}}
 \item  \visible<5->{haven't\,($=\text{have not}$)、hasn't\,($=\text{has not}$) という短縮形もあります}
% \item  She\textcolor{orange}{'s not} ということもあります
\end{itemize}
      \end{exampleblock}
}
\mbox{}\hfill\myaudio{./audio/012_have_pp_keizoku_07.mp3}
\end{frame}
%%%%%%%%%%%%%%%%%%%%%%%%%%%
\section{疑問文と答え方}
\begin{frame}[plain]{疑問文と答え方}
 \Large

\mbox{}\phantom{~You~~~~~}%
\visible<1->{You \myAnch{aux1}{NavyBlue}{have} known her for a long time.}


\vspace{10pt}

\visible<2->{\myAnch{aux2}{NavyBlue}{Have} you known her for a long time\fcolorbox{NavyBlue}{white}{?}}
\mbox{}\hfill{\scriptsize \myaudio{./audio/012_have_pp_keizoku_08.mp3}}


\mbox{}\hfill%
\visible<4->{$\left\{\begin{tabular}{l}
	 Yes, I have.\\
No, I have not($=$ haven't)
	\end{tabular}\right\}$}

\begin{tikzpicture}[remember picture,overlay]
\visible<3->{\draw[NavyBlue, line width=3pt,opacity=.5,->] (aux1.south) to[out=-90,in=90] (aux2.north);}
\end{tikzpicture}

\vfill

\begin{exampleblock}<5->{Topics for Today}\small
\begin{itemize}\setbeamertemplate{items}[square]
 \item \visible<6->{現在完了の疑問文は、$\text{Have} + \text{S} + \text{過去分詞\ldots\,?}$}\\
\mbox{}\hspace{0\zw}%
\hfill\visible<7->{\scriptsize 主語が三人称単数のときは$\text{Has} + \text{S} + \text{過去分詞\ldots{\,?}}$}
 \item \visible<8->{答え方はYes, S $+$ have.またはNo, S $+$ have not($=$haven't).}\\
\hfill\visible<9->{\scriptsize 主語が三人称単数のときはYes, S $+$ has.またはNo, S $+$ has not($=$hasn't).}
\end{itemize}
      \end{exampleblock}
\end{frame}
%%%%%%%%%%%%%%%%%%%%%%%%%%%%%%%%%%%
\begin{frame}[plain]{Exercises}
 
あたえられた日本語の意味になるよう、(~~~~~~~~) 内の語句を並べ替えましょう。なお、先頭の語は大文字ではじめてください\mbox{}\hfill\myaudio{./audio/012_have_pp_keizoku_09.mp3}


\begin{enumerate}
 \item {\small わたしたちは昨年からジョンにあっていません。}\\
We  ~( not / seen / have ) John since last year.\\
\visible<2->{We have not seen John since last year.}
 \item  {\small わたしは長いあいだ寿司を食べていません。\scalebox{3}{\twemoji{sushi}}\\
I ~~( eaten /  have / not )~~sushi~~( }long / a / fot /time ).\\
\visible<3->{I have not eaten sushi for a long time.}
 \item  {\small あなたのおばさんは先月からパリに滞在していますか}?\scalebox{3}{\twemoji{baguette bread}}\bcdfrance\\
( stayed /  your aunt / has ) in Paris (last / month / since) ? \\
\visible<4->{Has your aunt stayed in Paris since last month?}
\end{enumerate}
\end{frame}
%%%%%%%%%%%%%%%%%%%%%%%%%%%%%%%%%%%%%%%%%
\section{How long have you 〜?}
%%%%%%%%%%%%%%%%%%%%%%%%%%%%%%%%%%%%%%%%%
\begin{frame}[plain]{How long have you 〜?}

\visible<1->{You have stayed in Paris \myAnch{long1}{Maroon}{for two weeks}.}
 
\vspace{10pt}

\hfill\visible<2->{cf. Have you stayed in Paris for two weeks?}

\vspace{10pt}

\visible<4->{\myAnch{long2}{Maroon}{\bfseries How long} have you stayed in Paris?}%
\hfill\visible<6->{{\small どれくらいの期間}}

\begin{tikzpicture}[remember picture,overlay]
 \visible<5->{\draw[Maroon,->,line width=3pt,opacity=.5] (long1.south) to[out=-90,in=90] node[sloped,above,text=black,font=\tiny,pos=.667]{先頭へ} (long2.north);}
\end{tikzpicture}

\begin{exampleblock}<7->{Topics for Today}\small
\begin{itemize}\setbeamertemplate{items}[square]
 \item \visible<8->{$\text{How long have} + \text{S} + \text{過去分詞 \ldots ?}$は「継続している期間」をたずねる表現}\\
\mbox{}\hfill\visible<9->{「どのくらいのあいだ(期間)〜ですか」}\\
\hfill{\scriptsize 主語が三人称単数のときは$\text{How long has} + \text{S} + \text{過去分詞 \ldots ?}$}
 \item How longの後ろは疑問文の語順

\end{itemize}
      \end{exampleblock}
\mbox{}\hfill\myaudio{./audio/012_have_pp_keizoku_10.mp3}

\end{frame}
%%%%%%%%%%%%%%%%%%%%%%%%%%%%%%%
\begin{frame}[plain]{Exercises}

あたえられた日本語の意味になるよう、(~~~~~~~~) 内の語句を並べ替えましょう。なお、先頭の語は大文字ではじめてください\mbox{}\hfill\myaudio{./audio/012_have_pp_keizoku_11.mp3}

\vspace{-5pt}

\begin{enumerate}
 \item どのくらいのあいだスペイン語を勉強してきましたか?\\
( studied / long / have / you / how ) Spanish?\\
\visible<2->{How long have you studied Spanish?}
 \item 彼女はローマにどのくらいのあいだ住んでいますか?\\
( has / lived / long / she / how ) in Rome?\\
\visible<3->{How long has she lived in Rome?}
 \item これまでどのくらいのあいだこの会社で働いてきましたか?\\
( worked /  you / long / how / have ) at this company?\\
\visible<4->{How long have you worked at this company?}
 \item ジェニファーを知ってどのくらいの期間になりますか?\\
( long / known / you / have / how ) Jennifer?\\
\visible<5->{How long have you known Jennifer?}
\end{enumerate}
\end{frame}
%%%%%%%%%%%%%%%%%%%%%%
\section{まとめ}
\begin{frame}<1-12>[plain]{まとめ}

 \begin{exampleblock}<1->{Topics for Today \textcolor{black}{\mdseries --「継続」を表す現在完了--}}
\small

\begin{itemize}\setbeamertemplate{items}[square]
 \item[]<2-> 基本: $\text{現在完了(}=\textcolor{NavyBlue}{\text{\bfseries have} + \text{\bfseries 過去分詞\,})}$は「過去と現在にまたがる表現」

 \item \visible<3->{「継続」を表す用法があります}
  \item  \visible<4->{否定は、$\text{have} + \text{not} + \text{過去分詞}$%\\
%\mbox{}\hspace{9\zw}%
\hfill\visible<5->{{\scriptsize 主語が三人称単数のときは$\text{has} + \text{not} + \text{過去分詞}$}}}\\
\mbox{}\hfill\visible<6->{{\scriptsize 短縮形はhaven't($=\text{have not}$)、hasn't($=\text{has not}$)}}
% \item  She\textcolor{orange}{'s not} ということもあります

 \item \visible<7->{疑問文は、$\text{Have} + \text{S} + \text{過去分詞\ldots\,?}$}\hfill\visible<8->{{\scriptsize 主語が三人称単数のときは$\text{Has} + \text{S} + \text{過去分詞\ldots{\,?}}$}}
 \item \visible<9->{答え方はYes, S $+$ have.またはNo, S $+$ have not($=$haven't).}\\
\mbox{}\hfill\visible<10->{{\scriptsize 主語が三人称単数のときはYes, S $+$ has.またはNo, S $+$ has not($=$hasn't).}}

 \item \visible<11->{$\text{How long have} + \text{S} + \text{過去分詞 \ldots ?}$は「継続している期間」をたずねる表現}\\%
%\mbox{}\hfill\visible<7->{「どのくらいのあいだ(期間)〜ですか」   }\\
\mbox{}\hfill\visible<12->{{\scriptsize 主語が三人称単数のときは$\text{How long has} + \text{S} + \text{過去分詞 \ldots ?}$}}
\end{itemize}
      \end{exampleblock}
\end{frame}

\end{document}

