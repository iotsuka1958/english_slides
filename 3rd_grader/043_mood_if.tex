\documentclass[aspectratio=169,xcolor={dvipsnames,table}]{beamer}
\usepackage[no-math,deluxe,haranoaji]{luatexja-preset}
\renewcommand{\kanjifamilydefault}{\gtdefault}
\renewcommand{\emph}[1]{{\upshape\bfseries #1}}
\usetheme{metropolis}
\metroset{block=fill}
\setbeamertemplate{navigation symbols}{}
\setbeamertemplate{blocks}[rounded][shadow=false]
\usecolortheme[rgb={0.7,0.2,0.2}]{structure}
%%%%%%%%%%%%%%%%%%%%%%%%%%
%% Change alert block colors
%%% 1- Block title (background and text)
\setbeamercolor{block title alerted}{fg=mDarkTeal, bg=mLightBrown!45!yellow!45}
\setbeamercolor{block title example}{fg=magenta!10!black, bg=mLightGreen!60}
%%% 2- Block body (background)
\setbeamercolor{block body alerted}{bg=mLightBrown!25}
\setbeamercolor{block body example}{bg=mLightGreen!15}
%%%%%%%%%%%%%%%%%%%%%%%%%%%
%%%%%%%%%%%%%%%%%%%%%%%%%%%
%% さまざまなアイコン
%%%%%%%%%%%%%%%%%%%%%%%%%%%
%\usepackage{fontawesome}
\usepackage{fontawesome5}
\usepackage{figchild}
\usepackage{twemojis}
\usepackage{utfsym}
\usepackage{bclogo}
\usepackage{marvosym}
\usepackage{fontmfizz}
\usepackage{pifont}
\usepackage{phaistos}
\usepackage{worldflags}
\usepackage{jigsaw}
\usepackage{tikzlings}
\usepackage{tikzducks}
\usepackage{scsnowman}
\usepackage{epsdice}
\usepackage{halloweenmath}
\usepackage{svrsymbols}
\usepackage{countriesofeurope}
\usepackage{tipa}
\usepackage{manfnt}
%%%%%%%%%%%%%%%%%%%%%%%%%%%
\usepackage{tikz}
\usetikzlibrary{calc,patterns,decorations.pathmorphing,backgrounds}
\usepackage{tcolorbox}
\usepackage{tikzpeople}
\usepackage{circledsteps}
\usepackage{xcolor}
\usepackage{amsmath}
\usepackage{booktabs}
\usepackage{chronology}
\usepackage{signchart}
%%%%%%%%%%%%%%%%%%%%%%%%%%%
%% 場合分け
%%%%%%%%%%%%%%%%%%%%%%%%%%%
\usepackage{cases}
%%%%%%%%%%%%%%%%%%%%%%%%%%
\usepackage{pdfpages}
%%%%%%%%%%%%%%%%%%%%%%%%%%%
%% 音声リンク表示
\newcommand{\myaudio}[1]{\href{#1}{\faVolumeUp}}
%%%%%%%%%%%%%%%%%%%%%%%%%%
%% \myAnch{<名前>}{<色>}{<テキスト>}
%% 指定のテキストを指定の色の四角枠で囲み, 指定の名前をもつTikZの
%% ノードとして出力する. 図には remember picture 属性を付けている
%% ので外部から参照可能である.
\newcommand*{\myAnch}[3]{%
  \tikz[remember picture,baseline=(#1.base)]
    \node[draw,rectangle,line width=1pt,#2] (#1) {\normalcolor #3};
}
%%%%%%%%%%%%%%%%%%%%%%%%%%
%% \myEmph コマンドの定義
%%%%%%%%%%%%%%%%%%%%%%%%%%
%\newcommand{\myEmph}[3]{%
%    \textbf<#1>{\color<#1>{#2}{#3}}%
%}
\usepackage{xparse} % xparseパッケージの読み込み
\NewDocumentCommand{\myEmph}{O{} m m}{%
    \def\argOne{#1}%
    \ifx\argOne\empty
        \textbf{\color{#2}{#3}}% オプション引数が省略された場合
    \else
        \textbf<#1>{\color<#1>{#2}{#3}}% オプション引数が指定された場合
    \fi
}
%%%%%%%%%%%%%%%%%%%%%%%%%%%
%%%%%%%%%%%%%%%%%%%%%%%%%%%
%% 文末の上昇イントネーション記号 \myRisingPitch
%% 通常のイントネーション \myDownwardPitch
%% https://note.com/dan_oyama/n/n8be58e8797b2
%%%%%%%%%%%%%%%%%%%%%%%%%%%
\newcommand{\myRisingPitch}{
\begin{tikzpicture}[scale=0.3,baseline=0.3]
\draw[->,>=stealth] (0,0) to[bend right=45] (1,1);
\end{tikzpicture}
}
\newcommand{\myDownwardPitch}{
\begin{tikzpicture}[scale=0.3,baseline=0.3]
\draw[->,>=stealth] (0,1) to[bend left=45] (1,0);
\end{tikzpicture}
}
%%%%%%%%%%%%%%%%%%%%%%%%%%%%
%\AtBeginSection[%
%]{%
%  \begin{frame}[plain]\frametitle{授業の流れ}
%     \tableofcontents[currentsection]
%   \end{frame}%
%}

\usepackage{pxrubrica}
%%%%%%%%%%%%%%%%%%%%%%%%%%%
\title{English is fun.}
\subtitle{If I had a lot of money, I would buy a big house.}
\author{}
\institute[]{}
\date[]

%%%%%%%%%%%%%%%%%%%%%%%%%%%%
%% TEXT
%%%%%%%%%%%%%%%%%%%%%%%%%%%%
\begin{document}

\begin{frame}[plain]
  \titlepage
\end{frame}

\section*{授業の流れ}
\begin{frame}[plain]
  \frametitle{授業の流れ}
  \tableofcontents
\end{frame}
%%%%%%%%%%%%%%%%%%%%%%%%%
\section{If S $+$ 仮定法過去($=$ V)}

\begin{frame}[plain]{If S $+$ 仮定法過去($=$ V)}
 \large
\begin{enumerate} 
 \item I wish I \textcolor{Maroon}{\bfseries had} money.\hfill{\scriptsize お金があったらいいのに}
 \item If I \textcolor{Maroon}{\bfseries had} money, I would buy the cake.\hfill{\scriptsize お金があったらそのケーキを買うのに}
 \item If I \textcolor{NavyBlue}{\bfseries have} money, I will buy the cake.\hfill{\scriptsize お金があったらそのケーキを買います}
\end{enumerate}


\begin{block}{If S $+$ 仮定法過去($=$ V)}
\small
\begin{itemize}\setbeamertemplate{items}[square]
 \item ifと仮定法過去を組み合わせることがあります\par
If $\text{S}_{1} +$ \myAnch{mood1}{white}{仮定法過去}($= \text{V}_{1}$) \ldots \hspace{2pt},\hspace{10pt}
 $\text{S}_{2}\,+$ \Circled[fill color = white]{\,\,$\left\{ \begin{array}{l}
\text{would} \\
\text{could}
\end{array}\right\} + \text{原形}$\,\,}\,($=\text{V}_{2}$) \ldots\hspace{2pt} .\\[10pt]
\hfill{}\myAnch{mood2}{white}{現在の事実の反対}を表しています
\end{itemize}

\hfill{\scriptsize wouldは「(~なら)~するのに」、couldは「(~なら)~できるのに」の意味になります}

\hfill{\scriptsize would, couldはそれぞれwill, canの過去形です}
 
\begin{tikzpicture}[remember picture,overlay]
 \draw[thick,orange,->] (mood2.west) to[out=180, in=-45] (mood1.south east);
\end{tikzpicture}

\end{block}
\end{frame}
%%%%%%%%%%%%%%%%%%%%%%%%%
\begin{frame}[plain]{Exercises}
 日本文の意味になるように、適切なものを選びましょう

\begin{enumerate}
 \item 1,000円あったら、チーズバーガーとフライドポテトを買うのに(残念なことに手持ちがない)\\
If I ( get / have / \alt<2->{\Circled{had}}{had} ) 1,000 yen, I would buy a cheeseburger and fries.
 \item もしわたしがあなたなら、毎日朝食を食べますよ\\
If I ( am / are / \alt<3->{\Circled{were}}{were} ) you, I would eat breakfast every day. 
 \item もしわたしがあなたなら、早く寝ますよ\\
If I were you, I ( \alt<4->{\Circled{would}}{would} / will ) go to bed early.
\item もしわたしが魚だったら、じょうずに泳げるのに\\
If I were a fish, I ( can /  \alt<5->{\Circled{could}}{could} ) swim well.
\end{enumerate}
\end{frame}
%%%%%%%%%%%%%%%%%%%%%%%%%%
\begin{frame}[plain]{Exercises}
日本文の意味になるよう(~~~~~~~~)内の語句を並べ替えましょう
 \begin{enumerate}
  \item わたしがあなたなら、その本を読みますよ\\
If ( read / you / I / I / the book / were / would  ).\\
\visible<2->{If I were you, I would read the book.}
  \item たくさんお金があったら、大きな家を買うのになあ\\
If ( house / a /would / had / buy / money / I / I / large / a lot of ).\\
\visible<3->{If I had a lot of money, I would buy a large house.}
  \item きょう晴れていたら、外で遊べるのになあ\hfill{\scriptsize outdoors: 屋外で}\\
If ( sunny / I / it / play outdoors / were / could / today ).\\
\visible<4->{If it were sunny today, I could play outdoors.}

 \end{enumerate}
\end{frame}
%%%%%%%%%%%%%%%%%%%%%%%%%%%%
\begin{frame}[plain]{まとめ}
 \begin{block}{If S $+$ 仮定法過去($=$ V)}
\small
\begin{itemize}\setbeamertemplate{items}[square]
 \item ifと仮定法過去を組み合わせることがあります\par

If $\text{S}_{1} +$ \myAnch{mood1}{white}{仮定法過去}($= \text{V}_{1}$) \ldots\,\,\,,\hspace{10pt}
 $\text{S}_{2}\,+$ \Circled[fill color = white]{\,\,$\left\{ \begin{array}{l}
\text{would} \\
\text{could}
\end{array}\right\} + \text{原形}$\,\,}\,($=\text{V}_{2}$) \ldots\,\,\,.\\[10pt]
\hfill{}\myAnch{mood2}{white}{現在の事実の反対}を表しています
\end{itemize}

\hfill{\scriptsize wouldは「(~なら)~するのに」、couldは「(~なら)~できるのに」の意味になります}

\hfill{\scriptsize would, couldはそれぞれwill, canの過去形です}

 
\begin{tikzpicture}[remember picture,overlay]
 \draw[thick,orange,->] (mood2.west) to[out=180, in=-45] (mood1.south east);
\end{tikzpicture}

\end{block}
\end{frame}
\end{document}
