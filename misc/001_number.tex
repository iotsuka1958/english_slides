\documentclass[aspectratio=169,xcolor={dvipsnames,table}]{beamer}
\usepackage[no-math,deluxe,haranoaji]{luatexja-preset}
\renewcommand{\kanjifamilydefault}{\gtdefault}
\renewcommand{\emph}[1]{{\upshape\bfseries #1}}
\usetheme{metropolis}
\metroset{block=fill}
\setbeamertemplate{navigation symbols}{}
\setbeamertemplate{blocks}[rounded][shadow=false]
\usecolortheme[rgb={0.7,0.2,0.2}]{structure}
%%%%%%%%%%%%%%%%%%%%%%%%%%
%% Change alert block colors
%%% 1- Block title (background and text)
\setbeamercolor{block title alerted}{fg=mDarkTeal, bg=mLightBrown!45!yellow!45}
\setbeamercolor{block title example}{fg=magenta!10!black, bg=mLightGreen!60}
%%% 2- Block body (background)
\setbeamercolor{block body alerted}{bg=mLightBrown!25}
\setbeamercolor{block body example}{bg=mLightGreen!15}
%%%%%%%%%%%%%%%%%%%%%%%%%%%
%%%%%%%%%%%%%%%%%%%%%%%%%%%
%% さまざまなアイコン
%%%%%%%%%%%%%%%%%%%%%%%%%%%
%\usepackage{fontawesome}
\usepackage{fontawesome5}
\usepackage{figchild}
\usepackage{twemojis}
\usepackage{utfsym}
\usepackage{bclogo}
\usepackage{marvosym}
\usepackage{fontmfizz}
\usepackage{pifont}
\usepackage{phaistos}
\usepackage{worldflags}
\usepackage{jigsaw}
\usepackage{tikzlings}
\usepackage{tikzducks}
\usepackage{scsnowman}
\usepackage{epsdice}
\usepackage{halloweenmath}
\usepackage{svrsymbols}
\usepackage{countriesofeurope}
\usepackage{tipa}
%%%%%%%%%%%%%%%%%%%%%%%%%%%
\usepackage{tikz}
\usetikzlibrary{calc,patterns,decorations.pathmorphing,backgrounds}
\usepackage{tcolorbox}
\usepackage{tikzpeople}
\usepackage{circledsteps}
\usepackage{xcolor}
\usepackage{amsmath}
\usepackage{booktabs}
\usepackage{chronology}
\usepackage{signchart}
%%%%%%%%%%%%%%%%%%%%%%%%%%%
%% 場合分け
%%%%%%%%%%%%%%%%%%%%%%%%%%%
\usepackage{cases}
%%%%%%%%%%%%%%%%%%%%%%%%%%
\usepackage{pdfpages}
%%%%%%%%%%%%%%%%%%%%%%%%%%%
%% 音声リンク表示
\newcommand{\myaudio}[1]{\href{#1}{\faVolumeUp}}
%%%%%%%%%%%%%%%%%%%%%%%%%%
%% \myAnch{<名前>}{<色>}{<テキスト>}
%% 指定のテキストを指定の色の四角枠で囲み, 指定の名前をもつTikZの
%% ノードとして出力する. 図には remember picture 属性を付けている
%% ので外部から参照可能である.
\newcommand*{\myAnch}[3]{%
  \tikz[remember picture,baseline=(#1.base)]
    \node[draw,rectangle,line width=1pt,#2] (#1) {\normalcolor #3};
}
%%%%%%%%%%%%%%%%%%%%%%%%%%
%% \myEmph コマンドの定義
%%%%%%%%%%%%%%%%%%%%%%%%%%
%\newcommand{\myEmph}[3]{%
%    \textbf<#1>{\color<#1>{#2}{#3}}%
%}
\usepackage{xparse} % xparseパッケージの読み込み
\NewDocumentCommand{\myEmph}{O{} m m}{%
    \def\argOne{#1}%
    \ifx\argOne\empty
        \textbf{\color{#2}{#3}}% オプション引数が省略された場合
    \else
        \textbf<#1>{\color<#1>{#2}{#3}}% オプション引数が指定された場合
    \fi
}
%%%%%%%%%%%%%%%%%%%%%%%%%%%
%%%%%%%%%%%%%%%%%%%%%%%%%%%
%% 文末の上昇イントネーション記号 \myRisingPitch
%% 通常のイントネーション \myDownwardPitch
%% https://note.com/dan_oyama/n/n8be58e8797b2
%%%%%%%%%%%%%%%%%%%%%%%%%%%
\newcommand{\myRisingPitch}{
\begin{tikzpicture}[scale=0.3,baseline=0.3]
\draw[->,>=stealth] (0,0) to[bend right=45] (1,1);
\end{tikzpicture}
}
\newcommand{\myDownwardPitch}{
\begin{tikzpicture}[scale=0.3,baseline=0.3]
\draw[->,>=stealth] (0,1) to[bend left=45] (1,0);
\end{tikzpicture}
}
%%%%%%%%%%%%%%%%%%%%%%%%%%%%
%\AtBeginSection[%
%]{%
%  \begin{frame}[plain]\frametitle{授業の流れ}
%     \tableofcontents[currentsection]
%   \end{frame}%
%}

\usepackage{pxrubrica}
\UseTblrLibrary{counter}%%%%tabularrayとpauseが衝突することを回避する
%%%%%%%%%%%%%%%%%%%%%%%%%%%
\title{English is fun.}
\subtitle{I wish I had a lot of money.}
\author{}
\institute[]{}
\date[]

%%%%%%%%%%%%%%%%%%%%%%%%%%%%
%% TEXT
%%%%%%%%%%%%%%%%%%%%%%%%%%%%
\begin{document}

\begin{frame}[plain]
  \titlepage
\end{frame}

\section*{授業の流れ}
\begin{frame}[plain]
  \frametitle{授業の流れ}
  \tableofcontents
\end{frame}

\section{1, 2, 3 \ldots}
%%%%%%%%%%%%%%%%%%%%%%%%%%%%%%%%%%%%%%%%%%%%%%%%%%
\begin{frame}[plain]{1, 2, 3 \ldots\,18, 19, 20}
\small
 \hfill\begin{tblr}{
  colspec = {rll}, 
%  column{1} = {fg=red},    % 第1列の文字を赤に
%  column{2} = {fg=blue},   % 第7列の文字を青に
 row{odd} = {bg=azure8},
 row{1} = { bg=azure3, fg=white},
%  hline{1,2,Z} = {1pt},    % \toprule, \midrule, \bottomrule
%  hline{3} = {0.5pt}       % もう1つの \midrule
 baseline=t
}
  number  & \\
  1 & one & \textipa{/w\'\textturnv n/}\\
  2 & two & \textipa{/t\'u:/}\\ 
  3 & three & \textipa{/Tr\'\i:/}\\
  4 & four & \textipa{/f\'O\textrhookschwa /}\\
  5 & five & \textipa{/f\'aIv/}\\
  6 & six & \textipa{/s\'Iks/}\\
  7 & seven & \textipa{/s\'evn/}\\
  8 & eight & \textipa{/\'eIt/}\\
  9 & nine & \textipa{/n\'aIn/}\\
  10 & ten & \textipa{/t\'en/}\\
\end{tblr}
\hfill%
\pause
 \begin{tblr}{
  colspec = {rl}, 
%  column{1} = {fg=red},    % 第1列の文字を赤に
%  column{2} = {fg=blue},   % 第7列の文字を青に
 row{odd} = {bg=azure8},
 row{1} = { bg=azure3, fg=white},
%  hline{1,2,Z} = {1pt},    % \toprule, \midrule, \bottomrule
%  hline{3} = {0.5pt}       % もう1つの \midrule
 baseline=t
}
  number  & \\
  11 & eleven& \textipa{/Il\'evn/} \\
  12 & twelve & \textipa{/tw\'elv/}\\ 
  13 & thirteen & \textipa{/T\textrhookschwa :t\'\i:n/}\\
  14 & fourteen & \textipa{/f\'O\textrhookschwa t\'\i:n/}\\
  15 & fifteen & \textipa{/fIft\'\i:n/}\\
  16 & sixteen & \textipa{/sIkst\'\i:n/}\\
  17 & seventeen & \textipa{/sevnt\'\i:n/}\\
  18 & eighteen & \textipa{/eIt\'\i:n/}\\
  19 & nineteen & \textipa{/naInt\'\i:n/}\\
  20 & twenty & \textipa{/tw\'enti/}\\

\end{tblr}\hfill{\scriptsize \myaudio{./audio/001_number_01.mp3}} 
\end{frame}
%%%%%%%%%%%%%%%%%%%%%%%%%
%%%%%%%%%%%%%%%%%%%%%%%%%%%%%%%%%%%%%%%%%%%%%%%%%%
\begin{frame}[plain]{20, 21, 22 \ldots\,28, 29}
\small
 \hfill\begin{tblr}{
  colspec = {rll}, 
%  column{1} = {fg=red},    % 第1列の文字を赤に
%  column{2} = {fg=blue},   % 第7列の文字を青に
 row{odd} = {bg=azure8},
 row{1} = { bg=azure3, fg=white},
%  hline{1,2,Z} = {1pt},    % \toprule, \midrule, \bottomrule
%  hline{3} = {0.5pt}       % もう1つの \midrule
 baseline=t,
 cells={cmd=\onslide<\arabic{rownum}->} %%%%tabularrayとpauseが衝突することを回避する方法→https://github.com/lvjr/tabularray/issues/226
}
  number  & \\
  20 & twenty & \textipa{/tw\'enti/}\\
  21 & twenty-one & \textipa{/tw\'enti w\'\textturnv n/}\\
  22 & twenty-two & \textipa{/tw\'enti t\'u:/}\\ 
  23 & twenty-three & \textipa{/tw\'enti Tr\'\i:/}\\
  24 & twenty-four & \textipa{/tw\'enti f\'O\textrhookschwa /}\\
  25 & twenty-five & \textipa{/tw\'enti f\'aIv/}\\
  26 & twenty-six & \textipa{/tw\'enti s\'Iks/}\\
  27 & twenty-seven & \textipa{/tw\'enti s\'evn/}\\
  28 & twenty-eight & \textipa{/tw\'enti \'eIt/}\\
  29 & twenty-nine & \textipa{/tw\'enti n\'aIn/}\\
%  30 & thirty & \textipa{/T\'\textrhookschwa :ti/}\\
\end{tblr}
\hfill{\scriptsize \myaudio{./audio/001_number_02.mp3}} 
\end{frame}
%%%%%%%%%%%%%%%%%%%%%%%%%
%%%%%%%%%%%%%%%%%%%%%%%%%%%%%%%%%%%%%%%%%%%%%%%%%%
\begin{frame}[plain]{30, 31, 32 \ldots\,38, 39}
\small
 \hfill\begin{tblr}{
  colspec = {rll}, 
%  column{1} = {fg=red},    % 第1列の文字を赤に
%  column{2} = {fg=blue},   % 第7列の文字を青に
 row{odd} = {bg=azure8},
 row{1} = { bg=azure3, fg=white},
%  hline{1,2,Z} = {1pt},    % \toprule, \midrule, \bottomrule
%  hline{3} = {0.5pt}       % もう1つの \midrule
 baseline=t,
 cells={cmd=\onslide<\arabic{rownum}->} %%%%tabularrayとpauseが衝突することを回避する方法→https://github.com/lvjr/tabularray/issues/226
}
  number  & \\
  30 & thirty & \textipa{/T\'\textrhookschwa :ti/}\\
  31 & thirty-one & \textipa{/T\'\textrhookschwa :ti w\'\textturnv n/}\\
  32 & thirty-two & \textipa{/T\'\textrhookschwa :ti t\'u:/}\\ 
  33 & thirty-three & \textipa{/T\'\textrhookschwa :ti Tr\'\i:/}\\
  34 & thirty-four & \textipa{/T\'\textrhookschwa :ti f\'O\textrhookschwa /}\\
  35 & thirty-five & \textipa{/T\'\textrhookschwa :ti f\'aIv/}\\
  36 & thirty-six & \textipa{/T\'\textrhookschwa :ti s\'Iks/}\\
  37 & thirty-seven & \textipa{/T\'\textrhookschwa :ti s\'evn/}\\
  38 & thirty-eight & \textipa{/T\'\textrhookschwa :ti \'eIt/}\\
  39 & thirty-nine & \textipa{/T\'\textrhookschwa :ti n\'aIn/}\\
%  30 & thirty & \textipa{/T\'\textrhookschwa :ti/}\\
\end{tblr}
\hfill{\scriptsize \myaudio{./audio/001_number_03.mp3}} 
\end{frame}
%%%%%%%%%%%%%%%%%%%%%%%%%
%%%%%%%%%%%%%%%%%%%%%%%%%%%%%%%%%%%%%%%%%%%%%%%%%%
\begin{frame}[plain]{10, 20, 30, \ldots\,80, 90, 100}
\small
 \hfill\begin{tblr}{
  colspec = {rll}, 
%  column{1} = {fg=red},    % 第1列の文字を赤に
%  column{2} = {fg=blue},   % 第7列の文字を青に
 row{odd} = {bg=azure8},
 row{1} = { bg=azure3, fg=white},
%  hline{1,2,Z} = {1pt},    % \toprule, \midrule, \bottomrule
%  hline{3} = {0.5pt}       % もう1つの \midrule
 baseline=t,
 cells={cmd=\onslide<\arabic{rownum}->} %%%%tabularrayとpauseが衝突することを回避する方法→https://github.com/lvjr/tabularray/issues/226
}
  number  & \\
  10 & ten & \textipa{/t\'en/}\\
  20 & twenty & \textipa{/tw\'enti/}\\
  30 & thirty & \textipa{/T\'\textrhookschwa :ti/}\\ 
  40 & forty & \textipa{/f\'O\textrhookschwa :ti/}\\
  50 & fifty & \textipa{/f\'Ifti/}\\
  60 & sixty & \textipa{/s\'Iksti/}\\
  70 & seventy & \textipa{/s\'evnti/}\\
  80 & eighty & \textipa{/\'eIti/}\\
  90 & ninety & \textipa{/n\'aInti/}\\
  100 & one hundred & \textipa{/w\'\textturnv n h\'\textturnv ndr@d/}\\
%  30 & thirty & \textipa{/T\'\textrhookschwa :ti/}\\
\end{tblr}
\hfill{\scriptsize \myaudio{./audio/001_number_03.mp3}} 
\end{frame}
%%%%%%%%%%%%%%%%%%%%%%%%%
\end{document}
