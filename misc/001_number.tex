\documentclass[aspectratio=169,xcolor={dvipsnames,table}]{beamer}
\usepackage[no-math,deluxe,haranoaji]{luatexja-preset}
\renewcommand{\kanjifamilydefault}{\gtdefault}
\renewcommand{\emph}[1]{{\upshape\bfseries #1}}
\usetheme{metropolis}
\metroset{block=fill}
\setbeamertemplate{navigation symbols}{}
\setbeamertemplate{blocks}[rounded][shadow=false]
\usecolortheme[rgb={0.7,0.2,0.2}]{structure}
%%%%%%%%%%%%%%%%%%%%%%%%%%
%% Change alert block colors
%%% 1- Block title (background and text)
\setbeamercolor{block title alerted}{fg=mDarkTeal, bg=mLightBrown!45!yellow!45}
\setbeamercolor{block title example}{fg=magenta!10!black, bg=mLightGreen!60}
%%% 2- Block body (background)
\setbeamercolor{block body alerted}{bg=mLightBrown!25}
\setbeamercolor{block body example}{bg=mLightGreen!15}
%%%%%%%%%%%%%%%%%%%%%%%%%%%
\usepackage[absolute,overlay]{textpos}
%\usepackage[grid=true,gridcolor=Maroon,subgridcolor=gray,gridunit=pt,texcoord]{eso-pic} %場所決めのためのgrid表示
%%%%%%%%%%%%%%%%%%%%%%%%%%%
%% さまざまなアイコン
%%%%%%%%%%%%%%%%%%%%%%%%%%%
%\usepackage{fontawesome}
\usepackage{fontawesome5}
\usepackage{figchild}
\usepackage{twemojis}
\usepackage{utfsym}
\usepackage{bclogo}
\usepackage{marvosym}
\usepackage{fontmfizz}
\usepackage{pifont}
\usepackage{phaistos}
\usepackage{worldflags}
\usepackage{jigsaw}
\usepackage{tikzlings}
\usepackage{tikzducks}
\usepackage{scsnowman}
\usepackage{epsdice}
\usepackage{halloweenmath}
\usepackage{svrsymbols}
\usepackage{countriesofeurope}
\usepackage{tipa}
%%%%%%%%%%%%%%%%%%%%%%%%%%%
\usepackage{tikz}
\usetikzlibrary{calc,patterns,decorations.pathmorphing,backgrounds}
\usepackage{tcolorbox}
\usepackage{tikzpeople}
\usepackage{circledsteps}
\usepackage{xcolor}
\usepackage{amsmath}
\usepackage{booktabs}
\usepackage{chronology}
\usepackage{signchart}
%%%%%%%%%%%%%%%%%%%%%%%%%%%
%% 場合分け
%%%%%%%%%%%%%%%%%%%%%%%%%%%
\usepackage{cases}
%%%%%%%%%%%%%%%%%%%%%%%%%%
\usepackage{pdfpages}
%%%%%%%%%%%%%%%%%%%%%%%%%%%
%% 音声リンク表示
\newcommand{\myaudio}[1]{\href{#1}{\faVolumeUp}}
%%%%%%%%%%%%%%%%%%%%%%%%%%
%% \myAnch{<名前>}{<色>}{<テキスト>}
%% 指定のテキストを指定の色の四角枠で囲み, 指定の名前をもつTikZの
%% ノードとして出力する. 図には remember picture 属性を付けている
%% ので外部から参照可能である.
\newcommand*{\myAnch}[3]{%
  \tikz[remember picture,baseline=(#1.base)]
    \node[draw,rectangle,line width=1pt,#2] (#1) {\normalcolor #3};
}
%%%%%%%%%%%%%%%%%%%%%%%%%%
%% \myEmph コマンドの定義
%%%%%%%%%%%%%%%%%%%%%%%%%%
%\newcommand{\myEmph}[3]{%
%    \textbf<#1>{\color<#1>{#2}{#3}}%
%}
\usepackage{xparse} % xparseパッケージの読み込み
\NewDocumentCommand{\myEmph}{O{} m m}{%
    \def\argOne{#1}%
    \ifx\argOne\empty
        \textbf{\color{#2}{#3}}% オプション引数が省略された場合
    \else
        \textbf<#1>{\color<#1>{#2}{#3}}% オプション引数が指定された場合
    \fi
}
%%%%%%%%%%%%%%%%%%%%%%%%%%%
%%%%%%%%%%%%%%%%%%%%%%%%%%%
%% 文末の上昇イントネーション記号 \myRisingPitch
%% 通常のイントネーション \myDownwardPitch
%% https://note.com/dan_oyama/n/n8be58e8797b2
%%%%%%%%%%%%%%%%%%%%%%%%%%%
\newcommand{\myRisingPitch}{
\begin{tikzpicture}[scale=0.3,baseline=0.3]
\draw[->,>=stealth] (0,0) to[bend right=45] (1,1);
\end{tikzpicture}
}
\newcommand{\myDownwardPitch}{
\begin{tikzpicture}[scale=0.3,baseline=0.3]
\draw[->,>=stealth] (0,1) to[bend left=45] (1,0);
\end{tikzpicture}
}
%%%%%%%%%%%%%%%%%%%%%%%%%%%%
%\AtBeginSection[%
%]{%
%  \begin{frame}[plain]\frametitle{授業の流れ}
%     \tableofcontents[currentsection]
%   \end{frame}%
%}

\usepackage{pxrubrica}
\usepackage{lua-ul}
\UseTblrLibrary{counter}%%%%tabularrayとpauseが衝突することを回避する
\usepackage{lmodern}
\usetikzlibrary{tikzmark}
%%%%%%%%%%%%%%%%%%%%%%%%%%%
%%%%%%%%%%%%%%%%%%%%%%%%%%%
\title{English is fun.}
\subtitle{数字}
\author{}
\institute[]{}
\date[]

%%%%%%%%%%%%%%%%%%%%%%%%%%%%
%% TEXT
%%%%%%%%%%%%%%%%%%%%%%%%%%%%
\begin{document}

\begin{frame}[plain]
  \titlepage
\end{frame}

\section*{授業の流れ}
\begin{frame}[plain]
  \frametitle{授業の流れ}
  \tableofcontents
\end{frame}

\section{0, 1, 2, 3 \ldots\, 18, 19, 20}
%%%%%%%%%%%%%%%%%%%%%%%%%%%%%%%%%%%%%%%%%%%%%%%%%%
\begin{frame}[plain]{0, 1, 2, 3 \ldots\,18, 19, 20}
\small
% 0 zero \textipa{/z\'\i:roU/}\pause
\hfill\begin{tblr}{
  colspec = {rll}, 
%  column{1} = {fg=red},    % 第1列の文字を赤に
%  column{2} = {fg=blue},   % 第7列の文字を青に
 row{odd} = {bg=azure8},
 row{1} = { bg=azure3, fg=white},
 hline{Z} = {0.08em},    % \toprule, \midrule, \bottomrule
%  hline{3} = {0.5pt}       % もう1つの \midrule
 baseline=t,
 cell{1}{3} = {halign=r}
}
  number  & & {\tiny 0316}\,{\scriptsize \myaudio{./audio/001_number_01a.mp3}}\\
  0 & zero & \textipa{/z\'\i:roU/}\\
  1 & one & \textipa{/w\'\textturnv n/}\\
  2 & two & \textipa{/t\'u:/}\\ 
  3 & three & \textipa{/Tr\'\i:/}\\
  4 & four & \textipa{/f\'O\textrhookschwa /}\\
  5 & five & \textipa{/f\'aIv/}\\
  6 & six & \textipa{/s\'Iks/}\\
  7 & seven & \textipa{/s\'evn/}\\
  8 & eight & \textipa{/\'eIt/}\\
  9 & nine & \textipa{/n\'aIn/}\\
  10 & ten & \textipa{/t\'en/}\\
\end{tblr}
\hfill%
\pause
 \begin{tblr}{
  colspec = {rl}, 
%  column{1} = {fg=red},    % 第1列の文字を赤に
%  column{2} = {fg=blue},   % 第7列の文字を青に
 row{odd} = {bg=azure8},
 row{1} = { bg=azure3, fg=white},
 hline{Z} = {0.08em},    % \toprule, \midrule, \bottomrule
%  hline{3} = {0.5pt}       % もう1つの \midrule
 baseline=t,
 cell{1}{3} = {halign=r},
 cells={cmd=\onslide<\arabic{rownum}->} %%%%tabularrayとpauseが衝突することを回避する方法→https://github.com/lvjr/tabularray/issues/226
}
  number  & & {\tiny 0217}\,{\scriptsize \myaudio{./audio/001_number_01b.mp3}}\\
\\
  11 & eleven& \textipa{/Il\'evn/} \\
  12 & twelve & \textipa{/tw\'elv/}\\ 
  13 & thirteen & \textipa{/T\textrhookschwa :t\'\i:n/}\\
  14 & fourteen & \textipa{/fO\textrhookschwa t\'\i:n/}\\
  15 & fifteen & \textipa{/fIft\'\i:n/}\\
  16 & sixteen & \textipa{/sIkst\'\i:n/}\\
  17 & seventeen & \textipa{/sevnt\'\i:n/}\\
  18 & eighteen & \textipa{/eIt\'\i:n/}\\
  19 & nineteen & \textipa{/naInt\'\i:n/}\\
  20 & twenty & \textipa{/tw\'enti/}\\
\end{tblr}
\end{frame}
%%%%%%%%%%%%%%%%%%%%%%%%
\begin{frame}[plain]{11--19}

\visible<4->{13から19は、ぜんぶ \framebox{  } $+$ teen!\hspace{20pt}%
teen は $+10$ のこと}

 \begin{enumerate}\setcounter{enumi}{10}
  \item<1-> eleven\hfill{覚えるしかない}
  \item<2-> twelve\hfill{覚えるしかない}
  \item<3-> thirteen\visible<5->{($=3+10$)}\hfill\visible<6->{でもthreeteenじゃないので注意}
  \item<3-> fourteen\visible<7->{($=4+10$)}
  \item<3-> fifteen\visible<8->{($=5+10$)}\hfill\visible<9->{でもfiveteenじゃないので注意}
  \item<3-> sixteen\visible<10->{($=6+10$)}
  \item<3-> seventeen\visible<11->{($=7+10$)}
  \item<3-> eighteen\visible<12->{($=8+10$)}\hfill\visible<13->{でもeigh\underLine{tt}eenじゃないので注意}
  \item<3-> nineteen\visible<14->{($=9+10$)}
 \end{enumerate}
\end{frame}
%%%%%%%%%%%%%%%%%%%%%%%%%
\begin{frame}[plain]{Exercises}
 つぎの語があらわす数を数字で書きましょう\hspace{20pt}例: one $\rightarrow$ 1
\begin{enumerate}
 \item zero\hfill\visible<2->{\makebox[20pt][r]{0}}\hspace{250pt}\mbox{}
 \item three\hfill\visible<3->{\makebox[20pt][r]{3}}\hspace{250pt}\mbox{}
 \item four\hfill\visible<4->{\makebox[20pt][r]{4}}\hspace{250pt}\mbox{}
 \item eleven\hfill\visible<5->{\makebox[20pt][r]{11}}\hspace{250pt}\mbox{}
 \item twelve\hfill\visible<6->{\makebox[20pt][r]{12}}\hspace{250pt}\mbox{}
 \item thirteen\hfill\visible<7->{\makebox[20pt][r]{13}}\hspace{250pt}\mbox{}
 \item seventeen\hfill\visible<8->{\makebox[20pt][r]{17}}\hspace{250pt}\mbox{}
 \item eighteen\hfill\visible<9->{\makebox[20pt][r]{18}}\hspace{250pt}\mbox{}
 \item twenty\hfill\visible<10>{\makebox[20pt][r]{20}}\hspace{250pt}\mbox{}
\end{enumerate}
\end{frame}
%%%%%%%%%%%%%%%%%%%%%%%%%%%%%%%%%%%%%%%%%%%%%%%%%%
%%%%%%%%%%%%%%%%%%%%%%%%%
\begin{frame}[plain]{Exercises}
 つぎの数字を英語でつづりましょう\hspace{20pt}例: 2 $\rightarrow$ two
\begin{enumerate}
 \item \phantom{1}3\hfill\visible<2->{\makebox[20pt][l]{three}}\hspace{250pt}\mbox{}
 \item \phantom{1}4\hfill\visible<3->{\makebox[20pt][l]{four}}\hspace{250pt}\mbox{}
 \item \phantom{1}5\hfill\visible<4->{\makebox[20pt][l]{five}}\hspace{250pt}\mbox{}
 \item \phantom{1}9\hfill\visible<5->{\makebox[20pt][l]{nine}}\hspace{250pt}\mbox{}
 \item 11\hfill\visible<6->{\makebox[20pt][l]{eleven}}\hspace{250pt}\mbox{}
 \item 13\hfill\visible<7->{\makebox[20pt][l]{thirteen}}\hspace{250pt}\mbox{}
 \item 15\hfill\visible<8->{\makebox[20pt][l]{fifteen}}\hspace{250pt}\mbox{}
 \item 19\hfill\visible<9->{\makebox[20pt][l]{nineteen}}\hspace{250pt}\mbox{}
 \item 20\hfill\visible<10>{\makebox[20pt][l]{twenty}}\hspace{250pt}\mbox{}
\end{enumerate}
\end{frame}
%%%%%%%%%%%%%%%%%%%%%%%%%%%%%%%%%%%%%%%%%%%%%%%%%%
\section{10, 20, 30 \ldots\, 80, 90, 100}
%%%%%%%%%%%%%%%%%%%%%%%%%%%%%%%%%%%%%%%%%%%%%%%%%%
\begin{frame}[plain]{10, 20, 30 \ldots\,80, 90, 100}
\small
\centering
\begin{tblr}{
  colspec = {rll}, 
%  column{1} = {fg=red},    % 第1列の文字を赤に
%  column{2} = {fg=blue},   % 第7列の文字を青に
 row{odd} = {bg=azure8},
 row{1} = { bg=azure3, fg=white},
%  hline{1,2,Z} = {1pt},    % \toprule, \midrule, \bottomrule
%  hline{3} = {0.5pt}       % もう1つの \midrule
 baseline=t,
 cell{1}{3} = {halign=r},
 cells={cmd=\onslide<\arabic{rownum}->} %%%%tabularrayとpauseが衝突することを回避する方法→https://github.com/lvjr/tabularray/issues/226
}
  number  & &{\tiny 0214}\,{\scriptsize \myaudio{./audio/001_number_04.mp3}}\\
  10 & ten & \textipa{/t\'en/}\\
  20 & twenty & \textipa{/tw\'enti/}\\
  30 & thirty & \textipa{/T\'\textrhookschwa :ti/}\\ 
  40 & forty & \textipa{/f\'O\textrhookschwa :ti/}\\
  50 & fifty & \textipa{/f\'Ifti/}\\
  60 & sixty & \textipa{/s\'Iksti/}\\
  70 & seventy & \textipa{/s\'evnti/}\\
  80 & eighty & \textipa{/\'eIti/}\\
  90 & ninety & \textipa{/n\'aInti/}\\
  100 & one hundred & \textipa{/w\'\textturnv n h\'\textturnv ndr@d/}\\
%  30 & thirty & \textipa{/T\'\textrhookschwa :ti/}\\
\end{tblr}

\hfill\visible<12->{\scriptsize 20から90までは語尾がty}\\
\hfill\visible<13->{40は*fourtyじゃなく\textbf{forty}}
\end{frame}

%%%%%%%%%%%%%%%%%%%%%%%%%
%%%%%%%%%%%%%%%%%%%%%%%%
\begin{frame}[plain]{10, 20, 30, ... 70, 80, 90}

\visible<4->{20から90は、ぜんぶ \framebox{  } $+$ ty!\hspace{20pt}%
ty は $\times 10$ のこと}

 \begin{enumerate}\setcounter{enumi}{10}
  \item[10]<1-> ten
  \item[20]<2-> twenty\visible<5->{($=2\times 10$)}\hfill\visible<6->{でもtwotyじゃないので注意}
  \item[30]<3-> thirty\visible<7->{($=3\times 10$)}\hfill\visible<8->{でもthreetyじゃないので注意}
  \item[40]<3-> forty\visible<9->{($=4\times 10$)}\hfill\visible<10->{でもfo\underLine{u}rtyじゃないので注意}
  \item[50]<3-> fifty\visible<11->{($=5\times 10$)}\hfill\visible<12->{でもfivetyじゃないので注意}
  \item[60]<3-> sixty\visible<13->{($=6\times 10$)}
  \item[70]<3-> seventy\visible<14->{($=7\times 10$)}
  \item[80]<3-> eighty\visible<15->{($=8\times 10$)}\hfill\visible<16->{でもeigh\underLine{tt}yじゃないので注意}
  \item[90]<3-> ninety\visible<17->{($=9\times 10$)}
 \end{enumerate}
\end{frame}
%%%%%%%%%%%%%%%%%%%%%%%%%
\begin{frame}<1-5>[plain]{ここまでは}

\centering
\begin{tblr}{
colspec = rrrrrrrrrr,
cells={fg=gray6}
}
 \alt<1>{1}{\textcolor{black}{1}}&\alt<1>{2}{\textcolor{black}{2}}&\alt<1>{3}{\textcolor{black}{3}}&\alt<1>{4}{\textcolor{black}{4}}&\alt<1>{5}{\textcolor{black}{5}}&\alt<1>{6}{\textcolor{black}{6}}&\alt<1>{7}{\textcolor{black}{7}}&\alt<1>{8}{\textcolor{black}{8}}&\alt<1>{9}{\textcolor{black}{9}}&\alt<1>{10}{\textcolor{black}{10}}\\
 \alt<1-2>{11}{\textcolor{black}{11}}&\alt<1-2>{12}{\textcolor{black}{12}}&\alt<1-2>{13}{\textcolor{black}{13}}&\alt<1-2>{14}{\textcolor{black}{14}}&\alt<1-2>{15}{\textcolor{black}{15}}&\alt<1-2>{16}{\textcolor{black}{16}}&\alt<1-2>{17}{\textcolor{black}{17}}&\alt<1-2>{18}{\textcolor{black}{18}}&\alt<1-2>{19}{\textcolor{black}{19}}&\alt<1-2>{20}{\textcolor{black}{20}}\\
 \alt<1-4>{21}{\textcolor{Maroon}{21}}&\alt<1-4>{22}{\textcolor{Maroon}{22}}&\alt<1-4>{23}{\textcolor{Maroon}{23}}&\alt<1-4>{24}{\textcolor{Maroon}{24}}&\alt<1-4>{25}{\textcolor{Maroon}{25}}&\alt<1-4>{26}{\textcolor{Maroon}{26}}&\alt<1-4>{27}{\textcolor{Maroon}{27}}&\alt<1-4>{28}{\textcolor{Maroon}{28}}&\alt<1-4>{29}{\textcolor{Maroon}{29}}&\alt<1-3>{30}{\textcolor{black}{30}}\\
 31&32&33&34&35&36&37&38&39&\alt<1-3>{40}{\textcolor{black}{40}}\\
 41&42&43&44&45&46&47&48&49&\alt<1-3>{50}{\textcolor{black}{50}}\\
 51&52&53&54&55&56&57&58&59&\alt<1-3>{60}{\textcolor{black}{60}}\\
 61&62&63&64&65&66&67&68&69&\alt<1-3>{70}{\textcolor{black}{70}}\\
 71&72&73&74&75&76&77&78&79&\alt<1-3>{80}{\textcolor{black}{80}}\\
 81&82&83&84&85&86&87&88&89&\alt<1-3>{90}{\textcolor{black}{90}}\\
 91&92&93&94&95&96&97&98&99&\alt<1-3>{}{\textcolor{black}{}}\\
\end{tblr}
\end{frame}
%%%%%%%%%%%%%%%%%%%%%%%%%%%%%%%%%%%%%%%%%%%%%%%%%%
\section{20, 21, 22 \ldots\,27, 28, 29}
\begin{frame}[plain]{20, 21, 22 \ldots\,27, 28, 29}
\small\centering
\begin{tblr}{
  colspec = {rll}, 
%  column{1} = {fg=red},    % 第1列の文字を赤に
%  column{2} = {fg=blue},   % 第7列の文字を青に
 row{odd} = {bg=azure8},
 row{1} = { bg=azure3, fg=white},
%  hline{1,2,Z} = {1pt},    % \toprule, \midrule, \bottomrule
%  hline{3} = {0.5pt}       % もう1つの \midrule
 baseline=t,
 cell{1}{3} = {halign=r},
 cells={cmd=\onslide<\arabic{rownum}->} %%%%tabularrayとpauseが衝突することを回避する方法→https://github.com/lvjr/tabularray/issues/226
}
  number  & &{\tiny 0217}\,{\scriptsize \myaudio{./audio/001_number_02.mp3}}\\
  20 & twenty & \textipa{/tw\'enti/}\\
  21 & twenty-one & \textipa{/tw\'enti w\'\textturnv n/}\\
  22 & twenty-two & \textipa{/tw\'enti t\'u:/}\\ 
  23 & twenty-three & \textipa{/tw\'enti Tr\'\i:/}\\
  24 & twenty-four & \textipa{/tw\'enti f\'O\textrhookschwa /}\\
  25 & twenty-five & \textipa{/tw\'enti f\'aIv/}\\
  26 & twenty-six & \textipa{/tw\'enti s\'Iks/}\\
  27 & twenty-seven & \textipa{/tw\'enti s\'evn/}\\
  28 & twenty-eight & \textipa{/tw\'enti \'eIt/}\\
  29 & twenty-nine & \textipa{/tw\'enti n\'aIn/}\\
%  30 & thirty & \textipa{/T\'\textrhookschwa :ti/}\\
\end{tblr}
\end{frame}
%%%%%%%%%%%%%%%%%%%%%%%%%
\begin{frame}[plain]{21--29}
 
\large

21から29までは twenty-\fbox{  }\tikzmark{target}

\vspace{50pt}

\hfill{}\tikzmark{shooter}one, two, three ... eight, nine

\begin{tikzpicture}[remember picture,overlay]
 \visible<1->{\draw[<-,opacity=.4,line width=1pt] ([xshift=-15pt,yshift=-4pt]pic cs:target) to[out=-90, in=180] ([xshift=-2pt, yshift=2pt] pic cs:shooter);}
\end{tikzpicture}

\begin{textblock*}{0.4\linewidth}(50pt,120pt)
\visible<1->{\begin{tikzpicture}
\pig[
signpost=\scalebox{0.5}{
\parbox{2.2cm}{\color{black}
ハイフンをお忘れなく}},
signcolour= brown!70!gray,
signback=white!80!brown
]
\end{tikzpicture}}
\end{textblock*}
\end{frame}
%%%%%%%%%%%%%%%%%%%%%%%%
\section{30, 31, 32 \ldots\,37, 38, 39}
%%%%%%%%%%%%%%%%%%%%%%%%%%%%%%%%%%%%%%%%%%%%%%%%%%
\begin{frame}[plain]{30, 31, 32 \ldots\,38, 39}
\small
\centering
 \begin{tblr}{
  colspec = {rll}, 
%  column{1} = {fg=red},    % 第1列の文字を赤に
%  column{2} = {fg=blue},   % 第7列の文字を青に
 row{odd} = {bg=azure8},
 row{1} = { bg=azure3, fg=white},
%  hline{1,2,Z} = {1pt},    % \toprule, \midrule, \bottomrule
%  hline{3} = {0.5pt}       % もう1つの \midrule
 baseline=t,
 cell{1}{3} = {halign=r},
 cells={cmd=\onslide<\arabic{rownum}->} %%%%tabularrayとpauseが衝突することを回避する方法→https://github.com/lvjr/tabularray/issues/226
}
  number  & &{\tiny 0219}{\scriptsize \myaudio{./audio/001_number_03.mp3}}\\
  30 & thirty & \textipa{/T\'\textrhookschwa :ti/}\\
  31 & thirty-one & \textipa{/T\'\textrhookschwa :ti w\'\textturnv n/}\\
  32 & thirty-two & \textipa{/T\'\textrhookschwa :ti t\'u:/}\\ 
  33 & thirty-three & \textipa{/T\'\textrhookschwa :ti Tr\'\i:/}\\
  34 & thirty-four & \textipa{/T\'\textrhookschwa :ti f\'O\textrhookschwa /}\\
  35 & thirty-five & \textipa{/T\'\textrhookschwa :ti f\'aIv/}\\
  36 & thirty-six & \textipa{/T\'\textrhookschwa :ti s\'Iks/}\\
  37 & thirty-seven & \textipa{/T\'\textrhookschwa :ti s\'evn/}\\
  38 & thirty-eight & \textipa{/T\'\textrhookschwa :ti \'eIt/}\\
  39 & thirty-nine & \textipa{/T\'\textrhookschwa :ti n\'aIn/}\\
%  30 & thirty & \textipa{/T\'\textrhookschwa :ti/}\\
\end{tblr}
 
\end{frame}
%%%%%%%%%%%%%%%%%%%%%%%%%
%%%%%%%%%%%%%%%%%%%%%%%%%
\begin{frame}[plain]{21--99}
 
\large

31から39までは thirty\tikzmark{Target10}-\fbox{  }\tikzmark{Target01}

\vspace{50pt}

\hfill{}\tikzmark{Shooter01}one, two, three ... eight, nine

\hfill\visible<2->{\tikzmark{Shooter10}twenty, thirty, forty ... eighty, ninety}


\begin{tikzpicture}[remember picture,overlay]
 \visible<1->{\draw[<-,opacity=.4,line width=1pt] ([xshift=-15pt,yshift=-4pt]pic cs:Target01) to[out=-90, in=180] ([xshift=-2pt, yshift=2pt] pic cs:Shooter01);}
 \visible<2->{\draw[<-,opacity=.4,line width=1pt] ([xshift=-15pt,yshift=-4pt]pic cs:Target10) to[out=-90, in=180] ([xshift=-2pt, yshift=2pt] pic cs:Shooter10);}
\end{tikzpicture}

\begin{textblock*}{0.4\linewidth}(40pt,120pt)
\visible<3->{\begin{tikzpicture}
\cat[
scale=1.4,
signpost=\scalebox{0.7}{
\parbox{2.2cm}{\color{black}
99まで\\つくり方は同じです!}},
signcolour= brown!70!gray,
signback=white!80!brown
]
\end{tikzpicture}}
\end{textblock*}
\end{frame}
%%%%%%%%%%%%%%%%%%%%%%%%
%%%%%%%%%%%%%%%%%%%%%%%%%
\begin{frame}[plain]{Exercises}
 つぎの語があらわす数を数字で書きましょう\hspace{20pt}例: one $\rightarrow$ 1
\begin{enumerate}
 \item twenty-one\hfill\visible<2->{\makebox[20pt][r]{21}}\hspace{250pt}\mbox{}
 \item twenty-two\hfill\visible<3->{\makebox[20pt][r]{22}}\hspace{250pt}\mbox{}
 \item twenty-five\hfill\visible<4->{\makebox[20pt][r]{25}}\hspace{250pt}\mbox{}
 \item twenty-seven\hfill\visible<5->{\makebox[20pt][r]{27}}\hspace{250pt}\mbox{}
 \item thirty-three\hfill\visible<6->{\makebox[20pt][r]{33}}\hspace{250pt}\mbox{}
 \item thirty-four\hfill\visible<7->{\makebox[20pt][r]{34}}\hspace{250pt}\mbox{}
 \item thirty-six\hfill\visible<8->{\makebox[20pt][r]{36}}\hspace{250pt}\mbox{}
 \item thirty-eight\hfill\visible<9->{\makebox[20pt][r]{38}}\hspace{250pt}\mbox{}
 \item thirty-nine\hfill\visible<10>{\makebox[20pt][r]{39}}\hspace{250pt}\mbox{}
\end{enumerate}
\end{frame}
%%%%%%%%%%%%%%%%%%%%%%%%%%%%%%%%%%%%%%%%%%%%%%%%%%
%%%%%%%%%%%%%%%%%%%%%%%%%
\begin{frame}[plain]{Exercises}
 つぎの数字を英語でつづりましょう\hspace{20pt}例: 2 $\rightarrow$ two
\begin{enumerate}
 \item 23\hfill\visible<2->{\makebox[20pt][l]{twenty-three}}\hspace{250pt}\mbox{}
 \item 24\hfill\visible<3->{\makebox[20pt][l]{twenty-four}}\hspace{250pt}\mbox{}
 \item 25\hfill\visible<4->{\makebox[20pt][l]{twenty-five}}\hspace{250pt}\mbox{}
 \item 29\hfill\visible<5->{\makebox[20pt][l]{twenty-nine}}\hspace{250pt}\mbox{}
 \item 31\hfill\visible<6->{\makebox[20pt][l]{thirty-one}}\hspace{250pt}\mbox{}
 \item 36\hfill\visible<7->{\makebox[20pt][l]{thirty-six}}\hspace{250pt}\mbox{}
 \item 37\hfill\visible<8->{\makebox[20pt][l]{thirty-seven}}\hspace{250pt}\mbox{}
 \item 38\hfill\visible<9->{\makebox[20pt][l]{thirty-eight}}\hspace{250pt}\mbox{}
 \item 39\hfill\visible<10>{\makebox[20pt][l]{thirty-nine}}\hspace{250pt}\mbox{}
\end{enumerate}
\end{frame}
%%%%%%%%%%%%%%%%%%%%%%%%%%%%%%%%%%%%%%%%%%
\begin{frame}[plain]{Exercises}
 聞き取った数を数字で書きましょう\hspace{20pt}例: 音声one $\rightarrow$ 答1
\begin{enumerate}
 \item<2-> ten\hfill\visible<3->{\makebox[20pt][r]{10}}\hspace{250pt}\mbox{}
 \item<4-> twenty-three\hfill\visible<5->{\makebox[20pt][r]{23}}\hspace{250pt}\mbox{}
 \item<6-> thirty-five\hfill\visible<7->{\makebox[20pt][r]{35}}\hspace{250pt}\mbox{}
 \item<8-> forty\hfill\visible<9->{\makebox[20pt][r]{40}}\hspace{250pt}\mbox{}
 \item<10-> fifty-two\hfill\visible<11->{\makebox[20pt][r]{52}}\hspace{250pt}\mbox{}
 \item<12-> sixty\hfill\visible<13->{\makebox[20pt][r]{60}}\hspace{250pt}\mbox{}
 \item<14-> seventy-seven\hfill\visible<15->{\makebox[20pt][r]{77}}\hspace{250pt}\mbox{}
 \item<16-> eighty-eight\hfill\visible<17->{\makebox[20pt][r]{88}}\hspace{250pt}\mbox{}
 \item<18-> ninety-nine\hfill\visible<19->{\makebox[20pt][r]{99}}\hspace{250pt}\mbox{}
\end{enumerate}

{\tiny 0527}{\scriptsize \myaudio{./audio/001_number_99.mp3}}
\end{frame}
%%%%%%%%%%%%%%%%%%%%%%%%%%%%%%%%%%%%%%%%%%%%%%%%%%
%%%%%%%%%%%%%%%%%%%%%%%%%
\begin{frame}[plain]{Exercises}
 聞き取った数をを英語でつづりましょう\hspace{20pt}例: 音声2 $\rightarrow$ 答two
\begin{enumerate}
 \item<2-> \phantom{1}23\hfill\visible<3->{\makebox[20pt][l]{twenty-three}}\hspace{250pt}\mbox{}
 \item<4-> \phantom{1}34\hfill\visible<5->{\makebox[20pt][l]{thirty-four}}\hspace{250pt}\mbox{}
 \item<6-> \phantom{1}45\hfill\visible<7->{\makebox[20pt][l]{forty-five}}\hspace{250pt}\mbox{}
 \item<8-> \phantom{1}59\hfill\visible<9->{\makebox[20pt][l]{fifty-nine}}\hspace{250pt}\mbox{}
 \item<10-> \phantom{1}61\hfill\visible<11->{\makebox[20pt][l]{sixty-one}}\hspace{250pt}\mbox{}
 \item<12-> \phantom{1}73\hfill\visible<13->{\makebox[20pt][l]{seventy-three}}\hspace{250pt}\mbox{}
 \item<14-> \phantom{1}83\hfill\visible<15->{\makebox[20pt][l]{eighty-three}}\hspace{250pt}\mbox{}
 \item<16-> \phantom{1}99\hfill\visible<17->{\makebox[20pt][l]{ninety-nine}}\hspace{250pt}\mbox{}
 \item<18-> 100\,(2語で)\hfill\visible<19>{\makebox[20pt][l]{one hundred}}\hspace{250pt}\mbox{}
\end{enumerate}
{\tiny 0527}{\scriptsize \myaudio{./audio/001_number_99b.mp3}}

\end{frame}

%%%%%%%%%%%%%%%%%%%%%%%%%%%%%%%%%%%%%%%%%%%%%%%%%%
\section{more than 100}
%%%%%%%%%%%%%%%%%%%%%%%%%
\begin{frame}[plain]{200はなんていうの?}
\Huge\centering

\begin{tabular}{lll}
 \visible<2->{100}& \visible<3->{$\longrightarrow$\,\,\,\,one hundred} & \\
 \visible<4->{200}& \visible<5->{$\longrightarrow$\,\,\,\,two hundred\textcolor{Maroon}{s}} & \visible<6->{{\normalsize まちがい}}\\
&\visible<7->{$\longrightarrow$\,\,\,\,two hundred}&\visible<7->{{\normalsize 正解}}
\end{tabular}

\end{frame}
%%%%%%%%%%%%%%%%%%%%%%%%%%%
\begin{frame}[plain]{more than 100}
 \small
\centering
\begin{tblr}{
  colspec = {rl}, 
%  column{1} = {fg=red},    % 第1列の文字を赤に
%  column{2} = {fg=blue},   % 第7列の文字を青に
 row{odd} = {bg=azure8},
 row{1} = { bg=azure3, fg=white},
%  hline{1,2,Z} = {1pt},    % \toprule, \midrule, \bottomrule
%  hline{3} = {0.5pt}       % もう1つの \midrule
 baseline=t,
 cell{1}{2} = {halign=r},
 cells={cmd=\onslide<\arabic{rownum}->} %%%%tabularrayとpauseが衝突することを回避する方法→https://github.com/lvjr/tabularray/issues/226
}
  number  & {\tiny 0424}\,{\scriptsize \myaudio{./audio/001_number_05.mp3}}\\
101& one hundred (and) one\\
102& one hundred (and) two\\
111& one hundred (and) eleven\\
125& one hundred (and) twenty-five\\
134& one hundred (and) thirty-four\\
243& two hundred (and) forty-three\\
358& three hundred (and) fifty-eight\\
567& five hundred (and) sixty-seven\\
777& seven hundred (and) seventy-seven\\
999& nine hundred (and) ninety-nine\\
\end{tblr}

\hfill\visible<2->{\scriptsize hundredのあとのandはあってもなくてもOK}
\end{frame}
%%%%%%%%%%%%%%%%%%%%%%%%%
\begin{frame}[plain,t]{101--999}
 
\large

101から999までは

\hspace*{20pt}\fbox{百の位の数}\tikzmark{target100} hundred (and) \fbox{1から99までの数}\tikzmark{target10}

\vspace{50pt}

\hfill{}\visible<3->{\tikzmark{shooter10}one, two, three ... twenty-one ... ninety-nine}


\hfill\visible<2->{\tikzmark{shooter100}one, two, three ... eight, nine}


\begin{tikzpicture}[remember picture,overlay]
 \visible<3->{\draw[<-,opacity=.4,line width=1pt] ([xshift=-50pt,yshift=-4pt]pic cs:target10) to[out=-90, in=180] ([xshift=-2pt, yshift=2pt] pic cs:shooter10);}
 \visible<2->{\draw[<-,opacity=.4,line width=1pt] ([xshift=-20pt,yshift=-4pt]pic cs:target100) to[out=-90, in=180] ([xshift=-2pt, yshift=2pt] pic cs:shooter100);}
\end{tikzpicture}

\begin{textblock*}{0.4\linewidth}(35pt,120pt)
\visible<4->{\begin{tikzpicture}
\cat[
scale=1.4,
signpost=\scalebox{0.7}{
\parbox{2.2cm}{\color{black}
999まで\\つくり方は同じです!}},
signcolour= brown!70!gray,
signback=white!80!brown
]
\end{tikzpicture}}
\end{textblock*}
\end{frame}
%%%%%%%%%%%%%%%%%%%%%%%%
%%%%%%%%%%%%%%%%%%%%%%%%%%%%%%%%%%%%%%%%%%%%%%%%%%
\section{more than 999}
%%%%%%%%%%%%%%%%%%%%%%%%%%%%%%%%%%%%%%%%%%%%%%
\begin{frame}[plain]{{\scriptsize \textdbend}\,more than 999}
  \small
\centering
\begin{tblr}{
  colspec = {rl}, 
%  column{1} = {fg=red},    % 第1列の文字を赤に
%  column{2} = {fg=blue},   % 第7列の文字を青に
 row{odd} = {bg=azure8},
 row{1} = { bg=azure3, fg=white},
%  hline{1,2,Z} = {1pt},    % \toprule, \midrule, \bottomrule
%  hline{3} = {0.5pt}       % もう1つの \midrule
 baseline=t,
 cell{1}{2} = {halign=r},
 cells={cmd=\onslide<\arabic{rownum}->} %%%%tabularrayとpauseが衝突することを回避する方法→https://github.com/lvjr/tabularray/issues/226
}
  number  & {\tiny 0259}\,{\scriptsize \myaudio{./audio/001_number_06.mp3}}\\
1,000& one thousand\hspace{20pt}\textipa{/w\'\textturnv n T\'aUznd/}\\
1,400& one thousand four hundred\\
2,300 & two thousand three hundred\\
3,450& three thousand four hundred (and) fifty\\
4,567& four thousand five hundred (and) sixty-seven\\
10,000 & ten thousand\\
20,000 & twenty thousand\\
100,000& one hundred thousand\\
1,000,000& one million\hspace{30pt}\textipa{/w\'\textturnv n m\'Ilj@n/}\\
1,000,000,000& one billion\hspace{32.5pt}\textipa{/w\'\textturnv n b\'Ilj@n/}\\
\end{tblr}
\end{frame}
%%%%%%%%%%%%%%%%%%%%%%%%%%%%%%%%%%%%
\section{一覧表}
%%%%%%%%%%%%%%%%%%%%%%%%%%%%%%%%%%%%
\begin{frame}[plain,shrink=5,label=ichiran]{一覧表}

\small
\visible<2->{\begin{tblr}{
  colspec = {rl}, 
 row{odd} = {bg=azure8},
 row{1} = { bg=azure3, fg=white},
 hline{Z} = {0.08em},    % \toprule, \midrule, \bottomrule
%  hline{3} = {0.5pt}       % もう1つの \midrule
 baseline=t
}
  number  & \\
  0 & zero \\
  1 & one \\
  2 & two \\ 
  3 & three \\
  4 & four \\
  5 & five \\
  6 & six\\
  7 & seven \\
  8 & eight \\
  9 & nine \\
  10 & ten \\
\end{tblr}}
%\pause
 \visible<3->{\begin{tblr}{
  colspec = {rl}, 
%  column{1} = {fg=red},    % 第1列の文字を赤に
%  column{2} = {fg=blue},   % 第7列の文字を青に
 row{odd} = {bg=azure8},
 row{1} = { bg=azure3, fg=white},
 hline{Z} = {0.08em},    % \toprule, \midrule, \bottomrule
%  hline{3} = {0.5pt}       % もう1つの \midrule
 baseline=t,
%  cells={cmd=\onslide<\arabic{rownum}->} %%%%tabularrayとpauseが衝突することを回避する方法→https://github.com/lvjr/tabularray/issues/226
}
   & \\
\\
  11 & eleven \\
  12 & twelve\\ 
  13 & thirteen \\
  14 & fourteen \\
  15 & fifteen \\
  16 & sixteen \\
  17 & seventeen \\
  18 & eighteen \\
  19 & nineteen \\
  20 & twenty \\
\end{tblr}}
%\pause
\visible<4->{\begin{tblr}{
  colspec = {rll}, 
%  column{1} = {fg=red},    % 第1列の文字を赤に
%  column{2} = {fg=blue},   % 第7列の文字を青に
 row{odd} = {bg=azure8},
 row{1} = { bg=azure3, fg=white},
%  hline{1,2,Z} = {1pt},    % \toprule, \midrule, \bottomrule
%  hline{3} = {0.5pt}       % もう1つの \midrule
 hline{Z} = {0.08em},    % \toprule, \midrule, \bottomrule
 baseline=t,
% cells={cmd=\onslide<\arabic{rownum}->} %%%%tabularrayとpauseが衝突することを回避する方法→https://github.com/lvjr/tabularray/issues/226
}
    & \\
   &  \\
  21 & twenty-one \\
  22 & twenty-two \\ 
  23 & twenty-three \\
  24 & twenty-four \\
  25 & twenty-five \\
  26 & twenty-six \\
  27 & twenty-seven \\
  28 & twenty-eight \\
  29 & twenty-nine \\
  30 & thirty \\
\end{tblr}}
%\pause
\visible<5->{\begin{tblr}{
  colspec = {rll}, 
%  column{1} = {fg=red},    % 第1列の文字を赤に
%  column{2} = {fg=blue},   % 第7列の文字を青に
 row{odd} = {bg=azure8},
 row{1} = { bg=azure3, fg=white},
%  hline{1,2,Z} = {1pt},    % \toprule, \midrule, \bottomrule
%  hline{3} = {0.5pt}       % もう1つの \midrule
 hline{Z} = {0.08em},    % \toprule, \midrule, \bottomrule
 baseline=t,
% cells={cmd=\onslide<\arabic{rownum}->} %%%%tabularrayとpauseが衝突することを回避する方法→https://github.com/lvjr/tabularray/issues/226
}
    & \\
    & \\
  40 & forty \\
  50 & fifty \\
  60 & sixty \\
  70 & seventy \\
  80 & eighty \\
  90 & ninety \\
  100 & one hundred \\
  1,000 & one thousand\\
  1,000,000 & one million\\
  1,000,000,000 & one billion\\
\end{tblr}}

\hspace{110pt}\visible<3->{teen: $+ 10$}\hspace{55pt}\visible<4->{ty: $\times 10$}
\end{frame}
%%%%%%%%%%%%%%%%%%%%%%%%%%%%%%%%%%%%%%%%%%%%%
\section{大きな数}
%%%%%%%%%%%%%%%%%%%%%%%%%%%%%%%%%%%%%%%%%%%
\begin{frame}[plain,t,label=largenumber]{大きい数の表現}
 \Large

20,\tikzmark{comma_million}250,\tikzmark{comma_thousand}717

\large

\vspace{50pt}

\hfill{}\tikzmark{thousand}\visible<2->{thousand}

\hfill{}\tikzmark{million}\visible<3->{million}

\begin{tikzpicture}[remember picture,overlay]
 \visible<2->{\draw[<-] ([xshift=0pt, yshift=-5pt]pic cs:comma_thousand) to[out=-50, in=180] ([xshift=-2pt, yshift=3
pt]pic cs:thousand);}
 \visible<3->{\draw[<-] ([xshift=0pt, yshift=-5pt] pic cs:comma_million) to[out=-60,in=180] ([xshift=-2pt, yshift=3pt] pic cs:million);}
\end{tikzpicture}


\hfill\visible<4->{20 million, 250 thousand, 717}

\vfill

\hfill{}\visible<5->{twenty million, two hundred fifty thousand, seven hundred  seventeen}

{\tiny 0102}\,{\scriptsize \myaudio{./audio/001_number_06a.mp3}}
\end{frame}
%%%%%%%%%%%%%%%%%%%%%%%%%
\section{小数}
%%%%%%%%%%%%%%%%%%%%%%%%%
\begin{frame}[plain,label=decimal]{小数}
 \Huge

\[
 \pi \tikzmark{target_approx}\approx \tikzmark{target_pi}{3.14}
\]

\bigskip

\hfill{}\tikzmark{pi}\visible<3->{three point one four}
\bigskip

\bigskip

\bigskip

\small

\hfill{}\tikzmark{approx}\visible<2->{$\approx$は「ほぼ等しい」という記号}\hspace{7pt}\mbox{}



\begin{tikzpicture}[remember picture,overlay]
 \visible<3->{\draw[<-,line width=.8pt] ([xshift=14pt, yshift=-5pt] pic cs:target_pi) to[out=-90,in=130] ([xshift=-4pt, yshift=6pt] pic cs:pi);}
 \visible<2->{\draw[<-] ([xshift=14pt, yshift=-5pt]pic cs:target_approx) to[out=-120, in=190] ([xshift=-3pt, yshift=3pt]pic cs:approx);}
\end{tikzpicture}

\end{frame}
%%%%%%%%%%%%%%%%%%%%%%%%%%%%%%%%%%%%
\begin{frame}[plain]{2つのパイ}
\Large
\begin{description}
 \item[円周率のパイ] {\Huge $\pi$\,\,($=$ pi)}
 \item[食べ物のパイ] {\Huge pie}
\end{description}

\vspace{40pt}

\hspace*{40pt}{発音はどちらも\textipa{/p\'aI/}}

\begin{textblock*}{0.6\linewidth}(230pt,40pt)
\includegraphics[width=.85\textwidth]{./image/pie.png}
\end{textblock*}
\end{frame}
%%%%%%%%%%%%%%%%%%%%%%%%%%%%%%%%%%%%%%%%%%%%%%%%%%%%
\begin{frame}[plain]{Exercises}
\scriptsize

\vspace{-3pt}
\small
\begin{tcolorbox}[colframe=ForestGreen,
  colback=ForestGreen!10!white,
  colbacktitle=ForestGreen!40!white,
  coltitle=black, %fonttitle=\bfseries,
before upper={\setlength{\parindent}{1.25em}},
  title=次の英文を読み、あとの設問に答えましょう\mbox{}\hfill{\tiny 0054}\,{\scriptsize \myaudio{./audio/001_number_07.mp3}}
]

March 14 is called Pi\tikzmark{clue1} Day in the U.S. Why? Because, in numbers, March 14 is written as 3/14, and the number pi ($\pi$) begins with 3.14. We use pi to find the length\footnotemark{} around a circle.

Some people eat pie on March 14. Why? Because ``pie'' sou\tikzmark{clue2}nds just like ``pi.''
At some schools, students learn about cir\tikzmark{clue3}cles and pi, and eat sweet pie. It is a fun day to enjoy math.

How many digits\footnotemark{} are there in pi? It has many! 3.141592\ldots\,\, It goes on \tikzmark{clue4}and on\footnotemark{}!

\end{tcolorbox}

\vspace{-5pt}

\begin{enumerate}\scriptsize\setlength{\itemsep}{-2pt}
 \item<2-> アメリカで3月14日を円周率の日と呼ぶのはなぜですか\tikzmark{q1}\hfill\visible<4->{円周率の最初の3桁は3.14だから}
 \item<2-> アメリカで3月14日にパイを食べるのはなぜですか\tikzmark{q2}\hfill\visible<6->{パイの発音は円周率$\pi$の発音と同じだから}
 \item<2-> 3月14日にアメリカの生徒は平行四辺形について学習しますか\tikzmark{q3}\hfill\visible<8->{いいえ(円と円周率について学習する)}
 \item<2-> $\pi$の桁数は有限ですか\tikzmark{q4}\hfill\visible<10->{いいえ(無限に続く)}
\end{enumerate}

\vspace*{-15pt}

\footnotetext[1]{\scriptsize length:長さ}
\footnotetext[2]{\scriptsize digit:数字}
\footnotetext[3]{\scriptsize go on and on:えんえんと続く}

\begin{tikzpicture}[remember picture,overlay]
 \visible<3->{\draw[<-,opacity=.4,line width=2pt] ([yshift=-2pt]pic cs:clue1) to[bend left] ([xshift=2pt, yshift=2pt] pic cs:q1);}
 \visible<5->{\draw[<-,opacity=.4,line width=2pt] ([yshift=-2pt]pic cs:clue2) to[bend left] ([xshift=2pt, yshift=2pt] pic cs:q2);}
 \visible<7->{\draw[<-,opacity=.4,line width=2pt] ([yshift=-2pt]pic cs:clue3) to[in=30,out=-70] ([xshift=2pt, yshift=2pt] pic cs:q3);}
 \visible<9->{\draw[<-,opacity=.4,line width=2pt] ([yshift=-2pt]pic cs:clue4) to[out=-120, in=0] ([xshift=2pt, yshift=2pt] pic cs:q4);}
\end{tikzpicture}
\end{frame}
%%%%%%%%%%%%%%%%%%%%%%%%%%%%%%%%%%%
\begin{frame}[plain]{大意}

3月14日はアメリカで「円周率の日」と呼ばれています。なぜでしょうか?数字で3月14日は3/14と書き、円周率($\pi$)は3.14で始まるからです。円周率は円周を求めるときに使います。

3月14日にはパイを食べる人もいます。なぜでしょうか?食べ物の「パイ」は円周率の「パイ」と発音が似ているからです。学校によっては、生徒たちが円や円周率について学び、甘いパイを食べることもあります。数学を楽しむ楽しい一日です。

円周率には何桁の数字があるでしょうか?たくさんの桁があります!$3.141592 \ldots$と、えんえんと続きます!
\end{frame}
%%%%%%%%%%%%%%%%%%%%%%%%
\section{まとめ}
\againframe<5>[plain,shrink=5]{ichiran}
\againframe<5>[plain]{largenumber}
\againframe<3>[plain]{decimal}
%%%%%%%%%%%%%%%%%%%%%%%%%%%
\begin{frame}[plain]{音声概要と解説動画}
 
{\tiny audio\_overview 2220}\,{\scriptsize \myaudio{./audio/overview/001_number_audio_overview.m4a}}

{\tiny video 0805}\,{\scriptsize \href{./video/001_number_ja_subtitle.mp4}{\reflectbox{\twemoji{movie camera}}}}
\end{frame}
%%%%%%%%%%%%%%%%%%%%%
\end{document}
