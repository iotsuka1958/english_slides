\documentclass[aspectratio=169,xcolor={dvipsnames,table}]{beamer}
\usepackage[no-math,deluxe,haranoaji]{luatexja-preset}
\renewcommand{\kanjifamilydefault}{\gtdefault}
\renewcommand{\emph}[1]{{\upshape\bfseries #1}}
\usetheme{metropolis}
\metroset{block=fill}
\setbeamertemplate{navigation symbols}{}
\setbeamertemplate{blocks}[rounded][shadow=false]
\usecolortheme[rgb={0.7,0.2,0.2}]{structure}
%%%%%%%%%%%%%%%%%%%%%%%%%%
%% Change alert block colors
%%% 1- Block title (background and text)
\setbeamercolor{block title alerted}{fg=mDarkTeal, bg=mLightBrown!45!yellow!45}
\setbeamercolor{block title example}{fg=magenta!10!black, bg=mLightGreen!60}
%%% 2- Block body (background)
\setbeamercolor{block body alerted}{bg=mLightBrown!25}
\setbeamercolor{block body example}{bg=mLightGreen!15}
%%%%%%%%%%%%%%%%%%%%%%%%%%%
%%%%%%%%%%%%%%%%%%%%%%%%%%%
%% さまざまなアイコン
%%%%%%%%%%%%%%%%%%%%%%%%%%%
%\usepackage{fontawesome}
\usepackage{fontawesome5}
\usepackage{figchild}
\usepackage{twemojis}
\usepackage{utfsym}
\usepackage{bclogo}
\usepackage{marvosym}
\usepackage{fontmfizz}
\usepackage{pifont}
\usepackage{phaistos}
\usepackage{worldflags}
\usepackage{jigsaw}
\usepackage{tikzlings}
\usepackage{tikzducks}
\usepackage{scsnowman}
\usepackage{epsdice}
\usepackage{halloweenmath}
\usepackage{svrsymbols}
\usepackage{countriesofeurope}
\usepackage{tipa}
\usepackage{manfnt}
%%%%%%%%%%%%%%%%%%%%%%%%%%%
\usepackage{tikz}
\usetikzlibrary{calc,patterns,decorations.pathmorphing,backgrounds}
\usepackage{tcolorbox}
\usepackage{tikzpeople}
\usepackage{circledsteps}
\usepackage{xcolor}
\usepackage{amsmath}
\usepackage{booktabs}
\usepackage{chronology}
\usepackage{signchart}
%%%%%%%%%%%%%%%%%%%%%%%%%%%
%% 場合分け
%%%%%%%%%%%%%%%%%%%%%%%%%%%
\usepackage{cases}
%%%%%%%%%%%%%%%%%%%%%%%%%%
\usepackage{pdfpages}
%%%%%%%%%%%%%%%%%%%%%%%%%%%
%% 音声リンク表示
\newcommand{\myaudio}[1]{\href{#1}{\faVolumeUp}}
%%%%%%%%%%%%%%%%%%%%%%%%%%
%% \myAnch{<名前>}{<色>}{<テキスト>}
%% 指定のテキストを指定の色の四角枠で囲み, 指定の名前をもつTikZの
%% ノードとして出力する. 図には remember picture 属性を付けている
%% ので外部から参照可能である.
\newcommand*{\myAnch}[3]{%
  \tikz[remember picture,baseline=(#1.base)]
    \node[draw,rectangle,line width=1pt,#2] (#1) {\normalcolor #3};
}
%%%%%%%%%%%%%%%%%%%%%%%%%%
%% \myEmph コマンドの定義
%%%%%%%%%%%%%%%%%%%%%%%%%%
%\newcommand{\myEmph}[3]{%
%    \textbf<#1>{\color<#1>{#2}{#3}}%
%}
\usepackage{xparse} % xparseパッケージの読み込み
\NewDocumentCommand{\myEmph}{O{} m m}{%
    \def\argOne{#1}%
    \ifx\argOne\empty
        \textbf{\color{#2}{#3}}% オプション引数が省略された場合
    \else
        \textbf<#1>{\color<#1>{#2}{#3}}% オプション引数が指定された場合
    \fi
}
%%%%%%%%%%%%%%%%%%%%%%%%%%%
%%%%%%%%%%%%%%%%%%%%%%%%%%%
%% 文末の上昇イントネーション記号 \myRisingPitch
%% 通常のイントネーション \myDownwardPitch
%% https://note.com/dan_oyama/n/n8be58e8797b2
%%%%%%%%%%%%%%%%%%%%%%%%%%%
\newcommand{\myRisingPitch}{
\begin{tikzpicture}[scale=0.3,baseline=0.3]
\draw[->,>=stealth] (0,0) to[bend right=45] (1,1);
\end{tikzpicture}
}
\newcommand{\myDownwardPitch}{
\begin{tikzpicture}[scale=0.3,baseline=0.3]
\draw[->,>=stealth] (0,1) to[bend left=45] (1,0);
\end{tikzpicture}
}
%%%%%%%%%%%%%%%%%%%%%%%%%%%%
%\AtBeginSection[%
%]{%
%  \begin{frame}[plain]\frametitle{授業の流れ}
%     \tableofcontents[currentsection]
%   \end{frame}%
%}

\usepackage{pxrubrica}
\UseTblrLibrary{counter}%%%%tabularrayとpauseが衝突することを回避する
\usepackage{lmodern}
%%%%%%%%%%%%%%%%%%%%%%%%%%%
\title{English is fun.}
\subtitle{I wish I had a lot of money.}
\author{}
\institute[]{}
\date[]

%%%%%%%%%%%%%%%%%%%%%%%%%%%%
%% TEXT
%%%%%%%%%%%%%%%%%%%%%%%%%%%%
\begin{document}

\begin{frame}[plain]
  \titlepage
\end{frame}

\section*{授業の流れ}
\begin{frame}[plain]
  \frametitle{授業の流れ}
  \tableofcontents
\end{frame}

\section{0, 1, 2, 3 \ldots}
%%%%%%%%%%%%%%%%%%%%%%%%%%%%%%%%%%%%%%%%%%%%%%%%%%
\begin{frame}[plain]{0, 1, 2, 3 \ldots\,18, 19, 20}
\small
% 0 zero \textipa{/z\'\i:roU/}\pause
\hfill\begin{tblr}{
  colspec = {rll}, 
%  column{1} = {fg=red},    % 第1列の文字を赤に
%  column{2} = {fg=blue},   % 第7列の文字を青に
 row{odd} = {bg=azure8},
 row{1} = { bg=azure3, fg=white},
 hline{Z} = {0.08em},    % \toprule, \midrule, \bottomrule
%  hline{3} = {0.5pt}       % もう1つの \midrule
 baseline=t,
 cell{1}{3} = {halign=r}
}
  number  & & {\scriptsize \myaudio{./audio/001_number_01a.mp3}}\\
  0 & zero & \textipa{/z\'\i:roU/}\\
  1 & one & \textipa{/w\'\textturnv n/}\\
  2 & two & \textipa{/t\'u:/}\\ 
  3 & three & \textipa{/Tr\'\i:/}\\
  4 & four & \textipa{/f\'O\textrhookschwa /}\\
  5 & five & \textipa{/f\'aIv/}\\
  6 & six & \textipa{/s\'Iks/}\\
  7 & seven & \textipa{/s\'evn/}\\
  8 & eight & \textipa{/\'eIt/}\\
  9 & nine & \textipa{/n\'aIn/}\\
  10 & ten & \textipa{/t\'en/}\\
\end{tblr}
\hfill%
\pause
 \begin{tblr}{
  colspec = {rl}, 
%  column{1} = {fg=red},    % 第1列の文字を赤に
%  column{2} = {fg=blue},   % 第7列の文字を青に
 row{odd} = {bg=azure8},
 row{1} = { bg=azure3, fg=white},
 hline{Z} = {0.08em},    % \toprule, \midrule, \bottomrule
%  hline{3} = {0.5pt}       % もう1つの \midrule
 baseline=t,
 cell{1}{3} = {halign=r},
 cells={cmd=\onslide<\arabic{rownum}->} %%%%tabularrayとpauseが衝突することを回避する方法→https://github.com/lvjr/tabularray/issues/226
}
  number  & & {\scriptsize \myaudio{./audio/001_number_01b.mp3}}\\
\\
  11 & eleven& \textipa{/Il\'evn/} \\
  12 & twelve & \textipa{/tw\'elv/}\\ 
  13 & thirteen & \textipa{/T\textrhookschwa :t\'\i:n/}\\
  14 & fourteen & \textipa{/fO\textrhookschwa t\'\i:n/}\\
  15 & fifteen & \textipa{/fIft\'\i:n/}\\
  16 & sixteen & \textipa{/sIkst\'\i:n/}\\
  17 & seventeen & \textipa{/sevnt\'\i:n/}\\
  18 & eighteen & \textipa{/eIt\'\i:n/}\\
  19 & nineteen & \textipa{/naInt\'\i:n/}\\
  20 & twenty & \textipa{/tw\'enti/}\\
\end{tblr}
\end{frame}
%%%%%%%%%%%%%%%%%%%%%%%%%
\begin{frame}[plain]{Exercises}
 つぎの語があらわす数を数字で書きましょう\hspace{20pt}例: one $\rightarrow$ 1
\begin{enumerate}
 \item zero\hfill\visible<2->{\makebox[20pt][r]{0}}\hspace{250pt}\mbox{}
 \item three\hfill\visible<3->{\makebox[20pt][r]{3}}\hspace{250pt}\mbox{}
 \item four\hfill\visible<4->{\makebox[20pt][r]{4}}\hspace{250pt}\mbox{}
 \item eleven\hfill\visible<5->{\makebox[20pt][r]{11}}\hspace{250pt}\mbox{}
 \item twelve\hfill\visible<6->{\makebox[20pt][r]{12}}\hspace{250pt}\mbox{}
 \item thirteen\hfill\visible<7->{\makebox[20pt][r]{13}}\hspace{250pt}\mbox{}
 \item seventeen\hfill\visible<8->{\makebox[20pt][r]{17}}\hspace{250pt}\mbox{}
 \item eighteen\hfill\visible<9->{\makebox[20pt][r]{18}}\hspace{250pt}\mbox{}
 \item twenty\hfill\visible<10>{\makebox[20pt][r]{20}}\hspace{250pt}\mbox{}
\end{enumerate}
\end{frame}
%%%%%%%%%%%%%%%%%%%%%%%%%%%%%%%%%%%%%%%%%%%%%%%%%%
%%%%%%%%%%%%%%%%%%%%%%%%%
\begin{frame}[plain]{Exercises}
 つぎの数字を英語でつづりましょう\hspace{20pt}例: 2 $\rightarrow$ two
\begin{enumerate}
 \item \phantom{1}3\hfill\visible<2->{\makebox[20pt][l]{three}}\hspace{250pt}\mbox{}
 \item \phantom{1}4\hfill\visible<3->{\makebox[20pt][l]{four}}\hspace{250pt}\mbox{}
 \item \phantom{1}5\hfill\visible<4->{\makebox[20pt][l]{five}}\hspace{250pt}\mbox{}
 \item \phantom{1}9\hfill\visible<5->{\makebox[20pt][l]{nine}}\hspace{250pt}\mbox{}
 \item 11\hfill\visible<6->{\makebox[20pt][l]{eleven}}\hspace{250pt}\mbox{}
 \item 13\hfill\visible<7->{\makebox[20pt][l]{thirteen}}\hspace{250pt}\mbox{}
 \item 15\hfill\visible<8->{\makebox[20pt][l]{fifteen}}\hspace{250pt}\mbox{}
 \item 19\hfill\visible<9->{\makebox[20pt][l]{nineteen}}\hspace{250pt}\mbox{}
 \item 20\hfill\visible<10>{\makebox[20pt][l]{twenty}}\hspace{250pt}\mbox{}
\end{enumerate}
\end{frame}
%%%%%%%%%%%%%%%%%%%%%%%%%%%%%%%%%%%%%%%%%%%%%%%%%%
\begin{frame}[plain]{20, 21, 22 \ldots\,28, 29}
\small\centering
\begin{tblr}{
  colspec = {rll}, 
%  column{1} = {fg=red},    % 第1列の文字を赤に
%  column{2} = {fg=blue},   % 第7列の文字を青に
 row{odd} = {bg=azure8},
 row{1} = { bg=azure3, fg=white},
%  hline{1,2,Z} = {1pt},    % \toprule, \midrule, \bottomrule
%  hline{3} = {0.5pt}       % もう1つの \midrule
 baseline=t,
 cell{1}{3} = {halign=r},
 cells={cmd=\onslide<\arabic{rownum}->} %%%%tabularrayとpauseが衝突することを回避する方法→https://github.com/lvjr/tabularray/issues/226
}
  number  & &{\scriptsize \myaudio{./audio/001_number_02.mp3}}\\
  20 & twenty & \textipa{/tw\'enti/}\\
  21 & twenty-one & \textipa{/tw\'enti w\'\textturnv n/}\\
  22 & twenty-two & \textipa{/tw\'enti t\'u:/}\\ 
  23 & twenty-three & \textipa{/tw\'enti Tr\'\i:/}\\
  24 & twenty-four & \textipa{/tw\'enti f\'O\textrhookschwa /}\\
  25 & twenty-five & \textipa{/tw\'enti f\'aIv/}\\
  26 & twenty-six & \textipa{/tw\'enti s\'Iks/}\\
  27 & twenty-seven & \textipa{/tw\'enti s\'evn/}\\
  28 & twenty-eight & \textipa{/tw\'enti \'eIt/}\\
  29 & twenty-nine & \textipa{/tw\'enti n\'aIn/}\\
%  30 & thirty & \textipa{/T\'\textrhookschwa :ti/}\\
\end{tblr}
\end{frame}
%%%%%%%%%%%%%%%%%%%%%%%%%
%%%%%%%%%%%%%%%%%%%%%%%%%%%%%%%%%%%%%%%%%%%%%%%%%%
\begin{frame}[plain]{30, 31, 32 \ldots\,38, 39}
\small
\centering
 \begin{tblr}{
  colspec = {rll}, 
%  column{1} = {fg=red},    % 第1列の文字を赤に
%  column{2} = {fg=blue},   % 第7列の文字を青に
 row{odd} = {bg=azure8},
 row{1} = { bg=azure3, fg=white},
%  hline{1,2,Z} = {1pt},    % \toprule, \midrule, \bottomrule
%  hline{3} = {0.5pt}       % もう1つの \midrule
 baseline=t,
 cell{1}{3} = {halign=r},
 cells={cmd=\onslide<\arabic{rownum}->} %%%%tabularrayとpauseが衝突することを回避する方法→https://github.com/lvjr/tabularray/issues/226
}
  number  & &{\scriptsize \myaudio{./audio/001_number_03.mp3}}\\
  30 & thirty & \textipa{/T\'\textrhookschwa :ti/}\\
  31 & thirty-one & \textipa{/T\'\textrhookschwa :ti w\'\textturnv n/}\\
  32 & thirty-two & \textipa{/T\'\textrhookschwa :ti t\'u:/}\\ 
  33 & thirty-three & \textipa{/T\'\textrhookschwa :ti Tr\'\i:/}\\
  34 & thirty-four & \textipa{/T\'\textrhookschwa :ti f\'O\textrhookschwa /}\\
  35 & thirty-five & \textipa{/T\'\textrhookschwa :ti f\'aIv/}\\
  36 & thirty-six & \textipa{/T\'\textrhookschwa :ti s\'Iks/}\\
  37 & thirty-seven & \textipa{/T\'\textrhookschwa :ti s\'evn/}\\
  38 & thirty-eight & \textipa{/T\'\textrhookschwa :ti \'eIt/}\\
  39 & thirty-nine & \textipa{/T\'\textrhookschwa :ti n\'aIn/}\\
%  30 & thirty & \textipa{/T\'\textrhookschwa :ti/}\\
\end{tblr}
 
\end{frame}
%%%%%%%%%%%%%%%%%%%%%%%%%
%%%%%%%%%%%%%%%%%%%%%%%%%
\begin{frame}[plain]{Exercises}
 つぎの語があらわす数を数字で書きましょう\hspace{20pt}例: one $\rightarrow$ 1
\begin{enumerate}
 \item twenty-one\hfill\visible<2->{\makebox[20pt][r]{21}}\hspace{250pt}\mbox{}
 \item twenty-two\hfill\visible<3->{\makebox[20pt][r]{22}}\hspace{250pt}\mbox{}
 \item twenty-five\hfill\visible<4->{\makebox[20pt][r]{25}}\hspace{250pt}\mbox{}
 \item twenty-seven\hfill\visible<5->{\makebox[20pt][r]{27}}\hspace{250pt}\mbox{}
 \item thirty-three\hfill\visible<6->{\makebox[20pt][r]{33}}\hspace{250pt}\mbox{}
 \item thirty-four\hfill\visible<7->{\makebox[20pt][r]{34}}\hspace{250pt}\mbox{}
 \item thirty-six\hfill\visible<8->{\makebox[20pt][r]{36}}\hspace{250pt}\mbox{}
 \item thirty-eight\hfill\visible<9->{\makebox[20pt][r]{38}}\hspace{250pt}\mbox{}
 \item thirty-nine\hfill\visible<10>{\makebox[20pt][r]{39}}\hspace{250pt}\mbox{}
\end{enumerate}
\end{frame}
%%%%%%%%%%%%%%%%%%%%%%%%%%%%%%%%%%%%%%%%%%%%%%%%%%
%%%%%%%%%%%%%%%%%%%%%%%%%
\begin{frame}[plain]{Exercises}
 つぎの数字を英語でつづりましょう\hspace{20pt}例: 2 $\rightarrow$ two
\begin{enumerate}
 \item 23\hfill\visible<2->{\makebox[20pt][l]{twenty-three}}\hspace{250pt}\mbox{}
 \item 24\hfill\visible<3->{\makebox[20pt][l]{twenty-four}}\hspace{250pt}\mbox{}
 \item 25\hfill\visible<4->{\makebox[20pt][l]{twenty-five}}\hspace{250pt}\mbox{}
 \item 29\hfill\visible<5->{\makebox[20pt][l]{twenty-nine}}\hspace{250pt}\mbox{}
 \item 31\hfill\visible<6->{\makebox[20pt][l]{thirty-one}}\hspace{250pt}\mbox{}
 \item 36\hfill\visible<7->{\makebox[20pt][l]{thirty-six}}\hspace{250pt}\mbox{}
 \item 37\hfill\visible<8->{\makebox[20pt][l]{thirty-seven}}\hspace{250pt}\mbox{}
 \item 38\hfill\visible<9->{\makebox[20pt][l]{thirty-eight}}\hspace{250pt}\mbox{}
 \item 39\hfill\visible<10>{\makebox[20pt][l]{thirty-nine}}\hspace{250pt}\mbox{}
\end{enumerate}
\end{frame}
%%%%%%%%%%%%%%%%%%%%%%%%%%%%%%%%%%%%%%%%%%%%%%%%%%
%%%%%%%%%%%%%%%%%%%%%%%%%%%%%%%%%%%%%%%%%%%%%%%%%%
\begin{frame}[plain]{10, 20, 30, \ldots\,80, 90, 100}
\small
\centering
\begin{tblr}{
  colspec = {rll}, 
%  column{1} = {fg=red},    % 第1列の文字を赤に
%  column{2} = {fg=blue},   % 第7列の文字を青に
 row{odd} = {bg=azure8},
 row{1} = { bg=azure3, fg=white},
%  hline{1,2,Z} = {1pt},    % \toprule, \midrule, \bottomrule
%  hline{3} = {0.5pt}       % もう1つの \midrule
 baseline=t,
 cell{1}{3} = {halign=r},
 cells={cmd=\onslide<\arabic{rownum}->} %%%%tabularrayとpauseが衝突することを回避する方法→https://github.com/lvjr/tabularray/issues/226
}
  number  & &{\scriptsize \myaudio{./audio/001_number_04.mp3}}\\
  10 & ten & \textipa{/t\'en/}\\
  20 & twenty & \textipa{/tw\'enti/}\\
  30 & thirty & \textipa{/T\'\textrhookschwa :ti/}\\ 
  40 & forty & \textipa{/f\'O\textrhookschwa :ti/}\\
  50 & fifty & \textipa{/f\'Ifti/}\\
  60 & sixty & \textipa{/s\'Iksti/}\\
  70 & seventy & \textipa{/s\'evnti/}\\
  80 & eighty & \textipa{/\'eIti/}\\
  90 & ninety & \textipa{/n\'aInti/}\\
  100 & one hundred & \textipa{/w\'\textturnv n h\'\textturnv ndr@d/}\\
%  30 & thirty & \textipa{/T\'\textrhookschwa :ti/}\\
\end{tblr}

\end{frame}
%%%%%%%%%%%%%%%%%%%%%%%%%
%%%%%%%%%%%%%%%%%%%%%%%%%
\begin{frame}[plain]{Exercises}
 つぎの語があらわす数を数字で書きましょう\hspace{20pt}例: one $\rightarrow$ 1
\begin{enumerate}
 \item ten\hfill\visible<2->{\makebox[20pt][r]{10}}\hspace{250pt}\mbox{}
 \item twenty-three\hfill\visible<3->{\makebox[20pt][r]{23}}\hspace{250pt}\mbox{}
 \item thirty-five\hfill\visible<4->{\makebox[20pt][r]{35}}\hspace{250pt}\mbox{}
 \item forty\hfill\visible<5->{\makebox[20pt][r]{40}}\hspace{250pt}\mbox{}
 \item fifty-two\hfill\visible<6->{\makebox[20pt][r]{52}}\hspace{250pt}\mbox{}
 \item sixty\hfill\visible<7->{\makebox[20pt][r]{60}}\hspace{250pt}\mbox{}
 \item seventy-seven\hfill\visible<8->{\makebox[20pt][r]{77}}\hspace{250pt}\mbox{}
 \item eighty-eight\hfill\visible<9->{\makebox[20pt][r]{88}}\hspace{250pt}\mbox{}
 \item ninety-nine\hfill\visible<10>{\makebox[20pt][r]{99}}\hspace{250pt}\mbox{}
\end{enumerate}
\end{frame}
%%%%%%%%%%%%%%%%%%%%%%%%%%%%%%%%%%%%%%%%%%%%%%%%%%
%%%%%%%%%%%%%%%%%%%%%%%%%
\begin{frame}[plain]{Exercises}
 つぎの数字を英語でつづりましょう\hspace{20pt}例: 2 $\rightarrow$ two
\begin{enumerate}
 \item \phantom{1}23\hfill\visible<2->{\makebox[20pt][l]{twenty-three}}\hspace{250pt}\mbox{}
 \item \phantom{1}34\hfill\visible<3->{\makebox[20pt][l]{thirty-four}}\hspace{250pt}\mbox{}
 \item \phantom{1}45\hfill\visible<4->{\makebox[20pt][l]{forty-five}}\hspace{250pt}\mbox{}
 \item \phantom{1}59\hfill\visible<5->{\makebox[20pt][l]{fifty-nine}}\hspace{250pt}\mbox{}
 \item \phantom{1}61\hfill\visible<6->{\makebox[20pt][l]{sixty-eleven}}\hspace{250pt}\mbox{}
 \item \phantom{1}73\hfill\visible<7->{\makebox[20pt][l]{seventy-three}}\hspace{250pt}\mbox{}
 \item \phantom{1}83\hfill\visible<8->{\makebox[20pt][l]{eighty-three}}\hspace{250pt}\mbox{}
 \item \phantom{1}99\hfill\visible<9->{\makebox[20pt][l]{ninety-nine}}\hspace{250pt}\mbox{}
 \item 100\,(2語で)\hfill\visible<10>{\makebox[20pt][l]{one hundred}}\hspace{250pt}\mbox{}
\end{enumerate}
\end{frame}
%%%%%%%%%%%%%%%%%%%%%%%%%%%%%%%%%%%%%%%%%%%%%%%%%%
\begin{frame}[plain]{more than 100}
 \small
\centering
\begin{tblr}{
  colspec = {rl}, 
%  column{1} = {fg=red},    % 第1列の文字を赤に
%  column{2} = {fg=blue},   % 第7列の文字を青に
 row{odd} = {bg=azure8},
 row{1} = { bg=azure3, fg=white},
%  hline{1,2,Z} = {1pt},    % \toprule, \midrule, \bottomrule
%  hline{3} = {0.5pt}       % もう1つの \midrule
 baseline=t,
 cell{1}{2} = {halign=r},
 cells={cmd=\onslide<\arabic{rownum}->} %%%%tabularrayとpauseが衝突することを回避する方法→https://github.com/lvjr/tabularray/issues/226
}
  number  & {\scriptsize \myaudio{./audio/001_number_05.mp3}}\\
101& one hundred (and) one\\
102& one hundred (and) two\\
111& one hundred (and) eleven\\
125& one hundred (and) twenty-five\\
134& one hundred (and) thirty-four\\
243& two hundred (and) forty-three\\
358& one hundred (and) fifty-eight\\
567& five hundred (and) sixty-seven\\
777& seven hundered (and) seventy-seven\\
999& nine hundred (and) ninty-nine\\
\end{tblr}
\end{frame}
%%%%%%%%%%%%%%%%%%%%%%%%%%%%%%%%%%%%%%%%%%%%%%%%%%
\begin{frame}[plain]{{\scriptsize \textdbend}\,more than 999}
  \small
\centering
\begin{tblr}{
  colspec = {rl}, 
%  column{1} = {fg=red},    % 第1列の文字を赤に
%  column{2} = {fg=blue},   % 第7列の文字を青に
 row{odd} = {bg=azure8},
 row{1} = { bg=azure3, fg=white},
%  hline{1,2,Z} = {1pt},    % \toprule, \midrule, \bottomrule
%  hline{3} = {0.5pt}       % もう1つの \midrule
 baseline=t,
 cell{1}{2} = {halign=r},
 cells={cmd=\onslide<\arabic{rownum}->} %%%%tabularrayとpauseが衝突することを回避する方法→https://github.com/lvjr/tabularray/issues/226
}
  number  & {\scriptsize \myaudio{./audio/001_number_06.mp3}}\\
1,000& one thousand\hspace{20pt}\textipa{/w\'\textturnv n T\'aUznd/}\\
1,400& one thousand four hundred\\
2,300 & two thousand three hundred\\
3,450& three thousand four hundred (and) fifty\\
4,567& four thousand five hundred (and) sixty-seven\\
10,000 & ten thousand\\
20,000 & twenty thousand\\
100,000& one hundred thousand\\
1,000,000& one million\hspace{30pt}\textipa{/w\'\textturnv n m\'Ilj@n/}\\
1,000,000,000& one billion\hspace{32.5pt}\textipa{/w\'\textturnv n b\'Ilj@n/}\\
\end{tblr}
\end{frame}
%%%%%%%%%%%%%%%%%%%%%%%%%%%%%%%%%%%%
\begin{frame}[plain,shrink=5]{一覧表}
\small
\begin{tblr}{
  colspec = {rl}, 
 row{odd} = {bg=azure8},
 row{1} = { bg=azure3, fg=white},
 hline{Z} = {0.08em},    % \toprule, \midrule, \bottomrule
%  hline{3} = {0.5pt}       % もう1つの \midrule
 baseline=t
}
  number  & \\
  0 & zero \\
  1 & one \\
  2 & two \\ 
  3 & three \\
  4 & four \\
  5 & five \\
  6 & six\\
  7 & seven \\
  8 & eight \\
  9 & nine \\
  10 & ten \\
\end{tblr}
\pause
 \begin{tblr}{
  colspec = {rl}, 
%  column{1} = {fg=red},    % 第1列の文字を赤に
%  column{2} = {fg=blue},   % 第7列の文字を青に
 row{odd} = {bg=azure8},
 row{1} = { bg=azure3, fg=white},
 hline{Z} = {0.08em},    % \toprule, \midrule, \bottomrule
%  hline{3} = {0.5pt}       % もう1つの \midrule
 baseline=t,
%  cells={cmd=\onslide<\arabic{rownum}->} %%%%tabularrayとpauseが衝突することを回避する方法→https://github.com/lvjr/tabularray/issues/226
}
   & \\
\\
  11 & eleven \\
  12 & twelve\\ 
  13 & thirteen \\
  14 & fourteen \\
  15 & fifteen \\
  16 & sixteen \\
  17 & seventeen \\
  18 & eighteen \\
  19 & nineteen \\
  20 & twenty \\
\end{tblr}
\pause
\begin{tblr}{
  colspec = {rll}, 
%  column{1} = {fg=red},    % 第1列の文字を赤に
%  column{2} = {fg=blue},   % 第7列の文字を青に
 row{odd} = {bg=azure8},
 row{1} = { bg=azure3, fg=white},
%  hline{1,2,Z} = {1pt},    % \toprule, \midrule, \bottomrule
%  hline{3} = {0.5pt}       % もう1つの \midrule
 hline{Z} = {0.08em},    % \toprule, \midrule, \bottomrule
 baseline=t,
% cells={cmd=\onslide<\arabic{rownum}->} %%%%tabularrayとpauseが衝突することを回避する方法→https://github.com/lvjr/tabularray/issues/226
}
    & \\
   &  \\
  21 & twenty-one \\
  22 & twenty-two \\ 
  23 & twenty-three \\
  24 & twenty-four \\
  25 & twenty-five \\
  26 & twenty-six \\
  27 & twenty-seven \\
  28 & twenty-eight \\
  29 & twenty-nine \\
  30 & thirty \\
\end{tblr}
\pause
\begin{tblr}{
  colspec = {rll}, 
%  column{1} = {fg=red},    % 第1列の文字を赤に
%  column{2} = {fg=blue},   % 第7列の文字を青に
 row{odd} = {bg=azure8},
 row{1} = { bg=azure3, fg=white},
%  hline{1,2,Z} = {1pt},    % \toprule, \midrule, \bottomrule
%  hline{3} = {0.5pt}       % もう1つの \midrule
 hline{Z} = {0.08em},    % \toprule, \midrule, \bottomrule
 baseline=t,
% cells={cmd=\onslide<\arabic{rownum}->} %%%%tabularrayとpauseが衝突することを回避する方法→https://github.com/lvjr/tabularray/issues/226
}
    & \\
    & \\
  40 & forty \\
  50 & fifty \\
  60 & sixty \\
  70 & seventy \\
  80 & eighty \\
  90 & ninety \\
  100 & one hundred \\
  1,000 & one thousand\\
  1,000,000 & one million\\
  1,000,000,000 & one billion\\
\end{tblr}
\end{frame}
%%%%%%%%%%%%%%%%%%%%%%%%%
\begin{frame}[plain]{小数}
 \Huge

\[
 \pi \approx 3.14
\]

\hfill{}three point one four

\bigskip

$\approx$は「ほぼ等しい」という記号

\end{frame}
%%%%%%%%%%%%%%%%%%%%%%%%%%%%%%%%%%%%
\end{document}
