\documentclass[aspectratio=169,xcolor={dvipsnames,table}]{beamer}
\usepackage[no-math,deluxe,haranoaji]{luatexja-preset}
\renewcommand{\kanjifamilydefault}{\gtdefault}
\renewcommand{\emph}[1]{{\upshape\bfseries #1}}
\usetheme{metropolis}
\metroset{block=fill}
\setbeamertemplate{navigation symbols}{}
\setbeamertemplate{blocks}[rounded][shadow=false]
\usecolortheme[rgb={0.7,0.2,0.2}]{structure}
%%%%%%%%%%%%%%%%%%%%%%%%%%
%% Change alert block colors
%%% 1- Block title (background and text)
\setbeamercolor{block title alerted}{fg=mDarkTeal, bg=mLightBrown!45!yellow!45}
\setbeamercolor{block title example}{fg=magenta!10!black, bg=mLightGreen!60}
%%% 2- Block body (background)
\setbeamercolor{block body alerted}{bg=mLightBrown!25}
\setbeamercolor{block body example}{bg=mLightGreen!15}
%%%%%%%%%%%%%%%%%%%%%%%%%%%
%%%%%%%%%%%%%%%%%%%%%%%%%%%
%% さまざまなアイコン
%%%%%%%%%%%%%%%%%%%%%%%%%%%
%\usepackage{fontawesome}
\usepackage{fontawesome5}
\usepackage{figchild}
\usepackage{twemojis}
\usepackage{utfsym}
\usepackage{bclogo}
\usepackage{marvosym}
\usepackage{fontmfizz}
\usepackage{pifont}
\usepackage{phaistos}
\usepackage{worldflags}
\usepackage{jigsaw}
\usepackage{tikzlings}
\usepackage{tikzducks}
\usepackage{scsnowman}
\usepackage{epsdice}
\usepackage{halloweenmath}
\usepackage{svrsymbols}
\usepackage{countriesofeurope}
\usepackage{tipa}
\usepackage{manfnt}
%%%%%%%%%%%%%%%%%%%%%%%%%%%
\usepackage{tikz}
\usetikzlibrary{calc,patterns,decorations.pathmorphing,backgrounds}
\usepackage{tcolorbox}
\usepackage{tikzpeople}
\usepackage{circledsteps}
\usepackage{xcolor}
\usepackage{amsmath}
\usepackage{booktabs}
\usepackage{chronology}
\usepackage{signchart}
%%%%%%%%%%%%%%%%%%%%%%%%%%%
%% 場合分け
%%%%%%%%%%%%%%%%%%%%%%%%%%%
\usepackage{cases}
%%%%%%%%%%%%%%%%%%%%%%%%%%
\usepackage{pdfpages}
%%%%%%%%%%%%%%%%%%%%%%%%%%%
%% 音声リンク表示
\newcommand{\myaudio}[1]{\href{#1}{\faVolumeUp}}
%%%%%%%%%%%%%%%%%%%%%%%%%%
%% \myAnch{<名前>}{<色>}{<テキスト>}
%% 指定のテキストを指定の色の四角枠で囲み, 指定の名前をもつTikZの
%% ノードとして出力する. 図には remember picture 属性を付けている
%% ので外部から参照可能である.
\newcommand*{\myAnch}[3]{%
  \tikz[remember picture,baseline=(#1.base)]
    \node[draw,rectangle,line width=1pt,#2] (#1) {\normalcolor #3};
}
%%%%%%%%%%%%%%%%%%%%%%%%%%
%% \myEmph コマンドの定義
%%%%%%%%%%%%%%%%%%%%%%%%%%
%\newcommand{\myEmph}[3]{%
%    \textbf<#1>{\color<#1>{#2}{#3}}%
%}
\usepackage{xparse} % xparseパッケージの読み込み
\NewDocumentCommand{\myEmph}{O{} m m}{%
    \def\argOne{#1}%
    \ifx\argOne\empty
        \textbf{\color{#2}{#3}}% オプション引数が省略された場合
    \else
        \textbf<#1>{\color<#1>{#2}{#3}}% オプション引数が指定された場合
    \fi
}
%%%%%%%%%%%%%%%%%%%%%%%%%%%
%%%%%%%%%%%%%%%%%%%%%%%%%%%
%% 文末の上昇イントネーション記号 \myRisingPitch
%% 通常のイントネーション \myDownwardPitch
%% https://note.com/dan_oyama/n/n8be58e8797b2
%%%%%%%%%%%%%%%%%%%%%%%%%%%
\newcommand{\myRisingPitch}{
\begin{tikzpicture}[scale=0.3,baseline=0.3]
\draw[->,>=stealth] (0,0) to[bend right=45] (1,1);
\end{tikzpicture}
}
\newcommand{\myDownwardPitch}{
\begin{tikzpicture}[scale=0.3,baseline=0.3]
\draw[->,>=stealth] (0,1) to[bend left=45] (1,0);
\end{tikzpicture}
}
%%%%%%%%%%%%%%%%%%%%%%%%%%%%
%\AtBeginSection[%
%]{%
%  \begin{frame}[plain]\frametitle{授業の流れ}
%     \tableofcontents[currentsection]
%   \end{frame}%
%}

\usepackage{pxrubrica}
\UseTblrLibrary{counter}%%%%tabularrayとpauseが衝突することを回避する
\usepackage{lmodern}
%%%%%%%%%%%%%%%%%%%%%%%%%%%%%%%%%%%%%%%%%%
%%%%%%%%%%%%%%%%%%%%%%%%%%%
\title{English is fun.}
\subtitle{曜日、月、季節}
\author{}
\institute[]{}
\date[]

%%%%%%%%%%%%%%%%%%%%%%%%%%%%
%% TEXT
%%%%%%%%%%%%%%%%%%%%%%%%%%%%
\begin{document}

\begin{frame}[plain]
  \titlepage
\end{frame}

\section*{授業の流れ}
\begin{frame}[plain]
  \frametitle{授業の流れ}
  \tableofcontents
\end{frame}

\section{曜日}
%%%%%%%%%%%%%%%%%%%%%%%%%%%%%%%%%%%%%%%%%%%%%%%%%%
\begin{frame}[plain]{曜日}
\centering
\begin{tblr}{
  colspec = {rll}, 
%  column{2} = {fg=blue},   % 第7列の文字を青に
 row{odd} = {bg=azure8},
 row{1} = { bg=azure3, fg=white},
 row{2} = {fg=Maroon!80},    % 第2列の文字を赤に
 row{Z} = {fg=azure3},
 hline{Z} = {0.08em},    % \toprule, \midrule, \bottomrule
%  hline{3} = {0.5pt}       % もう1つの \midrule
 baseline=t,
 cell{1}{3} = {halign=r}
}
    & & {\scriptsize \myaudio{./audio/002_day_month_season_01.mp3}}\\
  日 & Sunday & \textipa{/s\'\textturnv nd\`eI/}\\
  月 & Monday & \textipa{/m\'\textturnv nd\`eI/}\\
  火 & Tuesday & \textipa{/t(j)\'u:zd\`eI/}\\ 
  水 & Wednesday & \textipa{/w\'enzd\`eI/}\\
  木 & Thursday & \textipa{/T\'\textrhookschwa :zd\`eI/}\\
  金 & Friday & \textipa{/fr\'aId\`eI/}\\
  土 & Saturday & \textipa{/s\'\ae t\textrhookschwa d\`eI/}\\
\end{tblr}
\end{frame}
%%%%%%%%%%%%%%%%%%%%%%%%%%%%%%%%%%%%%%%%%%%%%%%%%%
\section{month \textipa{/m\'\textturnv nT/}   season \textipa{/s\'\i:zn/}}
%%%%%%%%%%%%%%%%%%%%%%%%%%%%%%%%%%%%%%%%%%%%%%%%%%
\setbeamercolor{background canvas}{bg=}
\begin{frame}[plain]
\small

\begin{columns}
\begin{column}{.55\linewidth}
\begin{tblr}{
  colspec = {rlll}, 
%  column{2} = {fg=blue},   % 第7列の文字を青に
 row{1} = { bg=azure3, fg=white},
 row{2-3,13} = {bg=azure8},
 row{4-6} = {bg=SpringGreen!60},    
 row{7-9} = {bg=Goldenrod!60},    
 row{10-12} = {bg=Maroon!60!black!80, fg=white},
 column{4} = {bg=white},    
% row{Z} = {fg=azure3},
% hline{Z} = {0.08em},    % \toprule, \midrule, \bottomrule
%  hline{3} = {0.5pt}       % もう1つの \midrule
 baseline=t,
 cell{1}{3} = {halign=r},
 cells={cmd=\onslide<\arabic{rownum}->} %%%%tabularrayとpauseが衝突することを回避する方法→https://github.com/lvjr/tabularray/issues/226
}
    & month& {\scriptsize \textipa{/m\'\textturnv nT/}\hspace{15pt}\myaudio{./audio/002_day_month_season_02a.mp3}}\\
  1月 & January & \textipa{/dZ\'\ae nju\`eri/}&\myAnch{fuyu1}{white}{}\\
  2月 & February & \textipa{/f\'eb(r)ueri/}\\
  3月 & March & \textipa{/m\'A\textrhookschwa tS/}\\ 
  4月 & April & \textipa{/\'eIpr@l/}&\myAnch{haru}{white}{}\\
  5月 & May & \textipa{/m\'eI/}\\
  6月 & June & \textipa{/dZ\'u:n/}\\
  7月 & July & \textipa{/dZUl\'aI/}&\myAnch{natsu}{white}{}\\
  8月 & August & \textipa{/\'O:g@st/}\\ 
  9月 & September & \textipa{/sept\'emb\textrhookschwa /}\\
  10月 & October & \textipa{/Akt\'oUb@/}&\myAnch{aki}{white}{}\\
  11月 & November & \textipa{/nouv\'emb@/}\\
  12月 & December & \textipa{/dIs\'emb@/}&\myAnch{fuyu2}{white}{}\\
\end{tblr}
\end{column}
\begin{column}<14->{.4\linewidth}
\begin{tblr}{
  colspec = {lll}, 
%  column{2} = {fg=blue},   % 第7列の文字を青に
row{1} = { bg=azure3, fg=white},
 row{5} = {bg=azure8},
 row{2} = {bg=SpringGreen!60},    
 row{3} = {bg=Goldenrod!60},    
 row{4} = {bg=Maroon!60!black!80, fg=white},
column{1} = {bg=white},
  baseline=t,
 cells={cmd=\onslide<\arabic{rownum}->} %%%%tabularrayとpauseが衝突することを回避する方法→https://github.com/lvjr/tabularray/issues/226
}
    & season& {\scriptsize \textipa{/s\'\i:zn/}\hspace{27pt}\myaudio{./audio/002_day_month_season_02b.mp3}}\\
 \myAnch{spring}{white}{}&spring&\textipa{/spr\'IN/}\\
 \myAnch{summer}{white}{}&summer&\textipa{/s\'\textturnv m\textrhookschwa /}\\
 \myAnch{fall}{white}{}&fall / autumn&\textipa{/f\'O:l/} \textipa{/\'O:t@m/}\\
 \myAnch{winter}{white}{}&winter&\textipa{/w\'Int\textrhookschwa/}\\\
\end{tblr}
\end{column}
\end{columns}


\begin{tikzpicture}[remember picture,overlay]
 \onslide<15->{\draw[<-,line width=3pt,SpringGreen,opacity=.75] (haru.east) -- (spring.west);}
 \onslide<16->{\draw[<-,line width=3pt,Goldenrod,opacity=.75] (natsu.east) -- (summer.west);}
 {\onslide<17->\draw[<-,line width=3pt,Maroon,opacity=.75] (aki.east) -- (fall.west);}
 \onslide<18->{\draw[<-,line width=3pt,azure3,opacity=.75] (fuyu1.east) to[out=0,in=180] (winter.west);}
 \onslide<18->{\draw[<-,line width=3pt,azure3,opacity=.75] (fuyu2.315) to[out=0,in=200] (winter.west);}
\end{tikzpicture}

\end{frame}
%%%%%%%%%%%%%%%%%%%%%%%%%%%%%%%%%%%%%%%%%%%%%%%%%%
\setbeamercolor{background canvas}{bg=}
\end{document}
